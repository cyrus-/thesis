\PassOptionsToPackage{svgnames,dvipsnames,svgnames}{xcolor}

%for a more compact document, add the option openany to avoid
%starting all chapters on odd numbered pages
\documentclass[12pt]{cmuthesis}
%\usepackage[usenames,dvipsnames,svgnames]{xcolor}
\newif\ificfp
\icfpfalse
\newcommand{\todolater}[1]{{\color{magenta} TODO (Later): #1}}
\newcommand{\todo}[1]{{\color{red} TODO: #1}}
% some useful packages
\usepackage{times}     % use times font for document
%\usepackage{lmodern}
\usepackage{newtxtt}
\usepackage{bbm}
%\renewcommand{\ttdefault}{txtt} % use txtt for typewriter font
\usepackage{mathpazo}
\usepackage{mathpartir} % use package for inference rules
\usepackage{upgreek} % package for alternative greek letters (\uppi)
\usepackage{fullpage}
\usepackage{colortab}
%\usepackage{graphicx}
\usepackage[labelfont=bf]{caption}

% \usepackage{cleveref}
%\usepackage{MnSymbol}
\usepackage{fancyvrb}
\usepackage{microtype}
\usepackage[framemethod=tikz]{mdframed}

% Get just llangle and rrangle from MnSymbol
\makeatletter
\DeclareFontFamily{OMX}{MnSymbolE}{}
\DeclareSymbolFont{MnLargeSymbols}{OMX}{MnSymbolE}{m}{n}
\SetSymbolFont{MnLargeSymbols}{bold}{OMX}{MnSymbolE}{b}{n}
\DeclareFontShape{OMX}{MnSymbolE}{m}{n}{
    <-6>  MnSymbolE5
   <6-7>  MnSymbolE6
   <7-8>  MnSymbolE7
   <8-9>  MnSymbolE8
   <9-10> MnSymbolE9
  <10-12> MnSymbolE10
  <12->   MnSymbolE12
}{}
\DeclareFontShape{OMX}{MnSymbolE}{b}{n}{
    <-6>  MnSymbolE-Bold5
   <6-7>  MnSymbolE-Bold6
   <7-8>  MnSymbolE-Bold7
   <8-9>  MnSymbolE-Bold8
   <9-10> MnSymbolE-Bold9
  <10-12> MnSymbolE-Bold10
  <12->   MnSymbolE-Bold12
}{}

\let\llangle\@undefined
\let\rrangle\@undefined
\DeclareMathDelimiter{\llangle}{\mathopen}%
                     {MnLargeSymbols}{'164}{MnLargeSymbols}{'164}
\DeclareMathDelimiter{\rrangle}{\mathclose}%
                     {MnLargeSymbols}{'171}{MnLargeSymbols}{'171}
\makeatother

% \usepackage{rotating}
% \usepackage{pdflscape}

\usepackage[colorlinks=true,allcolors=Blue,backref,pageanchor=true,plainpages=false, pdfpagelabels, bookmarks,bookmarksnumbered,
pdfborder={0 0 0},  %removes outlines around hyper links in online display
]{hyperref}

\usepackage{amsmath,amssymb, amsthm}

\allowdisplaybreaks[1]
\newtheorem{theorem}{Theorem}[chapter]
\newtheorem{lemma}[theorem]{Lemma}
\newtheorem{corollary}[theorem]{Corollary}
\newtheorem{definition}[theorem]{Definition}
\newtheorem{assumption}[theorem]{Assumption}
\newtheorem{condition}[theorem]{Condition}

\usepackage{pfsteps}

\usepackage[noabbrev]{cleveref}

\newenvironment{proof-sketch}{\noindent{\emph{Proof Sketch.}}}{\qed}

\makeatletter
\renewenvironment{proof}[1][\proofname]{\par
  \vspace{-\topsep}% remove the space after the theorem
  \pushQED{\qed}%
  \normalfont
  \topsep0pt \partopsep0pt % no space before
  \trivlist
  \item[\hskip\labelsep
        \itshape
    #1\@addpunct{.}]\ignorespaces
}{%
  \popQED\endtrivlist\@endpefalse
  \addvspace{6pt plus 6pt} % some space after
}
\makeatother
\makeatletter
\renewenvironment{proof-sketch}[1][\proofname]{\par
  \vspace{-\topsep}% remove the space after the theorem
  \pushQED{\qed}%
  \normalfont
  \topsep0pt \partopsep0pt % no space before
  \trivlist
  \item[\hskip\labelsep
        \itshape
    Proof Sketch\@addpunct{.}]\ignorespaces
}{%
  \popQED\endtrivlist\@endpefalse
  \addvspace{6pt plus 6pt} % some space after
}
\makeatother




\usepackage{mathtools}
\usepackage{ stmaryrd }
\usepackage[numbers,sort]{natbib}

\usepackage{subfigure}

% Approximately 1" margins, more space on binding side
%\usepackage[letterpaper,twoside,vscale=.8,hscale=.75,nomarginpar]{geometry}
%for general printing (not binding)
\usepackage[letterpaper,twoside,vscale=.8,hscale=.75,nomarginpar,hmarginratio=1:1]{geometry}

\usepackage{listings}
\lstset{tabsize=2, 
basicstyle=\ttfamily\fontsize{11pt}{1em}\selectfont, 
commentstyle=\itshape\ttfamily\color{gray}, 
stringstyle=\ttfamily\color{red},
mathescape=false,escapechar=\#,
numbers=left, numberstyle=\scriptsize\color{gray}\ttfamily, language=ML,moredelim=[il][\sffamily]{?},showspaces=false,showstringspaces=false,xleftmargin=15pt, morekeywords=[1]{tyfam,opfam,let,fn,val,def,casetype,objtype,metadata,of,*,datatype,new,toast,syntax,module,where,import,for,ana,syn,opcon,tycon,metasignature,metamodule,metasig,metamod,static,at,tycase,mod,macro,match,pattern,in,patterns,expressions,implicit,forall,rectype,fold,unfold,inj,by,spliced},deletekeywords={double},classoffset=0,belowskip=\smallskipamount,
moredelim=**[is][\color{red}]{SSTR}{ESTR},
moredelim=**[is][\color{Green}]{SHTML}{EHTML},
moredelim=**[is][\color{purple}]{SCSS}{ECSS},
moredelim=**[is][\color{brown}]{SSQL}{ESQL},
moredelim=**[is][\color{orange}]{SCOLOR}{ECOLOR},
moredelim=**[is][\color{magenta}]{SPCT}{EPCT}, 
moredelim=**[is][\color{gray}]{SNAT}{ENAT}, 
moredelim=**[is][\color{Green}]{SURL}{EURL},
moredelim=**[is][\color{blue}]{SURI}{EURI},
moredelim=**[is][\color{SeaGreen}]{SQT}{EQT},
moredelim=**[is][\color{Periwinkle}]{SGRM}{EGRM},
moredelim=**[is][\color{YellowGreen}]{SID}{EID},
moredelim=**[is][\color{Sepia}]{SUS}{EUS},
deletestring=[d]{"},
}
\lstloadlanguages{Java,VBScript,XML,HTML,ML}
\let\li\lstinline

% http://tex.stackexchange.com/q/43526
% fix the apparently deliberate but undocumented behaviour of disabling escapes other than mathescape in TextStyle (used by \lstinline)
% there may be a good reason why this is disabled by default, so beware in case it causes any problems
\usepackage{etoolbox}
\makeatletter
\patchcmd{\lsthk@TextStyle}{\let\lst@DefEsc\@empty}{}{}{\errmessage{failed to patch}}
\makeatother

% Provides a draft mark at the bottom of the document. 
% \draftstamp{\today}{DRAFT}


% I hate hyphenation.
%\lefthyphenmin=5
\definecolor{light-gray}{gray}{0.95}

% !TEX root = omar-thesis.tex
\newcommand{\dolla}{\texttt{\$}}  % used so I don't screw up syntax highlighting when using $ in an identifier inline

% \newcommand{\gheading}[1]{\multicolumn{3}{l}{\textbf{#1}}}

\newcommand{\elided}{{\color{gray}\cdots}}

% Calculi Names
\newcommand{\miniVerseUE}{\mathsf{miniVerse}_\textbf{UE}}
\newcommand{\miniVersePat}{\mathsf{miniVerse}_\textbf{U}}
\newcommand{\miniVerseParam}{\mathsf{miniVerse}_\mathbf{\forall}}
\newcommand{\miniVerseTSL}{\mathsf{miniVerse}_\textbf{TSL}}

% General abstract syntax
\newcommand{\aboppz}[2]{\texttt{#1}\texttt{[}#2\texttt{]}}
\newcommand{\abop}[2]{\texttt{#1}\texttt{(}#2\texttt{)}}
\newcommand{\abopi}[3]{\texttt{#1}[#2]\texttt{(}#3\texttt{)}}
\newcommand{\abopii}[4]{\texttt{#1}[#2][#3]\texttt{(}#4\texttt{)}}
\newcommand{\abopic}[4]{\texttt{#1}[#2]\texttt{\{}#3\texttt{\}(}#4\texttt{)}}
\newcommand{\abopp}[3]{\texttt{#1}\texttt{[}#2\texttt{](}#3\texttt{)}}
\newcommand{\abopc}[3]{\texttt{#1}\texttt{\{}#2\texttt{\}(}#3\texttt{)}}
\newcommand{\abopbc}[4]{\texttt{#1}\texttt{[}#2\texttt{]\{}#3\texttt{\}(}#4\texttt{)}}
\newcommand{\abopibc}[5]{\texttt{#1}[#2]\texttt{[}#3\texttt{]\{}#4\texttt{\}(}#5\texttt{)}}
\newcommand{\abopcc}[4]{\texttt{#1}\texttt{\{}#2\texttt{\}\{}#3\texttt{\}(}#4\texttt{)}}

% Types / candidate expansion types
\newcommand{\parr}[2]{#1 \rightharpoonup #2}
\newcommand{\aparr}[2]{\abop{parr}{#1; #2}}
\newcommand{\aceparr}[2]{\abop{ceparr}{#1; #2}}

\newcommand{\forallt}[2]{\forall #1.#2}
\newcommand{\aall}[2]{\abop{all}{#1.#2}}
\newcommand{\aceall}[2]{\abop{ceall}{#1.#2}}

\newcommand{\rect}[2]{\mu #1.#2}
\newcommand{\arec}[2]{\abop{rec}{#1.#2}}
\newcommand{\acerec}[2]{\abop{cerec}{#1.#2}}

\newcommand{\prodt}[1]{\langle #1 \rangle}
\newcommand{\aprod}[2]{\abopi{prod}{#1}{#2}}
\newcommand{\aceprod}[2]{\abopi{ceprod}{#1}{#2}}

\newcommand{\sumt}[1]{[#1]}
\newcommand{\asum}[2]{\abopi{sum}{#1}{#2}}
\newcommand{\acesum}[2]{\abopi{cesum}{#1}{#2}}

% Labels and maps
\newcommand{\labelset}{L}
\newcommand{\mapschema}[3]{\{#2 \hookrightarrow #1_{#2}\}_{#2 \in #3}}
\newcommand{\mapschemab}[4]{\{#3 \hookrightarrow #1_{#3}.#2_{#3}\}_{#3 \in #4}}
\newcommand{\mapschemax}[4]{\{#3 \hookrightarrow #1(#2_{#3})\}_{#3 \in #4}}
\newcommand{\mapschemabx}[5]{\{#4 \hookrightarrow #2_{#4}.#1(#3_{#4})\}_{#4 \in #5}}
\newcommand{\finmap}[1]{#1}
\newcommand{\mapitem}[2]{#1 \hookrightarrow #2}
\newcommand{\lbltxt}[1]{\mathtt{#1}}

% sequences
\newcommand{\seqschema}[4]{\{#1_{#2}\}_{#3 \leq #2 \leq #4}}
\newcommand{\seqschemaX}[1]{\seqschema{#1}{i}{1}{n}}
\newcommand{\seqschemaXx}[2]{\{#1(#2_i)\}_{1 \leq i \leq n}}

% Expanded/Unexpanded/Candidate expressions
\newcommand{\lam}[3]{\lambda #1{:}#2.#3}
\newcommand{\aelam}[3]{\abopc{elam}{#1}{#2.#3}}
\newcommand{\aulam}[3]{\abopc{ulam}{#1}{#2.#3}}
\newcommand{\acelam}[3]{\abopc{celam}{#1}{#2.#3}}

\newcommand{\ap}[2]{#1(#2)}
\newcommand{\aeap}[2]{\abop{eap}{#1; #2}}
\newcommand{\auap}[2]{\abop{uap}{#1; #2}}
\newcommand{\aceap}[2]{\abop{ceap}{#1; #2}}

\newcommand{\Lam}[2]{\Lambda #1.#2}
\newcommand{\aetlam}[2]{\abop{etlam}{#1.#2}}
\newcommand{\autlam}[2]{\abop{utlam}{#1.#2}}
\newcommand{\acetlam}[2]{\abop{cetlam}{#1.#2}}

\newcommand{\App}[2]{#1\texttt{[}#2\texttt{]}}
\newcommand{\aetap}[2]{\abopc{etap}{#2}{#1}}
\newcommand{\autap}[2]{\abopc{utap}{#2}{#1}}
\newcommand{\acetap}[2]{\abopc{cetap}{#2}{#1}}

\newcommand{\fold}[1]{\texttt{fold}(#1)}
\newcommand{\aefold}[3]{\abopc{efold}{#1.#2}{#3}}
\newcommand{\aufold}[3]{\abopc{ufold}{#1.#2}{#3}}
\newcommand{\acefold}[3]{\abopc{cefold}{#1.#2}{#3}}

\newcommand{\unfold}[1]{\texttt{unfold}(#1)}
\newcommand{\aeunfold}[1]{\abop{eunfold}{#1}}
\newcommand{\auunfold}[1]{\abop{uunfold}{#1}}
\newcommand{\aceunfold}[1]{\abop{ceunfold}{#1}}

\newcommand{\tpl}[1]{\langle #1\rangle}
\newcommand{\aetpl}[2]{\abopi{etpl}{#1}{#2}}
\newcommand{\autpl}[2]{\abopi{utpl}{#1}{#2}}
\newcommand{\acetpl}[2]{\abopi{cetpl}{#1}{#2}}

\newcommand{\prj}[2]{#1 \cdot #2}
\newcommand{\aepr}[2]{\abopp{epr}{#1}{#2}}
\newcommand{\aupr}[2]{\abopp{upr}{#1}{#2}}
\newcommand{\acepr}[2]{\abopp{cepr}{#1}{#2}}

\newcommand{\inj}[2]{#1 \cdot #2}
\newcommand{\aein}[4]{\abopibc{ein}{#1}{#2}{#3}{#4}}
\newcommand{\auin}[4]{\abopibc{uin}{#1}{#2}{#3}{#4}}
\newcommand{\acein}[4]{\abopibc{cein}{#1}{#2}{#3}{#4}}

\newcommand{\caseof}[2]{\texttt{case}~#1~#2}
\newcommand{\aecase}[4]{\abopic{ecase}{#1}{#2}{#3; #4}}
\newcommand{\aucase}[4]{\abopic{ucase}{#1}{#2}{#3; #4}}
\newcommand{\acecase}[4]{\abopic{cecase}{#1}{#2}{#3; #4}}

% Expanded expressions
\newcommand{\etxt}[1]{e_\text{#1}}

\newcommand{\Uofv}{\mathcal{U}}
\newcommand{\Uof}[1]{\Uofv(#1)}
\newcommand{\VTypofv}{\mathcal{V}_\mathsf{Typ}}
\newcommand{\VTypof}[1]{\VTypofv(#1)}
\newcommand{\VExpofv}{\mathcal{V}_\mathsf{EExp}}
\newcommand{\VExpof}[1]{\VExpofv(#1)}

% Statics of miniVerseU expanded expressions
\newcommand{\istypeU}[2]{#1 \vdash #2~\mathsf{type}}
\newcommand{\isctxU}[2]{#1 \vdash #2~\mathsf{ctx}}
\newcommand{\hastypeU}[4]{#1~#2 \vdash #3 : #4}
\newcommand{\hastypeUC}[2]{\vdash #1 : #2}
\newcommand{\hastypeUCO}[3]{#1 \vdash #2 : #3}

\newcommand{\Dhyp}[1]{#1~\mathsf{type}}
\newcommand{\Dcons}[2]{{#1}\cup{#2}}
\newcommand{\Ghyp}[2]{#1 : #2}
\newcommand{\Gcons}[2]{{#1}\cup{#2}}
\newcommand{\Gconsi}[2]{\cup_{#1} #2}

% Dynamics of miniVerseU
\newcommand{\isvalU}[1]{#1~\mathsf{val}}
\newcommand{\stepsU}[2]{#1 \mapsto #2}
\newcommand{\multistepU}[2]{#1 \mapsto^{*} #2}
\newcommand{\evalU}[2]{#1 \Downarrow #2}

% Unexpanded types
\newcommand{\utau}{{\hat\tau}}
\newcommand{\ut}{{\hat{t}}}

% Unexpanded expressions
\newcommand{\ue}{{\hat e}}
\newcommand{\ux}{{\hat x}}
\newcommand{\uesyntax}[4]{\texttt{syntax}~#1~\texttt{at}~#2~\{#3\}~\texttt{in}~#4}
\newcommand{\audefuetsm}[4]{\abopcc{usyntaxue}{#1}{#2}{#3.#4}}
\newcommand{\utsmap}[2]{#1~\texttt{/}#2\texttt{/}}
\newcommand{\autsmap}[2]{\texttt{uapuetsm}[#1]\texttt{[}#2\texttt{]}}
\newcommand{\uet}[1]{\ue_\text{#1}}
\newcommand{\ueparse}{\uet{parse}}

% TSM expressions
\newcommand{\tsmv}{a}
\newcommand{\utsmdef}[2]{\texttt{syntax}~@~#1~\texttt{\{}#2\texttt{\}}}
\newcommand{\istsmU}[2]{#1 \vdash #2~\mathsf{tsm}}

\newcommand{\uGamma}{\hat{\Gamma}}
\newcommand{\uDelta}{\hat{\Delta}}
\newcommand{\uD}{\mathcal{D}}
\newcommand{\uG}{\mathcal{G}}
\newcommand{\uDD}[2]{#1 \vert #2}
\newcommand{\uGG}[2]{#1 \vert #2}
\newcommand{\uGammaOK}[1]{#1 \vdash \mathsf{ok}}
\newcommand{\uDeltaOK}[1]{#1 \vdash \mathsf{ok}}

\newcommand{\tBody}{\mathtt{Body}}
\newcommand{\tParseResultExp}{\mathtt{ParseResultExp}} 
\newcommand{\tCEExp}{\mathtt{CEExp}} % Typed expansion
\newcommand{\expandsU}[6]{#1~#2 \vdash_{#3} #4 \leadsto #5 : #6} % there's a multiline one in the document done manually
\newcommand{\expandsUX}[3]{\expandsU{\Delta}{\Gamma}{\Sigma}{#1}{#2}{#3}}
\newcommand{\expandsUC}[3]{\vdash #1 \leadsto #2 : #3}
\newcommand{\domof}[1]{\text{dom}(#1)}
\newcommand{\xuetsmdef}[2]{\abop{uetsm}{#1;\,#2}}
\newcommand{\xuetsmbnd}[3]{#1 \hookrightarrow \xuetsmdef{#2}{#3}}
\newcommand{\uetsmenv}[2]{#1 \vdash #2~\mathsf{ok}}
\newcommand{\encodeBody}[2]{#1 \downarrow #2}
\newcommand{\decodeBody}[2]{#1 \uparrow #2}
\newcommand{\ebody}{\etxt{body}}
\newcommand{\eparse}{\etxt{parse}}
\newcommand{\ecand}{\etxt{cand}}
\newcommand{\decodeCondE}[2]{#1 \uparrow_\mathsf{CEExp} #2}
\newcommand{\encodeCondE}[2]{#1 \downarrow_\mathsf{CEExp} #2}

% Candidate Expansions
\newcommand{\ce}{\grave{e}}
\newcommand{\ctau}{\grave{\tau}}

\newcommand{\splicedt}[2]{\texttt{spliced}\langle#1, #2\rangle}
\newcommand{\acesplicedt}[2]{\texttt{cesplicedt}[#1; #2]}
\newcommand{\splicede}[2]{\texttt{spliced}\langle#1, #2\rangle}
\newcommand{\acesplicede}[2]{\texttt{cesplicede}[#1; #2]}
\newcommand{\splicedp}[2]{\texttt{spliced}\langle#1, #2\rangle}
\newcommand{\acesplicedp}[2]{\texttt{cesplicedp}[#1; #2]}

\newcommand{\mtau}{\dot{\tau}}
\newcommand{\mtspliced}[1]{\texttt{spliced}(#1)}

% Candidate expansion validation
\newcommand{\cvalidT}[4]{#1\vdash^{#2} #3 \leadsto #4~\mathsf{type}}
\newcommand{\cvalidE}[6]{#1~#2\vdash^{#3} #4 \leadsto #5 : #6}
\newcommand{\cvalidEX}[3]{\cvalidE{\Delta}{\Gamma}{\escenev}{#1}{#2}{#3}}
\newcommand{\escenev}{\mathbbmss{E}}
\newcommand{\tscenev}{\mathbbmss{T}}
\newcommand{\esceneU}[4]{#1;\,#2;\,#3;\,#4}
\newcommand{\esceneUP}[5]{#1;\,#2;\,#3;\,#4;\,#5}
\newcommand{\tsceneU}[2]{#1;\,#2}
\newcommand{\tsceneUP}[2]{\tsceneU{#1}{#2}}
\newcommand{\tsfrom}[1]{\mathsf{ts}(#1)}
\newcommand{\parseTyp}[2]{\mathsf{parseTyp}(#1)=#2}
\newcommand{\parseUExp}[2]{\mathsf{parseUExp}(#1)=#2}
\newcommand{\bsubseq}[3]{\mathsf{subseq}(#1; #2; #3)}

\newcommand{\sizeof}[1]{\Vert #1 \Vert}

% Pattern matching
\newcommand{\matchwith}[2]{\texttt{match}~#1~#2}
\newcommand{\aematchwith}[4]{\abopic{ematch}{#1}{#2}{#3; #4}}
\newcommand{\aumatchwith}[4]{\abopic{umatch}{#1}{#2}{#3; #4}}
\newcommand{\acematchwith}[4]{\abopic{cematch}{#1}{#2}{#3; #4}}

\newcommand{\matchrule}[2]{#1 \Rightarrow #2}
\newcommand{\aematchrule}[4]{\abopic{erule}{#1}{#2}{#3.#4}}
\newcommand{\aumatchrule}[4]{\abopic{urule}{#1}{#2}{#3.#4}}
\newcommand{\acematchrule}[4]{\abopic{cerule}{#1}{#2}{#3.#4}}

\newcommand{\ruleType}[5]{#1~#2 \vdash #3 : #4 \Mapsto #5}
\newcommand{\patType}[3]{#1 \Vdash #2 : #3}
\newcommand{\pctx}{\Omega}

\newcommand{\matchfail}[1]{#1~\mathsf{matchfail}}

\newcommand{\eR}{R}
\newcommand{\uR}{\hat{R}}
\newcommand{\cR}{\grave{R}}

\newcommand{\erv}{r}
\newcommand{\urv}{\hat{r}}
\newcommand{\crv}{\grave{r}}

\newcommand{\epv}{p}
\newcommand{\upv}{\hat{p}}
\newcommand{\cpv}{\grave{p}}

\newcommand{\wildp}{\_}
\newcommand{\aewildp}{\texttt{ewildp}}
\newcommand{\auwildp}{\texttt{uwildp}}
\newcommand{\acewildp}{\texttt{cewildp}}

\newcommand{\foldp}[1]{\abop{fold}{#1}}
\newcommand{\aefoldp}[1]{\abop{efoldp}{#1}}
\newcommand{\aufoldp}[1]{\abop{ufoldp}{#1}}
\newcommand{\acefoldp}[1]{\abop{cefoldp}{#1}}

\newcommand{\tplp}[1]{\langle #1 \rangle}
\newcommand{\aetplp}[2]{\abopi{etplp}{#1}{#2}}
\newcommand{\autplp}[2]{\abopi{utplp}{#1}{#2}}
\newcommand{\acetplp}[2]{\abopi{cetplp}{#1}{#2}}

\newcommand{\injp}[2]{#1 \cdot #2}
\newcommand{\aeinjp}[2]{\abopp{einp}{#1}{#2}}
\newcommand{\auinjp}[2]{\abopp{uinp}{#1}{#2}}
\newcommand{\aceinjp}[2]{\abopp{ceinp}{#1}{#2}}


\newcommand{\usyntaxup}[4]{\texttt{syntax}~#1~\texttt{at}~#2~\texttt{for~patterns}~\{#3\}~\texttt{in}~#4}
\newcommand{\audefuptsm}[4]{\abopcc{usyntaxup}{#1}{#2}{#3.#4}}
\newcommand{\auapuptsm}[2]{\texttt{uapuptsm}[#1]\texttt{[}#2\texttt{]}}

\newcommand{\expandsUP}[7]{#1~#2 \vdash_{#3;\,#4} #5 \leadsto #6 : #7} % there's a multiline one in the document done manually
\newcommand{\expandsUPX}[3]{\expandsUP{\Delta}{\Gamma}{\Sigma}{\Phi}{#1}{#2}{#3}}

\newcommand{\ruleExpands}[8]{#1~#2 \vdash_{#3;\,#4} #5 \leadsto #6 : #7 \Mapsto #8}
\newcommand{\patExpands}[5]{#1 \vdash_{#2} #3 \leadsto #4 : #5}
\newcommand{\xuptsmdef}[2]{\abop{uptsm}{#1;\,#2}}
\newcommand{\xuptsmbnd}[3]{#1 \hookrightarrow \xuptsmdef{#2}{#3}}
\newcommand{\uptsmenv}[2]{#1 \vdash #2~\mathsf{ok}}

\newcommand{\tParseResultPat}{\mathtt{ParseResultPat}} 
\newcommand{\tCEPat}{\mathtt{CEPat}} % Typed expansion

\newcommand{\decodeCEPat}[2]{#1 \uparrow_\mathsf{CEPat} #2}
\newcommand{\encodeCEPat}[2]{#1 \downarrow_\mathsf{CEPat} #2}

\newcommand{\cvalidP}[5]{\vdash^{#1;\,#2} #3 \leadsto #4 : #5}
\newcommand{\cvalidR}[7]{#1~#2 \vdash^{#3} #4 \leadsto #5 : #6 \Mapsto #7}
\newcommand{\pscenev}{\mathbbmss{P}}
\newcommand{\pscene}[2]{#1;\,#2}

\newcommand{\parseUPat}[2]{\mathsf{parseUPat}(#1)=#2}


% allow interrupted equation numbering
% taken from http://tex.stackexchange.com/questions/101002/interrupting-and-resuming-subequations
% \makeatletter
% \def\user@resume{resume}
% \def\user@intermezzo{intermezzo}
% %
% \newcounter{previousequation}
% \newcounter{lastsubequation}
% \newcounter{savedparentequation}
% \setcounter{savedparentequation}{1}
% % 
% \renewenvironment{subequations}[1][]{%
%       \def\user@decides{#1}%
%       \setcounter{previousequation}{\value{equation}}%
%       \ifx\user@decides\user@resume 
%            \setcounter{equation}{\value{savedparentequation}}%
%       \else  
%       \ifx\user@decides\user@intermezzo
%            \refstepcounter{equation}%
%       \else
%            \setcounter{lastsubequation}{0}%
%            \refstepcounter{equation}%
%       \fi\fi
%       \protected@edef\theHparentequation{%
%           \@ifundefined {theHequation}\theequation \theHequation}%
%       \protected@edef\theparentequation{\theequation}%
%       \setcounter{parentequation}{\value{equation}}%
%       \ifx\user@decides\user@resume 
%            \setcounter{equation}{\value{lastsubequation}}%
%          \else
%            \setcounter{equation}{0}%
%       \fi
%       \def\theequation  {\theparentequation  \alph{equation}}%
%       \def\theHequation {\theHparentequation \alph{equation}}%
%       \ignorespaces
% }{%
% %  \arabic{equation};\arabic{savedparentequation};\arabic{lastsubequation}
%   \ifx\user@decides\user@resume
%        \setcounter{lastsubequation}{\value{equation}}%
%        \setcounter{equation}{\value{previousequation}}%
%   \else
%   \ifx\user@decides\user@intermezzo
%        \setcounter{equation}{\value{parentequation}}%
%   \else
%        \setcounter{lastsubequation}{\value{equation}}%
%        \setcounter{savedparentequation}{\value{parentequation}}%
%        \setcounter{equation}{\value{parentequation}}%
%   \fi\fi
% %  \arabic{equation};\arabic{savedparentequation};\arabic{lastsubequation}
%   \ignorespacesafterend
% }
% \makeatother

\begin{document} 
\frontmatter

%initialize page style, so contents come out right (see bot) -mjz
\pagestyle{empty}

\title{\textbf{Reasonably Programmable Syntax}}
\author{Cyrus Omar}
\date{\today}
\Year{2017}
\trnumber{CMU-CS-17-113}

\committee{Jonathan Aldrich, Chair\\
Robert Harper\\
Karl Crary\\
Eric Van Wyk, University of Minnesota}

\support{This research was supported by the DOE Computational Science Graduate Fellowship, the NSF Graduate Research Fellowship, by AFRL and DARPA under agreement \#FA8750-16-2-0042 and by NSA lablet contract \#H98230-14-C-0140.
% Any opinions or recommendations 
%expressed in this material are those of the author and do not necessarily
 %reflect the views of the DOE, NSF, AFRL, DARPA or NSA.
 }

% \disclaimer{}
\permission{This work is licensed under the Creative Commons Attribution 4.0 International License. To view a copy of this license, visit \url{http://creativecommons.org/licenses/by/4.0/}.}

% copyright notice generated automatically from Year and author.
% permission added if \permission{} given.

\keywords{syntax, notation, parsing, type systems, module systems, macro systems, hygiene, pattern matching, bidirectional typechecking, implicit dispatch}

\maketitle

% \begin{dedication}
% Dedicated to the memory of Daniel Schreiber (1986 -- 2010), my friend.
% \end{dedication}

\pagestyle{plain} % for toc, was empty

\begin{abstract}
\noindent
Programming languages commonly provide ``syntactic sugar'' that decreases the syntactic cost of working with certain standard library constructs.   
%, i.e. {derived forms} that decrease the cognitive cost of idioms involving select library constructs. 
For example, Standard ML builds in syntactic sugar for constructing and pattern matching on lists. %This decreases the cognitive cost of working with lists. %Semantically, lists are defined in the SML Basis library (i.e. SML's ``standard library''.)
Third-party library providers are, justifiably, envious of this special arrangement. After all, it is not difficult to find other examples of situationally useful library-specific syntactic sugar \cite{TSLs}. For example, 1) clients of a ``collections'' library might like syntactic sugar for finite sets and dictionaries; 2) clients of a ``web programming'' library might like syntactic sugar for HTML and JSON values; 3) a compiler writer might like syntactic sugar for the terms of the object language or various intermediate languages of interest; and 4) clients of a ``chemistry'' library might like syntactic sugar for chemical structures based on the SMILES standard \cite{anderson1987smiles}.

Defining a ``library-specific'' syntax dialect in each of these situations is problematic, because library clients cannot combine dialects like these in a manner that conserves syntactic determinism (i.e. syntactic conflicts can and do arise.) Moreover, it can become difficult for library clients to reason abstractly about types and binding when examining the text of a program that uses unfamiliar forms. Typed, hygienic term-rewriting macro systems, like Scala's macro system \cite{ScalaMacros2013}, while somewhat more reasonable, offer limited control over parsing.

% In other words, there are few clear \emph{abstract reasoning principles} available to client programmers. % As such, the dialect-oriented approach is difficult to reconcile with the best practices of  ``programming in the large.''

This thesis formally introduces \emph{typed literal macros (TLMs)}, which give library providers the ability to programmatically control the parsing and expansion, at ``compile-time'', of expressions and patterns of \emph{generalized literal form}. Library clients can use any combination of TLMs in a program without needing to consider the possibility of syntactic conflicts between them,  because the context-free syntax of the language is never extended (rather, it is  contextually repurposed.) Moreover, the language validates each expansion that a TLM generates in order to maintain useful abstract reasoning principles. Most notably, expansion validation maintains:
\begin{itemize}
\item a \emph{type discipline}, meaning that the client can reason about types while holding the literal expansion abstract; and 
\item a \emph{strictly hygienic binding discipline}, meaning that the client can always be sure that:
  \begin{enumerate}
    \item spliced terms, i.e. terms that appear within literal bodies, cannot capture bindings hidden within the literal expansion; and 
  \item the literal expansion does not refer to definition-site or application-site bindings directly. Instead, all interactions with bindings external to the expansion go explicitly through {spliced terms} or {parameters}.% Support for partial parameter application helps reduce the syntactic cost of this explicit parameter passing style.
  \end{enumerate}
\end{itemize}
\noindent
In short, we formally define a programming language in the ML tradition with a \emph{reasonably} programmable syntax.

%We discuss both explicit application of TLMs (with support for partial parameter application) and implicit, type-directed application of TLMs, which further reduces cognitive cost.
\end{abstract}

\begin{acknowledgments}
I owe a tremendous debt of gratitude to my advisor, Jonathan Aldrich, for being willing to take me on as a na\"ive neuroscience student interested in designing programming languages, and for guiding me patiently through many years of learning, experimentation and refinement. 
Jonathan's depth of expertise and breadth of perspective has been invaluable. Thank you.%My work seeks to apply type theory to solve usability problems, so it has been a great fit.

I would also like to thank Bob Harper and Karl Crary, who both generously served on my thesis committee and substantially influenced my approach. Through their teaching and scholarship, they taught me the type-theoretic foundations of programming languages, and more broadly, they taught me the tremendous value of precision in both formal and informal discourse on language design. These lessons were reinforced during long afternoons discussing theory papers with other POP students in the ConCert reading group,  and during long evenings grading and preparing for 15-150 (Functional Programming) and 15-312 (Principles of Programming Languages) with Dan Licata, Ian Voysey, Bob Harper, Shayak Sen and the rest of the course staff. Thank you all for being uncompromisingly mathematical in your approach.

I have also learned a great deal about the psychological and social aspects of software development from Brad Myers of the HCI Institute and from the faculty and students of the Institute for Software Research (ISR), particularly Thomas LaToza and Joshua Sunshine. In addition, I have collaborated with Alex Potanin, who visited us on sabbatical, and with students Darya Melicher, Ligia Nistor, Benjamin Chung and Chenglong Wang on projects related to the work presented here. Thank you all for broadening my perspective on the art and science of language design. 

During my time in graduate school, I have had the privilege to attend a great many  conferences, workshops and summer schools where I participated in more illuminating conversations than I could  hope to recall here. I am particularly grateful for the conversations with Eric Van Wyk, who never failed to appreciate the subtle contours of a design space and graciously served on my thesis committee. I would also like to thank the organizers and participants of the Oregon PL Summer School, where I had an amazing time learning how to properly prove the proper theorems. Finally, I am grateful to have collaborated with Ian Voysey, Michael Hilton and Matthew Hammer on Hazelnut, a side project that quite successfully delayed the completion of this dissertation. With friends and collaborators like them, why graduate?

I would be remiss not to mention Brent Doiron, who took me as a student when I entered graduate school in the Neural Computation PhD program, and Garrett Kenyon,  who was my practicum supervisor during my ``man vs. wild'' stint at Los Alamos National Lab. Both of them left me with a deep appreciation for the  mathematical and computational methods used to study the dynamics of neurobiological systems. I hope some day, far in the future, to return to neuroscience with a pack full of truly modern programming tools.

I also want to explicitly acknowledge Deb Cavlovich, Catherine Copetas, Victoria Poprocky and all of the other staff that keep things running smoothly around the school, and at conferences and other events. I really appreciate all of the work that you do.

Graduate school can be an emotionally taxing experience, to say the least. Fortunately, my friends were there whenever I came up for air -- sometimes after going under for weeks at a time. Tommy and Sanna, you are beautiful souls and our travels together have been incredibly rejuvenating. D, you have the most exquisite taste and it's so good to know you (yeh yeh yeh.) To the greater Miasma crew, Ian, V, Tom7 and the rest of the thursdz crew, and my former officemate, Harsha: yinz are such fascinating people and I really enjoy the time we spend together. The same goes for so many other individuals that I've connected with personally, whether at conferences, at office hours, through the CNBC, CSGF, LANL, WRCT, SCS, on Twitter, at shows and festivals, in the woods, or in apartments and backyards. You've made these years wonderfully memorable.

I especially want to remember Dan Schreiber. Dan was one of my very closest friends, a romantic visionary and the greatest debate partner I have ever had. He died in 2010. I am so glad to have known him and I only wish that I could have heard his take on so many of the topics I've learned about since then -- proof theory, type theory, tech cooperatives, experimental music, long-distance cycling, old books, psychedelic films,  spontaneous theater and colonizing Venus, to name just a few! Dan is truly missed.

Finally, so much of who I am is due to the love and support of my family. The diverse personalities of my aunts, uncles, cousins and their spouses made family gatherings so lively. My sister, Elisha, and now her husband, Pat, have been an endless source of great book recommendations and conversations. And I am forever grateful to Ami and Hibbi Abu, my mother and father, who have given me so much love, offered so many heartfelt prayers and provided me with so much practical assistance and advice throughout my life. From an early age, they encouraged me to adopt only the best practices of the cultures around me, and to remember the Big Picture at all times. Those lessons, rooted in the traditions of our family going back generations, have proven incredibly valuable in research, and in life, time and time again.

It's been an unforgettable journey. Thanks everyone. 
\end{acknowledgments}


\tableofcontents
\listoffigures
%\listoftables

\mainmatter

% The other requirements Catherine has:
%
%  - avoid large margins.  She wants the thesis to use fewer pages, 
%    especially if it requires colour printing.
%
%  - The thesis should be formatted for double-sided printing.  This
%    means that all chapters, acknowledgements, table of contents, etc.
%    should start on odd numbered (right facing) pages.
%
%  - You need to use the department standard tech report title page.  I
%    have tried to ensure that the title page here conforms to this
%    standard.
%
%  - Use a nice serif font, such as Times Roman.  Sans serif looks bad.
%
% Other than that, just make it look good...

% !TEX root = omar-thesis.tex
\chapter{Introduction}\label{chap:intro}
\begin{quote}\textit{The recent development of programming languages suggests that the simul\-taneous achievement of simplicity 
and generality in language design is a serious unsolved 
problem.}\begin{flushright}--- John Reynolds (1970) \cite{Reynolds70}\end{flushright}
\end{quote}
\section{Motivation}\label{sec:intro-motivation}

%``Full-scale' typed functional programming languages like Standard ML (SML) \cite{mthm97-for-dart,harper1997programming}, OCaml \cite{ocaml-manual} and Haskell \cite{jones2003haskell} are members of a conceptual lineage rooted in the typed lambda calculus \cite{Reynolds94anintroduction}. 

Programming languages come in many sizes. Small languages -- i.e. ``formal {calculi}'' -- allow language designers to  study the mathematical properties of language primitives of interest in isolation. These studies then inform the design of ``full-scale'' 
%\footnote{Throughout this work, words and phrases that should be read as having an intuitive or informal meaning, rather than a strict mathematical meaning, will be introduced with quotation marks.} 
 languages, which combine several such primitives, or generalizations thereof.

Because small-scale languages are of interest mainly as objects of mathematical study, their designers often choose to specify only the abstract syntax of their primitives (or, when typesetting documents, stylized representations thereof). Full-scale languages, on the other hand, are both interesting objects of mathematical study and, ideally, useful for write large programs, so they  typically also specify a more ``programmer-friendly'' textual concrete syntax that features various \emph{derived syntactic forms}, i.e. forms defined by a context-independent ``desugaring'' to the set of base forms, that decrease the syntactic cost of certain common idioms. 
For example, Standard ML (SML) \cite{mthm97-for-dart,harper1997programming}, OCaml \cite{ocaml-manual} and Haskell \cite{jones2003haskell}   build in 
%record types, generalizing the nullary and binary product types that suffice in simpler calculi, because labeled components are cognitively useful to human programmers. Similarly, these languages all build in 
derived forms that decrease the syntactic cost of working with lists. In these languages, the form \lstinline{[1, 2, 3, 4, 5]} desugars to: 
\begin{lstlisting}[numbers=none]
Cons(1, Cons(2, Cons(3, Cons(4, Cons(5, Nil)))))
\end{lstlisting}

The hope amongst many language designers is that a limited number of derived forms like these will suffice to produce a ``general-purpose'' programming language, i.e. one that satisfies programmers working in a wide variety of application domains. Unfortunately, a stable language design that fully achieves this ideal has yet to emerge, as evidenced by the diverse array of \emph{syntactic dialects} -- dialects that introduce only new derived forms -- that continue to proliferate around all major contemporary languages. For example, Ur/Web is a syntactic dialect of Ur (an ML-like full-scale language \cite{conf/pldi/Chlipala10}) that builds in derived forms for SQL queries, HTML elements and other datatypes used in the domain of web programming \cite{conf/popl/Chlipala15}. %Syntactic cost is often assessed qualitatively \cite{green1996usability}, though quantitative metrics can be defined. 
We will consider a large number of other examples of syntactic dialects in Sec. \ref{sec:motivating-examples}. 
Tools like Camlp4 \cite{ocaml-manual}, Sugar* \cite{erdweg2011sugarj,erdweg2013framework} and Racket's preprocessor \cite{Flatt:2012:CLR:2063176.2063195}, which we will discuss in Sec. \ref{sec:existing-approaches}, have decreased the engineering costs of constructing syntactic dialects, further contributing to their proliferation. 

%In fact, tools that aid in the construction of so-called  ``domain-specific'' language dialects (DSLs)\footnote{In some parts of the literature, such dialects are called ``external DSLs'', to distinguish them from  ``internal'' or ``embedded DSLs'', which are actually  library interfaces that only ``resemble'' distinct dialects \cite{fowler2010domain}.} seem only to be becoming more prominent over time. 

%\subsection{Why are there so many language dialects?}
%{This calls for an investigation}: why is it that programmers and researchers are still so often unable to satisfyingly express the constructs that they seek in libraries, as modes of use of the ``general-purpose'' primitives already available in major languages today, and instead see a need for new language dialects?

%Perhaps the most common sort of dialect is the \emph{syntactic dialect} -- a dialect that introduces only new derived syntactic forms, motivated by a desire to decrease the {syntactic cost} of working with one or more library constructs of interest. 
%Put another way, syntactic dialects can be specified by a context-independent expansion to the existing language that they are based on. 
%For example, Ur/Web is a syntactic dialect of Ur (a language that itself descends from ML \cite{conf/pldi/Chlipala10}) that builds in derived forms for SQL queries, HTML elements and other datatypes used in the domain of web programming \cite{conf/popl/Chlipala15}. %Syntactic cost is often assessed qualitatively \cite{green1996usability}, though quantitative metrics can be defined. 
%This is not an isolated example -- we will consider a number of additional types of data that similarly stand to benefit from the availability of specialized derived forms in Sec. \ref{sec:motivating-examples}. 
%Tools like Camlp4 \cite{ocaml-manual}, Sugar* \cite{erdweg2011sugarj,erdweg2013framework} and Racket \cite{Flatt:2012:CLR:2063176.2063195}, which we will discuss in Sec. \ref{sec:existing-approaches}, have lowered the engineering costs of constructing syntactic dialects in such situations, further contributing to their proliferation. 

%More advanced dialects introduce new type structure, going beyond what is possible with only new derived forms. As a simple example, the static and dynamic semantics of records cannot be expressed by context-independent expansion to a language with only nullary and binary products. Various languages have explored ``record-like'' primitives that go further, supporting functional update operators, width and depth coercions (sometimes implicit)%\cite{Cardelli:1984:SMI:1096.1098}
%, methods, prototypic dispatch and other such ``semantic embellishments'' that in turn cannot be expressed by context-independent expansion to a language with only standard record types (we will detail an  example in Sec. \ref{sec:metamodules-motivating-examples}). OCaml primitively builds in the type structure of polymorphic variants, open datatypes and  operations that use format strings like $\mathtt{sprintf}$ \cite{ocaml-manual}. ReactiveML builds in primitives for functional reactive programming \cite{mandel2005reactiveml}. ML5 builds in high-level primitives for distributed programming based on a modal lambda calculus \cite{Murphy:2007:TDP:1793574.1793585}. Manticore \cite{conf/popl/FluetRRSX07} and AliceML  \cite{AliceLookingGlass} build in parallel programming primitives with a more elaborate type structure than is found in simpler accounts of parallelism. 
%MLj builds in the type structure of the Java object system (motivated by a desire to interface safely and naturally with Java libraries) \cite{Benton:1999:IWW:317636.317791}. Other dialects do the same for other foreign languages, e.g. Furr and Foster describe a dialect of OCaml that builds in the type structure of C \cite{Furr:2005:CTS:1065010.1065019}. Tools like proof assistants and logical frameworks are used to specify and reason metatheoretically about dialects like these, and tools like compiler generators and language frameworks \cite{erdweg2013state} lower their implementation cost, again contributing to their proliferation. 

\subsection{Dialects Considered Harmful}
% express record types as syntactic sugar over the simply-typed lambda calculus with  binary product types.\footnote{Pairs can of course be expressed as syntactic sugar atop records, though one could argue that using binary products as the more primitive concept is simpler.} The static semantics need to be extended with new type and term operators. However, the simplest way to express the dynamic semantics of the newly introduced term operators is by translation to nested binary products, so we can leave the operational semantics alone. \todo{fill this out} %For example, there are dozens of constructs that go by the name of ``records'' in various languages, each defined by a slightly different collection of primitive operations. \todo{examples} %, encouraged  historically  by the availability of tools like compiler generators and,  more recently, language workbenches \cite{workbenches} and DSL frameworks \cite{dsl}. Unfortunately, taking this approach makes it substantially more difficult for clients to import high-level abstractions orthogonally. 
% test 
Some  view this proliferation of dialects as harmless or even as desirable, arguing that programmers can simply choose the right dialect for the job at hand \cite{journals/stp/Ward94}. However, this ``dialect-oriented'' approach is, in an important sense, anti-modular: programmers cannot always ``combine'' different dialects when they want to use the primitives that they feature together within a single program. For example, a programmer might have access to a dialect featuring HTML syntax and to a dialect featuring regular expression syntax, but it is not always straightforward to, from these, construct a dialect featuring both. Both HTML and regular expression syntax might be useful when constructing, for example, a web-based bioinformatics tool. 

In some cases, constructing the desired ``combined dialect'' is difficult simply because the constituent dialects are specified using different formalisms. In other cases, the constituent dialects may be specified using a formalism that does not operationalize the notion of dialect combination (e.g. Racket's preprocessor \cite{Flatt:2012:CLR:2063176.2063195}). But even if we restrict our interest to dialects specified using a formalism that does operationalize some notion of dialect combination (or, equivalently, one that allows programmers to combine ``dialect fragments''), there may still be a problem: the formalism may not guarantee that the combined dialect will conserve important properties that can be established about the dialects in isolation. %In other words, any putative ``combined language'' must formally be considered a  distinct system for which one must derive essentially all metatheorems of interest anew, guided only informally by those derived for the dialects individually. %There is no well-defined mechanism for constructing such a ``combined language'' in general. 
For example, consider two syntactic dialects specified using Camlp4, one specifying derived syntax for finite mappings, the other specifying overlapping syntax for \emph{ordered} finite mappings. Though each dialect has a deterministic grammar, when these grammars are na\"ively  combined, syntactic ambiguities will arise. We are aware of only one formalism that guarantees that determinism is conserved when syntactic dialects are combined \cite{conf/pldi/SchwerdfegerW09}, but it has limited expressive power, as we will discuss in Sec. \ref{sec:direct-syntax-extension}.
%It is thus infeasible to simply allow different contributors to a software system to choose their own favorite dialect for each component they are responsible for. 
%It it clear that dialects are better rhetorical devices than practical engineering artifacts. 

%Due to this paucity of modular reasoning principles, the ``dialect-oriented'' approach is problematic for software development ``in the large''. %Large software projects and software ecosystems must pick a single language that does provide powerful modular reasoning principles and, to benefit from them, stay inside it.

\subsection{Large Languages Considered Harmful}
Dialects do sometimes have a less direct influence on large-scale software development: they can help convince the designers in control of comparatively popular languages, like OCaml and Scala, to include some variant of the primitives that they feature into backwards-compatible language revisions. %These decisions are increasingly influenced by community processes, e.g. the Scala Improvement Process.  %This approach concentrates power as well as responsibility over maintaining metatheoretic guarantees in the hands of a small group of language designers, though increasingly influenced by various community processes (e.g. the Scala Improvement Process). 
%Dialects thus serve the role of rhetorical vehicles for new ideas, rather than direct artifacts. 
%Over time, accepting such extensions has caused these languages to balloon in size. 
This \emph{ad hoc} approach is unsustainable, for three main reasons. First, as we will demonstrate in Sec. \ref{sec:motivating-examples}, there are simply too  many potentially useful such primitives, and many of these capture idioms common only in relatively narrow application domains. It is unreasonable to expect language designers to be able to evaluate all of these use cases in a timely and informed manner. Second, primitives introduced earlier in a language's lifespan can end up monopolizing finite ``syntactic resources'', forcing subsequent primitives to use ever more esoteric forms. And third, primitives that prove after some time to be flawed in some way cannot be removed or modified without breaking backwards compatibility. For these reasons, language designers are justifiably reticent to add new primitives to major languages.%Because there is often no empirical data about how useful a construct is in practice until it is available in a major language, decisions about which constructs to include are often informed only by intuition (and are thus)
%Recalling the words of  Reynolds, which are clearly as relevant today as they were almost half a century ago \cite{Reynolds70}: %This approach is antithetical to the ideal of a truly \emph{general-purpose language} described at the beginning of this section.
%\newpage

\subsection{Toward More General Primitives}
This leaves two possible paths forward. One is to simply eschew ``niche'' primitives and settle on the existing designs, which might be considered to sit at a ``sweet spot'' in the overall language design space (accepting that in some circumstances, this leads to  high syntactic cost). 
The other path forward is to search for a small number of highly general primitives that allow us degrade many of the constructs that are built primitively into languages and their dialects today instead to modular library constructs. 
Encouragingly, primitives of this sort do occasionally arise. For example, a recent revision of OCaml added support for  generalized algebraic data types (GADTs), based on research on guarded recursive datatype constructors \cite{XiCheChe03}. Using GADTs, OCaml was able to move some of the \emph{ad hoc} machinery for typechecking operations that use format strings, like \li{sprintf}, out of the language and into a library. Syntactic machinery related to \li{sprintf}, however, remains built in. 

%Similarly, it recently introduced ``open datatypes'', which subsume its previous more specialized exception type, and captures many use cases for .

%Viewed ``dually'', one might equivalently ask for a language that builds in a core that is as small as possible, but provides expressive power comparable to languages with much larger cores. This is our goal in the work being proposed

%Similarly, it recently introduced ``open datatypes'', which subsume its previous more specialized exception type, and captures many use cases for .

%Viewed ``dually'', one might equivalently ask for a language that builds in a core that is as small as possible, but provides expressive power comparable to languages with much larger cores. This is our goal in the work being proposed. 

%\vspace{-10px}
\section{Overview of Contributions}\label{sec:contributions}
%%Our broad aim in the work being proposed is to introduce primitive language mechanisms that give library providers the ability to  express new syntactic expansions as well as new types and operators in a safe and modularly composable manner. 
Our aim in this work is to introduce primitive language constructs that reduce the need for syntactic dialects and \emph{ad hoc} derived syntactic forms. In particular, we introduce the following primitives:
% By supporting the primitives that we introduce, 1) VerseML will be smaller than comparable languages like ML and Scala, and 2) dialect formation will be less frequently necessary. In other words, these primitives reduce the need for many others:%This eliminates the needs to build in fewer \emph{ad hoc} constructs and dialects are less frequently necessary. %thereby reducing the need for language dialects and revisions. 

%VerseML features a module system taken directly from SML. Unlike SML, the VerseML core language is split into a \emph{typed external language} (EL) specified by {type-directed translation} to a minimal \emph{typed internal language} (IL). 
\begin{enumerate}
\item \textbf{Typed syntax macros}, or \textbf{TSMs}. TSMs are applied like functions to \emph{generalized literal forms} to programmatically control their  parsing and expansion. This occurs statically (i.e. simultaneously with typing). We  introduce TSMs first for a simple language of expressions and types in Chapter \ref{chap:tsms}, then add support for pattern matching  in Chapter \ref{sec:pattern-tsms} and type and module parameters in Chapter \ref{sec:tsms-parameterized}.
\item \textbf{Type-specific languages}, or \textbf{TSLs}. TSLs, described in Chapter \ref{chap:tsls}, further reduce syntactic cost by allowing library providers to designate a privileged TSM for each type that they introduce. Library clients can then rely on local type inference to invoke that TSM and apply its parameters implicitly. TSLs can reduce the syntactic cost of an idiom to very nearly the same extent that a special-purpose dialect can, while avoiding the problems described above.
%\item \textbf{Metamodules}, introduced in Sec. \ref{sec:metamodules}, reduce the need to primitively build in the type structure of constructs like records (and variants thereof),  labeled sums and other interesting constructs that we will introduce later by giving library providers programmatic ``hooks'' directly into the semantics, which are specified as a \emph{type-directed translation semantics} targeting a small \emph{typed internal language} (introduced in Sec. \ref{sec:VerseML}). %For example, a library provider can implement the type structure of records with a metamodule that:
%\begin{enumerate}
%\item introduces a type constructor, \lstinline{record}, parameterized by finite mappings from labels to types, and defines, programmatically, a translation to unary and binary product types (which are built in to the internal language); and 
%\item introduces operators used to work with records, minimally record introduction and elimination (but perhaps also various functional update operators), and directly implements the logic governing their typechecking and translation to the IL (which builds in only nullary and binary products). 
%\end{enumerate}
%We will see direct analogies between ML-style modules (which our mechanisms also support) and metamodules later.
\end{enumerate} 

As vehicles for this work, we will specify a small-scale typed lambda calculus in each of the chapters just mentioned, each building upon the previous one. For the sake of examples, we will also describe (but not formally specify) a full-scale functional language called VerseML.\footnote{We distinguish VerseML from Wyvern, which is the language described in our prior publications about some of the work that we will describe, because Wyvern is a group effort evolving independently.} VerseML is, as its name suggests, a dialect of ML. It diverges from other dialects of ML that have a similar underlying type structure, like Standard ML and OCaml, in that it uses a local type inference scheme \cite{Pierce:2000:LTI:345099.345100} (like, for example, Scala \cite{OdeZenZen01}) for reasons that have to do with the mechanisms described in Chapter \ref{chap:tsls}. The reason we will not follow Standard ML \cite{mthm97-for-dart} in giving a complete formal specification of VerseML in this work is both to emphasize that the primitives we introduce are fairly insensitive to the details of the underlying type structure of the language (so TSMs can be considered for inclusion in a variety of languages, not only dialects of ML), and to avoid distracting the reader (and the author) with specifications of primitives that are already well-understood in the literature and that are orthogonal to those that are the focus of this work. %We anticipate that future full-scale language specifications will be able to combine the ideas  in the proposed work without trouble. %The purpose of the work being proposed is to serve as a reference for those interested in the new constructs we introduce, not to serve as a language specification. 
%We will give a brief overview of these languages are organized in Sec. \ref{sec:VerseML}.

%TSMs, like other macro systems, perform \emph{static code generation} (also sometimes called \emph{static} or \emph{compile-time metaprogramming}), meaning that the relevant rules in the static semantics of the language call for the evaluation of \emph{static functions} that generate term encodings. Static functions are functions that are evaluated statically, i.e. during typing. %Library providers write these static functions using the VerseML \emph{static language} (SL).  
%Maintaining a separation between the static (or ``compile-time'') phase and the dynamic (or ``run-time'') phase is an important facet of VerseML's design. % static code generation. %We will  also introduce a simple variant of each of these primitives that leverages VerseML's support for local type inference to further reduce syntactic cost in certain common situations. 


The main challenge will come in maintaining the following:
\begin{itemize}
\item a \emph{type discipline}, meaning that the language must be type safe, and that programmers examining a well-typed expression must be able to determine its type without examining its expansion; 
\item a \emph{hygienic binding discipline}, meaning that the expansion logic must not be permitted to make ``hidden assumptions'' about the names of variables at macro application sites, nor  introduce ``hidden bindings'' into other terms; and 
\item \emph{modular reasoning principles}, meaning that library providers must have the ability to reason about the syntax that they have defined in isolation, and clients must be able to use macros safely in any combination, without the possibility of conflict.\footnote{This is not quite true --  name clashes of the usual sort can arise. We will tacitly assume that in practice, they can be avoided extrinsically, e.g. by using a URI-based naming scheme as in the Java ecosystem.} 
\end{itemize}
\noindent
We will, of course, make these notions more technically precise as we continue.

\subsection*{Thesis Statement}
In summary, this work defends the following statement:

\begin{quote}
A functional programming language can give library providers the ability to %meta\-pro\-gram\-matic\-ally 
express new syntactic expansions while maintaining a type discipline, a hygienic binding discipline and modular reasoning principles. %These  primitives are  expressive enough to subsume the need for a variety of primitives that are, or would need to be, built in to comparable contemporary languages.
\end{quote}
\section{Disclaimers}
Before we continue, it may be prudent to explicitly acknowledge that completely eliminating the need for dialects would indeed be asking for too much: certain language design decisions are fundamentally incompatible with others or require coordination across a language design. We aim only to decrease the need for syntactic dialects in this work. We will not consider situations that require modifications to the underlying type structure of a language (though this is a rich avenue for future work). % out a larger design space within a single language, VerseML.%a subset of constructs that can be specified by a semantics of a certain ``shape'' specified by VerseML (we will make this more specific later). %There is nothing ``universal'' about VerseML.

It may also be useful to explicitly acknowledge that library providers could leverage the primitives we introduce   to define constructs that are in ``poor taste''. We  expect that in practice, VerseML will come with a standard library defining an expertly curated collection of standard constructs, as well as guidelines for advanced users regarding when it would be sensible to use the mechanisms we introduce (following the example of languages that support operator overloading or type classes \cite{Hall:1996:TCH:227699.227700}, which also have some potential for ``abuse'' or ``overuse''). %For most programmers, using VerseML should not be substantially different from using a language like ML or one of its dialects.%The vast majority of programmers should not use the primitives that we introduce directly.

%Finally, VerseML is not designed as a dependently-typed language like Coq, Agda or Idris. %because these languages do not maintain a phase separation between ``compile-time'' and ``run-time.'' This phase separation is useful for programming tasks (where one would like to be able to discover errors before running a program, particularly programs that may have an effect) but less so for theorem proving tasks (where it is mainly the fact that a pure expression is well-typed that is of interest, by the propositions-as-types principle). 


\chapter{Background}\label{chap:background}
\vspace{-6px}
\begin{quote}\textit{The recent development of programming languages suggests that the simul\-taneous achievement of simplicity 
and generality in language design is a serious unsolved 
problem.}\begin{flushright}John Reynolds (1970) \cite{Reynolds70}\end{flushright}
\end{quote}
\vspace{-6px}
%VerseML, like most contemporary full-scale programming languages, has a textual concrete syntax (we will consider the topic of non-textual display forms as future work in Sec. \ref{sec:non-textual-display-forms}).
%\footnote{Although Wyvern specified a layout-sensitive concrete syntax, to avoid unnecessary distractions, we will describe a more conventional layout-insensitive concrete syntax for VerseML.} %We have chosen to specify a layout-sensitive textual concrete syntax (i.e. newlines and indentation are not ignored). This design choice is not  fundamental to our proposed contributions, but it will be useful for cleanly expressing a class of examples that we plan to discuss later. We plan to specify some novel aspects of VerseML's concrete syntax with an \emph{Adams grammar} \cite{Adams:2013:PPI:2429069.2429129} (such a specification for Wyvern, which has a very similar syntax, can be found in \cite{TSLs}), but for the purposes of this proposal, we will simply introduce VerseML's concrete syntax by example as we go on. %For constructs that have an obvious analog in ML, we will omit a detailed explanation.
%Because the purpose of concrete syntax is to serve as the programmer-facing user interface to the language, it is common practice to build in  derived syntactic forms (colloquially, \emph{syntactic sugar}) that capture common idioms more concisely (i.e. at lower \emph{syntactic cost}) or ``naturally'' (i.e. at lower \emph{cognitive cost}, which is usually considered qualitatively \cite{green1996usability}). % (i.e. considering cognitive dimensions \cite{green1996usability}). 
%For example, derived list syntax is built in to many functional languages, so that instead of having to write out 
%\begin{lstlisting}[numbers=none]
%Cons(1, Cons(2, Cons(3, Nil)))
%\end{lstlisting}
%the programmer can equivalently write \lstinline{[1, 2, 3]}. Many languages and dialects thereof go beyond this, building in derived syntax associated with various other types of data, like vectors (the SML/NJ dialect of SML), arrays (OCaml), monadic commands (Haskell), syntax trees (Scala, F\#), XML trees (Scala, Ur/Web) and SQL queries (F\#, Ur/Web). We will begin by describing these and several other examples in more detail in Sec. \ref{sec:motivating-examples}. %This is a rather \emph{ad hoc} process.% discussed previously, the usual approach is to require that the language designer build in new derived syntactic forms. %The desugaring from the latter to the former is specified by the language itself. %Typically, the language designer controls what forms of derived syntax are built in to the language.

%VerseML will take a less \emph{ad hoc} approach -- rather than privileging particular library constructs with primitive syntactic support, VerseML exposes primitives that allow library providers to introduce new expansion logic on their own, in a modular manner. Before describing these primitives in the remaining chapters, we will survey existing approaches to the problem of reducing syntactic cost (and in so doing, highlight some of the problems that we aim to resolve in our work) in Sec. \ref{sec:existing-approaches}. %Lists need no special consideration from the language specification.
%The purpose of this section is to  a and then introduce VerseML's syntax extension mechanisms. %For forms with a clear analogy to a form in Standard ML, we will assume the  semantics are analagous without providing details.

%We will begin in Sec. \ref{sec:examples} by detailing another example for which such a mechanism would be useful: regular expression (regex) patterns expressed using abstract data types. We will refer to this example throughout the proposal. In Sec. \ref{sec:syntax-existing}, we discuss how the usual approach of using dynamic string parsing to introduce regex patterns is not ideal. We also survey existing alternatives to dynamic string parsing, finding that they involve an unacceptable loss of modularity and other undesirable trade-offs. In Secs. \ref{sec:tsms} and \ref{sec:tsls}, we introduce our proposed alternatives -- \emph{typed literal macros} (TLMs) and the related \emph{type-specific languages} (TSLs) -- and discuss how they resolve these issues (as well as some limitations that they have). We  also give an overview of how TLMs are formally specified. We give a concrete timeline for the remaining work in Sec. \ref{sec:syntax-timeline}, and conclude in Sec. \ref{sec:conclusion}.
% !TEX root = omar-thesis.tex

\section{Preliminaries}\label{sec:preliminaries}
\vspace{-3px}
This work is rooted in the tradition of full-scale functional languages like Standard ML, OCaml and Haskell (as might have been obvious from Chapter \ref{chap:intro}.) Familiarity with basic concepts in these languages, e.g. variables, types, polymorphic and recursive functions, tuples, records, recursive datatypes and structural pattern matching, is assumed throughout this work. Readers who are not familiar with these concepts are encouraged to consult the early chapters of an introductory text like Harper's \emph{Programming in Standard ML} \cite{harper1997programming} (a working draft can be found online.) We discuss integrating TSMs into languages from other design traditions in Sec. \ref{sec:integration}.

In Chapter \ref{chap:ptsms} and onward, as well as in some of the motivating examples below, we also assume basic familiarity with ML-style module systems. Readers with experience in a language without such a module system (e.g. Haskell) are also advised to consult the relevant chapters in \emph{Programming in Standard ML} \cite{harper1997programming} as needed. We distinguish \emph{modules}, which are language constructs, from \emph{libraries}, which are extralinguistic packaging constructs managed by some implementation-defined compilation manager (e.g. \li{CM}, distributed with Standard ML of New Jersey (SML/NJ) \cite{DBLP:conf/plilp/AppelM91}.) A library can export any number of modules, signatures and TSM definitions.

The formal systems that we will consider are defined within the metatheoretic framework of type theory. More specifically, we will assume that abstract binding trees (ABTs, which enrich abstract syntax trees with the notions of binding and scope, as discussed in Chapter \ref{chap:intro}), renaming, alpha-equivalence, substitution, structural induction and rule induction are defined as described in Harper's \emph{Practical Foundations for Programming Languages, Second Edition} (\emph{PFPL}) \cite{pfpl}. Familiarity with other formal accounts of type systems, e.g. Pierce's \emph{Types and Programming Languages} (\emph{TAPL}) \cite{tapl}, should also suffice.% This document is organized so as to be readable even if the sections defining formal systems are skipped entirely, although much precision will, of course, be lost.


% !TEX root = omar-thesis.tex

\section{Cognitive Cost}\label{sec:syntactic-properties}
\begin{quote}
\emph{In the present inquiry, the idea is to adopt a much
wider conception of formal languages so as to investigate more broadly what
exactly is going on when a reasoner puts these tools to use.}

\begin{flushright}Catarina Dutilh Novaes\\
\emph{Formal Languages in Logic: A Philosophical and Cognitive Analysis} \cite{novaes2012formal}
\end{flushright}
\end{quote}

Central to our motivations is the notion that different drawings of a formal structure can and should be distinguished on the basis of the  \emph{cognitive costs} that humans incur as they interact with them. 

The broad notion of cognitive cost must ultimately be understood intuitively, relating as it does to the complexities of the human mind. Cognitive cost is also fundamentally a \emph{subjective} and \emph{situational} notion. 
As such, researchers cannot develop a truly comprehensive formal framework capable of settling questions of cognitive cost.\footnote{The fact that cognitive cost cannot be comprehensively characterized seems itself to create a cognitive hazard, in that those of us who favor comprehensive formal frameworks sometimes devalue or dismiss concerns related to cognitive cost, or consider them in an overly \emph{ad hoc} manner. This tendency must be resisted if programming language design is to progress as a human-oriented design discipline.} However, there are several situationally useful frameworks worth briefly reviewing \cite{box1987empirical}. % operationalize cognitive cost in a simpler and more tractable manner %These can serve as satisfying proxies in many circumstances. %In order to ground this concept, it is common for researchers to  operationalize this notion in order to simplify the arguments that they are making. 

% Notions of cognitive cost can perhaps be understood by informal analogy to notions of \emph{dynamic cost}, which distinguish semantically equivalent expressions based on their consumption of various resources, e.g. time or memory, as they are evaluated. Notions of cognitive cost analagously capture the consumption of human attentional resources as they are being drawn and examined by a human. Human attention resources are, of course, more difficult to quantify.


One useful quantitative framework reduces cognitive cost to \emph{syntactic cost}, which is measured by counting characters (or glyphs, more generally.) This is often a satisfying proxy for cognitive cost, in that smaller drawings are often easier to comprehend and produce. For example, the drawing \li{[x, y, z]} has lower syntactic cost than its desugaring, as discussed in the previous chapter. There is a limit to this approximation, of course. For example, one might argue that the drawings involving the syntax of K, like the drawing from Figure \ref{fig:K-dialect}, have high cognitive cost, despite their low syntactic cost, until one is experienced with the syntax of K. In other words, the relationship between syntactic cost and cognitive cost depends on the subject's progression along some \emph{learning curve}.

A related quantity of interest to human programmers is \emph{edit cost}, measured relative to a program editor as the minimum number of primitive edit actions that must be performed to produce a drawing. For example, when using a text editor (as most professional programmers today do), drawings in textual form typically have lower edit cost, as measured by the minimum number of keystrokes necessary to produce the drawing, than those in operational or stylized forms (indeed, some drawings in stylized form can be understood to have infinite text edit cost.) Edit cost can be modeled using, for example, \emph{keystroke-level models} (KLMs) as described by Card, Moran and Newell \cite{journals/cacm/CardMN80}.%which, for software developers, is their primary mode of interaction with a programming language.%Our choice might also be influenced (or determined) by the tool that we are using to write the program. In particular, stylized forms are suitable for use when typesetting a program, whereas textual forms are necessary for writing programs using a text editor for consumption by an implementation of the semantics on a computer. 

One can also analyze cognitive cost using disciplined qualitative methods. Green's \emph{Cognitive Dimensions of Notations} \cite{Green89,green1996usability} and Pane and Myers' \emph{Usability Issues} \cite{pane1996usability} (both of which synthesized much of the earlier work in the area) are highly cited heuristic frameworks. For example, Green's cognitive dimensions framework gives us a common vocabulary for  comparing the derived list forms described in Chapter \ref{chap:intro} to the primitive list forms. In particular, the derived list forms \emph{map more closely} to other notations used for sequences of elements (e.g. in typeset mathematics, or on a physical notepad) than the primitive list forms. They also make the elements of the list more clearly \emph{visible}, in that the identifier \li{Cons} is not interspersed throughout the term, and they have lower \emph{viscosity} because adding a new item to the middle of a list drawn in derived form requires only a local edit, whereas for a list constructed by applying list constructors in prefix position, one needs also to add a closing parenthesis to the end of the term. (Infix operators for lists, discussed in Sec. \ref{sec:Fixity-directives}, also have low viscosity.)

Finally, one might consider cognitive cost comparatively using quantitative empirical methods, e.g. by conducting randomized control trials to compare forms with respect to task completion time or error rate (for satisfyingly representative tasks.) Stefik et al. have performed many such studies, mainly on novice programmers (these are summarized, along with other such studies, in \cite{journals/jeric/StefikS13}.) Kaijanaho provides another review of evidence-based language design methodologies \cite{kaijanaho2015evidence}.

Our goal in this work is to provide a means by which library providers can introduce alternative syntactic forms of their own design. We leave it up to each library provider to establish the cognitive costs associated with the alternative forms that they introduce, according to whichever operationalization of the concept that they favor. For the examples in this document, we will mainly utilize syntactic cost, because claims about syntactic cost can be evaluated quantitatively. In a few cases, we also make heuristic arguments. 

We claim also that the abstract reasoning principles that TSMs come equipped with serve to limit cognitive costs that a client programmer that encounters an unfamiliar form would otherwise incur when attempting to reason about types and binding. This claim follows from the intuitive assumption that examining only type annotations is less costly than examining the full expansion of an unexpanded term and the logic that produced that expansion. 

% There is much that remains to be understood about cognitive cost, particularly when the subject is an experienced programmer. Many of the difficulties that we will confront in this work have to do with the fact that allowing programmers to add new derived forms unconstrained to a syntax definition can decrease cognitive cost ``in the small'', i.e. for programmers who understand all of the details of the newly introduced desugaring transformations, while increasing cognitive cost ``in the large'' because programmers have few clear modular reasoning principles that they can rely on when they encounter an unfamiliar form. Our aim is to control cognitive cost at all scales. % (Indeed, many of challenges of programming language design might be said to have this flavor.)% Our contributions, however, are not directly in this area, so we will stop here. 

%Put another way, the stylized and textual forms are preferrable when designing a \emph{user interface} of our programming language.


\section{Motivating Definitions}\label{sec:motivating-examples}
In this section, we give a number of VerseML definitions that will serve as the basis for many subsequent examples. This section also serves as an introduction to the textual syntax and semantics of VerseML.

\subsection{Lists}\label{sec:lists}
The Standard ML Basis Library (i.e.  the standard library) defines list types as follows:
\begin{lstlisting}[numbers=none]
datatype 'a list = nil | op:: of 'a * 'a list
\end{lstlisting}
This datatype declaration generates:
\begin{itemize}
\item a type function \li{list} that takes one type parameter; 
\item the value constructors \li{nil : 'a list} and \li{op:: : 'a * 'a list -> 'a list}; and
\item the corresponding list pattern constructors \li{nil} and \li{op::}.
\end{itemize}
We will return to the significance of the identifier \li{op::} in Sec. \ref{sec:Fixity-directives} below.

VerseML does not support SML-style datatype declarations directly. Instead, type functions, recursive types, sum types, product types, value constructors, pattern constructors and type generativity arise through orthogonal mechanisms, as in foundational accounts of these concepts (e.g. \emph{PFPL} \cite{pfpl}.) This is mainly for pedagogical purposes -- it will take until Chapter \ref{chap:ptsms} to build up all of the machinery that would be necessary to integrate TSMs into a language with SML-style datatype declarations. By exposing more granular primitives, we can define sub-languages of VerseML in Chapter \ref{chap:uetsms} and Chapter \ref{chap:uptsms} that communicate certain fundamental ideas more clearly and generally.

With that in mind, the family of list types are defined in VerseML as follows:
\begin{lstlisting}[numbers=none]
type list('a) = rec(self => Nil + Cons of 'a * self)
\end{lstlisting}
Here, \li{list} is a {type function} binding its type parameter to the type variable \li{'a}. We apply parameters in post-fix position (rather than in prefix position, as in SML.) For example, the type of integer lists is \li{list(int)}. This is equivalent, by substitution of \li{int} for \li{'a} on the right side of the definition above, to the following \emph{recursive type}:
\begin{lstlisting}[numbers=none]
rec(self => Nil + Cons of int * self)
\end{lstlisting}
%Here, the type variable \li{self} is bound as a \emph{self reference} in the type body. 

The values of a recursive type \li{T} are \li{fold(e)}, where \li{e} is a value of the \emph{unrolling} of \li{T}. The {unrolling} of a recursive type is determined by substituting the recursive type itself for the self reference in its type body. For example, the unrolling of \li{list(int)} is equivalent, by substitution of \li{list(int)} for \li{self}, to the following \emph{labeled sum type}:
\begin{lstlisting}[numbers=none]
Nil + Cons of int * list(int)
\end{lstlisting}
The values of a labeled sum type \li{T} are injections \li{inj[Lbl](e)}, where \li{Lbl} is a label specified by one of the classes specified by \li{T} and \li{e} is a value of the corresponding type. The {labeled sum type} above specifies two {classes}:
\begin{enumerate}
\item One class, labeled \li{Nil}, takes values of \li{unit} type (we can omit \li{of unit}.) The only value of \li{unit} type is the trivial value \li{()}.  
\item The other class, labeled \li{Cons}, takes values of the \emph{product type} \li{int * list(int)}, the values of which are tuples. 
\end{enumerate}

Let us now define two example values of type \li{list(int)}:
\begin{lstlisting}[numbers=none]
val nil_int : list(int) = fold(inj[Nil] ())
val one_int : list(int) = fold(inj[Cons] (1, nil_int))
\end{lstlisting}
Here, \li{nil_int} is the empty list and \li{one_int} is a list containing a single integer, \li{1}. %We omitted the type ascriptions on the folds and injections because VerseML can infer them.

One way to lower syntactic cost is to define the following polymorphic values, called the \emph{list value constructors}, which abstract away the necessary folds and injections:
\begin{lstlisting}[numbers=none]
val Nil : list('a) = fold(inj[Nil] ())
fun Cons(x : 'a * list('a)) : list('a) => fold(inj[Cons] x)
\end{lstlisting}
In fact, VerseML generates constructors like these automatically.\footnote{A more general mechanism that allows values to be generated from type definitions is beyond the scope of our work on TSMs.} 
Using these list value constructors, we can equivalently express the values above as follows:
\begin{lstlisting}[numbers=none]
val nil_int : list(int) = Nil
val one_int = Cons (1, Nil)
\end{lstlisting}
In SML, constructors like these are the only means by which a value of a datatype can be introduced -- folding and injection operators are not exposed directly to programmers. As such, it is not possible to construct a value of a type like \li{list(int)} in a context-independent manner, i.e. in contexts where the value constructors have been shadowed or are not bound. This will become relevant in the next section and in Chapter \ref{chap:uetsms}. %In Chapter \ref{chap:ptsms}, we will introduce the machinery that would be necessary to take the SML-style approach and suppress mention of \li{fold} and \li{inj} operators entirely.

Values of recursive type, labeled sum type and product type are deconstructed by pattern matching. %\footnote{Readers who are not familiar with structural pattern matching may wish to consult the introduction to Chapter \ref{chap:uptsms} for a somewhat more detailed description.} 
For example, we can write the polymorphic map function, which constructs a  list by applying a given function to each item in a given list, as follows:
\begin{lstlisting}[numbers=none]
fun map (f : 'a -> 'b) (xs : list('a)) : list('b) => 
  match xs with 
  | fold(inj[Nil] ()) => Nil
  | fold(inj[Cons] (y, ys)) => Cons (f y, map f ys)
  end
\end{lstlisting}


The primitive pattern forms above are drawn like the corresponding primitive value forms (though it is important to keep in mind that the syntactic overlap is superficial -- patterns and expressions are distinct sorts of trees.) To lower syntactic cost, VerseML automatically inserts folds, injections and trivial arguments into patterns of constructor form, i.e. those of the form \li{Lbl} and \li{Lbl p} where \li{Lbl} is a capitalized label and \li{p} is another pattern:\footnote{Pattern TSMs, introduced in Chapter \ref{chap:uptsms}, could be used to manually achieve a similar syntax for any particular type, or in Chapter \ref{chap:ptsms}, across a particular family of types, but because this syntactic sugar is useful for all recursive labeled sum types, we build it primitively into VerseML.}
\begin{lstlisting}[numbers=none]
fun map (f : 'a -> 'b) (xs : list('a)) : list('b) => 
  match xs with 
  | Nil => Nil 
  | Cons (y, ys) => Cons (f y, map f ys)
  end
\end{lstlisting}
%To avoid syntactic ambiguity, variables must not begin with an uppercase letter.

We group the type and value definitions above, as well as some other standard utility functions like \li{append}, into a \emph{module} \li{List : LIST}, where \li{LIST} is the \emph{signature} defined in Figure \ref{fig:LIST}. These definitions are not privileged in any way by the language definition. In particular, there are no list-specific derived forms built in to the textual syntax of VerseML. We will show how TSMs allow programmers to achieve a similar syntax for lists over the next several chapters.

\begin{figure}[h!]
\begin{lstlisting}[numbers=none]
signature LIST = 
sig 
  type list('a) = rec(self => Nil + Cons of 'a * self)
  val Nil : list('a)
  val Cons : 'a * list('a) -> list('a)
  val map : ('a -> 'b) -> list('a) -> list('b)
  val append : list('a) -> list('a) -> list('a)
  (* ... *)
end
\end{lstlisting}
\caption{Definition of the \li{LIST} signature}
\label{fig:LIST}
\end{figure}

\subsection{Regular Expressions}\label{sec:syntax-examples-regexps}
A regular expression, or \emph{regex}, is a description of a \emph{regular language} \cite{Thompson:1968:PTR:363347.363387}. Regexes arise with some frequency in fields like natural language processing and bioinformatics.

\paragraph{Recursive Sums}
One way to encode regular expressions in VerseML is as values of the recursive labeled sum type abbreviated \li{rx} in Figure \ref{fig:datatype-rx}.

\begin{figure}[h]
\begin{lstlisting}[numbers=none]
type rx = rec(rx => Empty + Str of string + Seq of rx * rx +
                    Or of rx * rx + Star of rx)
\end{lstlisting}
\caption{Definition of the recursive labeled sum type \li{rx}}
\label{fig:datatype-rx}
\end{figure}
Assuming the automatically generated value constructors as in Sec. \ref{sec:lists}, we can construct a regex that matches the strings \li{"SSTRAESTR"}, \li{"SSTRTESTR"}, \li{"SSTRGESTR"} or \li{"SSTRCESTR"} (i.e. DNA bases) as follows:
\begin{lstlisting}[numbers=none]
Or(Str "SSTRAESTR", Or(Str "SSTRTESTR", Or(Str "SSTRGESTR", Str "SSTRCESTR")))
\end{lstlisting}

Given a value of type \li{rx}, we can deconstruct it by pattern matching, again as in Sec. \ref{sec:lists}. For example, the function \li{is_dna_rx} defined in Figure \ref{fig:is_dna_rx} detects regular expressions that match DNA sequences.

\begin{figure}[h]
\begin{lstlisting}[numbers=none]
fun is_dna_rx(r : rx) : boolean => 
  match r with 
  | Str "SSTRAESTR" => True
  | Str "SSTRTESTR" => True
  | Str "SSTRGESTR" => True
  | Str "SSTRCESTR" => True
  | Seq (r1, r2) => (is_dna_rx r1) andalso (is_dna_rx r2)
  | Or  (r1, r2) => (is_dna_rx r1) andalso (is_dna_rx r2)
  | Star(r') => is_dna_rx r'
  | _ => False 
  end
\end{lstlisting}
\caption{Pattern matching over regexes in VerseML}
\label{fig:is_dna_rx}
\end{figure}


\paragraph{Abstract Types} Encoding regexes as values of type \li{rx} is straightforward, but there are reasons why one might not wish to expose this encoding to clients directly. 

First, regexes are usually identified up to their reduction to a normal form. For example, \li{Seq(Empty, Str "SSTRAESTR")} has normal form \li{Str("SSTRAESTR")}. It can be useful for regexes with the same normal form to be  indistinguishable from the perspective of client code. (The details of regex normalization are not important for our purposes, so we omit them.)

Second, it can be useful for performance reasons to maintain additional data alongside each regex (e.g. a corresponding finite automaton.) In fact, there may be many ways to represent regexes, each with different performance trade-offs, so we would like to provide a choice of representations behind a common interface.

To achieve these goals, we turn to the VerseML module system, which is based directly on the SML module system \cite{mthm97-for-dart,dreyer2005understanding} (which originates in early work by MacQueen \cite{MacQueen:1984:MSM:800055.802036}.) In particular, let us define the {signature} abbreviated \li{RX} in Figure \ref{fig:signature-RX}.
%Notice that it exposes an interface otherwise  to the one available using a case type:

\begin{figure}[ht]
\begin{lstlisting}[deletekeywords={case}]
(* abstract regex unfoldings *)
type u('a) = UEmpty + UStr of string + USeq of 'a * 'a + 
             UOr of 'a * 'a + UStar of 'a

signature RX = 
sig
  type t (* abstract *)

  (* constructors *)
  val Empty : t
  val Str : string -> t
  val Seq : t * t -> t
  val Or : t * t -> t
  val Star : t -> t

  (* produces the normal unfolding *)
  val unfold_norm : t -> u(t)
end

module R1 : RX = struct (* ... *) end
module R2 : RX = struct (* ... *) end
\end{lstlisting}
\vspace{-5px}
\caption{The \lstinline{RX} signature and two example implementations}
\label{fig:signature-RX}
\end{figure}

The clients of any module \lstinline{R} that has been sealed by \lstinline{RX}, e.g. \li{R1} or \li{R2}  in Figure \ref{fig:signature-RX}, manipulate regexes as values of type \li{R.t} using the interface specified by \li{RX}. For example, a client can construct a regex matching DNA bases by projecting the value constructors out of \li{R} and applying them as follows:
\begin{lstlisting}[numbers=none]
R.Or(R.Str "SSTRAESTR", R.Or(R.Str "SSTRTESTR", R.Or (R.Str "SSTRGESTR", R.Str "SSTRCESTR")))
\end{lstlisting}

Because the identity of the representation type \lstinline{R.t} is held abstract by the signature, the only way for a client to construct a value of this type is through the values that \li{RX} specifies (i.e. we have defined an \emph{abstract data type (ADT)}  \cite{liskov1974programming}.) Consequently, representation invariants need only be established locally within each module.




Similarly, clients cannot interrogate the structure of a value \li{r : R.t} directly. Instead, the signature specifies a function \li{R.unfold_norm} that produces the \emph{normal unfolding} of a given regex, i.e. a value of type \li{u(R.t)} that exposes only the outermost form of the regex in normal form (this normal form invariant is specified only as an unenforced side condition that implementations are expected to obey, as is common practice in languages like ML.) Clients can pattern match over the {normal unfolding} in the familiar manner, as shown in Figure \ref{fig:is_dna_rx_prime}. 
\begin{figure}
\begin{lstlisting}[numbers=none]
fun is_dna_rx'(r : R.t) : boolean => 
  match R.unfold_norm r with 
  | UStr "SSTRAESTR" => True
  | UStr "SSTRTESTR" => True
  | UStr "SSTRGESTR" => True
  | UStr "SSTRCESTR" => True
  | USeq (r1, r2) => (is_dna_rx' r1) andalso (is_dna_rx' r2)
  | UOr (r1, r2) => (is_dna_rx' r1) andalso (is_dna_rx' r2)
  | UStar r' => is_dna_rx' r'
  | _ => False
  end
\end{lstlisting}
\vspace{-5px}
\caption{Pattern matching over normal unfoldings of regexes}
\label{fig:is_dna_rx_prime}
\end{figure}

The normal unfolding suffices in situations where a client needs to examine only the outermost structure of a regex. However, in general, a client may want to pattern match more deeply into a regex. There are various ways to approach this problem. 

One approach is to define auxiliary functions that construct $n$-deep unfoldings of \li{r}, where $n$ is the deepest level at which the client wishes to expose the normal structure of the regex. For example, it is easy to define a function \li{unfold_norm2 : R.t -> u(u(R.t))} in terms of \li{R.unfold_norm} that allows pattern matching to depth $2$.\footnote{Defining an unfolding \emph{generic} in $n$ is a more subtle problem that is beyond the scope of this work.} 

Another approach is to \emph{completely unfold} a value of type \li{t} by applying a function \li{view : R.t -> rx} that recursively applies \li{R.unfold_norm} to exhaustion. The type \li{rx} was defined in Figure \ref{fig:datatype-rx}.  Computing the complete unfolding (also called the \emph{view}) can have higher dynamic cost than computing an incomplete unfolding of appropriate depth, but it is also a simpler approach (i.e.   lower cognitive cost can justify higher dynamic cost.)


\begin{figure}[t]
\begin{lstlisting}[numbers=none]
functor RXUtil(R : RX) = 
struct
  fun unfold_norm2(r : R.t) : u(u(R.t)) => (* ... *)

  fun view(r : R.t) : rx => 
    match R.unfold_norm r with 
    | UEmpty => Empty
    | UStr s => Str s
    | USeq (r1, r2) => Seq (view r1, view r2)
    | UOr (r1, r2) => Or (view r1, view r2)
    | UStar r => Star (view r)
    end 

  (* ... *)
end
\end{lstlisting}
\vspace{-5px}
\caption{The definition of \li{RXUtil}}
\vspace{-5px}
\label{fig:RXUtil}
\end{figure}
Typically, utility functions like \li{unfold_norm2} and \li{view} are defined in a \emph{functor} (i.e. a function at the level of modules) like \li{RXUtil} in Figure \ref{fig:RXUtil}, so that they need only be defined once, rather than separately for each module \li{R : RX}. The client can instantiate the functor by applying it to their choice of module as follows:
\begin{lstlisting}[numbers=none]
module RU = RXUtil(R)
\end{lstlisting}
% \subsection{Lists, Sets, Maps, Vectors and Other Containers}\label{sec:syntax-examples-containers}
% \todo{write this (Spring 2016)}
% \subsection{HTML and Other Web Languages}\label{sec:syntax-examples-html}
% \subsection{Dates, URLs and Other Standardized Formats}\label{sec:syntax-examples-dates}
% \subsection{Query Languages} The language of regular expressions can be considered a query language over strings. There are many other query languages that focus on different types of data, e.g. XQuery for XML trees, or that are associated with various database technologies, e.g. SQL for relational databases. \todo{finish this (Spring 2016)} 
% \subsection{Monadic Commands}\label{sec:syntax-examples-monads}
% \todo{write this; cite Bob's blog (Spring 2016)}

% \todo{http://www.cs.umd.edu/~mwh/papers/monadic.pdf}
% \subsection{Quasiquotation and Object Language Syntax}\label{sec:syntax-examples-quasiquotation}
% \todo{write this (Spring 2016)}
% \subsection{Grammars}\label{sec:syntax-examples-grammars}
% \todo{write this (Spring 2016)}
% \subsection{Mathematical and Scientific Notations}\label{sec:syntax-examples-math-science}
% \subsubsection{SMILES: Chemical Notation}
% \todo{write this; cite SMILES \url{https://en.wikipedia.org/wiki/Simplified_molecular-input_line-entry_system} (Spring 2016)}
% \subsubsection{\TeX~Mathematical Formula Notation}
% \todo{write this (Spring 2016)}

% \subsection{Others}

% Get examples from: \url{http://voelter.de/data/pub/mbeddr-cs-oopsla2015-preprint.pdf}


% !TEX root = omar-thesis.tex
\section{Existing Approaches}\label{sec:existing-approaches}
We will now review existing approaches that library providers seeking to decrease the cognitive cost of idioms involving definitions like those above might consider. 

\subsection{Standard Abstraction Mechanisms}
The simplest approach is to capture idioms using the standard abstraction mechanisms of our 
language, e.g. functions and modules. 

We already saw examples of this approach above. For example, we defined the ``list constructors'', which capture the idioms of list construction:
\begin{lstlisting}[numbers=none]
val Nil : 'a list = fold(inj[Nil] ())
fun Cons (x : 'a * 'a list) : 'a list => 
  fold(inj[Cons] x)
\end{lstlisting} 
Such definitions are common enough that VerseML generates them automatically.

We also defined a utility functor for regexes, \li{RXUtil}, in Figure \ref{fig:RXUtil}. As more idioms involving regexes arise, we can capture them by adding additional definitions to this functor (or some other such functor, if we cannot modify \li{RXUtil} itself.) For example, we can define a value that matches digits:
\begin{lstlisting}[numbers=none]
val digit = R.Or(R.Str "SSTR0ESTR", R.Or(R.Str "SSTR1ESTR", ...))
\end{lstlisting}
Similarly, we can define a function \li{repeat : R.t -> int -> R.t} that computes a regex by sequentially repeating the given regex a given number of times. Using these auxiliary definitions, a client can define a regex that matches social security numbers as follows:
\begin{lstlisting}[numbers=none]
val dash = R.Str "SSTR-ESTR"
val repeat_d = RU.repeat RU.digit
val ssn = R.Seq(repeat_d 3, R.Seq(dash, R.Seq(repeat_d 2, 
                R.Seq(dash, repeat_d 4))))
\end{lstlisting}

One limitation of this approach is that it provides no way to capture idioms at the level of patterns. Expressions and patterns are two different sorts of trees, and a pattern must be fully determined statically, whereas variables stand for dynamic values and functions perform dynamic computation.

Another limitation is that this approach does not give library providers control over form. For example, we cannot ``approximate'' SML-style derived list forms using only auxiliary definitions (otherwise, SML would not have defined derived list forms.) 
Similarly, consider the textual syntax for regexes defined in the POSIX standard \cite{STD95954} (which data suggests that most professional programmers are familiar with \cite{Omar:2012:ACC:2337223.2337324}.) Using this syntax, the regex that matches DNA bases is drawn:
\begin{lstlisting}[numbers=none]
A|T|G|C
\end{lstlisting}
Similarly, the regex that matches SSNs just discussed is drawn:
\begin{lstlisting}[numbers=none]
\d\d\d-\d\d-\d\d\d\d
\end{lstlisting}
or
\begin{lstlisting}[numbers=none]
\d{3}-\d{2}-\d{4}
\end{lstlisting}
These drawings have lower syntactic cost than those above, and programmers familiar with the POSIX syntax for regular expressions would likely agree that these drawings have lower cognitive cost as well. 

\vspace{-6px}
\subsubsection{String Parsing}\label{sec:dynamic-string-parsing}
We might attempt to approximate the POSIX syntax for regexes by defining a function \li{parse : string -> R.t} in \li{RXUtil} that parses a string representation of a POSIX regex form  to produce a regular expression value, or raises an exception if the input is malformed with respect to the POSIX specification. 

Using \li{RU.parse}, we can construct the regex matching DNA bases  as follows:
\begin{lstlisting}[numbers=none]
RU.parse "SSTRA|T|G|CESTR"
\end{lstlisting}
This approach is imperfect for several reasons:
\begin{enumerate} 
\item First, there are syntactic conflicts between standard string escape sequences and standard regex escape sequences. For example, the following is not a well-formed drawing in Standard ML:
\begin{lstlisting}[numbers=none,mathescape=|]
val ssn = RU.parse "SSTR\d\d\d-\d\d-\d\d\d\dESTR" (* ERROR *)
\end{lstlisting}
In practice, a parser would provide the client with an error message like:\footnote{This is the error message that \texttt{javac} produces. When compiling an analagous expression using SML of New Jersey (SML/NJ), we encounter a more confusing error message: \texttt{Error: unclosed string}.}
\begin{lstlisting}[numbers=none]
error: illegal escape character
\end{lstlisting}
because \verb|\d| is not a valid string escape character (even though it is a valid regex escape character.) In a small lab study, we observed that this class of error confused even experienced programmers if they had not used regexes recently \cite{Omar:2012:ACC:2337223.2337324}. 

One workaround -- escaping all backslashes -- nearly doubles syntactic cost:
\begin{lstlisting}[numbers=none]
val ssn = RU.parse "SSTR\\d\\d\\d-\\d\\d-\\d\\d\\d\\dESTR"
\end{lstlisting}

Some languages, anticipating this use of string literals, build in alternative string forms that leave escape sequences uninterpreted. For example, OCaml supports alternative string literals like \li+{rx|SSTR\d\d\d-\d\d-\d\d\d\dESTR|rx}+.

\item The next problem is that dynamic string parsing only decreases the syntactic cost of complete regexes. Regexes constructed compositionally do not benefit from this technique. For example, the function below constructs a regex from a string, \li{name}, and another regex, \li{ssn}:
\begin{figure}[h]
\begin{lstlisting}[numbers=none]
  fun lookup_rx(name : string) => 
    R.Seq(R.Str name, R.Seq(R.Str "SSTR: ESTR", ssn))
\end{lstlisting}
\caption{Compositional construction of regexes.}
\label{fig:lookup_rx}
\end{figure}
%We needed to use both dynamic string parsing and explicit applications of pattern constructors to achieve the intended semantics. 

We cannot use \li{RU.parse} to redraw this equivalently, but at lower syntactic cost. 

(We will describe a syntax dialect that does capture such idioms in Sec. \ref{sec:syntax-dialects}. In particular, we will compare Figure \ref{fig:lookup_rx} with Figure \ref{fig:derived-spliced-subexpressions}.)

This is also why dynamic string parsing is irrelevant for capturing idioms like list construction -- list expressions contain sub-expressions.

%(we will see an example of syntax that does capture such idioms below).

\item Using strings to introduce regexes also creates a \emph{cognitive hazard} for programmers who are coincidentally working with other data of type \li{string}. For example, consider the following seemingly ``more readable definition of \lstinline{lookup_rx}'', where the infix operator \li{^} means string concatenation:
\begin{lstlisting}[numbers=none,escapechar=~]
fun lookup_rx_insecure(name : string) => 
  RU.parse (name ^ {rx|SSTR: \d\d\d-\d\d-\d\d\d\dESTR|rx})
\end{lstlisting}

or equivalently, given the regex \li{ssn} as above and an auxiliary function \li{RU.to_string} that can generate its string representation:
\begin{lstlisting}[numbers=none,escapechar=~]
fun lookup_rx_insecure(name : string) => 
  RU.parse (name ^ "SSTR: ESTR" ^ (RU.to_string ssn))
\end{lstlisting}
%The (unstated) intent here was to treat \lstinline{name} as a sub-pattern matching only itself, but this is not the observed behavior when \lstinline{name} contains special characters that have other meanings in patterns.

Both \lstinline{lookup_rx} and \lstinline{lookup_rx_insecure} have the same type, \li{string -> R.t}, and behave identically at many inputs, particularly the ``typical'' inputs (i.e. alphabetic names.) It is only when \li{lookup_rx_insecure} is applied to a string that corresponds to a regex that matches more than just that string that it behaves incorrectly. 

In applications that query sensitive data, mistakes like this lead to \emph{injection attacks}, which are among the most common and catastrophic security threats today \cite{owasp2013}.

This problem is, fundamentally, attributable to the programmer making a mistake in a misguided effort to decrease syntactic cost. However, the availability of a better approach for reducing syntactic cost would serve to make this class of mistakes less prevalent \cite{Bravenboer:2007:PIA:1289971.1289975}. Given that our design philosophy is explicitly concerned with issues of cognitive cost, it is natural to also consider common cognitive hazards.

%Proving that mistakes like this have not been made involves reasoning about complex run-time data flows. 

 %Ultimately, of course, mistakes like this are the fault of a programmer using a flawed heuristic, and they could be avoided with discipline. The problem is once again that it is difficult to detect violations of this discipline automatically. 

 %Ideally, our library would be able to make it more difficult to inadvertently introduce subtle security bugs like this.
\item The final problem is that regex parsing does not occur until the call to \li{RU.parse} is dynamically evaluated. For example, the malformed string encoding of a regex in the program fragment below will only trigger an exception when this expression is evaluated during the full moon: %Achieving this goal is an explicit goal of this proposal, so we are obviously not happy with this.

\begin{lstlisting}[numbers=none]
match moon_phase with 
Full => RU.parse "SSTR(GCESTR" | _ => (* ... *)
end
\end{lstlisting}
Malformed string encodings of regexes can sometimes be discovered by testing, though empirical data gathered from large open source projects suggests that there remain many malformed regexes that are not detected by a project's test suite ``in the wild'' \cite{spishak2012type}.

One workaround is for the programmer to lift all such calls where the argument is a string literal out to the top level of the program, so that the exception is raised every time the program is evaluated. This subtly changes the performance profile of the program, and there is a cognitive penalty associated with moving the description of a regex away from its use site, but for complete regexes, this might be an acceptable trade-off.% Moreover, the dynamic cost of parsing the regex is incurred on every invocation of the program, even when the regex will never be used.
% Statically verifying that pattern formation errors will not dynamically arise requires reasoning about arbitrary dynamic behavior. This is an undecidable verification problem in general and can be difficult to even partially automate. In this example, the verification procedure would first need to be able to establish that the variable \lstinline{rxparse} is equal to the parse function \lstinline{RUtil.parse}. If the string argument had not been written literally but rather computed, e.g. as \lstinline{"SSTR(GESTR" ^ "SSTRCESTR"} where \lstinline{^} is the string concatenation function applied in infix style, it would also need to be able to establish that this expression is equivalent to the string \lstinline{"SSTR(GCESTR"}. For patterns that are dynamically constructed based on input to a function, evaluating the expression statically (or, more generally, in some earlier ``stage'' of evaluation \cite{Jones:Gomard:Sestoft:93:PartialEvaluation}) also does not suffice. 

% Of course, asking the client to provide a proof of well-formedness would defeat the purpose of lowering syntactic cost.

% In contrast, were our language to support  derived regex syntax, pattern parsing would occur at compile-time and so malformed patterns would produce a compile-time error, no matter where they appear in a program.

% \item Dynamic string parsing also necessarily incurs dynamic cost. Regular expression patterns are common when processing large datasets, so it is easy to inadvertently incur this cost repeatedly. For example, consider mapping over a list of strings:
% \begin{lstlisting}[numbers=none]
% map exmpl_list (fn s => rxmatch (rxparse "SSTRA|T|G|CESTR") s)
% \end{lstlisting}
% To avoid incurring the parsing cost for each element of \lstinline{exmpl_list}, the programmer or compiler must move the parsing step out of the closure (for example, by eta-reduction in this simple example).\footnote{Anecdotally, in major contemporary compilers, this optimization is not automatic.} If the programmer must do this, it can (in more complex examples) increase syntactic cost and cognitive cost by moving the pattern itself far away from its use site. Alternatively, an appropriately tuned memoization (i.e. caching) strategy could be used to amortize some of this cost, but it is difficult to reason compositionally about performance using such a strategy. %If the programmer does it, it can sometimes make the program more difficult to read. 

% %This too is difficult if a portion of the pattern is dynamically generated. % Regular expressions are often used across large datasets in scientific applications, so the absolute peformance penalty can be non-trivial.

% In contrast, were our language to primitively support derived pattern syntax, the expansion would be computed at compile-time and incur no dynamic cost.
\end{enumerate}

Problems like these arise whenever a library provider attempts to deploy dynamic string parsing as a solution to the problem of high syntactic cost. The reason is that syntactic cost is a property of a drawing of a program, so trying to address it by drawing a different program requires establishing that the alternative program is equivalent to the program that the client would write if syntactic cost was not a consideration (which is, at worst, an ill-posed problem, and at best, a rather difficult problem.) %Moreover, logically equivalent programs can differ in terms of performance.

This is not to say that one should simply refrain from defining or applying a function like \li{RU.parse}. There are  valid uses of string parsing that are not motivated by the desire to decrease syntactic cost, e.g. when parsing regular expressions received as dynamic input to the program.%Strings are, simply put, not ideally suited for this task. 

\subsection{Quotation Parsing}\label{sec:dynamic-quotation}
Some syntax dialects of ML, e.g. a syntax dialect available via a compiler flag in SML/NJ \cite{SML/Quote}, define \emph{quotation literals}:  derived forms for expressions of type \li{'a frag list}, where \li{'a frag} is defined as follows (using SML's datatype declaration syntax):
\begin{lstlisting}[numbers=none]
datatype 'a frag = QUOTE of 'a | ANTIQUOTE of string
\end{lstlisting}
Quotation literals are delimited by backticks, e.g. \li{`SCSSA|T|G|CECSS`} is the same as writing \li{[QUOTE "SSTRA|T|G|CESTR"]}. Expressions of variable or parenthesized form that appear prefixed by a caret in the body of a quotation literal  are parsed out and appear wrapped in the \li{ANTIQUOTE} constructor, e.g. \li{`SCSSGC^(ECSSdna_rxSCSS)GCECSS`}  is the same as writing 
\begin{lstlisting}[numbers=none]
[QUOTE "SSTRGCESTR", ANTIQUOTE dna_rx, QUOTE "SSTRGCESTR"]
\end{lstlisting}
It is possible to define a function \li{qparse : R.t frag list -> R.t} in \li{RXUtil} that allows clients to dynamically construct regexes from  fragment lists like these.

Similarly, it is possible to define a function \li{qparse : 'a frag list -> 'a list} in the \li{List} module that allows clients to use quotation literals to construct lists:
\begin{lstlisting}[numbers=none]
List.qparse `SCSS[^(ECSSx + ySCSS), ^ECSSySCSS, ^ECSSzSCSS]ECSS`
\end{lstlisting}

%This addresses some of the problems of dynamic string parsing, in that we no longer need to use string concatenation to emulate splicing. 

As with dynamic string parsing, parsing occurs dynamically. We cannot use the trick of lifting all calls to \li{qparse} to the top level of our program, because the arguments are no longer string literals. At best, we can lift these calls into the earliest possible ``stage'' of evaluation. Parse errors are only detected once this stage is entered, and the dynamic cost of parsing is incurred each time this stage is entered. For example, \li{List.qparse} is called $n$ times below, where $n$ is the length of \li{input}:
\begin{lstlisting}[numbers=none]
List.map (x => List.qparse `SCSS[^ECSSxSCSS, ^(ECSS2 * xSCSS)]ECSS`) input
\end{lstlisting}

  % the application of \li{qparse} is evaluated.
Another problem is that the antiquote character, i.e. the caret, is fixed \emph{a priori}. This is problematic for regexes, for example, because the caret has a different meaning in the POSIX standard (and in practice, appears quite often.) 

All antiquoted values must be of the same type. To support both spliced regexes and spliced strings, for example, we need to define an auxiliary datatype in \li{RXUtil} 
and the client needs to apply it in each antiquotation. These ``marking constructors'' increase syntactic cost. 
For example, we we would need to write \li{lookup_rx} as follows:
\begin{lstlisting}[numbers=none]
fun lookup_rx(string : name) =>
  RU.qparse' `SCSS^(ECSSRU.QS nameSCSS): ^(ECSSRU.QR readingSCSS)ECSS`
\end{lstlisting}

Similarly, we would need to use marking constructors to support quoted lists where the tail is explicitly given by the client (following OCaml's revised syntax \cite{ocaml-manual}):
\begin{lstlisting}[numbers=none]
List.qparse `SCSS[^(ECSSList.V xSCSS), ^(ECSSList.V ySCSS) :: ^(ECSSList.VS zsSCSS)]ECSS`
\end{lstlisting}

Finally, quotation parsing, like the other approaches considered thusfar, helps only with the problem of abbreviating expressions. It provides no solution to the problem of abbreviating patterns.% The reason is simple: these approaches require applying functions, which, by nature, are expressions, not patterns.

VerseML does not build in quotation literals.\footnote{In fact, quotation syntax can be expressed using parametric TSMs, which are the topic of Chapter \ref{chap:ptsms}, though we will leave the details as an exercise for the reader. As such, even in the odd situation where parsing fragment lists drawn using quotations is the right solution, VerseML is a suitable language.}

% \subsection{Syntax Dialects}
% To gain more precise control over form, a library provider, or another interested party, might also consider defining a syntax dialect. 
% A dialect of a syntax definition, $\mathcal{D}$, is a new syntax definition, $\mathcal{D}'$, that:
% \begin{enumerate}
% \item extends $\mathcal{D}$, meaning that all drawings that are well-formed in $\mathcal{D}$ are well-formed in $\mathcal{D}'$ and identify the same AST; and
% \item defines additional derived forms.
% \end{enumerate}
% We leave the notion of a ``syntax definition'' undefined here for the sake of generality -- there are many different syntax definition systems.

\subsection{Infix Function Application}
To gain more precise control over form, a library provider, or another interested party, might also consider defining a syntax dialect using a syntax definition system.

The simplest syntax definition systems allow programmers to designate identifiers as infix operators. For example, SML builds in such a system. In SML, we can define \li{::} as a right-associative infix operator at precedence level 5 as follows:
\begin{lstlisting}[numbers=none]
infixr 5 ::
\end{lstlisting}
In a context where the variable \li{op::} is bound to the list constructor that we identified as \li{Cons} earlier, we can construct a list with \li{x} as its head and \li{y :: zs} as its tail as follows:
\begin{lstlisting}[numbers=none]
x :: y :: zs
\end{lstlisting}

In the SML Basis library, the list datatype is defined such that the constructor that we labeled \li{Cons} is instead labeled \li{op::}. As such, clients can use infix \li{::} in patterns as well. Binding the variable \li{op::} to \li{Cons} is not equivalent in this regard (again, because variables do not stand for patterns.)

Figure \ref{fig:infix-RX} shows how our regex library might define several such fixity declarations, together with a functor \li{RXOps} that binds the corresponding operator variables to the appropriate functions. Assuming that the library packaging system has brought the fixity declarations and the definition of \li{RXOps} from Figure \ref{fig:infix-RX} into scope, the client can instantiate \li{RXOps} with their choice of module \li{R : RX} and then \li{open} this instantiated module (also called a \emph{structure} in SML) to make these bindings visible. 


\begin{figure}
\begin{lstlisting}
infix 5 ::
infix 6 <*>
infix 4 <|>

functor RXOps(R : RX) =
struct 
  module RU = RXUtil(R)
  val op:: = R.Seq
  val op<*> = RU.repeat
  val op<|> = R.Or
end
\end{lstlisting}
\caption{Fixity declarations and related bindings for \li{RX}.}
\label{fig:infix-RX}
\end{figure}
\begin{lstlisting}[numbers=none]
structure ROps = RXOps(R)
open ROps
\end{lstlisting}
From there, the client can draw the examples discussed earlier equivalently as follows:
\begin{lstlisting}[numbers=none]
val dna = (R.Str "SSTRAESTR") <|> (R.Str "SSTRTESTR") <|> (R.Str "SSTRGESTR") <|> 
          (R.Str "SSTRCESTR")
val ssn = (RU.digit)<*>3 :: (RU.digit)<*>2 :: (RU.digit)<*>4
fun lookup_rx(name : string) => 
  (Str name) :: (Str "SSTR: ESTR") :: ssn
\end{lstlisting}

This demonstrates the two main limitations of this approach. 

First, it grants only limited control over form -- we cannot express the POSIX forms in this way, only \emph{ad hoc} (and in this case, rather poor) approximations thereof. Moreover, the infix operators declared in Figure \ref{fig:infix-RX} cannot be used for pattern matching, again because variables only stand for values.% as given cannot be used in patterns.%The desugaring of an infix operator application is always of function application form. In particular, it is always the operator variable (e.g. \li{op::}) applied to two arguments, first the expression on the left, then the expression on the right (with the precedence and associativity determining what these expressions are.)

The second limitation has to do with the possibility of syntactic conflicts. Notice that both the list library and the regex library have declared \li{::} an  infix operator, but with different associativity. They also export different bindings for \li{op::}. As such, clients cannot use both in the same scope. There is no mechanism that allows a client to explicitly qualify the use of an infix operator as referring to the fixity declaration from one or the other library -- fixity declarations are purely syntactic (i.e. they only influence parsing) and are not exported from modules or otherwise integrated into the binding structure of ML (note that libraries are extralinguistic packaging constructs, distinct from modules.) 

%This identifier-oriented approach is also rather \emph{ad hoc}, in that renaming or substituting for an identifier can break or change the meaning of the program.

Formally, each fixity declaration induces a dialect of the subset of SML's textual syntax that does not allow the declared identifier to be used in prefix position. When two such dialects are combined, the resulting dialect is not necessarily a dialect of both of the constituent dialects (the fixity declaration that is used depends on the order in which they are combined.)

Due to these limitations, VerseML does not inherit this mechanism from SML (the infix operators that are available in VerseML, like \li{^} for string concatenation, have a fixed precedence and associativity.)


\subsection{More General Syntax Definition Systems}\label{sec:syntax-dialects}
Fixity declarations give their users no direct control over desugaring -- the desugaring is always the corresponding function application. More general syntax definition systems give users more direct control over desugaring. In this section, we first review these more general syntax definition systems, then give two examples that demonstrate the expressive power of these systems, followed by a discussion of the difficulties that programmers can expect to encounter if they  adopt these systems ``in the large'' (following up on the previous discussion in Sec. \ref{sec:problems-with-dialects}.)

\subsubsection{Mixfix Notational Definitions}
Some languages, including many popular type-theoretic proof assistants like Coq \cite{Coq:manual}, support ``mixfix'' notational definitions. These generalize SML-style fixity declarations, in that newly defined forms can contain any fixed number of sub-terms (rather than just two). Their desugarings are determined by a user-defined rewriting rule. Griffin gives a formal account of such systems \cite{5134}, and Taha and Johann incorporate Griffin's account into a more general and  well-behaved account of \emph{staged notational definitions} \cite{Taha2003}.

Although more expressive than fixity declarations, these syntax definition systems also do not allow us to express POSIX regex syntax as-is. The issue is fundamentally that these systems only give programmers the ability to extend the syntax of the existing sorts of trees, e.g. expressions and patterns. They do not give programmers the ability to define new sorts of trees, with their own distinct syntax. For example, we cannot define a sort for regular expressions, where sequences of characters are not recognized as identifiers but rather as regex character sequences. 

\subsubsection{Grammar-Based Syntax Definition Systems}
Many syntax definition systems are oriented around \emph{formal grammars} \cite{hopcroft1979introduction}. Formal grammars have been studied since at least the time of P\~anini, who developed a grammar for Sanskrit in or around the 4th century BCE \cite{Ingerman:1967:LFS:363162.363165}. 

\emph{Context-free grammars (CFGs)} were first used to define the textual syntax of a major programming language -- Algol 60 -- by Backus \cite{naur1963revised}. Since then, countless other syntax definition systems oriented around CFGs have emerged -- we will only summarize the systems that are particularly relevant to our work here. In these systems a syntax definition consists of a CFG (perhaps from some restricted class of CFGs) equipped with various auxiliary definitions (e.g. a scanner/lexer definition in some systems) and logic for computing an output value (e.g. a tree) based on the determined form of the input. A \emph{parser generator} is an implementation of a grammar-based syntax definition system.


Perhaps the most established syntax definition systems within the ML ecosystem are ML-Lex  and ML-Yacc, which are distributed with SML/NJ \cite{TarditiDR:mly}, and Camlp4, which was (until recently)  integrated into  the OCaml system (in recent releases of the OCaml system, it has been deprecated in favor of a simpler system, \li{ppx}, that we discuss in the next section) \cite{ocaml-manual}. In these systems, output is  computed by ML functions that appear associated with each production in the grammar (these functions are referred to as the \emph{semantic actions}.) 

The \emph{syntax definition formalism  (SDF)}  \cite{journals/sigplan/HeeringHKR89} is a syntactic formalism for describing CFGs. SDF is used by a number of syntax definition systems, e.g. the Spoofax ``language workbench'' \cite{kats2010spoofax}. These systems commonly use Stratego, a rule-based rewriting language, as the language that output logic is written in \cite{Visser-RTA01}. SugarJ is an extension of Java that allows programmers to define and combine fragments of SDF+Stratego-based syntax definitions directly from within the program text \cite{erdweg2011sugarj}. SugarHaskell is a similar system based on Haskell \cite{erdweg2012layout} and Sugar* simplifies the task of defining such extensions of other languages \cite{erdweg2013framework}. SoundExt and SugarFOmega add the requirement that new derived forms must come equipped with corresponding derived typing rules \cite{conf/icfp/LorenzenE13}. The system must be able to verify that the rewrite logic is sound with respect to these rules (their verification system defers to the proof search facilities of PLT-Redex \cite{Felleisen-Findler-Flatt09}.) SoundX generalizes this idea to support other base languages, and also adds the ability to define type-dependent rewritings \cite{conf/popl/LorenzenE16}. We will say more about SoundExt/SugarFOmega and SoundX below.

 %Erdweg et al. have developed many non-trivial examples, including . \to

Copper implements a CFG-based syntax definition system that uses a context-aware scanner, which is useful for reasons that we will discuss shortly \cite{conf/gpce/WykS07}. Silver is an \emph{attribute grammar} system based on Copper \cite{VanWyk:2010:SEA}. Attribute grammars are used to compositionally define output logic that requires information that is not local to each form \cite{knuth1968semantics}.


Some other syntax definition systems are instead oriented  around \emph{parsing expression grammars} (PEGs) \cite{Ford04a}. PEGs are similar to CFGs, distinguished mainly in that they are deterministic by construction (by allowing only for explicitly prioritized choice between alternative parses.)

\subsubsection{Parser Combinator Systems}
\emph{Parser combinator systems} define a functional interface for defining a parser, together with various functions that generate new parsers from given parsers and other values (these functions are referred to as the \emph{parser combinators}) \cite{Hutton1992d}. 

For example, Hutton describes a system where parsers are functions of some type in the following type family:% \li{'char 'tree parser}, bound in VerseML's syntax:
\begin{lstlisting}[numbers=none]
type 'char 'tree parser = 'char list -> ('tree*'char list) list
\end{lstlisting}
i.e., a parser is a function that takes a list of (abstract) characters and returns a list of valid parses, each of which consists of an (abstract) tree and a list of the characters that were not consumed. An input is malformed if this function returns the empty list, and it is ambiguous if this function returns more than one parse. A deterministic parser is one that never returns more than one parse. The parser combinator \li{alt}, declared 
\begin{lstlisting}[numbers=none]
val alt : 'c 't parser -> 'c 't parser -> 'c 't parser
\end{lstlisting}
combines the two input parsers by applying them both to the input and appending the lists that they return, i.e. it defines non-deterministic choice.

Various alternative formulations of this concept that better control dynamic cost or have other useful properties have also been described. For example, Hutton and Meijer describe a parser combinator in monadic style \cite{hutton1998monadic}. Okasaki has described an alternative design that uses continuations to control cost \cite{Okasaki98b}.

In some cases, it is acceptable to take a composition of parser combinators as definitional (as opposed to the usual view, where a parser is an \emph{implementation} of a separate syntax definition.)

Syntax dialects implemented using parser combinators or by a {parser generator} usually operate as language-external \emph{preprocessors}, transforming source text into a well-formed program. Some compilers integrate preprocessing into the build process, e.g. the OCaml compiler \cite{ocaml-manual}. Other systems use directives placed in source text to invoke a preprocessor. For example, using Racket's reader macro system, the programmer can direct the lexer (called the ``reader'') to shift control to a given parser when a certain directive or token is seen \cite{Flatt:2012:CLR:2063176.2063195}.

%The most minimal syntax definition systems, e.g. Racket's dialect preprocessor \cite{Flatt:2012:CLR:2063176.2063195}, take any function of a type like \li{string -> exp}, where \li{exp} is a system-defined encoding of the syntax of expressions (enriched perhaps with source code locations and other ``metadata''), as a syntax definition. Programmers using these systems are free to use an implementation of some other syntax definition system to define this function.

\subsubsection{Example 1: $\mathcal{V}_\texttt{rx}$}
\begin{figure}
\begin{lstlisting}[numbers=none]
val ssn = SURL/\d\d\d-\d\d-\d\d\d\d/EURL
fun lookup_rx(name : string) => SURL/@EURLnameSURL: %EURLssnSURL/EURL
\end{lstlisting}
\caption{Derived regex expression forms in $\mathcal{V}_\texttt{rx}$}
\label{fig:derived-spliced-subexpressions}
\end{figure}
Let us now consider a dialect of VerseML's textual syntax called $\mathcal{V}_\texttt{rx}$, defined using some syntax definition system like those just described, that  builds in derived forms related to the recursive type \li{rx} that was defined in Figure \ref{fig:datatype-rx}.\footnote{Technically, $\mathcal{V}_\texttt{rx}$ must be a dialect of the textual syntax of VerseML's expansion language; cf. Chap. \ref{chap:uetsms}.}

$\mathcal{V}_\texttt{rx}$ extends the syntax of expressions with  \emph{derived regex literals}, which are delimited by forward slashes, e.g.:
\begin{lstlisting}[numbers=none]
/SURLA|T|G|CEURL/
\end{lstlisting}
The desugaring of this form is equivalent to the following, assuming an environment where the variables \li{Or} and \li{Str} stand for the corresponding regex constructors:
\begin{lstlisting}[numbers=none]
Or(Str "SSTRAESTR", Or (Str "SSTRTESTR", Or (Str "SSTRGESTR", Str "SSTRCESTR")))
\end{lstlisting}
It is not reasonable to assume that \li{Or} and \li{Str} are bound appropriately at every use site. In order to maintain \emph{context independence}, the desugaring applies the explicit \li{fold} and \li{inj} operators as discussed in Sec. \ref{sec:lists}.\footnote{In SML, where datatypes are generative and the constructors values are necessary, we would need to use a technique like the one described for $\mathcal{V}_\text{RX}$, next.}



$\mathcal{V}_\texttt{rx}$ also supports regex literals that contain {subexpressions}, to capture the idioms that arise when constructing regexes compositionally. For example, the definition of \li{lookup_rx} in Figure \ref{fig:derived-spliced-subexpressions} is equivalent to the definition of \li{lookup_rx} that was given in Figure \ref{fig:lookup_rx}, i.e. it constructs a regex from a string, \li{name}, and another regex, \li{ssn}. The prefix \li{SURL@EURL} followed by the identifier \li{name} causes the expression \lstinline{name} to appear in the desugaring as if it was wrapped in the \li{Str} constructor, and the prefix \li{SURL%EURL} 
followed by the identifier \li{ssn} causes \lstinline{ssn} to appear in the desugaring directly. We refer to the subexpressions that appear inside literal forms as \emph{spliced subexpressions}. 


%  The body of \li{example_rx} could equivalently be written as follows:
% \begin{lstlisting}[numbers=none]
% Seq(Str(name), Seq(Str "SSTR: ESTR", ssn))
% \end{lstlisting}
% (Again, the desugaring itself must use the explicit \li{fold} and \li{inj} operators to maintain context-independence.)
%Notice that \li{name} appears wrapped in the label \li{Str} because it was prefixed by \li{@}, whereas \li{ssn} appears unadorned because it was prefixed by \li{%}. 

To splice in an expression that is not of variable form, e.g. a function application, we must delimit it with parentheses:
\begin{lstlisting}[numbers=none]
/SURL@(EURLcapitalize nameSURL): %(EURLssnSURL)EURL/
\end{lstlisting}

Finally, $\mathcal{V}_\texttt{rx}$ extends the syntax of patterns with analagous \emph{derived regex pattern literals}. For example, the definition of \li{is_dna_rx} in Figure \ref{fig:derived-pattern-syntax} is equivalent to the definition of \li{is_dna_rx} that was given in Figure \ref{fig:is_dna_rx}. Notice that the variables bound by the patterns in Figure \ref{fig:derived-pattern-syntax} appear inside \emph{spliced sub-patterns}.

\subsubsection{Example 2: $\mathcal{V}_\text{RX}$}
\begin{figure}
\begin{lstlisting}[numbers=none]
fun is_dna_rx(r : rx) : boolean => 
  match r with 
  | SURL/A/EURL => True
  | SURL/T/EURL => True
  | SURL/G/EURL => True
  | SURL/C/EURL => True
  | SURL/%(EURLr1SURL)%(EURLr2SURL)/EURL => (is_dna_rx r1) andalso (is_dna_rx r2)
  | SURL/%(EURLr1SURL)|%(EURLr2SURL)/EURL => (is_dna_rx r1) andalso (is_dna_rx r2)
  | SURL/%(EURLrSURL)*/EURL => is_dna_rx r'
  | _ => False
  end
\end{lstlisting}
\vspace{-5px}
\caption{Derived regex pattern forms in $\mathcal{V}_\texttt{rx}$}
\label{fig:derived-pattern-syntax}
\end{figure}
\begin{figure}
\begin{lstlisting}[numbers=none]
fun is_dna_rx'(r : R.t) : boolean => 
  match R.unfold_norm r with 
  | SURL/A/EURL => True
  | SURL/T/EURL => True
  | SURL/G/EURL => True
  | SURL/C/EURL => True
  | SURL/%(EURLr1SURL)%(EURLr2SURL)/EURL => (is_dna_rx' r1) andalso (is_dna_rx' r2)
  | SURL/%(EURLr1SURL)|%(EURLr2SURL)/EURL => (is_dna_rx' r1) andalso (is_dna_rx' r2)
  | SURL/%(EURLrSURL)*/EURL => is_dna_rx r'
  | _ => False
  end
\end{lstlisting}\vspace{-5px}
\caption{Derived pattern forms in $\mathcal{V}_\text{RX}$}
\label{fig:VRX-pats}
\end{figure}

In Sec. \ref{sec:syntax-examples-regexps}, we gave a more sophisticated formulation of our regex library organized around the signature \li{RX} defined in Figure \ref{fig:signature-RX}. Let us define another dialect of VerseML's textual syntax called $\mathcal{V}_\text{RX}$ that defines derived forms whose desugarings involve modules that implement \li{RX}. For this to work in a  context-independent manner, these forms must take the particular module that is to appear in the desugaring as a spliced term. For example, in the following program fragment, the module \li{R} is ``passed into'' each derived form for use in its desugaring as a spliced module:
\begin{lstlisting}[numbers=none]
val ssn = RSURL./\d\d\d-\d\d\d\d-\d\d\d/EURL
fun lookup_rx'(name : string) => RSURL./@EURLnameSURL: %EURLssnSURL/EURL
\end{lstlisting}
The desugaring of the body of \li{lookup_rx'} is:
\begin{lstlisting}[numbers=none]
R.Seq(R.Str(name), R.Seq(R.Str "SSTR: ESTR", ssn))
\end{lstlisting}
This is context-independent because the constructors are explicitly qualified (i.e. \li{Seq} and \li{Str} are \emph{field labels} here, not variables.) The only variables are \li{R}, \li{name} and \li{ssn}, all of which were given by the client at the application site.

Recall that \li{RX} specifies a function \li{unfold_norm : t -> t u} for computing the normal unfolding of the given regex, which is a value of type \li{R.t u}. $\mathcal{V}_\text{RX}$ defines derived forms for patterns matching values of types in the type family \li{'a u}. These appear in the definition of \li{is_dna_rx'} given in Figure \ref{fig:VRX-pats}.


\subsubsection{Combining $\mathcal{V}_\text{rx}$ and $\mathcal{V}_\text{RX}$}

Notice that the derived regex pattern forms that appear in Figure \ref{fig:VRX-pats} are identical to those that appear in Figure \ref{fig:derived-pattern-syntax}. Their desugarings are, however, different. In particular, the patterns in Figure \ref{fig:VRX-pats} match values of type \li{R.t u}, whereas the patterns in Figure \ref{fig:derived-pattern-syntax} match values of type \li{rx}. 

These two examples were written in different syntax dialects. However, it would be useful to have derived forms for values of type \li{rx} available even when we are working with a value of a type \li{R.t}, because we have defined a function \li{view : R.t -> rx} in \li{RXUtil}. This brings us to the first of the two main problems with the dialect-oriented approach, already described in Chapter \ref{chap:intro}: there is no good way to conservatively combine $\mathcal{V}_\text{rx}$ and $\mathcal{V}_\text{RX}$. In particular, any such ``combined dialect'' will either fail to conserve determinism, or the combined dialect will not be a dialect of both of the constituent dialects, i.e. some of the forms from one dialect will ``shadow'' the overlapping forms from the other dialect (depending on the order in which they were combined \cite{Ford04a}.) 

In response to this problem, Schwerdfeger and Van Wyk have developed a ``nearly'' modular analysis that accepts only deterministic extensions of a base LALR(1) grammar where all new forms must start with a ``marking'' terminal symbol and obey certain other constraints related to  the follow sets of the base grammar's non-terminals \cite{conf/pldi/SchwerdfegerW09}. By relying on a context-aware scanner (a feature of Copper \cite{conf/gpce/WykS07}) to transfer control when the marking terminals are seen, extensions of a base grammar that pass this analysis and specify distinct marking terminals can be combined without introducing conflict. The analysis is ``nearly'' modular in that only a relatively simple ``combine-time'' check that the set of marking terminals is disjoint is necessary.

For the two dialects just considered, these conditions are not satisfied. If we modify the grammar of $\mathcal{V}_\text{RX}$ so that, for example, the regex literal forms are marked with \li{#\dolla#r} and the regex unfolding forms were marked with \li{#\dolla#u}, the analysis will accept both grammars, and the combine-time disjointness check will pass, solving our immediate problem at only a small cost. However, a conflict could still  arise later when a client combines these extensions with another extension that also uses the marking terminals \li{#\dolla#r}, \li{#\dolla#u} or \li{/}. %There is no reason to believe that other dialect providers will avoid these marking terminals.

The solution given in \cite{conf/pldi/SchwerdfegerW09} is 1) to allow for the grammar's name to be used as an additional syntactic prefix when a conflict arises, and 2) to adopt a naming convention for grammars  based on the Internet domain name system (or some similar coordinating system.) Figure \ref{fig:VRX-pats} shows how a client needs to draw \li{is_dna_rx'} when a conflict arises. Clearly, this drawing has substantially higher syntactic cost than the drawing in Figure \ref{fig:VRX-pats}. Moreover, there is no simple way for clients to selectively control this cost by defining scoped abbreviations for marking tokens or grammar names (as one does for types, modules or values that are exported from deeply nested modules) because this mechanism is only syntactic, i.e. agnostic to the binding structure of the base language.

\todo{mention this? \url{http://www.ccs.neu.edu/home/ejs/papers/tfp12-island.pdf}}
\begin{figure}
\begin{lstlisting}[numbers=none]
fun is_dna_rx'(r : R.t) : boolean => 
  match R.unfold_norm r with 
  | SURL$cmu_edu_comar_rx $u/A/EURL => True 
  | SURL$cmu_edu_comar_rx $u/T/EURL => True
  | SURL$cmu_edu_comar_rx $u/G/EURL => True
  | SURL$cmu_edu_comar_rx $u/C/EURL => True
  (* and so on *)
  | _ => False
  end
\end{lstlisting}
\caption{Using URL-based marking tokens to avoid syntactic conflicts.}
\label{fig:vanwyk}
\end{figure}

\subsubsection{Abstract Reasoning About Derived Forms}
In addition to the difficulties of conservatively combining syntax dialects, there are a  number of other difficulties related to the fact that there is often no useful notion of syntactic abstraction that a programmer can rely on to reason about an unfamiliar derived form. The programmer may need to examine the desugaring, the desugaring logic or even the definitions of all of the constituent dialects, to definitively answer the questions given in Sec. \ref{sec:abs-reasoning-intro}. These questions were stated relative to a particular example involving the query processing language K.  
Here, we generalize from that example to develop an informal classification of the difficulties that programmers might encounter in analagous situations. In each case, we will discuss exceptional systems where these difficulties are ameliorated or avoided.% We discuss some exceptions from amongst the related work above:

\begin{enumerate}
\item \textbf{Search:} It is not always straightforward to determine which constituent dialect is responsible for any particular derived form.

The system implemented by Copper \cite{conf/pldi/SchwerdfegerW09} is an exception, in that the marking terminal (and the grammar name, if necessary) allows clients to search across the constituent dialect definitions for the corresponding declaration without needing to understand any of them deeply.
\item \textbf{Segmentation:} It is not always possible to segment a derived form such that each segment consists either of a spliced base language term (which we have drawn in black in the examples in this document) or a sequence of characters that are parsed otherwise (which we have drawn in color.) When it is possible, determining the segmentation is not always straightforward.

For example, consider a production in a grammar that looks like this: 
\begin{lstlisting}[numbers=none]
start <- "%(" verseml_exp ")"
\end{lstlisting}

The name of the non-terminal \li{verseml_exp} suggests that it will match any VerseML expression, but it is not certain that this is the case. Moreover, even if we know that this non-terminal matches VerseML expressions, it is not certain that the output logic will insert that expression as-is into the desugaring -- it may instead only examine its form, or transform it in some way (in which case highlighting it as a spliced expression might be misleading.)

Systems that support the generation of editor plug-ins, such as Spoofax \cite{kats2010spoofax} and Sugarclipse for SugarJ \cite{Erdweg:2012:GLE}, can generate syntax coloring logic from an annotated grammar definition, which often give programmers some indication of where a spliced term occurs. However, there is no definitive information about segmentation in how the editor displays the derived form. (Moreover, these editor plug-ins can themselves conflict, even if the syntax itself is deterministic.)
\item \textbf{Shadowing:} The desugaring of a derived form might place spliced terms under binders. These binders are not visible in the program text, but can shadow those that are. This obscures the binding structure of the program.

For derived forms that desugar to module-level definitions (e.g. to one or more \li{val} definitions), a desugaring might introduce locally-scoped bindings and, simultaneously, exported module components that are similarly invisible in the text. This can cause both local shadowing as well as non-local shadowing if a client \li{open}s the module into scope.

In most cases, shadowing is inadvertent. For example, a desugaring might bind an intermediate value to some temporary variable, \li{tmp}. This can cause problems at use sites where \li{tmp} is bound. If the types of the two bindings are incompatible, the problem will be caught statically. Otherwise, it will cause unanticipated dynamic behavior. It is easy to miss this problem in testing.

In some syntax dialects, shadowing is by design. For example, in (Sugar)Haskell, \li{do} notation for monadic values introduces a new binding construct \cite{erdweg2012layout}. For programmers who {are} familiar with \li{do} notation, this can be useful. But when a programmer encounters an unfamiliar form, this forces them to determine whether it similarly is designed as a new binding construct. A simple grammar provides no information about shadowing.%The point is simply that this is a double-edged sword.

In most systems, it is possible for dialect providers to generate identifiers that are guaranteed to be fresh at the use site. If dialect providers are disciplined about using this mechanism, they can prevent such  conflicts. However, this is awkward and most systems provide not guarantee that the dialect provider maintained this freshness discipline \cite{conf/ecoop/ErdwegSD14}.

To enforce a prohibition on shadowing, the system must be integrated into or otherwise made aware of the binding structure of the language. For example, some of the language-integrated mixfix systems discussed above, e.g. Coq's notation system \cite{Coq:manual}, enforce a prohibition on shadowing by alpha-renaming desugarings as necessary. Erdweg et al. have developed a formalism for directly describing the ``binding structure'' of program text, as well as contextual transformations that use these descriptions to rename the identifiers that appear in a desugaring (and more generally, a rewriting) to avoid shadowing \cite{conf/ecoop/ErdwegSD14,conf/sle/RitschelE15}.\footnote{These papers refer to this property as ``capture avoidance''. We use the term ``shadowing'' rather than ``capture'' because ``capture'' has several incompatible meanings in the literature.}

\item \textbf{Context Dependence:} If the desugaring of a derived form assumes that certain identifiers are bound at the use site (e.g. to particular values, or to values of some particular type), we refer to the desugaring as being \emph{context dependent}. 

%One might describe such desugarings as having ``captured'' bindings from the use site. Notice that this is distinct from the situation described above, where it is a term at the use site that ``captures'' bindings from the desugaring. %We will avoid the term ``capture''. %We use the word ``hygiene'' to refer collectively to context independence and shadowing avoidance. 

Context dependent desugarings take control over naming away from clients. Moreover, it is difficult to determine the assumptions that a desugaring is making. As such, it is difficult to reason about whether renaming an identifier or moving a binding is a meaning-preserving transformation. 

In our examples above, we maintained context independence as a ``courtesy'' by explicitly applying the \li{fold} and \li{inj} operators, or by taking the module for use in the desugaring as a ``syntactic argument''. 

To enforce context independence, the system must be aware of binding structure and have some way to distinguish those subterms of a desugaring that originate in the text at the use site (which should have access to bindings at the use site) from those that do not (which should only have access to bindings internal to the desugaring.) 
For example, language-integrated mixfix systems, e.g. Coq's notation system, use a simple rewriting system to compute desugarings, so they satisfy these requirements and can enforce context independence. Coq gives desugarings access only to the bindings visible where the notation was defined.

More flexible systems where desugarings are computed functionally, or language-external systems that have no understanding of binding structure, do not enforce context independence.

\item \textbf{Typing:} Finally, and perhaps most importantly, it is not always clear what type an expression drawn in derived form has, or what type of value that a pattern drawn in derived form matches.

Similarly, it is not always straightforward to determine what type a spliced expression has, or what type of value that a spliced pattern matches.

SoundExt/SugarFomega \cite{conf/popl/LorenzenE16} and SoundX \cite{conf/sle/RitschelE15} allow dialect providers to define derived typing rules alongside derived forms and desugaring rules. These systems automatically verify that the desugaring rules are sound with respect to these derived typing rules. This ensures that type errors are never reported in terms of the desugaring (which is the stated goal of their work). However, this helps only to a limited extent in answering the questions just given. In particular, the programmer must construct a derivation using the derived typing rules introduced by all of the constituent dialects, then examine this derivation to answer questions about the type of the desugaring and the spliced terms within it. 

Even for relatively simple base languages, like System $\mathbf{F}_\omega$, understanding a typing derivation requires significantly more expertise than programmers usually need.\footnote{At CMU, we teach ML to all first-year students (in 15-150.) However, understanding a judgemental specification of a language like System $\mathbf{F}_\omega$ involves skills that are taught only to some third and fourth year students (in 15-312.)} For languages like ML, the judgement forms are substantially more complex. 
\end{enumerate}

As discussed in Sec. \ref{sec:problems-with-dialects}, languages with a rich type and binding structure are designed to minimize or eliminate the cognitive cost of answering analagous questions. These reasoning principles are central to the claim that such languages are suitable for ``programming in the large'' \cite{DeRemer76}. 

Due to the problems of syntactic conflict and the reasoning difficulties enumerated above, the textual syntax of VerseML cannot be modified or extended from within.% (There is, of course, no way to stop programmers from defining dialects of VerseML using any language-external syntax definition system of their choosing. Our goal is only to make this a less compelling option for library providers seeking only to capture idioms like those that we have discussed.) 

 % reasonable to obscure the type bind and binding structure by using these mechanisms.
% One approach would be to define both this encoding and the recursive labeled sum type \li{Rx} and define a parameterized module (i.e. a \emph{functor} in SML) that maps between them, given any module \li{R : RX}: 

% \begin{lstlisting}[numbers=none]
% structure RxHelper(R : RX) = 
% struct
%   fun to_R : Rx -> R.t = (* ... *)
%   fun of_R : R.t -> Rx = (* ... *)
% end
% \end{lstlisting}

% For example, given a particular module \li{R : RX}, we can generate the helper module \li{RH} as follows:

% \begin{lstlisting}[numbers=none]
% structure RH = RxHelper(R)
% \end{lstlisting}

% then, if we are using the VerseML/Rx syntax dialect, we can use the derived forms described previously in the argument position of \li{RH.to_r}:

% \begin{lstlisting}[numbers=none]
% let ssn = RH.to_R /SURL\d\d\d-\d\d\d\d-\d\d\dEURL/
% \end{lstlisting}

% One problem with this approach is that it makes using the spliced forms awkward. For example, consider writing the function \li{example_rx} in this manner:

% \begin{lstlisting}[numbers=none]
% fun example_R(name : string) => RH.to_R /SURL@EURLnameSURL: %(EURLRH.of_R ssnSURL)EURL/\end{lstlisting}

% Notice that we had to transform \li{ssn}, which is of type \li{R.t}, back into a value of type \li{Rx} in order to splice it into the expression above. The value of this expression is then immediately transformed back into a value of type \li{R.t} by \li{RH.to_R}. This is both syntactically awkward and incurs dynamic cost, i.e. it is an $\mathcal{O}(n)$ operation, where $n$ is the size of the regex being spliced. In this particular case, the cost may be negligible, but for large data structures, this may no longer be the case.




% \subsubsection{Direct Syntax Extension}\label{sec:direct-syntax-extension}
% One tempting alternative to dynamic string parsing is to use a system that gives the users of a language the power to directly extend its concrete syntax with new derived forms. %for regular expression patterns.% for patterns.

% The simplest such systems are those where the elaboration of each new syntactic form is defined by a single rewrite rule. For example, Gallina, the ``external language'' of the Coq proof assistant, supports such extensions \cite{Coq:manual}. A formal account of such a system has been developed by Griffin \cite{5134}. Unfortunately, a single equation is not enough to allow us to express pattern syntax following the usual conventions. For example, a system like Coq's cannot handle escape characters, because there is no way to programmatically examine a form when generating its expansion.

% Other syntax extension systems are more flexible. For example, many are based on context-free grammars, e.g.  Sugar* \cite{erdweg2013framework} and Camlp4 \cite{ocaml-manual} (amongst many others). Other systems give library providers direct programmatic access to the parse stream, like Common Lisp's \emph{reader macros} \cite{steele1990common} (which are distinct from its term-rewriting macros, described in Sec. \ref{sec:term-rewriting} below) and Racket's preprocessor \cite{Flatt:2012:CLR:2063176.2063195}. All of these would allow us to add pattern syntax into our language's grammar, perhaps following Unix conventions and supporting splicing syntax as described above:
% \begin{lstlisting}[numbers=none]
% let val ssn = /SURL\d\d\d-\d\d-\d\d\d\dEURL/
% fun example_shorter(name : string) => /SURL@EURLnameSURL: %EURLssn/
% \end{lstlisting}
% %The body of this function elaborates to the body of \lstinline{example_fixed} as shown above. 
% %Had we mistakenly written \lstinline{%name}, we would encounter only a static type error, rather than the  silent injection  vulnerability discussed above. 

% We sidestep the problems of dynamic string parsing described above  when we directly extend the syntax of our language using any of these systems. Unfortunately, direct syntax extension introduces serious new problems. First, the systems mentioned thus far cannot guarantee that {syntactic conflicts} between such extensions will not arise. As stated directly in the  Coq manual: ``mixing different symbolic notations in [the] same text may cause serious parsing ambiguity''. If another library provider used similar syntax for a different implementation or variant of regular expressions, or for some other unrelated construct, then a client could not simultaneously use both libraries at the same time. So properly considered, every combination of extensions introduced using these mechanisms creates a \emph{de facto} syntactic dialect of our language. The benefit of these systems is only that they lower the implementation cost of constructing syntactic dialects. % Resolving such parsing amibiguities is left to each client of the library. 

% In response to this problem, Schwerdfeger and Van Wyk developed a modular analysis that accepts only context-free grammar extensions that begin with an identifying starting token and obey certain constraints on  the follow sets of base language's non-terminals \cite{conf/pldi/SchwerdfegerW09}. Extensions that specify distinct starting tokens and that satisfy these constraints can be used together in any combination without the possibility of syntactic conflict. However, the most natural starting tokens like \lstinline{rx} cannot be guaranteed to be unique. To address this problem, programmers must agree on a convention for defining ``globally unique identifiers'', e.g. the common URI convention used on the web and by the Java packaging system. However, this forces us to use a more verbose token like \lstinline{edu_cmu_VerseML_rx}. There is no simple way for clients of our extension to define scoped abbreviations for starting tokens because this mechanism operates purely at the level of the context-free grammar.

% In particular, if det(H) and det(R) and the set of marking terminals on dialects such that if OK(H) and OK(R) and starttokens(H) disjoint from starttokens(R) then det(H cup R). This is not quite modular, in that we still need to check that the start tokens are disjoint at ``combination-time''. To be confident that this check will not fail, a community might adopt a social convention, e.g. using URIs as start tokens. 


% Putting this aside, we must also consider another modularity-related question: which particular module should the expansion use? Clearly, simply assuming that some module identified as \lstinline{R} matching \lstinline{RX} is in scope is a brittle solution. In fact, we should expect that the system actively prevents such capture of specific variable names to ensure that variables (here, module variables) can be freely renamed. Such a \emph{hygiene discipline} is well-understood only when performing term-to-term rewriting (discussed below) or in simple language-integrated rewrite systems like those found in Coq. For mechanisms that operate strictly at the level of context-free grammars or the parse stream, it is not clear how one could address this issue. The onus is then on the library provider to make no assumptions about variable names and instead require that the client explicitly identify the module they intend to use as an ``argument'' within the newly introduced form:
% \begin{lstlisting}[numbers=none]
% let val ssn = edu_cmu_VerseML_rx R /SURL\d\d\d-\d\d-\d\d\d\dEURL/
% \end{lstlisting}

% Another problem with the approach of direct syntax extension is that, given an unfamiliar piece of syntax, there is no straightforward method for determining what type it will have, causing difficulties for both humans (related to code comprehension) and tools. 

% \todo{Related work I haven't mentioned yet:}
% \begin{itemize}
% \item Fan: http://zhanghongbo.me/fan/start.html
% \item Well-Typed Islands Parse Faster: \\\url{http://www.ccs.neu.edu/home/ejs/papers/tfp12-island.pdf}
% \item User-defined infix operators
% \item SML quote/unquote 
% \item That Modularity paper
% \item Template Haskell and similar
% \end{itemize}
\subsection{Rewriting Systems}\label{sec:term-rewriting}
Another approach is to leave the textual syntax of the language fixed, but repurpose it for novel ends using a \emph{term rewriting system}.

\subsubsection{Language-External Term Rewriting Systems}
Language-external rewriting systems operate as \emph{preprocessors}, transforming well-formed program fragments to produce other well-formed program fragments.

For example, one could define a preprocessor that rewrites every string literal that is followed by the comment \li{(*rx*)} to the corresponding expression (or pattern) of type \li{rx}, following the approach discussed in the previous section. For example, the following expression would be rewritten to a regex expression, with \li{dna} treated as a spliced subexpression as described in the previous section:
\begin{lstlisting}[numbers=none]
"SSTRGC%(dna)GCESTR"(*rx*)
\end{lstlisting}

OCaml 4.02 introduced \emph{extension points} into its textual syntax \cite{ocaml-manual-4.02}. Extension points serve as markers for the benefit of a preprocessor. They are less \emph{ad hoc} than comments, in that each extension point is associated with a single term in a well-defined way, and the compiler gives an error if any extension points remain after preprocessing is complete. For example, in the following program fragment, 
\begin{lstlisting}[numbers=none]
let%lwt (x, y) = f in x + y
\end{lstlisting}
the \li{%lwt} 
annotation on the let expression causes a preprocessor distributed with \li{Lwt}, a lightweight threading library, to rewrite this fragment to:
\begin{lstlisting}[numbers=none]
Lwt.bind f (fun (x, y) -> x + y)
\end{lstlisting}
The OCaml system is distributed with a library called \li{ppx_tools} that simplifies the task of writing  preprocessors that operate on terms annotated with extension points.

These systems present conceptual problems that are directly analagous to those that dialect-oriented systems present:
\begin{enumerate}
\item \textbf{Conflict:} Different preprocessors may recognize the same markers.
\item \textbf{Search:} It is not always clear which preprocessor handles each marked form.
\item \textbf{Segmentation:} It is not always clear where spliced sub-terms appear inside marked forms (particularly string literals).
\item \textbf{Shadowing:} The rewriting of a marked form might place terms under binders that shadow bindings visible in the program text.
\item \textbf{Context Dependence:} The rewriting of a marked form might assume that certain identifiers are bonud at the use site, making it difficult to reason about refactoring.
\item \textbf{Typing:} It is not always clear what type the rewriting of a marked form will have.
\end{enumerate}  

\todo{Astar macros? Extensible compilers that give you pattern matching?}

\subsubsection{Macro Systems}
Macro systems allow programmers to designate functions that operate over term encodings as macros, and then apply these macros directly to terms as rewritings. \todo{search is solved. delimitation is solved. conflict is solved.}

The LISP macro system \cite{Hart63a} is perhaps the most prominent example of such a system. Early variants of this system suffered from the problem of hygiene described earlier \todo{elaborate}, but  later variants, notably in the Scheme dialect of LISP, brought support for enforcing hygiene \cite{Kohlbecker86a}. 

In languages with a richer static type discipline, variants of macros that restrict rewriting to a particular type and perform the rewriting statically have also been studied \cite{Herman10:Theory,ganz2001macros} and integrated into languages, e.g. MacroML \cite{ganz2001macros} and Scala \cite{ScalaMacros2013}. \todo{actually MacroML isn't a term rewriting system; just a staging system. write about this.} \todo{typing is sort of solved...}

The most immediate problem with using these for our example is that we are not aware of any such statically-typed macro system that integrates cleanly with an ML-style module system. In other words, macros cannot be parameterized by modules. However, let us imagine such a macro system. We could use it to repurpose string syntax  as follows:
\begin{lstlisting}[numbers=none]
let val ssn = rx R {rx|SSTR\d\d\d-\d\d-\d\d\d\dESTR|rx}
\end{lstlisting}

The definition of the macro \lstinline{rx} might look like this:
\begin{lstlisting}
macro rx[Q : RX](e) at Q.t {
  static fun f(e : Exp) : Exp => case(e) {
      StrLit(s) => (* regex parser here *)
    | BinOp(Caret, e1, e2) => `SQTQ.Seq(Q.Str(%EQTe1SQT), %(EQTf e2SQT))EQT`
    | BinOp(Plus, e1, e2) => `SQTQ.Seq(%(EQTf e1SQT), %(EQTf e2SQT))EQT`
    | _ => raise Error
  }
}
\end{lstlisting}

Here, \lstinline{rx} is a macro parameterized by a module matching \lstinline{rx} (we identify it as \lstinline{Q} to emphasize that there is nothing special about the identifier \lstinline{R}) and taking a single argument, identified as \lstinline{e}. The macro specifies a type annotation, \lstinline{at Q.t}, which imposes the constraint that the expansion the macro statically generates must be of type \lstinline{Q.t} for the provided parameter \lstinline{Q}. This expansion is generated by a \emph{static function} that examines the syntax tree of the provided argument (syntax trees are of a type \lstinline{Exp} defined in the standard library; cf. SML/NJ's visible compiler \cite{SML/VisibleCompiler}). If it is a string literal, as in the example above, it statically parses the literal body to generate an expansion (the details of the parser, elided on line 3, would be entirely standard). 
By parsing the string statically, we avoid the problems of dynamic string parsing for statically-known patterns. 

For patterns that are constructed compositionally, we need to get more creative. For example, we might repurpose the infix operators that are normally used for other purposes to support string and pattern splicing, e.g. as follows:

\begin{lstlisting}[numbers=none,escapechar=|]
fun example_using_macro(name : string) => 
  rx R (name ^ "SSTR: ESTR" + ssn)
\end{lstlisting}

The binary operator \lstinline{^} is repurposed to indicate a spliced string and \lstinline{+} is repurposed to indicate a spliced pattern. The logic for handling these forms can be seen above on lines 4 and 5, respectively. We assume that there is derived syntax available at the type \lstinline{Exp}, i.e. \emph{quasiquotation} syntax as in Lisp \cite{Bawd99a} and Scala \cite{shabalin2013quasiquotes}, here delimited by backticks and using the prefix \lstinline{%} to indicate a spliced value (i.e. unquote). 

Having to creatively repurpose existing forms in this way limits the effect a library provider can have on cognitive cost (particularly when it would be desirable to express conventions that are quite different from the conventions adopted by the language). It also can create confusion for readers expecting parenthesized expressions to behave in a consistent manner. However,  this approach is preferable to direct syntax extension because it avoids many of the problems discussed above: there cannot be syntactic conflicts (because the syntax is not extended at all), we can define macro abbreviations because macros are integrated into the language, there is a hygiene discipline that guarantees that the expansion will not capture variables inadvertently, and by using a typed macro system, programmers need not examine the expansion to know what type the expansion produced by a macro must have. 

% \subsection{Active Libraries}
% The design we are proposing also has conceptual roots in earlier work on \emph{active libraries}, which similarly envisioned using compile-time computation to give library providers more control over various aspects of a programming system, including its concrete syntax (but did not take an approach rooted in the study of type systems) \cite{active-libraries-thesis}. 
% TODO FLESH THIS OUT 


% \part{Simple TLMs}\label{part:simple-tsms}
% !TEX root = omar-thesis.tex
\chapter{Unparameterized Expression TSMs (ueTSMs)}\label{chap:uetsms}
We now introduce a new primitive -- the \textbf{typed syntax macro} (TSM). TSMs, like term-rewriting macros (Sec. \ref{sec:term-rewriting}), generate expansions. Unlike term-rewriting macros, TSMs are applied to unparsed \emph{generalized literal forms}, which gives them substantially more syntactic flexibility. This chapter considers perhaps the simplest manifestation of TSMs: \textbf{unparameterized expression TSMs} (ueTSMs), which generate expressions of a single specified type. We will consider unparameterized pattern TSMs (upTSMs) in Chapter \ref{chap:uptsms} and parameterized TSMs (pTSMs) in Chapter \ref{chap:ptsms}.

%Like the term-rewriting macros just described, TSMs can be parameterized by modules, so they can be used to define syntax valid at any abstract type defined by a module satisfying a specified signature. As we will discuss in the remainder of this section, this addresses all of the problems brought up above, at moderate syntactic cost.

\section{Expression TSMs By Example}\label{sec:tsms-by-example}
%A typed syntax macro is invoked by applying it to a \emph{delimited form}, which can contain  arbitrary syntax in its \emph{body}.  
We begin in this section with a ``tutorial-style'' introduction to ueTSMs in VerseML. In particular, we discuss a ueTSM for constructing values of the recursive labeled sum type \li{Rx} that was defined in Figure \ref{fig:datatype-rx}. We then formally specify ueTSMs with a reduced calculus, $\miniVerseUE$, in Sec. \ref{sec:tsms-minimal-formalism}. %We conclude in Sec. \ref{sec:uetsms-discussion} 

\subsection{Usage}\label{sec:uetsms-usage}
In the following concrete VerseML expression, we apply a TSM named \li{#\dolla#rx} to the \emph{generalized literal form} \li{/SURLA|T|G|CEURL/}:
\begin{lstlisting}[numbers=none,mathescape=|]
$rx /SURLA|T|G|CEURL/
\end{lstlisting}
Generalized literal forms are left unparsed when concrete expressions are first parsed. It is only during the subsequent \emph{typed expansion} process that the TSM parses the \emph{body} of the provided literal form, i.e. the characters between forward slashes in blue here, to generate a \emph{candidate expansion}. The language then \emph{validates} the candidate expansion according to criteria that we will establish in Sec. \ref{sec:uetsms-validation}. If candidate expansion validation succeeds, the language generates the \emph{final expansion} (or more concisely, simply the \emph{expansion}) of the expression. The program will behave as if the expression above has been replaced by its expansion. The expansion of the expression above, written concretely, is:
\begin{lstlisting}[numbers=none]
Or(Str "SSTRAESTR", Or(Str "SSTRTESTR", Or(Str "SSTRGESTR", Str "SSTRCESTR")))
\end{lstlisting}
%The constructors above are those of the type \li{Rx} that was defined in Figure \ref{fig:datatype-rx}.

A number of literal forms, shown in Figure \ref{fig:literal-forms},  are available in VerseML's concrete syntax. Any literal form can be used with any TSM, e.g. we could have equivalently written the example above as \li{#\dolla#rx `SURLA|T|G|CEURL`} (in fact, this would be convenient if we had wanted to express a regex containing forward slashes but not backticks). TSMs have access only to the literal bodies. Because TSMs do not extend the concrete syntax of the language directly, there cannot be syntactic conflicts between TSMs.

 %The form does not directly determine the expansion. 

\begin{figure}
\begin{lstlisting}
'SURLbody cannot contain an apostropheEURL'
`SURLbody cannot contain a backtickEURL`
[SURLbody cannot contain unmatched square bracketsEURL]
{SURLbody cannot contain an unmatched curly braceEURL}
/SURLbody cannot contain a forward slashEURL/
\SURLbody cannot contain a backslashEURL\
\end{lstlisting}
%SURL<tag>body includes enclosing tags</tag>EURL
\caption[Available Generalized Literal Forms]{Generalized literal forms available for use in VerseML's concrete syntax. The characters in blue indicate where the literal bodies are located within each form. In this figure, each line describes how the literal body is constrained by the form shown on that line. The Wyvern language specifies additional forms, including whitespace-delimited forms \cite{TSLs} and multipart forms \cite{sac15}, but for simplicity we leave these out of VerseML.}
\label{fig:literal-forms}
\end{figure}
\subsection{Definition}\label{sec:uetsms-definition}
%The original expression, above, is statically rewritten to this expression.
Let us now take the perspective of the library provider. The definition of the TSM \lstinline{#\dolla#rx} shown being applied above has the following form:
\begin{lstlisting}[numbers=none,mathescape=|]
syntax $rx at Rx {
  static fn(body : Body) : CEExp ParseResult => 
    (* regex literal parser here *)
}
\end{lstlisting}
This {TSM definition} first names the TSM. 
 TSM names must begin with the dollar symbol (\li{#\dolla#}) to clearly distinguish them from variables (and thereby clearly distinguish TSM application from function application). This is inspired by a similar convention enforced by the Rust macro system \cite{Rust/Macros}.

The TSM definition then specifies a \emph{type annotation}, \lstinline{at Rx}, and a \emph{parse function} within curly braces. 
The {parse function} is a \emph{static function} responsible for parsing the literal body when the TSM is applied to generate an encoding of the candidate expansion, or an indication of an error if one cannot be generated (e.g. when the body is ill-formed according to the syntactic specification that the TSM implements). Static functions are functions that are applied during the typed expansion process. For this reason, they do not have access to surrounding variable bindings (because those variables stand in for dynamic values). For now, let us simply assume that static functions are closed (we discuss introducing a distinct class of static bindings so that static values can be shared between TSM definitions in Sec. \ref{sec:uetsms-static-language}).

The parse function must have type \li{Body -> CEExp ParseResult}. These types are defined in the VerseML \emph{prelude}, which is a collection of definitions available ambiently. The input type, \lstinline{Body}, gives the parse function access to the {body} of the provided literal form. For our purposes, it suffices to define \li{Body} as an abbreviation for the \li{string} type:
\begin{lstlisting}[numbers=none]
type Body = string
\end{lstlisting} 

The output type, \li{CEExp ParseResult}, is a labeled sum type that distinguishes between successful parses and parse errors. The parameterized type \li{'a ParseResult} is defined in Figure \ref{fig:indexrange-and-parseresult}.

If parsing succeeds, the parse function returns a value of the form \li{Success(#$\ecand$#)}, where $\ecand$ is the \emph{encoding of the candidate expansion}. Encodings of candidate expansions are, for expression TSMs, values of the type \lstinline{CEExp} defined in Figure \ref{fig:candidate-exp-verseml} (in Chapter \ref{chap:ptsms}, we will introduce pattern TSMs, which generate patterns rather than expressions; this is why \li{ParseResult} is defined as a parameterized type). Expressions can mention types, so we also need to define a type \li{CETyp} in Figure \ref{fig:candidate-exp-verseml}. 
\begin{figure}
\begin{lstlisting}[numbers=none]
type IndexRange = {startIndex : nat, endIndex : nat}

type 'a ParseResult = Success of 'a 
                    | ParseError of {
                        msg : string, loc : IndexRange
                      }
\end{lstlisting}
\caption[Definitions of \li{IndexRange} and \li{ParseResult} in VerseML]{Definitions of \li{IndexRange} and \li{ParseResult} in the VerseML prelude.}
\label{fig:indexrange-and-parseresult}
\end{figure}
\begin{figure}
\begin{lstlisting}[numbers=none]
type CETyp = TyVar of var_t 
           | Arrow of CETyp * CETyp 
           | (* ... *) 
           | Spliced of IndexRange

type CEExp = Var of var_t 
           | Fn of var_t * CETyp * CEExp
           | App of CEExp * CEExp
           | (* ... *) 
           | Spliced of IndexRange
\end{lstlisting}
\caption[Abbreviated definitions of \li{CETyp} and \li{CEExp} in VerseML]{Abbreviated definitions \li{CETyp} and \li{CEExp} in the VerseML prelude. We assume some suitable type \li{var_t} exists, not shown.}
\label{fig:candidate-exp-verseml}
\end{figure}
% We will show a complete encoding when we describe our reduced formal system $\miniVerseUE$ in Sec. \ref{sec:tsms-minimal-formalism}. 
We discuss the constructors labeled \li{Spliced} in Sec. \ref{sec:splicing-and-hygiene}; the remaining constructors (some of which are elided for concision) encode the abstract syntax of VerseML expressions and types. To decrease the syntactic cost of working with the types defined in Figure \ref{fig:candidate-exp-verseml}, the prelude provides \emph{quasiquotation syntax} at these types, which is itself implemented using TSMs. We will discuss these TSMs in more detail in Sec. \ref{sec:tsms-for-tsms}. The definitions in Figure \ref{fig:candidate-exp-verseml} are recursive labeled sum types to simplify our exposition, but we could have chosen alternative encodings of terms, e.g. based on abstract binding trees \cite{pfpl}, with only minor modification to the semantics. % It is extended with one additional form used to handled spliced subexpressions, 

If the parse function determines that a candidate expansion cannot be generated, i.e. there is a parse error in the literal body, it returns a value labeled by \li{ParseError}. It must provide an error message and indicate the location of the error within the body of the literal form as a value of type \li{IndexRange}, also defined in Figure \ref{fig:indexrange-and-parseresult}. This information can be used by VerseML compilers when reporting errors to the programmer.

%Notice that the types just described are those that one would expect to find in a typical parser.

%One would find types analagous to those just described in any parser, so for concision, we elide the details of \li{#\dolla#rx}'s parse function.
%The parse function must treat the TSM parameters parametrically, i.e. it does not have access to any values in the supplied module parameter. Only the expansion the parse function generates can refer to module parameters. 
%For example, the following definition is ill-sorted:
%\begin{lstlisting}[numbers=none]
%syntax pattern_bad[Q : PATTERN] at Q.t {
%  static fn (body : Body) : Exp => 
%    if Q.flag then (* ... *) else (* ... *)
%}
%\end{lstlisting}%So the parse function parses the body of the delimited form to generate an encoding of the elaboration.

\subsection{Splicing}\label{sec:splicing-and-hygiene}
To support splicing syntax, like that described in Sec. \ref{sec:syntax-examples-regexps}, the parse function must be able to parse subexpressions out of the supplied literal body. For example, consider the code snippet in Figure \ref{fig:derived-spliced-subexpressions}, expressed instead using the \li{#\dolla#rx} TSM:
\begin{lstlisting}[numbers=none]
val ssn = $rx /SURL\d\d\d-\d\d-\d\d\d\dEURL/
fun example_rx_tsm(name: string) => $rx /SURL@EURLnameSURL: %EURLssn/
\end{lstlisting}
The subexpressions \lstinline{name} and \lstinline{ssn} on the second line appear directly in the body of the literal form, so we call them \emph{spliced subexpressions} (and color them black when typesetting them in this document). When the parse function determines that a subsequence of the literal body should be treated as a spliced subexpression (here, by recognizing the characters \lstinline{@} or \lstinline{%} followed by a variable or parenthesized expression), 
it can refer to it within the candidate expansion it generates using the \li{Spliced} constructor of the \li{CEExp} type shown in Figure \ref{fig:candidate-exp-verseml}. The \li{Spliced} constructor requires a value of type \li{IndexRange} because spliced subexpressions are referred to indirectly by their position within the literal body. This prevents TSMs from ``forging'' a spliced subexpression (i.e. claiming that an expression is a spliced subexpression, even though it does not appear in the body of the literal form). Expressions can also contain types, so one can also mark spliced types in an analagous manner using the \li{Spliced} constructor of the \li{CETyp} type. %In particular, the parse function must provide the index range of spliced subexpressions to the \li{Spliced} constructor of the type \li{MarkedExp}. %Only subexpressions that actually appear in the body of the literal form can be marked as spliced subexpressions.

The candidate expansion generated by \li{#\dolla#rx} for the body of \lstinline{example_rx_tsm}, if written in a hypothetical concrete syntax for candidate expansions where references to spliced subexpressions are written \li{spliced<startIdx, endIndex>}, is:
\begin{lstlisting}[numbers=none]
Seq(Str(spliced<1, 4>), Seq(Str "SSTR: ESTR", spliced<8, 10>))
\end{lstlisting}
Here, \li{spliced<1, 4>} refers to the subexpression \li{name} by position and \li{spliced<8, 10>} refers to the subexpression \li{ssn} by position. 

%For example, had the  would not be a valid expansion, because the  that are not inside spliced subexpressions:
%\begin{lstlisting}[numbers=none]
%Q.Seq(Q.Str(name), Q.Seq(Q.Str ": ", ssn))
%\end{lstlisting}

\subsection{Typing}\label{sec:uetsms-validation}
The language \emph{validates} candidate expansions before a final expansion is generated. One aspect of candidate expansion validation is checking  the candidate expansion against the type annotation specified by the TSM, e.g. the type \li{Rx} in the example above. This maintains a \emph{type discipline}: if a programmer sees a TSM being applied when examining a well-typed program, they need only look up the TSM's type annotation to determine the type of the generated expansion. Determining the type does not require examine the expansion directly.


\subsection{Hygiene}
The spliced subexpressions that the candidate expansion refers to (by their position within the literal body, cf. above) must be parsed, typed and expanded during the candidate expansion validation process (otherwise, the language would not be able to check the type of the candidate expansion). To maintain a useful \emph{binding discipline}, i.e. to allow programmers to reason also about variable binding without examining expansions directly, the validation process maintains two additional properties related to spliced subexpressions: \textbf{context independent expansion} and \textbf{expansion independent splicing}. These are collectively referred to as the \emph{hygiene properties} (because they are conceptually related to the concept of hygiene in term rewriting macro systems, cf. Sec. \ref{sec:term-rewriting}.) 

\paragraph{Context Independent Expansion} Programmers expect to be able to choose variable and symbol names freely, i.e. without needing to satisfy ``hidden assumptions'' made by the TSMs that are applied in scope of a binding. For this reason, context-dependent candidate expansions, i.e. those with free variables or symbols, are deemed invalid (even at application sites where those variables happen to be bound). An example of a TSM that generates context-dependent candidate expansions is shown below:
\begin{lstlisting}[numbers=none]
syntax $bad1 at Rx {
	static fn(body : Body) : ParseResultExp => Success (Var 'SSTRxESTR')
}
\end{lstlisting}
The candidate expansion this TSM generates would be well-typed only when there is an assumption \li{x : Rx} in the application site typing context. This ``hidden assumption'' makes reasoning about binding and renaming especially difficult, so this candidate expansion is deemed invalid (even when \li{#\dolla#bad1} is applied in a context where \li{x} happens to be bound).

Of course, this prohibition does not extend into the spliced subexpressions referred to in a candidate expansion because spliced subexpressions are authored by the TSM client and appear at the application site, and so can justifiably refer to application site bindings. We saw examples of spliced subexpressions that referred to variables bound at the application site in Sec. \ref{sec:splicing-and-hygiene}. Because candidate expansions refer to spliced subexpressions indirectly, checking this property is straightforward -- we only allow access to the application site typing context when typing spliced subexpressions. In the next section, we will formalize this intuition. % The TSM provider can only refer to them opaquely.

In the examples in Sec. \ref{sec:uetsms-usage} and Sec. \ref{sec:splicing-and-hygiene}, the expansion used constructors associated with the \li{Rx} type, e.g. \li{Seq} and \li{Str}. This might appear to violate our prohibition on context-dependent expansions. This is not the case only because in VerseML, constructor labels are not variables or scoped symbols. Syntactically, they must begin with a capital letter (like Haskell's datatype constructors). Different labeled sum types can use common constructor labels without conflict because the type the term is being checked against -- e.g. \li{Rx}, due to the type ascription on \li{#\dolla#rx} -- determines which type of value will be constructed. For dialects of ML where datatype definitions do introduce new variables or scoped symbols, we need parameterized TSMs. We will return to this topic in Chapter \ref{chap:ptsms}. % Indeed, we used the label \li{Spliced} for two different recursive labeled sum types in Figure \ref{fig:candidate-exp-verseml}.

\paragraph{Expansion Independent Splicing} Spliced subexpressions, as just described, must be given access to application site bindings. The \emph{expansion independent splicing} property ensures that spliced subexpressions have access to \emph{only} those bindings, i.e. a TSM cannot introduce new bindings into spliced subexpressions. For example, consider the following hypothetical candidate expansion (written concretely as above):
\begin{lstlisting}[numbers=none]
fn(x : Rx) => spliced<0, 4>
\end{lstlisting}
The variable \li{x} is not available when typing the indicated spliced subexpression, nor can it shadow any bindings of \li{x} that might appear at the application site.

For TSM providers, the benefit of this property is that they can choose the names of variables used internally within expansions freely, without worrying about whether they might shadow those that a client might have defined at the application site.

TSM clients can, in turn, determine exactly which bindings are available in a spliced subexpression without examining the expansion it appears within. In other words, there can be no ``hidden variables''. 

The trade-off is that this prevents library providers from defining  alternative binding forms. For example, Haskell's derived form for monadic commands (i.e. \li{do}-notation) supports binding the result of executing a command to a variable that is then available in the subsequent commands in a command sequence. In VerseML, this cannot be expressed in the same way. We will show an alternative formulation of Haskell's syntax for monadic commands that uses VerseML's anonymous function syntax to bind variables in Sec. \ref{sec:application-monadic-commands}. We will discuss mechanisms that would allow us to relax this restriction while retaining client control over variable names as future work in Sec. \ref{sec:controlled-binding}.

%These properties suffice to ensure that programmers and tools can freely rename a variable without changing the meaning of the program. The only information that is necessary to perform such a \emph{rename refactoring} is the locations of spliced subexpressions within all the literal forms for which the variable being renamed is in scope; the expansions need not otherwise be examined. It would be straightforward to develop a tool and/or editor plugin to indicate the locations of spliced subexpressions to the user, like we do in this document (by coloring spliced subexpressions black). We discuss tool support as future work in Sec. \ref{sec:interaction-with-tools}.

\subsection{Final Expansion}
If validation succeeds, the language generates the \emph{final expansion} from the candidate expansion by replacing references to spliced subexpressions with their final expansions. The final expansion of the body of \li{example_rx_tsm} is:
\begin{lstlisting}[numbers=none]
Seq(Str(name), Seq(Str "SSTR: ESTR", ssn))
\end{lstlisting}

\subsection{Scoping}
A benefit of specifying TSMs as a language primitive, rather than relying on extralinguistic mechanisms to manipulate the concrete syntax of our language directly, is that TSMs follow standard scoping rules.

For example, we can define a TSM that is visible only to a single expression like this:
\begin{lstlisting}[numbers=none]
let x = (syntax $rx at Rx { (* ... *) } in 
           (* $rx is in scope here *) 
         end)
in (* $rx is no longer in scope *) end
\end{lstlisting}

If the \li{in} clause is omitted, the scope of the TSM extends to the end of the current block. 
We will consider the question of how TSM definitions can be exported from compilation units in Sec. \ref{sec:tsm-packaging}.

\subsection{Comparison to ML+Rx}
Let us compare the VerseML TSM \li{#\dolla#rx} to ML+Rx, the hypothetical syntactic dialect of ML with support for derived forms for regular expressions described in Sec. \ref{sec:syntax-examples-regexps}.

Both ML+Rx and \li{#\dolla#rx} give programmers the ability to use the same standard syntax for constructing regexes, including syntax for splicing in other strings and regexes. In VerseML, however, we incur the additional syntactic cost of explicitly applying the \li{#\dolla#rx} TSM each time we wish to use regex syntax. This cost does not grow with the size of the regex, so it would only be significant in programs that involve a large number of small regexes (which do, of course, exist). In Chapter \ref{chap:tsls} we will consider a design where even this syntactic cost can be eliminated in certain situations.

The benefit of this approach is that we can easily define other TSMs to use alongside the \li{#\dolla#rx} TSM without needing to consider the possibility of syntactic conflict. Furthermore, programmers can rely on the typing discipline and the hygienic binding discipline described above to reason about programs, including those that contain unfamiliar forms. Put pithily, VerseML helps programmers avoid ``conflict and confusion''. 


\section{\texorpdfstring{$\miniVerseUE$}{miniVerseUE}}\label{sec:tsms-minimal-formalism}\label{sec:miniVerseU}

% \begin{figure}[p!]
% $\begin{array}{lllllll}
% \textbf{variables} & \textbf{type variables} & \textbf{labels} & \textbf{label sets} & \textbf{TSM variables} & \textbf{literal bodies} & \textbf{nats}\\
% x & t & \ell & \labelset & \tsmv & b & n\\~\end{array}$\\
% $\begin{array}{ll}
% \textbf{type formation contexts} & \textbf{typing contexts}\\
% \Delta ::= \emptyset ~\vert~ \Delta, t & \Gamma ::= \emptyset ~\vert~ \Gamma, x : \tau\\
% ~
% \end{array}$\\
% ~\\
% $\begin{array}{lcl}
% \gheading{types}\\
% \tau & ::= & t ~\vert~ \parr{\tau}{\tau} ~\vert~ \forallt{t}{\tau} ~\vert~ \rect{t}{\tau} ~\vert~  \prodt{\mapschema{\tau}{i}{\labelset}} ~\vert~ \sumt{\mapschema{\tau}{i}{\labelset}}\\
% ~\\
% \gheading{expanded expressions}\\
% e & ::= & x ~\vert~ \lam{x}{\tau}{e} ~\vert~ \app{e}{e} ~\vert~ \Lam{t}{e} ~\vert~ \App{e}{\tau} ~\vert~ \fold{t}{\tau}{e} ~\vert~ \unfold{e} ~\vert~ \tpl{\mapschema{e}{i}{\labelset}} ~\vert~ \prj{e}{\ell} \\
% & \vert & \inj{\ell}{e} ~\vert~ \caseof{e}{\mapschemab{x}{e}{i}{\labelset}}\\
% ~\\
% \gheading{TSM expressions}\\
% \tsme & ::= & \tsmv ~\vert~ \utsmdef{\tau}{\ue}\\
% ~\\
% \gheading{unexpanded expressions}\\
% \ue & ::= & {x} ~\vert~ \lam{x}{\tau}{\ue} ~\vert~ \ue(\ue) ~\vert~ \Lam{t}{\ue} ~\vert~ \App{\ue}{\tau} ~\vert~ \fold{t}{\tau}{\ue} ~\vert~ \unfold{\ue} ~\vert~ \tpl{\mapschema{\ue}{i}{\labelset}} ~\vert~ \prj{\ue}{\ell} \\
% & \vert & \inj{\ell}{\ue} ~\vert~ \caseof{\ue}{\mapschemab{x}{\ue}{i}{\labelset}}\\
% & \vert & \uesyntax{\tsmv}{\tsme}{\ue} ~\vert~ \utsmapp{\eta}{b}\\
% ~\\
% \gheading{candidate expansion types}\\
% \mtau & ::= & t ~\vert~ \parr{\mtau}{\mtau} ~\vert~ \forallt{t}{\mtau} ~\vert~ \rect{t}{\mtau} ~\vert~ \prodt{\mapschema{\tau}{i}{\labelset}} ~\vert~ \sumt{\mapschema{\mtau}{i}{\labelset}} \\
% & \vert & \mtspliced{\tau}\\
% ~\\
% \gheading{candidate expansion expressions}\\
% \me & ::= & x ~\vert~ \lam{x}{\mtau}{\me} ~\vert~ \app{\me}{\me} ~\vert~ \Lam{t}{\me} ~|~ \App{\me}{\mtau} ~\vert~ \fold{t}{\mtau}{\me} ~\vert~ \unfold{\me} ~\vert~ \tpl{\mapschema{\me}{i}{\labelset}} ~\vert~ \prj{\me}{\ell} \\
% & \vert & \inj{\ell}{\me} ~\vert~ \caseof{\me}{\mapschemab{x}{\me}{i}{\labelset}}\\
% & \vert & \mspliced{e}
% % \\~
% \end{array}$
% \todo{finish breaking this up into syntax tables}
% \caption[Syntax of $\miniVerseUE$]{Syntax of $\miniVerseUE$. The forms $\mapschema{V}{i}{\labelset}$ and $\mapschemab{x}{V}{i}{\labelset}$ where $V$ is a metavariable indicate finite functions from each label $i \in \labelset$ to a term, $V_i$, or binder, $x_i.V_i$, respectively.}
% \label{fig:lambda-tsm-syntax}
% \end{figure}


To make the intuitions developed in the previous section mathematically precise, we will now introduce a reduced language called $\miniVerseUE$ with support for ueTSMs. $\miniVerseUE$ consists of an \emph{inner core} and an \emph{outer surface}.
%For reference, the syntax of $\miniVerseUE$ is specified in Figure \ref{fig:lambda-tsm-syntax}. We will reproduce relevant portions of this specification inline (in tabular form) as we continue. 
%We specify all formal systems in this document within the metatheoretic framework detailed in \emph{PFPL} \cite{pfpl}, and assume familiarity of fundamental background concepts (e.g. abstract binding trees, substitution, implicit identification of terms up to $\alpha$-equivalence, structural induction and rule induction) covered therein. %Familiarity with other accounts of typed lambda calculi should also suffice to understand the formal systems in this document. 



\subsection{Syntax of the Inner Core}\label{sec:U-expanded-terms}

\begin{figure}
\hspace{-5px}$\begin{array}{lllllll}
\textbf{Sort} & & & \textbf{Operational Form} & \textbf{Stylized Form} & \textbf{Description}\\
\mathsf{Typ} & \tau & ::= & t & t & \text{variable}\\
&&& \aparr{\tau}{\tau} & \parr{\tau}{\tau} & \text{partial function}\\
&&& \aall{t}{\tau} & \forallt{t}{\tau} & \text{polymorphic}\\
&&& \arec{t}{\tau} & \rect{t}{\tau} & \text{recursive}\\
&&& \aprod{\labelset}{\mapschema{\tau}{i}{\labelset}} & \prodt{\mapschema{\tau}{i}{\labelset}} & \text{labeled product}\\
&&& \asum{\labelset}{\mapschema{\tau}{i}{\labelset}} & \sumt{\mapschema{\tau}{i}{\labelset}} & \text{labeled sum}\\
\mathsf{Exp} & e & ::= & x & x & \text{variable}\\
&&& \aelam{\tau}{x}{e} & \lam{x}{\tau}{e} & \text{abstraction}\\
&&& \aeap{e}{e} & \ap{e}{e} & \text{application}\\
&&& \aetlam{t}{e} & \Lam{t}{e} & \text{type abstraction}\\
&&& \aetap{e}{\tau} & \App{e}{\tau} & \text{type application}\\
&&& \aefold{t}{\tau}{e} & \fold{e} & \text{fold}\\
&&& \aeunfold{e} & \unfold{e} & \text{unfold}\\
&&& \aetpl{\labelset}{\mapschema{e}{i}{\labelset}} & \tpl{\mapschema{e}{i}{\labelset}} & \text{labeled tuple}\\
&&& \aepr{\ell}{e} & \prj{e}{\ell} & \text{projection}\\
&&& \aein{\labelset}{\ell}{\mapschema{\tau}{i}{\labelset}}{e} & \inj{\ell}{e} & \text{injection}\\
&&& \aecase{\labelset}{\tau}{e}{\mapschemab{x}{e}{i}{\labelset}} & \caseof{e}{\mapschemab{x}{e}{i}{\labelset}} & \text{case analysis}
\end{array}$
\caption[Syntax of types and expanded expressions in $\miniVerseUE$]{Abstract syntax of types and expanded expressions, which form the \emph{inner core of }$\miniVerseUE$. Metavariables $x$ range over variables, $t$ over type variables, $\ell$ over labels and $\labelset$ over finite sets of labels. We adopt \emph{PFPL}'s conventions for operational forms, i.e. the names of operators and indexed families of operators are written in $\texttt{typewriter font}$, indexed families of operators specify non-symbolic indices within $[\text{mathematical braces}]$ and symbolic indices within \texttt{[}textual braces\text{]}, and term arguments are grouped arbitrarily (roughly, by ``phase'') using \texttt{\{}textual curly braces\texttt{\}} and \texttt{(}textual rounded braces\texttt{)} \cite{pfpl}. We write $\mapschema{\tau}{i}{\labelset}$ for a sequence of arguments $\tau_i$, one for each $i\in \labelset$, and similarly for arguments of other valences. Operations  parameterized by label sets, e.g. $\aprod{\labelset}{\mapschema{\tau}{i}{\labelset}}$, are identified up to mutual reordering of the label set and the corresponding argument sequence. %When using stylized forms, the label set is omitted when it can be inferred, e.g. the labeled product type $\prodt{\finmap{\mapitem{\ell_1}{e_1}, \mapitem{\ell_2}{e_2}}}$ leaves the label set $\{\ell_1, \ell_2\}$ implicit. 
When we use the stylized forms, we assume that the reader can infer suppressed indices and arguments from the surrounding context. Types and expanded expressions are identified up to $\alpha$-equivalence.}
\label{fig:U-expanded-terms}
\end{figure}

The \emph{inner core of} $\miniVerseUE$ consists of \emph{types}, $\tau$, and \emph{expanded expressions}, $e$. The syntax of the inner core is specified by the syntax chart in Figure \ref{fig:U-expanded-terms}. 
The {inner core} forms a pure language with support for partial functions, quantification over types, recursive types, labeled product types and labeled sum types.  The reader is directed to \emph{PFPL} \cite{pfpl} (or another text on type systems, e.g. \emph{TAPL} \cite{tapl}) for a detailed introductory account of these (or very similar) constructs. We will tersely define the statics of the inner core, and outline the structural dynamics, in the next two subsections, respectively.

\subsection{Statics of the Inner Core}
The \emph{statics of the inner core} is defined by hypothetical judgements of the following form:

\[\begin{array}{ll}
\textbf{Judgement Form} & \textbf{Description}\\
\istypeU{\Delta}{\tau} & \text{$\tau$ is a well-formed type assuming $\Delta$}\\
%\isctxU{\Delta}{\Gamma} & \text{$\Gamma$ is a well-formed typing context assuming $\Delta$}\\
\hastypeU{\Delta}{\Gamma}{e}{\tau} & \text{$e$ is assigned type $\tau$ assuming $\Delta$ and $\Gamma$}
\end{array}\]
\noindent
\emph{Type formation contexts}, $\Delta$, are finite sets of hypotheses of the form $\Dhyp{t}$. Empty finite sets are written $\emptyset$, or omitted entirely within judgements, and non-empty finite sets are written as comma-separated finite sequences identified up to exchange and contraction. We write $\Delta, \Dhyp{t}$, when $\Dhyp{t} \notin \Delta$, for $\Delta$ extended with the hypothesis $\Dhyp{t}$. %Finite sets are written as finite sequences identified up to exchange.% We write $\Dcons{\Delta}{\Delta'}$ for the union of $\Delta$ and $\Delta'$.

The \emph{type formation judgement}, $\istypeU{\Delta}{\tau}$, is inductively defined by the following rules:
\begin{subequations}\label{rules:istypeU}
\begin{equation}\label{rule:istypeU-var}
\inferrule{ }{\istypeU{\Delta, \Dhyp{t}}{t}}
\end{equation}
\begin{equation}\label{rule:istypeU-parr}
\inferrule{
  \istypeU{\Delta}{\tau_1}\\
  \istypeU{\Delta}{\tau_2}
}{\istypeU{\Delta}{\aparr{\tau_1}{\tau_2}}}
\end{equation}
\begin{equation}\label{rule:istypeU-all}
  \inferrule{
    \istypeU{\Delta, \Dhyp{t}}{\tau}
  }{
    \istypeU{\Delta}{\aall{t}{\tau}}
  }
\end{equation}
\begin{equation}\label{rule:istypeU-rec}
  \inferrule{
    \istypeU{\Delta, \Dhyp{t}}{\tau}
  }{
    \istypeU{\Delta}{\arec{t}{\tau}}
  }
\end{equation}
\begin{equation}\label{rule:istypeU-prod}
  \inferrule{
    \{\istypeU{\Delta}{\tau_i}\}_{i \in \labelset}
  }{
    \istypeU{\Delta}{\aprod{\labelset}{\mapschema{\tau}{i}{\labelset}}}
  }
\end{equation}
\begin{equation}\label{rule:istypeU-sum}
  \inferrule{
    \{\istypeU{\Delta}{\tau_i}\}_{i \in \labelset}
  }{
    \istypeU{\Delta}{\asum{\labelset}{\mapschema{\tau}{i}{\labelset}}}
  }
\end{equation}
\end{subequations}
Premises of the form $\{{J}_i\}_{i \in \labelset}$ mean that for each $i \in \labelset$, the judgement ${J}_i$ must hold. 

\emph{Typing contexts}, $\Gamma$, are finite functions that map each variable $x \in \domof{\Gamma}$, to the hypothesis $\Ghyp{x}{\tau}$, for some $\tau$. Empty typing contexts are written $\emptyset$, or omitted entirely within judgements, and non-empty typing contexts are written as finite sequences of hypotheses identified up to exchange (we do not separately write down the finite set $\domof{\Gamma}$ because it can be determined from the listed hypotheses). We write $\Gamma, \Ghyp{x}{\tau}$, when $x \notin \domof{\Gamma}$, for the extension of $\Gamma$ with a mapping from $x$ to $\Ghyp{x}{\tau}$, and $\Gcons{\Gamma}{\Gamma'}$ when $\domof{\Gamma} \cap \domof{\Gamma'} = \emptyset$ for the typing context mapping each $x \in \domof{\Gamma} \cup \domof{\Gamma'}$ to $x : \tau$ if $x : \tau \in \Gamma$ or $x : \tau \in \Gamma'$. We write $\isctxU{\Delta}{\Gamma}$ if every type in $\Gamma$ is well-formed relative to $\Delta$.
\begin{definition}[Typing Context Formation] \label{def:isctxU}
$\isctxU{\Delta}{\Gamma}$ iff for each hypothesis $x : \tau \in \Gamma$, we have $\istypeU{\Delta}{\tau}$.
\end{definition}

The typing judgement, $\hastypeU{\Delta}{\Gamma}{e}{\tau}$, assigns types to expressions. It is inductively defined by the following rules:
\begin{subequations}\label{rules:hastypeU}
\begin{equation}\label{rule:hastypeU-var}
  \inferrule{ }{
    \hastypeU{\Delta}{\Gamma, \Ghyp{x}{\tau}}{x}{\tau}
  }
\end{equation}
\begin{equation}\label{rule:hastypeU-lam}
  \inferrule{
    \istypeU{\Delta}{\tau}\\
    \hastypeU{\Delta}{\Gamma, \Ghyp{x}{\tau}}{e}{\tau'}
  }{
    \hastypeU{\Delta}{\Gamma}{\aelam{\tau}{x}{e}}{\aparr{\tau}{\tau'}}
  }
\end{equation}
\begin{equation}\label{rule:hastypeU-ap}
  \inferrule{
    \hastypeU{\Delta}{\Gamma}{e_1}{\aparr{\tau}{\tau'}}\\
    \hastypeU{\Delta}{\Gamma}{e_2}{\tau}
  }{
    \hastypeU{\Delta}{\Gamma}{\aeap{e_1}{e_2}}{\tau'}
  }
\end{equation}
\begin{equation}\label{rule:hastypeU-tlam}
  \inferrule{
    \hastypeU{\Delta, \Dhyp{t}}{\Gamma}{e}{\tau}
  }{
    \hastypeU{\Delta}{\Gamma}{\aetlam{t}{e}}{\aall{t}{\tau}}
  }
\end{equation}
\begin{equation}\label{rule:hastypeU-tap}
  \inferrule{
    \hastypeU{\Delta}{\Gamma}{e}{\aall{t}{\tau}}\\
    \istypeU{\Delta}{\tau'}
  }{
    \hastypeU{\Delta}{\Gamma}{\aetap{e}{\tau'}}{[\tau'/t]\tau}
  }
\end{equation}
\begin{equation}\label{rule:hastypeU-fold}
  \inferrule{\
    \istypeU{\Delta, \Dhyp{t}}{\tau}\\
    \hastypeU{\Delta}{\Gamma}{e}{[\arec{t}{\tau}/t]\tau}
  }{
    \hastypeU{\Delta}{\Gamma}{\aefold{t}{\tau}{e}}{\arec{t}{\tau}}
  }
\end{equation}
\begin{equation}\label{rule:hastypeU-unfold}
  \inferrule{
    \hastypeU{\Delta}{\Gamma}{e}{\arec{t}{\tau}}
  }{
    \hastypeU{\Delta}{\Gamma}{\aeunfold{e}}{[\arec{t}{\tau}/t]\tau}
  }
\end{equation}
\begin{equation}\label{rule:hastypeU-tpl}
  \inferrule{
    \{\hastypeU{\Delta}{\Gamma}{e_i}{\tau_i}\}_{i \in \labelset}
  }{
    \hastypeU{\Delta}{\Gamma}{\aetpl{\labelset}{\mapschema{e}{i}{\labelset}}}{\aprod{\labelset}{\mapschema{\tau}{i}{\labelset}}}
  }
\end{equation}
\begin{equation}\label{rule:hastypeU-pr}
  \inferrule{
    \hastypeU{\Delta}{\Gamma}{e}{\aprod{\labelset, \ell}{\mapschema{\tau}{i}{\labelset}; \ell \hookrightarrow \tau}}
  }{
    \hastypeU{\Delta}{\Gamma}{\aepr{\ell}{e}}{\tau}
  }
\end{equation}
\begin{equation}\label{rule:hastypeU-in}
  \inferrule{
    \{\istypeU{\Delta}{\tau_i}\}_{i \in \labelset}\\
    \istypeU{\Delta}{\tau}\\
    \hastypeU{\Delta}{\Gamma}{e}{\tau}
  }{
    \hastypeU{\Delta}{\Gamma}{\aein{\labelset, \ell}{\ell}{\mapschema{\tau}{i}{\labelset}; \ell \hookrightarrow \tau}{e}}{\asum{\labelset, \ell}{\mapschema{\tau}{i}{\labelset}; \ell \hookrightarrow \tau}}
  }
\end{equation}
\begin{equation}\label{rule:hastypeU-case}
  \inferrule{
    \hastypeU{\Delta}{\Gamma}{e}{\asum{\labelset}{\mapschema{\tau}{i}{\labelset}}}\\
    \istypeU{\Delta}{\tau}\\
    \{\hastypeU{\Delta}{\Gamma, x_i : \tau_i}{e_i}{\tau}\}_{i \in \labelset}
  }{
    \hastypeU{\Delta}{\Gamma}{\aecase{\labelset}{\tau}{e}{\mapschemab{x}{e}{i}{\labelset}}}{\tau}
  }
\end{equation}
\end{subequations}
Rules (\ref{rules:istypeU}) and (\ref{rules:hastypeU}) are syntax-directed, so we assume an inversion lemma for each rule as needed without stating it separately. The following standard lemmas also hold. 

The Weakening Lemma establishes that extending a context with unnecessary hypotheses preserves well-formedness and typing.
\begin{lemma}[Weakening]\label{lemma:weakening-U} All of the following hold: 
\begin{enumerate} 
\item If $\istypeU{\Delta}{\tau}$ then $\istypeU{\Delta, \Dhyp{t}}{\tau}$.
%\item If $\isctxU{\Delta}{\Gamma}$ then $\isctxU{\Delta, \Dhyp{t}}{\Gamma}$.
\item If $\hastypeU{\Delta}{\Gamma}{e}{\tau}$ then $\hastypeU{\Delta, \Dhyp{t}}{\Gamma}{e}{\tau}$.
\item If $\hastypeU{\Delta}{\Gamma}{e}{\tau}$ and $\istypeU{\Delta}{\tau'}$ then $\hastypeU{\Delta}{\Gamma, \Ghyp{x}{\tau'}}{e}{\tau}$.
\end{enumerate}
\end{lemma}
\begin{proof-sketch} For each part, by rule induction on the assumption. 
%\begin{enumerate} 
%\item By rule induction over Rules (\ref{rules:istypeU}).
%\item By rule induction over Rules (\ref{rules:isctxU}).
%\item By rule induction over Rules (\ref{rules:hastypeU}).
%\item By rule induction over Rules (\ref{rules:hastypeU}).
%\end{enumerate}
\end{proof-sketch}

We assume that renaming of bound variables, $\alpha$-equivalence and substitution are defined as in \emph{PFPL} \cite{pfpl}. The Substitution Lemma establishes that substitution of a well-formed type for a type variable, or an expanded expression of the appropriate type for an expanded expression variable, preserves well-formedness and typing. 
\begin{lemma}[Substitution]\label{lemma:substitution-U} All of the following hold:
\begin{enumerate}
\item If $\istypeU{\Delta, \Dhyp{t}}{\tau}$ and $\istypeU{\Delta}{\tau'}$ then $\istypeU{\Delta}{[\tau'/t]\tau}$.
%\item If $\isctxU{\Delta, \Dhyp{t}}{\Gamma}$ and $\istypeU{\Delta}{\tau'}$ then $\isctxU{\Delta}{[\tau'/t]\Gamma}$.
\item If $\hastypeU{\Delta, \Dhyp{t}}{\Gamma}{e}{\tau}$ and $\istypeU{\Delta}{\tau'}$ then $\hastypeU{\Delta}{[\tau'/t]\Gamma}{[\tau'/t]e}{[\tau'/t]\tau}$.
\item If $\hastypeU{\Delta}{\Gamma, \Ghyp{x}{\tau'}}{e}{\tau}$ and $\hastypeU{\Delta}{\Gamma}{e'}{\tau'}$ then $\hastypeU{\Delta}{\Gamma}{[e'/x]e}{\tau}$.
\end{enumerate}\end{lemma}
\begin{proof-sketch}
For each part, by rule induction on the first assumption.
\end{proof-sketch}

The Decomposition Lemma is the converse of the Substitution Lemma.
\begin{lemma}[Decomposition]\label{lemma:decomposition-U} All of the following hold:
\begin{enumerate}
\item If $\istypeU{\Delta}{[\tau'/t]\tau}$ and $\istypeU{\Delta}{\tau'}$ then $\istypeU{\Delta, \Dhyp{t}}{\tau}$.
%\item If $\isctxU{\Delta}{[\tau'/t]\Gamma}$ and $\istypeU{\Delta}{\tau'}$ then $\isctxU{\Delta, \Dhyp{t}}{\Gamma}$.
\item If $\hastypeU{\Delta}{[\tau'/t]\Gamma}{[\tau'/t]e}{[\tau'/t]\tau}$ and $\istypeU{\Delta}{\tau'}$ then $\hastypeU{\Delta, \Dhyp{t}}{\Gamma}{e}{\tau}$.
\item If $\hastypeU{\Delta}{\Gamma}{[e'/x]e}{\tau}$ and $\hastypeU{\Delta}{\Gamma}{e'}{\tau'}$ then $\hastypeU{\Delta}{\Gamma, \Ghyp{x}{\tau'}}{e}{\tau}$.
\end{enumerate}\end{lemma}
\begin{proof-sketch}
\begin{enumerate}
\item By rule induction over Rules (\ref{rules:istypeU}) and case analysis on the definition of substitution. In all cases, the derivation of $\istypeU{\Delta}{[\tau'/t]\tau}$ does not depend on the form of $\tau'$.
%\item Context formation of $[\tau'/t]\Gamma$ does not depend on the structure of $\tau'$.
\item By rule induction over Rules (\ref{rules:hastypeU}) and case analysis on the definition of substitution. In all cases, the derivation of $\hastypeU{\Delta}{[\tau'/t]\Gamma}{[\tau'/t]e}{[\tau'/t]\tau}$ does not depend on the form of $\tau'$.
\item By rule induction over Rules (\ref{rules:hastypeU}) and case analysis on the definition of substitution. In all cases, the derivation of $\hastypeU{\Delta}{\Gamma}{[e'/x]e}{\tau}$ does not depend on the form of $e'$.
\end{enumerate}
\end{proof-sketch}

The Regularity Lemma establishes that the type assigned to an expanded expression under a well-formed typing context is always well-formed. 
\begin{lemma}[Regularity]\label{lemma:regularity-U} If $\hastypeU{\Delta}{\Gamma}{e}{\tau}$ and $\isctxU{\Delta}{\Gamma}$ then $\istypeU{\Delta}{\tau}$.\end{lemma}
\begin{proof-sketch}
By rule induction over Rules (\ref{rules:hastypeU}) and application of Definition \ref{def:isctxU} and Lemma \ref{lemma:substitution-U}.
\end{proof-sketch}
\subsection{Structural Dynamics}\label{sec:dynamics-U}
The \emph{structural dynamics of }$\miniVerseUE$ is specified as a transition system by judgements of the following form:
\[\begin{array}{ll}
\textbf{Judgement Form} & \textbf{Description}\\
\stepsU{e}{e'} & \text{$e$ transitions to $e'$}\\
\isvalU{e} & \text{$e$ is a value}
\end{array}\]
We also define auxiliary judgements for \emph{iterated transition}, $\multistepU{e}{e'}$, and \emph{evaluation}, $\evalU{e}{e'}$.

\begin{definition}[Iterated Transition]\label{defn:iterated-transition-U} $\multistepU{e}{e'}$ is the reflexive, transitive closure of $\stepsU{e}{e'}$.\end{definition}

\begin{definition}[Evaluation]\label{defn:evaluation-U}  $\evalU{e}{e'}$ iff $\multistepU{e}{e'}$ and $\isvalU{e'}$. \end{definition}

Our subsequent developments do not require making reference to particular rules in the structural dynamics (because TSMs operate statically), so we do not reproduce the rules here. Instead, it suffices to state the following conditions.

The Canonical Forms condition characterizes well-typed values. Satisfying this condition requires an \emph{eager} (i.e. \emph{by-value}) formulation of the dynamics. 
\begin{condition}[Canonical Forms]\label{condition:canonical-forms-U} If $\hastypeUC{e}{\tau}$ and $\isvalU{e}$ then:
\begin{enumerate}
\item If $\tau=\aparr{\tau_1}{\tau_2}$ then $e=\aelam{\tau_1}{x}{e'}$ and $\hastypeUCO{\Ghyp{x}{\tau_1}}{e'}{\tau_2}$.
\item If $\tau=\aall{t}{\tau'}$ then $e=\aetlam{t}{e'}$ and $\hastypeUCO{\Dhyp{t}}{e'}{\tau'}$.
\item If $\tau=\arec{t}{\tau'}$ then $e=\aefold{t}{\tau'}{e'}$ and $\hastypeUC{e'}{[\abop{rec}{t.\tau'}/t]\tau'}$ and $\isvalU{e'}$. 
\item If $\tau=\aprod{\labelset}{\mapschema{\tau}{i}{\labelset}}$ then $e=\aetpl{\labelset}{\mapschema{e}{i}{\labelset}}$ and $\hastypeUC{e_i}{\tau_i}$ and $\isvalU{e_i}$ for each $i \in \labelset$.
\item If $\tau=\asum{\labelset}{\mapschema{\tau}{i}{\labelset}}$ then for some label set $L'$ and label $\ell$ and type $\tau_\ell$, we have that $\labelset=\labelset', \ell$ and $\tau=\asum{\labelset', \ell}{\mapschema{\tau}{i}{\labelset'}; \mapitem{\ell}{\tau_\ell}}$ and $e=\aein{\labelset', \ell}{\ell}{\mapschema{\tau}{i}{\labelset'}; \ell \hookrightarrow \tau_\ell}{e'}$ and $\hastypeUC{e'}{\tau_\ell}$ and $\isvalU{e'}$.
\end{enumerate}\end{condition}

The Preservation condition ensures that evaluation preserves typing.  
\begin{condition}[Preservation]\label{condition:preservation-U} If $\hastypeUC{e}{\tau}$ and $\multistepU{e}{e'}$ then $\hastypeUC{e'}{\tau}$. \end{condition}
The Progress condition ensures that evaluation of a well-typed expanded expression cannot ``get stuck''.
\begin{condition}[Progress]\label{condition:progress-U} If $\hastypeUC{e}{\tau}$ then either $\isvalU{e}$ or there exists an $e'$ such that $\stepsU{e}{e'}$. \end{condition}
 Together, these two conditions constitute the Type Safety Condition.

\subsection{Syntax of the Outer Surface}\label{sec:syntax-U}
\begin{figure}
\hspace{-6px}$\arraycolsep=3.5pt\begin{array}{lllllll}
\textbf{Sort} & & & \textbf{Operational Form} & \textbf{Stylized Form} & \textbf{Description}\\
\mathsf{UTyp} & \utau & ::= & \ut & \ut & \text{sigil}\\
&&& \auparr{\utau}{\utau} & \parr{\utau}{\utau} & \text{partial function}\\
&&& \auall{\ut}{\utau} & \forallt{\ut}{\utau} & \text{polymorphic}\\
&&& \aurec{\ut}{\utau} & \rect{\ut}{\utau} & \text{recursive}\\
&&& \auprod{\labelset}{\mapschema{\utau}{i}{\labelset}} & \prodt{\mapschema{\utau}{i}{\labelset}} & \text{labeled product}\\
&&& \ausum{\labelset}{\mapschema{\utau}{i}{\labelset}} & \sumt{\mapschema{\utau}{i}{\labelset}} & \text{labeled sum}\\
\mathsf{UExp} & \ue & ::= & \ux & \ux & \text{sigil}\\
&&& \aulam{\utau}{\ux}{\ue} & \lam{\ux}{\utau}{\ue} & \text{abstraction}\\
&&& \auap{\ue}{\ue} & \ap{\ue}{\ue} & \text{application}\\
&&& \autlam{\ut}{\ue} & \Lam{\ut}{\ue} & \text{type abstraction}\\
&&& \autap{\ue}{\utau} & \App{\ue}{\utau} & \text{type application}\\
&&& \aufold{\ut}{\utau}{\ue} & \fold{\ue} & \text{fold}\\
&&& \auunfold{\ue} & \unfold{\ue} & \text{unfold}\\
&&& \autpl{\labelset}{\mapschema{\ue}{i}{\labelset}} & \tpl{\mapschema{\ue}{i}{\labelset}} & \text{labeled tuple}\\
&&& \aupr{\ell}{\ue} & \prj{\ue}{\ell} & \text{projection}\\
&&& \auin{\labelset}{\ell}{\mapschema{\utau}{i}{\labelset}}{\ue} & \inj{\ell}{\ue} & \text{injection}\\
&&& \aucase{\labelset}{\utau}{\ue}{\mapschemab{\ux}{\ue}{i}{\labelset}} & \caseof{\ue}{\mapschemab{\ux}{\ue}{i}{\labelset}} & \text{case analysis}\\
\LCC  &  & & \lightgray & \lightgray & \lightgray \\
&&& \audefuetsm{\utau}{e}{\tsmv}{\ue} & \uesyntax{\tsmv}{\utau}{e}{\ue} & \text{ueTSM definition}\\ 
&&& \autsmap{b}{\tsmv} & \utsmap{\tsmv}{b} & \text{ueTSM application}\ECC
\end{array}$
\caption[Syntax of unexpanded types and expressions in $\miniVerseUE$]{Abstract syntax of unexpanded types and expressions in $\miniVerseUE$. Metavariable $\ut$ ranges over type sigils, $\ux$ ranges over expression sigils, $\tsmv$  over TSM names and $b$ over textual sequences, which, when they appear in an unexpanded expression, are called literal bodies. Literal bodies might contain spliced subterms that are only ``surfaced'' during typed expansion, so renaming of bound identifiers and substitution are not defined over unexpanded types and expressions.}
\label{fig:U-unexpanded-terms}
\end{figure}
A $\miniVerseUE$ program ultimately evaluates as an expanded expression. However, the programmer does not write the expanded expression directly. Instead, the programmer writes a textual sequence, $b$, consisting of characters in some suitable alphabet (e.g. in practice, \texttt{ASCII} or \texttt{Unicode}), which is parsed by some partial metafunction $\mathsf{parseUExp}(b)$ to produce an \emph{unex\-panded expression}, $\ue$. Unexpanded expressions can contain \emph{unexpanded types}, $\utau$, so we also need a partial metafunction $\mathsf{parseUTyp}(b)$. The abstract syntax of unexpanded types and expressions, which form  the \emph{outer surface} of $\miniVerseUE$, is defined in Figure \ref{fig:U-unexpanded-terms}. The full definition of the textual syntax of $\miniVerseUE$, which $\mathsf{parseUExp}(b)$ and $\mathsf{parseUTyp}(b)$ implement, is not important for our purposes, so we simply give the following condition, which states that there is some way to textually represent every unexpanded type and expression. %We also assume a metafunction $\mathsf{parseUTyp}(b)$ for parsing unexpanded types, and impose an analagous condition.
\begin{condition}[Textual Representability] Both of the following must hold:
\begin{enumerate}
\item For each $\utau$, there exists $b$ such that $\parseUTyp{b}{\utau}$. 
\item For each $\ue$, there exists $b$ such that $\parseUExp{b}{\ue}$.
\end{enumerate}
\end{condition}


Unexpanded types and expressions are given meaning by expansion to types and expanded expressions, respectively, according to the \emph{typed expansion judgements}, which are defined in the next subsection.

Unexpanded types and expressions bind \emph{type sigils}, $\ut$, \emph{expression sigils}, $\ux$, and \emph{TSM names}, $\tsmv$. Sigils are given meaning by expansion to variables during typed expansion. We \textbf{cannot} adopt the usual definitions of $\alpha$-renaming of identifiers, because unexpanded types and expressions are still in a ``partially parsed'' state -- the literal bodies, $b$, within an unexpanded expression might contain spliced subterms that are ``surfaced'' by a TSM only during typed expansion, as we will detail below. %Sigils are given meaning by expansion to variables. %In other words, unexpanded expressions are not abstract binding trees, nor sequences of characters, but a ``transitional'' structure with some characteristics of each of these. 
%For this reason, we will need to handle generating fresh variables explicitly at binding sites in our semantics. %To do so, we distinguish \emph{type sigils}, $\ut$, and \emph{expression sigils}, $\ux$, from type variables, $t$, and expression variables, $x$. Sigils will be given meaning by expansion to variables (which, in turn, are given meaning by substitution, as described above). 

Each inner core form (defined in Figure \ref{fig:U-expanded-terms}) maps onto an outer surface form. We refer to these as the \emph{shared forms}. In particular:
\begin{itemize}
\item Each type variable, $t$, maps onto a unique {type sigil}, written $\sigilof{t}$ (pronounced ``sigil of $t$''). Notice the distinction between $\ut$, which is a metavariable ranging over type sigils, and $\sigilof{t}$, which is a metafunction, written in stylized form, applied to a type variable to produce a type sigil.
\item Each type form, $\tau$, maps onto an unexpanded type form, $\Uof{\tau}$, as follows: 
  \begin{align*}
  \Uof{t} &= \sigilof{t}\\
  \Uof{\aparr{\tau_1}{\tau_2}} & = \auparr{\Uof{\tau_1}}{\Uof{\tau_2}}\\
  \Uof{\aall{t}{\tau}} & = \auall{\sigilof{t}}{\Uof{\tau}}\\
  \Uof{\arec{t}{\tau}} & = \aurec{\sigilof{t}}{\Uof{\tau}}\\
  \Uof{\aprod{\labelset}{\mapschema{\tau}{i}{\labelset}}} & = \auprod{\labelset}{\mapschemax{\Uofv}{\tau}{i}{\labelset}}\\
  \Uof{\asum{\labelset}{\mapschema{\tau}{i}{\labelset}}} & = \ausum{\labelset}{\mapschemax{\Uofv}{\tau}{i}{\labelset}}
  \end{align*}
\item Each expression variable, $x$, maps onto a unique expression sigil written $\sigilof{x}$. Again, notice the distinction between $\ux$ and $\sigilof{x}$.
\item Each expanded expression form, $e$, maps onto an unexpanded expression form, $\Uof{e}$, as follows:
\begin{align*}
\Uof{x} & = \sigilof{x}\\
\Uof{\aelam{\tau}{x}{e}} & = \aulam{\Uof{\tau}}{\sigilof{x}}{\Uof{e}}\\
\Uof{\aeap{e_1}{e_2}} & = \auap{\Uof{e_1}}{\Uof{e_2}}\\
\Uof{\aetlam{t}{e}} & = \autlam{\sigilof{t}}{\Uof{e}}\\
\Uof{\aetap{e}{\tau}} & = \autap{\Uof{e}}{\Uof{\tau}}\\
\Uof{\aefold{t}{\tau}{e}} & = \aufold{\sigilof{t}}{\Uof\tau}{\Uof e}\\
\Uof{\aeunfold{e}} & = \auunfold{\Uof{e}}\\
\Uof{\aetpl{\labelset}{\mapschema{e}{i}{\labelset}}} & = \autpl{\labelset}{\mapschemax{\Uofv}{e}{i}{\labelset}}\\
\Uof{\aein{\labelset}{\ell}{\mapschema{\tau}{i}{\labelset}}{e}} &= \auin{\labelset}{\ell}{\mapschemax{\Uofv}{\tau}{i}{\labelset}}{\Uof{e}}\\
\Uof{\aecase{\labelset}{\tau}{e}{\mapschemab{x}{e}{i}{\labelset}}} & = \aucase{\labelset}{\Uof\tau}{\Uof{e}}{\mapschemabx{\Uofv}{x}{e}{i}{\labelset}}
\end{align*}
\end{itemize}

There are only two unexpanded expression forms, highlighted in gray in Figure \ref{fig:U-unexpanded-terms}, that do not correspond to expanded expression forms -- the ueTSM definition form and the ueTSM application form. %These are the ``interesting'' forms. % These are the ``interesting'' forms. % Let us define this correspondence by the metafunction $\Uof{e}$:
%\[
%\begin{split}
%\Uof{x} & = x\\
%\Uof{\aelam{\tau}{x}{e}} & = \aulam{\tau}{x}{\Uof{e}}\\
%\Uof{\aeap{e_1}{e_2}} & = \auap{\Uof{e_1}}{\Uof{e_2}}
%\end{split}
%\] and so on for the remaining expanded expression forms.


\subsection{Typed Expansion}\label{sec:typed-expansion-U}
Unexpanded expressions, and the unexpanded types therein, are checked and expanded simultaneously according to the \emph{typed expansion judgements}:
\[\begin{array}{ll}
\textbf{Judgement Form} & \textbf{Description}\\
\expandsTU{\uDelta}{\utau}{\tau} & \text{$\utau$ is well-formed and has expansion $\tau$ assuming $\uDelta$}\\
\expandsUX{\ue}{e}{\tau} & \text{$\ue$ has expansion $e$ and type $\tau$ under ueTSM context $\uPsi$}\\
& \text{assuming $\uDelta$ and $\uGamma$}
\end{array}\]
%\newcommand{\gray}[1]{{\color{gray} #1}}

\subsubsection{Type Expansion}
The \emph{type expansion judgement}, $\expandsTU{\uDelta}{\utau}{\tau}$, is inductively defined by the following rules.
\begin{subequations}\label{rules:expandsTU}
\begin{equation}\label{rule:expandsTU-var}
\inferrule{ }{\expandsTU{\uDelta, \uDhyp{\ut}{t}}{\ut}{t}}
\end{equation}
\begin{equation}\label{rule:expandsTU-parr}
\inferrule{
  \expandsTU{\uDelta}{\utau_1}{\tau_1}\\
  \expandsTU{\uDelta}{\utau_2}{\tau_2}
}{\expandsTU{\uDelta}{\auparr{\utau_1}{\utau_2}}{\aparr{\tau_1}{\tau_2}}}
\end{equation}
\begin{equation}\label{rule:expandsTU-all}
  \inferrule{
    \expandsTU{\uDelta, \uDhyp{\ut}{t}}{\utau}{\tau}
  }{
    \expandsTU{\uDelta}{\auall{\ut}{\utau}}{\aall{t}{\tau}}
  }
\end{equation}
\begin{equation}\label{rule:expandsTU-rec}
  \inferrule{
    \expandsTU{\uDelta, \uDhyp{\ut}{t}}{\utau}{\tau}
  }{
    \expandsTU{\uDelta}{\aurec{\ut}{\utau}}{\arec{t}{\tau}}
  }
\end{equation}
\begin{equation}\label{rule:expandsTU-prod}
  \inferrule{
    \{\expandsTU{\uDelta}{\utau_i}{\tau_i}\}_{i \in \labelset}
  }{
    \expandsTU{\uDelta}{\auprod{\labelset}{\mapschema{\utau}{i}{\labelset}}}{\aprod{\labelset}{\mapschema{\tau}{i}{\labelset}}}
  }
\end{equation}
\begin{equation}\label{rule:expandsTU-sum}
  \inferrule{
    \{\expandsTU{\uDelta}{\utau_i}{\tau_i}\}_{i \in \labelset}
  }{
    \expandsTU{\uDelta}{\ausum{\labelset}{\mapschema{\utau}{i}{\labelset}}}{\asum{\labelset}{\mapschema{\tau}{i}{\labelset}}}
  }
\end{equation}
\end{subequations}
\emph{Unexpanded type formation contexts}, $\uDelta$, are of the form $\uDD{\uD}{\Delta}$, where $\uD$ is a \emph{type sigil expansion context}, and $\Delta$ is a type formation context. A type sigil expansion context, $\uD$, is a finite function that maps each type sigil $\ut \in \domof{\uD}$ to the hypothesis $\vExpands{\ut}{t}$, for some type variable $t$. We write $\ctxUpdate{\uD}{\ut}{t}$ for the type sigil expansion context that maps $\ut$ to $\vExpands{\ut}{t}$ and defers to $\uD$ for all other type sigils (i.e. the previous mapping, if it exists, is updated). 
We define $\uDelta, \uDhyp{\ut}{t}$ when $\uDelta=\uDD{\uD}{\Delta}$ as an abbreviation of  \[\uDD{\ctxUpdate{\uD}{\ut}{t}}{\Delta, \Dhyp{t}}\]%type identifier expansion context is always extended/updated together with 
%We write $\uDeltaOK{\uDelta}$ when $\uDelta=\uDD{\uD}{\Delta}$ and each type variable in $\uD$ also appears in $\Delta$.
%\begin{definition}\label{def:uDeltaOK} $\uDeltaOK{\uDD{\uD}{\Delta}}$ iff for each $\vExpands{\ut}{t} \in \uD$, we have $\Dhyp{t} \in \Delta$.\end{definition}

To understand how type sigil expansion contexts operate, it is instructive to derive an expansion for the unexpanded type $\forallt{\ut}{\forallt{\ut}{\ut}}$, or in operational form, $\auall{\ut}{\auall{\ut}{\ut}}$:
\begin{mathpar}
\inferrule{
  \inferrule{
    \inferrule{ }{
      \expandsTU{\uDD{\vExpands{\ut}{t'}}{{\Dhyp{t}}, {\Dhyp{t'}}}}{\ut}{t'}
    }~\text{(\ref*{rule:expandsTU-var})}
  }{
    \expandsTU{\uDD{\vExpands{\ut}{t}}{\Dhyp{t}}}{\auall{\ut}{\ut}}{\aall{t'}{t'}}
  }~\text{(\ref*{rule:expandsTU-all})}
}{
  \expandsTU{\uDD{\emptyset}{\emptyset}}{\auall{\ut}{\auall{\ut}{\ut}}}{\aall{t}{\aall{t'}{t'}}}
}~\text{(\ref*{rule:expandsTU-all})}
\end{mathpar}
Notice that when a type sigil is bound, a fresh type variable is generated. The type sigil expansion context is extended (when the outermost binding is encountered) or updated (at all inner bindings) and the type formation context is simultaneously extended at each binding (so that typing contexts and ueTSM contexts, discussed below, that contain types that refer to the previous binding remain well-formed). Had we used type variables in the syntax and type formation contexts in the rules above, rather than type sigils and type sigil expansion contexts, derivations for unexpanded types where an inner binding shadows an outer binding would not exist, because by definition we cannot extend a type formation context with a variable it already mentions nor implicitly $\alpha$-vary the unexpanded type to sidestep this problem. 

These rules validate the following lemmas. The Type Expansion Lemma establishes that the expansion of an unexpanded type is a well-formed type.

\begin{lemma}[Type Expansion]\label{lemma:type-expansion-U} If $\expandsTU{\uDD{\uD}{\Delta}}{\utau}{\tau}$ then $\istypeU{\Delta}{\tau}$.\end{lemma}
\begin{proof} By rule induction over Rules (\ref{rules:expandsTU}). In each case, we apply the IH to or over each premise, then apply the corresponding type formation rule in Rules (\ref{rules:istypeU}). \end{proof}

The Type Expressibility Lemma establishes that every well-formed type, $\tau$, can be expressed as a well-formed unexpanded type, $\Uof{\tau}$. This requires defining the metafunction $\Uof{\Delta}$ which maps $\Delta$ onto a an unexpanded type formation context as follows:
\begin{align*}
\Uof{\emptyset} &= \uDD{\emptyset}{\emptyset}\\
\Uof{\Delta, \Dhyp{t}} &= \Uof{\Delta}, \uDhyp{\sigilof{t}}{t}
\end{align*}
\begin{lemma}[Type Expressibility]\label{lemma:type-expressibility} If $\istypeU{\Delta}{\tau}$ then $\expandsTU{\Uof{\Delta}}{\Uof{\tau}}{\tau}$.\end{lemma}
\begin{proof} By rule induction over Rules (\ref{rules:istypeU}) using the definitions of $\Uof{\tau}$ and $\Uof{\Delta}$ above. In each case, we apply the IH to or over each premise, then apply the corresponding type expansion rule in Rules (\ref{rules:expandsTU}).\end{proof}

\subsubsection{Typed Expression Expansion}
\begin{subequations}\label{rules:expandsU}
\emph{Unexpanded typing contexts}, $\uGamma$, are of the form $\uGG{\uG}{\Gamma}$, where $\uG$ is an \emph{expression sigil expansion context}, and $\Gamma$ is a typing context. An expression sigil expansion context, $\uG$, is a finite function that maps each expression sigil $\ux \in \domof{\uG}$ to the hypothesis $\vExpands{\ux}{x}$, for some expression variable, $x$. We write $\ctxUpdate{\uG}{\ux}{x}$ for the expression sigil expansion context that maps $\ux$ to $\vExpands{\ux}{x}$ and defers to $\uG$ for all other expression sigils (i.e. the previous mapping, if it exists, is updated). %We write $\uGammaOK{\uGamma}$ when $\uGamma=\uGG{\uG}{\Gamma}$ and each expression variable in $\uG$ is assigned a type by $\Gamma$.
%\begin{definition} $\uGammaOK{\uGG{\uG}{\Gamma}}$ iff for each $\vExpands{\ux}{x} \in \uG$, we have $\Ghyp{x}{\tau} \in \Gamma$ for some $\tau$.\end{definition}
%\noindent 
We define $\uGamma, \uGhyp{\ux}{x}{\tau}$ when $\uGamma = \uGG{\uG}{\Gamma}$ as an abbreviation of \[\uGG{\uG, \vExpands{\ux}{x}}{\Gamma, \Ghyp{x}{\tau}}\]

The \emph{typed expression expansion judgement}, $\expandsUX{\ue}{e}{\tau}$, is inductively defined by Rules (\ref*{rules:expandsU}) as follows. %These rules validate the following theorem, which establishes that typed expansion produces an expansion of the assigned type. 
%\begin{theorem}[Typed Expression Expansion] If $\expandsU{\uDD{\uD}{\Delta}}{\uGG{\uG}{\Gamma}}{\uPsi}{\ue}{e}{\tau}$ and $\uetsmenv{\Delta}{\uPsi}$ then $\hastypeU{\Delta}{\Gamma}{e}{\tau}$.\end{theorem}
%\begin{proof} This is the first part of Theorem \ref{thm:typed-expansion-U}, defined and proven below.\end{proof}

\paragraph{Shared Forms} Rules (\ref*{rule:expandsU-var}) through (\ref*{rule:expandsU-case}) handle unexpanded expressions of shared form. The first five of these rules are defined below:
%Each of these rules is based on the corresponding typing rule, i.e. Rules (\ref{rule:hastypeU-var}) through (\ref{rule:hastypeU-case}), respectively. For example, the following typed expansion rules are based on the typing rules (\ref{rule:hastypeU-var}), (\ref{rule:hastypeU-lam}) and (\ref{rule:hastypeU-ap}), respectively:% for unexpanded expressions of variable, function and application form, respectively: 
\begin{equation}\label{rule:expandsU-var}
  \inferrule{ }{\expandsU{\uDelta}{\uGamma, \uGhyp{\ux}{x}{\tau}}{\uPsi}{\ux}{x}{\tau}}
\end{equation}
\begin{equation}\label{rule:expandsU-lam}
  \inferrule{
    \expandsTU{\uDelta}{\utau}{\tau}\\
    \expandsU{\uDelta}{\uGamma, \uGhyp{\ux}{x}{\tau}}{\uPsi}{\ue}{e}{\tau'}
  }{\expandsUX{\aulam{\utau}{\ux}{\ue}}{\aelam{\tau}{x}{e}}{\aparr{\tau}{\tau'}}}
\end{equation}
\begin{equation}\label{rule:expandsU-ap}
  \inferrule{
    \expandsUX{\ue_1}{e_1}{\aparr{\tau}{\tau'}}\\
    \expandsUX{\ue_2}{e_2}{\tau}
  }{
    \expandsUX{\auap{\ue_1}{\ue_2}}{\aeap{e_1}{e_2}}{\tau'}
  }
\end{equation}
\begin{equation}\label{rule:expandsU-tlam}
  \inferrule{
    \expandsU{\uDelta, \uDhyp{\ut}{t}}{\uGamma}{\uPsi}{\ue}{e}{\tau}
  }{
    \expandsUX{\autlam{\ut}{\ue}}{\aetlam{t}{e}}{\aall{t}{\tau}}
  }
\end{equation}
\begin{equation}\label{rule:expandsU-tap}
  \inferrule{
    \expandsUX{\ue}{e}{\aall{t}{\tau}}\\
    \expandsTU{\uDelta}{\utau'}{\tau'}
  }{
    \expandsUX{\autap{\ue}{\utau'}}{\aetap{e}{\tau'}}{[\tau'/t]\tau}
  }
\end{equation}
Observe that, in each of these rules, the unexpanded and expanded expression forms in the conclusion correspond, and the premises correspond to those of the typing rule for the expanded expression form, i.e. Rules (\ref{rule:hastypeU-var}) through (\ref{rule:hastypeU-tap}), respectively. In particular, each type expansion premise in each rule above corresponds to a  type formation premise in the corresponding typing rule, and each typed expression expansion premise in each rule above corresponds to a typing premise in the corresponding typing rule. The type assigned in the conclusion of each rule above is identical to the type assigned in the conclusion of the corresponding typing rule. The ueTSM context, $\uPsi$, passes opaquely through these rules (we will define ueTSM contexts below). Rules (\ref{rules:expandsTU}) were similarly generated by mechanically transforming Rules (\ref{rules:istypeU}).

We can express this scheme more precisely with the following rule transformation. For each rule in Rules (\ref{rules:istypeU}) and Rules (\ref{rules:hastypeU}),
\begin{mathpar}
\refstepcounter{equation}
% \label{rule:expandsU-tlam}
% \refstepcounter{equation}
% \label{rule:expandsU-tap}
% \refstepcounter{equation}
\label{rule:expandsU-fold}
\refstepcounter{equation}
\label{rule:expandsU-unfold}
\refstepcounter{equation}
\label{rule:expandsU-tpl}
\refstepcounter{equation}
\label{rule:expandsU-pr}
\refstepcounter{equation}
\label{rule:expandsU-in}
\refstepcounter{equation}
\label{rule:expandsU-case}
\inferrule{J_1\\ \cdots \\ J_k}{J}
\end{mathpar}
the corresponding typed expansion rule is 
\begin{mathpar}
\inferrule{
  \Uof{J_1} \\
  \cdots\\
  \Uof{J_k}
}{
  \Uof{J}
}
\end{mathpar}
where
\[\begin{split}
\Uof{\istypeU{\Delta}{\tau}} & = \expandsTU{\Uof{\Delta}}{\Uof{\tau}}{\tau} \\
\Uof{\hastypeU{\Gamma}{\Delta}{e}{\tau}} & = \expandsU{\Uof{\Gamma}}{\Uof{\Delta}}{\uPsi}{\Uof{e}}{e}{\tau}\\
\Uof{\{J_i\}_{i \in \labelset}} & = \{\Uof{J_i}\}_{i \in \labelset}
\end{split}\]
and where:
\begin{itemize}
\item $\Uof{\tau}$ is defined as follows:
  \begin{itemize}
  \item When $\tau$ is of definite form, $\Uof{\tau}$ is defined as in Sec. \ref{sec:syntax-U}.
  \item When $\tau$ is of indefinite form, $\Uof{\tau}$ is a uniquely corresponding metavariable of sort $\mathsf{UTyp}$ also of indefinite form. For example, in Rule (\ref{rule:istypeU-parr}), $\tau_1$ and $\tau_2$ are of indefinite form, i.e. they match arbitrary types. The rule transformation simply ``hats'' them, i.e. $\Uof{\tau_1}=\utau_1$ and $\Uof{\tau_2}=\utau_2$.
  \end{itemize}
\item $\Uof{e}$ is defined as follows
\begin{itemize}
\item When $e$ is of definite form, $\Uof{e}$ is defined as in Sec. \ref{sec:syntax-U}. 
\item When $e$ is of indefinite form, $\Uof{e}$ is a uniquely corresponding metavariable of sort $\mathsf{UExp}$ also of indefinite form. For example, $\Uof{e_1}=\ue_1$ and $\Uof{e_2}=\ue_2$.
\end{itemize}
\item $\Uof{\Delta}$ is defined as follows:
  \begin{itemize} 
  \item When $\Delta$ is of definite form, $\Uof{\Delta}$ is defined as above.
  \item When $\Delta$ is of indefinite form, $\Uof{\Delta}$ is a uniquely corresponding metavariable ranging over unexpanded type formation contexts. For example, $\Uof{\Delta} = \uDelta$.
  \end{itemize}
\item $\Uof{\Gamma}$ is defined as follows:
  \begin{itemize}
  \item When $\Gamma$ is of definite form, $\Uof{\Gamma}$ produces the corresponding unexpanded typing context as follows:
\begin{align*}
\Uof{\emptyset} & = \uGG{\emptyset}{\emptyset}\\
\Uof{\Gamma, \Ghyp{x}{\tau}} & = \Uof{\Gamma}, \uGhyp{\sigilof{x}}{x}{\tau}
\end{align*}
  \item When $\Gamma$ is of indefinite form, $\Uof{\Gamma}$ is a uniquely corresponding metavariable ranging over unexpanded typing contexts. For example, $\Uof{\Gamma} = \uGamma$.
\end{itemize}
\end{itemize}

It is instructive to use this rule transformation to generate Rules (\ref{rules:expandsTU}) and Rules (\ref{rule:expandsU-var}) through (\ref{rule:expandsU-tap}) above. We omit the remaining rules, i.e. Rules (\ref*{rule:expandsU-fold}) through (\ref*{rule:expandsU-case}). By instead defining these rules solely by the rule transformation just described, we avoid having to write down a number of rules that are of limited marginal interest. Moreover, this demonstrates the general technique for generating typed expansion rules for unexpanded types and expressions of shared form, so our exposition is somewhat ``robust'' to changes to the inner core. 

We can now establish the Expressibility Theorem -- that each well-typed expanded expression, $e$, can be expressed as an unexpanded expression, $\ue$, and assigned the same type under the corresponding contexts.

\begin{theorem}[Expressibility] If $\hastypeU{\Delta}{\Gamma}{e}{\tau}$ then $\expandsU{\Uof{\Delta}}{\Uof{\Gamma}}{\uPsi}{\Uof{e}}{e}{\tau}$.\end{theorem}
\begin{proof} By rule induction over Rules (\ref{rules:hastypeU}). The above rule transformation guarantees that this theorem holds by its construction. In particular, in each case, we can apply Lemma \ref{lemma:type-expressibility} to or over each type formation premise, the IH to or over each typing premise, then apply the corresponding rule in Rules (\ref{rules:expandsU}).\end{proof}
%o that when the inner core changes,  typed expansion rules  our exposition somewhat robust to changes to the inner core (though not to changes to the judgement forms in the statics of the inner core).% Even if changes to the judgement forms in the statics of the inner core are needed (e.g. the addition of a symbol context), it is easy to see would correspond to changes in the generic specification above.
% \begin{subequations}\label{rules:expandsU}
% \begin{equation}\label{rule:expandsU-var}
%   \inferrule{ }{\expandsU{\Delta}{\Gamma, x : \tau}{\uPsi}{x}{x}{\tau}}
% \end{equation}
% \begin{equation}\label{rule:expandsU-lam}
%   \inferrule{
%     \istypeU{\Delta}{\tau}\\
%     \expandsU{\Delta}{\Gamma, x : \tau}{\uPsi}{\ue}{e}{\tau'}
%   }{\expandsUX{\aulam{\tau}{x}{\ue}}{\aelam{\tau}{x}{e}}{\aparr{\tau}{\tau'}}}
% \end{equation}
% \begin{equation}\label{rule:expandsU-ap}
%   \inferrule{
%     \expandsUX{\ue_1}{e_1}{\aparr{\tau}{\tau'}}\\
%     \expandsUX{\ue_2}{e_2}{\tau}
%   }{
%     \expandsUX{\auap{\ue_1}{\ue_2}}{\aeap{e_1}{e_2}}{\tau'}
%   }
% \end{equation}
% \begin{equation}\label{rule:expandsU-tlam}
%   \inferrule{
%     \expandsU{\Delta, \Dhyp{t}}{\Gamma}{\uPsi}{\ue}{e}{\tau}
%   }{
%     \expandsUX{\autlam{t}{\ue}}{\aetlam{t}{e}}{\aall{t}{\tau}}
%   }
% \end{equation}
% \begin{equation}\label{rule:expandsU-tap}
%   \inferrule{
%     \expandsUX{\ue}{e}{\aall{t}{\tau}}\\
%     \istypeU{\Delta}{\tau'}
%   }{
%     \expandsUX{\autap{\ue}{\tau'}}{\aetap{e}{\tau'}}{[\tau'/t]\tau}
%   }
% \end{equation}
% \begin{equation}\label{rule:expandsU-fold}
%   \inferrule{
%     \istypeU{\Delta, \Dhyp{t}}{\tau}\\
%     \expandsUX{\ue}{e}{[\arec{t}{\tau}/t]\tau}
%   }{
%     \expandsUX{\aufold{t}{\tau}{\ue}}{\aefold{t}{\tau}{e}}{\arec{t}{\tau}}
%   }
% \end{equation}
% \begin{equation}\label{rule:expandsU-unfold}
%   \inferrule{
%     \expandsUX{\ue}{e}{\arec{t}{\tau}}
%   }{
%     \expandsUX{\auunfold{\ue}}{\aeunfold{e}}{[\arec{t}{\tau}/t]\tau}
%   }
% \end{equation}
% \begin{equation}\label{rule:expandsU-tpl}
%   \inferrule{
%     \{\expandsUX{\ue_i}{e_i}{\tau_i}\}_{i \in \labelset}
%   }{
%     \expandsUX{\autpl{\labelset}{\mapschema{\ue}{i}{\labelset}}}{\aetpl{\labelset}{\mapschema{e}{i}{\labelset}}}{\aprod{\labelset}{\mapschema{\tau}{i}{\labelset}}}
%   }
% \end{equation}
% \begin{equation}\label{rule:expandsU-pr}
%   \inferrule{
%     \expandsUX{\ue}{e}{\aprod{\labelset, \ell}{\mapschema{\tau}{i}{\labelset}; \mapitem{\ell}{\tau}}}
%   }{
%     \expandsUX{\aupr{\ell}{\ue}}{\aepr{\ell}{e}}{\tau}
%   }
% \end{equation}
% \begin{equation}\label{rule:expandsU-in}
%   \inferrule{
%     \{\istypeU{\Delta}{\tau_i}\}_{i \in \labelset}\\
%     \istypeU{\Delta}{\tau}\\
%     \expandsUX{\ue}{e}{\tau}
%   }{
%     \left\{\shortstack{$\Delta~\Gamma \vdash_\uPsi \auin{\labelset, \ell}{\ell}{\mapschema{\tau}{i}{\labelset}; \mapitem{\ell}{\tau}}{\ue}$\\$\leadsto$\\$\aein{\labelset, \ell}{\ell}{\mapschema{\tau}{i}{\labelset}; \mapitem{\ell}{\tau}}{e} : \asum{\labelset, \ell}{\mapschema{\tau}{i}{\labelset}; \mapitem{\ell}{\tau}}$\vspace{-1.2em}}\right\}
%   }
% \end{equation}
% \begin{equation}\label{rule:expandsU-case}
%   \inferrule{
%     \expandsUX{\ue}{e}{\asum{\labelset}{\mapschema{\tau}{i}{\labelset}}}\\
%     \{\expandsU{\Delta}{\Gamma, \Ghyp{x_i}{\tau_i}}{\uPsi}{\ue_i}{e_i}{\tau}\}_{i \in \labelset}
%   }{
%     \expandsUX{\aucase{\labelset}{\ue}{\mapschemab{x}{\ue}{i}{\labelset}}}{\aecase{\labelset}{e}{\mapschemab{x}{e}{i}{\labelset}}}{\tau}
%   }
% \end{equation}
\end{subequations}
\paragraph{ueTSM Definition and Application} The two remaining typed expansion rules, Rules (\ref{rule:expandsU-syntax}) and (\ref{rule:expandsU-tsmap}), govern the ueTSM definition and application forms, and are defined in the next two subsections, respectively. 

% \begin{equation}\label{rule:expandsU-syntax}
% \inferrule{
%   \istypeU{\Delta}{\tau}\\
%   \expandsU{\emptyset}{\emptyset}{\emptyset}{\ueparse}{\eparse}{\aparr{\tBody}{\tParseResultExp}}\\\\
%   a \notin \domof{\uPsi}\\
%   \expandsU{\Delta}{\Gamma}{\uPsi, \xuetsmbnd{\tsmv}{\tau}{\eparse}}{\ue}{e}{\tau'}
% }{
%   \expandsUX{\audefuetsm{\tau}{\ueparse}{\tsmv}{\ue}}{e}{\tau'}
% }
% \end{equation}
% \begin{equation}\label{rule:expandsU-tsmap}
% \inferrule{
%   \encodeBody{b}{\ebody}\\
%   \evalU{\ap{\eparse}{\ebody}}{\inj{\lbltxt{Success}}{\ecand}}\\
%   \decodeCondE{\ecand}{\ce}\\\\
%   \cvalidE{\emptyset}{\emptyset}{\esceneU{\Delta}{\Gamma}{\uPsi, \xuetsmbnd{\tsmv}{\tau}{\eparse}}{b}}{\ce}{e}{\tau}
% }{
%   \expandsU{\Delta}{\Gamma}{\uPsi, \xuetsmbnd{\tsmv}{\tau}{\eparse}}{\autsmap{b}{\tsmv}}{e}{\tau}
% }
% \end{equation}
%\end{subequations}

%Notice that each form of expanded expression (Figure \ref{fig:U-expanded-terms}) corresponds to a form of unexpanded expression (Figure \ref{fig:U-unexpanded-terms}). For each typing rule in Rules (\ref{rules:hastypeU}), there is a corresponding typed expansion rule -- Rules (\ref{rule:expandsU-var}) through (\ref{rule:expandsU-case}) -- where the unexpanded and expanded forms correspond. The premises also correspond -- if a typing judgement appears as a premise of a typing rule, then the corresponding premise in the corresponding typed expansion rule is the corresponding typed expansion judgement. The ueTSM context is not extended or inspected by these rules (it is only ``threaded through'' them opaquely).

%There are two unexpanded expression forms that do not correspond to an expanded expression form: the ueTSM definition form, and the ueTSM application form. The rules governing these two forms interact with the ueTSM context, and are the topics of the next two subsections, respectively.

\subsection{ueTSM Definitions}\label{sec:U-uetsm-definition}
The stylized ueTSM definition form is \[\uesyntax{\tsmv}{\utau}{\eparse}{\ue}\] 
%The operational form corresponding to this stylized form is \[\audefuetsm{\utau}{\eparse}{\tsmv}{\ue}\]
An unexpanded expression of this form defines a {ueTSM} named $\tsmv$ with \emph{unexpanded type annotation} $\utau$ and \emph{parse function} $\eparse$ for use within $\ue$. 

The parse function is an expanded expression because parse functions are applied statically (i.e. during typed expansion of $\ue$), as we will discuss when describing ueTSM application below, and evaluation is defined only for closed expanded expressions. This construction simplifies our exposition, though it is not entirely practical because it provides no way for TSM providers to share values between parse functions, nor any way to use TSMs when defining other TSMs. We discuss enriching the language to eliminate these limitations in Sec. \ref{sec:uetsms-static-language}, but it is pedagogically simpler to leave the necessary machinery out of our calculus for now.%$\miniVerseUE$.

Rule (\ref*{rule:expandsU-syntax}) defines typed expansion of ueTSM definitions (we use stylized forms for clarity):
\begin{subequations}[resume]
% \begin{equation}\label{rule:expandsU-syntax}
% \inferrule{
%   \istypeU{\Delta}{\tau}\\
%   \expandsU{\emptyset}{\emptyset}{\emptyset}{\ueparse}{\eparse}{\aparr{\tBody}{\tParseResultExp}}\\\\
%   \expandsU{\Delta}{\Gamma}{\uPsi, \xuetsmbnd{\tsmv}{\tau}{\eparse}}{\ue}{e}{\tau'}
% }{
%   \expandsUX{\audefuetsm{\tau}{\ueparse}{\tsmv}{\ue}}{e}{\tau'}
% }
% \end{equation}
\begin{equation}\label{rule:expandsU-syntax}
\inferrule{
  \expandsTU{\uDelta}{\utau}{\tau}\\
  \hastypeU{\emptyset}{\emptyset}{\eparse}{\parr{\tBody}{\tParseResultExp}}\\\\
  \expandsU{\uDelta}{\uGamma}{\uPsi, \uShyp{\tsmv}{a}{\tau}{\eparse}}{\ue}{e}{\tau'}
}{
  \expandsUX{\uesyntax{\tsmv}{\utau}{\eparse}{\ue}}{e}{\tau'}
}
\end{equation}
\end{subequations}
The premises of this rule can be understood as follows, in order:
\begin{enumerate}
\item The first premise ensures that the unexpanded type annotation is well-formed and expands it to produce the \emph{type annotation}, $\tau$.

\item The second premise checks that the parse function, $\eparse$, is closed and of type \[\parr{\tBody}{\tParseResultExp}\] %to generate the \emph{expanded parse function}, $\eparse$. 
 %Notice that this occurs under empty contexts, i.e. parse functions cannot refer to the surrounding bindings. 
%The parse function must be of type $\aparr{\tBody}{\tParseResultExp}$ where the type abbreviations $\tBody$ and $\tParseResultExp$ are defined as follows.

The type abbreviated $\tBody$ classifies encodings of literal bodies, $b$. The mapping from literal bodies to values of type $\tBody$ is defined by the \emph{body encoding judgement} $\encodeBody{b}{\ebody}$. An inverse mapping is defined   by the \emph{body decoding judgement} $\decodeBody{\ebody}{b}$.
\[\begin{array}{ll}
\textbf{Judgement Form} & \textbf{Description}\\
\encodeBody{b}{e} & \text{$b$ has encoding $e$}\\
\decodeBody{e}{b} & \text{$e$ has decoding $b$}
\end{array}\]
Rather than defining $\tBody$ explicitly, and these judgements inductively against that definition (which would be tedious and uninteresting), it suffices to define the following condition, which establishes an isomorphism between literal bodies and values of type $\tBody$ mediated by the judgements above.
\begin{condition}[Body Isomorphism] All of the following must hold:
\begin{enumerate}
\item For every literal body $b$, we have that $\encodeBody{b}{\ebody}$ for some $\ebody$ such that $\hastypeUC{\ebody}{\tBody}$ and $\isvalU{\ebody}$.
\item If $\hastypeUC{\ebody}{\tBody}$ and $\isvalU{\ebody}$ then $\decodeBody{\ebody}{b}$ for some $b$.
\item If $\encodeBody{b}{\ebody}$ then $\decodeBody{\ebody}{b}$.
\item If $\hastypeUC{\ebody}{\tBody}$ and $\isvalU{\ebody}$ and $\decodeBody{\ebody}{b}$ then $\encodeBody{b}{\ebody}$. 
\item If $\encodeBody{b}{\ebody}$ and $\encodeBody{b}{\ebody'}$ then $\ebody = \ebody'$.
\item If $\hastypeUC{\ebody}{\tBody}$ and $\isvalU{\ebody}$ and $\decodeBody{\ebody}{b}$ and $\decodeBody{\ebody}{b'}$ then $b=b'$.
\end{enumerate}
\end{condition}

$\tParseResultExp$ abbreviates a labeled sum type that distinguishes successful parses from parse errors\footnote{In VerseML, the \li{ParseError} constructor of \li{ParseResult} required an error message and an error location, but we omit these in our formalization for simplicity}:
\[\tParseResultExp \triangleq [\mapitem{\lbltxt{Success}}{\tCEExp}, \mapitem{\lbltxt{ParseError}}{\prodt{}}]\] 

The type abbreviated $\tCEExp$ classifies encodings of \emph{candidate expansion expressions} (or \emph{ce-expressions}), $\ce$ (pronounced ``grave $e$''). The syntax of ce-expressions will be described in Sec. \ref{sec:ce-syntax-U}. The mapping from ce-expressions to values of type $\tCEExp$ is defined by the \emph{ce-expression encoding judgement}, $\encodeCondE{\ce}{e}$. An inverse mapping is defined by the \emph{ce-expression decoding judgement}, $\decodeCondE{e}{\ce}$.

\[\begin{array}{ll}
\textbf{Judgement Form} & \textbf{Description}\\
\encodeCondE{\ce}{e} & \text{$\ce$ has encoding $e$}\\
\decodeCondE{e}{\ce} & \text{$e$ has decoding $\ce$}
\end{array}\]

Again, rather than picking a particular definition of $\tCEExp$ and defining the judgements above inductively against it, we only state the following condition, which establishes an isomorphism between values of type $\tCEExp$ and ce-expressions.

\begin{condition}[Candidate Expansion Expression Isomorphism] All of the following must hold:
\begin{enumerate}
\item For every $\ce$, we have $\encodeCondE{\ce}{\ecand}$ for some $\ecand$ such that $\hastypeUC{\ecand}{\tCEExp}$ and $\isvalU{\ecand}$.
\item If $\hastypeUC{\ecand}{\tCEExp}$ and $\isvalU{\ecand}$ then $\decodeCondE{\ecand}{\ce}$ for some $\ce$.
\item If $\encodeCondE{\ce}{\ecand}$ then $\decodeCondE{\ecand}{\ce}$.
\item If $\hastypeUC{\ecand}{\tCEExp}$ and $\isvalU{\ecand}$ and $\decodeCondE{\ecand}{\ce}$ then $\encodeCondE{\ce}{\ecand}$.
\item If $\encodeCondE{\ce}{\ecand}$ and $\encodeCondE{\ce}{\ecand'}$ then $\ecand=\ecand'$.
\item If $\hastypeUC{\ecand}{\tCEExp}$ and $\isvalU{\ecand}$ and $\decodeCondE{\ecand}{\ce}$ and $\decodeCondE{\ecand}{\ce'}$ then $\ce=\ce'$.
\end{enumerate}
\end{condition}


\item The final premise of Rule (\ref{rule:expandsU-syntax}) extends the ueTSM context, $\uPsi$, with the newly determined {ueTSM definition}, and proceeds to assign a type, $\tau'$, and expansion, $e$, to $\ue$. The conclusion of Rule (\ref{rule:expandsU-syntax}) assigns this type and expansion to the ueTSM definition as a whole.% i.e. TSMs define behavior that is relevant during typed expansion, but not during evaluation. 



\emph{ueTSM contexts}, $\uPsi$, are of the form $\uAS{\uA}{\Psi}$, where $\uA$ is a \emph{TSM naming context} and $\Psi$ is a \emph{ueTSM definition context}. 

A \emph{TSM naming context}, $\uA$, is a finite function mapping each TSM name $\tsmv \in \domof{\uA}$ to the \emph{TSM name-symbol mapping}, $\vExpands{\tsmv}{a}$, for some \emph{symbol}, $a$. We write $\ctxUpdate{\uA}{\tsmv}{a}$ for the ueTSM naming context that maps $\tsmv$ to $\vExpands{\tsmv}{a}$, and defers to $\uA$ for all other TSM names (i.e. the previous mapping, if it exists, is updated).

A \emph{ueTSM definition context}, $\Psi$, is a finite function mapping each symbol $a \in \domof{\Psi}$ to an \emph{expanded ueTSM definition}, $\xuetsmbnd{a}{\tau}{\eparse}$, where $\tau$ is the ueTSM's type annotation, and $\eparse$ is its parse function. We write $\Psi, \xuetsmbnd{a}{\tau}{\eparse}$ when $a \notin \domof{\Psi}$ for the extension of $\Psi$ that maps $a$ to $\xuetsmbnd{a}{\tau}{\eparse}$. We write $\uetsmenv{\Delta}{\Psi}$  when all the type annotations in $\Psi$ are well-formed assuming $\Delta$, and the parse functions in $\Psi$ are closed and of type $\parr{\tBody}{\tParseResultExp}$.

\begin{definition}[ueTSM Definition Context Formation]\label{def:ueTSM-def-ctx-formation} $\uetsmenv{\Delta}{\Psi}$ iff for each $\xuetsmbnd{\tsmv}{\tau}{\eparse} \in \Psi$, we have $\istypeU{\Delta}{\tau}$ and $\hastypeU{\emptyset}{\emptyset}{\eparse}{\parr{\tBody}{\tParseResultExp}}$.\end{definition}

We define $\uPsi, \uShyp{\tsmv}{a}{\tau}{\eparse}$, when $\uPsi=\uAS{\uA}{\Psi}$, as an abbreviation of \[\uAS{\ctxUpdate{\uA}{\tsmv}{a}}{\Psi, \xuetsmbnd{a}{\tau}{\eparse}}\]

\end{enumerate}

% \[\begin{array}{ll}
% \textbf{Judgement Form} & \textbf{Description}\\
% \uetsmenv{\Delta}{\uPsi} & \text{$\uPsi$ is well-formed assuming $\Delta$}\end{array}\]
% This judgement is inductively defined by the following rules:
% \begin{subequations}[intermezzo]\label{rules:uetsmenv-U}
% \begin{equation}\label{rule:uetsmenv-empty}
% \inferrule{ }{\uetsmenv{\Delta}{\emptyset}}
% \end{equation}
% \begin{equation}\label{rule:uetsmenv-ext}
% \inferrule{
%   \uetsmenv{\Delta}{\uPsi}\\
%   \istypeU{\Delta}{\tau}\\
%   \hastypeU{\emptyset}{\emptyset}{\eparse}{\aparr{\tBody}{\tParseResultExp}}
% }{
%   \uetsmenv{\Delta}{\uPsi, \xuetsmbnd{\tsmv}{\tau}{\eparse}}
% }
% \end{equation}
% \end{subequations}

\subsection{ueTSM Application}\label{sec:U-uetsm-application}
The stylized unexpanded expression form for applying a ueTSM named $\tsmv$ to a literal form with literal body $b$ is:
\[
\utsmap{\tsmv}{b}
\] 
This stylized form uses forward slashes to delimit the literal body, but stylized variants of any of the literal forms specified in Figure \ref{fig:literal-forms} could also be added to Figure \ref{fig:U-unexpanded-terms}. % (we omit them for simplicity).
The corresponding operational form is $\autsmap{b}{\tsmv}$. %i.e. for each literal body $b$, the operator $\texttt{uapuetsm}[b]$ is indexed by the TSM name $\tsmv$ and takes no arguments. %\footnote{This is in following the conventions in \emph{PFPL} \cite{pfpl}, where operators parameters allow for the use of metatheoretic objects that are not syntax trees or binding trees, e.g. $\mathsf{str}[s]$ and $\mathsf{num}[n]$.} This operator is indexed by the TSM name $\tsmv$ and takes no arguments. 

The typed expansion rule governing ueTSM application is below:
\begin{subequations}[resume]
% \begin{equation}\label{rule:expandsU-tsmap}
% \inferrule{
%   \encodeBody{b}{\ebody}\\
%   \evalU{\ap{\eparse}{\ebody}}{\inj{\lbltxt{Success}}{\ecand}}\\
%   \decodeCondE{\ecand}{\ce}\\\\
%   \cvalidE{\emptyset}{\emptyset}{\esceneU{\Delta}{\Gamma}{\uPsi, \xuetsmbnd{\tsmv}{\tau}{\eparse}}{b}}{\ce}{e}{\tau}
% }{
%   \expandsU{\Delta}{\Gamma}{\uPsi, \xuetsmbnd{\tsmv}{\tau}{\eparse}}{\autsmap{b}{\tsmv}}{e}{\tau}
% }
% \end{equation}
\begin{equation}\label{rule:expandsU-tsmap}
\inferrule{
  \encodeBody{b}{\ebody}\\
  \evalU{\ap{\eparse}{\ebody}}{\inj{\lbltxt{Success}}{\ecand}}\\
  \decodeCondE{\ecand}{\ce}\\\\
  \cvalidE{\emptyset}{\emptyset}{\esceneU{\uDelta}{\uGamma}{\uPsi, \uShyp{\tsmv}{a}{\tau}{\eparse}}{b}}{\ce}{e}{\tau}
}{
  \expandsU{\uDelta}{\uGamma}{\uPsi, \uShyp{\tsmv}{a}{\tau}{\eparse}}{\utsmap{\tsmv}{b}}{e}{\tau}
}
\end{equation}
\end{subequations}
The premises of Rule (\ref{rule:expandsU-tsmap}) can be understood as follows, in order:
\begin{enumerate}
\item The first premise determines the encoding of the literal body, $\ebody$ (see above).
\item The second premise applies the parse function $\eparse$, which appears in the ueTSM context associated with $\tsmv$, to $\ebody$. If parsing succeeds, i.e. a value of the (stylized) form $\inj{\lbltxt{Success}}{\ecand}$ results from evaluation, then $\ecand$ will be a value of type $\tCEExp$ (assuming a well-formed ueTSM context, by application of Assumption \ref{condition:preservation-U}). We call $\ecand$ the \emph{encoding of the candidate expansion}.

If the parse function produces a value labeled $\lbltxt{ParseError}$, then typed expansion fails. No rule is necessary to handle this case. 

\item The third premise decodes the encoding of the candidate expansion to produce the \emph{candidate expansion}, $\ce$ (see above).




\item The final premise of Rule (\ref{rule:expandsU-tsmap}) \emph{validates} the candidate expansion and simultaneously generates the \emph{final expansion}, $e$. This is the topic of Sec. \ref{sec:ce-validation-U}.
\end{enumerate}
\subsection{Syntax of Candidate Expansions}\label{sec:ce-syntax-U}

\begin{figure}
\hspace{-5px}$\arraycolsep=3.5pt\begin{array}{lllllll}
\textbf{Sort} & & & \textbf{Operational Form} & \textbf{Stylized Form} & \textbf{Description}\\
\mathsf{CETyp} & \ctau & ::= & t & t & \text{variable}\\
&&& \aceparr{\ctau}{\ctau} & \parr{\ctau}{\ctau} & \text{partial function}\\
&&& \aceall{t}{\ctau} & \forallt{t}{\ctau} & \text{polymorphic}\\
&&& \acerec{t}{\ctau} & \rect{t}{\ctau} & \text{recursive}\\
&&& \aceprod{\labelset}{\mapschema{\ctau}{i}{\labelset}} & \prodt{\mapschema{\ctau}{i}{\labelset}} & \text{labeled product}\\
&&& \acesum{\labelset}{\mapschema{\ctau}{i}{\labelset}} & \sumt{\mapschema{\ctau}{i}{\labelset}} & \text{labeled sum}\\
\LCC &&& \lightgray & \lightgray & \lightgray\\
&&& \acesplicedt{m}{n} & \splicedt{m}{n} & \text{spliced}\\\ECC
\mathsf{CEExp} & \ce & ::= & x & x & \text{variable}\\
&&& \acelam{\ctau}{x}{\ce} & \lam{x}{\ctau}{\ce} & \text{abstraction}\\
&&& \aceap{\ce}{\ce} & \ap{\ce}{\ce} & \text{application}\\
&&& \acetlam{t}{\ce} & \Lam{t}{\ce} & \text{type abstraction}\\
&&& \acetap{\ce}{\ctau} & \App{\ce}{\ctau} & \text{type application}\\
&&& \acefold{t}{\ctau}{\ce} & \fold{\ce} & \text{fold}\\
&&& \aceunfold{\ce} & \unfold{\ce} & \text{unfold}\\
&&& \acetpl{\labelset}{\mapschema{\ce}{i}{\labelset}} & \tpl{\mapschema{\ce}{i}{\labelset}} & \text{labeled tuple}\\
&&& \acepr{\ell}{\ce} & \prj{\ce}{\ell} & \text{projection}\\
&&& \acein{\labelset}{\ell}{\mapschema{\ctau}{i}{\labelset}}{\ce} & \inj{\ell}{\ce} & \text{injection}\\
&&& \acecase{\labelset}{\tau}{\ce}{\mapschemab{x}{\ce}{i}{\labelset}} & \caseof{\ce}{\mapschemab{x}{\ce}{i}{\labelset}} & \text{case analysis}\\
\LCC &&& \lightgray & \lightgray & \lightgray\\
&&& \acesplicede{m}{n} & \splicede{m}{n} & \text{spliced}\ECC
\end{array}$
\caption[Syntax of candidate expansion types and expressions in $\miniVerseUE$]{Abstract syntax of candidate expansion types and expressions in $\miniVerseUE$. Metavariables $m$ and $n$ range over natural numbers. Candidate expansion types and expressions are identified up to $\alpha$-equivalence.}
\label{fig:U-candidate-terms}
\end{figure}

Figure \ref{fig:U-candidate-terms} defines the syntax of candidate expansion types (or \emph{ce-types}), $\ctau$, and candidate expansion expressions  (or \emph{ce-expressions}), $\ce$. Candidate expansion types and expressions are identified up to $\alpha$-equivalence in the usual manner.

Each inner core form maps onto a candidate expansion form. We refer to these as the \emph{shared forms}. In particular:

\begin{itemize}
  \item Each type form maps onto a ce-type form according to the metafunction $\Cof{\tau}$, defined as follows:
  \begin{align*}
  \Cof{t} & = t\\
  \Cof{\aparr{\tau_1}{\tau_2}} & = \aceparr{\Cof{\tau_1}}{\Cof{\tau_2}}\\
  \Cof{\aall{t}{\tau}} & = \aceall{t}{\Cof{\tau}}\\
  \Cof{\arec{t}{\tau}} & = \acerec{t}{\Cof{\tau}}\\
  \Cof{\aprod{\labelset}{\mapschema{\tau}{i}{\labelset}}} & = \aceprod{\labelset}{\mapschemax{\Cofv}{\ctau}{i}{\labelset}}\\
  \Cof{\asum{\labelset}{\mapschema{\tau}{i}{\labelset}}} & = \acesum{\labelset}{\mapschemax{\Cofv}{\ctau}{i}{\labelset}}
  \end{align*}
  \item Each expanded expression form maps onto a ce-expression form according to the metafunction $\Cof{e}$, defined as follows:
  \begin{align*}
  \Cof{x} & = x\\
  \Cof{\aelam{\tau}{x}{e}} & = \acelam{\Cof{\tau}}{x}{\Cof{e}}\\
  \Cof{\aeap{e_1}{e_2}} & = \aceap{\Cof{e_1}}{\Cof{e_2}}\\
  \Cof{\aetlam{t}{e}} & = \acetlam{t}{\Cof{e}}\\
  \Cof{\aetap{e}{\tau}} & = \acetap{\Cof{e}}{\Cof{\tau}}\\
  \Cof{\aefold{t}{\tau}{e}} & = \acefold{t}{\Cof\tau}{\Cof e}\\
  \Cof{\aeunfold{e}} & = \aceunfold{\Cof{e}}\\
  \Cof{\aetpl{\labelset}{\mapschema{e}{i}{\labelset}}} & = \acetpl{\labelset}{\mapschemax{\Cofv}{e}{i}{\labelset}}\\
  \Cof{\aein{\labelset}{\ell}{\mapschema{\tau}{i}{\labelset}}{e}} &= \acein{\labelset}{\ell}{\mapschemax{\Cofv}{\tau}{i}{\labelset}}{\Cof{e}}\\
  \Cof{\aecase{\labelset}{\tau}{e}{\mapschemab{x}{e}{i}{\labelset}}} & = \acecase{\labelset}{\Cof\tau}{\Cof{e}}{\mapschemacx{\Cofv}{x}{e}{i}{\labelset}}
\end{align*}

\end{itemize}

There are two other candidate expansion forms, highlighted in gray in Figure \ref{fig:U-candidate-terms}: a ce-type form for \emph{references to spliced unexpanded types}, $\acesplicedt{m}{n}$, and a ce-expression form for \emph{references to spliced unexpanded expressions}, $\acesplicede{m}{n}$. %TSM utilize these to splice types and unexpanded expressions out of literal bodies.

\subsection{Candidate Expansion Validation}\label{sec:ce-validation-U}



The \emph{candidate expansion validation judgements} validate ce-types and ce-expressions and simultaneously generate their final expansions.% are types and expanded expressions, respectively.
\[\begin{array}{ll}
\textbf{Judgement Form} & \textbf{Description}\\
\cvalidT{\Delta}{\tscenev}{\ctau}{\tau} & \text{Candidate expansion type $\ctau$ is well-formed and has expansion $\tau$}\\
& \text{assuming $\Delta$ and type splicing scene $\tscenev$.}\\
\cvalidE{\Delta}{\Gamma}{\escenev}{\ce}{e}{\tau} & \text{Candidate expansion expression $\ce$ has expansion $e$ and type $\tau$}\\
& \text{assuming $\Delta$ and $\Gamma$ and expression splicing scene $\escenev$.}
\end{array}\]
\emph{Expression splicing scenes}, $\escenev$, are of the form $\esceneU{\uDelta}{\uGamma}{\uPsi}{b}$, and \emph{type splicing scenes}, $\tscenev$, are of the form $\tsceneU{\Delta}{b}$. We write $\tsfrom{\escenev}$ for the type splicing scene constructed by dropping the unexpanded typing context and ueTSM context from $\escenev$:
\[\tsfrom{\esceneU{\uDelta}{\uGamma}{\uPsi}{b}} = \tsceneU{\uDelta}{b}\]

The purpose of splicing scenes is to ``remember'', during the candidate expansion validation process, the unexpanded type formation context, $\uDelta$, unexpanded typing context, $\uGamma$, ueTSM context, $\uPsi$, and the literal body, $b$, from the ueTSM application site (cf. Rule (\ref{rule:expandsU-tsmap})), because these are necessary to validate references to spliced unexpanded types and expressions that appear within a candidate expansion.

\subsubsection{Candidate Expansion Type Validation}
The \emph{candidate expansion type validation judgement}, $\cvalidT{\Delta}{\tscenev}{\ctau}{\tau}$, is inductively defined by Rules (\ref*{rules:cvalidT-U}) as follows.

\paragraph{Shared Forms} Rules (\ref*{rule:cvalidT-U-tvar}) through (\ref*{rule:cvalidT-U-sum}), which validate ce-types of shared form, are defined below.
%Each of these rules is defined based on the corresponding type formation rule, i.e. Rules (\ref{rule:istypeU-var}) through (\ref{rule:istypeU-sum}), respectively. For example, the following candidate expansion type validation rules are based on type formation rules (\ref{rule:istypeU-var}), (\ref{rule:istypeU-parr}) and (\ref{rule:istypeU-all}), respectively: 
\begin{subequations}\label{rules:cvalidT-U}
\begin{equation}\label{rule:cvalidT-U-tvar}
\inferrule{ }{
  \cvalidT{\Delta, \Dhyp{t}}{\tscenev}{t}{t}
}
\end{equation}
\begin{equation}\label{rule:cvalidT-U-parr}
  \inferrule{
    \cvalidT{\Delta}{\tscenev}{\ctau_1}{\tau_1}\\
    \cvalidT{\Delta}{\tscenev}{\ctau_2}{\tau_2}
  }{
    \cvalidT{\Delta}{\tscenev}{\aceparr{\ctau_1}{\ctau_2}}{\aparr{\tau_1}{\tau_2}}
  }
\end{equation}
\begin{equation}\label{rule:cvalidT-U-all}
  \inferrule {
    \cvalidT{\Delta, \Dhyp{t}}{\tscenev}{\ctau}{\tau}
  }{
    \cvalidT{\Delta}{\tscenev}{\aceall{t}{\ctau}}{\aall{t}{\tau}}
  }
\end{equation}
\begin{equation}\label{rule:cvalidT-U-rec}
  \inferrule{
    \cvalidT{\Delta, \Dhyp{t}}{\tscenev}{\ctau}{\tau}
  }{
    \cvalidT{\Delta}{\tscenev}{\acerec{t}{\ctau}}{\arec{t}{\tau}}
  }
\end{equation}
\begin{equation}\label{rule:cvalidT-U-prod}
  \inferrule{
    \{\cvalidT{\Delta}{\tscenev}{\ctau_i}{\tau_i}\}_{i \in \labelset}
  }{
    \cvalidT{\Delta}{\tscenev}{\aceprod{\labelset}{\mapschema{\ctau}{i}{\labelset}}}{\aprod{\labelset}{\mapschema{\tau}{i}{\labelset}}}
  }
\end{equation}
\begin{equation}\label{rule:cvalidT-U-sum}
  \inferrule{
    \{\cvalidT{\Delta}{\tscenev}{\ctau_i}{\tau_i}\}_{i \in \labelset}
  }{
    \cvalidT{\Delta}{\tscenev}{\acesum{\labelset}{\mapschema{\ctau}{i}{\labelset}}}{\asum{\labelset}{\mapschema{\tau}{i}{\labelset}}}
  }
\end{equation}

Observe that, in each of these rules, the ce-type form and the type form in the conclusion correspond, and the premises correspond to those of the corresponding type formation rule, i.e. Rules (\ref{rules:istypeU}). The type splicing scene, $\tscenev$, passes opaquely through these rules. 
The following lemma establishes that each type can be expressed as a well-formed ce-type, under the same type formation context and any type splicing scene.
\begin{lemma}[Candidate Expansion Type Expressibility]\label{lemma:ce-type-expressibility-U} If $\istypeU{\Delta}{\tau}$ then $\cvalidT{\Delta}{\tscenev}{\Cof{\tau}}{\tau}$. \end{lemma}
\begin{proof}
By rule induction over Rules (\ref{rules:istypeU}). In each case, we apply the IH on or over each premise, then apply the corresponding ce-type validation rule in Rules (\ref{rules:cvalidT-U}).
\end{proof}
% We can express this scheme more precisely with the following rule transformation. For each rule in Rules (\ref{rules:istypeU}), 
% \begin{mathpar}
% % \refstepcounter{equation}
% % \label{rule:cvalidT-U-rec}
% % \refstepcounter{equation}
% % \label{rule:cvalidT-U-prod}
% % \refstepcounter{equation}
% % \label{rule:cvalidT-U-sum}
% % \inferrule{J_1\\\cdots\\J_k}{J}
% \end{mathpar}
% the corresponding candidate expansion type validation rule is
% \begin{mathpar}
% \inferrule{
%   \VTypof{J_1}\\
%   \cdots\\
%   \VTypof{J_k}
% }{
%   \VTypof{J}
% }
% \end{mathpar}
% where 
% \[\begin{split}
% \VTypof{\istypeU{\Delta}{\tau}} & = \cvalidT{\Delta}{\tscenev}{\VTypof{\tau}}{\tau}\\
% \VTypof{\{J_i\}_{i \in \labelset}} & = \{\VTypof{J_i}\}_{i \in \labelset}
% \end{split}\]
% and where $\VTypof{\tau}$, when $\tau$ is a metapattern of sort $\mathsf{Typ}$, is a metapattern of sort $\mathsf{CETyp}$ defined as follows:
% \begin{itemize}
% \item When $\tau$ is of definite form, $\VTypof{\tau}$ is defined as follows:
% \begin{align*}
% \VTypof{t} & = t\\
% \VTypof{\aparr{\tau_1}{\tau_2}} & = \aceparr{\VTypof{\tau_1}}{\VTypof{\tau_2}}\\
% \VTypof{\aall{t}{\tau}} & = \aceall{t}{\VTypof{\tau}}\\
% \VTypof{\arec{t}{\tau}} & = \acerec{t}{\VTypof{\tau}}\\
% \VTypof{\aprod{\labelset}{\mapschema{\tau}{i}{\labelset}}} & = \aceprod{\labelset}{\mapschemax{\VTypofv}{\tau}{i}{\labelset}}\\
% \VTypof{\asum{\labelset}{\mapschema{\tau}{i}{\labelset}}} & = \acesum{\labelset}{\mapschemax{\VTypofv}{\tau}{i}{\labelset}}
% \end{align*}
% \item When $\tau$ is of indefinite form, $\VTypof{\tau}$ is a uniquely corresponding metapattern also of indefinite form. For example, $\VTypof{\tau_1}=\ctau_1$ and $\VTypof{\tau_2}=\ctau_2$.
% \end{itemize}

% It is instructive to use this rule transformation to generate Rules (\ref{rule:cvalidT-U-tvar}) through (\ref{rule:cvalidT-U-all}) above. We omit the remaining rules, i.e. Rules (\ref*{rule:cvalidT-U-rec}) through (\ref*{rule:cvalidT-U-sum}). 

Notice that in Rule (\ref{rule:cvalidT-U-tvar}), only type variables tracked by the candidate expansion type formation context, $\Delta$, are validated. Type variables in the application site unexpanded type formation context, which appears within the type splicing scene, $\tscenev$, are not validated. Indeed, $\tscenev$ is not inspected by any of the rules above. This achieves \emph{context-independent expansion} as described in Sec. \ref{sec:splicing-and-hygiene} for type variables -- ueTSMs cannot impose ``hidden constraints'' on the application site unexpanded type formation context, because the type variables bound at the application site are simply not directly available to ce-types.

\paragraph{References to Spliced Types} The only ce-type form that does not correspond to a type form is $\acesplicedt{m}{n}$, which is a \emph{reference to a spliced unexpanded type}, i.e. it indicates that an unexpanded type should be parsed out from the literal body, $b$, which appears in the type splicing scene, beginning at position $m$ and ending at position $n$. 

Rule (\ref*{rule:cvalidT-U-splicedt}) governs this form:
\begin{equation}\label{rule:cvalidT-U-splicedt}
  \inferrule{
    \parseUTyp{\bsubseq{b}{m}{n}}{\utau}\\
    \expandsTU{\uDD{\uD}{\Delta_\text{app}}}{\utau}{\tau}\\
    \Delta \cap \Delta_\text{app} = \emptyset
  }{
    \cvalidT{\Delta}{\tsceneU{\uDD{\uD}{\Delta_\text{app}}}{b}}{\acesplicedt{m}{n}}{\tau}
  }
\end{equation}
The first premise of this rule extracts the indicated subsequence of $b$ using the partial metafunction $\bsubseq{b}{m}{n}$ and parses it using the partial metafunction $\mathsf{parseUTyp}(b)$, described in Sec. \ref{sec:syntax-U}, to produce the spliced unexpanded type itself, $\utau$.

The second premise of Rule (\ref{rule:cvalidT-U-splicedt}) performs type expansion of $\utau$ under the application site unexpanded type formation context, $\uDD{\uD}{\Delta_\text{app}}$, which appears in the type splicing scene. The hypotheses in the candidate expansion type formation context, $\Delta$, are not made available to $\tau$. %This enforces \emph{expansion independent splicing} as described in Sec. \ref{sec:splicing-and-hygiene} for type variables that appear in candidate expansion types. 

The third premise of Rule (\ref{rule:cvalidT-U-splicedt}) imposes the constraint that the candidate expansion's type formation context, $\Delta$, be disjoint from the application site type formation context, $\Delta_\text{app}$. This premise can always be discharged by $\alpha$-varying the candidate expansion that the reference to the spliced type appears within. 

This achieves \emph{expansion-independent splicing} as described in Sec. \ref{sec:splicing-and-hygiene} for type variables -- the TSM provider can choose type variable names freely within a candidate expansion, because the language prevents them from shadowing type variables at the application site (by $\alpha$-varying the candidate expansion as needed).%Such a change in bound variable names is possible again because variables bound by the ueTSM provider in a candidate expansion cannot ``leak into'' spliced terms because the hypotheses in $\Delta$ are not made available to the spliced type, $\tau$. 

Rules (\ref{rules:cvalidT-U}) validate the following lemma, which establishes that the final expansion of a valid ce-type is a well-formed type under the combined type formation context.
\begin{lemma}[Candidate Expansion Type Validation]\label{lemma:candidate-expansion-type-validation}
If $\cvalidT{\Delta}{\tsceneU{\uDD{\uD}{\Delta_\text{app}}}{b}}{\ctau}{\tau}$ then $\istypeU{\Dcons{\Delta}{\Delta_\text{app}}}{\tau}$.
\end{lemma}
\begin{proof} By rule induction over Rules (\ref{rules:cvalidT-U}).
\begin{byCases}
\item[\text{(\ref{rule:cvalidT-U-tvar})}] We have 
\begin{pfsteps*}
   \item $\Delta=\Delta', \Dhyp{t}$ \BY{assumption}
   \item $\ctau=t$ \BY{assumption}
   \item $\tau=t$ \BY{assumption}
   \item $\istypeU{\Delta', \Dhyp{t}}{t}$ \BY{Rule (\ref{rule:istypeU-var})} \pflabel{istype}
   \item $\istypeU{\Dcons{\Delta', \Dhyp{t}}{\Delta_\text{app}}}{t}$ \BY{Lemma \ref{lemma:weakening-U} over $\Delta_\text{app}$ to \pfref{istype}}
 \end{pfsteps*} 
\resetpfcounter

\item[\text{(\ref{rule:cvalidT-U-parr})}] We have
\begin{pfsteps*}
  \item $\ctau=\aceparr{\ctau_1}{\ctau_2}$ \BY{assumption}
  \item $\tau=\aparr{\tau_1}{\tau_2}$ \BY{assumption}
  \item $\cvalidT{\Delta}{\tsceneU{\uDD{\uD}{\Delta_\text{app}}}{b}}{\ctau_1}{\tau_1}$ \BY{assumption} \pflabel{cvalid-ctau1}
  \item $\cvalidT{\Delta}{\tsceneU{\uDD{\uD}{\Delta_\text{app}}}{b}}{\ctau_2}{\tau_2}$ \BY{assumption} \pflabel{cvalid-ctau2}
  \item $\istypeU{\Dcons{\Delta}{\Delta_\text{app}}}{\tau_1}$ \BY{IH on \pfref{cvalid-ctau1}} \pflabel{istype1}
  \item $\istypeU{\Dcons{\Delta}{\Delta_\text{app}}}{\tau_2}$ \BY{IH on \pfref{cvalid-ctau2}} \pflabel{istype2}
  \item $\istypeU{\Dcons{\Delta}{\Delta_\text{app}}}{\aparr{\tau_1}{\tau_2}}$ \BY{Rule (\ref{rule:istypeU-parr}) on \pfref{istype1} and \pfref{istype2}}
\end{pfsteps*}
\resetpfcounter

\item[\text{(\ref{rule:cvalidT-U-all})}] We have
\begin{pfsteps*}
  \item $\ctau=\aceall{t}{\ctau'}$ \BY{assumption}
  \item $\tau=\aall{t}{\tau'}$ \BY{assumption}
  \item $\cvalidT{\Delta, \Dhyp{t}}{\tsceneU{\uDD{\uD}{\Delta_\text{app}}}{b}}{\ctau'}{\tau'}$ \BY{assumption} \label{cvalidT}
  \item $\istypeU{\Dcons{\Delta, \Dhyp{t}}{\Delta_\text{app}}}{\tau'}$ \BY{IH on \pfref{cvalidT}} \pflabel{istypeU1}
  \item $\istypeU{\Dcons{\Delta}{\Delta_\text{app}}, \Dhyp{t}}{\tau'}$ \BY{exchange over $\Delta_\text{app}$ on \pfref{istypeU1}} \pflabel{istypeU2}
  \item $\istypeU{\Dcons{\Delta}{\Delta_\text{app}}}{\aall{t}{\tau'}}$ \BY{Rule (\ref{rule:istypeU-all}) on \pfref{istypeU2}}
\end{pfsteps*}
\resetpfcounter

\item[{\text{(\ref{rule:cvalidT-U-rec})}}~\text{through}~{\text{(\ref{rule:cvalidT-U-sum})}}] These cases follow analagously, i.e. we apply the IH to or over all ce-type validation premises, apply exchange as needed, and then apply the corresponding type formation rule.
\\
% \item[\text{(\ref{rule:cvalidT-U-rec})}] We have
% \begin{pfsteps*}
%   \item $\ctau=\acerec{t}{\ctau'}$ \BY{assumption}
%   \item $\tau=\arec{t}{\tau'}$ \BY{assumption}
%   \item $\cvalidT{\Delta, \Dhyp{t}}{\tsceneU{\Delta_\text{app}}{b}}{\ctau'}{\tau'}$ \BY{assumption} \label{cvalidT}
%   \item $\istypeU{\Dcons{\Delta, \Dhyp{t}}{\Delta_\text{app}}}{\tau'}$ \BY{IH on \pfref{cvalidT}} \pflabel{istypeU1}
%   \item $\istypeU{\Dcons{\Delta}{\Delta_\text{app}}, \Dhyp{t}}{\tau'}$ \BY{exchange over $\Delta_\text{app}$ on \pfref{istypeU1}} \pflabel{istypeU2}
%   \item $\istypeU{\Dcons{\Delta}{\Delta_\text{app}}}{\arec{t}{\tau'}}$ \BY{Rule (\ref{rule:istypeU-rec}) on \pfref{istypeU2}}
% \end{pfsteps*}
% \resetpfcounter

% \item[\text{(\ref{rule:cvalidT-U-prod})}] We have
% \begin{pfsteps*}
% \item $\ctau=\aceprod{\labelset}{\mapschema{\ctau}{i}{\labelset}}$ \BY{assumption}  
% \item $\tau=\aprod{\labelset}{\mapschema{\tau}{i}{\labelset}}$ \BY{assumption}
% \item $\{\cvalidT{\Delta}{\tsceneU{\Delta_\text{app}}{b}}{\ctau_i}{\tau_i}\}_{i \in \labelset}$ \BY{assumption} \pflabel{cvalidT-ass}
% \item $\{\istypeU{\Dcons{\Delta}{\Delta_\text{app}}}{\tau_i}\}_{i \in \labelset}$ \BY{IH on \pfref{cvalidT-ass}$_i$ for each $i \in \labelset$} \pflabel{istype}
% \item $\istypeU{\Dcons{\Delta}{\Delta_\text{app}}}{\aprod{\labelset}{\mapschema{\tau}{i}{\labelset}}}$ \BY{Rule (\ref{rule:istypeU-prod}) on \pfref{istype}}
% \end{pfsteps*}
% \resetpfcounter 

% \item[\text{(\ref{rule:cvalidT-U-sum})}] We have
% \begin{pfsteps*}
% \item $\ctau=\acesum{\labelset}{\mapschema{\ctau}{i}{\labelset}}$ \BY{assumption}  
% \item $\tau=\asum{\labelset}{\mapschema{\tau}{i}{\labelset}}$ \BY{assumption}
% \item $\{\cvalidT{\Delta}{\tsceneU{\Delta_\text{app}}{b}}{\ctau_i}{\tau_i}\}_{i \in \labelset}$ \BY{assumption} \pflabel{cvalidT-ass}
% \item $\{\istypeU{\Dcons{\Delta}{\Delta_\text{app}}}{\tau_i}\}_{i \in \labelset}$ \BY{IH on \pfref{cvalidT-ass}$_i$ for each $i \in \labelset$} \pflabel{istype}
% \item $\istypeU{\Dcons{\Delta}{\Delta_\text{app}}}{\asum{\labelset}{\mapschema{\tau}{i}{\labelset}}}$ \BY{Rule (\ref{rule:istypeU-sum}) on \pfref{istype}}
% \end{pfsteps*}
% \resetpfcounter

\item[\text{(\ref{rule:cvalidT-U-splicedt})}] We have
\begin{pfsteps*}
\item $\ctau=\acesplicedt{m}{n}$ \BY{assumption}
\item $\parseUTyp{\bsubseq{b}{m}{n}}{\utau}$ \BY{assumption}
\item $\expandsTU{\uDD{\uD}{\Delta_\text{app}}}{\utau}{\tau}$ \BY{assumption} \label{expandsTU}
\item $\istypeU{\Delta_\text{app}}{\tau}$ \BY{Lemma \ref{lemma:type-expansion-U} on \pfref{expandsTU}}\pflabel{istype}
\item $\istypeU{\Dcons{\Delta}{\Delta_\text{app}}}{\tau}$ \BY{Lemma \ref{lemma:weakening-U} over $\Delta$ on \pfref{istype} and exchange over $\Delta$}
\end{pfsteps*}
\resetpfcounter
\end{byCases}
\end{proof}
\end{subequations}
% \begin{subequations}\label{rules:cvalidT-U}
% \begin{equation}\label{rule:cvalidT-U-tvar}
% \inferrule{ }{
%   \cvalidT{\Delta, \Dhyp{t}}{\tscenev}{t}{t}
% }
% \end{equation}
% \begin{equation}\label{rule:cvalidT-U-parr}
%   \inferrule{
%     \cvalidT{\Delta}{\tscenev}{\ctau_1}{\tau_1}\\
%     \cvalidT{\Delta}{\tscenev}{\ctau_2}{\tau_2}
%   }{
%     \cvalidT{\Delta}{\tscenev}{\aceparr{\ctau_1}{\ctau_2}}{\aparr{\tau_1}{\tau_2}}
%   }
% \end{equation}
% \begin{equation}\label{rule:cvalidT-U-all}
%   \inferrule {
%     \cvalidT{\Delta, \Dhyp{t}}{\tscenev}{\ctau}{\tau}
%   }{
%     \cvalidT{\Delta}{\tscenev}{\aceall{t}{\ctau}}{\aall{t}{\tau}}
%   }
% \end{equation}
% \begin{equation}\label{rule:cvalidT-U-rec}
%   \inferrule{
%     \cvalidT{\Delta, \Dhyp{t}}{\tscenev}{\ctau}{\tau}
%   }{
%     \cvalidT{\Delta}{\tscenev}{\acerec{t}{\ctau}}{\arec{t}{\tau}}
%   }
% \end{equation}
% \begin{equation}\label{rule:cvalidT-U-prod}
%   \inferrule{
%     \{\cvalidT{\Delta}{\tscenev}{\ctau_i}{\tau_i}\}_{i \in \labelset}
%   }{
%     \cvalidT{\Delta}{\tscenev}{\aceprod{\labelset}{\mapschema{\ctau}{i}{\labelset}}}{\aprod{\labelset}{\mapschema{\tau}{i}{\labelset}}}
%   }
% \end{equation}
% \begin{equation}\label{rule:cvalidT-U-sum}
%   \inferrule{
%     \{\cvalidT{\Delta}{\tscenev}{\ctau_i}{\tau_i}\}_{i \in \labelset}
%   }{
%     \cvalidT{\Delta}{\tscenev}{\acesum{\labelset}{\mapschema{\ctau}{i}{\labelset}}}{\asum{\labelset}{\mapschema{\tau}{i}{\labelset}}}
%   }
% \end{equation}
% \end{subequations}
%Each form of type, $\tau$, corresponds to a form of candidate expansion type, $\ctau$ (compare Figures \ref{fig:U-expanded-terms} and \ref{fig:U-candidate-terms}). For each type formation rule in Rules (\ref{rules:istypeU}), there is a corresponding candidate expansion type validation rule -- Rules (\ref{rule:cvalidT-U-tvar}) to (\ref{rule:cvalidT-U-sum}) -- where the candidate expansion type and the final expansion correspond. The premises also correspond. 



\subsubsection{Candidate Expansion Expression Validation}
The \emph{candidate expansion expression validation judgement}, $\cvalidE{\Delta}{\Gamma}{\escenev}{\ce}{e}{\tau}$, is defined mutually inductively with the typed expansion judgement by Rules (\ref*{rules:cvalidE-U}) as follows.% This is necessary because a typed expansion judgement appears as a premise in Rule (\ref{rule:cvalidE-U-splicede}) below, and a candidate expansion expression validation judgement appears as a premise in Rule (\ref{rule:expandsU-tsmap}) above.

\paragraph{Shared Forms} Rules (\ref*{rule:cvalidE-U-var}) through (\ref*{rule:cvalidE-U-case}) validate ce-expressions of shared form. The first three of these rules are defined below:
%For each expanded expression form defined in Figure \ref{fig:U-expanded-terms}, Figure \ref{fig:U-candidate-terms} defines a corresponding candidate expansion expression form. The validation rules for candidate expansion expressions of these forms are each based on the corresponding typing rule in Rules (\ref{rules:hastypeU}). For example, the validation rules for candidate expansion expressions of variable, function and function application form  are based on Rules (\ref{rule:hastypeU-var}) through (\ref{rule:hastypeU-ap}), respectively:
\begin{subequations}\label{rules:cvalidE-U}
\begin{equation}\label{rule:cvalidE-U-var}
\inferrule{ }{
  \cvalidE{\Delta}{\Gamma, \Ghyp{x}{\tau}}{\escenev}{x}{x}{\tau}
}
\end{equation}
\begin{equation}\label{rule:cvalidE-U-lam}
\inferrule{
  \cvalidT{\Delta}{\tsfrom{\escenev}}{\ctau}{\tau}\\
  \cvalidE{\Delta}{\Gamma, \Ghyp{x}{\tau}}{\escenev}{\ce}{e}{\tau'}
}{
  \cvalidE{\Delta}{\Gamma}{\escenev}{\acelam{\ctau}{x}{\ce}}{\aelam{\tau}{x}{e}}{\aparr{\tau}{\tau'}}
}
\end{equation}
\begin{equation}\label{rule:cvalidE-U-ap}
  \inferrule{
    \cvalidE{\Delta}{\Gamma}{\escenev}{\ce_1}{e_1}{\aparr{\tau}{\tau'}}\\
    \cvalidE{\Delta}{\Gamma}{\escenev}{\ce_2}{e_2}{\tau}
  }{
    \cvalidE{\Delta}{\Gamma}{\escenev}{\aceap{\ce_1}{\ce_2}}{\aeap{e_1}{e_2}}{\tau'}
  }
\end{equation}
Observe that, in each of these rules, the ce-expression form and the expanded expression form in the conclusion correspond, and the premises correspond to those of the corresponding typing rule, i.e. Rules (\ref{rule:hastypeU-var}) through (\ref{rule:hastypeU-ap}), respectively. The expression splicing scene, $\escenev$, passes opaquely through these rules.

We can express this scheme more precisely with the following rule transformation. For each rule in Rules (\ref{rules:hastypeU}),
\begin{mathpar}\refstepcounter{equation}
\label{rule:cvalidE-U-tlam}
\refstepcounter{equation}
\label{rule:cvalidE-U-tap}
\refstepcounter{equation}
\label{rule:cvalidE-U-fold}
\refstepcounter{equation}
\label{rule:cvalidE-U-unfold}
\refstepcounter{equation}
\label{rule:cvalidE-U-tpl}
\refstepcounter{equation}
\label{rule:cvalidE-U-pr}
\refstepcounter{equation}
\label{rule:cvalidE-U-in}
\refstepcounter{equation}
\label{rule:cvalidE-U-case}
  \inferrule{
    J_1\\
    \cdots\\
    J_k
  }{
    J
  }
\end{mathpar}
the corresponding candidate expansion expression validation rule is 
\begin{mathpar}
  \inferrule{
    \Cof{J_1}\\
    \cdots\\
    \Cof{J_k}
  }{
    \Cof{J}
  }
\end{mathpar}
where 
\[\begin{split}
  \Cof{\istypeU{\Delta}{\tau}} & = \cvalidT{\Delta}{\tsfrom{\escenev}}{\Cof{\tau}}{\tau}\\
  \Cof{\hastypeU{\Delta}{\Gamma}{e}{\tau}} & = \cvalidE{\Delta}{\Gamma}{\escenev}{\Cof{e}}{e}{\tau}\\
  \Cof{\{J_i\}_{i \in \labelset}} & = \{\Cof{J_i}\}_{i \in \labelset}
\end{split}\]
and where:
\begin{itemize}
\item $\Cof{\tau}$ is defined as follows:
  \begin{itemize}
  \item When $\tau$ is of definite form, $\Cof{\tau}$ is defined as in Sec. \ref{sec:ce-syntax-U}.
  \item When $\tau$ is of indefinite form, $\Cof{\tau}$ is a uniquely corresponding metavariable of sort $\mathsf{CETyp}$ also of indefinite form. For example, $\Cof{\tau_1}=\ctau_1$ and $\Cof{\tau_2}=\ctau_2$.
  \end{itemize}
\item $\Cof{e}$ is defined as follows
  \begin{itemize}
  \item When $e$ is of definite form, $\Cof{e}$ is defined as in Sec. \ref{sec:ce-syntax-U}. 
  \item When $e$ is of indefinite form, $\Cof{e}$ is a uniquely corresponding metavariable of sort $\mathsf{CEExp}$ also of indefinite form. For example, $\Cof{e_1}=\ce_1$ and $\Cof{e_2}=\ce_2$.
  \end{itemize}
\end{itemize}

It is instructive to use this rule transformation to generate Rules (\ref{rule:cvalidE-U-var}) through (\ref{rule:cvalidE-U-ap}) above. We omit the remaining rules for shared forms, i.e. Rules (\ref*{rule:cvalidE-U-tlam}) through (\ref*{rule:cvalidE-U-case}).

The following lemma establishes that each well-typed expanded expression, $e$, can be expressed as a valid ce-expression, $\Cof{e}$, that is assigned the same type under any expression splicing scene.
\begin{theorem}[Candidate Expansion Expression Expressibility]\label{theorem:ce-expressions-expressibility-U} If $\hastypeU{\Delta}{\Gamma}{e}{\tau}$ then $\cvalidE{\Delta}{\Gamma}{\escenev}{\Cof{e}}{e}{\tau}$.\end{theorem}
\begin{proof} By rule induction over Rules (\ref{rules:hastypeU}). The rule transformation above guarantees that this lemma holds by construction. In particular, in each case, we apply Lemma \ref{lemma:ce-type-expressibility-U} to or over each type formation premise, the IH to or over each typing premise, then apply the corresponding ce-expression validation rule in Rules (\ref{rule:cvalidE-U-var}) through (\ref{rule:cvalidE-U-case}).
\end{proof}

Notice that in Rule (\ref{rule:cvalidE-U-var}), only variables tracked by the candidate expansion typing context, $\Gamma$, are validated. Variables  in the application site unexpanded typing context, which appears within the expression splicing scene $\escenev$, are not validated. Indeed, $\escenev$ is not inspected by any of the rules above. This achieves \emph{context-independent expansion} as described in Sec. \ref{sec:splicing-and-hygiene} -- ueTSMs cannot impose ``hidden constraints'' on the application site unexpanded typing context, because the variable bindings at the application site are not directly available to candidate expansions.

\paragraph{References to Spliced Unexpanded Expressions} The only ce-expression form that does not correspond to an expanded expression form is $\acesplicede{m}{n}$, which is a \emph{reference to a spliced unexpanded expression}, i.e. it indicates that an unexpanded expression should be parsed out from the literal body beginning at position $m$ and ending at position $n$. Rule (\ref*{rule:cvalidE-U-splicede}) governs this form:
\begin{equation}\label{rule:cvalidE-U-splicede}
\inferrule{
  \parseUExp{\bsubseq{b}{m}{n}}{\ue}\\
  \expandsU{\uDD{\uD}{\Delta_\text{app}}}{\uGG{\uG}{\Gamma_\text{app}}}{\uPsi}{\ue}{e}{\tau}\\\\
  \Delta \cap \Delta_\text{app} = \emptyset\\
  \domof{\Gamma} \cap \domof{\Gamma_\text{app}} = \emptyset
}{
  \cvalidE{\Delta}{\Gamma}{\esceneU{\uDD{\uD}{\Delta_\text{app}}}{\uGG{\uG}{\Gamma_\text{app}}}{\uPsi}{b}}{\acesplicede{m}{n}}{e}{\tau}
}
\end{equation}
% \begin{equation}\label{rule:cvalidE-U-splicede}
% \inferrule{
%   \parseUExp{\bsubseq{b}{m}{n}}{\ue}\\\\
%   \expandsU{\Delta_\text{app}}{\Gamma_\text{app}}{\uPsi}{\ue}{e}{\tau}\\
%   \Delta \cap \Delta_\text{app} = \emptyset\\
%   \domof{\Gamma} \cap \domof{\Gamma_\text{app}} = \emptyset
% }{
%   \cvalidE{\Delta}{\Gamma}{\esceneU{\Delta_\text{app}}{\Gamma_\text{app}}{\uPsi}{b}}{\splicede{m}{n}}{e}{\tau}
% }
% \end{equation}
The first premise of this rule extracts the indicated subsequence of $b$ using the partial metafunction $\bsubseq{b}{m}{n}$ and parses it using the partial metafunction $\mathsf{parseUExp}(b)$, described in Sec. \ref{sec:syntax-U}, to produce the referenced spliced unexpanded expression, $\ue$.

The second premise of Rule (\ref{rule:cvalidE-U-splicede}) types and expands the spliced unexpanded expression $\ue$ assuming the application site contexts that appear in the expression splicing scene. The hypotheses in the candidate expansion type formation context, $\Delta$, and typing context, $\Gamma$, are not made available to $\ue$. 

The third premise of Rule (\ref{rule:cvalidE-U-splicede}) imposes the constraint that the candidate expansion's type formation context, $\Delta$, be disjoint from the application site type formation context, $\Delta_\text{app}$. Similarly, the fourth premise requires that the candidate expansion's typing context, $\Gamma$, be disjoint from the application site typing context, $\Gamma_\text{app}$. These two premises can always be discharged by $\alpha$-varying the ce-expression that the reference to the spliced unexpanded expression appears within. 

This achieves \emph{expansion-independent splicing} as described in Sec. \ref{sec:splicing-and-hygiene} -- the TSM provider can choose variable names freely within a candidate expansion, because the language prevents them from shadowing those at the application site (by $\alpha$-varying the candidate expansion as needed).
\end{subequations}
% \begin{subequations}\label{rules:cvalidE-U}
% \begin{equation}\label{rule:cvalidE-U-var}
% \inferrule{ }{
%   \cvalidE{\Delta}{\Gamma, \Ghyp{x}{\tau}}{\escenev}{x}{x}{\tau}
% }
% \end{equation}
% \begin{equation}\label{rule:cvalidE-U-lam}
% \inferrule{
%   \cvalidT{\Delta}{\tsfrom{\escenev}}{\ctau}{\tau}\\
%   \cvalidE{\Delta}{\Gamma, \Ghyp{x}{\tau}}{\escenev}{\ce}{e}{\tau'}
% }{
%   \cvalidE{\Delta}{\Gamma}{\escenev}{\acelam{\ctau}{x}{\ce}}{\aelam{\tau}{x}{e}}{\aparr{\tau}{\tau'}}
% }
% \end{equation}
% \begin{equation}\label{rule:cvalidE-U-ap}
%   \inferrule{
%     \cvalidE{\Delta}{\Gamma}{\escenev}{\ce_1}{e_1}{\aparr{\tau}{\tau'}}\\
%     \cvalidE{\Delta}{\Gamma}{\escenev}{\ce_2}{e_2}{\tau}
%   }{
%     \cvalidE{\Delta}{\Gamma}{\escenev}{\aceap{\ce_1}{\ce_2}}{\aeap{e_1}{e_2}}{\tau'}
%   }
% \end{equation}
% \begin{equation}\label{rule:cvalidE-U-tlam}
%   \inferrule{
%     \cvalidE{\Delta, \Dhyp{t}}{\Gamma}{\escenev}{\ce}{e}{\tau}
%   }{
%     \cvalidEX{\acetlam{t}{\ce}}{\aetlam{t}{e}}{\aall{t}{\tau}}
%   }
% \end{equation}
% \begin{equation}\label{rule:cvalidE-U-tap}
%   \inferrule{
%     \cvalidEX{\ce}{e}{\aall{t}{\tau}}\\
%     \cvalidT{\Delta}{\tsfrom{\escenev}}{\ctau'}{\tau'}
%   }{
%     \cvalidEX{\acetap{\ce}{\ctau'}}{\aetap{e}{\tau'}}{[\tau'/t]\tau}
%   }
% \end{equation}
% \begin{equation}\label{rule:cvalidE-U-fold}
%   \inferrule{
%     \cvalidT{\Delta, \Dhyp{t}}{\escenev}{\ctau}{\tau}\\
%     \cvalidEX{\ce}{e}{[\arec{t}{\tau}/t]\tau}
%   }{
%     \cvalidEX{\acefold{t}{\ctau}{\ce}}{\aefold{t}{\tau}{e}}{\arec{t}{\tau}}
%   }
% \end{equation}
% \begin{equation}\label{rule:cvalidE-U-unfold}
%   \inferrule{
%     \cvalidEX{\ce}{e}{\arec{t}{\tau}}
%   }{
%     \cvalidEX{\aceunfold{\ce}}{\aeunfold{e}}{[\arec{t}{\tau}/t]\tau}
%   }
% \end{equation}
% \begin{equation}\label{rule:cvalidE-U-tpl}
%   \inferrule{
%     \{\cvalidEX{\ce_i}{e_i}{\tau_i}\}_{i \in \labelset}
%   }{
%     \cvalidEX{\acetpl{\labelset}{\mapschema{\ce}{i}{\labelset}}}{\aetpl{\labelset}{\mapschema{e}{i}{\labelset}}}{\aprod{\labelset}{\mapschema{\tau}{i}{\labelset}}}
%   }
% \end{equation}
% \begin{equation}\label{rule:cvalidE-U-pr}
%   \inferrule{
%     \cvalidEX{\ce}{e}{\aprod{\labelset, \ell}{\mapschema{\tau}{i}{\labelset}; \mapitem{\ell}{\tau}}}
%   }{
%     \cvalidEX{\acepr{\ell}{\ce}}{\aepr{\ell}{e}}{\tau}
%   }
% \end{equation}
% \begin{equation}\label{rule:cvalidE-U-in}
%   \inferrule{
%     \{\cvalidT{\Delta}{\tsfrom{\escenev}}{\ctau_i}{\tau_i}\}_{i \in \labelset}\\
%     \cvalidT{\Delta}{\tsfrom{\escenev}}{\ctau}{\tau}\\
%     \cvalidEX{\ce}{e}{\tau}
%   }{
%     \left\{\shortstack{$\Delta~\Gamma \vdash_\uPsi \acein{\labelset, \ell}{\ell}{\mapschema{\ctau}{i}{\labelset}; \mapitem{\ell}{\ctau}}{\ce}$\\$\leadsto$\\$\aein{\labelset, \ell}{\ell}{\mapschema{\tau}{i}{\labelset}; \mapitem{\ell}{\tau}}{e} : \asum{\labelset, \ell}{\mapschema{\tau}{i}{\labelset}; \mapitem{\ell}{\tau}}$\vspace{-1.2em}}\right\}
%   }
% \end{equation}
% \begin{equation}\label{rule:cvalidE-U-case}
%   \inferrule{
%     \cvalidEX{\ce}{e}{\asum{\labelset}{\mapschema{\tau}{i}{\labelset}}}\\
%     \{\cvalidE{\Delta}{\Gamma, \Ghyp{x_i}{\tau_i}}{\escenev}{\ue_i}{e_i}{\tau}\}_{i \in \labelset}
%   }{
%     \cvalidEX{\acecase{\labelset}{\ce}{\mapschemab{x}{\ce}{i}{\labelset}}}{\aecase{\labelset}{e}{\mapschemab{x}{e}{i}{\labelset}}}{\tau}
%   }
% \end{equation}
% \begin{equation}\label{rule:cvalidE-U-splicede}
% \inferrule{
%   \parseUExp{\bsubseq{b}{m}{n}}{\ue}\\\\
%   \Delta \cap \Delta_\text{app} = \emptyset\\
%   \domof{\Gamma} \cap \domof{\Gamma_\text{app}} = \emptyset\\
%   \expandsU{\Delta_\text{app}}{\Gamma_\text{app}}{\uPsi}{\ue}{e}{\tau}
% }{
%   \cvalidE{\Delta}{\Gamma}{\esceneU{\Delta_\text{app}}{\Gamma_\text{app}}{\uPsi}{b}}{\acesplicede{m}{n}}{e}{\tau}
% }
% \end{equation}
% \end{subequations}

% Each form of expanded expression, $e$, corresponds to a form of candidate expansion expression, $\ce$ (compare Figure \ref{fig:U-expanded-terms} and Figure \ref{fig:U-candidate-terms}). For each typing rule in Rules \ref{rules:hastypeU}, there is a corresponding candidate expansion expression validation rule -- Rules (\ref{rule:cvalidE-U-var}) to (\ref{rule:cvalidE-U-case}) -- where the candidate expansion expression and expanded expression correspond. The premises also correspond.


%Candidate expansions cannot themselves define or apply TSMs. This simplifies our metatheory, though it can be inconvenient at times for TSM providers. We discuss adding the ability to use TSMs within candidate expansions in Sec. \ref{sec:tsms-in-expansions}.


\subsection{Metatheory}
For the judgements we have defined to form a sensible language, we must have that typed expansion and candidate expansion expression validation be consistent with typing. Formally, this can be expressed as follows.

\begin{theorem}[Typed Expansion]\label{thm:typed-expansion-U} Both of the following hold:
\begin{enumerate}
\item If $\expandsU{\uDD{\uD}{\Delta}}{\uGG{\uG}{\Gamma}}{\uAS{\uA}{\Psi}}{\ue}{e}{\tau}$ then $\hastypeU{\Delta}{\Gamma}{e}{\tau}$.
\item If $\cvalidE{\Delta}{\Gamma}{\esceneU{\uDD{\uD}{\Delta_\text{app}}}{\uGG{\uG}{\Gamma_\text{app}}}{\uAS{\uA}{\Psi}}{b}}{\ce}{e}{\tau}$ and $\Delta \cap \Delta_\text{app} = \emptyset$ and $\domof{\Gamma} \cap \domof{\Gamma_\text{app}} = \emptyset$ then $\hastypeU{\Dcons{\Delta}{\Delta_\text{app}}}{\Gcons{\Gamma}{\Gamma_\text{app}}}{e}{\tau}$.
\end{enumerate}
\end{theorem}
\begin{proof}
By mutual rule induction over Rules (\ref{rules:expandsU}) and Rules (\ref{rules:cvalidE-U}). 

The proof of part 1 proceeds by inducting over the typed expansion assumption. In the following cases, let $\uDelta=\uDD{\uD}{\Delta}$ and $\uGamma=\uGG{\uG}{\Gamma}$ and $\uPsi=\uAS{\uA}{\Psi}$.
\begin{byCases}
\item[\text{(\ref{rule:expandsU-var})}] We have:
\begin{pfsteps}
  \item \ue=\ux \BY{assumption}
  \item e=x \BY{assumption}
  \item \Gamma=\Gamma', \Ghyp{x}{\tau} \BY{assumption}
  \item \hastypeU{\Delta}{\Gamma', \Ghyp{x}{\tau}}{x}{\tau} \BY{Rule (\ref{rule:hastypeU-var})}
\end{pfsteps}
\resetpfcounter

\item[\text{(\ref{rule:expandsU-lam})}] We have:
\begin{pfsteps}
  \item \ue=\aulam{\utau_1}{\ux}{\ue'} \BY{assumption}
  \item e=\aelam{\tau_1}{x}{e'} \BY{assumption}
  \item \tau=\aparr{\tau_1}{\tau_2} \BY{assumption}
  \item \expandsTU{\uDelta}{\utau_1}{\tau_1} \BY{assumption} \pflabel{istype}
  \item \expandsU{\uDelta}{\uGamma, \uGhyp{\ux}{x}{\tau_1}}{\uPsi}{\ue'}{e'}{\tau_2} \BY{assumption} \pflabel{expandsU}
%  \item \uetsmenv{\Delta}{\Psi} \BY{assumption} \pflabel{uetsmenv}
  \item \istypeU{\Delta}{\tau_1} \BY{Lemma \ref{lemma:type-expansion-U} on \pfref{istype}} \pflabel{istype2}
  \item \hastypeU{\Delta}{\Gamma, \Ghyp{x}{\tau_1}}{e'}{\tau_2} \BY{IH, part 1 on \pfref{expandsU}} \pflabel{hastypeU}
  \item \hastypeU{\Delta}{\Gamma}{\aelam{\tau_1}{x}{e'}}{\aparr{\tau_1}{\tau_2}} \BY{Rule (\ref{rule:hastypeU-lam}) on \pfref{istype2} and \pfref{hastypeU}}
\end{pfsteps}
\resetpfcounter

\item[\text{(\ref{rule:expandsU-ap})}] We have:
\begin{pfsteps}
  \item \ue=\auap{\ue_1}{\ue_2} \BY{assumption}
  \item e=\aeap{e_1}{e_2} \BY{assumption}
  \item \expandsU{\uDelta}{\uGamma}{\uPsi}{\ue_1}{e_1}{\aparr{\tau_1}{\tau}} \BY{assumption}\pflabel{expandsU1}
  \item \expandsU{\uDelta}{\uGamma}{\uPsi}{\ue_2}{e_2}{\tau_1} \BY{assumption}\pflabel{expandsU2}
%  \item \uetsmenv{\Delta}{\Psi} \BY{assumption} \pflabel{uetsmenv}
  \item \hastypeU{\Delta}{\Gamma}{e_1}{\aparr{\tau_1}{\tau}} \BY{IH, part 1 on \pfref{expandsU1}}\pflabel{hastypeU1}
  \item \hastypeU{\Delta}{\Gamma}{e_2}{\tau_1} \BY{IH on \pfref{expandsU2}}\pflabel{hastypeU2}
  \item \hastypeU{\Delta}{\Gamma}{\aeap{e_1}{e_2}}{\tau} \BY{Rule (\ref{rule:hastypeU-ap}) on \pfref{hastypeU1} and \pfref{hastypeU2}}
\end{pfsteps}
\resetpfcounter

\item[\text{(\ref{rule:expandsU-tlam})}~\text{through}~\text{(\ref{rule:expandsU-case})}] These cases follow analagously, i.e. we apply Lemma \ref{lemma:type-expansion-U} to or over the type expansion premises and the IH, part 1, to or over the typed expression expansion premises and then apply the corresponding typing rule in Rules (\ref{rule:hastypeU-tlam}) through (\ref{rule:hastypeU-case}).
\\

\item[\text{(\ref{rule:expandsU-syntax})}] We have 
\begin{pfsteps}
  \item \ue=\audefuetsm{\utau'}{\eparse}{\tsmv}{\ue'} \BY{assumption}
  \item \expandsTU{\uDelta}{\utau'}{\tau'} \BY{assumption} \pflabel{expandsTU}
%  \item \hastypeU{\emptyset}{\emptyset}{\eparse}{\aparr{\tBody}{\tParseResultExp}} \BY{assumption}\pflabel{eparse}
  \item \expandsU{\uDelta}{\uGamma}{\uPsi, \uShyp{\tsmv}{a}{\tau'}{\eparse}}{\ue'}{e}{\tau} \BY{assumption}\pflabel{expandsU}
%  \item \uetsmenv{\Delta}{\Psi} \BY{assumption}\pflabel{uetsmenv1}
%  \item \istypeU{\Delta}{\tau'} \BY{Lemma \ref{lemma:type-expansion-U} to \pfref{expandsTU}} \pflabel{istype}
%  \item \uetsmenv{\Delta}{\Psi, \xuetsmbnd{\tsmv}{\tau'}{\eparse}} \BY{Definition \ref{def:ueTSM-def-ctx-formation} on \pfref{uetsmenv1}, \pfref{istype} and \pfref{eparse}}\pflabel{uetsmenv3}
  \item \hastypeU{\Delta}{\Gamma}{e}{\tau} \BY{IH, part 1 on \pfref{expandsU}}
\end{pfsteps}
\resetpfcounter 

\item[\text{(\ref{rule:expandsU-tsmap})}] We have 
\begin{pfsteps}
  \item \ue=\autsmap{b}{\tsmv} \BY{assumption}
  \item \uA = \uA', \vExpands{\tsmv}{a} \BY{assumption}
  \item \Psi=\Psi', \xuetsmbnd{a}{\tau}{\eparse} \BY{assumption}
  \item \encodeBody{b}{\ebody} \BY{assumption}
  \item \evalU{\eparse(\ebody)}{\inj{\lbltxt{Success}}{\ecand}} \BY{assumption}
  \item \decodeCondE{\ecand}{\ce} \BY{assumption}
  \item \cvalidE{\emptyset}{\emptyset}{\esceneU{\uDelta}{\uGamma}{\uPsi}{b}}{\ce}{e}{\tau} \BY{assumption}\pflabel{cvalidE}
%  \item \uetsmenv{\Delta}{\Psi} \BY{assumption} \pflabel{uetsmenv}
  \item \emptyset \cap \Delta = \emptyset \BY{finite set intersection identity} \pflabel{delta-cap}
  \item {\emptyset} \cap \domof{\Gamma} = \emptyset \BY{finite set intersection identity} \pflabel{gamma-cap}
  \item \hastypeU{\emptyset \cup \Delta}{\emptyset \cup \Gamma}{e}{\tau} \BY{IH, part 2 on \pfref{cvalidE}, \pfref{delta-cap}, and \pfref{gamma-cap}} \pflabel{penultimate}
  \item \hastypeU{\Delta}{\Gamma}{e}{\tau} \BY{definition of finite set and finite function union over \pfref{penultimate}}
\end{pfsteps}
\resetpfcounter
\end{byCases}

The second part of the theorem proceeds by induction over the candidate expansion expression validation assumption as follows. In the following cases, let $\uDelta_\text{app}=\uDD{\uD}{\Delta_\text{app}}$ and $\uGamma_\text{app}=\uGG{\uG}{\Gamma_\text{app}}$ and $\uPsi = \uAS{\uA}{\Psi}$.
\begin{byCases}
\item[\text{(\ref{rule:cvalidE-U-var})}] We have
\begin{pfsteps*}
  \item $\ce=x$ \BY{assumption}
  \item $e=x$ \BY{assumption}
  \item $\Gamma=\Gamma', \Ghyp{x}{\tau}$ \BY{assumption}
  \item $\hastypeU{\Dcons{\Delta}{\Delta_\text{app}}}{\Gamma', \Ghyp{x}{\tau}}{x}{\tau}$ \BY{Rule (\ref{rule:hastypeU-var})} \pflabel{hastypeU}
  \item $\hastypeU{\Dcons{\Delta}{\Delta_\text{app}}}{\Gcons{\Gamma', \Ghyp{x}{\tau}}{\Gamma_\text{app}}}{x}{\tau}$ \BY{Lemma \ref{lemma:weakening-U} over $\Gamma_\text{app}$ to \pfref{hastypeU}}
\end{pfsteps*}
\resetpfcounter

\item[\text{(\ref{rule:cvalidE-U-lam})}] We have
\begin{pfsteps*}
  \item $\ce=\acelam{\ctau_1}{x}{\ce'}$ \BY{assumption}
  \item $e=\aelam{\tau_1}{x}{e'}$ \BY{assumption}
  \item $\tau=\aparr{\tau_1}{\tau_2}$ \BY{assumption}
  \item $\cvalidT{\Delta}{\tsceneU{\uDelta_\text{app}}{b}}{\ctau_1}{\tau_1}$ \BY{assumption} \pflabel{cvalidT}
  \item $\cvalidE{\Delta}{\Gamma, \Ghyp{x}{\tau_1}}{\esceneU{\uDelta_\text{app}}{\uGamma_\text{app}}{\uPsi}{b}}{\ce'}{e'}{\tau_2}$ \BY{assumption} \pflabel{cvalidE}
%  \item $\uetsmenv{\Delta_\text{app}}{\Psi}$ \BY{assumption} \pflabel{uetsmenv}
  \item $\Delta \cap \Delta_\text{app}=\emptyset$ \BY{assumption} \pflabel{delta-disjoint}
  \item $\domof{\Gamma} \cap \domof{\Gamma_\text{app}}=\emptyset$ \BY{assumption} \pflabel{gamma-disjoint}
  \item $x \notin \domof{\Gamma_\text{app}}$ \BY{identification convention} \pflabel{x-fresh}
  \item $\domof{\Gamma, x : \tau_1} \cap \domof{\Gamma_\text{app}}=\emptyset$ \BY{\pfref{gamma-disjoint} and \pfref{x-fresh}} \pflabel{gamma-disjoint2}
  \item $\istypeU{\Dcons{\Delta}{\Delta_\text{app}}}{\tau_1}$ \BY{Lemma \ref{lemma:candidate-expansion-type-validation} on \pfref{cvalidT}} \pflabel{istype}
  \item $\hastypeU{\Dcons{\Delta}{\Delta_\text{app}}}{\Gcons{\Gamma, \Ghyp{x}{\tau_1}}{\Gamma_\text{app}}}{e'}{\tau_2}$ \BY{IH, part 2 on \pfref{cvalidE}, \pfref{delta-disjoint} and \pfref{gamma-disjoint2}} \pflabel{hastype1}
  \item $\hastypeU{\Dcons{\Delta}{\Delta_\text{app}}}{\Gcons{\Gamma}{\Gamma_\text{app}}, \Ghyp{x}{\tau_1}}{e'}{\tau_2}$ \BY{exchange over $\Gamma_\text{app}$ on \pfref{hastype1}} \pflabel{hastype2}
  \item $\hastypeU{\Dcons{\Delta}{\Delta_\text{app}}}{\Gcons{\Gamma}{\Gamma_\text{app}}}{\aelam{\tau_1}{x}{e'}}{\aparr{\tau_1}{\tau_2}}$ \BY{Rule (\ref{rule:hastypeU-lam}) on \pfref{istype} and \pfref{hastype2}}
\end{pfsteps*}
\resetpfcounter

\item[\text{(\ref{rule:cvalidE-U-ap})}] We have
\begin{pfsteps*}
  \item $\ce=\aceap{\ce_1}{\ce_2}$ \BY{assumption}
  \item $e=\aeap{e_1}{e_2}$ \BY{assumption}
  \item $\cvalidE{\Delta}{\Gamma}{\esceneU{\uDelta_\text{app}}{\uGamma_\text{app}}{\uPsi}{b}}{\ce_1}{e_1}{\aparr{\tau_2}{\tau}}$ \BY{assumption} \pflabel{cvalidE1}
  \item $\cvalidE{\Delta}{\Gamma}{\esceneU{\uDelta_\text{app}}{\uGamma_\text{app}}{\uPsi}{b}}{\ce_2}{e_2}{\tau_2}$ \BY{assumption} \pflabel{cvalidE2}
%  \item $\uetsmenv{\Delta_\text{app}}{\Psi}$ \BY{assumption} \pflabel{uetsmenv}
  \item $\Delta \cap \Delta_\text{app}=\emptyset$ \BY{assumption} \pflabel{delta-disjoint}
  \item $\domof{\Gamma} \cap \domof{\Gamma_\text{app}}=\emptyset$ \BY{assumption} \pflabel{gamma-disjoint}
  \item $\hastypeU{\Dcons{\Delta}{\Delta_\text{app}}}{\Gcons{\Gamma}{\Gamma_\text{app}}}{e_1}{\aparr{\tau_2}{\tau}}$ \BY{IH, part 2 on \pfref{cvalidE1}, \pfref{delta-disjoint} and \pfref{gamma-disjoint}} \pflabel{hastypeU1}
  \item $\hastypeU{\Dcons{\Delta}{\Delta_\text{app}}}{\Gcons{\Gamma}{\Gamma_\text{app}}}{e_2}{\tau_2}$ \BY{IH, part 2 on \pfref{cvalidE2}, \pfref{delta-disjoint} and \pfref{gamma-disjoint}} \pflabel{hastypeU2}
  \item $\hastypeU{\Dcons{\Delta}{\Delta_\text{app}}}{\Gcons{\Gamma}{\Gamma_\text{app}}}{\aeap{e_1}{e_2}}{\tau}$ \BY{Rule (\ref{rule:hastypeU-ap}) on \pfref{hastypeU1} and \pfref{hastypeU2}}
\end{pfsteps*}
\resetpfcounter

\item[\text{(\ref{rule:cvalidE-U-tlam})}] We have
\begin{pfsteps}
  \item \ce=\acetlam{t}{\ce'} \BY{assumption}
  \item e = \aetlam{t}{e'} \BY{assumption}
  \item \tau = \aall{t}{\tau'}\BY{assumption}
  \item \cvalidE{\Delta, \Dhyp{t}}{\Gamma}{\esceneU{\uDelta_\text{app}}{\uGamma_\text{app}}{\uPsi}{b}}{\ce'}{e'}{\tau'} \BY{assumption} \pflabel{cvalidE}
%  \item \uetsmenv{\Delta_\text{app}}{\Psi} \BY{assumption} \pflabel{uetsmenv}
  \item \Delta \cap \Delta_\text{app}=\emptyset \BY{assumption} \pflabel{delta-disjoint}
  \item \domof{\Gamma} \cap \domof{\Gamma_\text{app}}=\emptyset \BY{assumption} \pflabel{gamma-disjoint}
  \item \Dhyp{t} \notin \Delta_\text{app} \BY{identification convention}\pflabel{t-fresh}
  \item \Delta, \Dhyp{t} \cap \Delta_\text{app} = \emptyset \BY{\pfref{delta-disjoint} and \pfref{t-fresh}}\pflabel{delta-disjoint2}
  \item \hastypeU{\Dcons{\Delta, \Dhyp{t}}{\Delta_\text{app}}}{\Gcons{\Gamma}{\Gamma_\text{app}}}{e'}{\tau'} \BY{IH, part 2 on \pfref{cvalidE}, \pfref{delta-disjoint2} and \pfref{gamma-disjoint}}\pflabel{hastype1}
  \item \hastypeU{\Dcons{\Delta}{\Delta_\text{app}, \Dhyp{t}}}{\Gcons{\Gamma}{\Gamma_\text{app}}}{e'}{\tau'} \BY{exchange over $\Delta_\text{app}$ on \pfref{hastype1}}\pflabel{hastype2}
  \item \hastypeU{\Dcons{\Delta}{\Delta_\text{app}}}{\Gcons{\Gamma}{\Gamma_\text{app}}}{\aetlam{t}{e'}}{\aall{t}{\tau'}} \BY{Rule (\ref{rule:hastypeU-tlam}) on \pfref{hastype2}}
\end{pfsteps}
\resetpfcounter

\item[{\text{(\ref{rule:cvalidE-U-tap})}}~\text{through}~{\text{(\ref{rule:cvalidE-U-case})}}] These cases follow analagously, i.e. we apply the IH, part 2 to all ce-expression validation judgements, Lemma \ref{lemma:candidate-expansion-type-validation} to all ce-type validation judgements, the identification convention to ensure that extended contexts remain disjoint, weakening and exchange as needed, and the corresponding typing rule in Rules (\ref{rule:hastypeU-tap}) through (\ref{rule:hastypeU-case}).
\\

\item[\text{(\ref{rule:cvalidE-U-splicede})}] We have
\begin{pfsteps*}
  \item $\ce=\acesplicede{m}{n}$ \BY{assumption}
  \item $\parseUExp{\bsubseq{b}{m}{n}}{\ue}$ \BY{assumption}
  \item $\expandsU{\uDelta_\text{app}}{\uGamma_\text{app}}{\uPsi}{\ue}{e}{\tau}$ \BY{assumption} \pflabel{expands}
%  \item $\uetsmenv{\Delta_\text{app}}{\Psi}$ \BY{assumption} \pflabel{uetsmenv}
  \item $\Delta \cap \Delta_\text{app}=\emptyset$ \BY{assumption} \pflabel{delta-disjoint}
  \item $\domof{\Gamma} \cap \domof{\Gamma_\text{app}}=\emptyset$ \BY{assumption} \pflabel{gamma-disjoint}
  \item $\hastypeU{\Delta_\text{app}}{\Gamma_\text{app}}{e}{\tau}$ \BY{IH, part 1 on \pfref{expands}} \pflabel{hastype}
  \item $\hastypeU{\Dcons{\Delta}{\Delta_\text{app}}}{\Gcons{\Gamma}{\Gamma_\text{app}}}{e}{\tau}$ \BY{Lemma \ref{lemma:weakening-U} over $\Delta$ and $\Gamma$ and exchange on \pfref{hastype}}
\end{pfsteps*}
\resetpfcounter
\end{byCases}

The mutual induction can be shown to be well-founded by showing that the following numeric metric on the judgements that we induct over is decreasing:
\begin{align*}
\sizeof{\expandsU{\uDelta}{\uGamma}{\uPsi}{\ue}{e}{\tau}} & = \sizeof{\ue}\\
\sizeof{\cvalidE{\Delta}{\Gamma}{\esceneU{\uDelta_\text{app}}{\uGamma_\text{app}}{\uPsi}{b}}{\ce}{e}{\tau}} & = \sizeof{b}
\end{align*}
where $\sizeof{b}$ is the length of $b$ and $\sizeof{\ue}$ is the sum of the lengths of the literal bodies in $\ue$,
\begin{align*}
\sizeof{\ux} & = 0\\
\sizeof{\aulam{\utau}{\ux}{\ue}} &= \sizeof{\ue}\\
\sizeof{\auap{\ue_1}{\ue_2}} & = \sizeof{\ue_1} + \sizeof{\ue_2}\\
\sizeof{\autlam{\ut}{\ue}} & = \sizeof{\ue}\\
\sizeof{\autap{\ue}{\utau}} & = \sizeof{\ue}\\
\sizeof{\aufold{\ut}{\utau}{\ue}} & = \sizeof{\ue}\\
\sizeof{\auunfold{\ue}} & = \sizeof{\ue}\\
%\end{align*}
%\begin{align*}
\sizeof{\autpl{\labelset}{\mapschema{\ue}{i}{\labelset}}} & = \sum_{i \in \labelset} \sizeof{\ue_i}\\
\sizeof{\aupr{\ell}{\ue}} & = \sizeof{\ue}\\
\sizeof{\auin{\labelset}{\ell}{\mapschema{\utau}{i}{\labelset}}{\ue}} & = \sizeof{\ue}\\
\sizeof{\aucase{\labelset}{\utau}{\ue}{\mapschemab{\ux}{\ue}{i}{\labelset}}} & = \sizeof{\ue} + \sum_{i \in \labelset} \sizeof{\ue_i}\\
\sizeof{\audefuetsm{\utau}{\eparse}{\tsmv}{\ue}} & = \sizeof{\ue}\\
\sizeof{\autsmap{b}{\tsmv}} & = \sizeof{b}
\end{align*}

The only case in the proof of part 1 that invokes part 2 is Case (\ref{rule:expandsU-tsmap}). There, we have that the metric remains stable: \begin{align*}
 & \sizeof{\expandsU{\uDelta}{\uGamma}{\uPsi, \uShyp{\tsmv}{a}{\tau}{\eparse}}{\autsmap{b}{\tsmv}}{e}{\tau}}\\
=& \sizeof{\cvalidE{\emptyset}{\emptyset}{\esceneU{\uDelta}{\uGamma}{\uPsi, \uShyp{\tsmv}{a}{\tau}{\eparse}}{b}}{\ce}{e}{\tau}}\\
=&\sizeof{b}\end{align*}

The only case in the proof of part 2 that invokes part 1 is Case (\ref{rule:cvalidE-U-splicede}). There, we have that $\parseUExp{\bsubseq{b}{m}{n}}{\ue}$ and the IH is applied to the judgement $\expandsU{\uDelta_\text{app}}{\uGamma_\text{app}}{\uPsi}{\ue}{e}{\tau}$ where $\uDelta_\text{app}=\uDD{\uD}{\Delta_\text{app}}$ and $\uGamma_\text{app}=\uGG{\uG}{\Gamma_\text{app}}$ and $\uPsi=\uAS{\uA}{\Psi}$. Because the metric is stable when passing from part 1 to part 2, we must have that it is strictly decreasing in the other direction:
\[\sizeof{\expandsU{\uDelta_\text{app}}{\uGamma_\text{app}}{\uPsi}{\ue}{e}{\tau}} < \sizeof{\cvalidE{\Delta}{\Gamma}{\esceneU{\uDelta_\text{app}}{\uGamma_\text{app}}{\uPsi}{b}}{\acesplicede{m}{n}}{e}{\tau}}\]
i.e. by the definitions above, 
\[\sizeof{\ue} < \sizeof{b}\]

This is established by appeal to the following two conditions. The first condition simply states that subsequences of $b$ are no longer than $b$.
\begin{condition}[Body Subsequencing]\label{condition:body-subsequences} If $\bsubseq{b}{m}{n}=b'$ then $\sizeof{b'} \leq \sizeof{b}$. \end{condition}
The second condition states that an unexpanded expression constructed by parsing a textual sequence $b$ is strictly smaller, as measured by the metric defined above, than the length of $b$, because some characters must necessarily be used to invoke a TSM and delimit each literal body.
\begin{condition}[Expression Parsing Monotonicity]\label{condition:body-parsing} If $\parseUExp{b}{\ue}$ then $\sizeof{\ue} < \sizeof{b}$.\end{condition}

Combining Conditions \ref{condition:body-subsequences} and \ref{condition:body-parsing}, we have that $\sizeof{\ue} < \sizeof{b}$ as needed.
\end{proof}
% We need to define the following theorem about candidate expansion expression validation mutually with Theorem \ref{thm:typed-expansion-U}. 
% \begin{theorem}[Candidate Expansion Expression Validation]\label{thm:candidate-expansion-validation-U}
% If $\cvalidE{\Delta}{\Gamma}{\esceneU{\Delta_\text{app}}{\Gamma_\text{app}}{\uPsi}{b}}{\ce}{e}{\tau}$ and $\uetsmenv{\Delta_\text{app}}{\uPsi}$ then $\hastypeU{\Dcons{\Delta}{\Delta_\text{app}}}{\Gcons{\Gamma}{\Gamma_\text{app}}}{e}{\tau}$.
% \end{theorem}
% \begin{proof} By rule induction over Rules (\ref{rules:cvalidE-U}).
% \begin{byCases}
% \item[\text{(\ref{rule:cvalidE-U-var})}] We have
% \begin{pfsteps*}
%   \item $\ce=x$ \BY{assumption}
%   \item $e=x$ \BY{assumption}
%   \item $\Gamma=\Gamma', \Ghyp{x}{\tau}$ \BY{assumption}
%   \item $\hastypeU{\Dcons{\Delta}{\Delta_\text{app}}}{\Gamma', \Ghyp{x}{\tau}}{x}{\tau}$ \BY{Rule (\ref{rule:hastypeU-var})} \pflabel{hastypeU}
%   \item $\hastypeU{\Dcons{\Delta}{\Delta_\text{app}}}{\Gcons{\Gamma', \Ghyp{x}{\tau}}{\Gamma_\text{app}}}{x}{\tau}$ \BY{Lemma \ref{lemma:weakening-U} over $\Gamma_\text{app}$ to \pfref{hastypeU}}
% \end{pfsteps*}
% \resetpfcounter

% \item[\text{(\ref{rule:cvalidE-U-lam})}] We have
% \begin{pfsteps*}
%   \item $\ce=\acelam{\ctau_1}{x}{\ce'}$ \BY{assumption}
%   \item $e=\aelam{\tau_1}{x}{e'}$ \BY{assumption}
%   \item $\tau=\aparr{\tau_1}{\tau_2}$ \BY{assumption}
%   \item $\cvalidT{\Delta}{\esceneU{\Delta_\text{app}}{\Gamma_\text{app}}{\uPsi}{b}}{\ctau_1}{\tau_1}$ \BY{assumption} \pflabel{cvalidT}
%   \item $\cvalidE{\Delta}{\Gamma, \Ghyp{x}{\tau_1}}{\esceneU{\Delta_\text{app}}{\Gamma_\text{app}}{\uPsi}{b}}{\ce'}{e'}{\tau_2}$ \BY{assumption} \pflabel{cvalidE}
%   \item $\uetsmenv{\Delta_\text{app}}{\uPsi}$ \BY{assumption} \pflabel{uetsmenv}
%   \item $\istypeU{\Dcons{\Delta}{\Delta_\text{app}}}{\tau_1}$ \BY{Lemma \ref{lemma:candidate-expansion-type-validation} on \pfref{cvalidT}} \pflabel{istype}
%   \item $\hastypeU{\Dcons{\Delta}{\Delta_\text{app}}}{\Gcons{\Gamma, \Ghyp{x}{\tau_1}}{\Gamma_\text{app}}}{e'}{\tau_2}$ \BY{IH on \pfref{cvalidE} and \pfref{uetsmenv}} \pflabel{hastype1}
%   \item $\hastypeU{\Dcons{\Delta}{\Delta_\text{app}}}{\Gcons{\Gamma}{\Gamma_\text{app}}, \Ghyp{x}{\tau_1}}{e'}{\tau_2}$ \BY{exchange over $\Gamma_\text{app}$ on \pfref{hastype1}} \pflabel{hastype2}
%   \item $\hastypeU{\Dcons{\Delta}{\Delta_\text{app}}}{\Gcons{\Gamma}{\Gamma_\text{app}}}{\aelam{\tau_1}{x}{e'}}{\aparr{\tau_1}{\tau_2}}$ \BY{Rule (\ref{rule:hastypeU-lam}) on \pfref{istype} and \pfref{hastype2}}
% \end{pfsteps*}
% \resetpfcounter

% \item[\text{(\ref{rule:cvalidE-U-ap})}] We have
% \begin{pfsteps*}
%   \item $\ce=\aceap{\ce_1}{\ce_2}$ \BY{assumption}
%   \item $e=\aeap{e_1}{e_2}$ \BY{assumption}
%   \item $\cvalidE{\Delta}{\Gamma}{\esceneU{\Delta_\text{app}}{\Gamma_\text{app}}{\uPsi}{b}}{\ce_1}{e_1}{\aparr{\tau_1}{\tau}}$ \BY{assumption} \pflabel{cvalidE1}
%   \item $\cvalidE{\Delta}{\Gamma}{\esceneU{\Delta_\text{app}}{\Gamma_\text{app}}{\uPsi}{b}}{\ce_2}{e_2}{\tau_1}$ \BY{assumption} \pflabel{cvalidE2}
%   \item $\uetsmenv{\Delta_\text{app}}{\uPsi}$ \BY{assumption} \pflabel{uetsmenv}
%   \item $\hastypeU{\Dcons{\Delta}{\Delta_\text{app}}}{\Gcons{\Gamma}{\Gamma_\text{app}}}{e_1}{\aparr{\tau_1}{\tau}}$ \BY{IH on \pfref{cvalidE1} and \pfref{uetsmenv}} \pflabel{hastypeU1}
%   \item $\hastypeU{\Dcons{\Delta}{\Delta_\text{app}}}{\Gcons{\Gamma}{\Gamma_\text{app}}}{e_2}{\tau_1}$ \BY{IH on \pfref{cvalidE2} and \pfref{uetsmenv}} \pflabel{hastypeU2}
%   \item $\hastypeU{\Dcons{\Delta}{\Delta_\text{app}}}{\Gcons{\Gamma}{\Gamma_\text{app}}}{\aeap{e_1}{e_2}}{\tau}$ \BY{Rule (\ref{rule:hastypeU-ap}) on \pfref{hastypeU1} and \pfref{hastypeU2}}
% \end{pfsteps*}
% \resetpfcounter

% \item[\VExpof{\text{\ref{rule:hastypeU-tlam}}}~\text{through}~\VExpof{\text{\ref{rule:hastypeU-case}}}] These cases follow analagously, i.e. we apply the IH to all candidate expansion expression validation premises, Lemma \ref{lemma:candidate-expansion-type-validation} to all candidate expansion type validation premises, weakening and exchange as needed, and then apply the corresponding typing rule.
% \\

% \item[\text{(\ref{rule:cvalidE-U-splicede})}] We have
% \begin{pfsteps*}
%   \item $\ce=\acesplicede{m}{n}$ \BY{assumption}
%   \item $\parseUExp{\bsubseq{b}{m}{n}}{\ue}$ \BY{assumption}
%   \item $\expandsU{\Delta_\text{app}}{\Gamma_\text{app}}{\uPsi}{\ue}{e}{\tau}$ \BY{assumption} \pflabel{expands}
%   \item $\uetsmenv{\Delta_\text{app}}{\uPsi}$ \BY{assumption} \pflabel{uetsmenv}
%   \item $\hastypeU{\Delta_\text{app}}{\Gamma_\text{app}}{e}{\tau}$ \BY{Theorem \ref{thm:typed-expansion-U} on \pfref{expands} and \pfref{uetsmenv}} \pflabel{hastype}
%   \item $\hastypeU{\Dcons{\Delta}{\Delta_\text{app}}}{\Gcons{\Gamma}{\Gamma_\text{app}}}{e}{\tau}$ \BY{Lemma \ref{lemma:weakening-U} on \pfref{hastype}}
% \end{pfsteps*}
% \resetpfcounter
% \end{byCases}
% \end{proof}


%\qed



% !TEX root = omar-thesis.tex
\chapter{Unparameterized Pattern TSMs}\label{chap:uptsms}
In Chapter \ref{chap:uetsms}, we considered situations where the programmer needed to \emph{construct} (a.k.a. \emph{introduce}) a value. In this chapter, we consider situations where the programmer needs to \emph{deconstruct} (a.k.a. \emph{eliminate}) a value. In full-scale functional languages like ML and Haskell, values are deconstructed by \emph{pattern matching} over their structure. For example, recall the recursive labeled sum type \lstinline{Rx} defined in Figure \ref{fig:datatype-rx}. We can pattern match over a value, \lstinline{r}, of type \lstinline{Rx} using VerseML's \lstinline{match} construct:
\begin{lstlisting}
fun read_example_rx(r : Rx) : (string * Rx) option => 
  match r with 
    Seq(Str(name), Seq(Str "SSTR: ESTR", ssn)) => Some (name, ssn)
  | _ => None
\end{lstlisting}

Match expressions consist of a \emph{scrutinee}, here \li{r}, and a sequence of \emph{rules} separated by vertical bars, \li{|}, in the concrete syntax. Each rule consists of a \emph{pattern} and an {expression} called the corresponding \emph{branch}, separated by a double arrow, \li{=>}, in the concrete syntax. When the {match} expression is evaluated, the value of the scrutinee is matched against each pattern sequentially. If the value matches, evaluation takes the corresponding branch. Variables in patterns match any value of the appropriate type. In the corresponding branch, the variable stands for that value. Variables can each appear only once in a pattern.  
For example, on Line 3, the pattern \li{Seq(Str(name), Seq(Str "SSTR: ESTR", ssn))} matches values of the form \li{Seq(Str(#$e_1$#), Seq(Str "SSTR: ESTR", #$e_2$#))}, where $e_1$ is a value of type \li{string} and $e_2$ is a value of type \li{Rx}. The variables \li{name} and \li{ssn} stand for the values of $e_1$ and $e_2$, respectively, in \li{Some (name, ssn)}. On Line 4, the pattern \li{_} is the \emph{wildcard pattern} -- it matches any value of the appropriate type and binds no variables.

The behavior of the \li{match} construct when no pattern in the rule sequence matches a value is to raise an exception indicating \emph{match failure}. It is possible to statically determine whether match failure is possible (i.e. whether there exist values of the scrutinee that are not matched by any pattern in the rule sequence). In the example above, our use of the wildcard pattern ensures that match failure cannot occur. A rule sequence that cannot lead to match failure is said to be \emph{exhaustive}. Most compilers warn the programmer when a rule sequence is non-exhaustive.

It is also possible to statically decide when a rule is \emph{redundant} relative to the preceding rules, i.e. when there does not exist a value matched by that rule but not matched by any of the preceding rules. For example, if we add  another rule at the end of the match expression above, it will be redundant because all values match the wildcard pattern. Again, most compilers warn the programmer when a rule is redundant.

Nested pattern matching generalizes the projection and case analysis operators (i.e. the \emph{eliminators}) for products and sums (cf. $\miniVerseUE$ from the previous section) and decreases syntactic cost in situations where eliminators would need to be nested. There remains room for improvement, however, because complex patterns sometimes    individually have high syntactic cost. In Sec. \ref{sec:syntax-examples-regexps}, we considered a hypothetical dialect of ML called ML+Rx that built in derived syntax both for constructing and pattern matching over values of the recursive labeled sum type \li{Rx}. In ML+Rx, we can express the example above at lower syntactic cost as follows:

\begin{lstlisting}
fun read_example_rx(r : Rx) : (string * Rx) option => 
  match r with 
    /SURL@EURLnameSURL: %EURLssn/ => Some (name, ssn)
  | _ => None\end{lstlisting}
\noindent
Dialect formation is not a modular approach, for the reasons discussed in Chapter \ref{chap:intro}, so we seek language constructs that allow us to decrease the syntactic cost of expressing complex patterns to a similar degree.

Expression TSMs -- introduced in Chapter \ref{chap:uetsms} -- can decrease the syntactic cost of constructing a value of a specified type. However, expressions are syntactically distinct from patterns, so we cannot simply apply an expression TSM to generate a pattern.\footnote{The fact that certain concrete expression and pattern forms overlap is immaterial to this fundamental distinction. There are many expression forms that the expansion generated by an expression TSM might use that have no corresponding pattern form, e.g.  lambda abstraction.} %For example, the expansion generated by an expression TSM might define or apply a function, but patterns do not contain functions or function applications. 
For this reason, we need to introduce a new (albeit closely related) construct -- the \textbf{pattern TSM}. In this chapter, we consider only \textbf{unparameterized pattern TSMs} (upTSMs), i.e. pattern TSMs that generate patterns that match values of a single specified type, like \li{Rx}. In Chapter \ref{chap:ptsms}, we will consider both expression and pattern TSMs that specify type and module parameters (peTSMs and ppTSMs). 

\section{Pattern TSMs By Example}\label{sec:ptsms-by-example}
The organization of the remainder of this chapter mirrors that of Chapter \ref{chap:uetsms}. We begin in this section with a ``tutorial-style'' introduction to upTSMs in VerseML. In particular, we  discuss an upTSM for patterns matching values of type \li{Rx}. In the next section, we specify a reduced formal system based on $\miniVerseUE$ called $\miniVersePat$ that makes the intuitions developed here mathematically precise.

\subsection{Usage}\label{sec:ptsms-usage}
The VerseML function \li{read_example_rx} defined at the beginning of this chapter can be concretely expressed at lower syntactic cost by applying a upTSM, \li{#\dolla#rx}, as follows:
\begin{lstlisting}
fun read_example_rx(r : Rx) : (string * Rx) option => 
  match r with 
    $rx /SURL@EURLnameSURL: %EURLssn/ => Some (name, ssn)
  | _ => None
\end{lstlisting}
Like expression TSMs, pattern TSMs are applied to \emph{generalized literal forms} (see Figure \ref{fig:literal-forms}). Generalized literal forms are left unparsed when patterns are first parsed. During the subsequent \emph{typed expansion} process, the pattern TSM parses the body of the literal form to generate a \emph{candidate expansion}. The language validates the candidate expansion according to criteria that we will establish in Sec. \ref{sec:ptsms-validation}. If validation succeeds, the language generates the final expansion (or more concisely, simply the expansion) of the pattern. The expansion of the unexpanded pattern \li{#\dolla#rx /SURL@EURLnameSURL: %EURLssn/} from the example above is the following pattern:
\begin{lstlisting}[numbers=none]
Seq(Str(name), Seq(Str "SSTR: ESTR", ssn))
\end{lstlisting}

The checks for exhaustiveness and redundancy can be performed post-expansion in the usual way, so we do not need to consider them further here. 
\subsection{Definition}\label{sec:ptsms-definition}
The definition of the pattern TSM \li{#\dolla#rx} shown being applied in the example above has the following form:
\begin{lstlisting}[numbers=none]
syntax $rx at Rx for patterns {
  static fn(body : Body) : CEPat ParseResult =>
    (* regex pattern parser here *)
}
\end{lstlisting}
This definition first names the pattern TSM. Pattern TSM names, like expression TSM names, must begin with the dollar symbol (\li{#\dolla#}) to distinguish them from labels. Pattern TSM names and expression TSM names are tracked separately, i.e. an expression TSM and a pattern TSM can have the same name without conflict (as is the case here -- the expression TSM described in Sec. \ref{sec:uetsms-definition} is also named \li{#\dolla#rx}). The \emph{sort qualifier} \li{for patterns} indicates that this is a pattern TSM definition, rather than an expression TSM definition (the sort qualifier \li{for expressions} can be written for expression TSMs, though when the sort qualifier is omitted this is the default). Because defining both an expression TSM and a pattern TSM with the same name at the same type is a common idiom, VerseML provides a primitive derived form for combining their definitions:
\begin{lstlisting}[numbers=none]
syntax $rx at Rx for expressions {
  static fn(body : Body) : CEExp ParseResult => 
    (* regex expression parser here *)
} for patterns {
  static fn(body : Body) : CEPat ParseResult => 
    (* regex pattern parser here *)
}
\end{lstlisting}

Pattern TSMs, like expression TSMs, must specify a static \emph{parse function}, delimited by curly braces in the concrete syntax. For a pattern TSM, the parse function must be of type \li{Body -> CEPat ParseResult}. The input type, \li{Body}, gives the parse function access to the body of the provided literal form, and is defined as in Sec. \ref{sec:uetsms-definition} as a synonym for the type \li{string}. The output type, \li{CEPat ParseResult}, is the parameterized type constructor \li{ParseResult}, defined in Figure \ref{fig:indexrange-and-parseresult}, applied to the type \li{CEPat} defined in Figure \ref{fig:CEPat}.  So if parsing succeeds, the pattern TSM returns a value of the form \li{Success #$\ecand$#} where $\ecand$ is a value of type \li{CEPat} that we call the \emph{encoding of the candidate expansion}. If parsing fails, then the pattern TSM returns a value constructed by \li{ParseError} and equipped with an error message and error location. 

The type \li{CEPat} is analagous to the types \li{CEExp} and \li{CETyp} defined in Figure \ref{fig:candidate-exp-verseml}. It encodes the abstract syntax of VerseML patterns (in Figure \ref{fig:CEPat}, some constructors are elided for concision), with the exception of variable patterns (for reasons explained in Sec. \ref{sec:ptsms-hygiene} below), and includes an additional constructor, \li{Spliced}, for referring to spliced subpatterns by their position within the parse stream, discussed next.

\begin{figure}
\begin{lstlisting}[numbers=none]
type CEPat = Wild
           | (* ... *)
           | Spliced of IndexRange
\end{lstlisting}
\caption[Abbreviated definition of \li{CEPat} in VerseML]{Abbreviated definition of \li{CEPat} in the VerseML prelude.}
\label{fig:CEPat}
\end{figure}

\subsection{Splicing}\label{sec:ptsms-splicing}
Patterns that appear directly within the literal body of an unexpanded pattern are called \emph{spliced subpatterns}. For example, the patterns \li{name} and \li{ssn} appear within the unexpanded pattern \li{#\dolla#rx /SURL@EURLnameSURL: %EURLssn/}. 
When the parse function determines that a subsequence of the literal body should be treated as a spliced subpatern (here, by recognizing the characters \li{@} or \li{%} followed by a variable or parenthesized pattern), 
it can refer to it within the candidate expansion that it construct a reference to it for use within the candidate expansion it generates using the \li{Spliced} constructor of the \li{CEPat} type shown in Figure \ref{fig:CEPat}. The \li{Spliced} constructor requires a value of type \li{IndexRange} because spliced subpatterns are referred to indirectly by their position within the literal body. This prevents pattern TSMs from ``forging'' a spliced subpattern (i.e. claiming that some pattern is a spliced subpattern, even though it does not appear in the literal body).

The candidate expansion generated by the pattern TSM \li{#\dolla#rx} for the example above, if written in a hypothetical concrete syntax where references to spliced subpatterns are written \li{spliced<startIdx, endIdx>}, is:
\begin{lstlisting}[numbers=none]
Seq(Str(spliced<1, 4>), Seq(Str "SSTR: ESTR", spliced<8, 10>))
\end{lstlisting}
Here, \li{spliced<1, 4>} refers to the subpattern \li{name} by position, and \li{spliced<8, 10>} refers to the subpattern \li{ssn} by position.

\subsection{Typing}\label{sec:ptsms-validation}
The language validates candidate expansion before a final expansion is generated. One aspect of candidate expansion validation is checking the candidate expansion against the type annotation specified by the pattern TSM, e.g. the type \li{Rx} in the example above.

\subsection{Hygiene}\label{sec:ptsms-hygiene}
In order to check that the candidate expansion is well-typed, the language must parse, type and expand the spliced subpatterns that the candidate expansion refers to (by their position within the literal body, cf. above). To maintain a useful binding discipline, i.e. to allow programmers to reason about variable binding without examining expansions directly, the validation process allows variables (e.g. \lstinline{name} and \lstinline{ssn} above) to occur only in spliced subpatterns (just as variables bound at the use site can only appear in spliced subexpressions when using TSMs). Indeed, there is no constructor for the type \li{CEPat} corresponding to a variable pattern. This protection against ``hidden bindings'' is beneficial because it leaves variable naming entirely up to the client of the pattern TSM. A pattern TSM cannot inadvertently shadow a binding at the application site.

\subsection{Final Expansion}\label{sec:ptsms-final-expansion}
If validation succeeds, the semantics generates the \emph{final expansion} of the pattern from the candidate expansion by replacing the references to spliced subpatterns with their final expansions. For example, the final expansion of \li{#\dolla#rx /SURL@EURLnameSURL: %EURLssn/} is:
\begin{lstlisting}[numbers=none]
Seq(Str(name), Seq(Str "SSTR: ESTR", ssn))
\end{lstlisting}

\section{\texorpdfstring{$\miniVersePat$}{miniVerseU}}\label{sec:miniVerseUP}
To make the intuitions developed in the previous section about pattern TSMs precise, we  now introduce $\miniVersePat$, a small language with support for both unparameterized expression TSMs and unparameterized pattern TSMs.
\subsection{Syntax of the Inner Core}\label{sec:UP-expanded-terms}
The \emph{inner core} of $\miniVersePat$ consists of \emph{types}, $\tau$, \emph{expanded expressions}, $e$, \emph{expanded rules}, $r$, and \emph{expanded patterns}, $p$. Their syntax is specified by the syntax chart in Figure \ref{fig:UP-expanded-terms}. The inner core of $\miniVersePat$ differs from that of $\miniVerseUE$  only in that the case analysis operator has been replaced by the pattern matching operator\footnote{We do not also remove the projection operator because it has lower syntactic cost than pattern matching when only a single field from a labeled tuple is needed.}, so we will gloss some definitions that are identical to those in Sec. \ref{sec:miniVerseU}. The new constructs are highlighted in gray. Our formulation of the semantics of pattern matching is adapted from Harper's formulation in \emph{Practical Foundations for Programming Languages, First Edition} \cite{pfple1}.\footnote{The chapter on pattern matching has, of this writing, been removed from the draft second edition of \emph{PFPL}, but a copy of the first edition can be found online.}

\begin{figure}
$\begin{array}{lllllll}
\textbf{Sort} & & & \textbf{Operational Form} & \textbf{Stylized Form} & \textbf{Description}\\
\mathsf{Typ} & \tau & ::= & t & t & \text{variable}\\
&&& \aparr{\tau}{\tau} & \parr{\tau}{\tau} & \text{partial function}\\
&&& \aall{t}{\tau} & \forallt{t}{\tau} & \text{polymorphic}\\
&&& \arec{t}{\tau} & \rect{t}{\tau} & \text{recursive}\\
&&& \aprod{\labelset}{\mapschema{\tau}{i}{\labelset}} & \prodt{\mapschema{\tau}{i}{\labelset}} & \text{labeled product}\\
&&& \asum{\labelset}{\mapschema{\tau}{i}{\labelset}} & \sumt{\mapschema{\tau}{i}{\labelset}} & \text{labeled sum}\\
\mathsf{Exp} & e & ::= & x & x & \text{variable}\\
&&& \aelam{\tau}{x}{e} & \lam{x}{\tau}{e} & \text{abstraction}\\
&&& \aeap{e}{e} & \ap{e}{e} & \text{application}\\
&&& \aetlam{t}{e} & \Lam{t}{e} & \text{type abstraction}\\
&&& \aetap{e}{\tau} & \App{e}{\tau} & \text{type application}\\
&&& \aefold{t}{\tau}{e} & \fold{e} & \text{fold}\\
&&& \aeunfold{e} & \unfold{e} & \text{unfold}\\
&&& \aetpl{\labelset}{\mapschema{e}{i}{\labelset}} & \tpl{\mapschema{e}{i}{\labelset}} & \text{labeled tuple}\\
&&& \aepr{\ell}{e} & \prj{e}{\ell} & \text{projection}\\
&&& \aein{\labelset}{\ell}{\mapschema{\tau}{i}{\labelset}}{e} & \inj{\ell}{e} & \text{injection}\\
\LCC \lightgray & \lightgray & \lightgray & \lightgray & \lightgray & \lightgray \\
&&& \aematchwith{n}{\tau}{e}{\seqschemaX{r}} & \matchwith{e}{\seqschemaX{r}} & \text{match}\\
\mathsf{ERule} & r & ::= & \aematchrule{n}{p}{\seqschemaX{x}}{e} & \matchrule{p}{e} & \text{rule}\\
\mathsf{EPat} & p & ::= & x & x & \text{variable pattern}\\
&&& \aewildp & \wildp & \text{wildcard pattern}\\
%&&& \aefoldp{p} & \foldp{p} & \text{fold pattern}\\
&&& \aetplp{\labelset}{\mapschema{p}{i}{\labelset}} & \tplp{\mapschema{p}{i}{\labelset}} & \text{labeled tuple pattern}\\
&&& \aeinjp{\ell}{p} & \injp{\ell}{p} & \text{injection pattern}\ECC
\end{array}$
\caption[Syntax of types and expanded expressions, rules and patterns in $\miniVersePat$]{Syntax of types and expanded expressions, rules and patterns (collectively, expanded terms) in $\miniVersePat$. We adopt the metatheoretic conventions established for our specification of $\miniVerseUE$ in Sec. \ref{sec:miniVerseU} without restating them, unless otherwise specified. We write $\seqschemaX{r}$ for sequences of $n \geq 0$ rule arguments and $\seqschemaX{x}.e$ for expressions binding the sequence of $n \geq 0$ variables $\seqschemaX{x}$. Variable patterns are not pattern variables, i.e. they do not stand for terms, but rather serve as references to the bindings in the rule that the pattern appears within. The semantics below will clarify this. Types and expanded terms are identified up to $\alpha$-equivalence.}
\label{fig:UP-expanded-terms}
\end{figure}


\subsection{Statics of the Inner Core}
The \emph{statics of the inner core} is specified by judgements of the following form:
\[\begin{array}{ll}
\textbf{Judgement Form} & \textbf{Description}\\
\istypeU{\Delta}{\tau} & \text{$\tau$ is a well-formed type assuming $\Delta$}\\
%\isctxU{\Delta}{\Gamma} & \text{$\Gamma$ is a well-formed typing context assuming $\Delta$}\\
\hastypeU{\Delta}{\Gamma}{e}{\tau} & \text{$e$ has type $\tau$ assuming $\Delta$ and $\Gamma$}\\
\ruleType{\Delta}{\Gamma}{r}{\tau}{\tau'} & \text{$r$ takes values of type $\tau$ to values of type $\tau'$ assuming $\Delta$ and $\Gamma$}\\
\patType{\pctx}{p}{\tau} & \text{$p$ matches values of type $\tau$ and generates hypotheses $\pctx$} 
\end{array}\]

The types of $\miniVersePat$ are exactly those of $\miniVerseUE$, described in Sec. \ref{sec:miniVerseU}, so the \emph{type formation judgement}, $\istypeU{\Delta}{\tau}$, is inductively defined by Rules (\ref{rules:istypeU}). 

The \emph{typing judgement}, $\hastypeU{\Delta}{\Gamma}{e}{\tau}$, assigns types to expressions and is inductively defined by Rules (\ref*{rules:hastypeUP}), which consist of:
\begin{subequations}\label{rules:hastypeUP}
\refstepcounter{equation}%
\begin{itemize}
\label{rule:hastypeUP-var}
\refstepcounter{equation}\label{rule:hastypeUP-lam}
\refstepcounter{equation}\label{rule:hastypeUP-ap}
\refstepcounter{equation}\label{rule:hastypeUP-tlam}
\refstepcounter{equation}\label{rule:hastypeUP-tap}
\refstepcounter{equation}\label{rule:hastypeUP-fold}
\refstepcounter{equation}\label{rule:hastypeUP-unfold}
\refstepcounter{equation}\label{rule:hastypeUP-tpl}
\refstepcounter{equation}\label{rule:hastypeUP-pr}
\refstepcounter{equation}\label{rule:hastypeUP-in}
\item Rules defined identically to Rules (\ref{rule:hastypeU-var}) through (\ref{rule:hastypeU-in}). We will refer to these rules as Rules (\ref*{rule:hastypeUP-var}) through (\ref*{rule:hastypeUP-in}). %Note that we cannot defer directly to the typing rules from Sec. \ref{sec:miniVerseU} because $e$ has been redefined here.
\item The following rule for match expressions: 
\end{itemize}
\begin{equation}\label{rule:hastypeUP-match}
\inferrule{
  \hastypeU{\Delta}{\Gamma}{e}{\tau}\\
  \istypeU{\Delta}{\tau'}\\
  \{\ruleType{\Delta}{\Gamma}{r_i}{\tau}{\tau'}\}_{1 \leq i \leq n}\\
}{\hastypeU{\Delta}{\Gamma}{\aematchwith{n}{\tau'}{e}{\seqschemaX{r}}}{\tau'}}
\end{equation}  
\end{subequations}
The first premise of Rule (\ref{rule:hastypeUP-match}) assigns a type, $\tau$, to the scrutinee, $e$. The second premise checks that the type of the expression as a whole, $\tau'$, is well-formed.\footnote{The second premise of Rule (\ref{rule:hastypeUP-match}), and the type argument in the match form, are necessary to maintain regularity, defined below, but only because when $n=0$, the type $\tau'$ is arbitrary. In all other cases, $\tau'$ can be determined by assigning types to the  branch expressions.} The third premise then ensures that each rule $r_i$, for $1 \leq i \leq n$, takes values of type $\tau$ to values of the type of the match expression as a whole, $\tau'$. This is expressed by the \emph{rule typing judgement}, $\ruleType{\Delta}{\Gamma}{r}{\tau}{\tau'}$, which is defined mutually with Rules (\ref{rules:hastypeUP}) by the following rule:
\begin{equation}\label{rule:ruleType}
\inferrule{
  \patType{\pctx}{p}{\tau}\\
  \domof{\pctx} = \seqschemaX{x}\\
  \hastypeU{\Delta}{\Gcons{\Gamma}{\pctx}}{e}{\tau'}
}{\ruleType{\Delta}{\Gamma}{\aematchrule{n}{p}{\seqschemaX{x}}{e}}{\tau}{\tau'}}
\end{equation}
The premises of Rule (\ref{rule:ruleType}) can be understood as follows, in order:
\begin{enumerate}
\item The first premise invokes the \emph{pattern typing judgement}, $\patType{\pctx}{p}{\tau}$, to check that the pattern, $p$, matches values of type $\tau$, and to gather the typing hypotheses that the pattern generates in a \emph{pattern typing context}, $\Omega$. Pattern typing contexts, like typing contexts, $\Gamma$, are finite functions from variables to  hypotheses of the form $x : \tau$. Algorithmically, however, one should consider the pattern typing context the ``output'' of the pattern typing judgement. %We use the letter $\pctx$ rather than $\Gamma$ only to emphasize that 

The pattern typing judgement is inductively defined by the following rules:
\begin{subequations}\label{rules:patType}
\begin{equation}\label{rule:patType-var}
\inferrule{ }{\patType{\Ghyp{x}{\tau}}{x}{\tau}}
\end{equation}
\begin{equation}\label{rule:patType-wild}
\inferrule{ }{\patType{\emptyset}{\aewildp}{\tau}}
\end{equation}
\begin{equation}\label{rule:patType-tpl}
\inferrule{
  \{\patType{\pctx_i}{p_i}{\tau_i}\}_{i \in \labelset}\\
  \{\{\domof{\pctx_i} \cap \domof{\pctx_j} = \emptyset\}_{j \in \labelset \setminus i}\}_{i \in \labelset}
}{
  \patType{\Gconsi{i \in \labelset}{\pctx_i}}{\aetplp{\labelset}{\mapschema{p}{i}{\labelset}}}{\aprod{\labelset}{\mapschema{\tau}{i}{\labelset}}}
}
\end{equation}
\begin{equation}\label{rule:patType-inj}
\inferrule{
  \patType{\pctx}{p}{\tau}
}{
  \patType{\pctx}{\aeinjp{\ell}{p}}{\asum{\labelset, \ell}{\mapschema{\tau}{i}{\labelset}; \mapitem{\ell}{\tau}}}
}
\end{equation}
\end{subequations}

Rule (\ref{rule:patType-var}) specifies that a variable pattern, $x$, can match values of any type, $\tau$, and generates the hypothesis that $x$ has the type $\tau$. 

Rule (\ref{rule:patType-wild}) specifies that a wildcard pattern can also match values of any type, $\tau$, but wildcard patterns generate no hypotheses. 

Labeled tuple patterns, $\aetplp{\labelset}{\mapschema{p}{i}{\labelset}}$, specify a subpattern $p_i$ for each label $i \in \labelset$. Rule (\ref{rule:patType-tpl}) specifies that a labeled tuple pattern of this form matches values of the labeled product type $\aprod{\labelset}{\mapschema{\tau}{i}{\labelset}}$. The first premise checks each subpattern $p_i$ against the corresponding type $\tau_i$, generating hypotheses $\pctx_i$. The second premise ensures that no variables are multiply bound by checking that the domains of the generated pattern typing contexts $\pctx_i$ are mutually disjoint. The hypotheses generated in the conclusion of the rule are the union of the hypotheses generated by the subpatterns. 

Injection patterns, $\aeinjp{\ell}{p}$, match values of labeled sum types of the form $\asum{\labelset, \ell}{\mapschema{\tau}{i}{\labelset}; \mapitem{\ell}{\tau}}$, i.e. labeled sum types that define a case for the label $\ell$. Rule (\ref{rule:patType-inj}) checks the subpattern $p$ against the corresponding type $\tau$, and passes through the assumptions that $p$ generates.

\item The second premise of Rule (\ref{rule:ruleType}) ensures that pattern typing of $p$ has generated hypotheses for all of the variables that the branch expression, $e$, binds. This is merely a matter of ``metatheoretic bookkeeping''. In the stylized form for rules, $\matchrule{p}{e}$, the variables bound in $e$ are, implicitly, exactly those mentioned in p.% The bindings for $e$ would be extracted from the pattern implicitly. 
\item The final premise of Rule (\ref{rule:ruleType}) extends the typing context, $\Gamma$, with the hypotheses generated by pattern typing, $\pctx$, and checks the branch expression, $e$, against the branch type, $\tau'$.
\end{enumerate}

The rules above are syntax-directed, so we assume an inversion lemma for each rule as needed without stating it separately or proving it explicitly. The following standard lemmas also hold.

The Weakening Lemma establishes that extending the context with unnecessary hypotheses preserves well-formedness and typing.
\begin{lemma}[Weakening]\label{lemma:weakening-UP} All of the following hold: 
\begin{enumerate} 
\item If $\istypeU{\Delta}{\tau}$ then $\istypeU{\Delta, \Dhyp{t}}{\tau}$.
%\item If $\isctxU{\Delta}{\Gamma}$ then $\isctxU{\Delta, \Dhyp{t}}{\Gamma}$.
\item \begin{enumerate}
  \item If $\hastypeU{\Delta}{\Gamma}{e}{\tau}$ then $\hastypeU{\Delta, \Dhyp{t}}{\Gamma}{e}{\tau}$.
  \item If $\ruleType{\Delta}{\Gamma}{r}{\tau}{\tau'}$ then $\ruleType{\Delta, \Dhyp{t}}{\Gamma}{r}{\tau}{\tau'}$.
  \end{enumerate}
\item \begin{enumerate}
  \item If $\hastypeU{\Delta}{\Gamma}{e}{\tau}$ and $\istypeU{\Delta}{\tau''}$ then $\hastypeU{\Delta}{\Gamma, \Ghyp{x}{\tau''}}{e}{\tau}$.
  \item If $\ruleType{\Delta}{\Gamma}{r}{\tau}{\tau'}$ and $\istypeU{\Delta}{\tau''}$ then $\ruleType{\Delta}{\Gamma, \Ghyp{x}{\tau''}}{r}{\tau}{\tau'}$.
  \end{enumerate}
\end{enumerate}
\end{lemma}
\begin{proof-sketch}
\begin{enumerate}
\item By rule induction over Rules (\ref{rules:istypeU}).
%\item By rule induction over Rules (\ref{rules:isctxU}).
\item By mutual rule induction over Rules (\ref{rules:hastypeUP}) and Rule (\ref{rule:ruleType}).
\item By mutual rule induction over Rules (\ref{rules:hastypeUP}) and Rule (\ref{rule:ruleType}).
\end{enumerate}
\end{proof-sketch}

The {pattern typing judgement} is a \emph{linear hypothetical judgement}, i.e. it does \emph{not} obey weakening of the pattern typing context. This is to ensure that the pattern typing context captures exactly those hypotheses generated by a pattern, and no others.

We assume that renaming of bound variables, $\alpha$-equivalence and substitution are defined as in \emph{PFPL} \cite{pfpl}, with the additional stipulation that the variables that are bound by the branch expression in an expanded rule are renamed together with those in the corresponding expanded pattern. The Substitution Lemma establishes that substitution of a well-formed type for a type variable, or an expanded expression of the appropriate type for an expanded expression variable, preserves well-formedness and typing.
\begin{lemma}[Substitution]\label{lemma:substitution-UP} All of the following hold:
\begin{enumerate}
\item If $\istypeU{\Delta, \Dhyp{t}}{\tau}$ and $\istypeU{\Delta}{\tau'}$ then $\istypeU{\Delta}{[\tau'/t]\tau}$.
%\item If $\isctxU{\Delta, \Dhyp{t}}{\Gamma}$ and $\istypeU{\Delta}{\tau'}$ then $\isctxU{\Delta}{[\tau'/t]\Gamma}$.
\item \begin{enumerate}
  \item If $\hastypeU{\Delta, \Dhyp{t}}{\Gamma}{e}{\tau}$ and $\istypeU{\Delta}{\tau'}$ then $\hastypeU{\Delta}{[\tau'/t]\Gamma}{[\tau'/t]e}{[\tau'/t]\tau}$.
  \item If $\ruleType{\Delta, \Dhyp{t}}{\Gamma}{r}{\tau}{\tau''}$ and $\istypeU{\Delta}{\tau'}$ then $\ruleType{\Delta}{[\tau'/t]\Gamma}{[\tau'/t]r}{[\tau'/t]\tau}{[\tau'/t]\tau''}$.
  \end{enumerate}
\item \begin{enumerate}
  \item If $\hastypeU{\Delta}{\Gamma, \Ghyp{x}{\tau'}}{e}{\tau}$ and $\hastypeU{\Delta}{\Gamma}{e'}{\tau'}$ then $\hastypeU{\Delta}{\Gamma}{[e'/x]e}{\tau}$.
  \item If $\ruleType{\Delta}{\Gamma, \Ghyp{x}{\tau'}}{r}{\tau}{\tau''}$ and $\hastypeU{\Delta}{\Gamma}{e'}{\tau''}$ then $\ruleType{\Delta}{\Gamma}{[e'/x]r}{\tau}{\tau''}$.
  \end{enumerate}
\end{enumerate}\end{lemma}
\begin{proof-sketch}
\begin{enumerate}
\item By rule induction over Rules (\ref{rules:istypeU}).
\item By mutual rule induction over Rules (\ref{rules:hastypeUP}) and Rule (\ref{rule:ruleType}).
\item By mutual rule induction over Rules (\ref{rules:hastypeUP}) and Rule (\ref{rule:ruleType}).
\end{enumerate}
\end{proof-sketch}

The Decomposition Lemma is the converse of the Substitution Lemma.
\begin{lemma}[Decomposition]\label{lemma:decomposition-UP} All of the following hold:
\begin{enumerate}
\item If $\istypeU{\Delta}{[\tau'/t]\tau}$ and $\istypeU{\Delta}{\tau'}$ then $\istypeU{\Delta, \Dhyp{t}}{\tau}$.
%\item If $\isctxU{\Delta}{[\tau'/t]\Gamma}$ and $\istypeU{\Delta}{\tau'}$ then $\isctxU{\Delta, \Dhyp{t}}{\Gamma}$.
\item \begin{enumerate}
  \item If $\hastypeU{\Delta}{[\tau'/t]\Gamma}{[\tau'/t]e}{[\tau'/t]\tau}$ and $\istypeU{\Delta}{\tau'}$ then $\hastypeU{\Delta, \Dhyp{t}}{\Gamma}{e}{\tau}$.
  \item If $\ruleType{\Delta}{[\tau'/t]\Gamma}{[\tau'/t]r}{[\tau'/t]\tau}{[\tau'/t]\tau''}$ and $\istypeU{\Delta}{\tau'}$ then $\ruleType{\Delta, \Dhyp{t}}{\Gamma}{r}{\tau}{\tau''}$.
  \end{enumerate}
\item \begin{enumerate}
  \item If $\hastypeU{\Delta}{\Gamma}{[e'/x]e}{\tau}$ and $\hastypeU{\Delta}{\Gamma}{e'}{\tau'}$ then $\hastypeU{\Delta}{\Gamma, \Ghyp{x}{\tau'}}{e}{\tau}$.
  \item If $\ruleType{\Delta}{\Gamma}{[e'/x]r}{\tau}{\tau''}$ and $\hastypeU{\Delta}{\Gamma}{e'}{\tau'}$ then $\ruleType{\Delta}{\Gamma, \Ghyp{x}{\tau'}}{r}{\tau}{\tau''}$.
  \end{enumerate}
\end{enumerate}\end{lemma}
\begin{proof-sketch}
\begin{enumerate}
\item By rule induction over Rules (\ref{rules:istypeU}) and case analysis over the definition of substitution. In all cases, the derivation of $\istypeU{\Delta}{[\tau'/t]\tau}$ does not depend on the form of $\tau'$.
%\item Context formation of $[\tau'/t]\Gamma$ does not depend on the structure of $\tau'$.
\item By mutual rule induction over Rules (\ref{rules:hastypeUP}) and Rule (\ref{rule:ruleType}) and case analysis over the definition of substitution. In all cases, the derivation of $\hastypeU{\Delta}{[\tau'/t]\Gamma}{[\tau'/t]e}{[\tau'/t]\tau}$ or $\ruleType{\Delta}{[\tau'/t]\Gamma}{[\tau'/t]r}{[\tau'/t]\tau}{[\tau'/t]\tau''}$ does not depend on the form of $\tau'$.
\item By mutual rule induction over Rules (\ref{rules:hastypeUP}) and Rule (\ref{rule:ruleType}) and case analysis over the definition of substitution. In all cases, the derivation of $\hastypeU{\Delta}{\Gamma}{[e'/x]e}{\tau}$ or $\ruleType{\Delta}{\Gamma}{[e'/x]r}{\tau}{\tau''}$ does not depend on the form of $e'$.
\end{enumerate}
\end{proof-sketch}

The Pattern Regularity Lemma establishes that the hypotheses generated by checking a pattern against a well-formed type involve only well-formed types.
\begin{lemma}[Pattern Regularity]\label{lemma:pattern-regularity-UP} 
If $\patType{\pctx}{p}{\tau}$ and $\istypeU{\Delta}{\tau}$ then $\istypeU{\Delta}{\tau_i}$ for each assumption $x_i : \tau_i$ in $\pctx$.
\end{lemma}
\begin{proof} By rule induction over Rules (\ref{rules:patType}).
\begin{byCases}
\item[\text{(\ref{rule:patType-var})}] We have:
\begin{pfsteps*}
  \item $p=x$ \BY{assumption}
  \item $\pctx=x : \tau$ \BY{assumption}
  \item $\istypeU{\Delta}{\tau}$ \BY{assumption}\pflabel{istypeU}
 \end{pfsteps*}
 \resetpfcounter
\item[\text{(\ref{rule:patType-wild})}] We have $\pctx=\emptyset$ by assumption, so the conclusion trivially holds.
\item[\text{(\ref{rule:patType-tpl})}] We have:
\begin{pfsteps*}
  \item $p=\aetplp{\labelset}{\mapschema{p}{i}{\labelset}}$ \BY{assumption}
  \item $\tau=\aprod{\labelset}{\mapschema{\tau}{i}{\labelset}}$ \BY{assumption}
  \item $\patType{\pctx_i}{p_i}{\tau_i}$ for each $i \in \labelset$ \BY{assumption}\pflabel{patType}
  \item $\pctx=\cup_{i \in \labelset} \pctx_i$ \BY{assumption}
  \item $\istypeU{\Delta}{\aprod{\labelset}{\mapschema{\tau}{i}{\labelset}}}$ \BY{assumption} \pflabel{istypeU}
  \item $\istypeU{\Delta}{\tau_i}$ for each $i \in \labelset$ \BY{Inversion of Rule (\ref{rule:istypeU-prod}) on \pfref{istypeU}}\pflabel{istypeU-each}
  \item $\istypeU{\Delta}{\tau_{ij}}$ for each $x_{ij} : \tau_{ij}$ in $\pctx_i$, for each $i \in \labelset$ \BY{IH on \pfref{patType} and \pfref{istypeU-each} for each $i \in \labelset$} \pflabel{biggy}
  \item $\istypeU{\Delta}{\tau_{ij}}$ for each $x_{ij} : \tau_{ij}$ in $\cup_{i \in \labelset}\pctx_i$ \BY{the definition of pattern context union and \pfref{biggy}}
\end{pfsteps*}
\resetpfcounter
\item[\text{(\ref{rule:patType-inj})}] We have:
\begin{pfsteps*}
  \item $p=\aeinjp{\ell}{p'}$ \BY{assumption}
  \item $\tau=\asum{\labelset, \ell}{\mapschema{\tau}{i}{\labelset}; \mapitem{\ell}{\tau'}}$ \BY{assumption}
  \item $\istypeU{\Delta}{\asum{\labelset, \ell}{\mapschema{\tau}{i}{\labelset}; \mapitem{\ell}{\tau'}}}$ \BY{assumption} \pflabel{istype}
  \item $\patType{\pctx}{p'}{\tau'}$ \BY{assumption} \pflabel{patType}
  \item $\istypeU{\Delta}{\tau'}$ \BY{Inversion of Rule (\ref{rule:istypeU-sum}) on \pfref{istype}} \pflabel{istypeTwo} 
  \item $\istypeU{\Delta}{\tau_i}$ for each assumption $x : \tau_i$ in $\pctx$ \BY{IH on \pfref{patType} and \pfref{istypeTwo}}
\end{pfsteps*}
\end{byCases}
\end{proof}

Finally, the Regularity Lemma establishes that the type assigned to an expression under a well-formed typing context is well-formed. 
\begin{lemma}[Regularity]\label{lemma:regularity-UP} All of the following hold:
\begin{enumerate}
\item If $\hastypeU{\Delta}{\Gamma}{e}{\tau}$ and $\istypeU{\Delta}{\tau_i}$ for each assumption $x_i : \tau_i$ in $\Gamma$ then $\istypeU{\Delta}{\tau}$.
\item If $\ruleType{\Delta}{\Gamma}{r}{\tau}{\tau'}$ and $\istypeU{\Delta}{\tau}$ and $\istypeU{\Delta}{\tau_i}$ for each assumption $x_i : \tau_i$ in $\Gamma$ then $\istypeU{\Delta}{\tau'}$.
\end{enumerate}
\end{lemma}
\begin{proof-sketch} By mutual rule induction over Rules (\ref{rules:hastypeUP}) and Rule (\ref{rule:ruleType}), and Lemma \ref{lemma:substitution-UP} and Lemma \ref{lemma:pattern-regularity-UP}.
\end{proof-sketch}
\subsection{Structural Dynamics}\label{sec:dynamics-UP}
The \emph{structural dynamics of }$\miniVersePat$ is specified as a transition system, and is organized around judgements of the following form:
\[\begin{array}{ll}
\textbf{Judgement Form} & \textbf{Description}\\
\stepsU{e}{e'} & \text{$e$ transitions to $e'$}\\
\isvalU{e} & \text{$e$ is a value}\\
\matchfail{e} & \text{$e$ raises match failure}
\end{array}\]
We also define auxiliary judgements for \emph{iterated transition}, $\multistepU{e}{e'}$, and \emph{evaluation}, $\evalU{e}{e'}$.

\begin{definition}[Iterated Transition]\label{defn:iterated-transition-UP} Iterated transition, $\multistepU{e}{e'}$, is the reflexive, transitive closure of the transition judgement, $\stepsU{e}{e'}$.\end{definition}

\begin{definition}[Evaluation]\label{defn:evaluation-UP}  $\evalU{e}{e'}$ iff $\multistepU{e}{e'}$ and $\isvalU{e'}$. \end{definition}

As in Sec. \ref{sec:dynamics-U}, our subsequent developments do not make mention of particular rules in the dynamics, nor do they make mention of judgements that are used only for defining the dynamics of the match operator, so we do not provide these details here. Instead, it suffices to state the following conditions.

The Canonical Forms condition characterizes well-typed values. Satisfying this condition requires an \emph{eager} (i.e. \emph{by-value}) formulation of the dynamics. This condition is identical to Condition \ref{condition:canonical-forms-U}.

\begin{condition}[Canonical Forms]\label{condition:canonical-forms-UP} If $\hastypeUC{e}{\tau}$ and $\isvalU{e}$ then:
\begin{enumerate}
\item If $\tau=\aparr{\tau_1}{\tau_2}$ then $e=\aelam{\tau_1}{x}{e'}$ and $\hastypeUCO{\Ghyp{x}{\tau_1}}{e'}{\tau_2}$.
\item If $\tau=\aall{t}{\tau'}$ then $e=\aetlam{t}{e'}$ and $\hastypeUCO{\Dhyp{t}}{e'}{\tau'}$.
\item If $\tau=\arec{t}{\tau'}$ then $e=\aefold{t}{\tau'}{e'}$ and $\hastypeUC{e'}{[\abop{rec}{t.\tau'}/t]\tau'}$ and $\isvalU{e'}$. 
\item If $\tau=\aprod{\labelset}{\mapschema{\tau}{i}{\labelset}}$ then $e=\aetpl{\labelset}{\mapschema{e}{i}{\labelset}}$ and $\hastypeUC{e_i}{\tau_i}$ and $\isvalU{e_i}$ for each $i \in \labelset$.
\item If $\tau=\asum{\labelset}{\mapschema{\tau}{i}{\labelset}}$ then for some label set $L'$ and label $\ell$, we have that $\labelset=\labelset', \ell$ and $\tau=\asum{\labelset', \ell}{\mapschema{\tau}{i}{\labelset'}; \mapitem{\ell}{\tau_\ell}}$ and $e=\aein{\labelset', \ell}{\ell}{\mapschema{\tau}{i}{\labelset'}; \ell \hookrightarrow \tau_\ell}{e'}$ and $\hastypeUC{e'}{\tau_\ell}$ and $\isvalU{e'}$.\end{enumerate}\end{condition}

The Preservation condition ensures that evaluation preserves typing.
\begin{condition}[Preservation]\label{condition:preservation-UP} If $\hastypeUC{e}{\tau}$ and $\stepsU{e}{e'}$ then $\hastypeUC{e'}{\tau}$. \end{condition}
The Progress condition ensures that evaluation of a well-typed expression preserves typing and cannot ``get stuck''.
\begin{condition}[Progress]\label{condition:progress-UP} If $\hastypeUC{e}{\tau}$ then either $\isvalU{e}$ or $\matchfail{e}$ or there exists an $e'$ such that $\stepsU{e}{e'}$. \end{condition}
 
The Preservation and Progress conditions together establish type safety.
%\noindent
%Condition \ref{condition:preservation-UP} is identical to Condition \ref{condition:preservation-U}, while Condition \ref{condition:progress-UP} modifies Condition \ref{condition:progress-U} to allow for match failure. 

We do not define exhaustiveness and redundancy properties here, because these can be checked post-expansion and so are also not relevant to our subsequent developments (but see \cite{pfple1} for a formal account).

\begin{figure}
\hspace{-8px}$\arraycolsep=4pt\begin{array}{lllllll}
\textbf{Sort} & & & \textbf{Operational Form} & \textbf{Stylized Form} & \textbf{Description}\\
\mathsf{UExp} & \ue & ::= & x & x & \text{variable}\\
&&& \aulam{\tau}{x}{\ue} & \lam{x}{\tau}{\ue} & \text{abstraction}\\
&&& \auap{\ue}{\ue} & \ap{\ue}{\ue} & \text{application}\\
&&& \autlam{t}{\ue} & \Lam{t}{\ue} & \text{type abstraction}\\
&&& \autap{\ue}{\tau} & \App{\ue}{\tau} & \text{type application}\\
&&& \aufold{t}{\tau}{\ue} & \fold{\ue} & \text{fold}\\
&&& \auunfold{\ue} & \unfold{\ue} & \text{unfold}\\
&&& \autpl{\labelset}{\mapschema{\ue}{i}{\labelset}} & \tpl{\mapschema{\ue}{i}{\labelset}} & \text{labeled tuple}\\
&&& \aupr{\ell}{\ue} & \prj{\ue}{\ell} & \text{projection}\\
&&& \auin{\labelset}{\ell}{\mapschema{\tau}{i}{\labelset}}{\ue} & \inj{\ell}{\ue} & \text{injection}\\
&&& \aumatchwith{n}{\tau}{\ue}{\seqschemaX{\urv}} & \matchwith{\ue}{\seqschemaX{\urv}} & \text{match}\\
\LCC &&& \gray & \gray & \gray \\
&&& \audefuetsm{\tau}{e}{\tsmv}{\ue} & \texttt{syntax}~a~\texttt{at}~\tau~\texttt{for} & \text{ueTSM definition}\\
&&&                                    & \texttt{expressions}~\{e\}~\texttt{in}~\ue\\
&&& \autsmap{b}{\tsmv} & \utsmap{\tsmv}{b} & \text{ueTSM application}\\\ECC
\LCC &&& \lightgray & \lightgray & \lightgray\\
&&& \audefuptsm{\tau}{e}{\tsmv}{\ue} & \texttt{syntax}~a~\texttt{at}~\tau~\texttt{for} & \text{upTSM definition}\\
&&&                                    & \texttt{patterns}~\{e\}~\texttt{in}~\ue\\\ECC
\mathsf{URule} & \urv & ::= & \aumatchrule{n}{\upv}{\seqschemaX{x}}{\ue} & \matchrule{\upv}{\ue} & \text{match rule}\\
\mathsf{UPat} & \upv & ::= & x & x & \text{variable pattern}\\
&&& \auwildp & \wildp & \text{wildcard pattern}\\
&&& \autplp{\labelset}{\mapschema{\upv}{i}{\labelset}} & \tplp{\mapschema{\upv}{i}{\labelset}} & \text{labeled tuple pattern}\\
&&& \auinjp{\ell}{\upv} & \injp{\ell}{\upv} & \text{injection pattern}\\
\LCC &&& \lightgray & \lightgray & \lightgray\\
&&& \auapuptsm{b}{\tsmv} & \utsmap{\tsmv}{b} & \text{upTSM application}\ECC
\end{array}$
\caption[Syntax of unexpanded expressions, rules and patterns in $\miniVersePat$]{Abstract syntax of unexpanded expressions, rules and patterns in $\miniVersePat$. Metavariable $\tsmv$ ranges over TSM names and $b$ ranges over literal bodies. Literal bodies might contain unparsed terms, so variable renaming and substitution cannot be defined in the usual manner over unexpanded terms.}
\label{fig:UP-unexpanded-terms}
\end{figure}

\subsection{Syntax of the Outer Surface}
Programs ultimately evaluate as expanded expressions, but programmers do not write expanded terms directly. Instead, the \emph{outer surface of} $\miniVersePat$ consists of types, defined above, and \emph{unexpanded expressions}, $\ue$, \emph{unexpanded rules}, $\urv$, and \emph{unexpanded patterns}, $\upv$ (collectively, \emph{unexpanded terms}). The syntax of unexpanded terms is specified in Figure \ref{fig:UP-unexpanded-terms}. 
Notice that each expanded term form corresponds to an unexpanded term form. We refer to these as the \emph{shared forms}. In addition, four unexpanded term forms do not correspond to expanded term forms: the ueTSM definition form, the ueTSM application form, the upTSM definition form and the upTSM application form. The forms related to ueTSMs are highlighted in dark gray, and the forms related to upTSMs are highlighted in light gray.

\subsection{Typed Expansion}
Unexpanded terms are typed and expanded according to the \emph{typed expansion judgements}:
\[\begin{array}{ll}
\textbf{Judgement Form} & \textbf{Description}\\
\expandsUP{\Delta}{\Gamma}{\uSigma}{\Phi}{\ue}{e}{\tau} & \text{$\ue$ has expansion $e$ and type $\tau$ under ueTSM context $\uSigma$}\\
& \text{and upTSM context $\Phi$ assuming $\Delta$ and $\Gamma$}\\
\ruleExpands{\Delta}{\Gamma}{\uSigma}{\Phi}{\urv}{r}{\tau}{\tau'} & \text{$\urv$ has expansion $r$ and takes values of type $\tau$ to values of}\\
& \text{type $\tau'$ under TSM environments $\uSigma$ and $\Phi$ assuming $\Delta$ and $\Gamma$}\\
\patExpands{\pctx}{\Phi}{\upv}{p}{\tau} & \text{$\upv$ has expansion $p$ and type $\tau$ and generates hypotheses $\pctx$ }\\
& \text{under upTSM context $\Phi$}
\end{array}\]

The \emph{typed expression expansion} judgement, $\expandsUP{\Delta}{\Gamma}{\uSigma}{\Phi}{\ue}{e}{\tau}$, and the \emph{typed rule expansion judgement}, $\ruleExpands{\Delta}{\Gamma}{\uSigma}{\Phi}{\urv}{r}{\tau}{\tau'}$ are defined mutually inductively by Rules (\ref*{rules:expandsUP}) and Rule (\ref*{rule:ruleExpands}), respectively, and the \emph{typed pattern expansion judgement}, $\patExpands{\pctx}{\Phi}{\upv}{p}{\tau}$, is inductively defined by Rules (\ref*{rules:patExpands}) as follows.

\paragraph{Shared Forms} Rules (\ref*{rule:expandsUP-var}) through (\ref*{rule:expandsUP-match}) define typed expansion of  unexpanded expressions of shared form. The first three of these rules are shown below:
\begin{subequations}\label{rules:expandsUP}
\begin{equation}\label{rule:expandsUP-var}
  \inferrule{ }{\expandsUP{\Delta}{\Gamma, x : \tau}{\uSigma}{\Phi}{x}{x}{\tau}}
\end{equation}
\begin{equation}\label{rule:expandsUP-lam}
  \inferrule{
    \istypeU{\Delta}{\tau}\\
    \expandsUP{\Delta}{\Gamma, x : \tau}{\uSigma}{\Phi}{\ue}{e}{\tau'}
  }{\expandsUPX{\aulam{\tau}{x}{\ue}}{\aelam{\tau}{x}{e}}{\aparr{\tau}{\tau'}}}
\end{equation}
\begin{equation}\label{rule:expandsUP-ap}
  \inferrule{
    \expandsUPX{\ue_1}{e_1}{\aparr{\tau}{\tau'}}\\
    \expandsUPX{\ue_2}{e_2}{\tau}
  }{
    \expandsUPX{\auap{\ue_1}{\ue_2}}{\aeap{e_1}{e_2}}{\tau'}
  }
\end{equation}
% \begin{equation}\label{rule:expandsUP-tlam}
%   \inferrule{
%     \expandsUP{\Delta, \Dhyp{t}}{\Gamma}{\uSigma}{\Phi}{\ue}{e}{\tau}
%   }{
%     \expandsUPX{\autlam{t}{\ue}}{\aetlam{t}{e}}{\aall{t}{\tau}}
%   }
% \end{equation}
% \begin{equation}\label{rule:expandsUP-tap}
%   \inferrule{
%     \expandsUPX{\ue}{e}{\aall{t}{\tau}}\\
%     \istypeU{\Delta}{\tau'}
%   }{
%     \expandsUPX{\autap{\ue}{\tau'}}{\aetap{e}{\tau'}}{[\tau'/t]\tau}
%   }
% \end{equation}
% \begin{equation}\label{rule:expandsUP-fold}
%   \inferrule{
%     \istypeU{\Delta, \Dhyp{t}}{\tau}\\
%     \expandsUPX{\ue}{e}{[\arec{t}{\tau}/t]\tau}
%   }{
%     \expandsUPX{\aufold{t}{\tau}{\ue}}{\aefold{t}{\tau}{e}}{\arec{t}{\tau}}
%   }
% \end{equation}
% \begin{equation}\label{rule:expandsUP-unfold}
%   \inferrule{
%     \expandsUPX{\ue}{e}{\arec{t}{\tau}}
%   }{
%     \expandsUPX{\auunfold{\ue}}{\aeunfold{e}}{[\arec{t}{\tau}/t]\tau}
%   }
% \end{equation}
% \begin{equation}\label{rule:expandsUP-tpl}
%   \inferrule{
%     \{\expandsUPX{\ue_i}{e_i}{\tau_i}\}_{i \in \labelset}
%   }{
%     \expandsUPX{\autpl{\labelset}{\mapschema{\ue}{i}{\labelset}}}{\aetpl{\labelset}{\mapschema{e}{i}{\labelset}}}{\aprod{\labelset}{\mapschema{\tau}{i}{\labelset}}}
%   }
% \end{equation}
% \begin{equation}\label{rule:expandsUP-pr}
%   \inferrule{
%     \expandsUPX{\ue}{e}{\aprod{\labelset, \ell}{\mapschema{\tau}{i}{\labelset}; \mapitem{\ell}{\tau}}}
%   }{
%     \expandsUPX{\aupr{\ell}{\ue}}{\aepr{\ell}{e}}{\tau}
%   }
% \end{equation}
% \begin{equation}\label{rule:expandsUP-in}
%   \inferrule{
%     \{\istypeU{\Delta}{\tau_i}\}_{i \in \labelset}\\
%     \istypeU{\Delta}{\tau}\\
%     \expandsUPX{\ue}{e}{\tau}
%   }{
%     \left\{\shortstack{$\Delta~\Gamma \vdash_{\uSigma;\,\Phi} \auin{\labelset, \ell}{\ell}{\mapschema{\tau}{i}{\labelset}; \mapitem{\ell}{\tau}}{\ue}$\\$\leadsto$\\$\aein{\labelset, \ell}{\ell}{\mapschema{\tau}{i}{\labelset}; \mapitem{\ell}{\tau}}{e} : \asum{\labelset, \ell}{\mapschema{\tau}{i}{\labelset}; \mapitem{\ell}{\tau}}$\vspace{-1.2em}}\right\}
%   }
% \end{equation}
Observe that, in each of these rules, the unexpanded and expanded expression forms in the conclusion correspond, and the premises correspond to those of the typing rule for the expanded expression form, i.e. Rules (\ref{rule:hastypeUP-var}), (\ref{rule:hastypeUP-lam}) and (\ref{rule:hastypeUP-ap}), respectively. In particular, the type formation premises correspond directly, and the typed expansion premises correspond to the typing premises. The ueTSM context, $\uSigma$, and upTSM context, $\Phi$, pass opaquely through these rules.
\refstepcounter{equation}
\label{rule:expandsUP-tlam}
\refstepcounter{equation}
\label{rule:expandsUP-tap}
\refstepcounter{equation}
\label{rule:expandsUP-fold}
\refstepcounter{equation}
\label{rule:expandsUP-unfold}
\refstepcounter{equation}
\label{rule:expandsUP-tpl}
\refstepcounter{equation}
\label{rule:expandsUP-pr}
\refstepcounter{equation}
\label{rule:expandsUP-in}

Rule (\ref*{rule:expandsUP-match}), below, defines typed expansion of unexpanded match expressions and corresponds analagously to Rule (\ref{rule:hastypeUP-match}).% The typed rule expansion premise corresponds to the rule typing premise of Rule (\ref{rule:ruleType}).
\begin{equation}\label{rule:expandsUP-match}
\inferrule{
  \expandsUPX{\ue}{e}{\tau}\\
  \istypeU{\Delta}{\tau'}\\
  \{\ruleExpands{\Delta}{\Gamma}{\uSigma}{\Phi}{\urv_i}{r_i}{\tau}{\tau'}\}_{1 \leq i \leq n}
}{\expandsUPX{\aumatchwith{n}{\tau'}{\ue}{\seqschemaX{\urv}}}{\aematchwith{n}{\tau'}{e}{\seqschemaX{r}}}{\tau'}}
\end{equation}  


We can express this scheme more precisely with the following rule transformation. For each rule in Rules (\ref{rules:hastypeUP}),
\begin{mathpar}
%\refstepcounter{equation}
%\label{rule:expandsU-case}
\inferrule{J_1\\ \cdots \\ J_k}{J}
\end{mathpar}
the corresponding typed expansion rule is 
\begin{mathpar}
\inferrule{
  \Uof{J_1} \\
  \cdots\\
  \Uof{J_k}
}{
  \Uof{J}
}
\end{mathpar}
where
\[\begin{split}
\Uof{\istypeU{\Delta}{\tau}} & = \istypeU{\Delta}{\tau} \\
\Uof{\hastypeU{\Gamma}{\Delta}{e}{\tau}} & = \expandsUP{\Gamma}{\Delta}{\uSigma}{\Phi}{\Uof{e}}{e}{\tau}\\
\Uof{\ruleType{\Gamma}{\Delta}{r}{\tau}{\tau'}} & = \ruleExpands{\Gamma}{\Delta}{\uSigma}{\Phi}{\Uof{r}}{r}{\tau}{\tau'}\\
\Uof{\{J_i\}_{i \in \labelset}} & = \{\Uof{J_i}\}_{i \in \labelset}
\end{split}\]
and where $\Uof{e}$, when $e$ is a metapattern of sort $\mathsf{Exp}$, is a metapattern of sort $\mathsf{UExp}$ defined as follows:
\begin{itemize}
\item When $e$ is of definite form, $\Uof{e}$ is defined as follows:
\begin{align*}
\Uof{x} & = x\\
\Uof{\aelam{\tau}{x}{e}} & = \aulam{\tau}{x}{\Uof{e}}\\
\Uof{\aeap{e_1}{e_2}} & = \auap{\Uof{e_1}}{\Uof{e_2}}\\
\Uof{\aetlam{t}{e}} & = \autlam{t}{\Uof{e}}\\
\Uof{\aetap{e}{\tau}} & = \autap{\Uof{e}}{\tau}\\
\Uof{\aefold{t}{\tau}{e}} & = \aufold{t}{\tau}{e}\\
\Uof{\aeunfold{e}} & = \auunfold{\Uof{e}}\\
\Uof{\aetpl{\labelset}{\mapschema{e}{i}{\labelset}}} & = \autpl{\labelset}{\mapschemax{\Uofv}{e}{i}{\labelset}}\\
\Uof{\aein{\labelset}{\ell}{\mapschema{\tau}{i}{\labelset}}{e}} &= \auin{\labelset}{\ell}{\mapschema{\tau}{i}{\labelset}}{\Uof{e}}\\
\Uof{\aematchwith{n}{\tau}{e}{\seqschemaX{r}}} &= \aumatchwith{n}{\tau}{\Uof{e}}{\seqschemaXx{\Uofv}{r}}
\end{align*}
\item When $e$ is of indefinite form, $\Uof{e}$ is a uniquely corresponding metapattern of indefinite form. %For example, in Rule (\ref{rule:hastypeUP-ap}), $e_1$ and $e_2$ are of indefinite form, i.e. they match arbitrary expanded expressions. The rule transformation simply ``hats'' them, i.e. $\Uof{e_1}=\ue_1$ and $\Uof{e_2}=\ue_2$.
\end{itemize}
and where $\Uof{r}$, when $r$ is a metapattern of sort $\mathsf{ERule}$ of indefinite form, is a uniquely corresponding  metapattern of sort $\mathsf{URule}$ of indefinite form. 

It is instructive to use this rule transformation to generate Rules (\ref{rule:expandsUP-var}) through (\ref{rule:expandsUP-ap}) and Rule (\ref{rule:expandsUP-match}) above. We omit the remaining rules generated by this transformation, i.e. Rules (\ref*{rule:expandsUP-tlam}) through (\ref*{rule:expandsUP-in}). 

The typed rule expansion judgement is defined by Rule (\ref*{rule:ruleExpands}), below.
\end{subequations}
\begin{equation}\label{rule:ruleExpands}
\inferrule{
  \patExpands{\pctx}{\Phi}{\upv}{p}{\tau}\\
  \domof{\pctx} = \seqschemaX{x}\\
  \expandsUP{\Delta}{\Gcons{\Gamma}{\pctx}}{\uSigma}{\Phi}{\ue}{e}{\tau'} 
}{
  \ruleExpands{\Delta}{\Gamma}{\uSigma}{\Phi}{\aumatchrule{n}{\upv}{\seqschemaX{x}}{\ue}}{\aematchrule{n}{p}{\seqschemaX{x}}{e}}{\tau}{\tau'}
}
\end{equation}
As in the typed expression expansion judgements, the unexpanded and expanded forms in the conclusion of the rule above correspond. The premises correspond to those of the rule typing rule, i.e. Rule (\ref{rule:ruleType}). In particular, the typed pattern expansion premise above corresponds to the pattern typing premise of Rule (\ref{rule:ruleType}), the second premise corresponds directly, and the typed expression expansion premise in the rule above corresponds to the typing premise of Rule (\ref{rule:ruleType}). 

The typed pattern expansion judgement for patterns of shared form is defined by the following rules.
\begin{subequations}[intermezzo]\label{rules:patExpands}
\begin{equation}\label{rule:patExpands-var}
\inferrule{ }{
  \patExpands{\Ghyp{x}{\tau}}{\Phi}{x}{x}{\tau}
}
\end{equation}
\begin{equation}\label{rule:patExpands-wild}
\inferrule{ }{
  \patExpands{\emptyset}{\Phi}{\auwildp}{\aewildp}{\tau}
}
\end{equation}
\begin{equation}\label{rule:patExpands-tpl}
\inferrule{
  \{\patExpands{\pctx_i}{\Phi}{\upv_i}{p_i}{\tau_i}\}_{i \in \labelset}\\
  \{\{\domof{\Omega_i} \cap \domof{\Omega_j} = \emptyset\}_{j \in \labelset \setminus i}\}_{i \in \labelset}
}{
  % \patExpands{\Gconsi{i \in \labelset}{\pctx_i}}{\Phi}{
  %   \autplp{\labelset}{\mapschema{\upv}{i}{\labelset}}
  % }{
  %   \aetplp{\labelset}{\mapschema{p}{i}{\labelset}}
  % }{
  %   \aprod{\labelset}{\mapschema{\tau}{i}{\labelset}}
  % } %{\autplp{\labelset}{\mapschema{\upv}{i}{\labelset}}}{\aetplp{\labelset}{\mapschema}{p}{i}{\labelset}}{...}
  \left(\shortstack{$\Gconsi{i \in \labelset}{\pctx_i} \vdash_\Phi \autplp{\labelset}{\mapschema{\upv}{i}{\labelset}}$\\$\leadsto$\\$\aetplp{\labelset}{\mapschema{p}{i}{\labelset}} : \aprod{\labelset}{\mapschema{\tau}{i}{\labelset}}$\vspace{-1.2em}}\right)
}
\end{equation}
\begin{equation}\label{rule:patExpands-in}
\inferrule{
  \patExpands{\pctx}{\Phi}{\upv}{p}{\tau}
}{
  \patExpands{\pctx}{\Phi}{\auinjp{\ell}{\upv}}{\aeinjp{\ell}{p}}{\asum{\labelset, \ell}{\mapschema{\tau}{i}{\labelset}; \mapitem{\ell}{\tau}}}
}
\end{equation}
\end{subequations}
Again, the unexpanded and expanded pattern forms in the conclusion correspond and the premises correspond to those of the corresponding pattern typing rule, i.e. Rules (\ref{rule:patType-var}) through (\ref{rule:patType-inj}), respectively. The upTSM context, $\Phi$, passes through these rules opaquely.
%By instead defining these rules by the rule transformation just described, we avoid having to list a number of rules that are individually uninteresting. Moreover, this approach makes our exposition somewhat robust to changes to the inner core (though not to changes to the judgement forms in the statics of the inner core).

\paragraph{ueTSM Definition and Application} Rules (\ref*{rule:expandsUP-syntax}) and (\ref*{rule:expandsUP-tsmap}) define typed expansion of ueTSM definitions and ueTSM application, respectively.  
\begin{subequations}[resume]
\begin{equation}\label{rule:expandsUP-syntax}
\inferrule{
  \istypeU{\Delta}{\tau}\\
  \hastypeU{\emptyset}{\emptyset}{\eparse}{\aparr{\tBody}{\tParseResultExp}}\\\\
  \expandsUP{\Delta}{\Gamma}{\uSigma, \xuetsmbnd{\tsmv}{\tau}{\eparse}}{\Phi}{\ue}{e}{\tau'}
}{
  \expandsUPX{\uesyntax{\tsmv}{\tau}{\eparse}{\ue}}{e}{\tau'}
}
\end{equation}
\begin{equation}\label{rule:expandsUP-tsmap}
\inferrule{
  \encodeBody{b}{\ebody}\\
  \evalU{\ap{\eparse}{\ebody}}{\inj{\lbltxt{Success}}{\ecand}}\\
  \decodeCondE{\ecand}{\ce}\\\\
  \cvalidE{\emptyset}{\emptyset}{\esceneUP{\Delta}{\Gamma}{\uSigma, \xuetsmbnd{\tsmv}{\tau}{\eparse}}{\Phi}{b}}{\ce}{e}{\tau}
}{
  \expandsUP{\Delta}{\Gamma}{\uSigma, \xuetsmbnd{\tsmv}{\tau}{\eparse}}{\Phi}{\utsmap{\tsmv}{b}}{e}{\tau}
}
\end{equation}
\end{subequations}
These rules are nearly identical to Rules (\ref{rule:expandsU-syntax}) and (\ref{rule:expandsU-tsmap}), respectively, differing only in that the upTSM context, $\Phi$, passes through them opaquely. The premises of these rules, and the following auxiliary definitions and conditions, can be understood as described in Sec. \ref{sec:U-uetsm-definition} and \ref{sec:U-uetsm-application}, respectively. 

The type $\tParseResultExp$ is defined as follows:
\[\tParseResultExp \triangleq [\mapitem{\lbltxt{Success}}{\tCEExp}, \mapitem{\lbltxt{ParseError}}{\prodt{}}]\]

The \emph{body encoding judgement} $\encodeBody{b}{\ebody}$ specifies a mapping from the literal body, $b$, to an expanded value, $\ebody$, of type $\tBody$. An inverse mapping is specified by the \emph{body decoding judgement} $\decodeBody{\ebody}{b}$.
\[\begin{array}{ll}
\textbf{Judgement Form} & \textbf{Description}\\
\encodeBody{b}{e} & \text{$b$ has encoding $e$}\\
\decodeBody{e}{b} & \text{$e$ has decoding $b$}
\end{array}\]
The following condition establishes an isomorphism between literal bodies and values of type $\tBody$.
\begin{condition}[Body Isomorphism] All of the following hold:
\begin{enumerate}
\item For every literal body $b$, we have that $\encodeBody{b}{\ebody}$ and $\hastypeUC{\ebody}{\tBody}$ and $\isvalU{\ebody}$.
\item If $\hastypeUC{\ebody}{\tBody}$ and $\isvalU{\ebody}$ then $\decodeBody{\ebody}{b}$ for some $b$.
\item If $\encodeBody{b}{\ebody}$ then $\decodeBody{\ebody}{b}$.
\item If $\hastypeUC{\ebody}{\tBody}$ and $\isvalU{\ebody}$ and $\decodeBody{\ebody}{b}$ then $\encodeBody{b}{\ebody}$. 
\item If $\encodeBody{b}{\ebody}$ and $\encodeBody{b}{\ebody'}$ then $\ebody = \ebody'$.
\item If $\hastypeUC{\ebody}{\tBody}$ and $\isvalU{\ebody}$ and $\decodeBody{\ebody}{b}$ and $\decodeBody{\ebody}{b'}$ then $b=b'$.
\end{enumerate}
\end{condition}

The \emph{candidate expansion expression decoding judgement}, $\decodeCondE{\ecand}{\ce}$, decodes $\ecand$ to produce a \emph{candidate expansion expression}, $\ce$  (pronounced ``grave $e$''). 
The inverse mapping is specified by the judgement $\encodeCondE{\ce}{\ecand}$. 
%The \emph{candidate expansion decoding judgement}, $\decodeCondE{e}{\ce}$, 
\[\begin{array}{ll}
\textbf{Judgement Form} & \textbf{Description}\\
\encodeCondE{\ce}{e} & \text{$\ce$ has encoding $e$}\\
\decodeCondE{e}{\ce} & \text{$e$ has decoding $\ce$}
\end{array}\]

The syntax of candidate expansion terms is defined in Figure \ref{fig:UP-candidate-terms}, and described in Sec. \ref{sec:ce-syntax-UP} below. The follow condition establishes an isomorphism between values of type $\tCEExp$ and candidate expansion expressions.
\begin{condition}[Candidate Expansion Expression Isomorphism] All of the following hold:
\begin{enumerate}
\item If $\hastypeUC{\ecand}{\tCEExp}$ and $\isvalU{\ecand}$ then $\decodeCondE{\ecand}{\ce}$ for some $\ce$.
\item For every $\ce$, we have $\encodeCondE{\ce}{\ecand}$ such that $\hastypeUC{\ecand}{\tCEExp}$ and $\isvalU{\ecand}$.
\item If $\hastypeUC{\ecand}{\tCEExp}$ and $\isvalU{\ecand}$ and $\decodeCondE{\ecand}{\ce}$ then $\encodeCondE{\ce}{\ecand}$.
\item If $\encodeCondE{\ce}{\ecand}$ then $\decodeCondE{\ecand}{\ce}$.
\item If $\hastypeUC{\ecand}{\tCEExp}$ and $\isvalU{\ecand}$ and $\decodeCondE{\ecand}{\ce}$ and $\decodeCondE{\ecand}{\ce'}$ then $\ce=\ce'$.
\item If $\encodeCondE{\ce}{\ecand}$ and $\encodeCondE{\ce}{\ecand'}$ then $\ecand=\ecand'$.
\end{enumerate}
\end{condition}


ueTSM contexts, $\uSigma$, are finite functions from TSM names, $\tsmv$, to {ueTSM definitions}, $\xuetsmdef{\tau}{\eparse}$, where $\tau$ is the ueTSM's {type annotation} and $\eparse$ is its {parse function}. The \emph{ueTSM context formation judgement}, $\uetsmenv{\Delta}{\uSigma}$, ensures that the type annotations in $\uSigma$ are well-formed assuming $\Delta$, and that the parse functions in $\uSigma$ are  of type $\aparr{\tBody}{\tParseResultExp}$.
\[\begin{array}{ll}
\textbf{Judgement Form} & \textbf{Description}\\
\uetsmenv{\Delta}{\uSigma} & \text{$\uSigma$ is well-formed assuming $\Delta$}\end{array}\]
This judgement is inductively defined by the following rules:
\begin{subequations}[intermezzo]\label{rules:uetsmenv-UP}
\begin{equation}\label{rule:uetsmenv-empty-UP}
\inferrule{ }{\uetsmenv{\Delta}{\emptyset}}
\end{equation}
\begin{equation}\label{rule:uetsmenv-ext-UP}
\inferrule{
  \uetsmenv{\Delta}{\uSigma}\\
  \istypeU{\Delta}{\tau}\\
  \hastypeU{\emptyset}{\emptyset}{\eparse}{\aparr{\tBody}{\tParseResultExp}}
}{
  \uetsmenv{\Delta}{\uSigma, \xuetsmbnd{\tsmv}{\tau}{\eparse}}
}
\end{equation}
\end{subequations}



\paragraph{upTSM Definition and Application}
Rules (\ref{rule:expandsUP-defuptsm}) and (\ref{rule:patExpands-apuptsm}) define upTSM definition and application, and are defined in the next two subsections, respectively.





\subsection{upTSM Definition}
The \emph{upTSM definition form}: 
\[\usyntaxup{\tsmv}{\tau}{\eparse}{\ue}\]
allows the programmer to introduce a upTSM named $\tsmv$ at type $\tau$ with parse function $\eparse$ into the upTSM context of $\ue$. The operational form corresponding to this stylized form is $\audefuptsm{\tau}{\eparse}{\tsmv}{\ue}$. Rule (\ref{rule:expandsUP-defuptsm}), defines typed expansion of upTSM definitions (for clarity, we use the stylized form in the conclusion of the rule):
\begin{subequations}[resume]
\begin{equation}\label{rule:expandsUP-defuptsm}
\inferrule{
  \istypeU{\Delta}{\tau}\\
  \hastypeU{\emptyset}{\emptyset}{\eparse}{\aparr{\tBody}{\tParseResultPat}}\\\\
  \expandsUP{\Delta}{\Gamma}{\uSigma}{\Phi, \xuptsmbnd{\tsmv}{\tau}{\eparse}}{\ue}{e}{\tau'} 
}{
  \expandsUP{\Delta}{\Gamma}{\uSigma}{\Phi}{\usyntaxup{\tsmv}{\tau}{\eparse}{\ue}}{e}{\tau'}
}
\end{equation}
\end{subequations}
This rule is similar to Rule (\ref{rule:expandsUP-syntax}), which governs ueTSM definitions. Its premises can be understood as follows, in order:
\begin{enumerate}
\item The first premise ensures that the type annotation is well-formed.
\item The second premise checks that the parse function, $\eparse$, is of type \[\aparr{\tBody}{\tParseResultPat}\] %to generate the \emph{expanded parse function}, $\eparse$. 
Parse functions are applied statically (i.e. during typed expansion), as we will discuss when describing ueTSM application below, and evaluation is defined only for closed expanded expressions, so the parse function must be closed. %Notice that this occurs under empty contexts, i.e. parse functions cannot refer to the surrounding bindings. 
%The parse function must be of type $\aparr{\tBody}{\tParseResultExp}$ where the type abbreviations $\tBody$ and $\tParseResultExp$ are defined as follows.

The type abbreviated $\tBody$ is characterized above. 

$\tParseResultPat$ abbreviates a labeled sum type that distinguishes successful parses from parse errors:%\footnote{In VerseML, the \li{ParseError} constructor of \li{ParseResult} required an error message and an error location, but we omit these in our formalization for simplicity}:
\[\tParseResultPat \triangleq [\mapitem{\lbltxt{Success}}{\tCEPat}, \mapitem{\lbltxt{ParseError}}{\prodt{}}]\] 

The type abbreviated $\tCEPat$ classifies encodings of \emph{candidate expansion patterns} (or \emph{ce-patterns}), $\cpv$ (pronounced ``grave $p$''). The syntax of ce-patterns will be described in Sec. \ref{sec:ce-syntax-UP}. The mapping from ce-patterns to values of type $\tCEPat$ is defined by the \emph{ce-pattern encoding judgement}, $\encodeCEPat{\cpv}{e}$. The inverse mapping is defined by the \emph{ce-pattern decoding judgement}, $\decodeCEPat{e}{\cpv}$.

\[\begin{array}{ll}
\textbf{Judgement Form} & \textbf{Description}\\
\encodeCEPat{\cpv}{e} & \text{$\cpv$ has encoding $e$}\\
\decodeCEPat{e}{\cpv} & \text{$e$ has decoding $\cpv$}
\end{array}\]

Again, rather than picking a particular definition of $\tCEPat$ and defining the judgements above inductively against it, we only state the following condition, which establishes an isomorphism between values of type $\tCEPat$ and ce-patterns.

\begin{condition}[Candidate Expansion Pattern Isomorphism] All of the following must hold:
\begin{enumerate}
\item If $\hastypeUC{\ecand}{\tCEPat}$ and $\isvalU{\ecand}$ then $\decodeCEPat{\ecand}{\cpv}$ for some $\cpv$.
\item For every $\cpv$, we have $\encodeCEPat{\cpv}{\ecand}$ such that $\hastypeUC{\ecand}{\tCEPat}$ and $\isvalU{\ecand}$.
\item If $\hastypeUC{\ecand}{\tCEPat}$ and $\isvalU{\ecand}$ and $\decodeCEPat{\ecand}{\cpv}$ then $\encodeCEPat{\cpv}{\ecand}$.
\item If $\encodeCEPat{\cpv}{\ecand}$ then $\decodeCEPat{\ecand}{\cpv}$.
\item If $\hastypeUC{\ecand}{\tCEPat}$ and $\isvalU{\ecand}$ and $\decodeCEPat{\ecand}{\cpv}$ and $\decodeCEPat{\ecand}{\cpv'}$ then $\cpv=\cpv'$.
\item If $\encodeCEPat{\cpv}{\ecand}$ and $\encodeCEPat{\cpv}{\ecand'}$ then $\ecand=\ecand'$.
\end{enumerate}
\end{condition}


\item The final premise of Rule (\ref{rule:expandsUP-defuptsm}) extends the upTSM context with the newly determined {upTSM definition}, and proceeds to assign a type, $\tau'$, and expansion, $e$, to $\ue$. The conclusion of Rule (\ref{rule:expandsUP-defuptsm}) assigns this type and expansion to the ueTSM definition as a whole.% i.e. TSMs define behavior that is relevant during typed expansion, but not during evaluation. 
\end{enumerate}
upTSM contexts, $\Phi$, are finite functions from TSM names, $\tsmv$, to \emph{upTSM definitions}, $\xuptsmdef{\tau}{\eparse}$, where $\tau$ is the upTSM's type annotation and $\eparse$ is the upTSM's parse function. The \emph{upTSM context formation judgement}, $\uptsmenv{\Delta}{\Phi}$, ensures that the type annotations in $\Phi$ are well-formed assuming $\Delta$ and the parse functions in $\Phi$ are of type $\aparr{\tBody}{\tParseResultPat}$.
\[\begin{array}{ll}
\textbf{Judgement Form} & \textbf{Description}\\
\uptsmenv{\Delta}{\Phi} & \text{upTSM context $\Phi$ is well-formed assuming $\Delta$}\end{array}\]
This judgement is inductively defined by the following rules:
\begin{subequations}\label{rules:uptsmenv-U}
\begin{equation}\label{rule:uptsmenv-empty}
\inferrule{ }{\uptsmenv{\Delta}{\emptyset}}
\end{equation}
\begin{equation}\label{rule:uptsmenv-ext}
\inferrule{
  \uptsmenv{\Delta}{\Phi}\\
  \istypeU{\Delta}{\tau}\\
  \hastypeU{\emptyset}{\emptyset}{\eparse}{\aparr{\tBody}{\tParseResultPat}}
}{
  \uptsmenv{\Delta}{\uSigma, \xuptsmbnd{\tsmv}{\tau}{\eparse}}
}
\end{equation}
\end{subequations}

\subsection{upTSM Application}\label{sec:uptsm-application}
The stylized unexpanded pattern form for applying a upTSM named $\tsmv$ to a literal form with literal body $b$ is:
\[
\utsmap{\tsmv}{b}
\] 
This stylized form is identical to the stylized form for ueTSM application, differing in that appears within the syntax of unexpanded patterns, $\upv$, rather than unexpanded expressions, $\ue$. %It uses forward slashes as delimiters, though stylized variants of any of the literal forms specified in Figure \ref{fig:literal-forms} would be straightforward to add to the syntax table in Figure \ref{fig:UP-unexpanded-terms} (we omit them for simplicity). 
The corresponding operational form is $\auapuptsm{b}{\tsmv}$.%, i.e. there is an operator $\texttt{uapuptsm}[b]$ for each literal body $b$ indexed by the TSM name $\tsmv$ and taking no arguments.

Rule (\ref{rule:patExpands-apuptsm}), below, governs upTSM application. 
\addtocounter{equation}{-3}
\begin{subequations}
\addtocounter{equation}{4}
\begin{equation}\label{rule:patExpands-apuptsm}
\inferrule{
  \encodeBody{b}{\ebody}\\
  \evalU{\ap{\eparse}{\ebody}}{\inj{\lbltxt{Success}}{\ecand}}\\
  \decodeCEPat{\ecand}{\cpv}\\\\
  \cvalidP{\pctx}{\pscene{\Phi, \xuptsmbnd{\tsmv}{\tau}{\eparse}}{b}}{\cpv}{p}{\tau}
}{
  \patExpands{\pctx}{\Phi, \xuptsmbnd{\tsmv}{\tau}{\eparse}}{\auapuptsm{b}{\tsmv}}{p}{\tau}
}
\end{equation}
\end{subequations}
\addtocounter{equation}{1}

\noindent
This rule is similar to Rule (\ref{rule:expandsUP-tsmap}), which governs ueTSM application. Its premises can be understood as follows, in order:
\begin{enumerate}
\item The first premise determines the encoding of the literal body, $\ebody$ (see above).
\item The second premise applies the parse function $\eparse$ to $\ebody$. If parsing succeeds, i.e. a value of the (stylized) form $\inj{\lbltxt{Success}}{\ecand}$ results from evaluation, then $\ecand$ will be a value of type $\tCEPat$ (assuming a well-formed upTSM context, by transitive application of Assumption \ref{condition:preservation-UP}). We call $\ecand$ the \emph{encoding of the candidate expansion}.
\item The third premise decodes the encoding of the candidate expansion to produce \emph{candidate expansion}, $\cpv$ (see above).
\item The final premise of Rule (\ref{rule:patExpands-apuptsm}) \emph{validates} the candidate expansion and simultaneously generates its final expansion, $p$, and its pattern typing context, $\pctx$. This is the topic of Sec. \ref{sec:ce-validation-UP}.
\end{enumerate}

\begin{figure}[p]
\hspace{-5px}$\begin{array}{lllllll}
\textbf{Sort} & & & \textbf{Operational Form} & \textbf{Stylized Form} & \textbf{Description}\\
\mathsf{CETyp} & \ctau & ::= & t & t & \text{variable}\\
&&& \aceparr{\ctau}{\ctau} & \parr{\ctau}{\ctau} & \text{partial function}\\
&&& \aceall{t}{\ctau} & \forallt{t}{\ctau} & \text{polymorphic}\\
&&& \acerec{t}{\ctau} & \rect{t}{\ctau} & \text{recursive}\\
&&& \aceprod{\labelset}{\mapschema{\ctau}{i}{\labelset}} & \prodt{\mapschema{\ctau}{i}{\labelset}} & \text{labeled product}\\
&&& \acesum{\labelset}{\mapschema{\ctau}{i}{\labelset}} & \sumt{\mapschema{\ctau}{i}{\labelset}} & \text{labeled sum}\\
\LCC &&& \gray & \gray & \gray\\
&&& \acesplicedt{m}{n} & \splicedt{m}{n} & \text{spliced}\\\ECC
\mathsf{CEExp} & \ce & ::= & x & x & \text{variable}\\
&&& \acelam{\ctau}{x}{\ce} & \lam{x}{\ctau}{\ce} & \text{abstraction}\\
&&& \aceap{\ce}{\ce} & \ap{\ce}{\ce} & \text{application}\\
&&& \acetlam{t}{\ce} & \Lam{t}{\ce} & \text{type abstraction}\\
&&& \acetap{\ce}{\ctau} & \App{\ce}{\ctau} & \text{type application}\\
&&& \acefold{t}{\ctau}{\ce} & \fold{\ce} & \text{fold}\\
&&& \aceunfold{\ce} & \unfold{\ce} & \text{unfold}\\
&&& \acetpl{\labelset}{\mapschema{\ce}{i}{\labelset}} & \tpl{\mapschema{\ce}{i}{\labelset}} & \text{labeled tuple}\\
&&& \acepr{\ell}{\ce} & \prj{\ce}{\ell} & \text{projection}\\
&&& \acein{\labelset}{\ell}{\mapschema{\ctau}{i}{\labelset}}{\ce} & \inj{\ell}{\ce} & \text{injection}\\
&&& \acematchwith{n}{\tau}{\ce}{\seqschemaX{\crv}} & \matchwith{\ce}{\seqschemaX{\crv}} & \text{match}\\
\LCC &&& \gray & \gray & \gray\\
&&& \acesplicede{m}{n} & \splicede{m}{n} & \text{spliced}\\\ECC
\mathsf{CERule} & \crv & ::= & \acematchrule{n}{p}{\seqschemaX{x}}{\ce} & \matchrule{p}{\ce} & \text{rule}\\
\mathsf{CEPat} & \cpv & ::= & \acewildp & \wildp & \text{wildcard pattern}\\
%&&& \aefoldp{p} & \foldp{p} & \text{fold pattern}\\
&&& \acetplp{\labelset}{\mapschema{\cpv}{i}{\labelset}} & \tplp{\mapschema{\cpv}{i}{\labelset}} & \text{labeled tuple pattern}\\
&&& \aceinjp{\ell}{\cpv} & \injp{\ell}{\cpv} & \text{injection pattern}\\
\LCC &&& \lightgray & \lightgray & \lightgray\\
&&& \acesplicedp{m}{n} & \splicedp{m}{n} & \text{spliced}\ECC
\end{array}$
\caption[Syntax of candidate expansion types and candidate expansion terms in $\miniVersePat$]{Abstract syntax of candidate expansion types and candidate expansion expressions, rules and patterns (collectively, candidate expansion terms) in $\miniVerseUE$. Candidate expansion types and terms are identified up to $\alpha$-equivalence.}
\label{fig:UP-candidate-terms}
\end{figure}

\subsection{Syntax of Candidate Expansions}\label{sec:ce-syntax-UP}
Figure \ref{fig:UP-candidate-terms} defines the syntax of candidate expansion types (or \emph{ce-types}), $\ctau$, candidate expansion expressions (or \emph{ce-expressions}), $\ce$, candidate expansion rules (or \emph{ce-rules}), $\crv$, and candidate expansion patterns (or \emph{ce-patterns}), $\cpv$. The syntax of ce-types is identical to that given in Figure \ref{fig:U-candidate-terms}, which was described in Sec. \ref{sec:ce-syntax-U}. 

Observe that for each expanded term form, except for the form for variable patterns, there is a corresponding ce-term form. We refer to these as the \emph{shared forms}. There are two other ce-term forms: a ce-expression form for \emph{references to spliced unexpanded expressions}, $\acesplicede{m}{n}$, highlighted in dark gray, and a ce-pattern form for \emph{references to spliced unexpanded patterns}, $\acesplicedp{m}{n}$, highlighted in light gray.

\subsection{Candidate Expansion Validation}\label{sec:ce-validation-UP}
The \emph{candidate expansion validation judgements} validate ce-types and ce-terms and simultaneously generate their final expansions.
\[\begin{array}{ll}
\textbf{Judgement Form} & \textbf{Description}\\
\cvalidT{\Delta}{\tscenev}{\ctau}{\tau} & \text{$\ctau$ is well-formed and has expansion $\tau$ assuming $\Delta$ and type}\\
& \text{splicing scene $\tscenev$}\\
\cvalidE{\Delta}{\Gamma}{\escenev}{\ce}{e}{\tau} & \text{$\ce$ has expansion $e$ and type $\tau$ assuming $\Delta$ and $\Gamma$ and expression}\\
& \text{splicing scene $\escenev$}\\
\cvalidR{\Delta}{\Gamma}{\escenev}{\crv}{r}{\tau}{\tau'} & \text{$\crv$ has expansion $r$ and takes values of type $\tau$ to values of type $\tau'$}\\
& \text{assuming $\Delta$ and $\Gamma$ and expression splicing scene $\escenev$}\\
\cvalidP{\pctx}{\pscenev}{\cpv}{p}{\tau} & \text{$\cpv$ expands to $p$ and matches values of type $\tau$ generating}\\
& \text{assumptions $\pctx$ assuming pattern splicing scene $\pscenev$}
\end{array}\]
\emph{Expression splicing scenes}, $\escenev$, are of the form $\esceneUP{\Delta}{\Gamma}{\uSigma}{\Phi}{b}$, \emph{type splicing scenes}, $\tscenev$, are of the form $\tsceneUP{\Delta}{b}$, and \emph{pattern splicing scenes}, $\pscenev$, are of the form $\pscene{\Phi}{b}$. Their purpose is to ``remember'', during candidate expansion validation, the contexts, TSM environments and literal bodies from the TSM application site (cf. Rules (\ref{rule:expandsUP-tsmap}) and (\ref{rule:patExpands-apuptsm})), because these are necessary to validate references to spliced types and terms. We write $\tsfrom{\escenev}$ for the type splicing scene constructed by dropping the typing context and TSM environments from $\escenev$:
\[\tsfrom{\esceneUP{\Delta}{\Gamma}{\uSigma}{\Phi}{b}} = \tsceneUP{\Delta}{b}\]

\subsubsection{Candidate Expansion Type Validation}
The \emph{candidate type validation judgement}, $\cvalidT{\Delta}{\tscenev}{\ctau}{\tau}$, is inductively defined by Rules (\ref{rules:cvalidT-U}), which were defined in Sec. \ref{sec:ce-validation-U}.

\subsubsection{Candidate Expansion Expression and Rule Validation}
\begin{subequations}\label{rules:cvalidE-UP}
The \emph{ce-expression validation judgement}, $\cvalidE{\Delta}{\Gamma}{\escenev}{\ce}{e}{\tau}$, and the \emph{ce-rule validation judgement}, $\cvalidR{\Delta}{\Gamma}{\escenev}{\crv}{r}{\tau}{\tau'}$, are defined mutually inductively with Rules (\ref{rules:expandsUP}) and Rule (\ref{rule:ruleExpands}) by Rules (\ref*{rules:cvalidE-UP}) and Rule (\ref*{rule:cvalidR-UP}), respectively, as follows.

Rules (\ref*{rules:cvalidE-UP}) define ce-expression validation and consist of the following rules:
\begin{itemize}
  \item \refstepcounter{equation}\label{rule:cvalidE-UP-var}
\refstepcounter{equation}\label{rule:cvalidE-UP-lam}
\refstepcounter{equation}\label{rule:cvalidE-UP-ap}
\refstepcounter{equation}\label{rule:cvalidE-UP-tlam}
\refstepcounter{equation}\label{rule:cvalidE-UP-tap}
\refstepcounter{equation}\label{rule:cvalidE-UP-fold}
\refstepcounter{equation}\label{rule:cvalidE-UP-unfold}
\refstepcounter{equation}\label{rule:cvalidE-UP-tpl}
\refstepcounter{equation}\label{rule:cvalidE-UP-prj}
\refstepcounter{equation}\label{rule:cvalidE-UP-in}
Rules defined identically to Rules (\ref{rule:cvalidE-U-var}) through (\ref{rule:cvalidE-U-in}). We will refer to these as Rules (\ref*{rule:cvalidE-UP-var}) through (\ref*{rule:cvalidE-UP-in}).
  \item The following rule for match ce-expressions:
  \begin{equation}\label{rule:cvalidE-UP-match}
\inferrule{
  \cvalidE{\Delta}{\Gamma}{\escenev}{\ce}{e}{\tau}\\
  \cvalidT{\Delta}{\tsfrom{\escenev}}{\ctau'}{\tau'}\\\\
  \{\cvalidR{\Delta}{\Gamma}{\escenev}{\crv_i}{r_i}{\tau}{\tau'}\}_{1 \leq i \leq n}
}{\cvalidE{\Delta}{\Gamma}{\escenev}{\acematchwith{n}{\ctau'}{\ce}{\seqschemaX{\crv}}}{\aematchwith{n}{\tau'}{e}{\seqschemaX{r}}}{\tau'}}
\end{equation}
\item The following rule for references to spliced unexpanded expressions, which can be understood as described in Sec. \ref{sec:ce-validation-U}.
\begin{equation}\label{rule:cvalidE-UP-splicede}
\inferrule{
  \parseUExp{\bsubseq{b}{m}{n}}{\ue}\\\\
  \expandsUP{\Delta_\text{app}}{\Gamma_\text{app}}{\uSigma}{\Phi_S}{\ue}{e}{\tau}\\
    \Delta \cap \Delta_\text{app} = \emptyset\\
  \domof{\Gamma} \cap \domof{\Gamma_\text{app}} = \emptyset\\
}{
  \cvalidE{\Delta}{\Gamma}{\esceneUP{\Delta_\text{app}}{\Gamma_\text{app}}{\uSigma}{\Phi_S}{b}}{\acesplicede{m}{n}}{e}{\tau}
}
\end{equation}
\end{itemize}

% \begin{equation}\label{rule:cvalidE-UP-var}
% \inferrule{ }{
%   \cvalidE{\Delta}{\Gamma, \Ghyp{x}{\tau}}{\escenev}{x}{x}{\tau}
% }
% \end{equation}
% \begin{equation}\label{rule:cvalidE-UP-lam}
% \inferrule{
%   \cvalidT{\Delta}{\tsfrom{\escenev}}{\ctau}{\tau}\\
%   \cvalidE{\Delta}{\Gamma, \Ghyp{x}{\tau}}{\escenev}{\ce}{e}{\tau'}
% }{
%   \cvalidE{\Delta}{\Gamma}{\escenev}{\acelam{\ctau}{x}{\ce}}{\aelam{\tau}{x}{e}}{\aparr{\tau}{\tau'}}
% }
% \end{equation}
% \begin{equation}\label{rule:cvalidE-UP-ap}
%   \inferrule{
%     \cvalidE{\Delta}{\Gamma}{\escenev}{\ce_1}{e_1}{\aparr{\tau}{\tau'}}\\
%     \cvalidE{\Delta}{\Gamma}{\escenev}{\ce_2}{e_2}{\tau}
%   }{
%     \cvalidE{\Delta}{\Gamma}{\escenev}{\aceap{\ce_1}{\ce_2}}{\aeap{e_1}{e_2}}{\tau'}
%   }
% \end{equation}
% \begin{equation}\label{rule:cvalidE-UP-tlam}
%   \inferrule{
%     \cvalidE{\Delta, \Dhyp{t}}{\Gamma}{\escenev}{\ce}{e}{\tau}
%   }{
%     \cvalidEX{\acetlam{t}{\ce}}{\aetlam{t}{e}}{\aall{t}{\tau}}
%   }
% \end{equation}
% \begin{equation}\label{rule:cvalidE-UP-tap}
%   \inferrule{
%     \cvalidEX{\ce}{e}{\aall{t}{\tau}}\\
%     \cvalidT{\Delta}{\tsfrom{\escenev}}{\ctau'}{\tau'}
%   }{
%     \cvalidEX{\acetap{\ce}{\ctau'}}{\aetap{e}{\tau'}}{[\tau'/t]\tau}
%   }
% \end{equation}
% \begin{equation}\label{rule:cvalidE-UP-fold}
%   \inferrule{
%     \cvalidT{\Delta, \Dhyp{t}}{\escenev}{\ctau}{\tau}\\
%     \cvalidEX{\ce}{e}{[\arec{t}{\tau}/t]\tau}
%   }{
%     \cvalidEX{\acefold{t}{\ctau}{\ce}}{\aefold{t}{\tau}{e}}{\arec{t}{\tau}}
%   }
% \end{equation}
% \begin{equation}\label{rule:cvalidE-UP-unfold}
%   \inferrule{
%     \cvalidEX{\ce}{e}{\arec{t}{\tau}}
%   }{
%     \cvalidEX{\aceunfold{\ce}}{\aeunfold{e}}{[\arec{t}{\tau}/t]\tau}
%   }
% \end{equation}
% \begin{equation}\label{rule:cvalidE-UP-tpl}
%   \inferrule{
%     \{\cvalidEX{\ce_i}{e_i}{\tau_i}\}_{i \in \labelset}
%   }{
%     \cvalidEX{\acetpl{\labelset}{\mapschema{\ce}{i}{\labelset}}}{\aetpl{\labelset}{\mapschema{e}{i}{\labelset}}}{\aprod{\labelset}{\mapschema{\tau}{i}{\labelset}}}
%   }
% \end{equation}
% \begin{equation}\label{rule:cvalidE-UP-pr}
%   \inferrule{
%     \cvalidEX{\ce}{e}{\aprod{\labelset, \ell}{\mapschema{\tau}{i}{\labelset}; \mapitem{\ell}{\tau}}}
%   }{
%     \cvalidEX{\acepr{\ell}{\ce}}{\aepr{\ell}{e}}{\tau}
%   }
% \end{equation}
% \begin{equation}\label{rule:cvalidE-UP-in}
%   \inferrule{
%     \{\cvalidT{\Delta}{\tsfrom{\escenev}}{\ctau_i}{\tau_i}\}_{i \in \labelset}\\
%     \cvalidT{\Delta}{\tsfrom{\escenev}}{\ctau}{\tau}\\
%     \cvalidEX{\ce}{e}{\tau}
%   }{
%     \left\{\shortstack{$\Delta~\Gamma \vdash_\uSigma \acein{\labelset, \ell}{\ell}{\mapschema{\ctau}{i}{\labelset}; \mapitem{\ell}{\ctau}}{\ce}$\\$\leadsto$\\$\aein{\labelset, \ell}{\ell}{\mapschema{\tau}{i}{\labelset}; \mapitem{\ell}{\tau}}{e} : \asum{\labelset, \ell}{\mapschema{\tau}{i}{\labelset}; \mapitem{\ell}{\tau}}$\vspace{-1.2em}}\right\}
%   }
% \end{equation}
% \begin{equation}\label{rule:cvalidE-UP-case}
%   \inferrule{
%     \cvalidEX{\ce}{e}{\asum{\labelset}{\mapschema{\tau}{i}{\labelset}}}\\
%     \{\cvalidE{\Delta}{\Gamma, \Ghyp{x_i}{\tau_i}}{\escenev}{\ue_i}{e_i}{\tau}\}_{i \in \labelset}
%   }{
%     \cvalidEX{\acecase{\labelset}{\tau}{\ce}{\mapschemab{x}{\ce}{i}{\labelset}}}{\aecase{\labelset}{\tau}{e}{\mapschemab{x}{e}{i}{\labelset}}}{\tau}
%   }
% \end{equation}
\end{subequations}
%The \emph{ce-rule validation judgement}, $\cvalidR{\Delta}{\Gamma}{\escenev}{\crv}{r}{\tau}{\tau'}$, is defined mutually inductively with Rules (\ref{rules:cvalidE-UP}) by 
Rule (\ref*{rule:cvalidR-UP}) defines ce-rule validation and is defined as follows:
\begin{equation}\label{rule:cvalidR-UP}
\inferrule{
  \patType{\pctx}{p}{\tau}\\
  \domof{\pctx} = \seqschemaX{x}\\
  \cvalidE{\Delta}{\Gcons{\Gamma}{\pctx}}{\escenev}{\ce}{e}{\tau'}
}{
  \cvalidR{\Delta}{\Gamma}{\escenev}{\acematchrule{n}{p}{\seqschemaX{x}}{\ce}}{\aematchrule{n}{p}{\seqschemaX{x}}{e}}{\tau}{\tau'}
}
\end{equation}
Notice that expanded patterns, $p$, not ce-patterns, $\cpv$, appear in ce-rules. This is because ce-expressions are generated only by ueTSMs. It would not be sensible for a ueTSM to extract a spliced subpattern from a literal body.

\subsubsection{Candidate Expansion Pattern Validation}
upTSMs generate candidate expansions of ce-pattern form, as described in Sec. \ref{sec:uptsm-application}. The \emph{ce-pattern validation judgement}, $\cvalidP{\pctx}{\pscenev}{\cpv}{p}{\tau}$, which appears as the final premise of Rule (\ref{rule:expandsUP-tsmap}), validates ce-patterns by checking that the pattern matches values of type $\tau$, and simultaneously generates the final expansion, $p$, and the hypotheses $\pctx$. Hypotheses can be generated only by spliced subpatterns, so there is no ce-pattern form corresponding to variable patterns (this is also why $\pctx$ does not appear to the left of the turnstile in the judgement form). The pattern splicing scene, $\pscenev$, is used to ``remember'' the upTSM context and literal body from the upTSM application site.

The ce-pattern validation judgement is defined mutually inductively with Rules (\ref{rules:patExpands}) by the following rules.

\begin{subequations}\label{rules:cvalidP-UP}
\begin{equation}\label{rule:cvalidP-UP-wild}
\inferrule{ }{
  \cvalidP{\emptyset}{\pscenev}{\acewildp}{\aewildp}{\tau}
}
\end{equation}
\begin{equation}\label{rule:cvalidP-UP-tpl}
\inferrule{
  \{\cvalidP{\pctx_i}{\pscenev}{\cpv_i}{p_i}{\tau_i}\}_{i \in \labelset}\\
  \{\{\domof{\Omega_i} \cap \domof{\Omega_j} = \emptyset\}_{j \in \labelset \setminus i}\}_{i \in \labelset}
}{
\left(\shortstack{$\vdash^{\Gconsi{i \in \labelset}{\pctx_i}; \pscenev} \acetplp{\labelset}{\mapschema{\cpv}{i}{\labelset}}$\\$\leadsto$\\$\aetplp{\labelset}{\mapschema{p}{i}{\labelset}} : \aprod{\labelset}{\mapschema{\tau}{i}{\labelset}}$\vspace{-1.2em}}\right)
}
\end{equation}
\begin{equation}\label{rule:cvalidP-UP-in}
\inferrule{
  \cvalidP{\pctx}{\pscenev}{\cpv}{p}{\tau}
}{
  \cvalidP{\pctx}{\pscenev}{\aceinjp{\ell}{\cpv}}{\aeinjp{\ell}{p}}{\asum{\labelset, \ell}{\mapschema{\tau}{i}{\labelset}; \mapitem{\ell}{\tau}}}
}
\end{equation}
\begin{equation}\label{rule:cvalidP-UP-spliced}
\inferrule{
  \parseUPat{\bsubseq{b}{m}{n}}{\upv}\\
  \patExpands{\pctx}{\Phi}{\upv}{p}{\tau}
}{
  \cvalidP{\pctx}{\pscene{\Phi}{b}}{\acesplicedp{m}{n}}{p}{\tau}
}
\end{equation}
\end{subequations}

Rules (\ref{rule:cvalidP-UP-wild}) through (\ref{rule:cvalidP-UP-in}) handle ce-patterns of shared form, and correspond to Rules (\ref{rule:patType-wild}) through (\ref{rule:patType-inj}). Rule (\ref{rule:cvalidP-UP-spliced}) handles references to spliced unexpanded patterns. The first premise parses the indicated subsequence of the literal body, $b$, to produce the referenced unexpanded pattern, $\upv$, and the second premise types and expands $\upv$ under the upTSM context $\Phi$ from the upTSM application site, producing the hypotheses $\pctx$. These are the hypotheses generated in the conclusion of the rule.

Notice that none of these rules explicitly add any hypotheses to the pattern typing context, so upTSMs cannot introduce any hypotheses other than those that come from such spliced subpatterns. This achieves the ``no hidden assumptions'' hygiene property described in Sec. \ref{sec:ptsms-hygiene}.

\subsection{Metatheory}
\begin{theorem}[Typed Pattern Expansion] Both of the following hold:
\begin{enumerate}
  \item If $\patExpands{\pctx}{\Phi}{\upv}{p}{\tau}$ and $\uptsmenv{\Delta}{\Phi}$ then $\patType{\pctx}{p}{\tau}$.
  \item If $\cvalidP{\pctx}{\pscene{\Phi}{b}}{\cpv}{p}{\tau}$ and $\uptsmenv{\Delta}{\Phi}$ then $\patType{\pctx}{p}{\tau}$.
\end{enumerate}
\end{theorem}
\begin{proof}
  By mutual rule induction on Rules (\ref{rules:patExpands}) and (\ref{rules:cvalidP-UP}).
\end{proof}

\begin{theorem}[Typed Expansion] All of the following hold:
\begin{enumerate}
  \item \begin{enumerate}
    \item If $\expandsUP{\Delta}{\Gamma}{\uSigma}{\Phi}{\ue}{e}{\tau}$ and $\uetsmenv{\Delta}{\uSigma}$ and $\uptsmenv{\Delta}{\Phi}$ then $\hastypeU{\Delta}{\Gamma}{e}{\tau}$.
    \item If $\ruleExpands{\Delta}{\Gamma}{\uSigma}{\Phi}{\urv}{r}{\tau}{\tau'}$ and $\uetsmenv{\Delta}{\uSigma}$ and $\uptsmenv{\Delta}{\Phi}$ then $\ruleType{\Delta}{\Gamma}{r}{\tau}{\tau'}$.
  \end{enumerate}
  \item \begin{enumerate}
    \item If $\cvalidE{\Delta}{\Gamma}{\esceneUP{\Delta_\text{app}}{\Gamma_\text{app}}{\uSigma}{\Phi_S}{b}}{\ce}{e}{\tau}$ and $\uetsmenv{\Delta_\text{app}}{\uSigma}$ and $\uptsmenv{\Delta_\text{app}}{\Phi_S}$ then $\hastypeU{\Dcons{\Delta}{\Delta_\text{app}}}{\Gcons{\Gamma}{\Gamma_\text{app}}}{e}{\tau}$. 
    \item If $\cvalidR{\Delta}{\Gamma}{\esceneUP{\Delta_\text{app}}{\Gamma_\text{app}}{\uSigma}{\Phi_S}{b}}{\crv}{r}{\tau}{\tau'}$ and $\uetsmenv{\Delta_\text{app}}{\uSigma}$ and $\uptsmenv{\Delta_\text{app}}{\Phi_S}$ then $\ruleType{\Dcons{\Delta}{\Delta_\text{app}}}{\Gcons{\Gamma}{\Gamma_\text{app}}}{r}{\tau}{\tau'}$.
  \end{enumerate}
\end{enumerate}
\end{theorem}
\begin{proof}
  By mutual rule induction on Rules (\ref{rules:expandsUP}), Rule (\ref{rule:ruleExpands}), Rules (\ref{rules:cvalidE-UP}) and Rule (\ref{rule:cvalidR-UP}).
\end{proof}


%\part[Parametric TLMs]{Parametric TLMs
% ~\\ ~\\ ~\\ 
% \begin{center}
%                      \begin{minipage}[l]{11cm}
%                        \textnormal{\normalsize
% }
%                      \end{minipage}
%                   \end{center}
% }\label{part:parametric-tsms}

% !TEX root = omar-thesis.tex

\chapter{Parametric TSMs (pTSMs)}\label{chap:ptsms}
% \begin{quote}\textit{The recent development of programming languages suggests that the simul\-taneous achievement of simplicity 
% and generality in language design is a serious unsolved 
% problem.}\begin{flushright}--- John Reynolds (1970) \cite{Reynolds70}\end{flushright}
% \end{quote}
% ~\\
This chapter introduces \emph{parametric TSMs} (pTSMs). Parametric TSMs can be defined over a type- and module-parameterized family of types, rather than just a single type, and the expansions that they generate can refer to  supplied type and module parameters. 

This chapter is organized like the preceding chapters. We begin in Sec. \ref{sec:parameterized-tsms-by-example} by introducing parametric TSMs by example in VerseML. In particular, we discuss type parameters in Sec. \ref{sec:type-parameters} and module parameters in Sec. \ref{sec:module-parameters}. We then develop a reduced calculus of parametric TSMs, $\miniVerseParam$, in Sec. \ref{sec:miniVerseP}.

\section{Parametric TSMs By Example}\label{sec:parameterized-tsms-by-example}

\subsection{Type Parameters}\label{sec:type-parameters}
Recall from Sec. \ref{sec:lists} the definition of the type-parameterized family of list types:
\begin{lstlisting}[numbers=none]
type list('a) = rec(self => Nil + Cons of 'a * self)
\end{lstlisting}

% \emph{Kinds} classify construction expressions, much like types classify expressions. Types are construction expressions of kind \li{T}, and type constructions are construction expressions of arrow kind. Here, \li{list} takes a single type parameter, so it has arrow kind \li{T -> T}.

% ML dialects commonly define derived syntactic forms for constructing and pattern matching over values of list type. VerseML, in contrast, does not build in derived list forms. Instead, 
Figure \ref{fig:petsm-list} defines a \emph{parametric expression TSM} (peTSM) and a \emph{parametric pattern TSM} (ppTSM), both named \li{#\dolla#list}. These TSMs operate uniformly over this family of types.
\begin{figure}[h]
\begin{lstlisting}
syntax $list('a) at list('a) for expressions by 
  static fn(b : body) : parse_result(proto_expr) => (* ... *)
and for patterns by 
  static fn(b : body) : parse_result(proto_pat) => (* ... *) 
end
\end{lstlisting}
\caption{The type-parameterized \texttt{\$list} TSMs.}
\label{fig:petsm-list}
\end{figure}

Line 1 specifies a single type parameter, \li{'a}. This type parameter appears in the type annotation, which establishes that:
\begin{enumerate}
\item The peTSM \li{#\dolla#list}, when applied to a type \li{T} and a generalized literal form, can only generate expansions of type \li{list(T)}.
\item The ppTSM \li{#\dolla#list}, when applied to a type \li{T} and a generalized literal form, can only generate expansions that match values of type \li{list(T)}.
\end{enumerate}
For example, we can apply \li{#\dolla#list} to \li{int} and a generalized literal form delimited by square brackets as follows:
% val y = $list int [3SURL, EURL4SURL, EURL5]
\begin{lstlisting}[numbers=none]
val x = $list int [xSURL, EURLySURL :: EURLxs]
\end{lstlisting}
The parse function (elided above for concision) segments the literal body into  spliced expressions. The trailing spliced expression is prefixed by two colons (\li{SURL::EURL}), which the TSM takes to mean that it should be the tail of the list. The final expansion of the example above is equivalent to the following when the list value constructors are in scope:
\begin{lstlisting}[numbers=none]
val x = Cons(x, Cons(y, xs))
\end{lstlisting}
As in the preceding chapters, the expansion itself must use the explicit \li{fold} and \li{inj} operators rather than the list value constructors \li{Cons} and \li{Nil} due to the prohibition on context dependence.

\subsection{Module Parameters}\label{sec:module-parameters}
We can finally address the inconvenience of needing to use explicit \li{fold} and \li{inj}  operators by  defining a module-parameterized TSM.

Recall that in Figure \ref{fig:LIST}, we defined a signature \li{LIST} that exported the definition of \li{list} and specified the list value constructors (and some other values.) The definition of \li{#\dolla#list'} shown in Figure \ref{fig:ptsm-listprime} takes modules matching this signature as an additional parameter.

\begin{figure}[h]
\begin{lstlisting}[numbers=none]
syntax $list' (L : LIST) 'a at 'a L.list for expressions by 
  static fn(b : body) : parse_result(proto_expr) => (* ... *)
for patterns by 
  static fn(b : body) : parse_result(proto_pat) => (* ... *)
end
\end{lstlisting}
\caption{The type- and module-parameterized \texttt{\$list'} TSMs.}
\label{fig:ptsm-listprime}
\end{figure}
% differing only in that any type parameter that the peTSM specifies can appear free in the generated expansion.

We can apply \li{#\dolla#list'} to the module \li{List} and the type \li{int} as follows:
\begin{lstlisting}[numbers=none]
val y = $list' List int [3SURL, EURL4SURL, EURL5]
val x = $list' List int [1SURL, EURL2SURL :: EURLy]
\end{lstlisting}
The expansion is:
\begin{lstlisting}[numbers=none]
val y = List.Cons(3, List.Cons(4, List.Cons(5, List.Nil)))
val x = List.Cons(1, List.Cons(2, y))
\end{lstlisting}
There is no need to use explicit \li{fold} and \li{inj} operators in this expansion, because the expansion projects the constructors out of the provided module parameter. The TSM itself did not assume that the module would be named \li{List} (internally, the proto-expansion refers to it as \li{L}.)

This makes matters simpler for the TSM provider, but there is a syntactic cost associated with supplying a module parameter at each TSM application site. To reduce this cost, VerseML supports partial parameter application in TSM abbreviations. For example, we can define \li{#\dolla#list} by partially applying \li{#\dolla#list'} as follows:
\begin{lstlisting}[numbers=none]
let syntax $list = $list' List
\end{lstlisting}
(This abbreviates both the expression and pattern TSMs -- sort qualifiers can be added to restrict the abbreviation if desired.)


% Similarly, in lieu of derived list pattern forms, we define the following \emph{parameterized pattern TSM} (ppTSM):
% \begin{lstlisting}[numbers=none]
% syntax $list('a) at list('a) for patterns {
%   static fn(body : Body) : ParseResult(CEPat) => (* ... *)
% }
% \end{lstlisting}
% Again, Line 1 names the ppTSM \li{#\dolla#list} and specifies a single type parameter, \li{'a}. This type parameter appears in the type annotation, which specifies that \li{#\dolla#list}, when apply to a type \li{'a} and a generalized literal form, will only generate patterns that match values of type \li{list('a)}. 

% For example, we can apply the ppTSM \li{#\dolla#list} and the \li{#\dolla#list} to define the polymorphic map function as follows.
% \begin{lstlisting}[numbers=none]
% fun map (f : 'a -> 'b) (x : list('a)) => match x { 
%   $list('a) [] => $list('b) []
% | $list('a) [hdSURL :: EURLtl] => $list('b) [f hdSURL :: EURLmap f tl]
% }
% \end{lstlisting}
% The expansion of this function definition, written textually, is:
% \begin{lstlisting}[numbers=none]
% fun map (f : 'a -> 'b) (x : list('a)) : 'b list => match x { 
%   Nil => Nil
% | Cons(hd, tl) => Cons(f hd, map f tl)
% }
% \end{lstlisting}
% This is somewhat unsatisfying, however, because the expansion is more concise than the unexpanded definition of \li{map}. To further reduce syntactic cost, we can designate \li{#\dolla#list} as the implicit TSM for both expressions and patterns at all types \li{'a list} around our definition of \li{map} as follows.
% \begin{lstlisting}[numbers=none]
% implicit syntax $list('a) in
%   fun map (f : 'a -> 'b) (x : 'a list) : 'b list => match x {
%     [] => []
%   | [hdSURL :: EURLtl] => [f hdSURL :: EURLmap f tl]
%   }
% end
% \end{lstlisting}
% By designating an implicit TSM, we no longer need to explicitly apply \li{#\dolla#list} within expressions in analytic position or patterns.

% When designating an implicit TSM, we assume that free type variables in the type annotation, e.g. here \li{'a}, range over all types. We can make this more explicit by specifying a type parameter explicitly as follows:
% \begin{lstlisting}[numbers=none]
% implicit syntax('a) $list('a) at list('a) in
% 	(* ... *)
% end
% \end{lstlisting}
% All type parameters must appear in the type annotation.
% \subsection{More Examples}
Module parameters also allow us to define TSMs that operate uniformly over module-parameterized families of abstract types. For example, the module-parameterized TSM \texttt{\$r} defined in Figure \ref{fig:param-tsm-r} supports the POSIX regex syntax for any type \li{R.t} where \li{R : RX}. 

\begin{figure}[h]
\begin{lstlisting}
syntax $r(R : RX) at R.t by 
  static fn(b : body) : parse_result(proto_expr) => (* ... *)
end
\end{lstlisting}
\vspace{-5px}
\caption{The module-parameterized TSM \texttt{\$r}.}
\label{fig:param-tsm-r}
\end{figure}
\noindent
For example, given \li{R1 : RX}, we can apply \li{#\dolla#r} as follows:
\begin{lstlisting}[numbers=none]
let dna = $r R1 /SURLA|T|G|CEURL/
\end{lstlisting}
The final expansion of this term is:
\begin{lstlisting}[numbers=none]
let dna = R1.Or(R1.Str "SSTRAESTR", R1.Or(R1.Str "SSTRTESTR", 
	        R1.Or(R1.Str "SSTRGESTR", R1.Str "SSTRCESTR")))
\end{lstlisting}

To be clear: parameters are available to the generated expansion, but they are not available to the parse function that generates the expansion. For example, the following TSM definition is not well-typed because it refers to \li{M} from within the parse function:
\begin{lstlisting}[numbers=none]
syntax $badM(M : A) at T by 
  static fn(b : body) => let x = M.x in (* ... *)
end
\end{lstlisting}
(In the next chapter, we will define a mechanism that gives parse functions access to a common static environment.)
% \subsubsection{Queues}
% Consider the following signature for working with persistent queues:
% \begin{lstlisting}[numbers=none]
% signature QUEUE = sig
%   type queue('a)
%   val empty  : queue('a)
%   val insert : 'a * queue('a) -> queue('a)
%   val remove : queue('a) -> option('a * queue('a))
% end 
% \end{lstlisting}
% Structures that match this signature must define a type constructor \li{queue} of kind \li{T -> T} and three values -- \li{empty} introduces the empty queue, \li{insert} inserts a value onto the back of a queue, and \li{remove} removes the element at the front of the queue and returns it and the remaining queue, or \li{None} if the queue is empty.%one for inserting an item into a queue, and one for removing a value from a queue.

% There are many possible structures that implement this signature. For example, we can define a structure \li{ListQueue} that represents queues internally as lists, where the head of the list is the back of the queue. With this representation, \li{insert} is a constant time operation, but \li{remove} is a linear time operation. Alternatively, we might define a structure \li{TwoListQueue} that represents queues internally as a pair of lists, maintaining the invariant that one is the reverse of the other, so that both \li{insert} and \li{remove} are constant time operations (see \cite{harper1997programming} for the details of this and other possibilities). 

% Regardless of the implementation that the client chooses, we would like for the client to be able to introduce queues more naturally and at lower syntactic cost than is possible by directly applying the functions specified by the signature above. In VerseML, we can give clients of structures matching the signature \li{QUEUE} this ability by defining the following parameterized expression TSM:
% \begin{lstlisting}[numbers=none]
% syntax $queue(Q : QUEUE)('a) at Q.queue('a) {
%   static fn(body : Body) : ParseResult(CEExp) => (* ... *)
% }
% \end{lstlisting}
% This peTSM specifies one module parameter, \li{Q}, which must match the signature \li{QUEUE}, and one type parameter, \li{'a} (implicitly of kind \li{T}). These appear in the type annotation, which specifies that expansions that arise from applying \li{#\dolla#queue} to a module \li{Q : QUEUE} and a type \li{'a} will be of type \li{Q.queue('a)}. For example:
% \begin{lstlisting}
% val q = $queue TwoListQueue int [SURL> EURL1SURL, EURL2SURL, EURL3]
% val q' = $queue TwoListQueue int [qSURL > EURL4SURL, EURL5]
% \end{lstlisting}
% On Line 1, the initial angle bracket (\li{SURL>EURL}) indicates that the items are inserted in left-to-right order. The items in the queue are given as spliced subexpressions separated by commas. Line 2 inserts two additional items onto the back of the queue \li{q}. The expansion of this example, written textually, is:
% \begin{lstlisting}
% val q : TwoListQueue.queue(int) = 
%   TwoListQueue.insert(1, 
%     TwoListQueue.insert(2, 
%       TwoListQueue.insert(3, 
%         TwoListQueue.empty)))
% val q' : TwoListQueue.queue(int) = 
%   TwoListQueue.insert(4, TwoListQueue.insert(5, q))
% \end{lstlisting}
% Notice that the expansion can refer to the module parameter \li{TwoListQueue}.

% We can further reduce syntactic cost by defining a synonym for the partial application of \li{#\dolla#queue} to the module parameter \li{TwoListQueue}:
% \begin{lstlisting}[numbers=none]
% syntax $tlq = $queue TwoListQueue
% val q = $tlq int [SURL> EURL1SURL, EURL2SURL, EURL3]
% \end{lstlisting}
% We can further define a synonym for the partial application of \li{#\dolla#tlq} to a type parameter:
% \begin{lstlisting}[numbers=none]
% syntax $tlqi = $tlq int (* = $queue TwoListQueue int *)
% val q' = $tlqi [qSURL > EURL4SURL, EURL5]
% \end{lstlisting}
% \subsection{Module Parameters}
% VerseML also provides a module language based on the Standard ML module language \cite{MacQueen:1984:MSM:800055.802036}. The module language consists of \emph{module expressions} classified by \emph{signatures}. %Signatures specify type components, which may be opaque or transparent, value components, and module components.h

% %In Sec. \ref{sec:motivating-examples}, we gave several examples of signatures and discussed how one might introduce derived forms that  across a module-parameterized family of types.



% Another way to reduce syntactic cost is by designating \li{#\dolla#queue Q 'a} the implicit TSM at all types of the form \li{Q.queue('a)} where \li{Q : QUEUE}. This is written as follows:
% \begin{lstlisting}[numbers=none]
% implicit syntax (Q : QUEUE) ('a) => $queue Q 'a in
%   val q : TwoListQueue.queue(int) = [SURL> EURL1SURL, EURL2SURL, EURL3]
%   val q' : TwoListQueue.queue(int) = [qSURL > EURL4SURL, EURL5]
% end
% \end{lstlisting}
% This designation is particularly useful for clients who need to construct a queue as an argument to a function. For example, consider a function 
% \begin{lstlisting}[numbers=none]
% enqueue_jobs : Q.queue(Job) -> Ticket
% \end{lstlisting}
% for some module \li{Q : QUEUE} and types \li{Job} and \li{Ticket}. We can enqueue a sequence of jobs \li{j1} through \li{j4} under the TSM designation above as follows:
% \begin{lstlisting}[numbers=none]
% enqueue_jobs [SURL> EURLj1SURL, EURLj2SURL, EURLj3SURL, EURLj4]
% \end{lstlisting}

\section{\texorpdfstring{$\miniVerseParam$}{miniVerseP}}\label{sec:miniVerseP}
We will now define a reduced dialect of VerseML called $\miniVerseParam$ that supports parametric expression and pattern TSMs (peTSMs and ppTSMs.) This language, like $\miniVersePat$, consists of an unexpanded language (UL) defined by typed expansion to an expanded language (XL). The full definition of $\miniVerseParam$ is given in Appendix \ref{appendix:miniVerseParam} -- we will detail only  particularly interesting constructs below.

\subsection{Syntax of the Expanded Language (XL)}\label{sec:P-expanded-terms}

\begin{figure}[p] 
\[\begin{array}{lllllll}
\textbf{Sort} & & & \textbf{Operational Form} 
%& \textbf{Stylized Form} 
& \textbf{Description}\\
\mathsf{Sig} & \sigma & ::= & \asignature{\kappa}{u}{\tau} 
%& \signature{u}{\kappa}{\tau} 
& \text{signature}\\
\mathsf{Mod} & M & ::= & X 
%& X 
& \text{module variable}\\
&&& \astruct{c}{e} 
%& \struct{c}{e} 
& \text{structure}\\
&&& \aseal{\sigma}{M} 
%& \seal{M}{\sigma} 
& \text{seal}\\
&&& \amlet{\sigma}{M}{X}{M} %& \mlet{X}{M}{M}{\sigma} 
& \text{definition}
\end{array}\]
\caption[Syntax of signatures and module expressions in $\miniVerseParam$]{Syntax of signatures and module expressions in $\miniVerseParam$.}
\label{fig:P-modules-signatures}
\end{figure}


\begin{figure}[p] 
\[\begin{array}{lrlllll}
\textbf{Sort} & & & \textbf{Operational Form} 
%& \textbf{Stylized Form} 
& \textbf{Description}\\
\mathsf{Kind} & \kappa & ::= & k & \text{kind variable}\\
&&& \akdarr{\kappa}{u}{\kappa} 
%& \kdarr{u}{\kappa}{\kappa} 
& \text{dependent function}\\
&&& \akunit 
%& \kunit 
& \text{nullary product}\\
&&& \akdbprod{\kappa}{u}{\kappa} 
%& \kdbprod{u}{\kappa}{\kappa} 
& \text{dependent product}\\
%&&& \akdprodstd & \kdprodstd & \text{labeled dependent product}\\
&&& \akty 
%& \kty
& \text{type}\\
&&& \aksing{\tau} 
%& \ksing{\tau} 
& \text{singleton}\\
\mathsf{Con} & c, \tau & ::= & u 
%& u 
& \text{construction variable}\\
&&& t 
%& t 
& \text{type variable}
\\
&&& \acabs{u}{c} 
%& \cabs{u}{c} 
& \text{abstraction}\\
&&& \acapp{c}{c} 
%& \capp{c}{c} 
& \text{application}\\
&&& \actriv 
%& \ctriv 
& \text{trivial}\\
&&& \acpair{c}{c}
% & \cpair{c}{c} 
& \text{pair}\\
&&& \acprl{c} 
%& \cprl{c} 
& \text{left projection}\\
&&& \acprr{c} 
%& \cprr{c} 
& \text{right projection}\\
%&&& \adtplX & \dtplX & \text{labeled dependent tuple}\\
%&&& \adprj{\ell}{c} & \prj{c}{\ell} & \text{projection}\\
&&& \aparr{\tau}{\tau} 
%& \parr{\tau}{\tau} 
& \text{partial function}\\
&&& \aallu{\kappa}{u}{\tau} 
%& \forallu{u}{\kappa}{\tau} 
& \text{polymorphic}\\
&&& \arec{t}{\tau} 
%& \rect{t}{\tau} 
& \text{recursive}\\
&&& \aprod{\labelset}{\mapschema{\tau}{i}{\labelset}} 
%& \prodt{\mapschema{\tau}{i}{\labelset}} 
& \text{labeled product}\\
&&& \asum{\labelset}{\mapschema{\tau}{i}{\labelset}} 
%& \sumt{\mapschema{\tau}{i}{\labelset}} 
& \text{labeled sum}\\
&&& \amcon{M} 
%& \mcon{M} 
& \text{construction component}
\end{array}\]
\caption[Syntax of kinds and constructions in $\miniVerseParam$]{Syntax of kinds and constructions in $\miniVerseParam$. By convention, we choose the metavariable $\tau$ for constructions that, in well-formed terms, must necessarily be of kind $\kty$, and the metavariable $c$ otherwise. Similarly, we use construction variables $t$ to stand for constructions of kind $\kty$, and construction variables $u$ otherwise. Kind variables, $k$, are necessary only for the metatheory.}
\label{fig:P-kinds-constructors}
\end{figure}

\begin{figure}
\[\begin{array}{lllllll}
\textbf{Sort} & & & \textbf{Operational Form} 
%& \textbf{Stylized Form} 
& \textbf{Description}\\
\mathsf{Exp} & e & ::= & x 
%& x 
& \text{variable}\\
&&& \aelam{\tau}{x}{e} 
%& \lam{x}{\tau}{e} 
& \text{abstraction}\\
&&& \aeap{e}{e} 
%& \ap{e}{e} 
& \text{application}\\
&&& \aeclam{\kappa}{u}{e} %& \clam{u}{\kappa}{e} 
& \text{construction abstraction}\\
&&& \aecap{e}{c} %& \cAp{e}{c} 
& \text{construction application}\\
&&& \aefold{e} %& \fold{e} 
& \text{fold}\\
&&& \aeunfold{e} %& \unfold{e} 
& \text{unfold}\\
&&& \aetpl{\labelset}{\mapschema{e}{i}{\labelset}} 
%& \tpl{\mapschema{e}{i}{\labelset}} 
& \text{labeled tuple}\\
&&& \aepr{\ell}{e} 
%& \prj{e}{\ell} 
& \text{projection}\\
&&& \aein{\ell}{e} 
%& \inj{\ell}{e} 
& \text{injection}\\
&&& \aematchwith{n}{e}{\seqschemaX{r}} 
%& \matchwith{e}{\seqschemaX{r}} 
& \text{match}\\
&&& \amval{M} 
%& \mval{M} 
& \text{value component}\\
\mathsf{Rule} & r & ::= & \aematchrule{p}{e} 
%& \matchrule{p}{e} 
& \text{rule}\\
\mathsf{Pat} & p & ::= & x 
%& x 
& \text{variable pattern}\\
&&& \aewildp 
%& \wildp 
& \text{wildcard pattern}\\
&&& \aefoldp{p} 
%& \foldp{p} 
& \text{fold pattern}\\
&&& \aetplp{\labelset}{\mapschema{p}{i}{\labelset}} 
%& \tplp{\mapschema{p}{i}{\labelset}} 
& \text{labeled tuple pattern}\\
&&& \aeinjp{\ell}{p} 
%& \injp{\ell}{p} 
& \text{injection pattern}
\end{array}\]
\caption[Syntax of expanded expressions, rules and patterns in $\miniVerseParam$]{Syntax of expanded expressions, rules and patterns in $\miniVerseParam$.}
\label{fig:P-expanded-terms}
\end{figure}


Figure \ref{fig:P-modules-signatures} defines the syntax of the \emph{expanded module language}. Figure \ref{fig:P-kinds-constructors} defines the syntax of the \emph{expanded type construction language}. Figure \ref{fig:P-expanded-terms} defines the syntax of the \emph{expanded expression language}.


\subsection{Statics of the Expanded Language}
The module and type construction languages are based closely on those defined by Harper in \emph{PFPL} \cite{pfpl}. These languages, in turn, are based on the languages developed by Lee et al. \cite{conf/popl/LeeCH07}, and also by Dreyer \cite{dreyer2005understanding}. All of these incorporate Stone and Harper's \emph{dependent singleton kinds} formalism to track type identity \cite{stone2006extensional}. The expression language is similar to that of $\miniVersePat$, defined in Chapter \ref{chap:uptsms}.

The \emph{statics of the expanded language} is defined by a collection of judgements that we organize into three groups. 

The first group of judgements, which we refer to as the \emph{statics of the expanded module language}, define the statics of expanded signatures and module expressions.

\vspace{5px}
$\begin{array}{ll}
\textbf{Judgement Form} & \textbf{Description}\\
\issigX{\sigma} & \text{$\sigma$ is a signature }\\
\sigequalX{\sigma}{\sigma'} & \text{$\sigma$ and $\sigma'$ are definitionally equal signatures}\\
\sigsubX{\sigma}{\sigma'} & \text{$\sigma$ is a sub-signature of $\sigma'$}\\
\hassigX{M}{\sigma} & \text{$M$ matches $\sigma$}\\
\ismvalX{M} & \text{$M$ is, or stands for, a module value}
\end{array}$
\vspace{5px}

The second group of judgements, which we refer to as the \emph{statics of the expanded type construction language}, define the statics of expanded kinds and constructions.

\vspace{5px}
$\begin{array}{ll}
\textbf{Judgement Form} & \textbf{Description}\\
\iskindX{\kappa} & \text{$\kappa$ is a kind}\\
\kequalX{\kappa}{\kappa'} & \text{$\kappa$ and $\kappa'$ are definitionally equal kinds}\\
\ksubX{\kappa}{\kappa'} & \text{$\kappa$ is a subkind of $\kappa'$}\\
\haskindX{c}{\kappa} & \text{$c$ has kind $\kappa$}\\
\cequalX{c}{c'}{\kappa} & \text{$c$ and $c'$ are equivalent as constructions of kind $\kappa$}
\end{array}$
\vspace{5px}

The third group of judgements, which we refer to as the \emph{statics of the expanded expression language}, define the statics of types, expanded expressions, rules and patterns. Types are constructions of kind $\akty$. We use the metavariable $\tau$ rather than $c$ for types.

\vspace{5px}
$\begin{array}{ll}
\textbf{Judgement Form} & \textbf{Description}\\
% \istypeP{\Omega}{\tau} & \text{$\tau$ is a well-formed type}\\
% \tequalPX{\tau}{\tau'} & \text{$\tau$ and $\tau'$ are definitionally equal types}\\
\issubtypePX{\tau}{\tau'} & \text{$\tau$ is a subtype of $\tau'$}\\
\hastypeP{\Omega}{e}{\tau} & \text{$e$ is assigned type $\tau$}\\
\ruleTypeP{\Omega}{r}{\tau}{\tau'} & \text{$r$ takes values of type $\tau$ to values of type $\tau'$}\\
\patTypeP{\Omega'}{p}{\tau} & \text{$p$ matches values of type $\tau$ and generates hypotheses $\Omega'$} 
\end{array}$
\vspace{5px}


A \emph{unified context}, $\Omega$, is a finite function over module, expression and construction variables. 
We write
\begin{itemize}
\item $\Omega, X : \sigma$ when $X \notin \domof{\Omega}$ and $\issigX{\sigma}$ for the extension of $\Omega$ with a mapping from $X$ to the hypothesis $X : \sigma$.
\item $\Omega, x : \tau$ when $x \notin \domof{\Omega}$ and $\haskindX{\tau}{\akty}$ for the extension of $\Omega$ with a mapping from $x$ to the hypothesis $x : \tau$
\item $\Omega, u :: \kappa$ when $u \notin \domof{\Omega}$ and $\iskindX{\kappa}$ for the extension of $\Omega$ with a mapping from $u$ to the hypothesis $u :: \kappa$
\end{itemize}
A well-formed unified context is one that can be constructed by some sequence of such extensions, starting from the empty context, $\emptyset$. We identify unified contexts up to exchange and contraction in the usual manner.

The complete set of rules is given in Appendix \ref{appendix:P-statics}. A comprehensive introductory account of these constructs is beyond the scope of this work (see \cite{pfpl}.) Instead, let us summarize the key features of the expanded language by example. 

Modules take the form $\astruct{c}{e}$, following a \emph{phase-splitting} approach -- the construction components of the module are ``tupled'' into a single construction component, $c$, and the value components of the module are ``tupled'' into a single value component, $e$ \cite{harper1989higher}. Signatures, $\sigma$, are also split in this way -- a single \emph{kind}, $\kappa$,  classifies the construction component and a single type, $\tau$, classifies the value component of the classified module. The type can refer to the construction component through a mediating construction variable, $u$. The key rule is reproduced below:
\begin{equation*}\tag{\ref{rule:hassig-struct}}
\inferrule{
  \haskindX{c}{\kappa}\\
  \hastypeP{\Omega}{e}{[c/u]\tau}
}{
  \hassigX{\astruct{c}{e}}{\asignature{\kappa}{u}{\tau}}
}
\end{equation*}

For example, consider the VerseML signature and the corresponding $\miniVerseParam$ signature in Figure \ref{fig:corresponding-signatures}. The kind on the right (Lines 1-3) is a \emph{dependent product kind} and the type (Lines 4-5) is a product type. Let us consider these in turn.

\begin{figure}
\begin{minipage}{0.35\textwidth}
\begin{lstlisting}
sig
  type t
  type t' = t * t
  val x : t
  val y : t -> t'
end
\end{lstlisting}
\end{minipage}
\begin{minipage}{0.5\textwidth}\vspace{3px}
{\footnotesize\[
\begin{array}{l}
\asignature{\akdbprod{\\
\quad\quad \akty}{t}{\\
\quad\quad \aksing{
  \aprod{\lbltxt{1}; \lbltxt{2}}{
    \mapitem{\lbltxt{1}}{t}; \mapitem{\lbltxt{2}}{t}
  }
}}\\}{u}{
 \aprod{\lbltxt{x}; \lbltxt{y}}{
  \mapitem{\lbltxt{x}}{\acprl{u}}; \\ 
\quad\quad \mapitem{\lbltxt{y}}{\aparr{
    \acprl{u}
  }{
    \acprr{u}
  }}
}\\
}
\end{array}
\]}
\end{minipage}
\caption{A VerseML signature and the corresponding $\miniVerseParam$ signature.}
\label{fig:corresponding-signatures}
\vspace{-10px}
\end{figure}



On Lines 2-3 (left), we specified an abstract type component \li{t}, and then a translucent type component \li{t'} equal to \li{t * t}. Abstract type components have kind $\akty$, so the first component of the dependent product kind is $\akty$ (Line 2, right). The construction variable $t$ stands for the first component in the second component of the dependent product kind. The second component is not held abstract, so it is classified by a corresponding \emph{singleton kind}, rather than by the kind $\akty$ (Line 3, right). A singleton kind $\aksing{\tau}$ classifies only those types definitionally equal to $\tau$. A subkinding system is necessary to ensure that constructions of singleton kind can appear where a construction of kind $\akty$ is needed -- the key rule is reproduced below:
\begin{equation*}\tag{\ref{rule:ksub-sing}}
\inferrule{
  \haskindX{\tau}{\akty}
}{
  \ksubX{\aksing{\tau}}{\akty}
}
\end{equation*}

Lines 4-5 (right) define a product type that classifies the value component of matching modules. The construction variable \li{u} stands for the construction component of the matching module. The left- and right-projection operations $\acprl{c}$ and $\acprr{c}$ on the right correspond to \li{t} and \li{t'} on the left. (In practice, we would use labeled dependent product kinds, but for simplicity, we stick to binary dependent product kinds here.)

Consider another example: the VerseML \li{LIST} signature from Figure \ref{fig:LIST}, partially reproduced below:
\begin{lstlisting}
sig 
  type list('a) = rec(self => Nil + Cons of 'a * self)
  val Nil : list('a)
  val Cons : 'a * list('a) -> list('a)
  (* ... *)
end
\end{lstlisting}
This signature corresponds to the $\miniVerseParam$ signature $\sigma_\texttt{LIST}$ defined in Figure \ref{fig:LIST-mini}. 

\begin{figure}
\[
\arraycolsep=1px\begin{array}{ll}

\sigma_\texttt{LIST} & \defeq \asignature{\kappa_\texttt{LIST}}{list}{\tau_\texttt{LIST}}\\
\kappa_\texttt{LIST} & \defeq \akdarr{\akty}{\alpha}{\aksing{
  \arec{self}{
    \asum{L_\texttt{list}}{\\
    & \quad\quad 
      \mapitem{\lbltxt{Nil}}{\aprod{}{}}; \\
    & \quad\quad 
      \mapitem{\lbltxt{Cons}}{
        \aprod{\lbltxt{1}; \lbltxt{2}}{
          \mapitem{\lbltxt{1}}{\alpha}; 
          \mapitem{\lbltxt{2}}{self}
        }
      }
    }
  } 
}}\\
L_\texttt{list} & \defeq \lbltxt{Nil}, \lbltxt{Cons}\\
\tau_\texttt{LIST} & \defeq \aprod{L_\texttt{list}}{\\&
  \quad\quad \mapitem{\lbltxt{Nil}}{
    \aallu{\akty}{\alpha}{\acapp{list}{\alpha}}
  }; \\&
  \quad\quad \mapitem{\lbltxt{Cons}}{
    \aallu{\akty}{\alpha}{
      \aparr{\\&\quad\quad\quad
        \aprod{\lbltxt{1}; \lbltxt{2}}{
          \mapitem{\lbltxt{1}}{\alpha}; 
          \mapitem{\lbltxt{2}}{\acapp{list}{\alpha}}
        }
      }{\\&\quad\quad\quad
        \acapp{list}{\alpha}
      }
    }
  }
}
\end{array}
\]
\caption{The $\miniVerseParam$ encoding of the \texttt{LIST} signature.}
\label{fig:LIST-mini}
\end{figure}
Here, the signature specifies only a single construction component, so no tupling of the construction component is necessary. This single construction component is a type function, so it has dependent function kind: the argument kind is $\akty$ and the return kind is a singleton kind, because the type function is not abstract. (Had we held the type function abstract, its kind would instead be $\akdarr{\akty}{\_}{\akty}$.)
%A well-formed unified inner context is one where there are no cycles in the dependency graph between the hypotheses (constructed in the obvious manner) and for each hypothesis, the construction, kinds or signature involved is well-formed relative to the unified inner context.

At the top level, a program consists of a module expression, $M$. The module let binding form allows the programmer to bind a module to a module variable, $X$:
\begin{equation*}\tag{\ref{rule:hassig-let}}
\inferrule{
  \hassigX{M}{\sigma}\\
  \issigX{\sigma'}\\
  \hassig{\Omega, X : \sigma}{M'}{\sigma'}  
}{
  \hassigX{\amlet{\sigma'}{M}{X}{M'}}{\sigma'}
}
\end{equation*}

The construction projection form, $\amcon{M}$, allows us to refer to the construction component of $M$ within a construction appearing in $M'$. The kinding rule for this form is reproduced below:
\begin{equation*}\tag{\ref{rule:haskind-stat}}
\inferrule{
  \ismvalX{M}\\
  \hassigX{M}{\asignature{\kappa}{u}{\tau}}
}{
  \haskindX{\amcon{M}}{\kappa}
}
\end{equation*}
Similarly, the value projection form, $\amval{M}$, projects out the value component of $M$ within an expression appearing in $M'$. The typing rule for this form is reproduced below:
\begin{equation*}\tag{\ref{rule:hastypeP-dyn}}
\inferrule{
  \ismvalX{M}\\
  \hassigX{M}{\asignature{\kappa}{u}{\tau}}
}{
  \hastypeP{\Omega}{\amval{M}}{[\amcon{M}/u]\tau}
}
\end{equation*}
The first premise of both of these rules requires that $M$ be, or stand for, a \emph{module value}, according to the following rules:
\begin{equation*}\tag{\ref{rule:ismval-struct}}
\inferrule{ }{
  \ismvalX{\astruct{c}{e}}
}
\end{equation*}
\begin{equation*}\tag{\ref{rule:ismval-var}}
\inferrule{ }{
  \ismval{\Omega, X : \sigma}{X}
}
\end{equation*}
The reason for this restriction has to do with the \emph{sealing} operation:
\begin{equation*}\tag{\ref{rule:hassig-seal}}
\inferrule{
  \issigX{\sigma}\\
  \hassigX{M}{\sigma}
}{
  \hassigX{\aseal{\sigma}{M}}{\sigma}
}
\end{equation*}
Sealing enforces \emph{representation independence} -- the abstract construction components of a sealed module are not treated as equivalent to those of any other sealed module within the program. In other words, sealing is \emph{generative}. The module value restriction above achieves this behavior by simple syntactic means -- a sealed module is not a module value, so all sealed modules have to be bound to distinct module variables.

The judgements above obey standard lemmas, including Weakening, Substitution and Decomposition (see Appendix \ref{appendix:P-statics}.)

We omit certain features of the ML module system in  $\miniVerseParam$, such as its support for hierarchical modules and functors. Our formulation also does not support ``width'' subtyping and subkinding for simplicity. These are straightforward extensions of $\miniVerseParam$, but because their inclusion would not change the semantics of parametric TSMs, we did not include them (see \cite{pfpl} for a discussion of these features.)

\subsection{Structural Dynamics}
The structural dynamics of modules is defined as a transition system, and is organized around judgements of the following form:

\vspace{10px}
$\begin{array}{ll}
\textbf{Judgement Form} & \textbf{Description}\\
\stepsU{M}{M'} & \text{$M$ transitions to $M'$}\\
\isvalP{M} & \text{$M$ is a module value}\\
\matchfail{M} & \text{$M$ raises match failure}
\end{array}$
\vspace{10px}

The structural dynamics of expressions is also defined as a transition system, and is organized around judgements of the following form:

\vspace{10px}
$\begin{array}{ll}
\textbf{Judgement Form} & \textbf{Description}\\
\stepsU{e}{e'} & \text{$e$ transitions to $e'$}\\
\isvalP{e} & \text{$e$ is a value}\\
\matchfail{e} & \text{$e$ raises match failure}
\end{array}$
\vspace{10px}

We also define auxiliary judgements for \emph{iterated transition}, $\multistepU{e}{e'}$, and \emph{evaluation}, $\evalU{e}{e'}$, of expressions.

\begingroup
\def\thetheorem{\ref{defn:iterated-transition-P}}
\begin{definition}[Iterated Transition] Iterated transition, $\multistepU{e}{e'}$, is the reflexive, transitive closure of the transition judgement, $\stepsU{e}{e'}$.\end{definition}
% \addtocounter{theorem}{-1}
\endgroup

\begingroup
\def\thetheorem{\ref{defn:evaluation-P}}
\begin{definition}[Evaluation] $\evalU{e}{e'}$ iff $\multistepU{e}{e'}$ and $\isvalU{e'}$. \end{definition}
% \addtocounter{theorem}{-1}
\endgroup

As in previous chapters, our subsequent developments do not make mention of particular rules in the dynamics, so we do not produce these details here. Instead, it suffices to state the following conditions.

The Preservation condition ensures that evaluation preserves typing.
\begingroup
\def\thetheorem{\ref{condition:preservation-P}}
\begin{condition}[Preservation] ~
\begin{enumerate}
\item If $\hassig{}{M}{\sigma}$ and $\stepsU{M}{M'}$ then $\hassig{}{M}{\sigma}$.
\item If $\hastypeUC{e}{\tau}$ and $\stepsU{e}{e'}$ then $\hastypeUC{e'}{\tau}$.
\end{enumerate}
\end{condition}
\endgroup

The Progress condition ensures that evaluation of a well-typed expanded expression cannot ``get stuck''. We must consider the possibility of match failure in this condition.
\begingroup
\def\thetheorem{\ref{condition:progress-P}}
\begin{condition}[Progress] ~
\begin{enumerate}
\item If $\hassig{}{M}{\sigma}$ then either $\isvalU{M}$ or $\matchfail{M}$ or there exists an $M'$ such that $\stepsU{M}{M'}$.
\item If $\hastypeUC{e}{\tau}$ then either $\isvalU{e}$ or $\matchfail{e}$ or there exists an $e'$ such that $\stepsU{e}{e'}$.
\end{enumerate}
\end{condition}
% \addtocounter{theorem}{-1}
\endgroup

Together, these two conditions constitute the Type Safety Condition.

\begin{figure}[p] \vspace{-15px}
$\arraycolsep=4pt\begin{array}{lllllll}
\textbf{Sort} & & 
%& \textbf{Operational Form} 
& \textbf{Stylized Form} & \textbf{Description}\\
\mathsf{USig} & \usigma & ::= 
%& \ausignature{\ukappa}{\uu}{\utau} 
& \signature{\uu}{\ukappa}{\utau} & \text{signature}\\
\mathsf{UMod} & \uM & ::= 
%& \uX 
& \uX & \text{module identifier}\\
&&
%& \austruct{\uc}{\ue} 
& \struct{\uc}{\ue} & \text{structure}\\
&&
%& \auseal{\usigma}{\uM} 
& \seal{\uM}{\usigma} & \text{seal}\\
&&
%& \aumlet{\usigma}{\uM}{\uX}{\uM} 
& \mlet{\uX}{\uM}{\uM}{\usigma} & \text{definition}\\
\LCC &&
%& \lightgray 
& \color{Yellow} & \color{Yellow}\\
&&
%& \aumdefpetsm{\urho}{e}{\tsmv}{\uM} 
& \defpetsm{\tsmv}{\urho}{e}{\uM} & \text{peTSM definition}\\
%&&&                                    & \texttt{expressions}~\{e\}~\texttt{in}~\uM\\
&&
%& \aumletpetsm{\uepsilon}{\tsmv}{\uM} 
& \uletpetsm{\tsmv}{\uepsilon}{\uM} & \text{peTSM binding}\\
% &&&                                  & \texttt{expressions}~\texttt{in}~\uM\\
% &&& ... & ... & \text{peTSM designation}\\
&&
%& \audefpptsm{\urho}{e}{\tsmv}{\uM} 
& \defpptsm{\tsmv}{\urho}{e}{\uM} & \text{ppTSM definition}\\
% &&&                                    & \texttt{patterns}~\{e\}~\texttt{in}~\uM\\
&&
%& \auletpptsm{\uepsilon}{\tsmv}{\uM} 
& \uletpptsm{\tsmv}{\uepsilon}{\uM} & \text{ppTSM binding}\ECC%
% &&& & \texttt{patterns}~\texttt{in}~\uM\\
% &&& ... & ... & \text{ppTSM designation}\ECC
\end{array}$\vspace{-5px}
\caption[Syntax of unexpanded module expressions and signatures in $\miniVerseParam$]{Syntax of unexpanded module expressions and signatures in $\miniVerseParam$.}\vspace{-5px}
\label{fig:P-unexpanded-modules-signatures}
\end{figure}
\begin{figure}[p] \vspace{-10px}
\[\begin{array}{lrlllll}
\textbf{Sort} & & 
%& \textbf{Operational Form} 
& \textbf{Stylized Form} & \textbf{Description}\\
\mathsf{UKind} & \ukappa & ::= 
%& \aukdarr{\ukappa}{\uu}{\ukappa} 
& \kdarr{\uu}{\ukappa}{\ukappa} & \text{dependent function}\\
&&
%& \aukunit 
& \kunit & \text{nullary product}\\
&&
%& \aukdbprod{\ukappa}{\uu}{\ukappa} 
& \kdbprod{\uu}{\ukappa}{\ukappa} & \text{dependent product}\\
%&&& \akdprodstd & \kdprodstd & \text{labeled dependent product}\\
&&
%& \aukty 
& \kty & \text{type}\\
&&
%& \auksing{\utau} 
& \ksing{\utau} & \text{singleton}\\
\mathsf{UCon} & \uc, \utau & ::= 
%& \uu 
& \uu & \text{construction identifier}\\
&&
%& \ut 
& \ut & \\
&&
%& \aucasc{\ukappa}{\uc} 
& \casc{\uc}{\ukappa} & \text{ascription}\\
&&
%& \aucabs{\uu}{\uc} 
& \cabs{\uu}{\uc} & \text{abstraction}\\
&&
%& \aucapp{c}{c} 
& \capp{c}{c} & \text{application}\\
&&
%& \auctriv 
& \ctriv & \text{trivial}\\
&&
%& \aucpair{\uc}{\uc} 
& \cpair{\uc}{\uc} & \text{pair}\\
&&
%& \aucprl{\uc} 
& \cprl{\uc} & \text{left projection}\\
&&
%& \aucprr{\uc} 
& \cprr{\uc} & \text{right projection}\\
%&&& \adtplX & \dtplX & \text{labeled dependent tuple}\\
%&&& \adprj{\ell}{c} & \prj{c}{\ell} & \text{projection}\\
&&
%& \auparr{\utau}{\utau} 
& \parr{\utau}{\utau} & \text{partial function}\\
&&
%& \auallu{\ukappa}{\uu}{\utau} 
& \forallu{\uu}{\ukappa}{\utau} & \text{polymorphic}\\
&&
%& \aurec{\ut}{\utau} 
& \rect{\ut}{\utau} & \text{recursive}\\
&&
%& \auprod{\labelset}{\mapschema{\utau}{i}{\labelset}} 
& \prodt{\mapschema{\utau}{i}{\labelset}} & \text{labeled product}\\
&&
%& \ausum{\labelset}{\mapschema{\utau}{i}{\labelset}} 
& \sumt{\mapschema{\utau}{i}{\labelset}} & \text{labeled sum}\\
&&
%& \aumcon{\uX} 
& \mcon{\uX} & \text{construction component}
\end{array}\]\vspace{-5px}
\caption[Syntax of unexpanded kinds and constructions in $\miniVerseParam$]{Syntax of unexpanded kinds and constructions in $\miniVerseParam$.}\vspace{-10px}
\label{fig:P-unexpanded-kinds-constructors}
\end{figure}

% \clearpage
\begin{figure}[p]
\[\begin{array}{lllllll}
\textbf{Sort} & & 
%& \textbf{Operational Form} 
& \textbf{Stylized Form} & \textbf{Description}\\
\mathsf{UExp} & \ue & ::= 
%& \ux 
& \ux & \text{identifier}\\
&&
% & \auasc{\utau}{\ue} 
& \asc{\ue}{\utau} & \text{ascription}\\
&&
% & \auletsyn{\ux}{\ue}{\ue} 
& \letsyn{\ux}{\ue}{\ue} & \text{value binding}\\
% &&
%& \auanalam{\ux}{\ue} 
% & \analam{\ux}{\ue} & \text{abstraction (unannotated)}\\
&&
%& \aulam{\utau}{\ux}{\ue} 
& \lam{\ux}{\utau}{\ue} & \text{abstraction}\\
&&
%& \auap{\ue}{\ue} 
& \ap{\ue}{\ue} & \text{application}\\
&&
%& \auclam{\ukappa}{\uu}{\ue} 
& \clam{\uu}{\ukappa}{\ue} & \text{construction abstraction}\\
&&
%& \aucap{\ue}{\uc} 
& \cAp{\ue}{\uc} & \text{construction application}\\
&&
%& \auanafold{\ue} 
& \fold{\ue} & \text{fold}\\
&&
%& \auunfold{\ue} 
& \unfold{\ue} & \text{unfold}\\
&&
%& \autpl{\labelset}{\mapschema{\ue}{i}{\labelset}} 
& \tpl{\mapschema{\ue}{i}{\labelset}} & \text{labeled tuple}\\
&&
%& \aupr{\ell}{\ue} 
& \prj{\ue}{\ell} & \text{projection}\\
&&
%& \auanain{\ell}{\ue} 
& \inj{\ell}{\ue} & \text{injection}\\
&&
%& \aumatchwithb{n}{\ue}{\seqschemaX{\urv}} 
& \matchwith{\ue}{\seqschemaX{\urv}} & \text{match}\\
&&
%& \aumval{\uX} 
& \mval{\uX} & \text{value component}\\
\LCC &&
% %& \color{Yellow} 
& \color{Yellow} & \color{Yellow} \\
% &&& \audefpetsm{\urho}{e}{\tsmv}{\ue} & \texttt{syntax}~\tsmv~\texttt{at}~\urho~\texttt{for} & \text{peTSM definition}\\
% &&&                                    & \texttt{expressions}~\{e\}~\texttt{in}~\ue\\
% &&& \auletpetsm{\uepsilon}{\tsmv}{\ue} & \texttt{let}~\texttt{syntax}~\tsmv=\uepsilon~\texttt{for} & \text{peTSM binding}\\
% &&&                                  & \texttt{expressions}~\texttt{in}~\ue\\
% &&& ... & ... & \text{peTSM designation}\\
&&
%& \auappetsm{b}{\uepsilon} 
& \utsmap{\uepsilon}{b} & \text{peTSM application}\ECC\\%\ECC
% &&& \auelit{b} & {\lit{b}}  & \text{peTSM unadorned literal}\\
% &&& \audefpptsm{\urho}{e}{\tsmv}{\ue} & \texttt{syntax}~\tsmv~\texttt{at}~\urho~\texttt{for} & \text{ppTSM definition}\\
% &&&                                    & \texttt{patterns}~\{e\}~\texttt{in}~\ue\\
% &&& \auletpptsm{\uepsilon}{\tsmv}{\ue} & \texttt{let}~\texttt{syntax}~\tsmv=\uepsilon~\texttt{for} & \text{ppTSM binding}\\
% &&& & \texttt{patterns}~\texttt{in}~\ue\\
% &&& ... & ... & \text{ppTSM designation}\\\ECC
\mathsf{URule} & \urv & ::= 
%& \aumatchrule{\upv}{\ue} 
& \matchrule{\upv}{\ue} & \text{match rule}\\
\mathsf{UPat} & \upv & ::= 
%& \ux 
& \ux & \text{identifier pattern}\\
&&
%& \auwildp 
& \wildp & \text{wildcard pattern}\\
&&
%& \aufoldp{\upv} 
& \foldp{\upv} & \text{fold pattern}\\
&&
%& \autplp{\labelset}{\mapschema{\upv}{i}{\labelset}} 
& \tplp{\mapschema{\upv}{i}{\labelset}} & \text{labeled tuple pattern}\\
&&
% & \auinjp{\ell}{\upv} 
& \injp{\ell}{\upv} 
& \text{injection pattern}\\
\LCC &&
%& \lightgray 
& \color{Yellow} & \color{Yellow}\\
&&
%& \auappptsm{b}{\uepsilon} 
& \utsmap{\uepsilon}{b} & \text{ppTSM application}\ECC
% &&& \auplit{b} & \lit{b} & \text{ppTSM unadorned literal}\ECC
\end{array}\]
\caption[Syntax of unexpanded expressions, rules and patterns in $\miniVerseParam$]{Syntax of unexpanded expressions, rules and patterns in $\miniVerseParam$.}
\label{fig:P-unexpanded-terms}
\end{figure}

% \clearpage



\begin{figure}[p]
\[\begin{array}{lllllll}
\textbf{Sort} & & 
%& \textbf{Operational Form} 
& \textbf{Stylized Form} 
& \textbf{Description}\\
\LCC \color{Yellow}&\color{Yellow}& \color{Yellow}
%& \lightgray 
& \color{Yellow} & \color{Yellow}\\
\mathsf{UMType} & \urho & ::= 
%& \autype{\utau} 
& \utau & \text{type annotation}\\
% &&
%& \aualltypes{\ut}{\urho} 
% & \alltypes{\ut}{\urho} & \text{type parameterization}\\
&&
%& \auallmods{\usigma}{\uX}{\urho} 
& \allmods{\uX}{\usigma}{\urho} & \text{module parameterization}\\
\mathsf{UMExp} & \uepsilon & ::= 
%& \abindref{\tsmv} 
& \tsmv & \text{TSM identifier reference}\\
% &&
%& \auabstype{\ut}{\uepsilon} 
% & \abstype{\ut}{\uepsilon} & \text{type abstraction}\\
&&
%& \auabsmod{\usigma}{\uX}{\uepsilon} 
& \absmod{\uX}{\usigma}{\uepsilon} & \text{module abstraction}\\
% &&
%& \auaptype{\utau}{\uepsilon} 
% & \aptype{\uepsilon}{\utau} & \text{type application}\\
&&
%& \auapmod{\uM}{\uepsilon} 
& \apmod{\uepsilon}{\uX} & \text{module application}\ECC
\end{array}
\]
\caption{Syntax of unexpanded TSM types and expressions.}
\label{fig:P-macro-expressions-types-u}
\end{figure}
\begin{figure}[t]
\[\begin{array}{lllllll}
\textbf{Sort} & & & \textbf{Operational Form} 
%& \textbf{Stylized Form} 
& \textbf{Description}\\
\LCC \color{Yellow}&\color{Yellow}& \color{Yellow}
%& \lightgray 
& \color{Yellow} & \color{Yellow}\\
\mathsf{MType} & \rho & ::= & \aetype{\tau} 
%& \tau 
& \text{type annotation}\\
% &&& \aealltypes{t}{\rho} 
%& \alltypes{t}{\rho} 
% & \text{type parameterization}\\
&&& \aeallmods{\sigma}{X}{\rho} 
%& \allmods{X}{\sigma}{\rho} 
& \text{module parameterization}\\
\mathsf{MExp} & \epsilon & ::= & \adefref{a} 
%& a 
& \text{TSM definition reference}\\
% &&& \aeabstype{t}{\epsilon} 
%& \abstype{t}{\epsilon} 
% & \text{type abstraction}\\
&&& \aeabsmod{\sigma}{X}{\epsilon} 
%& \absmod{X}{\sigma}{\epsilon} 
& \text{module abstraction}\\
% &&& \aeaptype{\tau}{\epsilon} 
%& \aptype{\epsilon}{\tau} 
% & \text{type application}\\
&&& \aeapmod{M}{\epsilon} 
%& \aptype{\epsilon}{M} 
& \text{module application}\ECC
\end{array}\]
\caption[Syntax of TSM types and expressions in $\miniVerseParam$]{Syntax of TSM types and expressions.}
\label{fig:P-macro-expressions-types}
\end{figure}
\subsection{Syntax of the Unexpanded Language}
The syntax of the unexpanded language is defined in Figures \ref{fig:P-unexpanded-modules-signatures} through \ref{fig:P-macro-expressions-types}.

Each expanded form, with three exceptions, has a corresponding unexpanded form. We refer to these as the \emph{common forms}. The correspondence is defined in Appendix \ref{appendix:P-shared-forms}.

Kind variables, $k$, are one exception. Kind variables are used only in the metatheory.

The other two exceptions are constructions of the form $\amcon{M}$ and expressions of the form $\amval{M}$ where $M$ is of the form $\astruct{c}{e}$. Projection out of a module expression of the form $\astruct{c}{e}$ was supported in the XL only because this is needed to give the language  a conventional structural dynamics. Programmers refer to modules exclusively through module identifiers in unexpanded programs. 

In addition to the common forms, there are several forms related to pTSMs, highlighted in yellow in these figures. We need syntax for unexpanded TSM types, $\urho$, and unexpanded TSM expressions, $\uepsilon$, to support parameterization and parameter application. Internally, these expand to TSM expressions, $\epsilon$, and TSM types, $\rho$, respectively.

There is also a context-free textual syntax for the UL. For our purposes, we need only posit the existence of partial metafunctions that satisfy the following condition. 
\begingroup
\def\thetheorem{\ref{condition:textual-representability-P}}
\begin{condition}[Textual Representability] All of the following must hold:
\begin{enumerate}
% \item For each $\usigma$, there exists $b$ such that $\parseUSig{b}{\usigma}$.
% \item For each $\uM$, there exists $b$ such that $\parseUMod{b}{\uM}$.
\item For each $\ukappa$, there exists $b$ such that $\parseUKind{b}{\ukappa}$.
\item For each $\uc$, there exists $b$ such that $\parseUCon{b}{\uc}$.
\item For each $\ue$, there exists $b$ such that $\parseUExp{b}{\ue}$.
\item For each $\upv$, there exists $b$ such that $\parseUPat{b}{\upv}$.
\end{enumerate}
\end{condition}
\endgroup


\subsection{Typed Expansion}

Typed expansion is defined by six groups of judgements. In these judgements, \emph{unexpanded unified contexts}, $\uOmega$, take the form $\uOmegaEx{\uD}{\uG}{\uMctx}{\Omega}$, where $\uMctx$ is a \emph{module identifier expansion context}, $\uD$ is a \emph{construction identifier expansion context}, $\uG$ is an \emph{expression identifier expansion context} and $\Omega$ is a unified context. Identifier expansion contexts are defined in Appendix \ref{appendix:u-unified-ctxs} and conceptually operate as described in Sec. \ref{sec:miniVerseU}, mapping identifiers to variables.

The first group of judgements defines signature and module expansion.

\vspace{6px}
$\begin{array}{ll}
\textbf{Judgement Form} & \textbf{Description}\\
\sigExpandsPX{\usigma}{\sigma} & \text{$\usigma$ has well-formed expansion $\sigma$}\\
\mExpandsPX{\uM}{M}{\sigma} & \text{$\uM$ has expansion $M$ matching $\sigma$}
\end{array}$
\vspace{6px}

The second group of judgements defines kind and construction expansion.

\vspace{6px}
$\begin{array}{ll}
\textbf{Judgement Form} & \textbf{Description}\\
\kExpandsX{\ukappa}{\kappa} & \text{$\ukappa$ has well-formed expansion $\kappa$}\\
\cExpandsX{\uc}{c}{\kappa} & \text{$\uc$ has expansion $c$ of kind $\kappa$}
\end{array}$
\vspace{6px}

The third group of judgements defines expression, rule and pattern expansion.

\vspace{6px}
$\begin{array}{ll}
\textbf{Judgement Form} & \textbf{Description}\\
% \tExpandsPX{\utau}{\tau} & \text{$\utau$ has well-formed expansion $\tau$}\\
\expandsPX{\ue}{e}{\tau} & \text{$\ue$ has expansion $e$ of type $\tau$}\\
% \eanaPX{\ue}{e}{\tau} & \text{$\ue$ has expansion $e$ when analyzed against type $\tau$}\\
\rExpandsSP{\uOmega}{\uPsi}{\uPhi}{\urv}{r}{\tau}{\tau'} & \text{$\urv$ has expansion $r$ taking values of type $\tau$ to values of type $\tau'$}\\
% & \text{synthesized type $\tau'$}\\
% \ranaPX{\urv}{r}{\tau}{\tau'} & \text{$\urv$ has expansion $r$ and takes values of type $\tau$ to values of}\\
% & \text{type $\tau'$ when $\tau's$ is provided for analysis}\\
\patExpandsP{\uOmega'}{\uPhi}{\upv}{p}{\tau} & \text{$\upv$ has expansion $p$ matching at $\tau$  generating hypotheses $\uOmega'$}
\end{array}$
\vspace{6px}

The judgements above are defined by the rules given in Appendix \ref{appendix:typed-expansion-P}. Most of these rules simply serve to ``mirror'' corresponding rules in the statics of the XL, as was described in Sec. \ref{sec:miniVerseU}. The interesting rules, governing the forms highlighted in yellow, will be reproduced as we discuss them below.

The remaining judgements assign meaning to TSM types and expressions. We will detail these below. In particular, the fourth group of judgements define TSM type and expression expansion.

\vspace{6px}
$\begin{array}{ll}
\textbf{Judgement Form} & \textbf{Description}\\
\tsmtyExpands{\uOmega}{\urho}{\rho} & \text{$\urho$ has well-formed expansion $\rho$}\\
\tsmexpExpandsExp{\uOmega}{\uPsi}{\uepsilon}{\epsilon}{\rho} & \text{$\uepsilon$ has peTSM expression expansion $\epsilon$ at $\rho$}\\
\tsmexpExpandsPat{\uOmega}{\uPsi}{\uepsilon}{\epsilon}{\rho} & \text{$\uepsilon$ has ppTSM expression expansion $\epsilon$ at $\rho$}
\end{array}$
\vspace{6px}

The fifth group of judgements define the statics of TSM expressions.

\vspace{6px}
$\begin{array}{ll}
\textbf{Judgement Form} & \textbf{Description}\\
\istsmty{\Omega}{\rho} & \text{$\rho$ is a TSM type}\\
\hastsmtypeExp{\Omega}{\Psi}{\epsilon}{\rho} & \text{$\epsilon$ is a peTSM expression at $\rho$}\\
\hastsmtypePat{\Omega}{\Phi}{\epsilon}{\rho} & \text{$\epsilon$ is a ppTSM expression at $\rho$}
\end{array}$
\vspace{6px}

The sixth group of judgements define the dynamics of TSM expressions.

\vspace{6px}
$\begin{array}{ll}
\textbf{Judgement Form} & \textbf{Description}\\
\tsmexpStepsExp{\Omega}{\Psi}{\epsilon}{\epsilon'} & \text{peTSM expression $\epsilon$ transitions to $\epsilon'$}\\
\tsmexpStepsPat{\Omega}{\Psi}{\epsilon}{\epsilon'} & \text{ppTSM expression $\epsilon$ transitions to $\epsilon'$}\\
\tsmexpNormalExp{\Omega}{\Psi}{\epsilon} & \text{$\epsilon$ is a normal peTSM expression}\\
\tsmexpNormalPat{\Omega}{\Psi}{\epsilon} & \text{$\epsilon$ is a normal ppTSM expression}
\end{array}$
\vspace{6px}

We define the multi-step transition judgements $\tsmexpMultistepsExp{\Omega}{\Psi}{\epsilon}{\epsilon'}$ and $\tsmexpMultistepsPat{\Omega}{\Phi}{\epsilon}{\epsilon'}$ as the reflexive transitive closures of the corresponding transition judgements. We also define the peTSM expression normalization judgement $\tsmexpEvalsExp{\Omega}{\Psi}{\epsilon}{\epsilon'}$ iff $\tsmexpMultistepsExp{\Omega}{\Psi}{\epsilon}{\epsilon'}$ and $\tsmexpNormalExp{\Omega}{\Psi}{\epsilon'}$. Similarly, we define the ppTSM expression normalization judgement $\tsmexpEvalsPat{\Omega}{\Phi}{\epsilon}{\epsilon'}$ iff $\tsmexpMultistepsPat{\Omega}{\Phi}{\epsilon}{\epsilon'}$ and $\tsmexpNormalPat{\Omega}{\Phi}{\epsilon'}$.

\subsection{TSM Definitions}
TSMs are scoped to module expressions. (Adding support for TSM definitions scoped to a single expression would be a straightforward exercise, so we omit the details for simplicity.)

\subsubsection{peTSM Definitions}
The rule governing peTSM definitions is reproduced below:
\begin{equation*}\tag{\ref{rule:mExpandsP-syntaxpe}}
\inferrule{
  \tsmtyExpands{\uOmega}{\urho}{\rho}\\
  \hastypeP{\emptyset}{\eparse}{\aparr{\tBody}{\tParseResultPCEExp}}\\\\
  \evalU{\eparse}{\eparse'}\\
  \mExpandsP{\uOmega}{\uAS{\uA \uplus \mapitem{\tsmv}{\adefref{a}}}{\Psi, \petsmdefn{a}{\rho}{\eparse'}}}{\uPhi}{\uM}{M}{\sigma}
}{
  \mExpandsP{\uOmega}{\uAS{\uA}{\Psi}}{\uPhi}{\defpetsm{\tsmv}{\urho}{\eparse}{\uM}}{M}{\sigma}
}
\end{equation*}

peTSM definitions differ from ueTSM definitions in that the unexpanded type annotation is an \emph{unexpanded TSM type}, $\urho$, rather than an unexpanded type, $\utau$. This unexpanded TSM type determines the parameterization of the TSM. The first premise of the rule above expands the unexpanded TSM type to produce a \emph{TSM type}, $\rho$. The straightforward rules governing TSM type expansion are reproduced below.
\begin{equation*}\tag{\ref{rule:tsmtyExpands-type}}
\inferrule{
  \cExpandsX{\utau}{\tau}{\akty}
}{
  \tsmtyExpands{\uOmega}{{\utau}}{\aetype{\tau}}
}
\end{equation*}
% \begin{equation*}\tag{\ref{rule:tsmtyExpands-alltypes}}
% \inferrule{
%   \tsmtyExpands{\uOmega, \uKhyp{\ut}{t}{\akty}}{\urho}{\rho}
% }{
%   \tsmtyExpands{\uOmega}{\alltypes{\ut}{\urho}}{\aealltypes{t}{\rho}}
% }
% \end{equation*}
\begin{equation*}\tag{\ref{rule:tsmtyExpands-allmods}}
\inferrule{
  \sigExpandsPX{\usigma}{\sigma}\\
  \tsmtyExpands{\uOmega, \uMhyp{\uX}{X}{\sigma}}{\urho}{\rho}
}{
  \tsmtyExpands{\uOmega}{\allmods{\uX}{\usigma}{\urho}}{\aeallmods{\sigma}{X}{\rho}}
}
\end{equation*}
Rule (\ref{rule:tsmtyExpands-type}) defines quantification over modules matching a given signature. There is no mechanism for quantification over types in the calculus because it can be understood as quantification over a module with a single type component.

The second premise of Rule (\ref{rule:mExpandsP-syntaxpe}) checks that the parse function is of the appropriate type. The types $\tBody$ and $\tParseResultPCEExp$ are characterized in Appendix \ref{appendix:typed-expansion-P}. The type $\tPProtoExpr$ classifies \emph{encodings of parameterized proto-expressions}, which we will return to when we discuss TSM application below.

The third premise of Rule (\ref{rule:mExpandsP-syntaxpe}) evaluates the parse function to a value.

The final premise of Rule (\ref{rule:mExpandsP-syntaxpe}) extends the \emph{peTSM context}, $\uPsi$, which consists of a \emph{TSM identifier expansion context}, $\uA$, and a \emph{peTSM definition context}, $\Psi$. A peTSM definition context maps TSM names, $a$, to an expanded peTSM definition, $\petsmdefn{a}{\rho}{\eparse}$, where $\rho$ is the TSM type determined from the annotation and $\eparse$ is its parse function. A TSM identifier context maps TSM identifiers, $\tsmv$, to \emph{TSM expressions}, $\epsilon$. In this case, the TSM expression is simply a reference to the newly introduced TSM definition, $\adefref{a}$. We discuss the other TSM expression forms when we discuss TSM abbreviations below.

\subsubsection{ppTSM Definitions}
The rule governing ppTSM definitions is similar, and is reproduced below:
\begin{equation*}\tag{\ref{rule:mExpandsP-syntaxpp}}
\inferrule{ 
  \tsmtyExpands{\uOmega}{\urho}{\rho}\\
  \hastypeP{\emptyset}{\eparse}{\aparr{\tBody}{\tParseResultCEPat }}\\\\
  \evalU{\eparse}{\eparse'}\\
  \mExpandsP{\uOmega}{\uPsi}{\uAS{\uA \uplus \mapitem{\tsmv}{\adefref{a}}}{\Phi, \pptsmdefn{a}{\rho}{\eparse'}}}{\uM}{M}{\sigma}
}{
  \mExpandsP{\uOmega}{\uPsi}{\uAS{\uA}{\Phi}}{\defpptsm{\tsmv}{\urho}{\eparse}{\uM}}{M}{\sigma}
}
\end{equation*}
This rule differs from Rule (\ref{rule:mExpandsP-syntaxpe}) in the type of the parse function and in the fact that the \emph{ppTSM context}, $\uPhi$, rather than the peTSM context, is updated.

\subsection{TSM Abbreviations}
It is possible to abbreviate a complex TSM expression by binding it to a TSM identifier.

\subsubsection{peTSM Abbreviations}
The rule governing peTSM abbreviations is reproduced below:
\begin{equation}\tag{\ref{rule:mExpandsP-letpetsm}}
\inferrule{
  \tsmexpExpandsExp{\uOmega}{\uAS{\uA}{\Psi}}{\uepsilon}{\epsilon}{\rho}\\
  \mExpandsP{\uOmega}{\uAS{\uA\uplus\mapitem{\tsmv}{\epsilon}}{\Psi}}{\uPhi}{\uM}{M}{\sigma}
}{
  \mExpandsP{\uOmega}{\uAS{\uA}{\Psi}}{\uPhi}{\uletpetsm{\tsmv}{\uepsilon}{\uM}}{M}{\sigma}
}
\end{equation}
Here, $\uepsilon$ is an \emph{unexpanded TSM expression}. The first premise of the rule above expands it, producing a TSM expression $\epsilon$ at TSM type $\rho$. The second premise updates the peTSM identifier expansion context with this TSM expression.

The rules below govern peTSM expression expansion. The first rule handles the base case, when the unexpanded TSM expression is a TSM identifier, $\tsmv$, by looking it up in $\uA$ and determining its TSM type according to the TSM expression typing judgement, $\hastsmtypeExp{\Omega}{\Psi}{\epsilon}{\rho}$ (which mirrors the rules below, and is defined in Appendix \ref{appendix:typed-expansion-P}.)
\begin{equation*}\tag{\ref{rule:tsmexpExpandsExp-bindref}}
\inferrule{
  \hastsmtypeExp{\Omega}{\Psi}{\epsilon}{\rho}  
}{
  \tsmexpExpandsExp{\uOmegaEx{\uD}{\uG}{\uMctx}{\Omega}}{\uAS{\uA, \mapitem{\tsmv}{\epsilon}}{\Psi}}{{\tsmv}}{\epsilon}{\rho}
}
\end{equation*}

The following rule allows a peTSM expression to itself abstract over a module. (This is necessary to support abbreviated application of parameters other than the first.)
% \begin{equation*}\tag{\ref{rule:tsmexpExpandsExp-abstype}}
% \inferrule{
%   \tsmexpExpandsExp{\uOmega, \uKhyp{\ut}{t}{\akty}}{\uPsi}{\uepsilon}{\epsilon}{\rho}
% }{
%   \tsmexpExpandsExp{\uOmega}{\uPsi}{\abstype{\ut}{\uepsilon}}{\aeabstype{t}{\epsilon}}{\aealltypes{t}{\rho}}
% }
% \end{equation*}
\begin{equation*}\tag{\ref{rule:tsmexpExpandsExp-absmod}}
\inferrule{
  \sigExpandsPX{\usigma}{\sigma}\\
  \tsmexpExpandsExp{\uOmega, \uMhyp{\uX}{X}{\sigma}}{\uPsi}{\uepsilon}{\epsilon}{\rho}
}{
  \tsmexpExpandsExp{\uOmega}{\uPsi}{\absmod{\uX}{\usigma}{\uepsilon}}{\aeabsmod{\sigma}{X}{\epsilon}}{\aeallmods{\sigma}{X}{\rho}}
}
\end{equation*}

The final rule defines the semantics of parameter application.
% \begin{equation*}\tag{\ref{rule:tsmexpExpandsExp-aptype}}
% \inferrule{
%   \tsmexpExpandsExp{\uOmega}{\uPsi}{\uepsilon}{\epsilon}{\aealltypes{t}{\rho}}\\
%   \cExpandsX{\utau}{\tau}{\akty}
% }{
%   \tsmexpExpandsExp{\uOmega}{\uPsi}{\aptype{\uepsilon}{\utau}}{\aeaptype{\tau}{\epsilon}}{[\tau/t]\rho} 
% }
% \end{equation*}
\begin{equation*}\tag{\ref{rule:tsmexpExpandsExp-apmod}}
\inferrule{
  \tsmexpExpandsExp{\uOmega}{\uPsi}{\uepsilon}{\epsilon}{\aeallmods{\sigma}{X'}{\rho}}\\
  \mExpandsPX{\uX}{X}{\sigma}
}{
  \tsmexpExpandsExp{\uOmega}{\uPsi}{\apmod{\uepsilon}{\uX}}{\aeapmod{X}{\epsilon}}{[X/X']\rho}
}
\end{equation*}

\subsubsection{ppTSM Abbreviations}
The rule governing ppTSM abbreviations is analagous:
\begin{equation*}\tag{\ref{rule:mExpandsP-letpptsm}}
\inferrule{
  \tsmexpExpandsPat{\uOmega}{\uAS{\uA}{\Phi}}{\uepsilon}{\epsilon}{\rho}\\
  \mExpandsP{\uOmega}{\uPsi}{\uAS{\uA\uplus\mapitem{\tsmv}{\epsilon}}{\Phi}}{\uM}{M}{\sigma}
}{
  \mExpandsP{\uOmega}{\uPsi}{\uAS{\uA}{\Phi}}{\uletpptsm{\tsmv}{\uepsilon}{\uM}}{M}{\sigma}
}
\end{equation*}
The ppTSM expression expansion judgement appearing as the first premise is defined analagously to the peTSM expression expansion judgement defined above, differing only in that the rule for TSM identifiers consults the ppTSM context rather than the peTSM context. The rules are reproduced in Appendix \ref{appendix:typed-expansion-P}.

\subsection{TSM Application}

\subsubsection{peTSM Application}
The rule for applying an unexpanded peTSM expression $\uepsilon$ to a generalized literal form with body $b$ is reproduced below:
\begin{equation*}\tag{\ref{rule:expandsP-apuetsm}}
\inferrule{
  \uOmega = \uOmegaEx{\uD}{\uG}{\uMctx}{\Omega_\text{app}}\\
  \uPsi=\uAS{\uA}{\Psi}\\\\
  \tsmexpExpandsExp{\uOmega}{\uPsi}{\uepsilon}{\epsilon}{\aetype{\tau_\text{final}}}\\
  \tsmexpEvalsExp{\Omega_\text{app}}{\Psi}{\epsilon}{\epsilon_\text{normal}}\\\\
  \tsmdefof{\epsilon_\text{normal}}=a\\
  \Psi = \Psi', \petsmdefn{a}{\rho}{\eparse}\\\\
  \encodeBody{b}{\ebody}\\
  \evalU{\ap{\eparse}{\ebody}}{\aein{\mathtt{SuccessE}}{e_\text{pproto}}}\\
  \decodePCEExp{e_\text{pproto}}{\pce}\\\\
  \prepce{\Omega_\text{app}}{\Psi}{\pce}{\ce}{\epsilon_\text{normal}}{\aetype{\tau_\text{proto}}}{\omega}{\Omega_\text{params}}\\\\
  \segOK{\segof{\ce}}{b}\\
  \cvalidEP{\Omega_\text{params}}{\esceneP{\omega : \OParams}{\uOmega}{\uPsi}{\uPhi}{b}}{\ce}{e}{\tau_\text{proto}}
}{
  \expandsP{\uOmega}{\uPsi}{\uPhi}{\utsmap{\uepsilon}{b}}{[\omega]e}{[\omega]\tau_\text{proto}}
}
\end{equation*}

The first two premises simply deconstruct $\uOmega$ and $\uPsi$. Next, we expand $\uepsilon$ according to the unexpanded peTSM expression expansion rules that we already described above. The resulting TSM expression, $\epsilon$, must be defined at a type (i.e. no quantification must remain.)

The fourth premise performs \emph{peTSM expression normalization}. Normalization is defined in terms of a simple structural dynamics with two stepping rules:
% \begin{equation*}\tag{\ref{rule:tsmexpEvalsExp}}
% \inferrule{
%   \tsmexpMultistepsExp{\Omega}{\Psi}{\epsilon}{\epsilon'}\\
%   \tsmexpNormalExp{\Omega}{\Psi}{\epsilon'}
% }{
%   \tsmexpEvalsExp{\Omega}{\Psi}{\epsilon}{\epsilon'}
% }
% \end{equation*}
% where the multistep judgement, $\tsmexpMultistepsExp{\Omega}{\Psi}{\epsilon}{\epsilon'}$, is defined as the reflexive, transitive closure of the stepping judgement defined by the following rules:
% \begin{equation*}\tag{\ref{rule:tsmexpStepsExp-aptype-1}}
% \inferrule{
%   \tsmexpStepsExp{\Omega}{\Psi}{\epsilon}{\epsilon'}
% }{
%   \tsmexpStepsExp{\Omega}{\Psi}{\aeaptype{\tau}{\epsilon}}{\aeaptype{\tau}{\epsilon'}}
% }
% \end{equation*}
% \begin{equation*}\tag{\ref{rule:tsmexpStepsExp-aptype-2}}
% \inferrule{ }{
%   \tsmexpStepsExp{\Omega}{\Psi}{\aeaptype{\tau}{\aeabstype{t}{\epsilon}}}{[\tau/t]\epsilon}
% }
% \end{equation*}
\begin{equation*}\tag{\ref{rule:tsmexpStepsExp-apmod-1}}
\inferrule{
  \tsmexpStepsExp{\Omega}{\Psi}{\epsilon}{\epsilon'}
}{
  \tsmexpStepsExp{\Omega}{\Psi}{\aeapmod{X}{\epsilon}}{\aeapmod{X}{\epsilon'}}
}
\end{equation*}
\begin{equation*}\tag{\ref{rule:tsmexpStepsExp-apmod-2}}
\inferrule{ }{
  \tsmexpStepsExp{\Omega}{\Psi}{\aeapmod{X}{\aeabsmod{\sigma}{X'}{\epsilon}}}{[X/X']\epsilon}
}
\end{equation*}
The peTSM expression normal forms are defined as follows:
\begin{equation*}\tag{\ref{rule:tsmexpNormalExp-defref}}
\inferrule{ }{
  \tsmexpNormalExp{\Omega}{\Psi, \petsmdefn{a}{\rho}{\eparse}}{\adefref{a}}
}
\end{equation*}
% \begin{equation*}\tag{\ref{rule:tsmexpNormalExp-abstype}}
% \inferrule{ }{
%   \tsmexpNormalExp{\Omega}{\Psi}{\aeabstype{t}{\epsilon}}
% }
% \end{equation*}
\begin{equation*}\tag{\ref{rule:tsmexpNormalExp-absmod}}
\inferrule{ }{
  \tsmexpNormalExp{\Omega}{\Psi}{\aeabsmod{\sigma}{X}{\epsilon}}
}
\end{equation*}
% \begin{equation*}\tag{\ref{rule:tsmexpNormalExp-aptype}}
% \inferrule{
%   \epsilon \neq \aeabstype{t}{\epsilon'}\\
%   \tsmexpNormalExp{\Omega}{\Psi}{\epsilon}
% }{
%   \tsmexpNormalExp{\Omega}{\Psi}{\aeaptype{\tau}{\epsilon}}
% }
% \end{equation*}
\begin{equation*}\tag{\ref{rule:tsmexpNormalExp-apmod}}
\inferrule{
  \epsilon \neq \aeabsmod{\sigma}{X'}{\epsilon'}\\
  \tsmexpNormalExp{\Omega}{\Psi}{\epsilon}
}{
  \tsmexpNormalExp{\Omega}{\Psi}{\aeapmod{X}{\epsilon}}
}
\end{equation*}
Normalization leaves only those parameter applications that cannot be reduced away immediately, i.e. those specified by the original TSM definition.

The TSM definition at the root of the normalized TSM expression is extracted by the third row of premises in Rule (\ref{rule:expandsP-apuetsm}). The first of these appeals to the following metafunction to produce the TSM definition's name.
\begin{align}
\tsmdefof{\adefref{a}} & = a \tag{\ref{eqn:tsmdefof-adefref}}\\
% \tsmdefof{\aeabstype{t}{\epsilon}} & = \tsmdefof{\epsilon} \tag{\ref{eqn:tsmdefof-abstype}}\\
\tsmdefof{\aeabsmod{\sigma}{X}{\epsilon}} & = \tsmdefof{\epsilon} \tag{\ref{eqn:tsmdefof-absmod}}\\
% \tsmdefof{\aeaptype{\tau}{\epsilon}} & = \tsmdefof{\epsilon} \tag{\ref{eqn:tsmdefof-aptype}}\\
\tsmdefof{\aeapmod{X}{\epsilon}} & = \tsmdefof{\epsilon} \tag{\ref{eqn:tsmdefof-apmod}}
\end{align}
The second premise on the third row then looks up this name within $\Psi$.

The fourth row of premises in Rule (\ref{rule:expandsP-apuetsm}) 1) encode the body as a value of the type $\tBody$; 2) apply the parse function; and 3) decode the result, producing a \emph{parameterized proto-expression}, $\pce$. Parameterized proto-expressions, $\pce$, are ABTs that serve to introduce the parameter bindings into a proto-expression, $\ce$. The operational and stylized syntax of parameterized proto-expression is given in Figure \ref{fig:P-pceexp}. 

\begin{figure}[h]
\[\begin{array}{lllllll}
\textbf{Sort} & & & \textbf{Operational Form} & \textbf{Stylized Form} & \textbf{Description}\\
\LCC \color{Yellow}&\color{Yellow}&\color{Yellow}& \color{Yellow} & \color{Yellow} & \color{Yellow}\\
\mathsf{PPrExpr} & \pce & ::= & \apceexp{\ce} & \pceexp{\ce} & \text{proto-expression}\\
% &&& \apcebindtype{t}{\pce} & \pcebindtype{t}{\pce} & \text{type binding}\\
&&& \apcebindmod{X}{\pce} & \pcebindmod{X}{\pce} & \text{module binding}\ECC
\end{array}\]
\caption[Syntax of parameterized proto-expressions in $\miniVerseParam$]{Syntax of parameterized proto-expressions.}
\label{fig:P-pceexp}
\end{figure}
\noindent 
There must be one binder in $\pce$ for each TSM parameter specified by $\tsmdefof{\epsilon_\text{normal}}$. (VerseML inserts these binders automatically as a convenience, but we consider only the underlying mechanism in this core calculus.) The judgement on the fifth row of Rule (\ref{rule:expandsP-apuetsm}) then \emph{deparameterizes} $\pce$ by peeling away these binders to produce 1) the underlying proto-expression, $\ce$, with the variables that stand for the parameters free; 2) a corresponding deparameterized type, $\tau_\text{proto}$, that uses the same free variables to stand for the parameters; 3) a \emph{substitution}, $\omega$, that pairs the applied parameters from $\epsilon_\text{normal}$ with the corresponding variables generated when peeling away the binders in $\pce$; and 4) a corresponding \emph{parameter context}, $\Omega_\text{params}$, that tracks the signatures of these variables. The two rules governing the proto-expression deparameterization judgement are reproduced below:
\begin{equation*}\tag{\ref{rule:prepce-ceexp}}
\inferrule{ }{
  \prepce{\Omega_\text{app}}{\Psi, \petsmdefn{a}{\rho}{\eparse}}{\apceexp{\ce}}{\ce}{\adefref{a}}{\rho}{\emptyset}{\emptyset}
}
\end{equation*}
% \begin{equation*}\tag{\ref{rule:prepce-alltypes}}
% \inferrule{
%   \prepce{\Omega_\text{app}}{\Psi}{\pce}{\ce}{\epsilon}{\aealltypes{t}{\rho}}{\omega}{\Omega}\\
%   t \notin \domof{\Omega_\text{app}}
% }{
%   \prepce{\Omega_\text{app}}{\Psi}{\apcebindtype{t}{\pce}}{\ce}{\aeaptype{\tau}{\epsilon}}{\rho}{\omega, \tau/t}{\Omega, t :: \akty}
% }
% \end{equation*}
\begin{equation*}\tag{\ref{rule:prepce-allmods}}
\inferrule{
  \prepce{\Omega_\text{app}}{\Psi}{\pce}{\ce}{\epsilon}{\aeallmods{\sigma}{X}{\rho}}{\omega}{\Omega}\\
  X \notin \domof{\Omega_\text{app}}
}{
  \prepce{\Omega_\text{app}}{\Psi}{\apcebindmod{X}{\pce}}{\ce}{\aeapmod{X'}{\epsilon}}{\rho}{(\omega, X'/X)}{(\Omega, X : \sigma)}
}
\end{equation*}
This judgement can be pronounced ``when applying peTSM $\epsilon$, $\pce$ has deparameterization $\ce$ leaving $\rho$ with parameter substitution $\omega$''. Notice from Rule (\ref{rule:prepce-allmods}) that every module binding in $\pce$ must pair with a corresponding module parameter application. Moreover, the variables standing for parameters must not appear in $\Omega_\text{app}$, i.e. $\domof{\Omega_\text{params}}$ must be disjoint from $\domof{\Omega_\text{app}}$ (this requirement can always be discharged by alpha-variation.)

The final row of premises in Rule (\ref{rule:expandsP-apuetsm}) performs proto-expansion validation. This involves first checking that the segmentation of $\ce$ is valid, and then checking that the proto-expansion is well-typed under the parameter context, $\Omega_\text{param}$ (rather than the empty context, as was the case in $\miniVersePat$.) The conclusion of the rule applies the parameter substitution, $\omega$, to the resulting expression and the deparameterized type it was checked against. 

\subsubsection{ppTSM Application}

The rule governing ppTSM application is similar:
\begin{equation*}\tag{\ref{rule:patExpandsP-apuptsm}}
\inferrule{
  \uOmega=\uOmegaEx{\uD}{\uG}{\uMctx}{\Omega_\text{app}}\\
  \uPhi=\uAS{\uA}{\Phi}\\\\
  \tsmexpExpandsPat{\uOmega}{\uPhi}{\uepsilon}{\epsilon}{\aetype{\tau_\text{final}}}\\
  \tsmexpEvalsPat{\Omega_\text{app}}{\Phi}{\epsilon}{\epsilon_\text{normal}}\\\\
  \tsmdefof{\epsilon_\text{normal}}=a\\
  \Phi = \Phi', \pptsmdefn{a}{\rho}{\eparse}\\\\
  \encodeBody{b}{\ebody}\\
  \evalU{\ap{\eparse}{\ebody}}{\aein{\mathtt{SuccessP}}{e_\text{pproto}}}\\
  \decodePCEPat{e_\text{pproto}}{\pcp}\\\\
  \prepcp{\Omega_\text{app}}{\Phi}{\pcp}{\cpv}{\epsilon_\text{normal}}{\aetype{\tau_\text{proto}}}{\omega}{\Omega_\text{params}}\\\\
      \segOK{\segof{\cpv}}{b}\\
  \cvalidPP{\uOmega'}{\psceneP{\omega : \Omega_\text{params}}{\uOmega}{\uPhi}{b}}{\cpv}{p}{\tau_\text{proto}}
}{
  \patExpandsP{\uOmega'}{\uPhi}{\utsmap{\uepsilon}{b}}{p}{[\omega]\tau_\text{proto}}
}
\end{equation*}

Although patterns themselves cannot make reference to surrounding bindings, the type annotations on spliced patterns can, so we need the notion of a \emph{parameterized proto-pattern}, $\pcp$, and a corresponding deparameterization judgement. The necessary definitions, which are analagous to those given above for peTSMs, are given in Appendix \ref{appendix:typed-expansion-P}.

\subsection{Syntax of Proto-Expansions}\label{sec:ce-syntax-P}

\begin{figure}[p] 
\[\begin{array}{lrlllll}
\textbf{Sort} & & & \textbf{Operational Form} & \textbf{Stylized Form} & \textbf{Description}\\
\mathsf{PrKind} & \cekappa & ::= & \acekdarr{\cekappa}{u}{\cekappa} & \kdarr{u}{\cekappa}{\cekappa} & \text{dependent function}\\
&&& \acekunit & \kunit & \text{nullary product}\\
&&& \acekdbprod{\cekappa}{u}{\cekappa} & \kdbprod{u}{\cekappa}{\cekappa} & \text{dependent product}\\
%&&& \akdprodstd & \kdprodstd & \text{labeled dependent product}\\
&&& \acekty & \kty & \text{type}\\
&&& \aceksing{\ctau} & \ksing{\ctau} & \text{singleton}\\
\LCC &&& \color{Yellow} & \color{Yellow} & \color{Yellow}\\
&&& \acesplicedk{m}{n} & \splicedk{m}{n} & \text{spliced kind}\ECC\\
\mathsf{PrCon} & \cec, \ctau & ::= & u & u & \text{construction variable}\\
&&& t & t & \text{type variable}\\
% &&& \acecasc{\cekappa}{\cec} & \casc{\cec}{\cekappa} & \text{ascription}\\
&&& \acecabs{u}{\cec} & \cabs{u}{\cec} & \text{abstraction}\\
&&& \acecapp{\cec}{\cec} & \capp{\cec}{\cec} & \text{application}\\
&&& \acectriv & \ctriv & \text{trivial}\\
&&& \acecpair{\cec}{\cec} & \cpair{\cec}{\cec} & \text{pair}\\
&&& \acecprl{\cec} & \cprl{\cec} & \text{left projection}\\
&&& \acecprr{\cec} & \cprr{\cec} & \text{right projection}\\
%&&& \adtplX & \dtplX & \text{labeled dependent tuple}\\
%&&& \adprj{\ell}{c} & \prj{c}{\ell} & \text{projection}\\
&&& \aceparr{\ctau}{\ctau} & \parr{\ctau}{\ctau} & \text{partial function}\\
&&& \aceallu{\cekappa}{u}{\ctau} & \forallu{u}{\cekappa}{\ctau} & \text{polymorphic}\\
&&& \acerec{t}{\ctau} & \rect{t}{\ctau} & \text{recursive}\\
&&& \aceprod{\labelset}{\mapschema{\ctau}{i}{\labelset}} & \prodt{\mapschema{\ctau}{i}{\labelset}} & \text{labeled product}\\
&&& \acesum{\labelset}{\mapschema{\ctau}{i}{\labelset}} & \sumt{\mapschema{\ctau}{i}{\labelset}} & \text{labeled sum}\\
&&& \acemcon{X} & \mcon{X} & \text{construction component}\\
\LCC &&& \color{Yellow} & \color{Yellow} & \color{Yellow}\\
&&& \acesplicedc{m}{n}{\cekappa} & \splicedc{m}{n}{\cekappa} & \text{spliced construction}\ECC
\end{array}\]
\caption[Syntax of proto-kinds and proto-constructions in $\miniVerseParam$]{Syntax of proto-kinds and proto-constructions in $\miniVerseParam$.}
\label{fig:P-ce-kinds-constructors}
\end{figure}

\begin{figure}[p]
\[\arraycolsep=4pt\begin{array}{lllllll}
\textbf{Sort} & & & \textbf{Operational Form} & \textbf{Stylized Form} & \textbf{Description}\\
\mathsf{PrExp} & \ce & ::= & x & x & \text{variable}\\
&&& \aceasc{\ctau}{\ce} & \asc{\ce}{\ctau} & \text{ascription}\\
&&& \aceletsyn{x}{\ce}{\ce} & \letsyn{x}{\ce}{\ce} & \text{value binding}\\
% &&& \aceasc{\ctau}{\ce} & \asc{\ce}{\ctau} & \text{ascription}\\
% &&& \aceletsyn{x}{\ce}{\ce} & \letsyn{x}{\ce}{\ce} & \text{value binding}\\
% &&& \aceanalam{x}{\ce} & \analam{x}{\ce} & \text{abstraction (unannotated)}\\
&&& \acelam{\ctau}{x}{\ce} & \lam{x}{\ctau}{\ce} & \text{abstraction}\\
&&& \aceap{\ce}{\ce} & \ap{\ce}{\ce} & \text{application}\\
&&& \aceclam{\cekappa}{u}{\ce} & \clam{u}{\cekappa}{\ce} & \text{construction abstraction}\\
&&& \acecap{\ce}{\cec} & \cAp{\ce}{\cec} & \text{construction application}\\
&&& \acefold{\ce} & \fold{\ce} & \text{fold}\\
&&& \aceunfold{\ce} & \unfold{\ce} & \text{unfold}\\
&&& \acetpl{\labelset}{\mapschema{\ce}{i}{\labelset}} & \tpl{\mapschema{\ce}{i}{\labelset}} & \text{labeled tuple}\\
&&& \acepr{\ell}{\ce} & \prj{\ce}{\ell} & \text{projection}\\
&&& \aceanain{\ell}{\ce} & \inj{\ell}{\ce} & \text{injection}\\
&&& \acematchwith{n}{\ce}{\seqschemaX{\urv}} & \matchwith{\ce}{\seqschemaX{\crv}} & \text{match}\\
&&& \acemval{X} & \mval{X} & \text{value component}\\
\LCC &&& \color{Yellow} & \color{Yellow} & \color{Yellow}\\
&&& \acesplicede{m}{n}{\ctau} & \splicede{m}{n}{\ctau} & \text{spliced expression}\ECC\\
\mathsf{PrRule} & \crv & ::= & \acematchrule{p}{\ce} & \matchrule{p}{\ce} & \text{rule}\\
\mathsf{PrPat} & \cpv & ::= & \acewildp & \wildp & \text{wildcard pattern}\\
&&& \acefoldp{p} & \foldp{p} & \text{fold pattern}\\
&&& \acetplp{\labelset}{\mapschema{\cpv}{i}{\labelset}} & \tplp{\mapschema{\cpv}{i}{\labelset}} & \text{labeled tuple pattern}\\
&&& \aceinjp{\ell}{\cpv} & \injp{\ell}{\cpv} & \text{injection pattern}\\
&&& \acemval{X} & \mval{X} & \text{value component}\\
\LCC &&& \color{Yellow} & \color{Yellow} & \color{Yellow}\\
&&& \acesplicedp{m}{n}{\ctau} & \splicedp{m}{n}{\ctau} & \text{spliced pattern} \ECC
\end{array}\]
\caption[Syntax of proto-expressions, proto-rules and proto-patterns in $\miniVerseParam$]{Syntax of proto-expressions, proto-rules and proto-patterns in $\miniVerseParam$.}
\label{fig:P-candidate-terms}
\end{figure}
Figure \ref{fig:P-ce-kinds-constructors} defines the syntax of proto-kinds, $\cekappa$ and proto-constructions, $\cec$. Figure \ref{fig:P-candidate-terms} defines the syntax of proto-expressions, $\ce$, proto-rules, $\crv$, and proto-patterns, $\cpv$. All of these are ABTs. %The syntax of ce-types is identical to that given in Figure \ref{fig:U-candidate-terms}, which was described in Sec. \ref{sec:ce-syntax-U}. 

The mapping from expanded forms to proto-expansion forms is given in Appendix \ref{appendix:P-proto-expansion-validation}. The only ``interesting'' forms are the forms for references to spliced unexpanded terms, highlighted in yellow in Figure \ref{fig:P-ce-kinds-constructors} and Figure \ref{fig:P-candidate-terms}.

\subsection{Proto-Expansion Validation}
Proto-expansion validation operates essentially as described in Sec. \ref{sec:ce-validation-U}. It is governed by two groups of judgements. The first group of judgements defines proto-kind and proto-construction validation.

\vspace{10px}\noindent
$\begin{array}{ll}
\textbf{Judgement Form} & \textbf{Description}\\
\cvalidKX{\cekappa}{\kappa} & \text{$\cekappa$ has well-formed expansion $\kappa$}\\
\cvalidCX{\cec}{c}{\kappa} & \text{$\cec$ has expansion $c$ of kind $\kappa$}\\
\end{array}$
\vspace{10px}

The second group of judgements defines proto-expression, proto-rule and proto-pattern validation.

\vspace{10px}\noindent
$\arraycolsep=4pt\begin{array}{ll}
\textbf{Judgement Form} & \textbf{Description}\\
% \cvalidTP{\Omega}{\cscenev}{\ctau}{\tau} & \text{$\ctau$ has expansion $\tau$}\\
\cvalidEPX{\ce}{e}{\tau} & \text{$\ce$ has expansion $e$ of type $\tau$}\\
\cvalidRP{\Omega}{\escenev}{\crv}{r}{\tau}{\tau'} & \text{$\crv$ has expansion $r$ taking values of type $\tau$ to values of type $\tau'$}\\
\cvalidPPE{\uOmega}{\pscenev}{\cpv}{p}{\tau} & \text{$\cpv$ has expansion $p$ matching against $\tau$ generating hypotheses $\uOmega$}
\end{array}$
\vspace{10px}

\emph{Expression splicing scenes}, $\escenev$, are of the form $\esceneP{\omega : \Omega_\text{params}}{\uOmega}{\uPsi}{\uPhi}{b}$, \emph{construction splicing scenes}, $\cscenev$, are of the form $\csceneP{\omega : \Omega_\text{params}}{\uOmega}{b}$, and \emph{pattern splicing scenes}, $\pscenev$, are of the form $\psceneP{\omega : \Omega_\text{params}}{\uOmega}{\uPhi}{b}$. Their purpose is to ``remember'', during proto-expansion validation, the contexts and literal bodies from the TSM application site (cf. Rules (\ref{rule:expandsP-apuetsm}) and (\ref{rule:patExpandsP-apuptsm}) above), because these are necessary to validate references to spliced terms. They also keep around the parameter substitution and corresponding context, $\omega : \Omega_\text{params}$, because type/kind annotations on spliced terms need to be able to access parameters (but not expansion-local bindings.) 
We write $\csfrom{\escenev}$ for the construction splicing scene constructed by dropping the TSM contexts from $\escenev$:
\[\csfrom{\esceneP{\omega : \OParams}{\uOmega}{\uPsi}{\uPhi}{b}} = \csceneP{\omega : \OParams}{\uOmega}{b}\]

The rules governing references to spliced terms are reproduced below:
\begin{equation*}\tag{\ref{rule:cvalidK-spliced}}
\inferrule{
  \parseUKind{\bsubseq{b}{m}{n}}{\ukappa}\\
  \kExpands{\uOmega}{\ukappa}{\kappa}\\\\
  \uOmega=\uOmegaEx{\uD}{\uG}{\uMctx}{\Omega_\text{app}}\\
  \domof{\Omega} \cap \domof{\Omega_\text{app}} = \emptyset
}{
  \cvalidK{\Omega}{\csceneP{\omega : \OParams}{\uOmega}{b}}{\acesplicedk{m}{n}}{\kappa}
}
\end{equation*}
\begin{equation*}\tag{\ref{rule:cvalidC-spliced}}
\inferrule{
  \cscenev=\csceneP{\omega : \OParams}{\uOmega}{b}\\
  \cvalidK{\OParams}{\cscenev}{\cekappa}{\kappa}\\\\
  \parseUCon{\bsubseq{b}{m}{n}}{\uc}\\
  \cExpands{\uOmega}{\uc}{c}{[\omega]\kappa}\\\\
  \uOmega=\uOmegaEx{\uD}{\uG}{\uMctx}{\Omega_\text{app}}\\
  \domof{\Omega} \cap \domof{\Omega_\text{app}} = \emptyset
}{
  \cvalidC{\Omega}{\cscenev}{\acesplicedc{m}{n}{\cekappa}}{c}{[\omega]\kappa}
}
\end{equation*}
\begin{equation*}\tag{\ref{rule:cvalidE-P-splicede}}
\inferrule{
  \escenev = \esceneP{\omega : \OParams}{\uOmega}{\uPsi}{\uPhi}{b}\\
  \cvalidC{\OParams}{\csfrom{\escenev}}{\ctau}{\tau}{\akty}\\\\
  \parseUExp{\bsubseq{b}{m}{n}}{\ue}\\
  \expandsP{\uOmega}{\uPsi}{\uPhi}{\ue}{e}{[\omega]\tau}\\\\
  \uOmega=\uOmegaEx{\uD}{\uG}{\uMctx}{\Omega_\text{app}}\\
  \domof{\Omega} \cap \domof{\Omega_\text{app}} = \emptyset
}{
  \cvalidEP{\Omega}{\escenev}{\acesplicede{m}{n}{\ctau}}{e}{[\omega]\tau}
}
\end{equation*}
\begin{equation*}\tag{\ref{rule:cvalidPP-spliced}}
\inferrule{
  \cvalidC{\OParams}{\csceneP{\omega : \OParams}{\uOmega}{b}}{\ctau}{\tau}{\akty}\\
  \parseUPat{\bsubseq{b}{m}{n}}{\upv}\\
  \patExpandsP{\uOmega'}{\uPhi}{\upv}{p}{[\omega]\tau}
}{
  \cvalidPP{\uOmega'}{\psceneP{\omega : \Omega_\text{params}}{\uOmega}{\uPhi}{b}}{\acesplicedp{m}{n}{\ctau}}{p}{[\omega]\tau}
}
\end{equation*}


Notice that the kind/type annotations on spliced terms can refer to the provided parameters, but not to bindings local to the expansion. The parameter substitution, $\omega$, must be applied after expanding the annotations because the parameter names are not bound at the application site.

\subsection{Metatheory}
A more detailed account of the metatheory is given in Appendix \ref{appendix:metatheory-P}. We will summarize the key theorems below.

\subsubsection{TSM Expression Evaluation}
The following theorems establish a notion of TSM type safety based on preservation and progress for TSM expression evaluation.

\begingroup
\def\thetheorem{\ref{thm:peTSM-preservation}}
\begin{theorem}[peTSM Preservation]
% \label{thm:peTSM-preservation}
If $\hastsmtypeExp{\Omega}{\Psi}{\epsilon}{\rho}$ and $\tsmexpStepsExp{\Omega}{\Psi}{\epsilon}{\epsilon'}$ then $\hastsmtypeExp{\Omega}{\Psi}{\epsilon'}{\rho}$.
\end{theorem}
\endgroup

\begingroup
\def\thetheorem{\ref{thm:ppTSM-preservation}}
\begin{theorem}[ppTSM Preservation]
% \label{thm:ppTSM-preservation}
If $\hastsmtypePat{\Omega}{\Phi}{\epsilon}{\rho}$ and $\tsmexpStepsPat{\Omega}{\Phi}{\epsilon}{\epsilon'}$ then $\hastsmtypePat{\Omega}{\Phi}{\epsilon'}{\rho}$.
\end{theorem}
\endgroup

\begingroup
\def\thetheorem{\ref{thm:peTSM-progress}}
\begin{theorem}[peTSM Progress]
% \label{thm:peTSM-progress}
If $\hastsmtypeExp{\Omega}{\Psi}{\epsilon}{\rho}$ then either $\tsmexpStepsExp{\Omega}{\Psi}{\epsilon}{\epsilon'}$ for some $\epsilon'$ or $\tsmexpNormalExp{\Omega}{\Psi}{\epsilon}$.
\end{theorem}
\endgroup

\begingroup
\def\thetheorem{\ref{thm:ppTSM-progress}}
\begin{theorem}[ppTSM Progress]
% \label{thm:ppTSM-progress}
If $\hastsmtypePat{\Omega}{\Phi}{\epsilon}{\rho}$ then either $\tsmexpStepsPat{\Omega}{\Phi}{\epsilon}{\epsilon'}$ for some $\epsilon'$ or $\tsmexpNormalPat{\Omega}{\Phi}{\epsilon}$.
\end{theorem}
\endgroup

\subsubsection{Typed Expansion}
There are also a number of theorems that establish that typed expansion generates a well-typed expansion.

The top-level theorem is the typed expansion theorem for modules. 

\begingroup
\def\thetheorem{\ref{thm:module-expansion-P}}
\begin{theorem}[Module Expansion]
% \label{thm:module-expansion-P}
If $\mExpandsP{\uOmegaEx{\uD}{\uG}{\uMctx}{\Omega}}{\uPsi}{\uPhi}{\uM}{M}{\sigma}$ then $\hassig{\Omega}{M}{\sigma}$.
\end{theorem}
\endgroup

(The proof of this theorem requires proving the corresponding theorems about the other typed expansion judgements, as well as the proto-expansion validation judgements -- see Appendix \ref{appendix:metatheory-P}.)

\subsubsection{peTSM Abstract Reasoning Principles}
The following theorem summarizes the abstract reasoning principles available to programmers when applying a peTSM. In words:
\begin{enumerate}
	\item \textbf{Segmentation}: The segmentation determined by the proto-expansion actually segments the literal body (i.e. each segment is in-bounds and the segments are non-overlapping.)
	\item \textbf{Typing 1}: The type of the expansion is consistent with the type annotation on the peTSM definition.
	\item \textbf{Kinding 1}: Each spliced kind has a well-formed expansion at the application site.
	\item \textbf{Kinding 2}: Each kind annotation on a spliced construction has a well-formed expansion at the application site.
	\item \textbf{Kinding 3}: Each spliced construction is well-kinded consistent with its kind annotation.
	\item \textbf{Kinding 4}: Each type annotation on a spliced expression has a well-formed expansion at the application site.
	\item \textbf{Typing 2}: Each spliced expression is well-typed consistent with its type annotation.
	\item \textbf{Capture Avoidance}: The final expansion can be decomposed into a term with variables in place of each spliced kind, construction, expression and parameter. The expansions of these spliced kinds, constructions and expressions, as well as the provided parameters, can be substituted into this term in the standard capture avoiding manner.
	\item \textbf{Context Independence}: The decomposed term is indeed well-typed independent of the application site context.
\end{enumerate}

\begingroup
\def\thetheorem{\ref{thm:petsm-abstract-reasoning-principles}}
\begin{theorem}[peTSM Abstract Reasoning Principles]
If $\expandsP{\uOmega}{\uPsi}{\uPhi}{\utsmap{\uepsilon}{b}}{e}{\tau}$ then:
\begin{enumerate}
	\item $\uOmega=\uOmegaEx{\uD}{\uG}{\uMctx}{\Omega_\text{app}}$
	\item $\uPsi=\uAS{\uA}{\Psi}$
	\item (\textbf{Typing 1}) $\tsmexpExpandsExp{\uOmega}{\uPsi}{\uepsilon}{\epsilon}{\aetype{\tau}}$ and $\hastypeP{\Omega_\text{app}}{e}{\tau}$
	\item $\tsmexpEvalsExp{\Omega_\text{app}}{\Psi}{\epsilon}{\epsilon_\text{normal}}$
	\item $\tsmdefof{\epsilon_\text{normal}}=a$
	\item $\Psi = \Psi', \petsmdefn{a}{\rho}{\eparse}$
	\item $\encodeBody{b}{\ebody}$
  	\item $\evalU{\ap{\eparse}{\ebody}}{\aein{\mathtt{SuccessE}}{e_\text{pproto}}}$
	\item $\decodePCEExp{e_\text{pproto}}{\pce}$
	\item $\prepce{\Omega_\text{app}}{\Psi}{\pce}{\ce}{\epsilon_\text{normal}}{\aetype{\tau_\text{proto}}}{\omega}{\Omega_\text{params}}$
	\item (\textbf{Segmentation}) $\segOK{\segof{\ce}}{b}$
	\item $\cvalidEP{\Omega_\text{params}}{\esceneP{\omega : \OParams}{\uOmega}{\uPsi}{\uPhi}{b}}{\ce}{e'}{\tau_\text{proto}}$
	\item $e = [\omega]e'$
	\item $\tau = [\omega]\tau_\text{proto}$
	\item $
		\summaryOf{\ce} = \sseq{\acesplicedk{m_i}{n_i}}{\nkind} \cup \sseq{\acesplicedc{m'_i}{n'_i}{\cekappa'_i}}{\ncon} \cup\\
					     \sseq{\acesplicede{m''_i}{n''_i}{\ctau_i}}{\nexp}
		$
	\item (\textbf{Kinding 1}) $\sseq{\kExpands{\uOmega}{\parseUKindF{\bsubseq{b}{m_i}{n_i}}}{\kappa_i}}{\nkind}$ and $\sseq{\iskind{\Omega_\text{app}}{\kappa_i}}{\nkind}$
	\item (\textbf{Kinding 2}) $\sseq{\cvalidK{\OParams}{\csceneP{\omega : \OParams}{\uOmega}{b}}{\cekappa'_i}{\kappa'_i}}{\ncon}$ and $\sseq{\iskind{\Omega_\text{app}}{[\omega]\kappa'_i}}{\ncon}$
	\item (\textbf{Kinding 3}) $\sseq{\cExpands{\uOmega}{\parseUConF{\bsubseq{b}{m'_i}{n'_i}}}{c_i}{[\omega]\kappa'_i}}{\ncon}$ and $\sseq{\haskind{\Omega_\text{app}}{c_i}{[\omega]\kappa'_i}}{\ncon}$
	\item (\textbf{Kinding 4}) $\sseq{\cvalidC{\OParams}{\csceneP{\omega : \OParams}{\uOmega}{b}}{\ctau_i}{\tau_i}{\akty}}{\nexp}$ and $\sseq{\haskind{\Omega_\text{app}}{[\omega]\tau_i}{\akty}}{\nexp}$
	\item (\textbf{Typing 2}) $\sseq{\expandsP{\uOmega}{\uPsi}{\uPhi}{\parseUExpF{\bsubseq{b}{m''_i}{n''_i}}}{e_i}{[\omega]\tau_i}}{\nexp}$ and $\sseq{\hastypeP{\Omega_\text{app}}{e_i}{[\omega]\tau_i}}{\nexp}$
	\item (\textbf{Capture Avoidance}) $e = [\sseq{\kappa_i/k_i}{\nkind}, \sseq{c_i/u_i}{\ncon}, \sseq{e_i/x_i}{\nexp}, \omega]e''$ for some $e''$ and fresh $\sseq{k_i}{\nkind}$ and fresh $\sseq{u_i}{\ncon}$ and fresh $\sseq{x_i}{\nexp}$
	\item (\textbf{Context Independence}) \[\mathsf{fv}(e'') \subset \sseq{k_i}{\nkind} \cup \sseq{u_i}{\ncon} \cup \sseq{x_i}{\nexp} \cup \domof{\OParams}\]
	% $\hastypeP{\sseq{\Khyp{k_i}}{\nkind} \cup \sseq{u_i :: [\omega]\kappa'_i}{\ncon} \cup \sseq{x_i : [\omega]\tau_i}{\nexp}}{[\omega]e''}{\tau}$\todo{maybe restate this in terms of free variables of e'' here and elsewhere, because context isn't technically well-formed here?}
\end{enumerate}
\end{theorem}
\endgroup

\subsubsection{ppTSM Abstract Reasoning Principles}
The following theorem summarizes the abstract reasoning principles available to programmers when applying a ppTSM. In words:
\begin{enumerate}
	  \item \textbf{Typing 1}: The final expansion matches values of the type specified by the ppTSM's type annotation.
	\item \textbf{Segmentation}: The segmentation determined by the proto-expansion actually segments the literal body (i.e. each segment is in-bounds and the segments are non-overlapping.)
	\item \textbf{Kinding 1}: Each spliced kind has a well-formed expansion at the application site.
	\item \textbf{Kinding 2}: Each kind annotation on a spliced construction has a well-formed expansion at the application site.
	\item \textbf{Kinding 3}: Each spliced construction is well-kinded consistent with its kind annotation.
	\item \textbf{Kinding 4}: Each type annotation on a spliced expression has a well-formed expansion at the application site.
	\item \textbf{Typing 2}: Each spliced pattern has a well-typed expansion that matches values of the type indicated by the corresponding type annotation in the splice summary.
	  \item \textbf{No Hidden Bindings}: The hypotheses generated by the TSM application are exactly those generated by the spliced patterns.
\end{enumerate}
\begingroup
\def\thetheorem{\ref{thm:pptsm-abstract-reasoning-principles}}
\begin{theorem}[ppTSM Abstract Reasoning Principles]
If $\patExpandsP{\uOmega'}{\uPhi}{\utsmap{\uepsilon}{b}}{p}{\tau}$ then:
\begin{enumerate}
  \item $\uOmega=\uOmegaEx{\uD}{\uG}{\uMctx}{\Omega_\text{app}}$
  \item $\uPhi=\uAS{\uA}{\Phi}$
  \item (\textbf{Typing 1}) $\tsmexpExpandsPat{\uOmega}{\uPhi}{\uepsilon}{\epsilon}{\aetype{\tau}}$ and $\patTypePC{\Omega_\text{app}}{\uOmega'}{p}{\tau}$
  \item $\tsmexpEvalsPat{\Omega_\text{app}}{\Phi}{\epsilon}{\epsilon_\text{normal}}$
  \item $\tsmdefof{\epsilon_\text{normal}}=a$
  \item $\Phi = \Phi', \pptsmdefn{a}{\rho}{\eparse}$
  \item $\encodeBody{b}{\ebody}$
  \item $\evalU{\ap{\eparse}{\ebody}}{\aein{\mathtt{SuccessP}}{\ecand}}$
  \item $\decodePCEPat{\ecand}{\pcp}$
  \item $\prepcp{\Omega_\text{app}}{\Phi}{\pcp}{\cpv}{\epsilon_\text{normal}}{\aetype{\tau_\text{proto}}}{\omega}{\Omega_\text{params}}$
  \item (\textbf{Segmentation}) $\segOK{\segof{\cpv}}{b}$
	\item $
	\summaryOf{\ce} = \sseq{\acesplicedk{m_i}{n_i}}{\nkind} \cup \sseq{\acesplicedc{m'_i}{n'_i}{\cekappa'_i}}{\ncon} \cup\\
				     \sseq{\acesplicedp{m''_i}{n''_i}{\ctau_i}}{\npat}
	$
	\item (\textbf{Kinding 1}) $\sseq{\kExpands{\uOmega}{\parseUKindF{\bsubseq{b}{m_i}{n_i}}}{\kappa_i}}{\nkind}$ and $\sseq{\iskind{\Omega_\text{app}}{\kappa_i}}{\nkind}$
	\item (\textbf{Kinding 2}) $\sseq{\cvalidK{\OParams}{\csceneP{\omega : \OParams}{\uOmega}{b}}{\cekappa'_i}{\kappa'_i}}{\ncon}$ and $\sseq{\iskind{\Omega_\text{app}}{[\omega]\kappa'_i}}{\ncon}$
	\item (\textbf{Kinding 3}) $\sseq{\cExpands{\uOmega}{\parseUConF{\bsubseq{b}{m'_i}{n'_i}}}{c_i}{[\omega]\kappa'_i}}{\ncon}$ and $\sseq{\haskind{\Omega_\text{app}}{c_i}{[\omega]\kappa'_i}}{\ncon}$
	\item (\textbf{Kinding 4}) $\sseq{\cvalidC{\OParams}{\csceneP{\omega : \OParams}{\uOmega}{b}}{\ctau_i}{\tau_i}{\akty}}{\npat}$ and $\sseq{\haskind{\Omega_\text{app}}{[\omega]\tau_i}{\akty}}{\npat}$
	\item (\textbf{Typing 3}) $\sseq{\patExpandsP{\uOmega'}{\uPhi}{\parseUPatF{\bsubseq{b}{m''_i}{n''_i}}}{p_i}{[\omega]\tau_i}}{\npat}$
      \item (\textbf{No Hidden Bindings}) $\uOmega' = \biguplus_{0 \leq i < \npat} \uOmega _i$

  % \item $\cvalidPP{\uOmega'}{\psceneP{\omega : \Omega_\text{params}}{\uOmega}{\uPhi}{b}}{\cpv}{p}{\tau_\text{proto}}$
  % \item $\tau = [\omega]\tau_\text{proto}$
  % \item (\textbf{Typing}) $\tau_\text{final} = [\omega]\tau_\text{proto}$
\end{enumerate}
\end{theorem}
\endgroup


% !TEX root = omar-thesis.tex
\chapter{Static Evaluation}\label{chap:static-eval}
In the previous chapters, we have assumed that the parse functions in TSM definitions are closed expanded expressions. This is unrealistic in practice -- writing a parser generally requires access to various libraries. Moreover, the parse function might itself be written more concisely using TSMs. In this chapter, we address these problems by introducing a \emph{static environment} shared between parse functions.

\section{Static Values}
Figure \ref{fig:static-module-example} shows an example of a module, \li{ParserCombos} (see Sec. \ref{sec:parser-combinators}), bound \emph{statically} for use within the static parse functions in the subsequent TSM definitions.
\begin{figure}[h]
\begin{lstlisting}
static module ParserCombos = 
struct 
  type parser('c, 't) = list('c) -> list('t * list('c))
  val alt : parser('c, 't) -> parser('c, 't) -> parser('c, 't)
  (* ... *)
end

syntax $a at T by 
  static fn(b) => 
  	(* ... *) ParserCombos.alt (* ... *)
end

syntax $b at T' by 
  static fn(b) => 
    (* ... *) ParserCombos.alt (* ... *)
end

val y = (* ParserCombos CANNOT be used here *)
\end{lstlisting}
\caption{Binding a static module for use within parse functions.}
\label{fig:static-module-example}
\end{figure}
\clearpage

\li{ParserCombos} can only be used within other static values (e.g. the parse functions.) Static values do not persist from ``compile-time'' to ``run-time'', so we cannot use \li{ParserCombos} when giving the value of \li{y} on the last line. This distinguishes our approach from that taken by staged computation systems \cite{Taha99multi-stageprogramming:}. Notionally, static values operate much like a read-evaluate-print loop (REPL), in that they are evaluated immediately and the evaluated values are tracked by a \emph{static environment}.


\section{Applying TSMs Within TSM Definitions}\label{sec:tsms-for-tsms}
TSMs and TSM abbreviations can also be qualified as \li{static} and then used within parse functions and other static terms. Let us consider some examples of particular relevance to TSM providers.

\subsection{Quasiquotation}
TSMs must generate values of type \li{proto_expr} or \li{proto_pat}. Constructing values of these types explicitly can have high syntactic cost. To decrease the syntactic cost of constructing values of these types, we can define TSMs that provide support for \emph{quasiquotation syntax} (similar to that built in to languages like Lisp \cite{Bawd99a} and Scala \cite{shabalin2013quasiquotes}):
\begin{lstlisting}[numbers=none]
static syntax $proto_expr at proto_expr {
  static fn(b) => 
    (* proto-expression quasiquotation parser here *)
}

static syntax $proto_pat at proto_pat {
  static fn(b) => 
    (* proto-pattern quasiquotation parser here *)
}
\end{lstlisting}
For example, the following expression:
\begin{lstlisting}[numbers=none]
val gx = $proto_expr `SQTg(x)EQT`
\end{lstlisting}
is more concise than its expansion:
\begin{lstlisting}[numbers=none]
val gx = App(Var 'SSTRgESTR', Var 'SSTRxESTR')
\end{lstlisting}
Anti-quotation, i.e. splicing in an expression of type \li{proto_expr} (or \li{proto_pat}), is performed by prefixing a variable or parenthesized expression with \li{%}:
\begin{lstlisting}[numbers=none]
val fgx = $proto_expr `SQTf(%EQTgxSQT)EQT`
\end{lstlisting}
The expansion of this term is:
\begin{lstlisting}[numbers=none]
val fgx = App(Var 'SSTRfESTR', gx)
\end{lstlisting}

\subsection{Grammar-Based Parser Generators}
In Sec. \ref{sec:grammars}, we discussed a number of grammar-based parser generators. Abstractly, a parser generator is a module matching the signature \li{PARSEGEN} defined in Figure \ref{fig:PARSEGEN}.

\begin{figure}
\begin{lstlisting}
signature PARSEGEN = 
sig 
  type grammar('a)
  (* ... operations on grammars ... *)
  type parser('a) = string -> 'a parse_result
  val generate : grammar('a) -> parser('a)
end
\end{lstlisting}
\vspace{-8px}
\caption{A signature for parser generators.}
\vspace{-8px}
\label{fig:PARSEGEN}
\end{figure}

Rather than constructing a grammar using various operations (whose specifications are elided in \li{PARSEGEN}), we wish to use a syntax for grammars that follows standard conventions. We can do so by defining a parametric TSM \li{#\dolla#grammar}, qualified so as to be usable in the static phase, as follows:
\begin{lstlisting}[numbers=none]
static syntax $grammar (P : PARSEGEN) 'a at P.grammar('a) by 
  static fn(b) => (* ... *)
end
\end{lstlisting}

Using this definition, and given a module \li{P : PARSEGEN} and a static value defining the grammar of spliced unexpanded expressions, \li{spliced_uexp : P.grammar(proto_expr)}, we can define a TSM for regexes (implementing only a subset of the POSIX regex syntax here for simplicity) as shown in Figure \ref{fig:rx-grammar-based}.

\begin{figure}[h!]
\vspace{-5px}
\begin{lstlisting}[deletekeywords={as}]
syntax $rx(R : RX) at R.t by static 
  P.generate ($grammar P proto_expr {|SHTML #\label{line:rx_parse_fn_start}#
    start <- ""
      EHTMLfn () => $proto_expr `SCSSR.EmptyECSS`SHTML
    start <- "(" start ")"
      EHTMLfn e => eSHTML
    token str_tok #\label{line:str_tok_start}#
      EHTMLRU.parse "SSTR[^(@$]+ESTR" (* cannot use $rx within its own def *)SHTML #\label{line:str_tok_end}#
    start <- str_tok
      EHTMLfn s => $proto_expr `SCSSR.Str %(ECSSstr_to_proto_lit sSCSS)ECSS`SHTML
    start <- start start
      EHTMLfn e1 e2 => $proto_expr `SCSSR.Seq (%ECSSe1SCSS, %ECSSe2SCSS)ECSS`SHTML
    start <- start "|" start 
      EHTMLfn e1 e2 => $proto_expr `SCSSR.Or (%ECSSe1SCSS, %ECSSe2SCSS)ECSS`SHTML
    start <- start "*"
      EHTMLfn e => $proto_expr `SCSSR.Star %ECSSe`SHTML

    using EHTMLspliced_uexpSHTML as spliced_uexp #\label{line:splicede_using}#
    start <- "${" spliced_uexp "}" #\label{line:splicing-start}#
      EHTMLfn e => eSHTML
    start <- "@{" spliced_uexp "}"
      EHTMLfn e => $proto_expr `SCSSR.Str %(ECSSeSCSS)ECSS`SHTML #\label{line:splicing-end}#
  EHTML|})
end #\label{line:rx_parse_fn_end}#
\end{lstlisting}
\vspace{-12px}
\caption{A grammar-based definition of \texttt{\$rx}.}
\vspace{-15px}
\label{fig:rx-grammar-based}
\end{figure}


\section{Library Management}
In the examples above, we explicitly qualified various definitions with the \li{static} keyword to make them available within static values. This captures the essential nature of the problem of static evaluation, but in practice, we would like to be able to use libraries within both static values and standard values as needed without duplicating code. This can be accomplished by the use of a language-external library and compilation manager. For example, a library and compilation manager for VerseML similar to SML/NJ's CM \cite{blume:smlnj-cm} could support a \li{static} qualifier on libraries, which would place the definitions exported by the imported library (without qualification) into the static phase of the library being defined:
\begin{lstlisting}[numbers=none,morekeywords={Library,is}]
Library 
  (* ... exports of library being defined ... *)
is 
  (* ... *)

  (* we do not need static qualifiers within parsegen.cm *)
  static parsegen.cm 
\end{lstlisting}
For the sake of generality and simplicity, we will leave the details of library and compilation management out of our formal developments (following the approach taken by the definition of Standard ML \cite{Tofte:89:TheDefinitionOfStandardML}.)

% \part{TLM Implicits}\label{part:implicits}
% !TEX root = omar-thesis.tex
\chapter{Unparameterized TSM Implicits}\label{chap:tsls}
Using TSMs, a library provider can control the expansion of generalized literal forms. However, the library client must explicitly prefix each such form with a TSM name. To further lower the syntactic cost of using TSMs, so that it compares to the syntactic cost of using derived forms built primitively into a language, VerseML allows clients to designate, for any type, one TSM as that type's \emph{implicit TSM} within a delimited scope. When VerseML's \emph{local type inference} system encounters a generalized literal form not prefixed by a TSM name (an \emph{unadorned literal form}), it applies the TSM associated with the type that the expression or pattern is being checked against.

\section{TSM Implicits By Example}\label{sec:tsm-implicits-by-example}
We begin in this section by introducing TSM implicits by example in VerseML. In Sec. \ref{sec:b-miniverse}, we will formalize TSM implicits with a reduced calculus, {Bidirectional} $\miniVersePat$. 

\subsection{Designation and Usage}
In the following example, the expression TSM named \li{#\dolla#rx}, defined in Section \ref{sec:uetsms-definition}, is designated the implicit expression TSM at type \li{Rx}, and the pattern TSM named \li{#\dolla#rx}, defined in Sec. \ref{sec:ptsms-definition}, is designated the implicit pattern TSM at type \li{Rx},  both within the indicated scope.  %The scope of this declaration could be further restricted using the clauses shown in comments below. 
\begin{lstlisting}
implicit syntax 
  $rx at Rx for expressions
  $rx at Rx for patterns
in
  fun is_ssn(s : string) => rx_match /SURL\d\d\d-\d\d-\d\d\d\dEURL/ s
  fun name_from_example_rx(r : Rx) : string option => 
    match r with 
      /SURL@EURLnameSURL: %EURL_/ => Some name
    | _ => None
end
\end{lstlisting}
For convenience, VerseML also provides a derived designation form that combines the two designations above:
\begin{lstlisting}[numbers=none]
implicit syntax 
  $rx at Rx 
in 
  (* ... *)
end 
\end{lstlisting}

On Line 5 of the example above, we apply a function \li{rx_match} (not shown), which has type \li{Rx -> string -> MatchResult}, to an expression of unadorned literal form. The expression TSM \li{#\dolla#rx} is applied implicitly to this expression to determine its expansion because the expression appears in a syntactic position where it must be of type \li{Rx}, and we have designated \li{#\dolla#rx} as the implicit expression TSM at this type. %f we had instead applied it explicitly, Line 2 would be written as follows:
% \begin{lstlisting}[numbers=none]
% fun is_ssn(s : string) => rx_match ($rx /SURL\d\d\d-\d\d-\d\d\d\dEURL/) s
% \end{lstlisting}

Similarly, a pattern of unadorned literal form appears on Line 8. Because it appears in a syntactic position where it must match values of type \li{Rx}, the pattern TSM \li{#\dolla#rx} is implicitly applied to determine its expansion.

\subsection{Analytic and Synthetic Positions}
When typechecking a subexpression, $e'$, of an expresssion, $e$, we say that $e'$ appears in an \emph{analytic position} if the type that $e'$ must have is fully determined by its position within $e$. For example, an expression appearing as a function argument is in an analytic position because the function's type determines each of the argument types. Similarly, an expression may be in analytic position due to a \emph{type ascription}, either directly on the expression, or on the binding or definition that the expression appears within:
\begin{lstlisting}[numbers=none]
val ssn = /SURL\d\d\d-\d\d-\d\d\d\dEURL/ : Rx
val ssn : Rx = /SURL\d\d\d-\d\d-\d\d\d\dEURL/
fun ssn() : Rx => /SURL\d\d\d-\d\d-\d\d\d\dEURL/
\end{lstlisting}

If the type of $e'$ is not fully determined by its position within $e$, we instead say that the expression appears in a \emph{synthetic position}. For example, a top-level expression, or an expression appearing in a binding or definition without a type ascription, appears in a synthetic position.

Expressions of unadorned literal form are only valid in analytic position, because their type must be known to be able to determine the appropriate TSM to implicitly apply. The following expressions cannot be typechecked because expressions of unadorned literal form appear in synthetic position:
\begin{lstlisting}[numbers=none]
let 
  val ssn = /SURL\d\d\d-\d\d-\d\d\d\dEURL/ (* INVALID *)
  fun ssn() => /SURL\d\d\d-\d\d-\d\d\d\dEURL/ (* INVALID *)
in 
  (* ... *) 
end
\end{lstlisting}

In VerseML, the scrutinee of a match expression is always in synthetic position. Consequently, the type of value that a pattern appearing within a match expression must match is always known, so patterns can always be of unadorned literal form.
\section{Bidirectional $\miniVersePat$}\label{sec:b-miniverse}
\subsection{Inner Core}
\subsection{Syntax of the Outer Surface}
\subsection{Bidirectionally Typed Expansion}
\subsection{uTSL Definition}
\subsection{uTSL Application}
\subsection{Candidate Expansion Validation}


% \part{Conclusion}
% !TEX root = omar-thesis.tex
\chapter{Discussion \& Future Directions}\label{chap:conclusion}
\section{Interesting Applications}
\subsection{TSMs For Defining TSMs}\label{sec:tsms-for-tsms}
Static functions can also make use of TSMs. In this section, we will show how quasiquotation syntax and grammar-based parser generators can be expressed using TSMs. These TSMs are quite useful for writing other TSMs.
\subsubsection{Quasiquotation}
TSMs must generate values of type \li{CEExp}. Doing so explicitly can have high syntactic cost. To decrease the syntactic cost of constructing values of this type, the prelude includes a TSM that provides quasiquotation syntax (cf. Sec. \ref{sec:syntax-examples-quasiquotation}):
\begin{lstlisting}[numbers=none]
syntax $qqexp at CEExp {
	static fn(body : Body) : ParseResult => (* expression parser here *)
}

syntax $qqtype at CETyp {
	static fn(body : Body) : ParseResult => (* type parser here *)
}
\end{lstlisting}
For example, the following expression:
\begin{lstlisting}[numbers=none]
let gx = $qqexp `SQTg(x)EQT`
\end{lstlisting}
is more concise than its expansion:
\begin{lstlisting}[numbers=none]
let gx = App(Var 'SSTRgESTR', Var 'SSTRxESTR')
\end{lstlisting}
The full concrete syntax of the language can be used. Anti-quotation, i.e. splicing in an expression of type \li{MarkedExp}, is indicated by the prefix \li{%}:
\begin{lstlisting}[numbers=none]
let fgx = $qqexp `SQTf(%EQTgxSQT)EQT`
\end{lstlisting}
The expansion of this expression is:
\begin{lstlisting}[numbers=none]
let fgx = App(Var 'SSTRfESTR', gx)
\end{lstlisting}


\subsubsection{Parser Generators}
\todo{grammars, compile function, TSM for grammar, example of IP address}

\subsection{Monadic Commands}\label{sec:application-monadic-commands}

\section{Summary}
\todo{Write summary}

\section{Future Directions}
\subsection{Mechanically Verifying TSM Definitions}\label{sec:verifying-tsms}
Finally, VerseML is not designed for advanced theorem proving tasks where languages like Coq, Agda or Idris might be used today. That said, we conjecture that the primitives we describe could be integrated into languages like Gallina (the ``external language'' of the Coq proof assistant  \cite{Coq:manual}) with  modifications, but do not plan to pursue this line of research here.

In such a setting, you could verify TSM definitions \todo{finish writing this}
\subsection{Improved Error Reporting}\label{sec:error-handling}
\subsection{Controlled Binding}\label{sec:controlled-binding}
\subsection{Type-Aware Splicing}\label{sec:type-aware-splicing}
\subsection{Integration With Code Editors}\label{sec:interaction-with-tools}
\subsection{Resugaring}\label{sec:resugaring}
\todo{Cite recent work at PLDI (?) and ICFP from Brown}
\subsection{Non-Textual Display Forms}\label{sec:non-textual-display-forms}
\todo{Talk about active code completion work and future ideas}





% !TEX root = omar-thesis.tex
\part{Appendix}

\appendix
\chapter{Conventions}
\section{Typographic Conventions}\label{appendix:typographic-conventions}
We adopt \emph{PFPL}'s typographic conventions for operational forms throughout the paper. For example, consider the operational form for injections into a labeled sum type:
\[... 
\]
In particular, the names of operators and indexed families of operators are written in $\texttt{typewriter font}$, indexed families of operators specify non-symbolic indices within $[\text{mathematical braces}]$ and symbolic indices within \texttt{[}textual braces\texttt{]}, and term arguments are grouped arbitrarily (roughly, by sort) using \texttt{\{}textual curly braces\texttt{\}} and \texttt{(}textual rounded braces\texttt{)} \cite{pfpl}. \todo{do we actually use symbols anymore?}

Moreover, we write $\mapschema{\tau}{i}{\labelset}$ for a sequence of arguments $\tau_i$, one for each $i\in \labelset$, and similarly for arguments of other valences. Operations  that are parameterized by label sets, e.g. $\aprod{\labelset}{\mapschema{\tau}{i}{\labelset}}$, are identified up to mutual reordering of the label set and the corresponding argument sequence. 

We write $\seqschemaX{r}$ for sequences of $n \geq 0$ rule arguments and $p.e$ for expressions binding the variables that appear in the pattern $p$.


Empty finite sets are written $\emptyset$, or omitted entirely within judgements, and non-empty finite sets are written as comma-separated finite sequences identified up to exchange and contraction. 

Empty typing contexts are written $\emptyset$, or omitted entirely within judgements, and non-empty typing contexts are written as finite sequences of hypotheses identified up to exchange and contraction. 


\section{Judgemental Conventions}\label{appendix:judgemental-conventions}

\chapter{\texorpdfstring{$\miniVerseUE$ and $\miniVersePat$}{miniVerseSE and miniVerseS}}\label{appendix:miniVerseSES}

This section defines $\miniVersePat$, the language of Chapter \ref{chap:uptsms}. The language of Chapter \ref{chap:uetsms}, $\miniVerseUE$, can be recovered by omitting the syntactic forms, judgements, rules, proof clauses and proof cases typeset with gray backgrounds below.

\clearpage

\section{Expanded Language (XL)}\label{appendix:SES-XL}
\subsection{Syntax}
% \begin{figure}[h!]
\[\begin{array}{lllllll}
\textbf{Sort} & & 
& \textbf{Operational Form} 
% & \textbf{Stylized Form} 
& \textbf{Description}\\
\mathsf{Typ} & \tau & ::= & t 
%& t 
& \text{variable}\\
&&& \aparr{\tau}{\tau} 
%& \parr{\tau}{\tau} 
& \text{partial function}\\
&&& \aall{t}{\tau} 
%& \forallt{t}{\tau} 
& \text{polymorphic}\\
&&& \arec{t}{\tau} 
%& \rect{t}{\tau} 
& \text{recursive}\\
&&& \aprod{\labelset}{\mapschema{\tau}{i}{\labelset}} 
%& \prodt{\mapschema{\tau}{i}{\labelset}} 
& \text{labeled product}\\
&&& \asum{\labelset}{\mapschema{\tau}{i}{\labelset}} 
%& \sumt{\mapschema{\tau}{i}{\labelset}} 
& \text{labeled sum}\\
\mathsf{Exp} & e & ::= & x 
%& x 
& \text{variable}\\
&&& \aelam{\tau}{x}{e} 
%& \lam{x}{\tau}{e} 
& \text{abstraction}\\
&&& \aeap{e}{e} 
%& \ap{e}{e} 
& \text{application}\\
&&& \aetlam{t}{e} 
%& \Lam{t}{e} 
& \text{type abstraction}\\
&&& \aetap{e}{\tau} 
%& \App{e}{\tau} 
& \text{type application}\\
&&& \aefold{e} 
%& \fold{e} : \tau 
& \text{fold}\\
&&& \aeunfold{e} 
%& \unfold{e} 
& \text{unfold}\\
&&& \aetpl{\labelset}{\mapschema{e}{i}{\labelset}} 
%& \tpl{\mapschema{e}{i}{\labelset}} 
& \text{labeled tuple}\\
&&& \aepr{\ell}{e} 
%& \prj{e}{\ell} 
& \text{projection}\\
&&& \aein{\ell}{e} 
%& \inj{\ell}{e} 
& \text{injection}\\
&&& \aecase{\labelset}{e}{\mapschemab{x}{e}{i}{\labelset}} 
%& \caseof{e}{\mapschemab{x}{e}{i}{\labelset}} 
& \text{case analysis}\\
\LCC \lightgray & \lightgray & \lightgray 
% & \lightgray 
& \lightgray & \lightgray \\
&&
& \aematchwith{n}{e}{\seqschemaX{r}}
% & \matchwith{e}{\seqschemaX{r}} 
& \text{match}\\
\mathsf{Rule} & r & ::= 
& \aematchrule{p}{e} 
%& \matchrule{p}{e} 
& \text{rule}\\
\mathsf{Pat} & p & ::= 
& x  
%& x 
& \text{variable pattern}\\
&&& \aewildp 
%& \wildp 
& \text{wildcard pattern}\\
&&& \aefoldp{p} 
%& \foldp{p} 
& \text{fold pattern}\\
&&& \aetplp{\labelset}{\mapschema{p}{i}{\labelset}} 
%& \tplp{\mapschema{p}{i}{\labelset}} 
& \text{labeled tuple pattern}\\
&&& \aeinjp{\ell}{p} 
%& \injp{\ell}{p} 
& \text{injection pattern} \ECC
\end{array}\]
% \caption{Syntax of the $\miniVersePat$ expanded language (XL).}
% \end{figure}

\subsection{Statics}
\emph{Type formation contexts}, $\Delta$, are finite sets of hypotheses of the form $\Dhyp{t}$. We write $\Delta, \Dhyp{t}$ when $\Dhyp{t} \notin \Delta$ for $\Delta$ extended with the hypothesis $\Dhyp{t}$. %Finite sets are written as finite sequences identified up to exchange.% We write $\Dcons{\Delta}{\Delta'}$ for the union of $\Delta$ and $\Delta'$.

\emph{Typing contexts}, $\Gamma$, are finite functions that map each variable $x \in \domof{\Gamma}$, where $\domof{\Gamma}$ is a finite set of variables, to the hypothesis $\Ghyp{x}{\tau}$, for some $\tau$. We write $\Gamma, \Ghyp{x}{\tau}$, when $x \notin \domof{\Gamma}$, for the extension of $\Gamma$ with a mapping from $x$ to $\Ghyp{x}{\tau}$, and $\Gcons{\Gamma}{\Gamma'}$ when $\domof{\Gamma} \cap \domof{\Gamma'} = \emptyset$ for the typing context mapping each $x \in \domof{\Gamma} \cup \domof{\Gamma'}$ to $x : \tau$ if $x : \tau \in \Gamma$ or $x : \tau \in \Gamma'$. We write $\isctxU{\Delta}{\Gamma}$ if every type in $\Gamma$ is well-formed relative to $\Delta$.
\begin{definition}[Typing Context Formation] \label{def:isctxU}
$\isctxU{\Delta}{\Gamma}$ iff for each hypothesis $x : \tau \in \Gamma$, we have $\istypeU{\Delta}{\tau}$.
\end{definition}

\noindent\fbox{\strut$\istypeU{\Delta}{\tau}$}~~$\tau$ is a well-formed type
\begin{subequations}\label{rules:istypeU}
\begin{equation}\label{rule:istypeU-var}
\inferrule{ }{\istypeU{\Delta, \Dhyp{t}}{t}}
\end{equation}
\begin{equation}\label{rule:istypeU-parr}
\inferrule{
  \istypeU{\Delta}{\tau_1}\\
  \istypeU{\Delta}{\tau_2}
}{\istypeU{\Delta}{\aparr{\tau_1}{\tau_2}}}
\end{equation}
\begin{equation}\label{rule:istypeU-all}
  \inferrule{
    \istypeU{\Delta, \Dhyp{t}}{\tau}
  }{
    \istypeU{\Delta}{\aall{t}{\tau}}
  }
\end{equation}
\begin{equation}\label{rule:istypeU-rec}
  \inferrule{
    \istypeU{\Delta, \Dhyp{t}}{\tau}
  }{
    \istypeU{\Delta}{\arec{t}{\tau}}
  }
\end{equation}
\begin{equation}\label{rule:istypeU-prod}
  \inferrule{
    \{\istypeU{\Delta}{\tau_i}\}_{i \in \labelset}
  }{
    \istypeU{\Delta}{\aprod{\labelset}{\mapschema{\tau}{i}{\labelset}}}
  }
\end{equation}
\begin{equation}\label{rule:istypeU-sum}
  \inferrule{
    \{\istypeU{\Delta}{\tau_i}\}_{i \in \labelset}
  }{
    \istypeU{\Delta}{\asum{\labelset}{\mapschema{\tau}{i}{\labelset}}}
  }
\end{equation}
\end{subequations}

\noindent\fbox{\strut$\hastypeU{\Delta}{\Gamma}{e}{\tau}$}~~$e$ is assigned type $\tau$
\begin{subequations}\label{rules:hastypeU}\label{rules:hastypeUP}
\begin{equation}\label{rule:hastypeU-var}
  \inferrule{ }{
    \hastypeU{\Delta}{\Gamma, \Ghyp{x}{\tau}}{x}{\tau}
  }
\end{equation}
\begin{equation}\label{rule:hastypeU-lam}
  \inferrule{
    \istypeU{\Delta}{\tau}\\
    \hastypeU{\Delta}{\Gamma, \Ghyp{x}{\tau}}{e}{\tau'}
  }{
    \hastypeU{\Delta}{\Gamma}{\aelam{\tau}{x}{e}}{\aparr{\tau}{\tau'}}
  }
\end{equation}
\begin{equation}\label{rule:hastypeU-ap}
  \inferrule{
    \hastypeU{\Delta}{\Gamma}{e_1}{\aparr{\tau}{\tau'}}\\
    \hastypeU{\Delta}{\Gamma}{e_2}{\tau}
  }{
    \hastypeU{\Delta}{\Gamma}{\aeap{e_1}{e_2}}{\tau'}
  }
\end{equation}
\begin{equation}\label{rule:hastypeU-tlam}
  \inferrule{
    \hastypeU{\Delta, \Dhyp{t}}{\Gamma}{e}{\tau}
  }{
    \hastypeU{\Delta}{\Gamma}{\aetlam{t}{e}}{\aall{t}{\tau}}
  }
\end{equation}
\begin{equation}\label{rule:hastypeU-tap}
  \inferrule{
    \hastypeU{\Delta}{\Gamma}{e}{\aall{t}{\tau}}\\
    \istypeU{\Delta}{\tau'}
  }{
    \hastypeU{\Delta}{\Gamma}{\aetap{e}{\tau'}}{[\tau'/t]\tau}
  }
\end{equation}
\begin{equation}\label{rule:hastypeU-fold}
  \inferrule{\
    % \istypeU{\Delta, \Dhyp{t}}{\tau}\\
    \hastypeU{\Delta}{\Gamma}{e}{[\arec{t}{\tau}/t]\tau}
  }{
    \hastypeU{\Delta}{\Gamma}{\aefold{e}}{\arec{t}{\tau}}
  }
\end{equation}
\begin{equation}\label{rule:hastypeU-unfold}
  \inferrule{
    \hastypeU{\Delta}{\Gamma}{e}{\arec{t}{\tau}}
  }{
    \hastypeU{\Delta}{\Gamma}{\aeunfold{e}}{[\arec{t}{\tau}/t]\tau}
  }
\end{equation}
\begin{equation}\label{rule:hastypeU-tpl}
  \inferrule{
    \{\hastypeU{\Delta}{\Gamma}{e_i}{\tau_i}\}_{i \in \labelset}
  }{
    \hastypeU{\Delta}{\Gamma}{\aetpl{\labelset}{\mapschema{e}{i}{\labelset}}}{\aprod{\labelset}{\mapschema{\tau}{i}{\labelset}}}
  }
\end{equation}
\begin{equation}\label{rule:hastypeU-pr}
  \inferrule{
    \hastypeU{\Delta}{\Gamma}{e}{\aprod{\labelset, \ell}{\mapschema{\tau}{i}{\labelset}; \ell \hookrightarrow \tau}}
  }{
    \hastypeU{\Delta}{\Gamma}{\aepr{\ell}{e}}{\tau}
  }
\end{equation}
\begin{equation}\label{rule:hastypeU-in}
  \inferrule{
    % \{\istypeU{\Delta}{\tau_i}\}_{i \in \labelset}\\
    % \istypeU{\Delta}{\tau}\\
    \hastypeU{\Delta}{\Gamma}{e}{\tau}
  }{
    \hastypeU{\Delta}{\Gamma}{\aein{\ell}{e}}{\asum{\labelset, \ell}{\mapschema{\tau}{i}{\labelset}; \ell \hookrightarrow \tau}}
  }
\end{equation}
\begin{equation}\label{rule:hastypeU-case}
  \inferrule{
    \hastypeU{\Delta}{\Gamma}{e}{\asum{\labelset}{\mapschema{\tau}{i}{\labelset}}}\\
    % \istypeU{\Delta}{\tau}\\
    \{\hastypeU{\Delta}{\Gamma, x_i : \tau_i}{e_i}{\tau}\}_{i \in \labelset}
  }{
    \hastypeU{\Delta}{\Gamma}{\aecase{\labelset}{e}{\mapschemab{x}{e}{i}{\labelset}}}{\tau}
  }
\end{equation}
\begin{grayparbox}
\begin{equation}\label{rule:hastypeUP-match}
\graybox{\inferrule{
  \hastypeU{\Delta}{\Gamma}{e}{\tau}\\
  % \istypeU{\Delta}{\tau'}\\
  \{\ruleType{\Delta}{\Gamma}{r_i}{\tau}{\tau'}\}_{1 \leq i \leq n}\\
}{\hastypeU{\Delta}{\Gamma}{\aematchwith{n}{e}{\seqschemaX{r}}}{\tau'}}}
\end{equation}
\end{grayparbox}
\end{subequations}

\vspace{-5px}\begin{grayparbox}
\vspace{5px}\noindent\fcolorbox{black}{lightgray}{\strut$\ruleType{\Delta}{\Gamma}{r}{\tau}{\tau'}$}~~$r$ takes values of type $\tau$ to values of type $\tau'$
\begin{equation}\label{rule:ruleType}
\graybox{\inferrule{
  \patType{\pctx'}{p}{\tau}\\
  \hastypeU{\Delta}{\Gcons{\Gamma}{\pctx'}}{e}{\tau'}
}{\ruleType{\Delta}{\Gamma}{\aematchrule{p}{e}}{\tau}{\tau'}}}
\end{equation}
Rule (\ref{rule:ruleType}) is defined mutually inductively with Rules (\ref{rules:hastypeUP}).

\noindent\fcolorbox{black}{lightgray}{\strut$\patType{\Gamma}{p}{\tau}$}~~$p$ matches values of type $\tau$ and generates hypotheses $\pctx$
\begin{subequations}\label{rules:patType}
\begin{equation}\label{rule:patType-var}
\graybox{\inferrule{ }{\patType{\Ghyp{x}{\tau}}{x}{\tau}}}
\end{equation}
\begin{equation}\label{rule:patType-wild}
\graybox{\inferrule{ }{\patType{\emptyset}{\aewildp}{\tau}}}
\end{equation}
\begin{equation}\label{rule:patType-fold}
\graybox{\inferrule{
  \patType{\pctx}{p}{[\arec{t}{\tau}/t]\tau}
}{
  \patType{\pctx}{\aefoldp{p}}{\arec{t}{\tau}}
}}
\end{equation}
\begin{equation}\label{rule:patType-tpl}
\graybox{\inferrule{
  \{\patType{\pctx_i}{p_i}{\tau_i}\}_{i \in \labelset}
}{
  \patType{\Gconsi{i \in \labelset}{\pctx_i}}{\aetplp{\labelset}{\mapschema{p}{i}{\labelset}}}{\aprod{\labelset}{\mapschema{\tau}{i}{\labelset}}}
}}
\end{equation}
\begin{equation}\label{rule:patType-inj}
\graybox{\inferrule{
  \patType{\pctx}{p}{\tau}
}{
  \patType{\pctx}{\aeinjp{\ell}{p}}{\asum{\labelset, \ell}{\mapschema{\tau}{i}{\labelset}; \mapitem{\ell}{\tau}}}
}}
\end{equation}
\end{subequations}
\end{grayparbox}

\subsubsection{Metatheory}
The rules above are syntax-directed, so we assume an inversion lemma for each rule as needed without stating it separately or proving it explicitly. The following standard lemmas also hold.

The Weakening Lemma establishes that extending the context with unnecessary hypotheses preserves well-formedness and typing.
\begin{lemma}[Weakening]\label{lemma:weakening-UP}\label{lemma:weakening-U} ~
\begin{enumerate} 
\item If $\istypeU{\Delta}{\tau}$ then $\istypeU{\Delta, \Dhyp{t}}{\tau}$.
%\item If $\isctxU{\Delta}{\Gamma}$ then $\isctxU{\Delta, \Dhyp{t}}{\Gamma}$.
\item \begin{enumerate}
  \item If $\hastypeU{\Delta}{\Gamma}{e}{\tau}$ then $\hastypeU{\Delta, \Dhyp{t}}{\Gamma}{e}{\tau}$.
  \item \graytxtbox{If $\ruleType{\Delta}{\Gamma}{r}{\tau}{\tau'}$ then $\ruleType{\Delta, \Dhyp{t}}{\Gamma}{r}{\tau}{\tau'}$.}
  \end{enumerate}
\item \begin{enumerate}
  \item If $\hastypeU{\Delta}{\Gamma}{e}{\tau}$ and $\istypeU{\Delta}{\tau''}$ then $\hastypeU{\Delta}{\Gamma, \Ghyp{x}{\tau''}}{e}{\tau}$.
  \item \graytxtbox{If $\ruleType{\Delta}{\Gamma}{r}{\tau}{\tau'}$ and $\istypeU{\Delta}{\tau''}$ then $\ruleType{\Delta}{\Gamma, \Ghyp{x}{\tau''}}{r}{\tau}{\tau'}$.}
  \end{enumerate}
\item \graytxtbox{If $\patType{\pctx}{p}{\tau}$ then $\patTypeD{\Delta, \Dhyp{t}}{\pctx}{p}{\tau}$.}
\end{enumerate}
\end{lemma}
\begin{proof-sketch} ~
\begin{enumerate}
\item By rule induction over Rules (\ref{rules:istypeU}).
%\item By rule induction over Rules (\ref{rules:isctxU}).
\item By \graytxtbox{mutual} rule induction over Rules (\ref{rules:hastypeUP}) \graytxtbox{and Rule (\ref{rule:ruleType})}, and part 1.
\item By \graytxtbox{mutual} rule induction over Rules (\ref{rules:hastypeUP}) \graytxtbox{and Rule (\ref{rule:ruleType})}, and part 1.
\item \graytxtbox{By rule induction over Rules (\ref{rules:patType}).}
\end{enumerate}
\end{proof-sketch}

\begin{grayparbox}
The {pattern typing judgement} is \emph{linear} in the pattern typing context, i.e. it does \emph{not} obey weakening of the pattern typing context. This is to ensure that the pattern typing context captures exactly those hypotheses generated by a pattern, and no others.
\end{grayparbox}

The Substitution Lemma establishes that substitution of a well-formed type for a type variable, or an expanded expression of the appropriate type for an expanded expression variable, preserves well-formedness and typing.
\begin{lemma}[Substitution]\label{lemma:substitution-UP} ~
\begin{enumerate}
\item If $\istypeU{\Delta, \Dhyp{t}}{\tau}$ and $\istypeU{\Delta}{\tau'}$ then $\istypeU{\Delta}{[\tau'/t]\tau}$.
%\item If $\isctxU{\Delta, \Dhyp{t}}{\Gamma}$ and $\istypeU{\Delta}{\tau'}$ then $\isctxU{\Delta}{[\tau'/t]\Gamma}$.
\item \begin{enumerate}
  \item If $\hastypeU{\Delta, \Dhyp{t}}{\Gamma}{e}{\tau}$ and $\istypeU{\Delta}{\tau'}$ then $\hastypeU{\Delta}{[\tau'/t]\Gamma}{[\tau'/t]e}{[\tau'/t]\tau}$.
  \item \begin{grayparbox} 
  {If} $\ruleType{\Delta, \Dhyp{t}}{\Gamma}{r}{\tau}{\tau''}$ and $\istypeU{\Delta}{\tau'}$ then $\ruleType{\Delta}{[\tau'/t]\Gamma}{[\tau'/t]r}{[\tau'/t]\tau}{[\tau'/t]\tau''}$.
  \end{grayparbox}
  \end{enumerate}
\item \begin{enumerate}
  \item If $\hastypeU{\Delta}{\Gamma, \Ghyp{x}{\tau'}}{e}{\tau}$ and $\hastypeU{\Delta}{\Gamma}{e'}{\tau'}$ then $\hastypeU{\Delta}{\Gamma}{[e'/x]e}{\tau}$.
  \item \graytxtbox{
  If $\ruleType{\Delta}{\Gamma, \Ghyp{x}{\tau'}}{r}{\tau}{\tau''}$ and $\hastypeU{\Delta}{\Gamma}{e'}{\tau''}$ then $\ruleType{\Delta}{\Gamma}{[e'/x]r}{\tau}{\tau''}$.}
  \end{enumerate}
\end{enumerate}\end{lemma}
\begin{proof-sketch} ~
\begin{enumerate}
\item By rule induction over Rules (\ref{rules:istypeU}).
\item By \graytxtbox{mutual} rule induction over Rules (\ref{rules:hastypeUP}) \graytxtbox{and Rule (\ref{rule:ruleType})}.
\item By \graytxtbox{mutual} rule induction over Rules (\ref{rules:hastypeUP}) \graytxtbox{and Rule (\ref{rule:ruleType})}.
\end{enumerate}
\end{proof-sketch}

The Decomposition Lemma is the converse of the Substitution Lemma.
\begin{lemma}[Decomposition]\label{lemma:decomposition-UP} ~
\begin{enumerate}
\item If $\istypeU{\Delta}{[\tau'/t]\tau}$ and $\istypeU{\Delta}{\tau'}$ then $\istypeU{\Delta, \Dhyp{t}}{\tau}$.
%\item If $\isctxU{\Delta}{[\tau'/t]\Gamma}$ and $\istypeU{\Delta}{\tau'}$ then $\isctxU{\Delta, \Dhyp{t}}{\Gamma}$.
\item \begin{enumerate}
  \item If $\hastypeU{\Delta}{[\tau'/t]\Gamma}{[\tau'/t]e}{[\tau'/t]\tau}$ and $\istypeU{\Delta}{\tau'}$ then $\hastypeU{\Delta, \Dhyp{t}}{\Gamma}{e}{\tau}$.
  \item \begin{grayparbox}
  If $\ruleType{\Delta}{[\tau'/t]\Gamma}{[\tau'/t]r}{[\tau'/t]\tau}{[\tau'/t]\tau''}$ and $\istypeU{\Delta}{\tau'}$ then $\ruleType{\Delta, \Dhyp{t}}{\Gamma}{r}{\tau}{\tau''}$.
  \end{grayparbox}
  \end{enumerate}
\item \begin{enumerate}
  \item If $\hastypeU{\Delta}{\Gamma}{[e'/x]e}{\tau}$ and $\hastypeU{\Delta}{\Gamma}{e'}{\tau'}$ then $\hastypeU{\Delta}{\Gamma, \Ghyp{x}{\tau'}}{e}{\tau}$.
  \item \graytxtbox{If $\ruleType{\Delta}{\Gamma}{[e'/x]r}{\tau}{\tau''}$ and $\hastypeU{\Delta}{\Gamma}{e'}{\tau'}$ then $\ruleType{\Delta}{\Gamma, \Ghyp{x}{\tau'}}{r}{\tau}{\tau''}$.}
  \end{enumerate}
\end{enumerate}\end{lemma}
\begin{proof-sketch} ~
\begin{enumerate}
\item By rule induction over Rules (\ref{rules:istypeU}) and case analysis over the definition of substitution. In all cases, the derivation of $\istypeU{\Delta}{[\tau'/t]\tau}$ does not depend on the form of $\tau'$.
%\item Context formation of $[\tau'/t]\Gamma$ does not depend on the structure of $\tau'$.
\item By \graytxtbox{mutual} rule induction over Rules (\ref{rules:hastypeUP}) \graytxtbox{and Rule (\ref{rule:ruleType})} and case analysis over the definition of substitution. In all cases, the derivation of $\hastypeU{\Delta}{[\tau'/t]\Gamma}{[\tau'/t]e}{[\tau'/t]\tau}$ \graytxtbox{or $\ruleType{\Delta}{[\tau'/t]\Gamma}{[\tau'/t]r}{[\tau'/t]\tau}{[\tau'/t]\tau''}$} does not depend on the form of $\tau'$.
\item By \graytxtbox{mutual} rule induction over Rules (\ref{rules:hastypeUP}) \graytxtbox{and Rule (\ref{rule:ruleType})} and case analysis over the definition of substitution. In all cases, the derivation of $\hastypeU{\Delta}{\Gamma}{[e'/x]e}{\tau}$ \graytxtbox{or $\ruleType{\Delta}{\Gamma}{[e'/x]r}{\tau}{\tau''}$} does not depend on the form of $e'$.
\end{enumerate}
\end{proof-sketch}

\begin{grayparbox}
The Pattern Regularity Lemma establishes that the hypotheses generated by checking a pattern against a well-formed type involve only well-formed types.
\begin{lemma}[Pattern Regularity]\label{lemma:pattern-regularity-UP} 
If $\patType{\pctx}{p}{\tau}$ and $\istypeU{\Delta}{\tau}$ then $\isctxU{\Delta}{\pctx}$.
\end{lemma}
\begin{proof} By rule induction over Rules (\ref{rules:patType}).
\begin{byCases}
\item[\text{(\ref{rule:patType-var})}] ~
\begin{pfsteps*}
  \item $p=x$ \BY{assumption}
  \item $\pctx=x : \tau$ \BY{assumption}
  \item $\istypeU{\Delta}{\tau}$ \BY{assumption}\pflabel{istypeU}
  \item $\isctxU{\Delta}{\Ghyp{x}{\tau}}$ \BY{Definition \ref{def:isctxU} on \pfref{istypeU}}
 \end{pfsteps*}
 \resetpfcounter
\item[\text{(\ref{rule:patType-wild})}] ~
\begin{pfsteps}
\item \pctx=\emptyset \BY{assumption}
\item \isctxU{\Delta}{\emptyset} \BY{Definition \ref{def:isctxU}}
\end{pfsteps}
\resetpfcounter

\item[\text{(\ref{rule:patType-tpl})}] ~
\begin{pfsteps*}
  \item $p=\aetplp{\labelset}{\mapschema{p}{i}{\labelset}}$ \BY{assumption}
  \item $\tau=\aprod{\labelset}{\mapschema{\tau}{i}{\labelset}}$ \BY{assumption}
  \item $\pctx=\cup_{i \in \labelset} \pctx_i$ \BY{assumption}
  \item $\{\patType{\pctx_i}{p_i}{\tau_i}\}_{i \in \labelset}$ \BY{assumption}\pflabel{patType}
  \item $\istypeU{\Delta}{\aprod{\labelset}{\mapschema{\tau}{i}{\labelset}}}$ \BY{assumption} \pflabel{istypeU}
  \item $\{\istypeU{\Delta}{\tau_i}\}_{i \in \labelset}$ \BY{Inversion of Rule (\ref{rule:istypeU-prod}) on \pfref{istypeU}}\pflabel{istypeU-each}
  \item $\{\isctxU{\Delta}{\pctx_i}\}_{i \in \labelset}$ \BY{IH over \pfref{patType} and \pfref{istypeU-each}} \pflabel{biggy}
  \item $\isctxU{\Delta}{\cup_{i \in \labelset} \pctx_i}$ \BY{Definition \ref{def:isctxU} on \pfref{biggy}, then Definition \ref{def:isctxU} again, using the definition of typing context union iteratively}
\end{pfsteps*}
\resetpfcounter

\item[\text{(\ref{rule:patType-inj})}] ~
\begin{pfsteps*}
  \item $p=\aeinjp{\ell}{p'}$ \BY{assumption}
  \item $\tau=\asum{\labelset, \ell}{\mapschema{\tau}{i}{\labelset}; \mapitem{\ell}{\tau'}}$ \BY{assumption}
  \item $\istypeU{\Delta}{\asum{\labelset, \ell}{\mapschema{\tau}{i}{\labelset}; \mapitem{\ell}{\tau'}}}$ \BY{assumption} \pflabel{istype}
  \item $\patType{\pctx}{p'}{\tau'}$ \BY{assumption} \pflabel{patType}
  \item $\istypeU{\Delta}{\tau'}$ \BY{Inversion of Rule (\ref{rule:istypeU-sum}) on \pfref{istype}} \pflabel{istypeTwo} 
  \item $\isctxU{\Delta}{\pctx}$ \BY{IH on \pfref{patType} and \pfref{istypeTwo}}
\end{pfsteps*}
\resetpfcounter
\end{byCases}
\end{proof}
\end{grayparbox}

% Finally, the Regularity Lemma establishes that the type assigned to an expression under a well-formed typing context is well-formed. 
% \begin{lemma}[Regularity]\label{lemma:regularity-UP} ~
% \begin{enumerate}
% \item If $\hastypeU{\Delta}{\Gamma}{e}{\tau}$ and $\isctxU{\Delta}{\Gamma}$ then $\istypeU{\Delta}{\tau}$.
% \item \graytxtbox{If $\ruleType{\Delta}{\Gamma}{r}{\tau}{\tau'}$ and $\isctxU{\Delta}{\Gamma}$ then $\istypeU{\Delta}{\tau'}$.}
% \end{enumerate}
% \end{lemma}
% \begin{proof-sketch} By \graytxtbox{mutual} rule induction over Rules (\ref{rules:hastypeUP}) \graytxtbox{and Rule (\ref{rule:ruleType})}, and Lemma \ref{lemma:substitution-UP} \graytxtbox{and Lemma \ref{lemma:pattern-regularity-UP}}.
% \end{proof-sketch}

\subsection{Structural Dynamics}\vspace{-4px}
The \emph{structural dynamics} is specified as a transition system, and is organized around judgements of the following form:
\vspace{-4px}\[\begin{array}{ll}
\textbf{Judgement Form} & \textbf{Description}\\
\stepsU{e}{e'} & \text{$e$ transitions to $e'$}\\
\isvalU{e} & \text{$e$ is a value}\\
\LCC \lightgray & \lightgray \\
\matchfail{e} & \text{$e$ raises match failure} \ECC
\end{array}\]\vspace{-4px}
We also define auxiliary judgements for \emph{iterated transition}, $\multistepU{e}{e'}$, and \emph{evaluation}, $\evalU{e}{e'}$.


\begin{definition}[Iterated Transition]\label{defn:iterated-transition-UP} Iterated transition, $\multistepU{e}{e'}$, is the reflexive, transitive closure of the transition judgement, $\stepsU{e}{e'}$.\end{definition}


\begin{definition}[Evaluation]\label{defn:evaluation-UP}  $\evalU{e}{e'}$ iff $\multistepU{e}{e'}$ and $\isvalU{e'}$. \end{definition}

Our subsequent developments do not make mention of particular rules in the dynamics, nor do they make mention of other judgements, not listed above,  that are used only for defining the dynamics of the match operator, so we do not produce these details here. Instead, it suffices to state the following conditions.

\begin{condition}[Canonical Forms]\label{condition:canonical-forms-UP} If $\hastypeUC{e}{\tau}$ and $\isvalU{e}$ then:
\begin{enumerate}
\item If $\tau=\aparr{\tau_1}{\tau_2}$ then $e=\aelam{\tau_1}{x}{e'}$ and $\hastypeUCO{\Ghyp{x}{\tau_1}}{e'}{\tau_2}$.
\item If $\tau=\aall{t}{\tau'}$ then $e=\aetlam{t}{e'}$ and $\hastypeUCO{\Dhyp{t}}{e'}{\tau'}$.
\item If $\tau=\arec{t}{\tau'}$ then $e=\aefold{e'}$ and $\hastypeUC{e'}{[\abop{rec}{t.\tau'}/t]\tau'}$ and $\isvalU{e'}$. 
\item If $\tau=\aprod{\labelset}{\mapschema{\tau}{i}{\labelset}}$ then $e=\aetpl{\labelset}{\mapschema{e}{i}{\labelset}}$ and $\hastypeUC{e_i}{\tau_i}$ and $\isvalU{e_i}$ for each $i \in \labelset$.
\item If $\tau=\asum{\labelset}{\mapschema{\tau}{i}{\labelset}}$ then for some label set $L'$ and label $\ell$ and type $\tau'$, we have that $\labelset=\labelset', \ell$ and $\tau=\asum{\labelset', \ell}{\mapschema{\tau}{i}{\labelset'}; \mapitem{\ell}{\tau'}}$ and $e=\aein{\ell}{e'}$ and $\hastypeUC{e'}{\tau'}$ and $\isvalU{e'}$.
\end{enumerate}\end{condition}


\begin{condition}[Preservation]\label{condition:preservation-UP} If $\hastypeUC{e}{\tau}$ and $\stepsU{e}{e'}$ then $\hastypeUC{e'}{\tau}$. \end{condition}

\begin{condition}[Progress]\label{condition:progress-UP} If $\hastypeUC{e}{\tau}$ then either $\isvalU{e}$ \graytxtbox{or $\matchfail{e}$} or there exists an $e'$ such that $\stepsU{e}{e'}$. \end{condition}

\section{Unexpanded Language (UL)}\label{appendix:SES-uexps}
\subsection{Syntax}\label{appendix:SES-syntax}\label{appendix:SES-shared-forms}
\subsubsection{Stylized Syntax}
\[\begin{array}{lllllll}
\textbf{Sort} & &  
%&\textbf{Operational Form} 
& \textbf{Stylized Form} & \textbf{Description}\\
\mathsf{UTyp} & \utau & ::= 
% &\ut 
& \ut & \text{identifier}\\
&& 
%& \auparr{\utau}{\utau} 
& \parr{\utau}{\utau} & \text{partial function}\\
&&
%& \auall{\ut}{\utau} 
& \forallt{\ut}{\utau} & \text{polymorphic}\\
&&
%& \aurec{\ut}{\utau} 
& \rect{\ut}{\utau} & \text{recursive}\\
&&
%& \auprod{\labelset}{\mapschema{\utau}{i}{\labelset}} 
& \prodt{\mapschema{\utau}{i}{\labelset}} & \text{labeled product}\\
&&
%& \ausum{\labelset}{\mapschema{\utau}{i}{\labelset}} 
& \sumt{\mapschema{\utau}{i}{\labelset}} & \text{labeled sum}\\
\mathsf{UExp} & \ue & ::= 
%& \ux 
& \ux & \text{identifier}\\
&&
%
& \asc{\ue}{\utau} & \text{ascription}\\
&&
%
& \letsyn{\ux}{\ue}{\ue} & \text{value binding}\\
&&
%& \aulam{\utau}{\ux}{\ue} 
& \lam{\ux}{\utau}{\ue} & \text{abstraction}\\
&&
%& \auap{\ue}{\ue} 
& \ap{\ue}{\ue} & \text{application}\\
&&
%& \autlam{\ut}{\ue} 
& \Lam{\ut}{\ue} & \text{type abstraction}\\
&&
%& \autap{\ue}{\utau} 
& \App{\ue}{\utau} & \text{type application}\\
&&
%& \aufold{\ut}{\utau}{\ue} 
& \fold{\ue} & \text{fold}\\
&&
%& \auunfold{\ue} 
& \unfold{\ue} & \text{unfold}\\
&&
%& \autpl{\labelset}{\mapschema{\ue}{i}{\labelset}} 
& \tpl{\mapschema{\ue}{i}{\labelset}} & \text{labeled tuple}\\
&&
%& \aupr{\ell}{\ue} 
& \prj{\ue}{\ell} & \text{projection}\\
&&
%& \auin{\labelset}{\ell}{\mapschema{\utau}{i}{\labelset}}{\ue} 
& \inj{\ell}{\ue} & \text{injection}\\
&&
%& \aucase{\labelset}{\utau}{\ue}{\mapschemab{\ux}{\ue}{i}{\labelset}} 
& \caseof{\ue}{\mapschemab{\ux}{\ue}{i}{\labelset}} & \text{case analysis}\\
&&
%& \audefuetsm{\utau}{e}{\tsmv}{\ue} 
& \uesyntax{\tsmv}{\utau}{e}{\ue} & \text{seTSM definition}\\ 
&&
%& \autsmap{b}{\tsmv} 
& \utsmap{\tsmv}{b} & \text{seTSM application}\\%\ECC
\LCC  \lightgray & \lightgray & \lightgray
& \lightgray 
& \lightgray & \lightgray \\
&&
%& \aumatchwith{n}{\utau}{\ue}{\seqschemaX{\urv}} 
& \matchwith{\ue}{\seqschemaX{\urv}} & \text{match}\\
&&
%& \audefuptsm{\utau}{e}{\tsmv}{\ue} 
& \usyntaxup{\tsmv}{\utau}{e}{\ue}
& \text{spTSM definition}\\
\mathsf{URule} & \urv & ::= 
%& \aumatchrule{\upv}{\ue} 
& \matchrule{\upv}{\ue} & \text{match rule}\\
\mathsf{UPat} & \upv & ::= 
%& \ux 
& \ux & \text{identifier pattern}\\
&&
%& \auwildp 
& \wildp & \text{wildcard pattern}\\
&&
%& \aufoldp{\upv} 
& \foldp{\upv} & \text{fold pattern}\\
&&
%& \autplp{\labelset}{\mapschema{\upv}{i}{\labelset}} 
& \tplp{\mapschema{\upv}{i}{\labelset}} & \text{labeled tuple pattern}\\
&&
%& \auinjp{\ell}{\upv} 
& \injp{\ell}{\upv} & \text{injection pattern}\\
% \LCC &&& \lightgray & \lightgray & \lightgray\\
&&
%& \auapuptsm{b}{\tsmv} 
& \utsmap{\tsmv}{b} & \text{spTSM application}\ECC
\end{array}\]

\clearpage

\paragraph{Body Lengths}\label{appendix:SES-body-lengths}
We write $\sizeof{b}$ for the length of $b$. The metafunction $\sizeof{\ue}$ computes the sum of the lengths of expression literal bodies in $\ue$:
\[
\begin{array}{ll}
\sizeof{\ux} & = 0\\
\sizeof{\asc{\ue}{\utau}} & = \sizeof{\ue}\\
\sizeof{\letsyn{\ux}{\ue_1}{\ue_2}} & = \sizeof{\ue_1} + \sizeof{\ue_2}\\
\sizeof{\lam{\ux}{\utau}{\ue}} &= \sizeof{\ue}\\
\sizeof{\ap{\ue_1}{\ue_2}} & = \sizeof{\ue_1} + \sizeof{\ue_2}\\
\sizeof{\Lam{\ut}{\ue}} & = \sizeof{\ue}\\
\sizeof{\App{\ue}{\utau}} & = \sizeof{\ue}\\
\sizeof{\fold{\ue}} & = \sizeof{\ue}\\
\sizeof{\unfold{\ue}} & = \sizeof{\ue}\\
%\end{align*}
%\begin{align*}
\sizeof{\tpl{\mapschema{\ue}{i}{\labelset}}} & = \sum_{i \in \labelset} \sizeof{\ue_i}\\
\sizeof{\prj{\ell}{\ue}} & = \sizeof{\ue}\\
\sizeof{\inj{\ell}{\ue}} & = \sizeof{\ue}\\
\sizeof{\caseof{\ue}{\mapschemab{\ux}{\ue}{i}{\labelset}}} & = \sizeof{\ue} + \sum_{i \in \labelset} \sizeof{\ue_i}\\
\sizeof{\uesyntax{\tsmv}{\utau}{\eparse}{\ue}} & = \sizeof{\ue}\\
\sizeof{\utsmap{\tsmv}{b}} & = \sizeof{b}\\
\LCC \lightgray & \lightgray\\
\sizeof{\matchwith{\ue}{\seqschemaX{\urv}}} & = \sizeof{\ue} + \sum_{1 \leq i \leq n} \sizeof{r_i}\\
\sizeof{\usyntaxup{\tsmv}{\utau}{\eparse}{\ue}} & = \sizeof{\ue}\ECC
\end{array}
\]
\vspace{-3px}\begin{grayparbox}\vspace{3px}and $\sizeof{\urv}$ computes the sum of the lengths of expression literal bodies in $\urv$:
\begin{align*}
\sizeof{\matchrule{\upv}{\ue}} & = \sizeof{\ue}
\end{align*}
Similarly, the metafunction $\sizeof{\upv}$ computes the sum of the lengths of the pattern literal bodies in $\upv$:
\begin{align*}
\sizeof{\ux} & = 0\\
\sizeof{\foldp{\upv}} & = \sizeof{\upv}\\
\sizeof{\tplp{\mapschema{\upv}{i}{\labelset}}} & = \sum_{i \in \labelset} \sizeof{\upv_i}\\
\sizeof{\injp{\ell}{\upv}} & = \sizeof{\upv}\\
\sizeof{\utsmap{\tsmv}{b}} & = \sizeof{b}
\end{align*}
\end{grayparbox}

\paragraph{Common Unexpanded Forms} Each expanded form maps onto an unexpanded form. We refer to these as the \emph{common forms}. In particular:
\begin{itemize}
\item Each type variable, $t$, maps onto a unique {type identifier}, written $\sigilof{t}$.
\item Each type, $\tau$, maps onto an unexpanded type, $\Uof{\tau}$, as follows: 
  \begin{align*}
  \Uof{t} &= \sigilof{t}\\
  \Uof{\aparr{\tau_1}{\tau_2}} & = \parr{\Uof{\tau_1}}{\Uof{\tau_2}}\\
  \Uof{\aall{t}{\tau}} & = \forallt{\sigilof{t}}{\Uof{\tau}}\\
  \Uof{\arec{t}{\tau}} & = \rect{\sigilof{t}}{\Uof{\tau}}\\
  \Uof{\aprod{\labelset}{\mapschema{\tau}{i}{\labelset}}} & = \prodt{\mapschemax{\Uofv}{\tau}{i}{\labelset}}\\
  \Uof{\asum{\labelset}{\mapschema{\tau}{i}{\labelset}}} & = \sumt{\mapschemax{\Uofv}{\tau}{i}{\labelset}}
  \end{align*}
\item Each expression variable, $x$, maps onto a unique expression identifier, written $\sigilof{x}$.
\item Each expanded expression, $e$, maps onto an unexpanded expression, $\Uof{e}$, as follows:
\[\arraycolsep=1pt\begin{array}{rl}
\Uof{x} & = \sigilof{x}\\
\Uof{\aelam{\tau}{x}{e}} & = \lam{\sigilof{x}}{\Uof{\tau}}{\Uof{e}}\\
\Uof{\aeap{e_1}{e_2}} & = \ap{\Uof{e_1}}{\Uof{e_2}}\\
\Uof{\aetlam{t}{e}} & = \Lam{\sigilof{t}}{\Uof{e}}\\
\Uof{\aetap{e}{\tau}} & = \App{\Uof{e}}{\Uof{\tau}}\\
\Uof{\aefold{e}} & = \fold{\Uof e}\\
\Uof{\aeunfold{e}} & = \unfold{\Uof{e}}\\
\Uof{\aetpl{\labelset}{\mapschema{e}{i}{\labelset}}} & = \tpl{\mapschemax{\Uofv}{e}{i}{\labelset}}\\
\Uof{\aepr{\ell}{e}} & = \prj{\Uof{e}}{\ell}\\
\Uof{\aein{\ell}{e}} &= \inj{\ell}{\Uof{e}}\\
\LCC \lightgray & \lightgray \\
\Uof{\aematchwith{n}{e}{\seqschemaX{r}}} & = \matchwith{\Uof{e}}{\seqschemaXx{\Uofv}{r}}\ECC
\end{array}\]
\end{itemize}
\begin{grayparbox}
\begin{itemize}
\item Each expanded rule, $r$, maps onto an unexpanded rule, $\Uof{r}$, as follows:
\[\arraycolsep=1pt\begin{array}{rl}
\LCC \lightgray & \lightgray \\
\Uof{\aematchrule{p}{e}} & = \aumatchrule{\Uof{p}}{\Uof{e}}\ECC
\end{array}\]
\item Each expanded pattern, $p$, maps onto the unexpanded pattern, $\Uof{p}$, as follows:
\[\arraycolsep=1pt\begin{array}{rl}
\LCC \lightgray & \lightgray \\
\Uof{x} & = \sigilof{x}\\
\Uof{\aewildp} &= \auwildp\\
\Uof{\aefoldp{p}} &= \aufoldp{\Uof{p}}\\
\Uof{\aetplp{\labelset}{\mapschema{p}{i}{\labelset}}} & = \autplp{\labelset}{\mapschemax{\Uofv}{p}{i}{\labelset}}\\
\Uof{\aeinjp{\ell}{p}} & = \auinjp{\ell}{\Uof{p}}\ECC
\end{array}\]
\end{itemize}
\end{grayparbox}
\vspace{-10px}
\subsubsection{Textual Syntax}\vspace{-3px} In addition to the stylized syntax, there is also a context-free textual syntax for the UL. For our purposes, we need only posit the existence of partial metafunctions $\parseUTypF{b}$ and $\parseUExpF{b}$\graytxtbox{~and $\parseUPatF{b}$}. 

\begin{condition}[Textual Representability]\label{condition:textual-representability-SES} ~
\begin{enumerate}
\item For each $\utau$, there exists $b$ such that $\parseUTyp{b}{\utau}$. 
\item For each $\ue$, there exists $b$ such that $\parseUExp{b}{\ue}$.
% \item For each $\urv$, there exists $b$ such that $\parseURule{b}{\urv}$.
\item \graytxtbox{For each $\upv$, there exists $b$ such that $\parseUPat{b}{\upv}$.}
\end{enumerate}
\end{condition}

We also impose the following technical condition\graytxtbox{s}.

\begin{condition}[Expression Parsing Monotonicity]\label{condition:body-parsing} If $\parseUExp{b}{\ue}$ then $\sizeof{\ue} < \sizeof{b}$.\end{condition}

\begin{grayparbox}\begin{condition}[Pattern Parsing Monotonicity]\label{condition:pattern-parsing} If $\parseUPat{b}{\upv}$ then $\sizeof{\upv} < \sizeof{b}$.\end{condition}\end{grayparbox}

\subsection{Type Expansion}
\emph{Unexpanded type formation contexts}, $\uDelta$, are of the form $\uDD{\uD}{\Delta}$, i.e. they consist of a \emph{type identifier expansion context}, $\uD$, paired with a type formation context, $\Delta$. 

A \emph{type identifier expansion context}, $\uD$, is a finite function that maps each type identifier $\ut \in \domof{\uD}$ to the hypothesis $\vExpands{\ut}{t}$, for some type variable $t$. We write $\ctxUpdate{\uD}{\ut}{t}$ for the type identifier expansion context that maps $\ut$ to $\vExpands{\ut}{t}$ and defers to $\uD$ for all other type identifiers (i.e. the previous mapping is \emph{updated}.) 

We define $\uDelta, \uDhyp{\ut}{t}$ when $\uDelta=\uDD{\uD}{\Delta}$ as an abbreviation of  \[\uDD{\ctxUpdate{\uD}{\ut}{t}}{\Delta, \Dhyp{t}}\]%type identifier expansion context is always extended/updated together with 

\begin{definition}[Unexpanded Type Formation Context Formation] $\uDOK{\uDD{\uD}{\Delta}}$ iff for each $\uDhyp{\ut}{t} \in \uD$ we have $\Dhyp{t} \in \Delta$. \end{definition}

\vspace{10px}\noindent\fbox{\strut$\expandsTU{\uDelta}{\utau}{\tau}$}~~$\utau$ has well-formed expansion $\tau$
\begin{subequations}\label{rules:expandsTU}
\begin{equation}\label{rule:expandsTU-var}
\inferrule{ }{\expandsTU{\uDelta, \uDhyp{\ut}{t}}{\ut}{t}}
\end{equation}
\begin{equation}\label{rule:expandsTU-parr}
\inferrule{
  \expandsTU{\uDelta}{\utau_1}{\tau_1}\\
  \expandsTU{\uDelta}{\utau_2}{\tau_2}
}{\expandsTU{\uDelta}{\auparr{\utau_1}{\utau_2}}{\aparr{\tau_1}{\tau_2}}}
\end{equation}
\begin{equation}\label{rule:expandsTU-all}
  \inferrule{
    \expandsTU{\uDelta, \uDhyp{\ut}{t}}{\utau}{\tau}
  }{
    \expandsTU{\uDelta}{\auall{\ut}{\utau}}{\aall{t}{\tau}}
  }
\end{equation}
\begin{equation}\label{rule:expandsTU-rec}
  \inferrule{
    \expandsTU{\uDelta, \uDhyp{\ut}{t}}{\utau}{\tau}
  }{
    \expandsTU{\uDelta}{\aurec{\ut}{\utau}}{\arec{t}{\tau}}
  }
\end{equation}
\begin{equation}\label{rule:expandsTU-prod}
  \inferrule{
    \{\expandsTU{\uDelta}{\utau_i}{\tau_i}\}_{i \in \labelset}
  }{
    \expandsTU{\uDelta}{\auprod{\labelset}{\mapschema{\utau}{i}{\labelset}}}{\aprod{\labelset}{\mapschema{\tau}{i}{\labelset}}}
  }
\end{equation}
\begin{equation}\label{rule:expandsTU-sum}
  \inferrule{
    \{\expandsTU{\uDelta}{\utau_i}{\tau_i}\}_{i \in \labelset}
  }{
    \expandsTU{\uDelta}{\ausum{\labelset}{\mapschema{\utau}{i}{\labelset}}}{\asum{\labelset}{\mapschema{\tau}{i}{\labelset}}}
  }
\end{equation}
\end{subequations}
% \emph{Unexpanded type formation contexts}, $\uDelta$, are of the form $\uDD{\uD}{\Delta}$, where $\uD$ is a \emph{type identifier expansion context}, and $\Delta$ is a type formation context. A type identifier expansion context, $\uD$, is a finite function that maps each type identifier $\ut \in \domof{\uD}$ to the hypothesis $\vExpands{\ut}{t}$, for some type variable $t$. We write $\ctxUpdate{\uD}{\ut}{t}$ for the type identifier expansion context that maps $\ut$ to $\vExpands{\ut}{t}$ and defers to $\uD$ for all other type identifiers (i.e. the previous mapping, if it exists, is updated). 
% We define $\uDelta, \uDhyp{\ut}{t}$ when $\uDelta=\uDD{\uD}{\Delta}$ as an abbreviation of  \[\uDD{\ctxUpdate{\uD}{\ut}{t}}{\Delta, \Dhyp{t}}\]%type identifier expansion context is always extended/updated together with 
% %We write $\uDeltaOK{\uDelta}$ when $\uDelta=\uDD{\uD}{\Delta}$ and each type variable in $\uD$ also appears in $\Delta$.
% %\begin{definition}\label{def:uDeltaOK} $\uDeltaOK{\uDD{\uD}{\Delta}}$ iff for each $\vExpands{\ut}{t} \in \uD$, we have $\Dhyp{t} \in \Delta$.\end{definition}

\subsection{Typed Expression Expansion}\label{appendix:typed-expression-expansion-S}
\subsubsection{Contexts}
\emph{Unexpanded typing contexts}, $\uGamma$, are, similarly, of the form $\uGG{\uG}{\Gamma}$, where $\uG$ is an \emph{expression identifier expansion context}, and $\Gamma$ is a typing context. An expression identifier expansion context, $\uG$, is a finite function that maps each expression identifier $\ux \in \domof{\uG}$ to the hypothesis $\vExpands{\ux}{x}$, for some expression variable, $x$. We write $\ctxUpdate{\uG}{\ux}{x}$ for the expression identifier expansion context that maps $\ux$ to $\vExpands{\ux}{x}$ and defers to $\uG$ for all other expression identifiers (i.e. the previous mapping is updated.) 
%We write $\uGammaOK{\uGamma}$ when $\uGamma=\uGG{\uG}{\Gamma}$ and each expression variable in $\uG$ is assigned a type by $\Gamma$.
%\noindent 

We define $\uGamma, \uGhyp{\ux}{x}{\tau}$ when $\uGamma = \uGG{\uG}{\Gamma}$ as an abbreviation of \[\uGG{\uG, \vExpands{\ux}{x}}{\Gamma, \Ghyp{x}{\tau}}\]

\begin{definition}[Unexpanded Typing Context Formation] $\uGammaOK{\uGG{\uG}{\Gamma}}$ iff $\isctxU{\Delta}{\Gamma}$ and for each $\vExpands{\ux}{x} \in \uG$, we have $x \in \domof{\Gamma}$.\end{definition}


\subsubsection{Body Encoding and Decoding}
An assumed type abbreviated $\tBody$ classifies encodings of literal bodies, $b$. The mapping from literal bodies to values of type $\tBody$ is defined by the \emph{body encoding judgement} $\encodeBody{b}{\ebody}$. An inverse mapping is defined   by the \emph{body decoding judgement} $\decodeBody{\ebody}{b}$.
\[\begin{array}{ll}
\textbf{Judgement Form} & \textbf{Description}\\
\encodeBody{b}{e} & \text{$b$ has encoding $e$}\\
\decodeBody{e}{b} & \text{$e$ has decoding $b$}
\end{array}\]
The following condition establishes an isomorphism between literal bodies and values of type $\tBody$ mediated by the judgements above.
\begin{condition}[Body Isomorphism]\label{condition:body-isomorphism} ~
\begin{enumerate}
\item For every literal body $b$, we have that $\encodeBody{b}{\ebody}$ for some $\ebody$ such that $\hastypeUC{\ebody}{\tBody}$ and $\isvalU{\ebody}$.
\item If $\hastypeUC{\ebody}{\tBody}$ and $\isvalU{\ebody}$ then $\decodeBody{\ebody}{b}$ for some $b$.
\item If $\encodeBody{b}{\ebody}$ then $\decodeBody{\ebody}{b}$.
\item If $\hastypeUC{\ebody}{\tBody}$ and $\isvalU{\ebody}$ and $\decodeBody{\ebody}{b}$ then $\encodeBody{b}{\ebody}$. 
\item If $\encodeBody{b}{\ebody}$ and $\encodeBody{b}{\ebody'}$ then $\ebody = \ebody'$.
\item If $\hastypeUC{\ebody}{\tBody}$ and $\isvalU{\ebody}$ and $\decodeBody{\ebody}{b}$ and $\decodeBody{\ebody}{b'}$ then $b=b'$.
\end{enumerate}
\end{condition}
We also assume a partial metafunction, $\bsubseq{b}{m}{n}$, which extracts a subsequence of $b$ starting at position $m$ and ending at position $n$, inclusive, where $m$ and $n$ are natural numbers. The following condition is technically necessary.
\begin{condition}[Body Subsequencing]\label{condition:body-subsequences} If $\bsubseq{b}{m}{n}=b'$ then $\sizeof{b'} \leq \sizeof{b}$. \end{condition}

\subsubsection{Parse Results}
 The type abbreviated $\tParseResultExp$, and an auxiliary abbreviation used below, is defined as follows:
\begin{align*}
L_\mathtt{SE} & \defeq \lbltxt{ParseError}, \lbltxt{SuccessE}\\
\tParseResultExp & \defeq \asum{L_\mathtt{SE}}{
  \mapitem{\lbltxt{ParseError}}{\prodt{}}, 
  \mapitem{\lbltxt{SuccessE}}{\tCEExp}
}\\
\lbltxt{SuccessE}\cdot e & \defeq \aein{L_\mathtt{SE}}{\mathtt{SuccessE}}{\mapitem{\mathtt{ParseError}}{\tpl{}}, \mapitem{\mathtt{SuccessE}}{\tCEExp}}{e}
\end{align*} %[\mapitem{\lbltxt{ParseError}}{\prodt{}}, \mapitem{\lbltxt{SuccessE}}{\tCEExp}]

\begin{grayparbox}
 The type abbreviated $\tParseResultPat$, and an auxiliary abbreviation used below, is defined as follows:
\begin{align*}
L_\mathtt{SP} & \defeq \lbltxt{ParseError}, \lbltxt{SuccessP}\\
\tParseResultExp & \defeq \asum{L_\mathtt{SP}}{
  \mapitem{\lbltxt{ParseError}}{\prodt{}}, 
  \mapitem{\lbltxt{SuccessP}}{\tCEPat}
}\\
\lbltxt{SuccessP}\cdot e & \defeq \aein{L_\mathtt{SP}}{\mathtt{SuccessP}}{\mapitem{\mathtt{ParseError}}{\tpl{}}, \mapitem{\mathtt{SuccessP}}{\tCEPat}}{e}
\end{align*} %[\mapitem{\lbltxt{ParseError}}{\prodt{}}, \mapitem{\lbltxt{SuccessE}}{\tCEExp}]
\end{grayparbox}

\subsubsection{seTSM Contexts}

\emph{seTSM contexts}, $\uPsi$, are of the form $\uAS{\uA}{\Psi}$, where $\uA$ is a \emph{TSM identifier expansion context} and $\Psi$ is a \emph{seTSM definition context}. 

A \emph{TSM identifier expansion context}, $\uA$, is a finite function mapping each TSM identifier $\tsmv \in \domof{\uA}$ to the \emph{TSM identifier expansion}, $\vExpands{\tsmv}{a}$, for some \emph{TSM name}, $a$. We write $\ctxUpdate{\uA}{\tsmv}{a}$ for the TSM identifier expansion context that maps $\tsmv$ to $\vExpands{\tsmv}{a}$, and defers to $\uA$ for all other TSM identifiers (i.e. the previous mapping is \emph{updated}.)

An \emph{seTSM definition context}, $\Psi$, is a finite function mapping each TSM name $a \in \domof{\Psi}$ to an \emph{expanded seTSM definition}, $\xuetsmbnd{a}{\tau}{\eparse}$, where $\tau$ is the seTSM's type annotation, and $\eparse$ is its parse function. We write $\Psi, \xuetsmbnd{a}{\tau}{\eparse}$ when $a \notin \domof{\Psi}$ for the extension of $\Psi$ that maps $a$ to $\xuetsmbnd{a}{\tau}{\eparse}$. We write $\uetsmenv{\Delta}{\Psi}$  when all the type annotations in $\Psi$ are well-formed assuming $\Delta$, and the parse functions in $\Psi$ are closed and of the appropriate type.

\begin{definition}[seTSM Definition Context Formation]\label{def:seTSM-def-ctx-formation} $\uetsmenv{\Delta}{\Psi}$ iff for each $\xuetsmbnd{a}{\tau}{\eparse} \in \Psi$, we have $\istypeU{\Delta}{\tau}$ and $\hastypeU{\emptyset}{\emptyset}{\eparse}{\aparr{\tBody}{\tParseResultExp}}$.\end{definition}

\begin{definition}[seTSM Context Formation] $\uetsmctx{\Delta}{\uAS{\uA}{\Psi}}$ iff $\uetsmenv{\Delta}{\Psi}$ and for each $\vExpands{\tsmv}{a} \in \uA$ we have $a \in \domof{\Psi}$.
\end{definition}

We define $\uPsi, \uShyp{\tsmv}{a}{\tau}{\eparse}$, when $\uPsi=\uAS{\uA}{\Phi}$, as an abbreviation of \[\uAS{\ctxUpdate{\uA}{\tsmv}{a}}{\Psi, \xuetsmbnd{a}{\tau}{\eparse}}\]
%\vspace{10px}

\begin{grayparbox}\vspace{-15px}\subsubsection{spTSM Contexts}
\emph{spTSM contexts}, $\uPhi$, are of the form $\uAS{\uA}{\Phi}$, where $\uA$ is a {TSM identifier expansion context}, defined above, and $\Phi$ is a \emph{spTSM definition context}. 

An \emph{spTSM definition context}, $\Phi$, is a finite function mapping each TSM name $a \in \domof{\Phi}$ to an \emph{expanded seTSM definition}, $\xuptsmbnd{a}{\tau}{\eparse}$, where $\tau$ is the spTSM's type annotation, and $\eparse$ is its parse function. We write $\Phi, \xuptsmbnd{a}{\tau}{\eparse}$ when $a \notin \domof{\Phi}$ for the extension of $\Phi$ that maps $a$ to $\xuptsmbnd{a}{\tau}{\eparse}$. We write $\uptsmenv{\Delta}{\Phi}$  when all the type annotations in $\Phi$ are well-formed assuming $\Delta$, and the parse functions in $\Phi$ are closed and of the appropriate type.

\begin{definition}[spTSM Definition Context Formation]\label{def:spTSM-def-ctx-formation} $\uptsmenv{\Delta}{\Phi}$ iff for each $\xuptsmbnd{a}{\tau}{\eparse} \in \Phi$, we have $\istypeU{\Delta}{\tau}$ and $\hastypeU{\emptyset}{\emptyset}{\eparse}{\aparr{\tBody}{\tParseResultPat}}$.\end{definition}

\begin{definition}[spTSM Context Formation] $\uptsmctx{\Delta}{\uAS{\uA}{\Phi}}$ iff $\uptsmenv{\Delta}{\Phi}$ and for each $\vExpands{\tsmv}{a} \in \uA$ we have $a \in \domof{\Phi}$.
\end{definition}

We define $\uPhi, \uPhyp{\tsmv}{a}{\tau}{\eparse}$, when $\uPhi=\uAS{\uA}{\Phi}$, as an abbreviation of \[\uAS{\ctxUpdate{\uA}{\tsmv}{a}}{\Phi, \xuptsmbnd{a}{\tau}{\eparse}}\]
\end{grayparbox}

\subsubsection{Typed Expression Expansion}\label{appendix:typed-expression-expansion-SES}
\vspace{8px}\noindent\fbox{\strut$\expandsSG{\uDelta}{\uGamma}{\uPsi}{\uPhi}{\ue}{e}{\tau}$}~~$\ue$ has expansion $e$ of type $\tau$
\begin{subequations}\label{rules:expandsU}
\begin{equation}\label{rule:expandsU-var}
  \inferrule{ }{
    \expandsSG{\uDelta}{\uGamma, \uGhyp{\ux}{x}{\tau}}{\uPsi}{\uPhi}{\ux}{x}{\tau}
  }
\end{equation}
\begin{equation}\label{rule:expandsU-asc}
  \inferrule{
    \expandsTU{\uDelta}{\utau}{\tau}\\
    \expandsSG{\uDelta}{\uGamma}{\uPsi}{\uPhi}{\ue}{e}{\tau}
  }{
    \expandsSG{\uDelta}{\uGamma}{\uPsi}{\uPhi}{\asc{\ue}{\utau}}{e}{\tau}
  }
\end{equation}
\begin{equation}\label{rule:expandsU-letsyn}
  \inferrule{
    \expandsSG{\uDelta}{\uGamma}{\uPsi}{\uPhi}{\ue_1}{e_1}{\tau_1}\\
    \expandsSG{\uDelta}{\uGamma, \uGhyp{\ux}{x}{\tau_1}}{\uPsi}{\uPhi}{\ue_2}{e_2}{\tau_2}
  }{
    \expandsSG{\uDelta}{\uGamma}{\uPsi}{\uPhi}{\letsyn{\ux}{\ue_1}{\ue_2}}{
      \aeap{\aelam{\tau_1}{x}{e_2}}{e_1}
    }{\tau_2}
  }
\end{equation}
\begin{equation}\label{rule:expandsU-lam}
  \inferrule{
    \expandsTU{\uDelta}{\utau}{\tau}\\
    \expandsSG{\uDelta}{\uGamma, \uGhyp{\ux}{x}{\tau}}{\uPsi}{\uPhi}{\ue}{e}{\tau'}
  }{
    \expandsSG{\uDelta}{\uGamma}{\uPsi}{\uPhi}{\lam{\ux}{\utau}{\ue}}{\aelam{\tau}{x}{e}}{\aparr{\tau}{\tau'}}
  }
\end{equation}
\begin{equation}\label{rule:expandsU-ap}
  \inferrule{
    \expandsSG{\uDelta}{\uGamma}{\uPsi}{\uPhi}{\ue_1}{e_1}{\aparr{\tau}{\tau'}}\\
    \expandsSG{\uDelta}{\uGamma}{\uPsi}{\uPhi}{\ue_2}{e_2}{\tau}
  }{
    \expandsSG{\uDelta}{\uGamma}{\uPsi}{\uPhi}{\ap{\ue_1}{\ue_2}}{\aeap{e_1}{e_2}}{\tau'}
  }
\end{equation}
\begin{equation}\label{rule:expandsU-tlam}
  \inferrule{
    \expandsSG{\uDelta, \uDhyp{\ut}{t}}{\uGamma}{\uPsi}{\uPhi}{\ue}{e}{\tau}
  }{
    \expandsSG{\uDelta}{\uGamma}{\uPsi}{\uPhi}{\Lam{\ut}{\ue}}{\aetlam{t}{e}}{\aall{t}{\tau}}
  }
\end{equation}
\begin{equation}\label{rule:expandsU-tap}
  \inferrule{
    \expandsSG{\uDelta}{\uGamma}{\uPsi}{\uPhi}{\ue}{e}{\aall{t}{\tau}}\\
    \expandsTU{\uDelta}{\utau'}{\tau'}
  }{
    \expandsSG{\uDelta}{\uGamma}{\uPsi}{\uPhi}{\App{\ue}{\utau'}}{\aetap{e}{\tau'}}{[\tau'/t]\tau}
  }
\end{equation}
\begin{equation}\label{rule:expandsU-fold}
  \inferrule{
    % \istypeU{\Delta, \Dhyp{t}}{\tau}\\
    \expandsSG{\uDelta}{\uGamma}{\uPsi}{\uPhi}{\ue}{e}{[\arec{t}{\tau}/t]\tau}
  }{
    \expandsSG{\uDelta}{\uGamma}{\uPsi}{\uPhi}{\fold{\ue}}{\aefold{e}}{\arec{t}{\tau}}
  }
\end{equation}
\begin{equation}\label{rule:expandsU-unfold}
  \inferrule{
    \expandsSG{\uDelta}{\uGamma}{\uPsi}{\uPhi}{\ue}{e}{\arec{t}{\tau}}
  }{
    \expandsSG{\uDelta}{\uGamma}{\uPsi}{\uPhi}{\unfold{\ue}}{\aeunfold{e}}{[\arec{t}{\tau}/t]\tau}
  }
\end{equation}
\begin{equation}\label{rule:expandsU-tpl}
  \inferrule{
    \{\expandsSG{\uDelta}{\uGamma}{\uPsi}{\uPhi}{\ue_i}{e_i}{\tau_i}\}_{i \in \labelset}
  }{
    \expandsSG{\uDelta}{\uGamma}{\uPsi}{\uPhi}{\tpl{\mapschema{\ue}{i}{\labelset}}}{\aetpl{\labelset}{\mapschema{e}{i}{\labelset}}}{\aprod{\labelset}{\mapschema{\tau}{i}{\labelset}}}
  }
\end{equation}
\begin{equation}\label{rule:expandsU-pr}
  \inferrule{
    \expandsSG{\uDelta}{\uGamma}{\uPsi}{\uPhi}{\ue}{e}{\aprod{\labelset, \ell}{\mapschema{\tau}{i}{\labelset}; \ell \hookrightarrow \tau}}
  }{
    \expandsSG{\uDelta}{\uGamma}{\uPsi}{\uPhi}{\prj{\ue}{\ell}}{\aepr{\ell}{e}}{\tau}
  }
\end{equation}
\begin{equation}\label{rule:expandsU-in}
  \inferrule{
   % \{\istypeU{\Delta}{\tau_i}\}_{i \in \labelset}\\
    % \istypeU{\Delta}{\tau'}\\
    \expandsSG{\uDelta}{\uGamma}{\uPsi}{\uPhi}{\ue}{e}{\tau'}
  }{
    % \left(\shortstack{
    %   $\uDelta~\uGamma~{\vdash_{\uPhi}}{\setlength{\fboxsep}{0px}\colorbox{lightgray}{$_{\mathstrut; \uPsi}$}}~ \inj{\ell}{\ue}$\\
    %   $\leadsto$\\
    %   $\aein{\ell}{e} : \asum{\labelset, \ell}{\mapschema{\tau}{i}{\labelset}; \ell \hookrightarrow \tau}$\vspace{-1.2em}}\right)
    \expandsSG{\uDelta}{\uGamma}{\uPsi}{\uPhi}{\inj{\ell}{\ue}}{\aein{\ell}{e}}{\asum{\labelset, \ell}{\mapschema{\tau}{i}{\labelset}; \ell \hookrightarrow \tau'}}
  }
\end{equation}
\begin{equation}\label{rule:expandsU-case}
  \inferrule{
    \expandsSG{\uDelta}{\uGamma}{\uPsi}{\uPhi}{\ue}{e}{\asum{\labelset}{\mapschema{\tau}{i}{\labelset}}}\\
    % \istypeU{\Delta}{\tau}\\
    \{\expandsSG{\uDelta}{\uGamma, \uGhyp{\ux_i}{x_i}{\tau_i}}{\uPsi}{\uPhi}{\ue_i}{e_i}{\tau}\}_{i \in \labelset}
  }{
    \expandsSG{\uDelta}{\uGamma}{\uPsi}{\uPhi}{\caseof{\ue}{\mapschemab{\ux}{\ue}{i}{\labelset}}}{\aecase{\labelset}{e}{\mapschemab{x}{e}{i}{\labelset}}}{\tau}
  }
\end{equation}
\begin{equation}\label{rule:expandsU-syntax}
\inferrule{
  \expandsTU{\uDelta}{\utau}{\tau}\\
  \hastypeU{\emptyset}{\emptyset}{\eparse}{\aparr{\tBody}{\tParseResultExp}}\\\\
  \evalU{\eparse}{\eparse'}\\
  \expandsU{\uDelta}{\uGamma}{\uPsi, \uShyp{\tsmv}{a}{\tau}{\eparse'}}{\ue}{e}{\tau'}
}{
  \expandsSG{\uDelta}{\uGamma}{\uPsi}{\uPhi}{\uesyntax{\tsmv}{\utau}{\eparse}{\ue}}{e}{\tau'}
}
\end{equation}
\begin{equation}\label{rule:expandsU-tsmap}
\inferrule{
  \uPsi = \uPsi', \uShyp{\tsmv}{a}{\tau}{\eparse}\\\\
  \encodeBody{b}{\ebody}\\
  \evalU{\ap{\eparse}{\ebody}}{\lbltxt{SuccessE}\cdot\ecand}\\
  \decodeCondE{\ecand}{\ce}\\\\
    \segOK{\segof{\ce}}{b}\\
  \cvalidE{\emptyset}{\emptyset}{\esceneSG{\uDelta}{\uGamma}{\uPsi}{\uPhi}{b}}{\ce}{e}{\tau}
}{
  \expandsSG{\uDelta}{\uGamma}{\uPsi}{\uPhi}{\utsmap{\tsmv}{b}}{e}{\tau}
}
\end{equation}
\begin{grayparbox}
\begin{equation}\label{rule:expandsU-match}
\inferrule{
  \expandsSG{\uDelta}{\uGamma}{\uPsi}{\uPhi}{\ue}{e}{\tau}\\
  % \istypeU{\Delta}{\tau'}\\
  \{\ruleExpands{\uDelta}{\uGamma}{\uPsi}{\uPhi}{\urv_i}{r_i}{\tau}{\tau'}\}_{1 \leq i \leq n}\\
}{
  \expandsSG
    {\uDelta}{\uGamma}{\uPsi}{\uPhi}
    {\matchwith
      {\ue}
      {\seqschemaX{\urv}}
    }{\aematchwith
      {n}
      {e}
      {\seqschemaX{r}}
    }{\tau'}
}
\end{equation}
\begin{equation}\label{rule:expandsU-defuptsm}
\graybox{\inferrule{
  \expandsTU{\uDelta}{\utau}{\tau}\\
  \hastypeU{\emptyset}{\emptyset}{\eparse}{\aparr{\tBody}{\tParseResultPat}}\\\\
  \evalU{\eparse}{\eparse'}\\
  \expandsUP{\uDelta}{\uGamma}{\uPsi}{\uPhi, \uPhyp{\tsmv}{a}{\tau}{\eparse'}}{\ue}{e}{\tau'}
}{
  \expandsUPX{\usyntaxup{\tsmv}{\utau}{\eparse}{\ue}}{e}{\tau'}
}}
\end{equation}
\end{grayparbox}
\end{subequations}

% \begin{subequations}\label{rules:expandsU}
% Rules (\ref*{rule:expandsU-var}) through (\ref*{rule:expandsU-case}) handle unexpanded expressions of common form. The first five of these rules are defined below:
% %Each of these rules is based on the corresponding typing rule, i.e. Rules (\ref{rule:hastypeU-var}) through (\ref{rule:hastypeU-case}), respectively. For example, the following typed expansion rules are based on the typing rules (\ref{rule:hastypeU-var}), (\ref{rule:hastypeU-lam}) and (\ref{rule:hastypeU-ap}), respectively:% for unexpanded expressions of variable, function and application form, respectively: 
% \begin{equation}\label{rule:expandsU-var}
%   \inferrule{ }{\expandsU{\uDelta}{\uGamma, \uGhyp{\ux}{x}{\tau}}{\uPsi}{\ux}{x}{\tau}}
% \end{equation}
% \begin{equation}\label{rule:expandsU-lam}
%   \inferrule{
%     \expandsTU{\uDelta}{\utau}{\tau}\\
%     \expandsU{\uDelta}{\uGamma, \uGhyp{\ux}{x}{\tau}}{\uPsi}{\ue}{e}{\tau'}
%   }{\expandsUX{\aulam{\utau}{\ux}{\ue}}{\aelam{\tau}{x}{e}}{\aparr{\tau}{\tau'}}}
% \end{equation}
% \begin{equation}\label{rule:expandsU-ap}
%   \inferrule{
%     \expandsUX{\ue_1}{e_1}{\aparr{\tau}{\tau'}}\\
%     \expandsUX{\ue_2}{e_2}{\tau}
%   }{
%     \expandsUX{\auap{\ue_1}{\ue_2}}{\aeap{e_1}{e_2}}{\tau'}
%   }
% \end{equation}
% \begin{equation}\label{rule:expandsU-tlam}
%   \inferrule{
%     \expandsU{\uDelta, \uDhyp{\ut}{t}}{\uGamma}{\uPsi}{\ue}{e}{\tau}
%   }{
%     \expandsUX{\autlam{\ut}{\ue}}{\aetlam{t}{e}}{\aall{t}{\tau}}
%   }
% \end{equation}
% \begin{equation}\label{rule:expandsU-tap}
%   \inferrule{
%     \expandsUX{\ue}{e}{\aall{t}{\tau}}\\
%     \expandsTU{\uDelta}{\utau'}{\tau'}
%   }{
%     \expandsUX{\autap{\ue}{\utau'}}{\aetap{e}{\tau'}}{[\tau'/t]\tau}
%   }
% \end{equation}
% Observe that, in each of these rules, the unexpanded and expanded expression forms in the conclusion correspond, and the premises correspond to those of the typing rule for the expanded expression form, i.e. Rules (\ref{rule:hastypeU-var}) through (\ref{rule:hastypeU-tap}), respectively. In particular, each type expansion premise in each rule above corresponds to a  type formation premise in the corresponding typing rule, and each typed expression expansion premise in each rule above corresponds to a typing premise in the corresponding typing rule. The type assigned in the conclusion of each rule above is identical to the type assigned in the conclusion of the corresponding typing rule. The ueTSM context, $\uPsi$, passes opaquely through these rules (we will define ueTSM contexts below). Rules (\ref{rules:expandsTU}) were similarly generated by mechanically transforming Rules (\ref{rules:istypeU}).

% We can express this scheme more precisely with the following rule transformation. For each rule in Rules (\ref{rules:istypeU}) and Rules (\ref{rules:hastypeU}),
% \begin{mathpar}
% \refstepcounter{equation}
% % \label{rule:expandsU-tlam}
% % \refstepcounter{equation}
% % \label{rule:expandsU-tap}
% % \refstepcounter{equation}
% \label{rule:expandsU-fold}
% \refstepcounter{equation}
% \label{rule:expandsU-unfold}
% \refstepcounter{equation}
% \label{rule:expandsU-tpl}
% \refstepcounter{equation}
% \label{rule:expandsU-pr}
% \refstepcounter{equation}
% \label{rule:expandsU-in}
% \refstepcounter{equation}
% \label{rule:expandsU-case}
% \inferrule{J_1\\ \cdots \\ J_k}{J}
% \end{mathpar}
% the corresponding typed expansion rule is 
% \begin{mathpar}
% \inferrule{
%   \Uof{J_1} \\
%   \cdots\\
%   \Uof{J_k}
% }{
%   \Uof{J}
% }
% \end{mathpar}
% where
% \[\begin{split}
% \Uof{\istypeU{\Delta}{\tau}} & = \expandsTU{\Uof{\Delta}}{\Uof{\tau}}{\tau} \\
% \Uof{\hastypeU{\Gamma}{\Delta}{e}{\tau}} & = \expandsU{\Uof{\Gamma}}{\Uof{\Delta}}{\uPsi}{\Uof{e}}{e}{\tau}\\
% \Uof{\{J_i\}_{i \in \labelset}} & = \{\Uof{J_i}\}_{i \in \labelset}
% \end{split}\]
% and where:
% \begin{itemize}
% \item $\Uof{\tau}$ is defined as follows:
%   \begin{itemize}
%   \item When $\tau$ is of definite form, $\Uof{\tau}$ is defined as in Sec. \ref{sec:syntax-U}.
%   \item When $\tau$ is of indefinite form, $\Uof{\tau}$ is a uniquely corresponding metavariable of sort $\mathsf{UTyp}$ also of indefinite form. For example, in Rule (\ref{rule:istypeU-parr}), $\tau_1$ and $\tau_2$ are of indefinite form, i.e. they match arbitrary types. The rule transformation simply ``hats'' them, i.e. $\Uof{\tau_1}=\utau_1$ and $\Uof{\tau_2}=\utau_2$.
%   \end{itemize}
% \item $\Uof{e}$ is defined as follows
% \begin{itemize}
% \item When $e$ is of definite form, $\Uof{e}$ is defined as in Sec. \ref{sec:syntax-U}. 
% \item When $e$ is of indefinite form, $\Uof{e}$ is a uniquely corresponding metavariable of sort $\mathsf{UExp}$ also of indefinite form. For example, $\Uof{e_1}=\ue_1$ and $\Uof{e_2}=\ue_2$.
% \end{itemize}
% \item $\Uof{\Delta}$ is defined as follows:
%   \begin{itemize} 
%   \item When $\Delta$ is of definite form, $\Uof{\Delta}$ is defined as above.
%   \item When $\Delta$ is of indefinite form, $\Uof{\Delta}$ is a uniquely corresponding metavariable ranging over unexpanded type formation contexts. For example, $\Uof{\Delta} = \uDelta$.
%   \end{itemize}
% \item $\Uof{\Gamma}$ is defined as follows:
%   \begin{itemize}
%   \item When $\Gamma$ is of definite form, $\Uof{\Gamma}$ produces the corresponding unexpanded typing context as follows:
% \begin{align*}
% \Uof{\emptyset} & = \uGG{\emptyset}{\emptyset}\\
% \Uof{\Gamma, \Ghyp{x}{\tau}} & = \Uof{\Gamma}, \uGhyp{\sigilof{x}}{x}{\tau}
% \end{align*}
%   \item When $\Gamma$ is of indefinite form, $\Uof{\Gamma}$ is a uniquely corresponding metavariable ranging over unexpanded typing contexts. For example, $\Uof{\Gamma} = \uGamma$.
% \end{itemize}
% \end{itemize}

% It is instructive to use this rule transformation to generate Rules (\ref{rules:expandsTU}) and Rules (\ref{rule:expandsU-var}) through (\ref{rule:expandsU-tap}) above. We omit the remaining rules, i.e. Rules (\ref*{rule:expandsU-fold}) through (\ref*{rule:expandsU-case}). By instead defining these rules solely by the rule transformation just described, we avoid having to write down a number of rules that are of limited marginal interest. Moreover, this demonstrates the general technique for generating typed expansion rules for unexpanded types and expressions of common form, so our exposition is somewhat ``robust'' to changes to the inner core. 
\vspace{-5px}\begin{grayparbox}
\vspace{8px}
\noindent\fcolorbox{black}{lightgray}{\strut$\ruleExpands{\uDelta}{\uGamma}{\uPsi}{\uPhi}{\urv}{r}{\tau}{\tau'}$}~~$\urv$ has expansion $r$ taking values of type $\tau$ to values of type $\tau'$
\begin{equation}\label{rule:ruleExpands}
\graybox{\inferrule{
  \patExpands{\uAS{\uG'}{\pctx'}}{\uPhi}{\upv}{p}{\tau}\\
  \expandsUP{\uDelta}{\uGG{\uGcons{\uG}{\uG'}}{\Gcons{\Gamma}{\pctx'}}}{\uPsi}{\uPhi}{\ue}{e}{\tau'} 
}{
  \ruleExpands{\uDelta}{\uGG{\uG}{\Gamma}}{\uPsi}{\uPhi}{\aumatchrule{\upv}{\ue}}{\aematchrule{p}{e}}{\tau}{\tau'}
}}
\end{equation}

Rule (\ref{rule:ruleExpands}) is defined mutually with Rules (\ref{rules:expandsU}).

\subsubsection{Typed Pattern Expansion}
% \vspace{8px}
\noindent\fcolorbox{black}{lightgray}{\strut$\patExpands{\uGamma}{\uPhi}{\upv}{p}{\tau}$}~~$\upv$ has expansion $p$ matching against $\tau$ generating hypotheses $\uGamma$
\begin{subequations}\label{rules:patExpands}
\begin{equation}\label{rule:patExpands-var}
\graybox{\inferrule{ }{
  \patExpands{\uGG{\vExpands{\ux}{x}}{\Ghyp{x}{\tau}}}{\uPhi}{\ux}{x}{\tau}
}}
\end{equation}
\begin{equation}\label{rule:patExpands-wild}
\graybox{\inferrule{ }{
  \patExpands{\uGG{\emptyset}{\emptyset}}{\uPhi}{\wildp}{\aewildp}{\tau}
}}
\end{equation}
\begin{equation}\label{rule:patExpands-fold}
\graybox{\inferrule{ 
  \patExpands{\upctx}{\uPhi}{\upv}{p}{[\arec{t}{\tau}/t]\tau}
}{
  \patExpands{\upctx}{\uPhi}{\foldp{\upv}}{\aefoldp{p}}{\arec{t}{\tau}}
}}
\end{equation}
\begin{equation}\label{rule:patExpands-tpl}
\graybox{
  \inferrule{
    \tau = \aprod{\labelset}{\mapschema{\tau}{i}{\labelset}}\\\\
    \{\patExpands{{\upctx_i}}{\uPhi}{\upv_i}{p_i}{\tau_i}\}_{i \in \labelset}
  }{
    % \left(\shortstack{
    %   $\Delta \vdash_{\uPhi} \tplp{\mapschema{\upv}{i}{\labelset}}$\\
    %   $\leadsto$\\
    %   $\aetplp{\labelset}{\mapschema{p}{i}{\labelset}} : \aprod{\labelset}{\mapschema{\tau}{i}{\labelset}}$\vspace{-1.2em}}\right)
    \patExpands{\GIconsi{i \in \labelset}{\upctx_i}}{\uPhi}{\tplp{\mapschema{\upv}{i}{\labelset}}}{\aetplp{\labelset}{\mapschema{p}{i}{\labelset}}}{\tau}
  }
}
% \graybox{\inferrule{
%   \{\patExpands{{\upctx_i}}{\uPhi}{\upv_i}{p_i}{\tau_i}\}_{i \in \labelset}\\
% }{
%   % \patExpands{\Gconsi{i \in \labelset}{\pctx_i}}{\Phi}{
%   %   \autplp{\labelset}{\mapschema{\upv}{i}{\labelset}}
%   % }{
%   %   \aetplp{\labelset}{\mapschema{p}{i}{\labelset}}
%   % }{
%   %   \aprod{\labelset}{\mapschema{\tau}{i}{\labelset}}
%   % } %{\autplp{\labelset}{\mapschema{\upv}{i}{\labelset}}}{\aetplp{\labelset}{\mapschema}{p}{i}{\labelset}}{...}
%   \left(\shortstack{$\Delta \vdash_{\uPhi} \autplp{\labelset}{\mapschema{\upv}{i}{\labelset}}$\\$\leadsto$\\$\aetplp{\labelset}{\mapschema{p}{i}{\labelset}} : \aprod{\labelset}{\mapschema{\tau}{i}{\labelset}} \dashV \Gconsi{i \in \labelset}{\upctx_i}$\vspace{-1.2em}}\right)
% }}
\end{equation}
\begin{equation}\label{rule:patExpands-in}
\graybox{\inferrule{
  \patExpands{\upctx}{\uPhi}{\upv}{p}{\tau}
}{
  \patExpands{\upctx}{\uPhi}{\injp{\ell}{\upv}}{\aeinjp{\ell}{p}}{\asum{\labelset, \ell}{\mapschema{\tau}{i}{\labelset}; \mapitem{\ell}{\tau}}}
}}
\end{equation}
\begin{equation}\label{rule:patExpands-apuptsm}
\graybox{\inferrule{
  \uPhi = \uPhi', \uPhyp{\tsmv}{a}{\tau}{\eparse}\\\\
  \encodeBody{b}{\ebody}\\
  \evalU{\ap{\eparse}{\ebody}}{{\lbltxt{SuccessP}}\cdot{\ecand}}\\
  \decodeCEPat{\ecand}{\cpv}\\\\
    \segOK{\segof{\cpv}}{b}\\
  \cvalidP{\upctx}{\pscene{\uDelta}{\uPhi}{b}}{\cpv}{p}{\tau}
}{
  \patExpands{\upctx}{\uPhi}{\utsmap{\tsmv}{b}}{p}{\tau}
}}
\end{equation}

\end{subequations}

In Rule (\ref{rule:patExpands-tpl}), $\upctx_i$ is shorthand for $\uGG{\uG_i}{\pctx_i}$ and $\GIconsi{i \in \labelset}{\upctx_i}$ is shorthand for \[\uGG{\GIconsi{i \in \labelset}{\uG_i}}{\Gconsi{i \in \labelset}{\pctx_i}}\] 
\end{grayparbox}


% \end{subequations}
% \clearpage
\section{Proto-Expansion Validation}\label{appendix:proto-expansions-SES}
\subsection{Syntax of Proto-Expansions}
$\arraycolsep=2pt\begin{array}{lllllll}
\textbf{Sort} & & & \textbf{Operational Form} & \textbf{Stylized Form} & \textbf{Description}\\
\mathsf{PrTyp} & \ctau & ::= & t & t & \text{variable}\\
&&& \aceparr{\ctau}{\ctau} & \parr{\ctau}{\ctau} & \text{partial function}\\
&&& \aceall{t}{\ctau} & \forallt{t}{\ctau} & \text{polymorphic}\\
&&& \acerec{t}{\ctau} & \rect{t}{\ctau} & \text{recursive}\\
&&& \aceprod{\labelset}{\mapschema{\ctau}{i}{\labelset}} & \prodt{\mapschema{\ctau}{i}{\labelset}} & \text{labeled product}\\
&&& \acesum{\labelset}{\mapschema{\ctau}{i}{\labelset}} & \sumt{\mapschema{\ctau}{i}{\labelset}} & \text{labeled sum}\\
% \LCC &&& \lightgray & \lightgray & \lightgray\\
&&& \acesplicedt{m}{n} & \splicedt{m}{n} & \text{spliced type ref.}\\%\ECC
\mathsf{PrExp} & \ce & ::= & x & x & \text{variable}\\
&&& \aceasc{\ctau}{\ce} & \asc{\ce}{\ctau} & \text{ascription}\\
&&& \aceletsyn{x}{\ce}{\ce} & \letsyn{x}{\ce}{\ce} & \text{value binding}\\
&&& \acelam{\ctau}{x}{\ce} & \lam{x}{\ctau}{\ce} & \text{abstraction}\\
&&& \aceap{\ce}{\ce} & \ap{\ce}{\ce} & \text{application}\\
&&& \acetlam{t}{\ce} & \Lam{t}{\ce} & \text{type abstraction}\\
&&& \acetap{\ce}{\ctau} & \App{\ce}{\ctau} & \text{type application}\\
&&& \acefold{\ce} & \fold{\ce} & \text{fold}\\
&&& \aceunfold{\ce} & \unfold{\ce} & \text{unfold}\\
&&& \acetpl{\labelset}{\mapschema{\ce}{i}{\labelset}} & \tpl{\mapschema{\ce}{i}{\labelset}} & \text{labeled tuple}\\
&&& \acepr{\ell}{\ce} & \prj{\ce}{\ell} & \text{projection}\\
&&& \acein{\ell}{\ce} & \inj{\ell}{\ce} & \text{injection}\\
&&& \acecase{\labelset}{\ce}{\mapschemab{x}{\ce}{i}{\labelset}} & \caseof{\ce}{\mapschemab{x}{\ce}{i}{\labelset}} & \text{case analysis}\\
&&& \acesplicede{m}{n}{\ctau} & \splicede{m}{n}{\ctau} & \text{spliced expr. ref.}\\
\LCC \lightgray &\lightgray & \lightgray& \lightgray & \lightgray & \lightgray\\
&&& \acematchwith{n}{\ce}{\seqschemaX{\crv}} & \matchwith{\ce}{\seqschemaX{\crv}} & \text{match}\\
\mathsf{PrRule} & \crv & ::= & \acematchrule{p}{\ce} & \matchrule{p}{\ce} & \text{rule}\\
\mathsf{PrPat} & \cpv & ::= & \acewildp & \wildp & \text{wildcard pattern}\\
&&& \acefoldp{p} & \foldp{p} & \text{fold pattern}\\
&&& \acetplp{\labelset}{\mapschema{\cpv}{i}{\labelset}} & \tplp{\mapschema{\cpv}{i}{\labelset}} & \text{labeled tuple pattern}\\
&&& \aceinjp{\ell}{\cpv} & \injp{\ell}{\cpv} & \text{injection pattern}\\
% \LCC &&& \color{Yellow} & \color{Yellow} & \color{Yellow}\\
&&& \acesplicedp{m}{n}{\ctau} & \splicedp{m}{n}{\ctau} & \text{spliced pattern ref.}\ECC
\end{array}$

\subsubsection{Common Proto-Expansion Terms} Each expanded term\graytxtbox{, except variable patterns,} maps onto a proto-expansion term. We refer to these as the \emph{common proto-expansion terms}. In particular:
\begin{itemize}
  \item Each type, $\tau$, maps onto a proto-type, $\Cof{\tau}$, as follows:
  \[\arraycolsep=1pt\begin{array}{rl}
  \Cof{t} & = t\\
  \Cof{\aparr{\tau_1}{\tau_2}} & = \aceparr{\Cof{\tau_1}}{\Cof{\tau_2}}\\
  \Cof{\aall{t}{\tau}} & = \aceall{t}{\Cof{\tau}}\\
  \Cof{\arec{t}{\tau}} & = \acerec{t}{\Cof{\tau}}\\
  \Cof{\aprod{\labelset}{\mapschema{\tau}{i}{\labelset}}} & = \aceprod{\labelset}{\mapschemax{\Cofv}{\tau}{i}{\labelset}}\\
  \Cof{\asum{\labelset}{\mapschema{\tau}{i}{\labelset}}} & = \acesum{\labelset}{\mapschemax{\Cofv}{\tau}{i}{\labelset}}
  \end{array}\]
  \item Each expanded expression, $e$, maps onto a proto-expression, $\Cof{e}$, as follows:
  \[\arraycolsep=1pt\begin{array}{rl}
  \Cof{x} & = x\\
  \Cof{\aelam{\tau}{x}{e}} & = \acelam{\Cof{\tau}}{x}{\Cof{e}}\\
  \Cof{\aeap{e_1}{e_2}} & = \aceap{\Cof{e_1}}{\Cof{e_2}}\\
  \Cof{\aetlam{t}{e}} & = \acetlam{t}{\Cof{e}}\\
  \Cof{\aetap{e}{\tau}} & = \acetap{\Cof{e}}{\Cof{\tau}}\\
  \Cof{\aefold{e}} & = \acefold{\Cof e}\\
  \Cof{\aeunfold{e}} & = \aceunfold{\Cof{e}}\\
  \Cof{\aetpl{\labelset}{\mapschema{e}{i}{\labelset}}} & = \acetpl{\labelset}{\mapschemax{\Cofv}{e}{i}{\labelset}}\\
  \Cof{\aein{\ell}{e}} &= \acein{\ell}{\Cof{e}}\\
  \LCC \lightgray & \lightgray \\
  \Cof{\aematchwith{n}{e}{\seqschemaX{r}}} & = \acematchwith{n}{\Cof{e}}{\seqschemaXx{\Cofv}{r}} \ECC
  \end{array}\]
  \end{itemize}
  \begin{grayparbox}
  \begin{itemize}
  \item Each expanded rule, $r$, maps onto the proto-rule, $\Cof{r}$, as follows:
  \begin{align*}
  \Cof{\aematchrule{p}{e}} & = \acematchrule{p}{\Cof{e}}
  \end{align*}
  Notice that proto-rules bind expanded patterns, not proto-patterns. This is because proto-rules appear in proto-expressions, which are generated by seTSMs. It would not be sensible for an seTSM to splice a pattern out of a literal body.
  \item Each expanded pattern, $p$, except for the variable patterns, maps onto a proto-pattern, $\Cof{p}$, as follows:
  \begin{align*}
  \Cof{\aewildp} & = \acewildp\\
  \Cof{\aefoldp{p}} & = \acefoldp{\Cof{p}}\\
  \Cof{\aetplp{\labelset}{\mapschema{p}{i}{\labelset}}} & = \acetplp{\labelset}{\mapschemax{\Cofv}{p}{i}{\labelset}}\\
  \Cof{\aeinjp{\ell}{p}} & = \aceinjp{\ell}{\Cof{p}}
  \end{align*}
\end{itemize}
\end{grayparbox}

\subsubsection{Proto-Expression Encoding and Decoding}
The type abbreviated $\tCEExp$ classifies encodings of \emph{proto-expressions}. The mapping from proto-expressions to values of type $\tCEExp$ is defined by the \emph{proto-expression encoding judgement}, $\encodeCondE{\ce}{e}$. An inverse mapping is defined by the \emph{proto-expression decoding judgement}, $\decodeCondE{e}{\ce}$.

\[\begin{array}{ll}
\textbf{Judgement Form} & \textbf{Description}\\
\encodeCondE{\ce}{e} & \text{$\ce$ has encoding $e$}\\
\decodeCondE{e}{\ce} & \text{$e$ has decoding $\ce$}
\end{array}\]

Rather than picking a particular definition of $\tCEExp$ and defining the judgements above inductively against it, we only state the following condition, which establishes an isomorphism between values of type $\tCEExp$ and proto-expressions.

\begin{condition}[Proto-Expression Isomorphism]\label{condition:proto-expression-isomorphism} ~
\begin{enumerate}
\item For every $\ce$, we have $\encodeCondE{\ce}{\ecand}$ for some $\ecand$ such that $\hastypeUC{\ecand}{\tCEExp}$ and $\isvalU{\ecand}$.
\item If $\hastypeUC{\ecand}{\tCEExp}$ and $\isvalU{\ecand}$ then $\decodeCondE{\ecand}{\ce}$ for some $\ce$.
\item If $\encodeCondE{\ce}{\ecand}$ then $\decodeCondE{\ecand}{\ce}$.
\item If $\hastypeUC{\ecand}{\tCEExp}$ and $\isvalU{\ecand}$ and $\decodeCondE{\ecand}{\ce}$ then $\encodeCondE{\ce}{\ecand}$.
\item If $\encodeCondE{\ce}{\ecand}$ and $\encodeCondE{\ce}{\ecand'}$ then $\ecand=\ecand'$.
\item If $\hastypeUC{\ecand}{\tCEExp}$ and $\isvalU{\ecand}$ and $\decodeCondE{\ecand}{\ce}$ and $\decodeCondE{\ecand}{\ce'}$ then $\ce=\ce'$.
\end{enumerate}
\end{condition}\vspace{10px}

\begin{grayparbox}\vspace{-16px}
\subsubsection{Proto-Pattern Encoding and Decoding}
The type abbreviated $\tCEPat$ classifies encodings of \emph{proto-patterns}. The mapping from proto-patterns to values of type $\tCEPat$ is defined by the \emph{proto-pattern encoding judgement}, $\encodeCEPat{\cpv}{p}$. An inverse mapping is defined by the \emph{proto-expression decoding judgement}, $\decodeCEPat{p}{\cpv}$.

\[\begin{array}{ll}
\textbf{Judgement Form} & \textbf{Description}\\
\encodeCEPat{\cpv}{p} & \text{$\cpv$ has encoding $p$}\\
\decodeCEPat{p}{\cpv} & \text{$p$ has decoding $\cpv$}
\end{array}\]

Again, rather than picking a particular definition of $\tCEPat$ and defining the judgements above inductively against it, we only state the following condition, which establishes an isomorphism between values of type $\tCEPat$ and proto-patterns.

\begin{condition}[Proto-Pattern Isomorphism]\label{condition:proto-pattern-isomorphism} ~
\begin{enumerate}
\item For every $\cpv$, we have $\encodeCEPat{\cpv}{\ecand}$ for some $\ecand$ such that $\hastypeUC{\ecand}{\tCEPat}$ and $\isvalU{\ecand}$.
\item If $\hastypeUC{\ecand}{\tCEPat}$ and $\isvalU{\ecand}$ then $\decodeCEPat{\ecand}{\cpv}$ for some $\cpv$.
\item If $\encodeCEPat{\cpv}{\ecand}$ then $\decodeCEPat{\ecand}{\cpv}$.
\item If $\hastypeUC{\ecand}{\tCEPat}$ and $\isvalU{\ecand}$ and $\decodeCEPat{\ecand}{\cpv}$ then $\encodeCEPat{\cpv}{\ecand}$.
\item If $\encodeCEPat{\cpv}{\ecand}$ and $\encodeCEPat{\cpv}{\ecand'}$ then $\ecand=\ecand'$.
\item If $\hastypeUC{\ecand}{\tCEPat}$ and $\isvalU{\ecand}$ and $\decodeCEPat{\ecand}{\cpv}$ and $\decodeCEPat{\ecand}{\cpv'}$ then $\cpv=\cpv'$.
\end{enumerate}
\end{condition}
\end{grayparbox}

\subsubsection{Splice Summaries}
The \emph{splice summary} of a proto-expression, $\summaryOf{\ce}$, \graytxtbox{or proto-pattern, $\summaryOf{\cpv}$,} is the finite set of references to spliced types, expressions \graytxtbox{and patterns} that it mentions.

\subsubsection{Segmentations}
A \emph{segment set}, $\psi$, is a finite set of pairs of natural numbers indicating the locations of spliced terms. The \emph{segmentation} of a proto-expression, $\segof{\ce}$, or proto-pattern, $\segof{\cpv}$, is the segment set implied by its splice summary.


% Segments consist of two natural numbers and a sort, i.e. segments are of the form $\segExp{m}{n}$ or $\segTyp{m}{n}$\graytxtbox{ or $\segPat{m}{n}$}.

% The metafunction $\segof{\ce}$ determines the segmentation of $\ce$ by generating one segment for each reference to a spliced expression or type, respectively. More specifically:
% \begin{itemize}
% \item We define $\segof{\ctau}$ as follows:
% \[\arraycolsep=1pt\begin{array}{rl}
%   \segof{t} & = \emptyset\\
%   \segof{\aceparr{\ctau_1}{\ctau_2}} & = \segof{\ctau_1} \cup \segof{\ctau_2}\\
%   \segof{\aceall{t}{\ctau}} &= \segof{\ctau}\\
%   \segof{\acerec{t}{\ctau}} & = \segof{\ctau}\\
%   \segof{\aceprod{\labelset}{\mapschema{\ctau}{i}{\labelset}}} & = \cup_{i \in \labelset} \segof{\ctau_i}\\
%   \segof{\acesum{\labelset}{\mapschema{\ctau}{i}{\labelset}}} & = \cup_{i \in \labelset} \segof{\ctau_i}\\
%   \segof{\acesplicedt{m}{n}} & = \{ \segTyp{m}{n} \}
%   \end{array}\]
% \item We define $\segof{\ce}$ as follows:
% \[\arraycolsep=1pt\begin{array}{rl} 

% \segof{x} & = \emptyset\\
% \segof{\acelam{\ctau}{x}{\ce}} & = \segof{\ctau} \cup \segof{\ce} \\
% \segof{\acetlam{t}{\ce}} & = \segof{\ce}\\
% \segof{\acetap{\ce}{\ctau}} & = \segof{\ctau} \cup \segof{\ce}\\
% \segof{\acefold{\ce}} & = \segof{\ce}\\
% \segof{\aceunfold{\ce}} & = \segof{\ce}\\
% \segof{\acetpl{\labelset}{\mapschema{\ce}{i}{\labelset}}} & = \cup_{i \in \labelset} \segof{\ce_i}\\
% \segof{\acepr{\ell}{\ce}} & = \segof{\ce}\\
% \segof{\acein{\ell}{\ce}} & = \segof{\ce}\\
% \segof{\acecase{\labelset}{\ce}{\mapschemab{x}{\ce}{i}{\labelset}}} & = \segof{\ce} \cup_{i \in \labelset} \segof{\ce_i}\\
% \segof{\acesplicede{m}{n}{\ctau}} & = \{ \segExp{m}{n} \} \cup \segof{\ctau}\\
% \LCC \lightgray & \lightgray\\
% \segof{\acematchwith{n}{\ce}{\seqschemaX{\crv}}} & = \segof{\ce} \cup_{1 \leq i \leq n} \segof{\crv_i}\ECC 
% \end{array}\]
% \end{itemize}
% \begin{grayparbox}
% \begin{itemize}
% \item We define $\segof{\crv}$ as follows:
% \[\arraycolsep=1pt\begin{array}{rl} 

% \segof{\acematchrule{p}{\ce}} & = \segof{\ce}
% \end{array}\]
% \end{itemize}

% The metafunction $\segof{\cpv}$ determines the segmentation of $\cpv$ by generating one segment for each reference to a spliced type or pattern:
% \[
% \arraycolsep=1pt\begin{array}{rl}

% \segof{\acewildp} & = \emptyset\\
% \segof{\acefoldp{\cpv}} & = \segof{\cpv}\\
% \segof{\acetplp{\labelset}{\mapschema{\cpv}{i}{\labelset}}} & = \cup_{i \in \labelset} \segof{\cpv_i}\\
% \segof{\aceinjp{\ell}{\cpv}} & = \segof{\cpv}\\
% \segof{\acesplicedp{m}{n}{\ctau}} & = \{ \segPat{m}{n} \} \cup \segof{\ctau}
% \end{array}
% \]

% \end{grayparbox}

The predicate $\segOK{\psi}{b}$ checks that each segment in $\psi$, has non-negative length and is within bounds of $b$, and that the segments in $\psi$ do not overlap.


\subsection{Proto-Type Validation}\label{appendix:proto-type-validation-SES}
%Each of these rules is defined based on the corresponding type formation rule, i.e. Rules (\ref{rule:istypeU-var}) through (\ref{rule:istypeU-sum}), respectively. For example, the following candidate expansion type validation rules are based on type formation rules (\ref{rule:istypeU-var}), (\ref{rule:istypeU-parr}) and (\ref{rule:istypeU-all}), respectively: 
\emph{Type splicing scenes}, $\tscenev$, are of the form $\tsceneUP{\uDelta}{b}$.

\vspace{10px}\noindent\fbox{\strut$\cvalidT{\Delta}{\tscenev}{\ctau}{\tau}$}~~$\ctau$ has well-formed expansion $\tau$
\begin{subequations}\label{rules:cvalidT-U}
\begin{equation}\label{rule:cvalidT-U-tvar}
\inferrule{ }{
  \cvalidT{\Delta, \Dhyp{t}}{\tscenev}{t}{t}
}
\end{equation}
\begin{equation}\label{rule:cvalidT-U-parr}
  \inferrule{
    \cvalidT{\Delta}{\tscenev}{\ctau_1}{\tau_1}\\
    \cvalidT{\Delta}{\tscenev}{\ctau_2}{\tau_2}
  }{
    \cvalidT{\Delta}{\tscenev}{\aceparr{\ctau_1}{\ctau_2}}{\aparr{\tau_1}{\tau_2}}
  }
\end{equation}
\begin{equation}\label{rule:cvalidT-U-all}
  \inferrule {
    \cvalidT{\Delta, \Dhyp{t}}{\tscenev}{\ctau}{\tau}
  }{
    \cvalidT{\Delta}{\tscenev}{\aceall{t}{\ctau}}{\aall{t}{\tau}}
  }
\end{equation}
\begin{equation}\label{rule:cvalidT-U-rec}
  \inferrule{
    \cvalidT{\Delta, \Dhyp{t}}{\tscenev}{\ctau}{\tau}
  }{
    \cvalidT{\Delta}{\tscenev}{\acerec{t}{\ctau}}{\arec{t}{\tau}}
  }
\end{equation}
\begin{equation}\label{rule:cvalidT-U-prod}
  \inferrule{
    \{\cvalidT{\Delta}{\tscenev}{\ctau_i}{\tau_i}\}_{i \in \labelset}
  }{
    \cvalidT{\Delta}{\tscenev}{\aceprod{\labelset}{\mapschema{\ctau}{i}{\labelset}}}{\aprod{\labelset}{\mapschema{\tau}{i}{\labelset}}}
  }
\end{equation}
\begin{equation}\label{rule:cvalidT-U-sum}
  \inferrule{
    \{\cvalidT{\Delta}{\tscenev}{\ctau_i}{\tau_i}\}_{i \in \labelset}
  }{
    \cvalidT{\Delta}{\tscenev}{\acesum{\labelset}{\mapschema{\ctau}{i}{\labelset}}}{\asum{\labelset}{\mapschema{\tau}{i}{\labelset}}}
  }
\end{equation}
\begin{equation}\label{rule:cvalidT-U-splicedt}
  \inferrule{
    \parseUTyp{\bsubseq{b}{m}{n}}{\utau}\\
    \expandsTU{\uDD{\uD}{\Delta_\text{app}}}{\utau}{\tau}\\
    \Delta \cap \Delta_\text{app} = \emptyset
  }{
    \cvalidT{\Delta}{\tsceneU{\uDD{\uD}{\Delta_\text{app}}}{b}}{\acesplicedt{m}{n}}{\tau}
  }
\end{equation}
\end{subequations}


%Rule (\ref*{rule:cvalidT-U-splicedt}) governs this form:
%\chapter{Dependent Labeled Product Kinds}
% \begin{landscape}
% \begin{equation}\label{rule:iskind-dlprod}
% \inferrule{
% 	\{\iskind{\Omega}{\Delta \cup \{u_{i, j} :: \kappa_j\}_{1 \leq j < i}}{\Gamma}{\kappa_i}\}_{1 \leq i \leq n}
% }{
% 	\iskindX{\akdprodstd}
% }
% \end{equation}

% \begin{equation}\label{rule:kequal-dlprod}
% \inferrule{
% 	\{\kequal{\Omega}{\Delta \cup \{u_{i, j} :: \kappa_j\}_{1 \leq j < i}}{\Gamma}{\kappa_i}{\kappa'_i}\}_{1 \leq i \leq n}
% }{
% 	\kequalX{\akdprodstd}{\akdprod{n}{\seqschemaX{\ell}}{\seqschemaijb{u}{\kappa'}{i}{1}{n}{j}{1}{i}}}
% }
% \end{equation}
% \begin{equation}\label{rule:ksub-dlprod}
% \inferrule{
% 	\{\ksub{\Omega}{\Delta \cup \{u_{i,j} :: \kappa_j\}_{1 \leq j < i}}{\Gamma}{\kappa_i}{\kappa'_i}\}_{1 \leq i \leq n}
% }{
% 	\ksubX{\akdprodstd}{\akdprod{n}{\seqschemaX{\ell}}{\seqschemaijb{u}{\kappa'}{i}{1}{n}{j}{1}{i}}}
% }
% \end{equation}
% \begin{equation}\label{rule:haskind-dtpl}
% \inferrule{
% 	\{\haskind{\Omega}{\Delta \cup \{u_{i, j} :: \aksing{c_j}\}_{1 \leq j < i}}{\Gamma}{c_i}{\kappa_i}\}_{1 \leq i \leq n}
% }{
% 	\haskindX{\adtplX}{\akdprodstd}
% }
% \end{equation}
% \begin{equation}\label{rule:haskind-prj}
% \inferrule{
% 	\haskindX{c}{
% 		\akdprod{
% 			n' + 1 + n''
% 		}{
% 			\seqschema{\ell'}{i}{1}{n'}, \ell, \seqschema{\ell''}{i}{1}{n''}
% 		}{
% 			\seqschemaijb{u'}{\kappa'}{i}{1}{n'}{j}{1}{i};
% 			\{u_{j}\}_{1 \leq j \leq n'}.\kappa;
% 			\seqschemaijb{u''}{\kappa''}{i}{1}{n''}{j}{1}{i}
% 		}
% 	}
% }{
% 	\haskindX{\adprj{\ell}{c}}{[\{\adprj{\ell'_j}{c}/u_{j}\}_{1 \leq j \leq n'}]\kappa}
% }
% \end{equation}
% \begin{equation}\label{rule:cequal-dtpl}
% \inferrule{
% 	c=\adtplX\\
% 	c'=\adtpl{n}{\seqschemaX{\ell}}{\seqschemaijb{u}{c'}{i}{1}{n}{j}{1}{i}}\\\\
% 	\{\cequal{\Omega}{\Delta \cup \{u_{i, j} :: \aksing{\kappa_j}\}_{1 \leq j < i}}{\Gamma}{c_i}{c'_i}{\kappa_i}\}_{1 \leq i \leq n}
% }{
% 	\cequalX{c}{c'}{\akdprodstd}
% }
% \end{equation}
% \begin{equation}\label{rule:cequal-prj-1}
% \inferrule{
%   \cequalX{c}{c'}{
% 		\akdprod{
% 			n' + 1 + n''
% 		}{
% 			\seqschema{\ell'}{i}{1}{n'}, \ell, \seqschema{\ell''}{i}{1}{n''}
% 		}{
% 			\seqschemaijb{u'}{\kappa'}{i}{1}{n'}{j}{1}{i};
% 			\{u_{j}\}_{1 \leq j \leq n'}.\kappa;
% 			\seqschemaijb{u''}{\kappa''}{i}{1}{n''}{j}{1}{i}
% 		}
% 	}	
% }{
% 	\cequalX{\adprj{\ell}{c}}{\adprj{\ell}{c'}}{\kappa}
% }
% \end{equation}
% \begin{equation}\label{rule:cequal-prj-2}
% \inferrule{
% 	c = \adtpl{
% 				n' + 1 + n''
% 			}{
% 				\seqschema{\ell'}{i}{1}{n'}, \ell, \seqschema{\ell''}{i}{1}{n''}
% 			}{
% 				\seqschemaijb{u'}{c'}{i}{1}{n'}{j}{1}{i};
% 				\{u_j\}_{1 \leq j \leq n}.c_\ell; 
% 				\seqschemaijb{u''}{c''}{i}{1}{n''}{j}{1}{i}
% 			}\\
% 	\haskindX{c}{
% 		\akdprod{
% 			n' + 1 + n''
% 		}{
% 			\seqschema{\ell'}{i}{1}{n'}, \ell, \seqschema{\ell''}{i}{1}{n''}
% 		}{
% 			\seqschemaijb{u'}{\kappa'}{i}{1}{n'}{j}{1}{i};
% 			\{u_{j}\}_{1 \leq j \leq n'}.\kappa;
% 			\seqschemaijb{u''}{\kappa''}{i}{1}{n''}{j}{1}{i}
% 		}
% 	}
% }{
% 	\cequalX{
% 		\adprj{\ell}{
% 			c
% 		}
% 	}{[\{\adprj{\ell'_j}{c}/u_j\}_{1 \leq j \leq n'}]c_\ell}{[\{\adprj{\ell'_j}{c}/u_j\}_{1 \leq j \leq n'}]\kappa}
% }
% \end{equation}

% \end{landscape}

\subsection{Proto-Expression Validation}

\emph{Expression splicing scenes}, $\escenev$, are of the form $\esceneSGB{\uDelta}{\uGamma}{\uPsi}{\uPhi}{b}$. We write $\tsfrom{\escenev}$ for the type splicing scene constructed by dropping unnecessary contexts from $\escenev$:
\[\tsfrom{\esceneSGB{\uDelta}{\uGamma}{\uPsi}{\uPhi}{b}} = \tsceneUP{\uDelta}{b}\]

\vspace{10px}\noindent\fbox{\strut$\cvalidE{\Delta}{\Gamma}{\escenev}{\ce}{e}{\tau}$}~~$\ce$ has expansion $e$ of type $\tau$
\begin{subequations}\label{rules:cvalidE-U}
\begin{equation}\label{rule:cvalidE-U-var}
\inferrule{ }{
  \cvalidE{\Delta}{\Gamma, \Ghyp{x}{\tau}}{\escenev}{x}{x}{\tau}
}
\end{equation}
\begin{equation}\label{rule:cvalidE-U-asc}
\inferrule{
  \cvalidT{\Delta}{\tsfrom{\escenev}}{\ctau}{\tau}\\
  \cvalidE{\Delta}{\Gamma}{\escenev}{\ce}{e}{\tau}
}{
  \cvalidE{\Delta}{\Gamma}{\escenev}{\aceasc{\ctau}{\ce}}{e}{\tau}
}
\end{equation}
\begin{equation}\label{rule:cvalidE-U-letsyn}
  \inferrule{
    \cvalidE{\Delta}{\Gamma}{\escenev}{\ce_1}{e_1}{\tau_1}\\
    \cvalidE{\Delta}{\Gamma, x : \tau_1}{\ce_2}{e_2}{\tau_2}
  }{
    \cvalidE{\Delta}{\Gamma}{\escenev}{\aceletsyn{x}{\ce_1}{\ce_2}}{
      \aeap{\aelam{\tau_1}{x}{e_2}}{e_1}
    }{\tau_2}
  }
\end{equation}
\begin{equation}\label{rule:cvalidE-U-lam}
\inferrule{
  \cvalidT{\Delta}{\tsfrom{\escenev}}{\ctau}{\tau}\\
  \cvalidE{\Delta}{\Gamma, \Ghyp{x}{\tau}}{\escenev}{\ce}{e}{\tau'}
}{
  \cvalidE{\Delta}{\Gamma}{\escenev}{\acelam{\ctau}{x}{\ce}}{\aelam{\tau}{x}{e}}{\aparr{\tau}{\tau'}}
}
\end{equation}
\begin{equation}\label{rule:cvalidE-U-ap}
  \inferrule{
    \cvalidE{\Delta}{\Gamma}{\escenev}{\ce_1}{e_1}{\aparr{\tau}{\tau'}}\\
    \cvalidE{\Delta}{\Gamma}{\escenev}{\ce_2}{e_2}{\tau}
  }{
    \cvalidE{\Delta}{\Gamma}{\escenev}{\aceap{\ce_1}{\ce_2}}{\aeap{e_1}{e_2}}{\tau'}
  }
\end{equation}
\begin{equation}\label{rule:cvalidE-U-tlam}
  \inferrule{
    \cvalidE{\Delta, \Dhyp{t}}{\Gamma}{\escenev}{\ce}{e}{\tau}
  }{
    \cvalidE{\Delta}{\Gamma}{\escenev}{\acetlam{t}{\ce}}{\aetlam{t}{e}}{\aall{t}{\tau}}
  }
\end{equation}
\begin{equation}\label{rule:cvalidE-U-tap}
  \inferrule{
    \cvalidE{\Delta}{\Gamma}{\escenev}{\ce}{e}{\aall{t}{\tau}}\\
    \cvalidT{\Delta}{\tsfrom{\escenev}}{\ctau'}{\tau'}
  }{
    \cvalidE{\Delta}{\Gamma}{\escenev}{\acetap{\ce}{\ctau'}}{\aetap{e}{\tau'}}{[\tau'/t]\tau}
  }
\end{equation}
\begin{equation}\label{rule:cvalidE-U-fold}
  \inferrule{\
    % \cvalidT{\Delta, \Dhyp{t}}{\tsfrom{\escenev}}{\ctau}{\tau}\\
    \cvalidE{\Delta}{\Gamma}{\escenev}{\ce}{e}{[\arec{t}{\tau}/t]\tau}
  }{
    \cvalidE{\Delta}{\Gamma}{\escenev}{\acefold{\ce}}{\aefold{e}}{\arec{t}{\tau}}
  }
\end{equation}
\begin{equation}\label{rule:cvalidE-U-unfold}
  \inferrule{
    \cvalidE{\Delta}{\Gamma}{\escenev}{\ce}{e}{\arec{t}{\tau}}
  }{
    \cvalidE{\Delta}{\Gamma}{\escenev}{\aceunfold{\ce}}{\aeunfold{e}}{[\arec{t}{\tau}/t]\tau}
  }
\end{equation}
\begin{equation}\label{rule:cvalidE-U-tpl}
  \inferrule{
    \tau = \aprod{\labelset}{\mapschema{\tau}{i}{\labelset}}\\\\
    \{\cvalidE{\Delta}{\Gamma}{\escenev}{\ce_i}{e_i}{\tau_i}\}_{i \in \labelset}
  }{
  % \left(\shortstack{$\Delta~\Gamma \vdash^{\escenev} \acetpl{\labelset}{\mapschema{\ce}{i}{\labelset}}$\\$\leadsto$\\$\aetpl{\labelset}{\mapschema{e}{i}{\labelset}} : \aprod{\labelset}{\mapschema{\tau}{i}{\labelset}}$\vspace{-1.2em}}\right)
    \cvalidE{\Delta}{\Gamma}{\escenev}{\acetpl{\labelset}{\mapschema{\ce}{i}{\labelset}}}{\aetpl{\labelset}{\mapschema{e}{i}{\labelset}}}{\tau}
  }
\end{equation}
\begin{equation}\label{rule:cvalidE-U-pr}
  \inferrule{
    \cvalidE{\Delta}{\Gamma}{\escenev}{\ce}{e}{\aprod{\labelset, \ell}{\mapschema{\tau}{i}{\labelset}; \ell \hookrightarrow \tau}}
  }{
    \cvalidE{\Delta}{\Gamma}{\escenev}{\acepr{\ell}{\ce}}{\aepr{\ell}{e}}{\tau}
  }
\end{equation}
\begin{equation}\label{rule:cvalidE-U-in}
  \inferrule{
    % \{\cvalidT{\Delta}{\tsfrom{\escenev}}{\ctau_i}{\tau_i}\}_{i \in \labelset}\\
    % \cvalidT{\Delta}{\tsfrom{\escenev}}{\ctau}{\tau}\\
    \cvalidE{\Delta}{\Gamma}{\escenev}{\ce}{e}{\tau'}
  }{
    % \left(\shortstack{
    %   $\Delta~\Gamma \vdash^{\escenev} \acein{\ell}{\ce}$\\
    %   $\leadsto$\\
    %   $\aein{\ell}{e} : \asum{\labelset, \ell}{\mapschema{\tau}{i}{\labelset}; \ell \hookrightarrow \tau}$\vspace{-1.2em}
    % }\right)
    \cvalidE{\Delta}{\Gamma}{\escenev}{\acein{\ell}{\ce}}{\aein{\ell}{e}}{\asum{\labelset, \ell}{\mapschema{\tau}{i}{\labelset}; \ell \hookrightarrow \tau'}}
  }
\end{equation}
\begin{equation}\label{rule:cvalidE-U-case}
  \inferrule{
    \cvalidE{\Delta}{\Gamma}{\escenev}{\ce}{e}{\asum{\labelset}{\mapschema{\tau}{i}{\labelset}}}\\
    % \cvalidT{\Delta}{\tsfrom{\escenev}}{\ctau}{\tau}\\
    \{\cvalidE{\Delta}{\Gamma, x_i : \tau_i}{\escenev}{\ce_i}{e_i}{\tau}\}_{i \in \labelset}
  }{
    \cvalidE{\Delta}{\Gamma}{\escenev}{\acecase{\labelset}{\ce}{\mapschemab{x}{\ce}{i}{\labelset}}}{\aecase{\labelset}{e}{\mapschemab{x}{e}{i}{\labelset}}}{\tau}
  }
\end{equation}
\begin{equation}\label{rule:cvalidE-U-splicede}
\inferrule{
  \cvalidT{\emptyset}{\tsfrom{\escenev}}{\ctau}{\tau}\\
  \escenev=\esceneSGB{\uDD{\uD}{\Delta_\text{app}}}{\uGG{\uG}{\Gamma_\text{app}}}{\uPsi}{\uPhi}{b}\\
  \parseUExp{\bsubseq{b}{m}{n}}{\ue}\\
  \expandsSG{\uDD{\uD}{\Delta_\text{app}}}{\uGG{\uG}{\Gamma_\text{app}}}{\uPsi}{\uPhi}{\ue}{e}{\tau}\\\\
  \Delta \cap \Delta_\text{app} = \emptyset\\
  \domof{\Gamma} \cap \domof{\Gamma_\text{app}} = \emptyset
}{
  \cvalidE{\Delta}{\Gamma}{\escenev}{\acesplicede{m}{n}{\ctau}}{e}{\tau}
}
\end{equation}
\begin{grayparbox}
\begin{equation}\label{rule:cvalidE-U-match}
\graybox{\inferrule{
  % \escenev = \esceneUP{\uDD{\uD}{\Delta_\text{app}}}{\uGamma}{\uPsi}{\uPhi}{b}\\\\
  % \istypeU{\Delta \cup \Delta_\text{app}}{\tau'}\\
  \cvalidE{\Delta}{\Gamma}{\escenev}{\ce}{e}{\tau}\\
  % \cvalidT{\Delta}{\tsfrom{\escenev}}{\ctau'}{\tau'}\\\\
  \{\cvalidR{\Delta}{\Gamma}{\escenev}{\crv_i}{r_i}{\tau}{\tau'}\}_{1 \leq i \leq n}\\
}{\cvalidE{\Delta}{\Gamma}{\escenev}{\acematchwith{n}{\ce}{\seqschemaX{\crv}}}{\aematchwith{n}{e}{\seqschemaX{r}}}{\tau'}}}
\end{equation}
\end{grayparbox}
\end{subequations}
% \clearpage
\vspace{-5px}
\begin{grayparbox}
\vspace{15px}
\noindent\fbox{\strut$\cvalidR{\Delta}{\Gamma}{\escenev}{\crv}{r}{\tau}{\tau'}$}~~$\crv$ has expansion $r$ taking values of type $\tau$ to values of type $\tau'$
\begin{equation}\label{rule:cvalidR-UP}
\inferrule{
  % \escenev = \esceneUP{\uDD{\uD}{\Delta_\text{app}}}{\uGamma}{\uPsi}{\uPhi}{b}\\\\
  \patTypeD{\Delta \cup \Delta_\text{app}}{\pctx'}{p}{\tau}\\
  \cvalidE{\Delta}{\Gcons{\Gamma}{\pctx'}}{\escenev}{\ce}{e}{\tau'}
}{
  \cvalidR{\Delta}{\Gamma}{\escenev}{\acematchrule{p}{\ce}}{\aematchrule{p}{e}}{\tau}{\tau'}
}
\end{equation}
\end{grayparbox}
\vspace{-5px}\begin{grayparbox}
\subsection{Proto-Pattern Validation}\label{appendix:proto-pattern-validation-P}
\emph{Pattern splicing scenes}, $\pscenev$, are of the form $\pscene{\uDelta}{\uPhi}{b}$.

\vspace{10px}\noindent\fbox{\strut$\cvalidP{\upctx}{\pscenev}{\cpv}{p}{\tau}$}~~$\cpv$ has expansion $p$ matching against $\tau$ generating hypotheses $\upctx$
% \begin{grayparbox}
\begin{subequations}\label{rules:cvalidP-UP}
\begin{equation}\label{rule:cvalidP-UP-wild}
\inferrule{ }{
  \cvalidP{\uGG{\emptyset}{\emptyset}}{\pscenev}{\acewildp}{\aewildp}{\tau}
}
\end{equation}
\begin{equation}\label{rule:cvalidP-UP-fold}
\inferrule{
  \cvalidP{\upctx}{\pscenev}{\cpv}{p}{[\arec{t}{\tau}/t]\tau}
}{
  \cvalidP{\upctx}{\pscenev}{\acefoldp{\cpv}}{\aefoldp{p}}{\arec{t}{\tau}}
}
\end{equation}
\begin{equation}\label{rule:cvalidP-UP-tpl}
\inferrule{
  \tau = \aprod{\labelset}{\mapschema{\tau}{i}{\labelset}}\\\\
  \{\cvalidP{\upctx_i}{\pscenev}{\cpv_i}{p_i}{\tau_i}\}_{i \in \labelset}
}{
% \left(\shortstack{$\vdash^{\pscenev} \acetplp{\labelset}{\mapschema{\cpv}{i}{\labelset}}$\\$\leadsto$\\$\aetplp{\labelset}{\mapschema{p}{i}{\labelset}} : \aprod{\labelset}{\mapschema{\tau}{i}{\labelset}}~\dashVx^{\,\Gconsi{i \in \labelset}{\upctx_i}}$\vspace{-1.2em}}\right)
  \cvalidP{\GIconsi{i \in \labelset}{\upctx_i}}{\pscenev}{\acetplp{\labelset}{\mapschema{\cpv}{i}{\labelset}}}{\aetplp{\labelset}{\mapschema{p}{i}{\labelset}}}{\tau}
}
\end{equation}
\begin{equation}\label{rule:cvalidP-UP-in}
\inferrule{
  \cvalidP{\upctx}{\pscenev}{\cpv}{p}{\tau}
}{
  \cvalidP{\upctx}{\pscenev}{\aceinjp{\ell}{\cpv}}{\aeinjp{\ell}{p}}{\asum{\labelset, \ell}{\mapschema{\tau}{i}{\labelset}; \mapitem{\ell}{\tau}}}
}
\end{equation}
\begin{equation}\label{rule:cvalidP-UP-spliced}
\inferrule{
  \cvalidT{\emptyset}{\tsceneUP{\uDelta}{b}}{\ctau}{\tau}\\
  \parseUPat{\bsubseq{b}{m}{n}}{\upv}\\
  \patExpands{\upctx}{\uPhi}{\upv}{p}{\tau}
}{
  \cvalidP{\upctx}{\pscene{\uDelta}{\uPhi}{b}}{\acesplicedp{m}{n}{\ctau}}{p}{\tau}
}
\end{equation}
\end{subequations}
\end{grayparbox}
% Observe that, in each of these rules, the proto-expression form and the expanded expression form in the conclusion correspond, and the premises correspond to those of the corresponding typing rule, i.e. Rules (\ref{rule:hastypeU-var}) through (\ref{rule:hastypeU-ap}), respectively. The expression splicing scene, $\escenev$, passes opaquely through these rules.


% We can express this scheme more precisely with the following rule transformation. For each rule in Rules (\ref{rules:hastypeU}),
% \begin{mathpar}\refstepcounter{equation}
% \label{rule:cvalidE-U-tlam}
% \refstepcounter{equation}
% \label{rule:cvalidE-U-tap}
% \refstepcounter{equation}
% \label{rule:cvalidE-U-fold}
% \refstepcounter{equation}
% \label{rule:cvalidE-U-unfold}
% \refstepcounter{equation}
% \label{rule:cvalidE-U-tpl}
% \refstepcounter{equation}
% \label{rule:cvalidE-U-pr}
% \refstepcounter{equation}
% \label{rule:cvalidE-U-in}
% \refstepcounter{equation}
% \label{rule:cvalidE-U-case}
%   \inferrule{
%     J_1\\
%     \cdots\\
%     J_k
%   }{
%     J
%   }
% \end{mathpar}
% the corresponding proto-expression validation rule is 
% \begin{mathpar}
%   \inferrule{
%     \Cof{J_1}\\
%     \cdots\\
%     \Cof{J_k}
%   }{
%     \Cof{J}
%   }
% \end{mathpar}
% where 
% \[\begin{split}
%   \Cof{\istypeU{\Delta}{\tau}} & = \cvalidT{\Delta}{\tsfrom{\escenev}}{\Cof{\tau}}{\tau}\\
%   \Cof{\hastypeU{\Delta}{\Gamma}{e}{\tau}} & = \cvalidE{\Delta}{\Gamma}{\escenev}{\Cof{e}}{e}{\tau}\\
%   \Cof{\{J_i\}_{i \in \labelset}} & = \{\Cof{J_i}\}_{i \in \labelset}
% \end{split}\]
% and where:
% \begin{itemize}
% \item $\Cof{\tau}$ is defined as follows:
%   \begin{itemize}
%   \item When $\tau$ is of definite form, $\Cof{\tau}$ is defined as in Sec. \ref{sec:ce-syntax-U}.
%   \item When $\tau$ is of indefinite form, $\Cof{\tau}$ is a uniquely corresponding metavariable of sort $\mathsf{CETyp}$ also of indefinite form. For example, $\Cof{\tau_1}=\ctau_1$ and $\Cof{\tau_2}=\ctau_2$.
%   \end{itemize}
% \item $\Cof{e}$ is defined as follows
%   \begin{itemize}
%   \item When $e$ is of definite form, $\Cof{e}$ is defined as in Sec. \ref{sec:ce-syntax-U}. 
%   \item When $e$ is of indefinite form, $\Cof{e}$ is a uniquely corresponding metavariable of sort $\mathsf{CEExp}$ also of indefinite form. For example, $\Cof{e_1}=\ce_1$ and $\Cof{e_2}=\ce_2$.
%   \end{itemize}
% \end{itemize}

% It is instructive to use this rule transformation to generate Rules (\ref{rule:cvalidE-U-var}) through (\ref{rule:cvalidE-U-ap}) above. We omit the remaining rules for common forms, i.e. Rules (\ref*{rule:cvalidE-U-tlam}) through (\ref*{rule:cvalidE-U-case}).

\section{Metatheory}\label{appendix:metatheory-SES}
\subsection{Type Expansion}
% The Type Expansion Lemma establishes that the expansion of an unexpanded type is a well-formed type.

\begin{lemma}[Type Expansion]\label{lemma:type-expansion-U} If $\expandsTU{\uDD{\uD}{\Delta}}{\utau}{\tau}$ then $\istypeU{\Delta}{\tau}$.\end{lemma}
\begin{proof} By rule induction over Rules (\ref{rules:expandsTU}). In each case, we apply the IH to or over each premise, then apply the corresponding type formation rule in Rules (\ref{rules:istypeU}). \end{proof}

\begin{lemma}[Proto-Type Validation]\label{lemma:candidate-expansion-type-validation}
If $\cvalidT{\Delta}{\tsceneU{\uDD{\uD}{\Delta_\text{app}}}{b}}{\ctau}{\tau}$ and $\Delta \cap \Delta_\text{app}=\emptyset$ then $\istypeU{\Dcons{\Delta}{\Delta_\text{app}}}{\tau}$.
\end{lemma}
\begin{proof} By rule induction over Rules (\ref{rules:cvalidT-U}).
\begin{byCases}
\item[\text{(\ref{rule:cvalidT-U-tvar})}] ~
\begin{pfsteps*}
   \item $\Delta=\Delta', \Dhyp{t}$ \BY{assumption}
   \item $\ctau=t$ \BY{assumption}
   \item $\tau=t$ \BY{assumption}
   \item $\istypeU{\Delta', \Dhyp{t}}{t}$ \BY{Rule (\ref{rule:istypeU-var})} \pflabel{istype}
   \item $\istypeU{\Dcons{\Delta', \Dhyp{t}}{\Delta_\text{app}}}{t}$ \BY{Lemma \ref{lemma:weakening-U} over $\Delta_\text{app}$ to \pfref{istype}}
 \end{pfsteps*} 
\resetpfcounter

\item[\text{(\ref{rule:cvalidT-U-parr})}] ~
\begin{pfsteps*}
  \item $\ctau=\aceparr{\ctau_1}{\ctau_2}$ \BY{assumption}
  \item $\tau=\aparr{\tau_1}{\tau_2}$ \BY{assumption}
  \item $\cvalidT{\Delta}{\tsceneU{\uDD{\uD}{\Delta_\text{app}}}{b}}{\ctau_1}{\tau_1}$ \BY{assumption} \pflabel{cvalid-ctau1}
  \item $\cvalidT{\Delta}{\tsceneU{\uDD{\uD}{\Delta_\text{app}}}{b}}{\ctau_2}{\tau_2}$ \BY{assumption} \pflabel{cvalid-ctau2}
  \item $\istypeU{\Dcons{\Delta}{\Delta_\text{app}}}{\tau_1}$ \BY{IH on \pfref{cvalid-ctau1}} \pflabel{istype1}
  \item $\istypeU{\Dcons{\Delta}{\Delta_\text{app}}}{\tau_2}$ \BY{IH on \pfref{cvalid-ctau2}} \pflabel{istype2}
  \item $\istypeU{\Dcons{\Delta}{\Delta_\text{app}}}{\aparr{\tau_1}{\tau_2}}$ \BY{Rule (\ref{rule:istypeU-parr}) on \pfref{istype1} and \pfref{istype2}}
\end{pfsteps*}
\resetpfcounter

\item[\text{(\ref{rule:cvalidT-U-all})}] ~
\begin{pfsteps*}
  \item $\ctau=\aceall{t}{\ctau'}$ \BY{assumption}
  \item $\tau=\aall{t}{\tau'}$ \BY{assumption}
  \item $\cvalidT{\Delta, \Dhyp{t}}{\tsceneU{\uDD{\uD}{\Delta_\text{app}}}{b}}{\ctau'}{\tau'}$ \BY{assumption} \label{cvalidT}
  \item $\istypeU{\Dcons{\Delta, \Dhyp{t}}{\Delta_\text{app}}}{\tau'}$ \BY{IH on \pfref{cvalidT}} \pflabel{istypeU1}
  \item $\istypeU{\Dcons{\Delta}{\Delta_\text{app}}, \Dhyp{t}}{\tau'}$ \BY{exchange over $\Delta_\text{app}$ on \pfref{istypeU1}} \pflabel{istypeU2}
  \item $\istypeU{\Dcons{\Delta}{\Delta_\text{app}}}{\aall{t}{\tau'}}$ \BY{Rule (\ref{rule:istypeU-all}) on \pfref{istypeU2}}
\end{pfsteps*}
\resetpfcounter

% \item[{\text{(\ref{rule:cvalidT-U-rec})}}~\textbf{through}~{\text{(\ref{rule:cvalidT-U-sum})}}] These cases follow analagously, i.e. we apply the IH to or over all proto-type validation premises, apply exchange as needed, and then apply the corresponding type formation rule.
% \\
\item[\text{(\ref{rule:cvalidT-U-rec})}] ~
\begin{pfsteps*}
  \item $\ctau=\acerec{t}{\ctau'}$ \BY{assumption}
  \item $\tau=\arec{t}{\tau'}$ \BY{assumption}
  \item $\cvalidT{\Delta, \Dhyp{t}}{\tsceneU{\Delta_\text{app}}{b}}{\ctau'}{\tau'}$ \BY{assumption} \label{cvalidT}
  \item $\istypeU{\Dcons{\Delta, \Dhyp{t}}{\Delta_\text{app}}}{\tau'}$ \BY{IH on \pfref{cvalidT}} \pflabel{istypeU1}
  \item $\istypeU{\Dcons{\Delta}{\Delta_\text{app}}, \Dhyp{t}}{\tau'}$ \BY{exchange over $\Delta_\text{app}$ on \pfref{istypeU1}} \pflabel{istypeU2}
  \item $\istypeU{\Dcons{\Delta}{\Delta_\text{app}}}{\arec{t}{\tau'}}$ \BY{Rule (\ref{rule:istypeU-rec}) on \pfref{istypeU2}}
\end{pfsteps*}
\resetpfcounter

\item[\text{(\ref{rule:cvalidT-U-prod})}] ~
\begin{pfsteps*}
\item $\ctau=\aceprod{\labelset}{\mapschema{\ctau}{i}{\labelset}}$ \BY{assumption}  
\item $\tau=\aprod{\labelset}{\mapschema{\tau}{i}{\labelset}}$ \BY{assumption}
\item $\{\cvalidT{\Delta}{\tsceneU{\Delta_\text{app}}{b}}{\ctau_i}{\tau_i}\}_{i \in \labelset}$ \BY{assumption} \pflabel{cvalidT-ass}
\item $\{\istypeU{\Dcons{\Delta}{\Delta_\text{app}}}{\tau_i}\}_{i \in \labelset}$ \BY{IH over \pfref{cvalidT-ass}} \pflabel{istype}
\item $\istypeU{\Dcons{\Delta}{\Delta_\text{app}}}{\aprod{\labelset}{\mapschema{\tau}{i}{\labelset}}}$ \BY{Rule (\ref{rule:istypeU-prod}) on \pfref{istype}}
\end{pfsteps*}
\resetpfcounter 

\item[\text{(\ref{rule:cvalidT-U-sum})}] ~
\begin{pfsteps*}
\item $\ctau=\acesum{\labelset}{\mapschema{\ctau}{i}{\labelset}}$ \BY{assumption}  
\item $\tau=\asum{\labelset}{\mapschema{\tau}{i}{\labelset}}$ \BY{assumption}
\item $\{\cvalidT{\Delta}{\tsceneU{\Delta_\text{app}}{b}}{\ctau_i}{\tau_i}\}_{i \in \labelset}$ \BY{assumption} \pflabel{cvalidT-ass}
\item $\{\istypeU{\Dcons{\Delta}{\Delta_\text{app}}}{\tau_i}\}_{i \in \labelset}$ \BY{IH over \pfref{cvalidT-ass}} \pflabel{istype}
\item $\istypeU{\Dcons{\Delta}{\Delta_\text{app}}}{\asum{\labelset}{\mapschema{\tau}{i}{\labelset}}}$ \BY{Rule (\ref{rule:istypeU-sum}) on \pfref{istype}}
\end{pfsteps*}
\resetpfcounter

\item[\text{(\ref{rule:cvalidT-U-splicedt})}] ~
\begin{pfsteps*}
\item $\ctau=\acesplicedt{m}{n}$ \BY{assumption}
\item $\parseUTyp{\bsubseq{b}{m}{n}}{\utau}$ \BY{assumption}
\item $\expandsTU{\uDD{\uD}{\Delta_\text{app}}}{\utau}{\tau}$ \BY{assumption} \label{expandsTU}
\item $\Delta \cap \Delta_\text{app} = \emptyset$ \BY{assumption}
\item $\istypeU{\Delta_\text{app}}{\tau}$ \BY{Lemma \ref{lemma:type-expansion-U} on \pfref{expandsTU}}\pflabel{istype}
\item $\istypeU{\Dcons{\Delta}{\Delta_\text{app}}}{\tau}$ \BY{Lemma \ref{lemma:weakening-U} over $\Delta$ on \pfref{istype} and exchange over $\Delta$}
\end{pfsteps*}
\resetpfcounter
\end{byCases}
\end{proof}

\vspace{15px}
\begin{grayparbox}\vspace{-15px}
\subsection{Typed Pattern Expansion}\label{appendix:SES-typed-pattern-expansion}
\begin{theorem}[Typed Pattern Expansion]\label{thm:typed-pattern-expansion} ~
\begin{enumerate}
  \item If $\pExpandsSP{\uDD{\uD}{\Delta}}{\uAS{\uA}{\Phi}}{\upv}{p}{\tau}{\uGG{\uG}{\pctx}}$ then $\patType{\pctx}{p}{\tau}$.
  \item If $\cvalidP{\uGG{\uG}{\pctx}}{\pscene{\uDD{\uD}{\Delta}}{\uAP{\uA}{\Phi}}{b}}{\cpv}{p}{\tau}$ then $\patType{\pctx}{p}{\tau}$.
\end{enumerate}
\end{theorem}
\begin{proof}
  By mutual rule induction over Rules (\ref{rules:patExpands}) and Rules (\ref{rules:cvalidP-UP}).
  \begin{enumerate}
  \item We induct on the premise. In the following, let $\uDelta=\uDD{\uD}{\Delta}$ and $\upctx=\uGG{\uG}{\pctx}$ and $\uPhi=\uAP{\uA}{\Phi}$.
  \begin{byCases}
    \item[\text{(\ref{rule:patExpands-var})}] ~
      \begin{pfsteps*}
        \item $\upv=\ux$ \BY{assumption}
        \item $p=x$ \BY{assumption}
        \item $\pctx=\Ghyp{x}{\tau}$ \BY{assumption}
        \item $\patType{\Ghyp{x}{\tau}}{x}{\tau}$ \BY{Rule (\ref{rule:patType-var})}
      \end{pfsteps*}
      \resetpfcounter
    \item[\text{(\ref{rule:patExpands-wild})}] ~
      \begin{pfsteps*}
        \item $p=\aewildp$ \BY{assumption}
        \item $\pctx = \emptyset$ \BY{assumption}
        \item $\patType{\emptyset}{\aewildp}{\tau}$ \BY{Rule (\ref{rule:patType-wild})}
      \end{pfsteps*}
      \resetpfcounter
    \item[\text{(\ref{rule:patExpands-fold})}] ~
      \begin{pfsteps*}
        \item $\upv=\foldp{\upv'}$ \BY{assumption}
        \item $p=\aefoldp{p'}$ \BY{assumption}
        \item $\tau=\arec{t}{\tau'}$ \BY{assumption}
        %\item $\uptsmenv{\Delta}{\Phi}$ \BY{assumption} \pflabel{env}
        \item $\patExpands{\upctx}{\uPhi}{\upv'}{p'}{[\arec{t}{\tau'}/t]\tau'}$ \BY{assumption} \pflabel{patExpands}
        \item $\patType{\pctx}{p'}{[\arec{t}{\tau'}/t]\tau'}$ \BY{IH, part 1 on \pfref{patExpands}} \pflabel{patType}
        \item $\patType{\pctx}{\aefoldp{p'}}{\arec{t}{\tau'}}$ \BY{Rule (\ref{rule:patType-fold}) on \pfref{patType}}
      \end{pfsteps*}
      \resetpfcounter
    \item[\text{(\ref{rule:patExpands-tpl})}] ~
      \begin{pfsteps*}
        \item $\upv=\tplp{\mapschema{\upv}{i}{\labelset}}$ \BY{assumption}
        \item $p=\aetplp{\labelset}{\mapschema{p}{i}{\labelset}}$ \BY{assumption}
        \item $\tau=\aprod{\labelset}{\mapschema{\tau}{i}{\labelset}}$ \BY{assumption}
        \item $\{\patExpands{\uGG{\uG_i}{\pctx_i}}{\uPhi}{\upv_i}{p_i}{\tau_i}\}_{i \in \labelset}$ \BY{assumption} \pflabel{patExpands}
        \item $\pctx = \Gconsi{i \in \labelset}{\pctx_i}$ \BY{assumption}
        %\item $\uptsmenv{\Delta}{\Phi}$ \BY{assumption} \pflabel{env}
        \item $\{\patType{\pctx_i}{p_i}{\tau_i}\}_{i \in \labelset}$ \BY{IH, part 1 over \pfref{patExpands}}\pflabel{patType}
        \item $\patType{\Gconsi{i \in \labelset}{\pctx_i}}{\aetplp{\labelset}{\mapschema{p}{i}{\labelset}}}{\aprod{\labelset}{\mapschema{\tau}{i}{\labelset}}}$ \BY{Rule (\ref{rule:patType-tpl}) on \pfref{patType}}
      \end{pfsteps*}
      \resetpfcounter
    \item[\text{(\ref{rule:patExpands-in})}] ~
      \begin{pfsteps*}
        \item $\upv=\injp{\ell}{\upv'}$ \BY{assumption}
        \item $p=\aeinjp{\ell}{p'}$ \BY{assumption}
        \item $\tau=\asum{\labelset, \ell}{\mapschema{\tau}{i}{\labelset}; \mapitem{\ell}{\tau'}}$ \BY{assumption}
        \item $\patExpands{\upctx}{\uPhi}{\upv'}{p'}{\tau'}$ \BY{assumption} \pflabel{patExpands}
%        \item $\uptsmenv{\Delta}{\Phi}$ \BY{assumption} \pflabel{env}
        \item $\patType{\pctx}{p'}{\tau'}$ \BY{IH, part 1 on \pfref{patExpands}} \pflabel{patType}
        \item $\patType{\pctx}{\aeinjp{\ell}{p'}}{\asum{\labelset, \ell}{\mapschema{\tau}{i}{\labelset}; \mapitem{\ell}{\tau'}}}$ \BY{Rule (\ref{rule:patType-inj}) on \pfref{patType}}
      \end{pfsteps*}
      \resetpfcounter
    \item[\text{(\ref{rule:patExpands-apuptsm})}] ~
      \begin{pfsteps*}
        \item $\upv=\utsmap{\tsmv}{b}$ \BY{assumption}
        \item $\uA=\uA', \vExpands{\tsmv}{a}$ \BY{assumption}
        \item $\Phi=\Phi', \xuptsmbnd{a}{\tau}{\eparse}$ \BY{assumption}
        \item $\encodeBody{b}{\ebody}$ \BY{assumption}
        \item $\evalU{\eparse(\ebody)}{{\lbltxt{SuccessP}}\cdot{\ecand}}$ \BY{assumption}
        \item $\decodeCEPat{\ecand}{\cpv}$ \BY{assumption}
        \item $\cvalidP{\uGG{\uG}{\pctx}}{\pscene{\uDelta}{\uAP{\uA}{\Phi}}{b}}{\cpv}{p}{\tau}$ \BY{assumption} \pflabel{cvalidP}
%        \item $\uptsmenv{\Delta}{\Phi', \xuptsmbnd{a}{\tau}{\eparse}}$ \BY{assumption} \pflabel{env}
        \item $\patType{\pctx}{p}{\tau}$ \BY{IH, part 2 on \pfref{cvalidP}}
      \end{pfsteps*}
      \resetpfcounter
  \end{byCases}

  \item We induct on the premise. In the following, let $\upctx=\uGG{\uG}{\pctx}$ and $\uDelta = \uDD{\uD}{\Delta}$ and $\uPhi=\uAP{\uA}{\Phi}$.
  \begin{byCases}
    \item[\text{(\ref{rule:cvalidP-UP-wild})}] ~
      \begin{pfsteps*}
        \item $p=\aewildp$ \BY{assumption}
        \item $\pctx=\emptyset$ \BY{assumption}
        \item $\patType{\emptyset}{\aewildp}{\tau}$ \BY{Rule (\ref{rule:patType-wild})}
      \end{pfsteps*}
      \resetpfcounter
    \item[\text{(\ref{rule:cvalidP-UP-fold})}] ~
      \begin{pfsteps*}
        \item $\cpv=\acefoldp{\cpv'}$ \BY{assumption}
        \item $p=\aefoldp{p'}$ \BY{assumption}
        \item $\tau=\arec{t}{\tau'}$ \BY{assumption}
        % \item $\uptsmenv{\Delta}{\Phi}$ \BY{assumption} \pflabel{env}
        \item $\cvalidP{\upctx}{\pscene{\uDelta}{\uPhi}{b}}{\cpv'}{p'}{[\arec{t}{\tau'}/t]\tau'}$ \BY{assumption} \pflabel{cvalidP}
        \item $\patType{\pctx}{p'}{[\arec{t}{\tau'}/t]\tau'}$ \BY{IH, part 2 on \pfref{cvalidP}} \pflabel{patType}
        \item $\patType{\pctx}{\aefoldp{p'}}{\arec{t}{\tau'}}$ \BY{Rule (\ref{rule:patType-fold}) on \pfref{patType}}
      \end{pfsteps*}
      \resetpfcounter
    \item[\text{(\ref{rule:cvalidP-UP-tpl})}] ~
      \begin{pfsteps*}
        \item $\cpv=\acetplp{\labelset}{\mapschema{\cpv}{i}{\labelset}}$ \BY{assumption}
        \item $p=\aetplp{\labelset}{\mapschema{p}{i}{\labelset}}$ \BY{assumption}
        \item $\tau=\aprod{\labelset}{\mapschema{\tau}{i}{\labelset}}$ \BY{assumption}
        \item $\{\cvalidP{\uGG{\uG_i}{\pctx_i}}{\pscene{\uDelta}{\uPhi}{b}}{\cpv_i}{p_i}{\tau_i}\}_{i \in \labelset}$ \BY{assumption} \pflabel{cvalidP}
        \item $\pctx = \Gconsi{i \in \labelset}{\pctx_i}$ \BY{assumption}
        %\item $\uptsmenv{\Delta}{\Phi}$ \BY{assumption} \pflabel{env}
        \item $\{\patType{\pctx_i}{p_i}{\tau_i}\}_{i \in \labelset}$ \BY{IH, part 2 over \pfref{cvalidP}}\pflabel{patType}
        \item $\patType{\Gconsi{i \in \labelset}{\pctx_i}}{\aetplp{\labelset}{\mapschema{p}{i}{\labelset}}}{\aprod{\labelset}{\mapschema{\tau}{i}{\labelset}}}$ \BY{Rule (\ref{rule:patType-tpl}) on \pfref{patType}}
      \end{pfsteps*}
      \resetpfcounter
    \item[\text{(\ref{rule:cvalidP-UP-in})}] ~
      \begin{pfsteps*}
        \item $\cpv=\aceinjp{\ell}{\cpv'}$ \BY{assumption}
        \item $p=\aeinjp{\ell}{p'}$ \BY{assumption}
        \item $\tau=\asum{\labelset, \ell}{\mapschema{\tau}{i}{\labelset}; \mapitem{\ell}{\tau'}}$ \BY{assumption}
        \item $\cvalidP{\upctx}{\pscene{\uDelta}{\uPhi}{b}}{\cpv'}{p'}{\tau'}$ \BY{assumption} \pflabel{cvalidP}
%        \item $\uptsmenv{\Delta}{\Phi}$ \BY{assumption} \pflabel{env}
        \item $\patType{\pctx}{p'}{\tau'}$ \BY{IH, part 2 on \pfref{cvalidP}} \pflabel{patType}
        \item $\patType{\pctx}{\aeinjp{\ell}{p'}}{\asum{\labelset, \ell}{\mapschema{\tau}{i}{\labelset}; \mapitem{\ell}{\tau'}}}$ \BY{Rule (\ref{rule:patType-inj}) on \pfref{patType}}
      \end{pfsteps*}
      \resetpfcounter
    \item[\text{(\ref{rule:cvalidP-UP-spliced})}] ~
      \begin{pfsteps*}
        \item $\cpv=\acesplicedp{m}{n}{\ctau}$ \BY{assumption}
        \item $\cvalidT{\emptyset}{\tsceneUP{\uDelta}{b}}{\ctau}{\tau}$ \BY{assumption}
        \item $\parseUExp{\bsubseq{b}{m}{n}}{\upv}$ \BY{assumption}
        \item $\patExpands{\upctx}{\uPhi}{\upv}{p}{\tau}$ \BY{assumption} \pflabel{patExpands}
        \item $\patType{\pctx}{p}{\tau}$ \BY{IH, part 1 on \pfref{patExpands}}
      \end{pfsteps*}
      \resetpfcounter
  \end{byCases}
  \end{enumerate}
The mutual induction can be shown to be well-founded by showing that the following numeric metric on the judgements that we induct on is decreasing:
\begin{align*}
\sizeof{\patExpands{\upctx}{\uPhi}{\upv}{p}{\tau}} & = \sizeof{\upv}\\
\sizeof{{\cvalidP{\upctx}{\pscene{\uDelta}{\uPhi}{b}}{\cpv}{p}{\tau}}} & = \sizeof{b}
\end{align*}
where $\sizeof{b}$ is the length of $b$ and $\sizeof{\upv}$ is the sum of the lengths of the literal bodies in $\upv$, as defined in Sec. \ref{appendix:SES-syntax}.

The only case in the proof of part 1 that invokes part 2 is Case (\ref{rule:patExpands-apuptsm}). There, we have that the metric remains stable: \begin{align*}
 & \sizeof{\patExpands{\upctx}{\uPhi}{\utsmap{\tsmv}{b}}{p}{\tau}}\\
=& \sizeof{{\cvalidP{\upctx}{\pscene{\uDelta}{\uPhi}{b}}{\cpv}{p}{\tau}}}\\
=&\sizeof{b}\end{align*}

The only case in the proof of part 2 that invokes part 1 is Case (\ref{rule:cvalidP-UP-spliced}). There, we have that $\parseUPat{\bsubseq{b}{m}{n}}{\upv}$ and the IH is applied to the judgement $\patExpands{\upctx}{\uPhi}{\upv}{p}{\tau}$. Because the metric is stable when passing from part 1 to part 2, we must have that it is strictly decreasing in the other direction:
\[\sizeof{\patExpands{\upctx}{\uPhi}{\upv}{p}{\tau}} < \sizeof{{\cvalidP{\upctx}{\pscene{\uDelta}{\uPhi}{b}}{\acesplicedp{m}{n}{\ctau}}{p}{\tau}}}\]
i.e. by the definitions above, 
\[\sizeof{\upv} < \sizeof{b}\]

This is established by appeal to Condition \ref{condition:body-subsequences}, which states that subsequences of $b$ are no longer than $b$, and the Condition \ref{condition:pattern-parsing}, which states that an unexpanded pattern constructed by parsing a textual sequence $b$ is strictly smaller, as measured by the metric defined above, than the length of $b$, because some characters must necessarily be used to apply the pattern TSM and delimit each literal body. Combining Conditions \ref{condition:body-subsequences} and \ref{condition:pattern-parsing}, we have that $\sizeof{\upv} < \sizeof{b}$ as needed.
\end{proof}

\end{grayparbox}
\subsection{Typed Expression Expansion}\label{appendix:SES-typed-expression-expansion-metatheory}
\begin{theorem}[Typed Expansion (Full)]\label{thm:typed-expansion-full-U} ~
\begin{enumerate}
  \item \begin{enumerate}
    \item If $\expandsSG{\uDD{\uD}{\Delta}}{\uGG{\uG}{\Gamma}}{\uPsi}{\uPhi}{\ue}{e}{\tau}$ then $\hastypeU{\Delta}{\Gamma}{e}{\tau}$.
    \item \graytxtbox{If $\ruleExpands{\uDD{\uD}{\Delta}}{\uGG{\uG}{\Gamma}}{\uPsi}{\uPhi}{\urv}{r}{\tau}{\tau'}$  then $\ruleType{\Delta}{\Gamma}{r}{\tau}{\tau'}$.}
  \end{enumerate}
  \item \begin{enumerate}
    \item If $\cvalidE{\Delta}{\Gamma}{\esceneSG{\uDD{\uD}{\Delta_\text{app}}}{\uGG{\uG}{\Gamma_\text{app}}}{\uPsi}{\uPhi}{b}}{\ce}{e}{\tau}$ and $\Delta \cap \Delta_\text{app}=\emptyset$ and $\domof{\Gamma} \cap \domof{\Gamma_\text{app}}=\emptyset$ then $\hastypeU{\Dcons{\Delta}{\Delta_\text{app}}}{\Gcons{\Gamma}{\Gamma_\text{app}}}{e}{\tau}$. 
    \item \begin{grayparbox}If $\cvalidR{\Delta}{\Gamma}{\esceneUP{\uDD{\uD}{\Delta_\text{app}}}{\uGG{\uG}{\Gamma_\text{app}}}{\uPsi}{\uPhi}{b}}{\crv}{r}{\tau}{\tau'}$ and $\Delta \cap \Delta_\text{app}=\emptyset$ and $\domof{\Gamma} \cap \domof{\Gamma_\text{app}}=\emptyset$ then $\ruleType{\Dcons{\Delta}{\Delta_\text{app}}}{\Gcons{\Gamma}{\Gamma_\text{app}}}{r}{\tau}{\tau'}$.\end{grayparbox}
  \end{enumerate}
\end{enumerate}
\end{theorem}
\begin{proof}
By mutual rule induction over Rules (\ref{rules:expandsU}), \graytxtbox{Rule (\ref{rule:ruleExpands}),} Rules (\ref{rules:cvalidE-U}) \graytxtbox{and Rule (\ref{rule:cvalidR-UP})}.

\begin{enumerate}
\item In the following, let $\uDelta=\uDD{\uD}{\Delta}$ and $\uGamma=\uGG{\uG}{\Gamma}$.
  \begin{enumerate}
  \item 
  \begin{byCases} \item[\text{(\ref{rule:expandsU-var})}] ~
\begin{pfsteps}
  \item \ue=\ux \BY{assumption}
  \item e=x \BY{assumption}
  \item \Gamma=\Gamma', \Ghyp{x}{\tau} \BY{assumption}
  \item \hastypeU{\Delta}{\Gamma', \Ghyp{x}{\tau}}{x}{\tau} \BY{Rule (\ref{rule:hastypeU-var})}
\end{pfsteps}
\resetpfcounter

\item[\text{(\ref{rule:expandsU-asc})}] ~
\begin{pfsteps}
  \item \ue=\asc{\ue'}{\utau} \BY{assumption}
  \item \expandsTU{\uDelta}{\utau}{\tau} \BY{assumption} \pflabel{expandsTU}
  \item \expandsSG{\uDelta}{\uGamma}{\uPsi}{\uPhi}{\ue'}{e}{\tau} \BY{assumption} \pflabel{expandsSG}
  \item \hastypeU{\Delta}{\Gamma}{e}{\tau} \BY{IH, part 1(a) on \pfref{expandsSG}}
\end{pfsteps}
\resetpfcounter

\item[\text{(\ref{rule:expandsU-letsyn})}] ~
\begin{pfsteps}
  \item \ue=\letsyn{\ux}{\ue_1}{\ue_2} \BY{assumption}
  \item e =   \aeap{\aelam{\tau_1}{x}{e_2}}{e_1} \BY{assumption}
  \item     \expandsSG{\uDelta}{\uGamma}{\uPsi}{\uPhi}{\ue_1}{e_1}{\tau_1} \BY{assumption} \pflabel{expandsSG1}
  \item     \expandsSG{\uDelta}{\uGamma, \uGhyp{\ux}{x}{\tau_1}}{\uPsi}{\uPhi}{\ue_2}{e_2}{\tau} \BY{assumption} \pflabel{expandsSG2}
  \item \hastypeU{\Delta}{\Gamma}{e_1}{\tau_1} \BY{IH, part 1(a) on \pfref{expandsSG1}} \pflabel{hastype1}
  \item \hastypeU{\Delta}{\Gamma, x : \tau}{e_2}{\tau} \BY{IH, part 1(a) on \pfref{expandsSG2}} \pflabel{hastype2}
  \item \hastypeU{\Delta}{\Gamma}{\aelam{\tau_1}{x}{e_2}}{\aparr{\tau_1}{\tau}} \BY{Rule (\ref{rule:hastypeU-lam}) on \pfref{hastype2}} \pflabel{hastype3}
  \item \hastypeU{\Delta}{\Gamma}{\aeap{\aelam{\tau_1}{x}{e_2}}{e_1}}{\tau} \BY{Rule (\ref{rule:hastypeU-ap}) on \pfref{hastype3} and \pfref{hastype1}}
\end{pfsteps}
\resetpfcounter

\item[\text{(\ref{rule:expandsU-lam})}] ~
\begin{pfsteps}
  \item \ue=\lam{\ux}{\utau_1}{\ue'} \BY{assumption}
  \item e=\aelam{\tau_1}{x}{e'} \BY{assumption}
  \item \tau=\aparr{\tau_1}{\tau_2} \BY{assumption}
  \item \expandsTU{\uDelta}{\utau_1}{\tau_1} \BY{assumption} \pflabel{istype}
  \item \expandsSG{\uDelta}{\uGamma, \uGhyp{\ux}{x}{\tau_1}}{\uPsi}{\uPhi}{\ue'}{e'}{\tau_2} \BY{assumption} \pflabel{expandsU}
%  \item \uetsmenv{\Delta}{\Psi} \BY{assumption} \pflabel{uetsmenv}
  \item \istypeU{\Delta}{\tau_1} \BY{Lemma \ref{lemma:type-expansion-U} on \pfref{istype}} \pflabel{istype2}
  \item \hastypeU{\Delta}{\Gamma, \Ghyp{x}{\tau_1}}{e'}{\tau_2} \BY{IH, part 1(a) on \pfref{expandsU}} \pflabel{hastypeU}
  \item \hastypeU{\Delta}{\Gamma}{\aelam{\tau_1}{x}{e'}}{\aparr{\tau_1}{\tau_2}} \BY{Rule (\ref{rule:hastypeU-lam}) on \pfref{istype2} and \pfref{hastypeU}}
\end{pfsteps}
\resetpfcounter

\item[\text{(\ref{rule:expandsU-ap})}] ~
\begin{pfsteps}
  \item \ue=\ap{\ue_1}{\ue_2} \BY{assumption}
  \item e=\aeap{e_1}{e_2} \BY{assumption}
  \item \expandsSG{\uDelta}{\uGamma}{\uPsi}{\uPhi}{\ue_1}{e_1}{\aparr{\tau_2}{\tau}} \BY{assumption}\pflabel{expandsU1}
  \item \expandsSG{\uDelta}{\uGamma}{\uPsi}{\uPhi}{\ue_2}{e_2}{\tau_2} \BY{assumption}\pflabel{expandsU2}
%  \item \uetsmenv{\Delta}{\Psi} \BY{assumption} \pflabel{uetsmenv}
  \item \hastypeU{\Delta}{\Gamma}{e_1}{\aparr{\tau_2}{\tau}} \BY{IH, part 1(a) on \pfref{expandsU1}}\pflabel{hastypeU1}
  \item \hastypeU{\Delta}{\Gamma}{e_2}{\tau_2} \BY{IH, part 1(a) on \pfref{expandsU2}}\pflabel{hastypeU2}
  \item \hastypeU{\Delta}{\Gamma}{\aeap{e_1}{e_2}}{\tau} \BY{Rule (\ref{rule:hastypeU-ap}) on \pfref{hastypeU1} and \pfref{hastypeU2}}
\end{pfsteps}
\resetpfcounter

\item[\text{(\ref{rule:expandsU-tlam})}~\textbf{through}~\text{(\ref{rule:expandsU-case})}] These cases follow analagously, i.e. we apply Lemma \ref{lemma:type-expansion-U} to or over the type expansion premises and the IH part 1(a) to or over the typed expression expansion premises and then apply the corresponding typing rule in Rules (\ref{rule:hastypeU-tlam}) through (\ref{rule:hastypeU-case}).
\\
\item[\text{(\ref{rule:expandsU-syntax})}] ~ 
\begin{pfsteps}
  \item \ue=\uesyntax{\tsmv}{\utau'}{\eparse}{\ue'} \BY{assumption}
  \item \expandsTU{\uDelta}{\utau'}{\tau'} \BY{assumption} \pflabel{expandsTU}
 \item \hastypeU{\emptyset}{\emptyset}{\eparse}{\aparr{\tBody}{\tParseResultExp}} \BY{assumption}\pflabel{eparse}
  \item \expandsSG{\uDelta}{\uGamma}{\uPsi, \uShyp{\tsmv}{a}{\tau'}{\eparse}}{\uPhi}{\ue'}{e}{\tau} \BY{assumption}\pflabel{expandsU}
%  \item \uetsmenv{\Delta}{\Psi} \BY{assumption}\pflabel{uetsmenv1}
 \item \istypeU{\Delta}{\tau'} \BY{Lemma \ref{lemma:type-expansion-U} to \pfref{expandsTU}} \pflabel{istype}
%  \item \uetsmenv{\Delta}{\Psi, \xuetsmbnd{\tsmv}{\tau'}{\eparse}} \BY{Definition \ref{def:seTSM-def-ctx-formation} on \pfref{uetsmenv1}, \pfref{istype} and \pfref{eparse}}\pflabel{uetsmenv3}
  \item \hastypeU{\Delta}{\Gamma}{e}{\tau} \BY{IH, part 1(a) on \pfref{expandsU}}
\end{pfsteps}
\resetpfcounter 

\item[\text{(\ref{rule:expandsU-tsmap})}] ~ 
\begin{pfsteps}
  \item \ue=\utsmap{\tsmv}{b} \BY{assumption}
  \item \uA = \uA', \vExpands{\tsmv}{a} \BY{assumption}
  \item \Psi=\Psi', \xuetsmbnd{a}{\tau}{\eparse} \BY{assumption}
  \item \encodeBody{b}{\ebody} \BY{assumption}
  \item \evalU{\eparse(\ebody)}{{\lbltxt{SuccessE}}\cdot{\ecand}} \BY{assumption}
  \item \decodeCondE{\ecand}{\ce} \BY{assumption}
  \item \cvalidE{\emptyset}{\emptyset}{\esceneSG{\uDelta}{\uGamma}{\uPsi}{\uPhi}{b}}{\ce}{e}{\tau} \BY{assumption}\pflabel{cvalidE}
%  \item \uetsmenv{\Delta}{\Psi} \BY{assumption} \pflabel{uetsmenv}
  \item \emptyset \cap \Delta = \emptyset \BY{finite set intersection} \pflabel{delta-cap}
  \item {\emptyset} \cap \domof{\Gamma} = \emptyset \BY{finite set intersection} \pflabel{gamma-cap}
  \item \hastypeU{\emptyset \cup \Delta}{\emptyset \cup \Gamma}{e}{\tau} \BY{IH, part 2(a) on \pfref{cvalidE}, \pfref{delta-cap}, and \pfref{gamma-cap}} \pflabel{penultimate}
  \item \hastypeU{\Delta}{\Gamma}{e}{\tau} \BY{finite set and finite function identity over \pfref{penultimate}}
\end{pfsteps}
\resetpfcounter
\end{byCases}
\end{enumerate}
\begin{grayparbox}
\begin{enumerate}
\item[\hphantom{(a)}] \begin{byCases}
    \item[\text{(\ref{rule:expandsU-match})}] ~
      \begin{pfsteps*}
        \item $\ue=\matchwith{\ue'}{\seqschemaX{\urv}}$ \BY{assumption}
        \item $e=\aematchwith{n}{e'}{\seqschemaX{r}}$ \BY{assumption}
        \item $\expandsUP{\uDelta}{\uGamma}{\uPsi}{\uPhi}{\ue'}{e'}{\tau'}$ \BY{assumption} \pflabel{expandsUP}
        % \item $\istypeU{\Delta}{\tau}$ \BY{assumption}\pflabel{istype}
        % \item $\expandsTU{\uDelta}{\utau}{\tau}$ \BY{assumption} \pflabel{expandsTU}
        \item $\{\ruleExpands{\uDelta}{\uGamma}{\uPsi}{\uPhi}{\urv_i}{r_i}{\tau'}{\tau}\}_{1 \leq i \leq n}$ \BY{assumption}\pflabel{ruleExpands}
        \item $\hastypeU{\Delta}{\Gamma}{e'}{\tau'}$ \BY{IH, part 1(a) on \pfref{expandsUP}}\pflabel{hasType}
        \item $\{\ruleType{\Delta}{\Gamma}{r_i}{\tau'}{\tau}\}_{1 \leq i \leq n}$ \BY{IH, part 1(b) over \pfref{ruleExpands}}\pflabel{ruleType}
        \item $\hastypeU{\Delta}{\Gamma}{\aematchwith{n}{e'}{\seqschemaX{r}}}{\tau}$ \BY{Rule (\ref{rule:hastypeUP-match}) on \pfref{hasType} and \pfref{ruleType}}
      \end{pfsteps*}
      \resetpfcounter

    \item[\text{(\ref{rule:expandsU-defuptsm})}] ~
      \begin{pfsteps}
          \item \ue=\usyntaxup{\tsmv}{\utau'}{\eparse}{\ue'} \BY{assumption}
          \item \expandsTU{\uDelta}{\utau'}{\tau'} \BY{assumption} \pflabel{expandsTU}
         \item \hastypeU{\emptyset}{\emptyset}{\eparse}{\aparr{\tBody}{\tParseResultExp}} \BY{assumption}\pflabel{eparse}
          \item \expandsUP{\uDelta}{\uGamma}{\uPsi}{\uPhi, \uPhyp{\tsmv}{a}{\tau'}{\eparse}}{\ue'}{e}{\tau} \BY{assumption}\pflabel{expandsU}
        %  \item \uetsmenv{\Delta}{\Psi} \BY{assumption}\pflabel{uetsmenv1}
         \item \istypeU{\Delta}{\tau'} \BY{Lemma \ref{lemma:type-expansion-U} to \pfref{expandsTU}} \pflabel{istype}
        %  \item \uetsmenv{\Delta}{\Psi, \xuetsmbnd{\tsmv}{\tau'}{\eparse}} \BY{Definition \ref{def:seTSM-def-ctx-formation} on \pfref{uetsmenv1}, \pfref{istype} and \pfref{eparse}}\pflabel{uetsmenv3}
          \item \hastypeU{\Delta}{\Gamma}{e}{\tau} \BY{IH, part 1(a) on \pfref{expandsU}}
        \end{pfsteps}
        \resetpfcounter 
  \end{byCases}
  \end{enumerate}
  \end{grayparbox}
  \vspace{-4px}\begin{grayparbox}\vspace{4px}
  \begin{enumerate}
  \item[(b)] \begin{byCases}
    \item[\text{(\ref{rule:ruleExpands})}] ~
      \begin{pfsteps*}
        \item $\urv=\matchrule{\upv}{\ue}$ \BY{assumption}
        \item $r=\aematchrule{p}{e}$ \BY{assumption}
        \item $\patExpands{\uGG{\uA'}{\pctx}}{\uPhi}{\upv}{p}{\tau}$ \BY{assumption} \pflabel{patExpands}
        \item $\expandsUP{\uDelta}{\uGG{{\uA}\uplus{\uA'}}{\Gcons{\Gamma}{\pctx}}}{\uPsi}{\uPhi}{\ue}{e}{\tau'}$ \BY{assumption} \pflabel{expandsUP}
        \item $\patType{\pctx}{p}{\tau}$ \BY{Theorem \ref{thm:typed-pattern-expansion}, part 1 on \pfref{patExpands}}\pflabel{patType}
        \item $\hastypeU{\Delta}{\Gcons{\Gamma}{\pctx}}{e}{\tau'}$ \BY{IH, part 1(a) on \pfref{expandsUP}} \pflabel{hasType}
        \item $\ruleType{\Delta}{\Gamma}{\aematchrule{p}{e}}{\tau}{\tau'}$ \BY{Rule (\ref{rule:ruleType}) on \pfref{patType} and \pfref{hasType}}
      \end{pfsteps*}
      \resetpfcounter
  \end{byCases}
  \end{enumerate}
  \end{grayparbox}

\item In the following, let $\uDelta=\uDD{\uD}{\Delta_\text{app}}$ and $\uGamma=\uGG{\uG}{\Gamma_\text{app}}$. \begin{enumerate}
  \item 
  \begin{byCases}
    \item[\text{(\ref{rule:cvalidE-U-var})}] ~
\begin{pfsteps*}
  \item $\ce=x$ \BY{assumption}
  \item $e=x$ \BY{assumption}
  \item $\Gamma=\Gamma', \Ghyp{x}{\tau}$ \BY{assumption}
  \item $\hastypeU{\Dcons{\Delta}{\Delta_\text{app}}}{\Gamma', \Ghyp{x}{\tau}}{x}{\tau}$ \BY{Rule (\ref{rule:hastypeU-var})} \pflabel{hastypeU}
  \item $\hastypeU{\Dcons{\Delta}{\Delta_\text{app}}}{\Gcons{\Gamma', \Ghyp{x}{\tau}}{\Gamma_\text{app}}}{x}{\tau}$ \BY{Lemma \ref{lemma:weakening-U} over $\Gamma_\text{app}$ to \pfref{hastypeU}}
\end{pfsteps*}
\resetpfcounter

\item[\text{(\ref{rule:cvalidE-U-lam})}] ~
\begin{pfsteps*}
  \item $\ce=\acelam{\ctau_1}{x}{\ce'}$ \BY{assumption}
  \item $e=\aelam{\tau_1}{x}{e'}$ \BY{assumption}
  \item $\tau=\aparr{\tau_1}{\tau_2}$ \BY{assumption}
  \item $\cvalidT{\Delta}{\tsceneU{\uDelta_\text{app}}{b}}{\ctau_1}{\tau_1}$ \BY{assumption} \pflabel{cvalidT}
  \item $\cvalidE{\Delta}{\Gamma, \Ghyp{x}{\tau_1}}{\esceneSG{\uDelta_\text{app}}{\uGamma_\text{app}}{\uPsi}{\uPhi}{b}}{\ce'}{e'}{\tau_2}$ \BY{assumption} \pflabel{cvalidE}
%  \item $\uetsmenv{\Delta_\text{app}}{\Psi}$ \BY{assumption} \pflabel{uetsmenv}
  \item $\Delta \cap \Delta_\text{app}=\emptyset$ \BY{assumption} \pflabel{delta-disjoint}
  \item $\domof{\Gamma} \cap \domof{\Gamma_\text{app}}=\emptyset$ \BY{assumption} \pflabel{gamma-disjoint}
  \item $x \notin \domof{\Gamma_\text{app}}$ \BY{identification convention} \pflabel{x-fresh}
  \item $\domof{\Gamma, x : \tau_1} \cap \domof{\Gamma_\text{app}}=\emptyset$ \BY{\pfref{gamma-disjoint} and \pfref{x-fresh}} \pflabel{gamma-disjoint2}
  \item $\istypeU{\Dcons{\Delta}{\Delta_\text{app}}}{\tau_1}$ \BY{Lemma \ref{lemma:candidate-expansion-type-validation} on \pfref{cvalidT} and \pfref{delta-disjoint}} \pflabel{istype}
  \item $\hastypeU{\Dcons{\Delta}{\Delta_\text{app}}}{\Gcons{\Gamma, \Ghyp{x}{\tau_1}}{\Gamma_\text{app}}}{e'}{\tau_2}$ \BY{IH, part 2(a) on \pfref{cvalidE}, \pfref{delta-disjoint} and \pfref{gamma-disjoint2}} \pflabel{hastype1}
  \item $\hastypeU{\Dcons{\Delta}{\Delta_\text{app}}}{\Gcons{\Gamma}{\Gamma_\text{app}}, \Ghyp{x}{\tau_1}}{e'}{\tau_2}$ \BY{exchange over $\Gamma_\text{app}$ on \pfref{hastype1}} \pflabel{hastype2}
  \item $\hastypeU{\Dcons{\Delta}{\Delta_\text{app}}}{\Gcons{\Gamma}{\Gamma_\text{app}}}{\aelam{\tau_1}{x}{e'}}{\aparr{\tau_1}{\tau_2}}$ \BY{Rule (\ref{rule:hastypeU-lam}) on \pfref{istype} and \pfref{hastype2}}
\end{pfsteps*}
\resetpfcounter

\item[\text{(\ref{rule:cvalidE-U-ap})}] ~
\begin{pfsteps*}
  \item $\ce=\aceap{\ce_1}{\ce_2}$ \BY{assumption}
  \item $e=\aeap{e_1}{e_2}$ \BY{assumption}
  \item $\cvalidE{\Delta}{\Gamma}{\esceneSG{\uDelta_\text{app}}{\uGamma_\text{app}}{\uPsi}{\uPhi}{b}}{\ce_1}{e_1}{\aparr{\tau_2}{\tau}}$ \BY{assumption} \pflabel{cvalidE1}
  \item $\cvalidE{\Delta}{\Gamma}{\esceneSG{\uDelta_\text{app}}{\uGamma_\text{app}}{\uPsi}{\uPhi}{b}}{\ce_2}{e_2}{\tau_2}$ \BY{assumption} \pflabel{cvalidE2}
%  \item $\uetsmenv{\Delta_\text{app}}{\Psi}$ \BY{assumption} \pflabel{uetsmenv}
  \item $\Delta \cap \Delta_\text{app}=\emptyset$ \BY{assumption} \pflabel{delta-disjoint}
  \item $\domof{\Gamma} \cap \domof{\Gamma_\text{app}}=\emptyset$ \BY{assumption} \pflabel{gamma-disjoint}
  \item $\hastypeU{\Dcons{\Delta}{\Delta_\text{app}}}{\Gcons{\Gamma}{\Gamma_\text{app}}}{e_1}{\aparr{\tau_2}{\tau}}$ \BY{IH, part 2(a) on \pfref{cvalidE1}, \pfref{delta-disjoint} and \pfref{gamma-disjoint}} \pflabel{hastypeU1}
  \item $\hastypeU{\Dcons{\Delta}{\Delta_\text{app}}}{\Gcons{\Gamma}{\Gamma_\text{app}}}{e_2}{\tau_2}$ \BY{IH, part 2(a) on \pfref{cvalidE2}, \pfref{delta-disjoint} and \pfref{gamma-disjoint}} \pflabel{hastypeU2}
  \item $\hastypeU{\Dcons{\Delta}{\Delta_\text{app}}}{\Gcons{\Gamma}{\Gamma_\text{app}}}{\aeap{e_1}{e_2}}{\tau}$ \BY{Rule (\ref{rule:hastypeU-ap}) on \pfref{hastypeU1} and \pfref{hastypeU2}}
\end{pfsteps*}
\resetpfcounter

\item[\text{(\ref{rule:cvalidE-U-tlam})}] ~
\begin{pfsteps}
  \item \ce=\acetlam{t}{\ce'} \BY{assumption}
  \item e = \aetlam{t}{e'} \BY{assumption}
  \item \tau = \aall{t}{\tau'}\BY{assumption}
  \item \cvalidE{\Delta, \Dhyp{t}}{\Gamma}{\esceneSG{\uDelta_\text{app}}{\uGamma_\text{app}}{\uPsi}{\uPhi}{b}}{\ce'}{e'}{\tau'} \BY{assumption} \pflabel{cvalidE}
%  \item \uetsmenv{\Delta_\text{app}}{\Psi} \BY{assumption} \pflabel{uetsmenv}
  \item \Delta \cap \Delta_\text{app}=\emptyset \BY{assumption} \pflabel{delta-disjoint}
  \item \domof{\Gamma} \cap \domof{\Gamma_\text{app}}=\emptyset \BY{assumption} \pflabel{gamma-disjoint}
  \item \Dhyp{t} \notin \Delta_\text{app} \BY{identification convention}\pflabel{t-fresh}
  \item \Delta, \Dhyp{t} \cap \Delta_\text{app} = \emptyset \BY{\pfref{delta-disjoint} and \pfref{t-fresh}}\pflabel{delta-disjoint2}
  \item \hastypeU{\Dcons{\Delta, \Dhyp{t}}{\Delta_\text{app}}}{\Gcons{\Gamma}{\Gamma_\text{app}}}{e'}{\tau'} \BY{IH, part 2(a) on \pfref{cvalidE}, \pfref{delta-disjoint2} and \pfref{gamma-disjoint}}\pflabel{hastype1}
  \item \hastypeU{\Dcons{\Delta}{\Delta_\text{app}, \Dhyp{t}}}{\Gcons{\Gamma}{\Gamma_\text{app}}}{e'}{\tau'} \BY{exchange over $\Delta_\text{app}$ on \pfref{hastype1}}\pflabel{hastype2}
  \item \hastypeU{\Dcons{\Delta}{\Delta_\text{app}}}{\Gcons{\Gamma}{\Gamma_\text{app}}}{\aetlam{t}{e'}}{\aall{t}{\tau'}} \BY{Rule (\ref{rule:hastypeU-tlam}) on \pfref{hastype2}}
\end{pfsteps}
\resetpfcounter

\item[{\text{(\ref{rule:cvalidE-U-tap})}}~\textbf{through}~{\text{(\ref{rule:cvalidE-U-case})}}] These cases follow analagously, i.e. we apply the IH, part 2(a) to all proto-expression validation judgements, Lemma \ref{lemma:candidate-expansion-type-validation} to all proto-type validation judgements, the identification convention to ensure that extended contexts remain disjoint, weakening and exchange as needed, and the corresponding typing rule in Rules (\ref{rule:hastypeU-tap}) through (\ref{rule:hastypeU-case}).
\\

\item[\text{(\ref{rule:cvalidE-U-splicede})}] ~
\begin{pfsteps*}
  \item $\ce=\acesplicede{m}{n}{\ctau}$ \BY{assumption}
  \item $\escenev=\esceneU{\uDD{\uD}{\Delta_\text{app}}}{\uGG{\uG}{\Gamma_\text{app}}}{\uPsi}{b}$ \BY{assumption}
  \item   $\cvalidT{\emptyset}{\tsfrom{\escenev}}{\ctau}{\tau}$ \BY{assumption}
  \item $\parseUExp{\bsubseq{b}{m}{n}}{\ue}$ \BY{assumption}
  \item $\expandsU{\uDelta_\text{app}}{\uGamma_\text{app}}{\uPsi}{\ue}{e}{\tau}$ \BY{assumption} \pflabel{expands}
%  \item $\uetsmenv{\Delta_\text{app}}{\Psi}$ \BY{assumption} \pflabel{uetsmenv}
  \item $\Delta \cap \Delta_\text{app}=\emptyset$ \BY{assumption} \pflabel{delta-disjoint}
  \item $\domof{\Gamma} \cap \domof{\Gamma_\text{app}}=\emptyset$ \BY{assumption} \pflabel{gamma-disjoint}
  \item $\hastypeU{\Delta_\text{app}}{\Gamma_\text{app}}{e}{\tau}$ \BY{IH, part 1 on \pfref{expands}} \pflabel{hastype}
  \item $\hastypeU{\Dcons{\Delta}{\Delta_\text{app}}}{\Gcons{\Gamma}{\Gamma_\text{app}}}{e}{\tau}$ \BY{Lemma \ref{lemma:weakening-U} over $\Delta$ and $\Gamma$ and exchange on \pfref{hastype}}
\end{pfsteps*}
\resetpfcounter
\end{byCases}
\end{enumerate}
\begin{grayparbox}
\begin{enumerate}
\item[\hphantom{(a)}] \begin{byCases}
    \item[\text{(\ref{rule:cvalidE-U-match})}] ~
      \begin{pfsteps*}
        \item $\ce=\acematchwith{n}{\ce'}{\seqschemaX{\crv}}$ \BY{assumption}
        \item $e=\aematchwith{n}{e'}{\seqschemaX{r}}$ \BY{assumption}
        \item $\cvalidE{\Delta}{\Gamma}{\esceneUP{\uDelta}{\uGamma}{\uPsi}{\uPhi}{b}}{\ce'}{e'}{\tau'}$ \BY{assumption} \pflabel{cvalidE}        
        % \item $\istypeU{\Delta \cup \Delta_\text{app}}{\tau}$ \BY{assumption} \pflabel{istype}
        % \item $\cvalidT{\Delta}{\tsceneUP{\uDelta}{b}}{\ctau}{\tau}$ \BY{assumption} \pflabel{cvalidT}
        \item $\{\cvalidR{\Delta}{\Gamma}{\esceneUP{\uDelta}{\uGamma}{\uPsi}{\uPhi}{b}}{\crv_i}{r_i}{\tau'}{\tau}\}_{1 \leq i \leq n}$ \BY{assumption} \pflabel{cvalidR}
        \item $\Delta \cap \Delta_\text{app} = \emptyset$ \BY{assumption} \pflabel{delta-disjoint}
        \item $\domof{\Gamma} \cap \domof{\Gamma_\text{app}} = \emptyset$ \BY{assumption} \pflabel{gamma-disjoint}
        \item $\hastypeU{\Delta \cup \Delta_\text{app}}{\Gamma \cup \Gamma_\text{app}}{e'}{\tau'}$ \BY{IH, part 2(a) on \pfref{cvalidE}, \pfref{delta-disjoint} and \pfref{gamma-disjoint}} \pflabel{hastype}
        \item $\ruleType{\Delta \cup \Delta_\text{app}}{\Gamma \cup \Gamma_\text{app}}{r}{\tau'}{\tau}$ \BY{IH, part 2(b) on \pfref{cvalidR}, \pfref{delta-disjoint} and \pfref{gamma-disjoint}} \pflabel{ruleType}
        \item $\hastypeU{\Delta \cup \Delta_\text{app}}{\Gamma \cup \Gamma_\text{app}}{\aematchwith{n}{e'}{\seqschemaX{r}}}{\tau}$ \BY{Rule (\ref{rule:hastypeUP-match}) on \pfref{hastype} and \pfref{ruleType}}
      \end{pfsteps*}
      \resetpfcounter
  \end{byCases}
    \end{enumerate}
  \end{grayparbox}\vspace{-3px}
  \begin{grayparbox}\vspace{3px}
  \begin{enumerate}
  \item[(b)] There is only one case. 
    \begin{byCases}
     \item[\text{(\ref{rule:cvalidR-UP})}] ~
      \begin{pfsteps*}
        \item $\crv=\acematchrule{p}{\ce}$ \BY{assumption}
        \item $r=\aematchrule{p}{e}$ \BY{assumption}
        \item $\patType{\pctx'}{p}{\tau}$ \BY{assumption} \pflabel{patType}
        \item $\cvalidE{\Delta}{\Gcons{\Gamma}{\pctx'}}{\esceneUP{\uDelta}{\uGamma}{\uPsi}{\uPhi}{b}}{\ce}{e}{\tau'}$ \BY{assumption} \pflabel{cvalidE}
        \item $\Delta \cap \Delta_\text{app} = \emptyset$ \BY{assumption}\pflabel{delta-disjoint}
        \item $\domof{\Gamma} \cap \domof{\pctx'} = \emptyset$ \BY{identification convention}\pflabel{gamma-disjoint1}
        \item $\domof{\Gamma_\text{app}} \cap \domof{\pctx'} = \emptyset$ \BY{identification convention}\pflabel{gamma-disjoint2}
        \item $\domof{\Gamma} \cap \domof{\Gamma_\text{app}} = \emptyset$ \BY{assumption}\pflabel{gamma-disjoint3}
        \item $\domof{\Gcons{\Gamma}{\pctx'}} \cap \domof{\Gamma_\text{app}} = \emptyset$ \BY{standard finite set definitions and identities on \pfref{gamma-disjoint1}, \pfref{gamma-disjoint2} and \pfref{gamma-disjoint3}}\pflabel{gamma-disjoint4}
        \item $\hastypeU{\Dcons{\Delta}{\Delta_\text{app}}}{\Gcons{\Gcons{\Gamma}{\pctx'}}{\Gamma_\text{app}}}{e}{\tau'}$ \BY{IH, part 2(a) on \pfref{cvalidE}, \pfref{delta-disjoint} and \pfref{gamma-disjoint4}}\pflabel{hastype}
        \item $\hastypeU{\Dcons{\Delta}{\Delta_\text{app}}}{\Gcons{\Gcons{\Gamma}{\Gamma_\text{app}}}{\pctx'}}{e}{\tau'}$ \BY{exchange of $\pctx'$ and $\Gamma_\text{app}$ on \pfref{hastype}}\pflabel{hastype2}
        \item $\ruleType{\Dcons{\Delta}{\Delta_\text{app}}}{\Gcons{\Gamma}{\Gamma_\text{app}}}{\aematchrule{p}{e}}{\tau}{\tau'}$ \BY{Rule (\ref{rule:ruleType}) on \pfref{patType} and \pfref{hastype2}}
      \end{pfsteps*}
      \resetpfcounter
   \end{byCases} 
\end{enumerate}
\end{grayparbox}
\end{enumerate}
\vspace{10px}

The mutual induction can be shown to be well-founded by showing that the following numeric metric on the judgements that we induct on is decreasing:
\begin{align*}
\sizeof{\expandsSG{\uDelta}{\uGamma}{\uPsi}{\uPhi}{\ue}{e}{\tau}} & = \sizeof{\ue}\\
\sizeof{\cvalidE{\Delta}{\Gamma}{\esceneSG{\uDelta}{\uGamma}{\uPsi}{\uPhi}{b}}{\ce}{e}{\tau}} & = \sizeof{b}
\end{align*}
where $\sizeof{b}$ is the length of $b$ and $\sizeof{\ue}$ is the sum of the lengths of the seTSM literal bodies in $\ue$, as defined in Sec. \ref{appendix:SES-syntax}.

The only case in the proof of part 1 that invokes part 2 is Case (\ref{rule:expandsU-tsmap}). There, we have that the metric remains stable: \begin{align*}
 & \sizeof{\expandsSG{\uDelta}{\uGamma}{\uPsi}{\uPhi}{\utsmap{\tsmv}{b}}{e}{\tau}}\\
=& \sizeof{\cvalidE{\emptyset}{\emptyset}{\esceneSG{\uDelta}{\uGamma}{\uPsi}{\uPhi}{b}}{\ce}{e}{\tau}}\\
=&\sizeof{b}\end{align*}

The only case in the proof of part 2 that invokes part 1 is Case (\ref{rule:cvalidE-U-splicede}). There, we have that $\parseUExp{\bsubseq{b}{m}{n}}{\ue}$ and the IH is applied to the judgement $\expandsSG{\uDelta}{\uGamma}{\uPsi}{\uPhi}{\ue}{e}{\tau}$. Because the metric is stable when passing from part 1 to part 2, we must have that it is strictly decreasing in the other direction:
\[\sizeof{\expandsSG{\uDelta}{\uGamma}{\uPsi}{\uPhi}{\ue}{e}{\tau}} < \sizeof{\cvalidE{\Delta}{\Gamma}{\esceneSG{\uDelta}{\uGamma}{\uPsi}{\uPhi}{b}}{\acesplicede{m}{n}{\ctau}}{e}{\tau}}\]
i.e. by the definitions above, 
\[\sizeof{\ue} < \sizeof{b}\]

This is established by appeal to Condition \ref{condition:body-subsequences}, which states that subsequences of $b$ are no longer than $b$, and Condition \ref{condition:body-parsing}, which states that an unexpanded expression constructed by parsing a textual sequence $b$ is strictly smaller, as measured by the metric defined above, than the length of $b$, because some characters must necessarily be used to apply a TSM and delimit each literal body. 
Combining these conditions, we have that $\sizeof{\ue} < \sizeof{b}$ as needed.
\end{proof}

\begin{theorem}[Typed Expression Expansion]\label{thm:typed-expansion-short-U} If $\expandsSG{\uDD{\uD}{\Delta}}{\uGG{\uG}{\Gamma}\hspace{-3px}}{\uPsi}{\uPhi}{\ue}{e}{\tau}$ then $\hastypeU{\Delta}{\Gamma}{e}{\tau}$.
\end{theorem}
\begin{proof} This theorem follows immediately from Theorem \ref{thm:typed-expansion-full-U}, part 1(a). \end{proof}

% % \subsection{Expressibility}
% The following lemma establishes that each type can be expressed as a well-formed proto-type, under the same type formation context and any type splicing scene.
% \begin{lemma}[Proto-Expansion Type Expressibility]\label{lemma:proto-type-expressibility-U} If $\istypeU{\Delta}{\tau}$ then $\cvalidT{\Delta}{\tscenev}{\Cof{\tau}}{\tau}$. \end{lemma}
% \begin{proof}
% By rule induction over Rules (\ref{rules:istypeU}). In each case, we apply the IH on or over each premise, then apply the corresponding proto-type validation rule in Rules (\ref{rules:cvalidT-U}).
% \end{proof}

% The Type Expressibility Lemma establishes that every well-formed type, $\tau$, can be expressed as a well-formed unexpanded type, $\Uof{\tau}$. This requires defining the metafunction $\Uof{\Delta}$ which maps $\Delta$ onto an unexpanded type formation context as follows:
% \begin{align*}
% \Uof{\emptyset} &= \uDD{\emptyset}{\emptyset}\\
% \Uof{\Delta, \Dhyp{t}} &= \Uof{\Delta}, \uDhyp{\sigilof{t}}{t}
% \end{align*}
% \begin{lemma}[Type Expressibility]\label{lemma:type-expressibility} If $\istypeU{\Delta}{\tau}$ then $\expandsTU{\Uof{\Delta}}{\Uof{\tau}}{\tau}$.\end{lemma}
% \begin{proof} By rule induction over Rules (\ref{rules:istypeU}) using the definitions of $\Uof{\tau}$ and $\Uof{\Delta}$ above. In each case, we apply the IH to or over each premise, then apply the corresponding type expansion rule in Rules (\ref{rules:expandsTU}).\end{proof}


% The following lemma establishes that each well-typed expanded expression, $e$, can be expressed as a valid proto-expression, $\Cof{e}$, that is assigned the same type under any expression splicing scene.
% \begin{theorem}[Proto-Expansion Expression Expressibility]\label{theorem:proto-expressions-expressibility-U} If $\hastypeU{\Delta}{\Gamma}{e}{\tau}$ then $\cvalidE{\Delta}{\Gamma}{\escenev}{\Cof{e}}{e}{\tau}$.\end{theorem}
% \begin{proof} By rule induction over Rules (\ref{rules:hastypeU}). The rule transformation above guarantees that this lemma holds by construction. In particular, in each case, we apply Lemma \ref{lemma:proto-type-expressibility-U} to or over each type formation premise, the IH to or over each typing premise, then apply the corresponding proto-expression validation rule in Rules (\ref{rule:cvalidE-U-var}) through (\ref{rule:cvalidE-U-case}).
% \end{proof}

% The following lemma establishes that each well-typed expanded expression, $e$, can be expressed as a valid ce-expression, $\Cof{e}$, that is assigned the same type under any expression splicing scene.
% \begin{theorem}[Candidate Expansion Expression Expressibility]\label{lemma:ce-expressions-expressibility-UP} Both of the following hold:
% \begin{enumerate}
% \item If $\hastypeU{\Delta}{\Gamma}{e}{\tau}$ then $\cvalidE{\Delta}{\Gamma}{\escenev}{\Cof{e}}{e}{\tau}$.
% \item If $\ruleType{\Delta}{\Gamma}{r}{\tau}{\tau'}$ then $\cvalidR{\Delta}{\Gamma}{\escenev}{\Cof{r}}{r}{\tau}{\tau'}$.
% \end{enumerate}
% \end{theorem}
% \begin{proof} By mutual rule induction over Rules (\ref{rules:hastypeUP}) and Rule (\ref{rule:ruleType}). 

% For part 1, we induct on the assumption. 
% \begin{byCases}
% \item[\text{(\ref{rule:hastypeUP-var}) through (\ref{rule:hastypeUP-in})}] In each of these cases, we apply Lemma \ref{lemma:ce-type-expressibility-U} to or over each type formation premise, the IH (part 1) to or over each typing premise, then apply the corresponding ce-expression validation rule in Rules (\ref{rule:cvalidE-UP-var}) through (\ref{rule:cvalidE-UP-in}).
% \item[\text{(\ref{rule:hastypeUP-match})}] ~
%   \begin{pfsteps}
%   \item e = \aematchwith{n}{e'}{\seqschemaX{r}} \BY{assumption}
%   \item \Cof{e} = \acematchwith{n}{\Cof{\tau}}{\Cof{e'}}{\seqschemaXx{\Cofv}{r}} \BY{definition of $\Cof{e}$}
%   \item \hastypeU{\Delta}{\Gamma}{e'}{\tau'} \BY{assumption} \pflabel{hasType}
%   \item \istypeU{\Delta}{\tau} \BY{assumption} \pflabel{isType}
%   \item \{\ruleType{\Delta}{\Gamma}{r_i}{\tau'}{\tau}\}_{1 \leq i \leq n} \BY{assumption} \pflabel{ruleType}
%   \item \cvalidE{\Delta}{\Gamma}{\escenev}{\Cof{e'}}{e'}{\tau'} \BY{IH, part 1 on \pfref{hasType}} \pflabel{cvalidE}
%   \item \cvalidT{\Delta}{\tsfrom{\escenev}}{\Cof{\tau}}{\tau} \BY{Lemma \ref{lemma:candidate-expansion-type-validation} on \pfref{isType}} \pflabel{cvalidT}
%   \item \{\cvalidR{\Delta}{\Gamma}{\escenev}{\Cof{r_i}}{r_i}{\tau'}{\tau}\}_{1 \leq i \leq n} \BY{IH, part 2 over \pfref{ruleType}} \pflabel{cvalidR}
%   \item \cvalidE{\Delta}{\Gamma}{\escenev}{\acematchwith{n}{\Cof{\tau}}{\Cof{e'}}{\seqschemaXx{\Cofv}{r}}}{\aematchwith{n}{e'}{\seqschemaX{r}}}{\tau} \BY{Rule (\ref{rule:cvalidE-UP-match}) on \pfref{cvalidE}, \pfref{cvalidT} and \pfref{cvalidR}}
%   \end{pfsteps}
% \end{byCases}
% \resetpfcounter

% For part 2, we induct on the assumption. There is only one case.
% \begin{byCases}
% \item[\text{(\ref{rule:ruleType})}] ~
%   \begin{pfsteps}
%     \item r = \aematchrule{p}{e} \BY{assumption}
%     \item \Cof{r} = \acematchrule{p}{\Cof{e}} \BY{definition of $\Cof{r}$}
%     \item \patType{\pctx}{p}{\tau} \BY{assumption} \pflabel{patType}
%     \item \hastypeU{\Delta}{\Gcons{\Gamma}{\pctx}}{e}{\tau'} \BY{assumption} \pflabel{hasType}
%     \item \cvalidE{\Delta}{\Gcons{\Gamma}{\pctx}}{\escenev}{\Cof{e}}{e}{\tau'} \BY{IH, part 1 on \pfref{hasType}} \pflabel{cvalidE}
%     \item \cvalidR{\Delta}{\Gamma}{\escenev}{\acematchrule{p}{\Cof{e}}}{\aematchrule{p}{e}}{\tau}{\tau'} \BY{Rule (\ref{rule:cvalidR-UP}) on \pfref{patType} and \pfref{cvalidE}}
%   \end{pfsteps}
%   \resetpfcounter
% \end{byCases}
% \end{proof}

% The following lemma establishes that every well-typed expanded pattern that generates no hypotheses can be expressed as a ce-pattern.
% \begin{lemma}[Candidate Expansion Pattern Expressibility]\label{lemma:ce-pattern-expressibility-U} If $\patType{\emptyset}{p}{\tau}$ then $\cvalidP{\uGG{\emptyset}{\emptyset}}{\pscene{\uDelta}{\uPhi}{b}}{\Cof{p}}{p}{\tau}$.\end{lemma}
% \begin{proof} By rule induction over Rules (\ref{rules:patType}).
% \begin{byCases}
% \item[\text{(\ref{rule:patType-var})}] This case does not apply.
% \item[\text{(\ref{rule:patType-wild})}] ~
%   \begin{pfsteps*}
%     \item $p=\aewildp$ \BY{assumption}
%     \item $\Cof{p}=\acewildp$ \BY{definition of $\Cof{p}$}
%     \item $\cvalidP{\uGG{\emptyset}{\emptyset}}{\pscene{\uDelta}{\uPhi}{b}}{\acewildp}{\aewildp}{\tau}$ \BY{Rule (\ref{rule:cvalidP-UP-wild})}
%   \end{pfsteps*}
%   \resetpfcounter
% \item[\text{(\ref{rule:patType-fold})}] ~
%   \begin{pfsteps*}
%     \item $p=\aefoldp{p'}$ \BY{assumption}
%     \item $\Cof{p}=\acefoldp{\Cof{p'}}$ \BY{definition of $\Cof{p}$}
%     \item $\tau=\arec{t}{\tau'}$ \BY{assumption}
%     \item $\patType{\emptyset}{p'}{[\arec{t}{\tau'}/t]\tau'}$ \BY{assumption} \pflabel{patType}
%     \item $\cvalidP{\uGG{\emptyset}{\emptyset}}{\pscene{\uDelta}{\uPhi}{b}}{\Cof{p'}}{p}{[\arec{t}{\tau'}/t]\tau'}$ \BY{IH on \pfref{patType}} \pflabel{cvalidP}
%     \item $\cvalidP{\uGG{\emptyset}{\emptyset}}{\pscene{\uDelta}{\uPhi}{b}}{\acefoldp{\Cof{p'}}}{\aefoldp{p'}}{\arec{t}{\tau'}}$ \BY{Rule (\ref{rule:cvalidP-UP-fold}) on \pfref{cvalidP}}
%   \end{pfsteps*}
%   \resetpfcounter
% \item[\text{(\ref{rule:patType-tpl})}] ~
%   \begin{pfsteps*}
%     \item $p=\aetplp{\labelset}{\mapschema{p}{i}{\labelset}}$ \BY{assumption}
%     \item $\Cof{p}=\acetpl{\labelset}{\mapschemax{\Cofv}{p}{i}{\labelset}}$ \BY{definition of $\Cof{p}$}
%     \item $\tau=\aprod{\labelset}{\mapschema{\tau}{i}{\labelset}}$ \BY{assumption}
%     \item $\{\patType{\emptyset}{p_i}{\tau_i}\}_{i \in \labelset}$ \BY{assumption} \pflabel{patType}
%     \item $\{\cvalidP{\uGG{\emptyset}{\emptyset}}{\pscene{\uDelta}{\uPhi}{b}}{\Cof{p_i}}{p_i}{\tau_i}\}_{i \in \labelset}$ \BY{IH over \pfref{patType}} \pflabel{cvalidP}
%     \item $\cvalidP{\uGG{\emptyset}{\emptyset}}{\pscene{\uDelta}{\uPhi}{b}}{\acetpl{\labelset}{\mapschemax{\Cofv}{p}{i}{\labelset}}}{\aetplp{\labelset}{\mapschema{p}{i}{\labelset}}}{\aprod{\labelset}{\mapschema{\tau}{i}{\labelset}}}$ \BY{Rule (\ref{rule:cvalidP-UP-tpl}) on \pfref{cvalidP}}
%   \end{pfsteps*}
%   \resetpfcounter
% \item[\text{(\ref{rule:patType-inj})}] ~
%   \begin{pfsteps*}
%     \item $p=\aeinjp{\ell}{p'}$ \BY{assumption}
%     \item $\Cof{p}=\aceinjp{\ell}{\Cof{p'}}$ \BY{definition of $\Cof{p}$}
%     \item $\tau=\asum{\labelset, \ell}{\mapschema{\tau}{i}{\labelset}; \mapitem{\ell}{\tau'}}$ \BY{assumption}
%     \item $\patType{\emptyset}{p'}{\tau'}$ \BY{assumption}\pflabel{patType}
%     \item $\cvalidP{\uGG{\emptyset}{\emptyset}}{\pscene{\uDelta}{\uPhi}{b}}{\Cof{p'}}{p'}{\tau'}$ \BY{IH on \pfref{patType}}\pflabel{cvalidP}
%     \item $\cvalidP{\uGG{\emptyset}{\emptyset}}{\pscene{\uDelta}{\uPhi}{b}}{\aceinjp{\ell}{\Cof{p'}}}{\aeinjp{\ell}{p'}}{\asum{\labelset, \ell}{\mapschema{\tau}{i}{\labelset}; \mapitem{\ell}{\tau'}}}$ \BY{Rule (\ref{rule:cvalidP-UP-in}) on \pfref{cvalidP}}
%   \end{pfsteps*}
%   \resetpfcounter
% \end{byCases}
% \end{proof}

% \subsubsection{Expressibility}
% The following lemma establishes that each well-typed expanded pattern can be expressed as an unexpanded pattern matching values of the same type and generating the same hypotheses and corresponding identifier updates. The metafunction $\Uof{\pctx}$ maps $\pctx$ to an unexpanded typing context as follows:
% \begin{align*}
% \Uof{\emptyset} & = \uGG{\emptyset}{\emptyset}\\
% \Uof{\pctx, x : \tau} & = \Uof{\pctx}, \uGhyp{\sigilof{x}}{x}{\tau}\\
% \Uof{\Gconsi{i \in \labelset}{\pctx_i}} & = \Gconsi{i \in \labelset}{\Uof{\pctx_i}}
% \end{align*}
% \begin{lemma}[Pattern Expressibility]\label{lemma:pattern-expressibility} If $\patType{\pctx}{p}{\tau}$ then $\patExpands{\Uof{\pctx}}{\uPhi}{\Uof{p}}{p}{\tau}$.\end{lemma}
% \begin{proof} By rule induction over Rules (\ref{rules:patType}), using the definitions of $\Uof{\pctx}$ and $\Uof{p}$ given above. In each case, we can apply the IH to or over each premise, then apply the corresponding rule in Rules (\ref{rules:patExpands}).\end{proof}

% We can now establish the Expressibility Theorem -- that each well-typed expanded expression, $e$, can be expressed as an unexpanded expression, $\ue$, and assigned the same type under the corresponding contexts.

% \begin{theorem}[Expressibility] Both of the following hold:
% \begin{enumerate}
% \item If $\hastypeU{\Delta}{\Gamma}{e}{\tau}$ then $\expandsUP{\Uof{\Delta}}{\Uof{\Gamma}}{\uPsi}{\uPhi}{\Uof{e}}{e}{\tau}$.
% \item If $\ruleType{\Delta}{\Gamma}{r}{\tau}{\tau'}$ then $\ruleExpands{\Uof{\Delta}}{\Uof{\Gamma}}{\uPsi}{\uPhi}{\Uof{r}}{r}{\tau}{\tau'}$.
% \end{enumerate}
% \end{theorem}
% \begin{proof} By mutual rule induction over Rules (\ref{rules:hastypeUP}) and Rule (\ref{rule:ruleType}). 

% For part 1, we induct on the assumption. The rule transformation defined above guarantees that this part holds by its construction. In particular, in each case, we can apply Lemma \ref{lemma:type-expressibility} to or over each type formation premise, the IH (part 1) to or over each typing premise, the IH (part 2) over each rule typing premise, then apply the corresponding rule in Rules (\ref{rules:expandsUP}).

% For part 2, we induct on the assumption. There is only one case:
% \begin{byCases}
% \item[(\ref{rule:ruleType})] ~
% \begin{pfsteps*}
% \item $r = \aematchrule{p}{e}$ \BY{assumption}
% \item $\patType{\pctx}{p}{\tau}$ \BY{assumption} \pflabel{patType}
% \item $\hastypeU{\Delta}{\Gamma \cup \pctx}{e}{\tau'}$ \BY{assumption} \pflabel{hasType}
% \item $\Uof{\Gamma}=\uGG{\uG}{\Gamma}$, for some $\uG$ \BY{definition of $\Uof{\Gamma}$}
% \item $\Uof{\pctx} =\uGG{\uG'}{\pctx}$, for some $\uG'$ \BY{definition of $\Uof{\pctx}$}
% \item $\Uof{\Gamma \cup \pctx} = \uGG{\uG \uplus \uG'}{\Gamma \cup \pctx}$ \BY{definition of $\Uof{\pctx}$}
% \item $\Uof{r} = \aumatchrule{\Uof{p}}{\Uof{e}}$ \BY{definition of $\Uof{r}$}
% \item $\patExpands{\uGG{\uG'}{\pctx}}{\uPhi}{\Uof{p}}{p}{\tau}$ \BY{Lemma \ref{lemma:pattern-expressibility} on \pfref{patType}} \pflabel{patExpands}
% \item $\expandsUP{\uDelta}{\uGG{\uGcons{\uG}{\uG'}}{\Gcons{\Gamma}{\pctx}}}{\uPsi}{\uPhi}{\Uof{e}}{e}{\tau'}$ \BY{IH, part 1 on \pfref{hasType}} \pflabel{expandsUP}
% \item $\ruleExpands{\Uof{\Delta}}{\uGG{\uG}{\Gamma}}{\uPsi}{\uPhi}{\aumatchrule{\Uof{p}}{\Uof{e}}}{\aematchrule{p}{e}}{\tau}{\tau'}$ \BY{Rule (\ref{rule:ruleExpands}) on \pfref{patExpands} and \pfref{expandsUP}}
% \end{pfsteps*}
% \resetpfcounter
% \end{byCases}
% \end{proof}

\subsection{Reasoning Principles}\label{appendix:SES-reasoning-principles}
\begin{lemma}[Proto-Type Expansion Decomposition] 
\label{thm:proto-type-expansion-decomposition-SES}
If $\cvalidT{\Delta}{\tsceneU{\uDD{\uD}{\Delta_\text{app}}}{b}}{\ctau}{\tau}$ and $\summaryOf{\ctau} = \sseq{\acesplicedt{m_i}{n_i}}{n}$ then all of the following hold:
\begin{enumerate}
\item $\sseq{\expandsTU{\uDD{\uD}{\Delta_\text{app}}}{
  \parseUTypF{\bsubseq{b}{m_i}{n_i}}
}{\tau_i}}{n}$
% \item $\sseq{\istypeU{\Delta_\text{app}}{\tau_i}}{n}$
\item $\tau = [\sseq{\tau_i/t_i}{n}]\tau'$ for some $\sseq{t_i}{n}$ and $\tau'$
\item $\istypeU{\Delta \cup \sseq{\Dhyp{t_i}}{n}}{\tau'}$
\end{enumerate}
\end{lemma}
\begin{proof}
By rule induction over Rules (\ref{rules:cvalidT-U}). In the following, let $\uDelta = \uDD{\uD}{\Delta_\text{app}}$ and $\tscenev=\tsceneU{\uDelta}{b}$.
\begin{byCases}
  \item[\text{(\ref{rule:cvalidT-U-tvar})}] 
    \begin{pfsteps}
    \item \ctau = t \BY{assumption}
    \item \tau = t \BY{assumption}
    \item \Delta = \Delta', \Dhyp{t} \BY{assumption}
    \item \summaryOf{\ctau} = \{ \} \BY{definition}
    \end{pfsteps}
    \resetpfcounter
    The conclusions hold as follows:
    \begin{enumerate}
    \item This conclusion holds trivially because $n=0$.
    % \item This conclusion holds trivially because $n=0$.
    \item Choose the empty set and $\tau'=t$.
    \item $\istypeU{\Delta', \Dhyp{t}}{t}$ by Rule (\ref{rule:istypeU-var})
    \end{enumerate}
  \item[\text{(\ref{rule:cvalidT-U-parr})}] 
    \begin{pfsteps}
    \item \ctau = \aceparr{\ctau_1}{\ctau_2} \BY{assumption}
    \item \tau = \aparr{\tau'_1}{\tau'_2} \BY{assumption}
    \item \cvalidT{\Delta}{\tscenev}{\ctau_1}{\tau_1} \BY{assumption} \pflabel{cvalidT1}
    \item \cvalidT{\Delta}{\tscenev}{\ctau_2}{\tau_2} \BY{assumption} \pflabel{cvalidT2}
    \item \summaryOf{\ctau} = \summaryOf{\ctau_1} \cup \summaryOf{\ctau_2} \BY{definition} \pflabel{summaryOf}
    \item \summaryOf{\ctau_1} = \sseqS{\acesplicedt{m_{i}}{n_{i}}}{0}{n'} \BY{definition} \pflabel{summaryOf1}
    \item \summaryOf{\ctau_2} = \sseqS{\acesplicedt{m_{i}}{n_{i}}}{n'}{n} \BY{definition} \pflabel{summaryOf2}
    \item \sseq{\expandsTU{\uDD{\uD}{\Delta_\text{app}}}{
  \parseUTypF{\bsubseq{b}{m_i}{n_i}}
}{\tau_i}}{n'} 
    \BY{IH on \pfref{cvalidT1} and \pfref{summaryOf1}}
    \pflabel{expands1}
    % \item \sseq{\istypeU{\Delta_\text{app}}{\tau_i}}{n}
    \item \tau'_1 = [\sseq{\tau_i/t_i}{n'}]\tau''_1 \text{~for some $\sseq{t_i}{n'}$ and $\tau''_1$}
        \BY{IH on \pfref{cvalidT1} and \pfref{summaryOf1}}
        \pflabel{decompose1}
    \item \istypeU{\Delta \cup \sseq{\Dhyp{t_i}}{n'}}{\tau''_1}
        \BY{IH on \pfref{cvalidT1} and \pfref{summaryOf1}}
        \pflabel{istype1}
    \item \sseqS{\expandsTU{\uDD{\uD}{\Delta_\text{app}}}{
  \parseUTypF{\bsubseq{b}{m_i}{n_i}}
}{\tau_i}}{n'}{n} 
    \BY{IH on \pfref{cvalidT2} and \pfref{summaryOf2}}
    \pflabel{expands2}
    % \item \sseq{\istypeU{\Delta_\text{app}}{\tau_i}}{n}
    \item \tau'_2 = [\sseqS{\tau_i/t_i}{n'}{n}]\tau''_2 \text{~for some $\sseqS{t_i}{n'}{n}$ and $\tau''_2$}
        \BY{IH on \pfref{cvalidT2} and \pfref{summaryOf2}}
        \pflabel{decompose2}
    \item \istypeU{\Delta \cup \sseqS{\Dhyp{t_i}}{n'}{n}}{\tau''_2}
        \BY{IH on \pfref{cvalidT2} and \pfref{summaryOf2}}
        \pflabel{istype2}
    \item \tau_1' = [\sseq{\tau_i/t_i}{n}]\tau_1'' \BY{substitution properties and \pfref{decompose1}} \pflabel{fdecompose1}
    \item \tau_2' = [\sseq{\tau_i/t_i}{n}]\tau_2'' \BY{substitution properties and \pfref{decompose2}} \pflabel{fdecompose2}
    \item \aparr{\tau_1'}{\tau_2'} = [\sseq{\tau_i/t_i}{n}]\aparr{\tau_1''}{\tau_2''} \BY{substitution and \pfref{fdecompose1} and \pfref{fdecompose2}}
    \pflabel{finaldecompose}
    \item \istypeU{\Delta \cup \sseq{\Dhyp{t_i}}{n}}{\tau''_1} \BY{Weakening and \pfref{istype1}} \pflabel{istype3}
    \item \istypeU{\Delta \cup \sseq{\Dhyp{t_i}}{n}}{\tau''_2} \BY{Weakening and \pfref{istype2}} \pflabel{istype4}
    \item \istypeU{\Delta \cup \sseq{\Dhyp{t_i}}{n}}{\aparr{\tau''_1}{\tau''_2}} \BY{Rule (\ref{rule:istypeU-parr}) on \pfref{istype3} and \pfref{istype4}}
    \pflabel{finalistype}
    \end{pfsteps}
    The conclusions hold as follows:
    \begin{enumerate}
      \item \pfref{expands1} $\cup$ \pfref{expands2}
      \item Choosing $\sseq{t_i}{n}$ and $\aparr{\tau''_1}{\tau''_2}$, by \pfref{finaldecompose}.
      \item \pfref{finalistype}
    \end{enumerate}
    \resetpfcounter
  \item[\text{(\ref{rule:cvalidT-U-all}) \textbf{through} (\ref{rule:cvalidT-U-sum})}] These cases follow by analagous inductive argument.
  \item[\text{(\ref{rule:cvalidT-U-splicedt})}] ~
  \begin{pfsteps}
  \item \ctau = \acesplicedt{m}{n} \BY{assumption}
  \item \summaryOf{\acesplicedt{m}{n}} = \{ \acesplicedt{m}{n} \} \BY{definition}
  \item \parseUTyp{\bsubseq{b}{m}{n}}{\utau} \BY{assumption} \pflabel{parseUTyp}
  \item \expandsTU{\uDD{\uD}{\Delta_\text{app}}}{\utau}{\tau} \BY{assumption} \pflabel{expandsTU}
  \item \istypeU{\Delta, \Dhyp{t}}{t} \BY{Rule (\ref{rule:istypeU-var})} \pflabel{istype}
  \end{pfsteps}
  The conclusions hold as follows:
  \begin{enumerate}
    \item \pfref{parseUTyp} and \pfref{expandsTU}
    \item Choose $\{t\}$ and $t$. Then $\tau = [\tau/t]t$ by definition.
    \item \pfref{istype}
  \end{enumerate}
  \resetpfcounter
\end{byCases}
\end{proof}

\begin{lemma}[Proto-Expression \graytxtbox{and Proto-Rule} Expansion Decomposition] 
\label{thm:proto-expression-expansion-decomposition} ~
\begin{enumerate}
\item If $\cvalidE
  {\Delta}{\Gamma}
  {\esceneSG
    {\uDD{\uD}{\Delta_\text{app}}}
    {\uGG{\uG}{\Gamma_\text{app}}}
    {\uPsi}{\uPhi}{b}
  }{\ce}{e}{\tau}$ and \[\summaryOf{\ce} = \sseq{\acesplicedt{m'_i}{n'_i}}{\nty} \cup \sseq{\acesplicede{m_i}{n_i}{\ctau_i}}{\nexp}\] then all of the following hold:
  \begin{enumerate}
    \item $\sseq{
          \expandsTU{\uDD{\uD}{\Delta_\text{app}}}
          {
            \parseUTypF{\bsubseq{b}{m'_i}{n'_i}}
          }{\tau'_i}
        }{\nty}$
    \item $\sseq{
      \cvalidT{\emptyset}{
        \tsceneUP
          {\uDD
            {\uD}{\Delta_\text{app}}
          }{b}
      }{
        \ctau_i
      }{\tau_i}
    }{\nexp}$
    \item $\sseq{
      \expandsSG
        {\uDD{\uD}{\Delta_\text{app}}}
        {\uGG{\uG}{\Gamma_\text{app}}}
        {\uPsi}
        {\uPhi}
        {\parseUExpF{\bsubseq{b}{m_i}{n_i}}}
        {e_i}
        {\tau_i}
    }{\nexp}$
    \item $e = [\sseq{\tau'_i/t_i}{\nty}][\sseq{e_i/x_i}{\nexp}]e'$ for some $\sseq{t_i}{\nty}$ and $\sseq{x_i}{\nexp}$ and $e'$
    \item $\hastypeU
      {\Delta \cup \sseq{\Dhyp{t_i}}{\nty}}
      {\Gamma \cup \sseq{x_i : \tau_i}{\nexp}}
      {e'}{\tau}$
  \end{enumerate}
\end{enumerate}
\begin{grayparbox}
\begin{enumerate}
\item[2.] If $\cvalidR{\Delta}{\Gamma}{\esceneUP{\uDD{\uD}{\Delta_\text{app}}}{\uGG{\uG}{\Gamma_\text{app}}}{\uPsi}{\uPhi}{b}}{\crv}{r}{\tau}{\tau'}$ and \[\summaryOf{\crv} = \sseq{\acesplicedt{m'_i}{n'_i}}{\nty} \cup \sseq{\acesplicede{m_i}{n_i}{\ctau_i}}{\nexp}\] then all of the following hold:
  \begin{enumerate}
    \item $\sseq{
          \expandsTU{\uDD{\uD}{\Delta_\text{app}}}
          {
            \parseUTypF{\bsubseq{b}{m'_i}{n'_i}}
          }{\tau'_i}
        }{\nty}$
    \item $\sseq{
      \cvalidT{\emptyset}{
        \tsceneUP
          {\uDD
            {\uD}{\Delta_\text{app}}
          }{b}
      }{
        \ctau_i
      }{\tau_i}
    }{\nexp}$
    \item $\sseq{
      \expandsSG
        {\uDD{\uD}{\Delta_\text{app}}}
        {\uGG{\uG}{\Gamma_\text{app}}}
        {\uPsi}
        {\uPhi}
        {\parseUExpF{\bsubseq{b}{m_i}{n_i}}}
        {e_i}
        {\tau_i}
    }{\nexp}$
    \item $r = [\sseq{\tau'_i/t_i}{\nty}][\sseq{e_i/x_i}{\nexp}]r'$ for some $\sseq{t_i}{\nty}$ and $\sseq{x_i}{\nexp}$ and $e'$
    \item $\ruleType
      {\Delta \cup \sseq{\Dhyp{t_i}}{\nty}}
      {\Gamma \cup \sseq{x_i : \tau_i}{\nexp}}
      {r'}{\tau}{\tau'}$
  \end{enumerate}
\end{enumerate}
\end{grayparbox}
\end{lemma}
\begin{proof} By rule induction over Rules (\ref{rules:cvalidE-U}) and Rule (\ref{rule:cvalidR-UP}). In the following, let $\uDelta=\uDD{\uD}{\Delta_\text{app}}$ and $\uGamma=\uGG{\uG}{\Gamma_\text{app}}$ and $\escenev=\esceneUP{\uDelta}{\uGamma}{\uPsi}{\uPhi}{b}$.
\begin{enumerate}
\item \begin{byCases}
  \item[\text{(\ref{rule:cvalidE-U-var})}] \begin{pfsteps}
    \item \ce=x \BY{assumption}
    \item e=x \BY{assumption}
    \item \Gamma = \Gamma', x : \tau \BY{assumption}
    \item \summaryOf{x}=\{ \} \BY{definition}
  \end{pfsteps}
  The conclusions hold as follows:
  \begin{enumerate}
    \item This conclusion holds trivially because $\nty=0$.
    \item This conclusion holds trivially because $\nexp=0$.
    \item This conclusion holds trivially because $\nexp=0$.
    \item Choose the empty set, the empty set and $x$. 
    \item $\hastypeU{\Delta}{\Gamma', x : \tau}{x}{\tau}$ by Rule (\ref{rule:hastypeU-var})
  \end{enumerate}
  \resetpfcounter
  \item[\text{(\ref{rule:cvalidE-U-asc}) \textbf{through} (\ref{rule:cvalidE-U-case})}] These cases follow by straightforward inductive argument.
  \item[\text{(\ref{rule:cvalidE-U-splicede})}] \begin{pfsteps}
    \item \ce=\acesplicede{m}{n}{\ctau} \BY{assumption}
    \item \summaryOf{\acesplicede{m}{n}{\ctau}} = \summaryOf{\ctau} \cup \{ \acesplicede{m}{n}{\ctau} \} \BY{definition}
    \item \summaryOf{\ctau} = \sseq{\acesplicedt{m'_i}{n'_i}}{\nty} \BY{definition} \pflabel{ctau}
    \item \cvalidT{\emptyset}{\tsfrom{\escenev}}{\ctau}{\tau} \BY{assumption} \pflabel{cvalidT}
    \item \parseUExp{\bsubseq{b}{m}{n}}{\ue} \BY{assumption} \pflabel{parseUExp}
    \item \expandsSG{\uDD{\uD}{\Delta_\text{app}}}{\uGG{\uG}{\Gamma_\text{app}}}{\uPsi}{\uPhi}{\ue}{e}{\tau}
    \BY{assumption} \pflabel{expandsSG}
    \item \sseq{\expandsTU{\uDD{\uD}{\Delta_\text{app}}}{
  \parseUTypF{\bsubseq{b}{m'_i}{n'_i}}
}{\tau'_i}}{\nty}
% \item $\sseq{\istypeU{\Delta_\text{app}}{\tau_i}}{n}$
   \BY{Lemma \ref{thm:proto-type-expansion-decomposition-SES} on \pfref{cvalidT} and \pfref{ctau}} \pflabel{parseUTyp}
   \item \hastypeU{\Delta \cup \sseq{\Dhyp{t_i}}{n_ty}}{\Gamma, x : \tau}{x}{\tau} \BY{Rule (\ref{rule:hastypeU-var})} \pflabel{hastypeU}
  \end{pfsteps}
  The conclusions hold as follows:
  \begin{enumerate}
    \item \pfref{parseUTyp}
    \item \pfref{cvalidT}
    \item \pfref{parseUExp} and \pfref{expandsSG}
    \item Choosing $\sseq{t_i}{\nty}$, $\{x\}$ and $x$, we have $e = [\sseq{\tau'_i/t_i}{\nty}][e/x]x$ by definition.
    \item \pfref{hastypeU}
  \end{enumerate}
  \resetpfcounter
\end{byCases}
\begin{grayparbox}
\begin{byCases}
  \item[\text{(\ref{rule:cvalidE-U-match})}] \begin{pfsteps*}
    \item $\ce = \acematchwith{\nrules}{\ce'}{\seqschemaX{\crv}}$ \BY{assumption}
    \item $e = \aematchwith{\nrules}{\tau}{e'}{\seqschemaX{r}}$ \BY{assumption}
    % \item $\istypeU{\Delta \cup \Delta_\text{app}}{\tau'}$ \BY{assumption} \pflabel{istype}
    \item $\cvalidE{\Delta}{\Gamma}{\escenev}{\ce}{e}{\tau'}$ \BY{assumption} \pflabel{cvalidE}
    \item $\{\cvalidR{\Delta}{\Gamma}{\escenev}{\crv_j}{r_j}{\tau'}{\tau}\}_{1 \leq j \leq \nrules}$
    \BY{assumption} \pflabel{cvalidR}
    \item $\summaryOf{\acematchwith{\nrules}{\ce'}{\seqschemaX{\crv}}} = \summaryOf{\ce} \cup \bigcup_{0 \leq i < \nrules} \summaryOf{\crv_i}$ \BY{definition} 
    \item $\summaryOf{\ce'} = \sseq{\acesplicedt{m'_i}{n'_i}}{\nty'} \cup \sseq{\acesplicede{m_i}{n_i}{\ctau_i}}{\nexp'}$ \BY{definition} \pflabel{summaryOfscrut}
    \item $\{\summaryOf{\crv_j} = \sseq{\acesplicedt{m'_{i,j}}{n'_{i,j}}}{\ntyj} \cup \sseq{\acesplicede{m_{i,j}}{n_{i,j}}{\ctau_{i,j}}}{\nexpj}\}_{0 \leq j < \nrules}$ \BY{definition} \pflabel{summaryOfr}
    \item $\sseq{
              \expandsTU{\uDD{\uD}{\Delta_\text{app}}}
              {
                \parseUTypF{\bsubseq{b}{m'_i}{n'_i}}
              }{\tau'_i}
            }{\nty'}$
        \BY{IH, part 1 on \pfref{cvalidE} and \pfref{summaryOfscrut}} \pflabel{c1scrut}
    \item $\sseq{
          \cvalidT{\emptyset}{
            \tsceneUP
              {\uDD
                {\uD}{\Delta_\text{app}}
              }{b}
          }{
            \ctau_i
          }{\tau_i}
        }{\nexp'}$
            \BY{IH, part 1 on \pfref{cvalidE} and \pfref{summaryOfscrut}} \pflabel{c2scrut}
    \item $\sseq{
          \expandsSG
            {\uDD{\uD}{\Delta_\text{app}}}
            {\uGG{\uG}{\Gamma_\text{app}}}
            {\uPsi}
            {\uPhi}
            {\parseUExpF{\bsubseq{b}{m_i}{n_i}}}
            {e_i}
            {\tau_i}
        }{\nexp'}$
            \BY{IH, part 1 on \pfref{cvalidE} and \pfref{summaryOfscrut}} \pflabel{c3scrut}
    \item $e' = [\sseq{\tau'_i/t_i}{\nty'}][\sseq{e_i/x_i}{\nexp'}]e''$ for some $\sseq{t_i}{\nty'}$ and $\sseq{x_i}{\nexp'}$ and $e''$
        \BY{IH, part 1 on \pfref{cvalidE} and \pfref{summaryOfscrut}} \pflabel{c4scrut}
    \item $\hastypeU
          {\Delta \cup \sseq{\Dhyp{t_i}}{\nty'}}
          {\Gamma \cup \sseq{x_i : \tau_i}{\nexp'}}
          {e''}{\tau'}$
              \BY{IH, part 1 on \pfref{cvalidE} and \pfref{summaryOfscrut}} \pflabel{c5scrut}
    \item $\{\sseq{
              \expandsTU{\uDD{\uD}{\Delta_\text{app}}}
              {
                \parseUTypF{\bsubseq{b}{m'_{i,j}}{n'_{i,j}}}
              }{\tau'_{i,j}}
            }{\ntyj}\}_{0 \leq j < \nrules}$
        \BY{IH, part 2 over \pfref{cvalidR} and \pfref{summaryOfr}} \pflabel{c1r}
    \item $\{\sseq{
          \cvalidT{\emptyset}{
            \tsceneUP
              {\uDD
                {\uD}{\Delta_\text{app}}
              }{b}
          }{
            \ctau_{i,j}
          }{\tau_{i,j}}
        }{\nexpj}\}_{0 \leq j < \nrules}$
            \BY{IH, part 2 over \pfref{cvalidR} and \pfref{summaryOfr}} \pflabel{c2r}
    \item $\{\sseq{
          \expandsSG
            {\uDD{\uD}{\Delta_\text{app}}}
            {\uGG{\uG}{\Gamma_\text{app}}}
            {\uPsi}
            {\uPhi}
            {\parseUExpF{\bsubseq{b}{m_{i,j}}{n_{i,j}}}}
            {e_{i,j}}
            {\tau_{i,j}}
        }{\nexpj}\}_{0 \leq j < \nrules}$
            \BY{IH, part 2 over \pfref{cvalidR} and \pfref{summaryOfr}} \pflabel{c3r}
    \item $\{r_j = [\sseq{\tau'_{i,j}/t_{i,j}}{\ntyj}][\sseq{e_{i,j}/x_{i,j}}{\nexpj}]r_j'\}_{0 \leq j < \nrules}$ for some $\{\sseq{t_{i,j}}{\ntyj}\}_{0 \leq j < \nrules}$ and $\{\sseq{x_{i,j}}{\nexpj}\}_{0 \leq j < \nrules}$  and $\{r_j'\}_{0 \leq j < \nrules}$ 
        \BY{IH, part 2 over \pfref{cvalidR} and \pfref{summaryOfr}} \pflabel{c4r}
    \item $\{\ruleType
          {\Delta \cup \sseq{\Dhyp{t_{i,j}}}{\ntyj}}
          {\Gamma \cup \sseq{x_{i,j} : \tau_{i,j}}{\nexpj}}
          {r_j'}{\tau'}{\tau}\}_{0 \leq j < \nrules}$
              \BY{IH, part 2 over \pfref{cvalidR} and \pfref{summaryOfr}} \pflabel{c5r}
    \item $e' = [\sseq{\tau'_i/t_i}{\nty'} \cup \{ \sseq{\tau_{i,j}/t_{i,j}}{\ntyj} \}_{0 \leq j < \nrules}][\sseq{e_i/x_i}{\nexp} \cup \{ \sseq{\tau_{i,j}/t_{i,j}}{\ntyj} \}_{0 \leq j < \nrules}]e''$ \BY{substitution properties and \pfref{c4scrut}} \pflabel{c4fscrut}
    \item $\{r_j = [\sseq{\tau'_i/t_i}{\nty'} \cup \sseq{\tau'_{i,j}/t_{i,j}}{\ntyj}][\sseq{e_i/x_i}{\nexp} \cup \sseq{e_{i,j}/x_{i,j}}{\nexpj}]r_j'\}_{0 \leq j < \nrules}$ \BY{substitution properties and \pfref{c4r}} \pflabel{c4fr}
    \item $e = [\sseq{\tau'_i/t_i}{\nty'} \cup \{ \sseq{\tau_{i,j}/t_{i,j}}{\ntyj} \}_{0 \leq j < \nrules}][\sseq{e_i/x_i}{\nexp'} \cup \{ \sseq{e_{i,j}/x_{i,j}}{\nexpj} \}_{0 \leq j < \nrules}]\aematchwith{\nrules}{e''}{\seqschemaX{r'}}$ \BY{\pfref{c4fscrut} and \pfref{c4fr} and definition of substitution} \pflabel{c4e}
    \item $\hastypeU
          {\Delta \cup \sseq{\Dhyp{t_i}}{\nty'} \cup \{ \sseq{t_{i,j}}{\ntyj} \}_{0 \leq j < \nrules}}
          {\Gamma \cup \sseq{x_i : \tau_i}{\nexp'} \cup \{ \sseq{x_{i,j}}{\nexpj} \}_{0 \leq j < \nrules}}
          {e''}{\tau'}$ \BY{Weakening on \pfref{c5scrut}} \pflabel{c5scrutf}
    \item $\{\ruleType
          {\Delta \cup \sseq{\Dhyp{t_i}}{\nty'} \cup \{ \sseq{t_{i,j}}{\ntyj} \}_{0 \leq j < \nrules}}
          {\Gamma \cup \sseq{x_{i,j} : \tau_{i,j}}{\nexpj} \cup \{ \sseq{x_{i,j}}{\nexpj} \}_{0 \leq j < \nrules}}
          {r_j'}{\tau'}{\tau}\}_{0 \leq j < \nrules}$ \BY{Weakening over \pfref{c5r}} \pflabel{c5rf}
    \item $\hastypeU
              {\Delta \cup \sseq{\Dhyp{t_i}}{\nty'} \cup \{ \sseq{t_{i,j}}{\ntyj} \}_{0 \leq j < \nrules}}
              {\Gamma \cup \sseq{x_i : \tau_i}{\nexp'} \cup \{ \sseq{x_{i,j}}{\nexpj} \}_{0 \leq j < \nrules}}
              {\aematchwith{\nrules}{e''}{\seqschemaX{r'}}}{\tau}$ \BY{Rule (\ref{rule:hastypeUP-match}) on \pfref{c5scrutf} and \pfref{c5rf}} \pflabel{c5ff}
  \end{pfsteps*} 

  The conclusions hold as follows:
  \begin{enumerate}
    \item $\text{\pfref{c1scrut}} \cup \bigcup_{0 \leq j < \nrules} \text{\pfref{c1r}}_j$
    \item $\text{\pfref{c2scrut}} \cup \bigcup_{0 \leq j < \nrules} \text{\pfref{c2r}}_j$
    \item $\text{\pfref{c3scrut}} \cup \bigcup_{0 \leq j < \nrules} \text{\pfref{c3r}}_j$
    \item Choose:
      \begin{enumerate}
      \item $\sseq{t_i}{\nty'} \cup \{ \sseq{t_{i,j}}{\ntyj} \}_{0 \leq j < \nrules}$; and
      \item $\sseq{x_i}{\nexp'} \cup \{ \sseq{x_{i,j}}{\nexpj} \}_{0 \leq j < \nrules}$; and 
      \item $\aematchwith{\nrules}{e''}{\seqschemaX{r'}}$
      \end{enumerate}
      We have $e = [\sseq{\tau'_i/t_i}{\nty'} \cup \{ \sseq{\tau_{i,j}/t_{i,j}}{\ntyj} \}_{0 \leq j < \nrules}][\sseq{e_i/x_i}{\nexp'} \cup \{ \sseq{e_{i,j}/x_{i,j}}{\nexpj} \}_{0 \leq j < \nrules}]\aematchwith{\nrules}{e''}{\seqschemaX{r'}}$ by \pfref{c4e}.
    \item \pfref{c5ff}
  \end{enumerate}
  \resetpfcounter
\end{byCases}
\end{grayparbox}
\end{enumerate}
\vspace{-2px}\begin{grayparbox}\vspace{2px}
\begin{enumerate}
\item[2.] By rule induction over the rule typing assumption. There is only one case. In the following, let $\uDelta=\uDD{\uD}{\Delta_\text{app}}$ and $\uGamma=\uGG{\uG}{\Gamma_\text{app}}$ and $\escenev=\esceneUP{\uDelta}{\uGamma}{\uPsi}{\uPhi}{b}$.
  \begin{byCases}
    \item[\text{(\ref{rule:cvalidR-UP})}] ~
    \begin{pfsteps*}
      \item $\crv = \acematchrule{p}{\ce}$ \BY{assumption}
      \item $r=\aematchrule{p}{e}$ \BY{assumption}
      \item $\patType{\pctx'}{p}{\tau}$ \BY{assumption} \pflabel{patType}
      \item $\cvalidE{\Delta}{\Gcons{\Gamma}{\pctx'}}{\escenev}{\ce}{e}{\tau'}$ \BY{assumption} \pflabel{cvalid}
      \item $\summaryOf{\crv} = \summaryOf{\ce}$ \BY{definition}
      \item $\summaryOf{\ce} = \sseq{\acesplicedt{m'_i}{n'_i}}{\nty} \cup \sseq{\acesplicede{m_i}{n_i}{\ctau_i}}{\nexp}$ \BY{definition} \pflabel{summaryOfce}
      \item $\sseq{
            \expandsTU{\uDD{\uD}{\Delta_\text{app}}}
            {
              \parseUTypF{\bsubseq{b}{m'_i}{n'_i}}
            }{\tau'_i}
          }{\nty}$
          \BY{IH, part 1 on \pfref{cvalid} and \pfref{summaryOfce}}
          \pflabel{c1e}
      \item $\sseq{
        \cvalidT{\emptyset}{
          \tsceneUP
            {\uDD
              {\uD}{\Delta_\text{app}}
            }{b}
        }{
          \ctau_i
        }{\tau_i}
      }{\nexp}$
                \BY{IH, part 1 on \pfref{cvalid} and \pfref{summaryOfce}}
          \pflabel{c2e}
      \item $\sseq{
        \expandsSG
          {\uDD{\uD}{\Delta_\text{app}}}
          {\uGG{\uG}{\Gamma_\text{app}}}
          {\uPsi}
          {\uPhi}
          {\parseUExpF{\bsubseq{b}{m_i}{n_i}}}
          {e_i}
          {\tau_i}
      }{\nexp}$
          \BY{IH, part 1 on \pfref{cvalid} and \pfref{summaryOfce}}
          \pflabel{c3e}      
      \item $e = [\sseq{\tau'_i/t_i}{\nty}][\sseq{e_i/x_i}{\nexp}]e'$ for some $\sseq{t_i}{\nty}$ and $\sseq{x_i}{\nexp}$ and $e'$
          \BY{IH, part 1 on \pfref{cvalid} and \pfref{summaryOfce}}
          \pflabel{c4e}
      \item $\hastypeU
        {\Delta \cup \sseq{\Dhyp{t_i}}{\nty}}
        {\Gamma \cup \pctx' \cup \sseq{x_i : \tau_i}{\nexp}}
        {e'}{\tau}$
          \BY{IH, part 1 on \pfref{cvalid} and \pfref{summaryOfce}}
          \pflabel{c5e}
      \item $r=[\sseq{\tau'_i/t_i}{\nty}][\sseq{e_i/x_i}{\nexp}]\aematchrule{p}{e'}$ \BY{substitution properties and \pfref{c4e}} \pflabel{c4r}
      \item $\ruleType
              {\Delta \cup \sseq{\Dhyp{t_i}}{\nty}}
              {\Gamma \cup \sseq{x_i : \tau_i}{\nexp}}
              {\aematchrule{p}{e'}}
              {\tau}{\tau'}$ \BY{Rule (\ref{rule:ruleType}) on \pfref{patType} and \pfref{c5e}} \pflabel{c5r}
    \end{pfsteps*}
    The conclusions hold as follows:
    \begin{enumerate}
      \item \pfref{c1e}
      \item \pfref{c2e}
      \item \pfref{c3e}
      \item Choosing $\sseq{t_i}{\nty}$ and $\sseq{x_i}{\nexp}$ and $\aematchrule{p}{e'}$, by \pfref{c4r}
      \item \pfref{c5r}
    \end{enumerate}
    \resetpfcounter
  \end{byCases}
\end{enumerate}
\end{grayparbox}
\end{proof}
% \begin{equation}\label{rule:cvalidR-UP}
% \inferrule{
%   \patType{\pctx}{p}{\tau}\\
%   \cvalidE{\Delta}{\Gcons{\Gamma}{\pctx}}{\escenev}{\ce}{e}{\tau'}
% }{
%   \cvalidR{\Delta}{\Gamma}{\escenev}{\acematchrule{p}{\ce}}{\aematchrule{p}{e}}{\tau}{\tau'}
% }
% \end{equation}

\begin{theorem}[seTSM Abstract Reasoning Principles]
\label{thm:tsc-SES}
If $\expandsSG{\uDD{\uD}{\Delta}}{\uGG{\uG}{\Gamma}}{\uPsi}{\uPhi}{\utsmap{\tsmv}{b}}{e}{\tau}$ then:
\begin{enumerate}
\item (\textbf{Typing 1}) $\uPsi = \uPsi', \uShyp{\tsmv}{a}{\tau}{\eparse}$ and $\hastypeU{\Delta}{\Gamma}{e}{\tau}$
\item $\encodeBody{b}{\ebody}$
\item $\evalU{\ap{\eparse}{\ebody}}{\lbltxt{SuccessE}\cdot\ecand}$
\item $\decodeCondE{\ecand}{\ce}$
\item (\textbf{Segmentation}) $\segOK{\segof{\ce}}{b}$
\item $\summaryOf{\ce} = \sseq{\acesplicedt{m'_i}{n'_i}}{\nty} \cup \sseq{\acesplicede{m_i}{n_i}{\ctau_i}}{\nexp}$
\item \textbf{(Typing 2)} $\sseq{
      \expandsTU{\uDD{\uD}{\Delta}}
      {
        \parseUTypF{\bsubseq{b}{m'_i}{n'_i}}
      }{\tau'_i}
    }{\nty}$ and $\sseq{\istypeU{\Delta}{\tau'_i}}{\nty}$
\item \textbf{(Typing 3)} $\sseq{
  \cvalidT{\emptyset}{
    \tsceneUP
      {\uDD
        {\uD}{\Delta}
      }{b}
  }{
    \ctau_i
  }{\tau_i}
}{\nexp}$ and $\sseq{\istypeU{\Delta}{\tau_i}}{\nexp}$
\item \textbf{(Typing 4)} $\sseq{
  \expandsSG
    {\uDD{\uD}{\Delta}}
    {\uGG{\uG}{\Gamma}}
    {\uPsi}
    {\uPhi}
    {\parseUExpF{\bsubseq{b}{m_i}{n_i}}}
    {e_i}
    {\tau_i}
}{\nexp}$ and $\sseq{\hastypeU{\Delta}{\Gamma}{e_i}{\tau_i}}{\nexp}$
\item (\textbf{Capture Avoidance}) $e = [\sseq{\tau'_i/t_i}{\nty}][\sseq{e_i/x_i}{\nexp}]e'$ for some $\sseq{t_i}{\nty}$ and $\sseq{x_i}{\nexp}$ and $e'$
\item (\textbf{Context Independence}) $\hastypeU
  {\sseq{\Dhyp{t_i}}{\nty}}
  {\sseq{x_i : \tau_i}{\nexp}}
  {e'}{\tau}$
\end{enumerate}
\end{theorem}
\begin{proof} By rule induction over Rules (\ref{rules:expandsU}). There is only one rule that applies. In the following, let $\uDelta = \uDD{\uD}{\Delta}$ and $\uGamma = \uGG{\uG}{\Gamma}$.
\begin{byCases}
\item[\text{(\ref{rule:expandsU-tsmap})}] ~
\begin{pfsteps*}
  \item $\uPsi = \uPsi', \uShyp{\tsmv}{a}{\tau}{\eparse}$ \BY{assumption} \pflabel{uPsidef}
  \item $\expandsSG{\uDD{\uD}{\Delta}}{\uGG{\uG}{\Gamma}}{\uPsi}{\uPhi}{\utsmap{\tsmv}{b}}{e}{\tau}$ \BY{assumption} \pflabel{expandsSG}
  \item $\hastypeU{\Delta}{\Gamma}{e}{\tau}$ \BY{Theorem \ref{thm:typed-expansion-short-U} on \pfref{expandsSG}} \pflabel{hastype}
  \item $\encodeBody{b}{\ebody}$ \BY{assumption} \pflabel{encodeBody}
  \item $\evalU{\eparse(\ebody)}{{\lbltxt{SuccessE}}\cdot{\ecand}}$ \BY{assumption} \pflabel{evalU}
  \item $\decodeCondE{\ecand}{\ce}$ \BY{assumption} \pflabel{decodeCondE}
  \item $\segOK{\segof{\ce}}{b}$ \BY{assumption} \pflabel{segOK}
  \item $\cvalidE{\emptyset}{\emptyset}{\esceneSG{\uDelta}{\uGamma}{\uPsi}{\uPhi}{b}}{\ce}{e}{\tau}$ \BY{assumption}\pflabel{cvalidE}
  \item $\summaryOf{\ce} = \sseq{\acesplicedt{m'_i}{n'_i}}{\nty} \cup \sseq{\acesplicede{m_i}{n_i}{\ctau_i}}{\nexp}$ \BY{definition} \pflabel{summaryOf}
  \item $\sseq{
      \expandsTU{\uDD{\uD}{\Delta}}
      {
        \parseUTypF{\bsubseq{b}{m'_i}{n'_i}}
      }{\tau'_i}
    }{\nty}$ \BY{Lemma \ref{thm:proto-expression-expansion-decomposition} on \pfref{cvalidE} and \pfref{summaryOf}} \pflabel{c1}
\item $\sseq{\istypeU{\Delta}{\tau'_i}}{\nty}$ \BY{Lemma \ref{lemma:type-expansion-U}, part 1 over \pfref{c1}} \pflabel{t1}
\item $\sseq{
  \cvalidT{\emptyset}{
    \tsceneUP
      {\uDD
        {\uD}{\Delta}
      }{b}
  }{
    \ctau_i
  }{\tau_i}
}{\nexp}$ \BY{Lemma \ref{thm:proto-expression-expansion-decomposition} on \pfref{cvalidE} and \pfref{summaryOf}} \pflabel{c2}
\item $\emptyset \cap \Delta = \emptyset$ \BY{definition} \pflabel{emptyintersect}
\item $\sseq{\istypeU{\Delta}{\tau_i}}{\nexp}$ \BY{Lemma \ref{lemma:type-expansion-U}, part 2 over \pfref{c2} and \pfref{emptyintersect}} \pflabel{t2}
\item $\sseq{
  \expandsSG
    {\uDD{\uD}{\Delta}}
    {\uGG{\uG}{\Gamma}}
    {\uPsi}
    {\uPhi}
    {\parseUExpF{\bsubseq{b}{m_i}{n_i}}}
    {e_i}
    {\tau_i}
}{\nexp}$ \BY{Lemma \ref{thm:proto-expression-expansion-decomposition} on \pfref{cvalidE} and \pfref{summaryOf}} \pflabel{c3}
\item $\sseq{\hastypeU{\Delta}{\Gamma}{e_i}{\tau_i}}{\nexp}$ \BY{Theorem \ref{thm:typed-expansion-short-U} over \pfref{c3}} \pflabel{t3}
\item $e = [\sseq{\tau'_i/t_i}{\nty}][\sseq{e_i/x_i}{\nexp}]e'$ for some $\sseq{t_i}{\nty}$ and $\sseq{x_i}{\nexp}$ and $e'$ \BY{Lemma \ref{thm:proto-expression-expansion-decomposition} on \pfref{cvalidE} and \pfref{summaryOf}} \pflabel{c4}
\item $\hastypeU
  {\sseq{\Dhyp{t_i}}{\nty}}
  {\sseq{x_i : \tau_i}{\nexp}}
  {e'}{\tau}$ \BY{Lemma \ref{thm:proto-expression-expansion-decomposition} on \pfref{cvalidE} and \pfref{summaryOf}} \pflabel{c5}
\end{pfsteps*}
The conclusions hold as follows:
\begin{enumerate}
  \item \pfref{uPsidef} and \pfref{hastype}
  \item \pfref{encodeBody}
  \item \pfref{evalU}
  \item \pfref{decodeCondE}
  \item \pfref{segOK}
  \item \pfref{summaryOf}
  \item \pfref{c1} and \pfref{t1}
  \item \pfref{c2} and \pfref{t2}
  \item \pfref{c3} and \pfref{t3}
  \item \pfref{c4}
  \item \pfref{c5}
\end{enumerate}
\resetpfcounter
\end{byCases}
\end{proof}

% The following theorem establishes a prohibition on \textbf{Shadowing} as discussed in Sec. \ref{sec:uetsms-validation}.

% \begin{theorem}[Shadowing Prohibition]
% \label{thm:shadowing-prohibition-SES} ~
% \begin{enumerate}
% \item If $\cvalidT{\Delta}{\tsceneU{\uDD{\uD}{\Delta_\text{app}}}{b}}{\acesplicedt{m}{n}}{\tau}$ then:\begin{enumerate}
% \item $\parseUTyp{\bsubseq{b}{m}{n}}{\utau}$
% \item $\expandsTU{\uDD{\uD}{\Delta_\text{app}}}{\utau}{\tau}$
% \item $\Delta \cap \Delta_\text{app} = \emptyset$
% \end{enumerate}
% \item If $\cvalidE{\Delta}{\Gamma}{\escenev}{\acesplicede{m}{n}{\ctau}}{e}{\tau}$ then:
% \begin{enumerate}
% \item $\cvalidT{\emptyset}{\tsfrom{\escenev}}{\ctau}{\tau}$
% \item $  \escenev=\esceneU{\uDD{\uD}{\Delta_\text{app}}}{\uGG{\uG}{\Gamma_\text{app}}}{\uPsi}{b}$
% \item $\parseUExp{\bsubseq{b}{m}{n}}{\ue}$
% \item $\expandsU{\uDD{\uD}{\Delta_\text{app}}}{\uGG{\uG}{\Gamma_\text{app}}}{\uPsi}{\ue}{e}{\tau}$
% \item $\Delta \cap \Delta_\text{app} = \emptyset$
% \item $\domof{\Gamma} \cap \domof{\Gamma_\text{app}} = \emptyset$
% \end{enumerate}
% \end{enumerate}
% \end{theorem}
% \begin{proof} ~
% \begin{enumerate}
% \item By rule induction over Rules (\ref{rules:cvalidT-U}). The only rule that applies is Rule (\ref{rule:cvalidT-U-splicedt}). The conclusions are the premises of this rule.
% \item By rule induction over Rules (\ref{rules:cvalidE-U}). The only rule that applies is Rule (\ref{rule:cvalidE-U-splicede}). The conclusions are the premises of this rule.
% \end{enumerate}
% \end{proof}

\begin{grayparbox}
\begin{lemma}[Proto-Pattern Expansion Decomposition]
\label{lemma:proto-pattern-expansion-decomposition-S}
If $\cvalidP{\upctx}{\pscene{\uDelta}{\uPhi}{b}}{\cpv}{p}{\tau}$ and 
\[ 
\summaryOf{\cpv} = \sseq{\acesplicedt{m'_i}{n'_i}}{\nty} \cup \sseq{\acesplicedp{m_i}{n_i}{\ctau_i}}{\npat}
\]
then all of the following hold:
\begin{enumerate}
    \item $\sseq{
          \expandsTU{\uDelta}
          {
            \parseUTypF{\bsubseq{b}{m'_i}{n'_i}}
          }{\tau'_i}
        }{\nty}$
    \item $\sseq{
      \cvalidT{\emptyset}{
        \tsceneUP
          {\uDelta}{b}
      }{
        \ctau_i
      }{\tau_i}
    }{\npat}$
    \item $\sseq{
      \patExpands
        {\upctx_i}
        {\uPhi}
        {\parseUPatF{\bsubseq{b}{m_i}{n_i}}}
        {p_i}
        {\tau_i}
    }{\npat}$
  \item $\upctx = \biguplus_{0 \leq i < \npat} \upctx_i$
\end{enumerate}
\end{lemma}
\begin{proof} By rule induction over Rules (\ref{rules:cvalidP-UP}). In the following, let $\pscenev=\pscene{\uDelta}{\uPhi}{b}$.
\begin{byCases}
  \item[\text{(\ref{rule:cvalidP-UP-wild})}] ~
    \begin{pfsteps*}
      \item $\cpv=\acewildp$ \BY{assumption}
      \item $e = \aewildp$ \BY{assumption}
      \item $\upctx = \uGG{\emptyset}{\emptyset}$ \BY{assumption}
      \item $\summaryOf{\acewildp} = \emptyset$ \BY{definition}
    \end{pfsteps*}
    The conclusions hold as follows:
    \begin{enumerate}
      \item This conclusion holds trivially because $\nty=0$.
      \item This conclusion holds trivially because $\npat=0$.
      \item This conclusion holds trivially because $\npat=0$.
      \item This conclusion holds trivially because $\upctx=\emptyset$ and $\npat=0$.
    \end{enumerate}
    \resetpfcounter
  \item[\text{(\ref{rule:cvalidP-UP-fold})}] ~
    \begin{pfsteps*}
      \item $\cpv=\acefoldp{\cpv'}$ \BY{assumption}
      \item $p=\aefoldp{p'}$ \BY{assumption}
      \item $\tau=\arec{t}{\tau'}$ \BY{assumption}
      \item $\cvalidP{\upctx}{\pscenev}{\cpv}{p}{[\arec{t}{\tau'}/t]\tau'}$ \BY{assumption} \pflabel{cvalidP}
      \item $\summaryOf{\acefoldp{\cpv'}} = \summaryOf{\cpv'}$ \BY{definition} \pflabel{summaryOf}
      \item $\summaryOf{\cpv'} = \sseq{\acesplicedt{m'_i}{n'_i}}{\nty} \cup \sseq{\acesplicedp{m_i}{n_i}{\ctau_i}}{\npat}$ \BY{definition} \pflabel{summaryOf2}
      \item $\sseq{
            \expandsTU{\uDelta}
            {
              \parseUTypF{\bsubseq{b}{m'_i}{n'_i}}
            }{\tau'_i}
          }{\nty}$ \BY{IH on \pfref{cvalidP} and \pfref{summaryOf2}} \pflabel{c1}
      \item $\sseq{
        \cvalidT{\emptyset}{
          \tsceneUP
            {\uDelta}{b}
        }{
          \ctau_i
        }{\tau_i}
      }{\npat}$ \BY{IH on \pfref{cvalidP} and \pfref{summaryOf2}} \pflabel{c2}
      \item $\sseq{
        \patExpands
          {\upctx_i}
          {\uPhi}
          {\parseUPatF{\bsubseq{b}{m_i}{n_i}}}
          {p_i}
          {\tau_i}
      }{\npat}$ \BY{IH on \pfref{cvalidP} and \pfref{summaryOf2}} \pflabel{c3}
    \item $\upctx = \biguplus_{0 \leq i < \npat} \upctx_i$ \BY{IH on \pfref{cvalidP} and \pfref{summaryOf2}}\pflabel{c4}\end{pfsteps*}
    The conclusions hold as follows:
    \begin{enumerate}
    \item \pfref{c1}
    \item \pfref{c2}
    \item \pfref{c3}
    \item \pfref{c4}
    \end{enumerate}
    \resetpfcounter
  \item[\text{(\ref{rule:cvalidP-UP-tpl})}] ~
    \begin{pfsteps*}
      \item $\cpv=\acetplp{\labelset}{\mapschema{\cpv}{j}{\labelset}}$ \BY{assumption}
      \item $p=\aetplp{\labelset}{\mapschema{p}{j}{\labelset}}$ \BY{assumption}
      \item $\tau = \aprod{\labelset}{\mapschema{\tau}{j}{\labelset}}$ \BY{assumption}
      \item $\upctx=\GIconsi{j \in \labelset}{\upctx_j}$ \BY{assumption} \pflabel{Gconsi}
      \item $\{\cvalidP{\upctx_j}{\pscenev}{\cpv_j}{p_j}{\tau_j}\}_{j \in \labelset}$ \BY{assumption} \pflabel{cvalidP}
      \item $\summaryOf{\acetplp{\labelset}{\mapschema{\cpv}{j}{\labelset}}} = \bigcup_{j \in \labelset} \summaryOf{\cpv_j}$ \BY{definition}
      \item $\{ \summaryOf{\cpv_j} = \sseq{\acesplicedt{m'_{i,j}}{n'_{i,j}}}{\ntyj} \cup \sseq{\acesplicedp{m_{i,j}}{n_{i,j}}{\ctau_{i,j}}}{\npatj} \}_{j \in \labelset}$ \BY{definition}\pflabel{summaryOf2}
      \item $\npat = \Sigma_{j \in \labelset} \npatj$ \BY{definition}
      \item $\{\sseq{
            \expandsTU{\uDelta}
            {
              \parseUTypF{\bsubseq{b}{m'_{i,j}}{n'_{i,j}}}
            }{\tau'_{i,j}}
          }{\ntyj}\}_{j \in \labelset}$ \BY{IH over \pfref{cvalidP} and \pfref{summaryOf2}} \pflabel{c1}
      \item $\{\sseq{
        \cvalidT{\emptyset}{
          \tsceneUP
            {\uDelta}{b}
        }{
          \ctau_{i,j}
        }{\tau_{i,j}}
      }{\npatj}\}_{j \in \labelset}$ \BY{IH over \pfref{cvalidP} and \pfref{summaryOf2}} \pflabel{c2}
      \item $\{\sseq{
        \patExpands
          {\upctx_{i,j}}
          {\uPhi}
          {\parseUPatF{\bsubseq{b}{m_{i,j}}{n_{i,j}}}}
          {p_{i,j}}
          {\tau_{i,j}}
      }{\npatj}\}_{j \in \labelset}$ \BY{IH over \pfref{cvalidP} and \pfref{summaryOf2}} \pflabel{c3}
    \item $\{\upctx_j = \biguplus_{0 \leq i < \npatj} \upctx_{i,j}\}_{j \in \labelset}$ \BY{IH over \pfref{cvalidP} and \pfref{summaryOf2}}\pflabel{c4}
    \item $\biguplus{j \in \labelset} \upctx_{j} = \biguplus_{j \in \labelset} \biguplus{i \in \npatj} \upctx_{i,j}$ \BY{definition and \pfref{c4}}\pflabel{c4x}
    \end{pfsteps*}
    The conclusions hold as follows:
    \begin{enumerate}
      \item $\bigcup_{j \in \labelset} \bigcup_{i \in \ntyj} \text{\pfref{c1}}_{i,j}$
      \item $\bigcup_{j \in \labelset} \bigcup_{i \in \npatj} \text{\pfref{c2}}_{i,j}$
      \item $\bigcup_{j \in \labelset} \bigcup_{i \in \npatj} \text{\pfref{c3}}_{i,j}$
      \item \pfref{c4x}
    \end{enumerate}
    \resetpfcounter
  \item[\text{(\ref{rule:cvalidP-UP-in})}]
    \begin{pfsteps*}
      \item $\cpv=\aceinjp{\ell}{\cpv'}$ \BY{assumption}
      \item $p=\aeinjp{\ell}{p'}$ \BY{assumption}
      \item $\tau=\asum{\labelset, \ell}{\mapschema{\tau}{i}{\labelset}; \mapitem{\ell}{\tau'}}$ \BY{assumption}
      \item $\cvalidP{\upctx}{\pscenev}{\cpv}{p}{\tau'}$ \BY{assumption} \pflabel{cvalidP}
      \item $\summaryOf{\aceinjp{\ell}{\cpv'}} = \summaryOf{\cpv'}$ \BY{definition} \pflabel{summaryOf}
      \item $\summaryOf{\cpv'} = \sseq{\acesplicedt{m'_i}{n'_i}}{\nty} \cup \sseq{\acesplicedp{m_i}{n_i}{\ctau_i}}{\npat}$ \BY{definition} \pflabel{summaryOf2}
      \item $\sseq{
            \expandsTU{\uDelta}
            {
              \parseUTypF{\bsubseq{b}{m'_i}{n'_i}}
            }{\tau'_i}
          }{\nty}$ \BY{IH on \pfref{cvalidP} and \pfref{summaryOf2}} \pflabel{c1}
      \item $\sseq{
        \cvalidT{\emptyset}{
          \tsceneUP
            {\uDelta}{b}
        }{
          \ctau_i
        }{\tau_i}
      }{\npat}$ \BY{IH on \pfref{cvalidP} and \pfref{summaryOf2}} \pflabel{c2}
      \item $\sseq{
        \patExpands
          {\upctx_i}
          {\uPhi}
          {\parseUPatF{\bsubseq{b}{m_i}{n_i}}}
          {p_i}
          {\tau_i}
      }{\npat}$ \BY{IH on \pfref{cvalidP} and \pfref{summaryOf2}} \pflabel{c3}
    \item $\upctx = \biguplus_{0 \leq i < \npat} \upctx_i$ \BY{IH on \pfref{cvalidP} and \pfref{summaryOf2}}\pflabel{c4}\end{pfsteps*}
    The conclusions hold as follows:
    \begin{enumerate}
    \item \pfref{c1}
    \item \pfref{c2}
    \item \pfref{c3}
    \item \pfref{c4}
    \end{enumerate}
    \resetpfcounter
  \item[\text{(\ref{rule:cvalidP-UP-spliced})}] ~
    \begin{pfsteps*}
      \item $\cpv=\acesplicedp{m}{n}{\ctau}$ \BY{assumption}
      \item $\cvalidT{\emptyset}{\tsceneUP{\uDelta}{b}}{\ctau}{\tau}$ \BY{assumption} \pflabel{cvalidT}
      \item $\parseUPat{\bsubseq{b}{m}{n}}{\upv}$ \BY{assumption} \pflabel{parseUPat}
      \item $\patExpands{\upctx}{\uPhi}{\upv}{p}{\tau}$ \BY{assumption} \pflabel{patExpands}
      \item $\summaryOf{\acesplicedp{m}{n}{\ctau}} = \summaryOf{\ctau} \cup \{ \acesplicedp{m}{n}{\ctau} \}$ \BY{definition} \pflabel{summaryOf}
      \item $\summaryOf{\ctau} = \sseq{\acesplicedt{m'_i}{n'_i}}{\nty}$ \BY{definition} \pflabel{summaryOf2}
      \item $\sseq{\expandsTU{\uDD{\uD}{\Delta_\text{app}}}{
  \parseUTypF{\bsubseq{b}{m_i}{n_i}}
}{\tau_i}}{n}$ \BY{Lemma \ref{thm:proto-type-expansion-decomposition-SES} on \pfref{cvalidT} and \pfref{summaryOf2}} \pflabel{expandsTU}
% \item $\sseq{\istypeU{\Delta_\text{app}}{\tau_i}}{n}$
    \end{pfsteps*}
    The conclusions hold as follows:
    \begin{enumerate}
      \item \pfref{expandsTU}
      \item \pfref{cvalidT}
      \item \pfref{parseUPat} and \pfref{patExpands}
      \item This conclusion holds by \pfref{patExpands} because $\npat=1$.
    \end{enumerate}
    \resetpfcounter
\end{byCases}
\end{proof}

\begin{theorem}[spTSM Abstract Reasoning Principles]
\label{thm:spTSM-Typing-Segmentation}
If $\patExpands{\upctx}{\uPhi}{\utsmap{\tsmv}{b}}{p}{\tau}$ where $\uDelta=\uDD{\uD}{\Delta}$ and $\uGamma=\uGG{\uG}{\Gamma}$ then all of the following hold:
\begin{enumerate}
        \item (\textbf{Typing 1}) $\uPhi=\uPhi', \uPhyp{\tsmv}{a}{\tau}{\eparse}$ and $\patType{\pctx}{p}{\tau}$
        \item $\encodeBody{b}{\ebody}$
        \item $\evalU{\eparse(\ebody)}{{\lbltxt{SuccessP}}\cdot{\ecand}}$
        \item $\decodeCEPat{\ecand}{\cpv}$
        \item (\textbf{Segmentation}) $\segOK{\segof{\cpv}}{b}$
        \item $\summaryOf{\cpv} = \sseq{\acesplicedt{n'_i}{m'_i}}{\nty} \cup \sseq{\acesplicedp{m_i}{n_i}{\ctau_i}}{\npat}$
        \item (\textbf{Typing 2}) $\sseq{
              \expandsTU{\uDelta}
              {
                \parseUTypF{\bsubseq{b}{m'_i}{n'_i}}
              }{\tau'_i}
            }{\nty}$ and $\sseq{\istypeU{\Delta}{\tau'_i}}{\nty}$
        \item (\textbf{Typing 3}) $\sseq{
          \cvalidT{\emptyset}{
            \tsceneUP
              {\uDelta}{b}
          }{
            \ctau_i
          }{\tau_i}
        }{\npat}$ and $\sseq{\istypeU{\Delta}{\tau_i}}{\npat}$
        \item (\textbf{Typing 4}) $\sseq{
          \patExpands
            {\upctx_i}
            {\uPhi}
            {\parseUPatF{\bsubseq{b}{m_i}{n_i}}}
            {p_i}
            {\tau_i}
        }{\npat}$ and $\sseq{\patType{\upctx_i}{p_i}{\tau_i}}{\npat}$
      \item (\textbf{No Hidden Bindings}) $\upctx = \biguplus_{0 \leq i < \npat} \upctx_i$
\end{enumerate}
\end{theorem}
\begin{proof} By rule induction over Rules (\ref{rules:patExpands}). There is only one rule that applies.
\begin{byCases}
  \item[\text{(\ref{rule:patExpands-apuptsm})}] 
    \begin{pfsteps*}
      \item $\patExpands{\upctx}{\uPhi}{\utsmap{\tsmv}{b}}{p}{\tau}$ \BY{assumption} \pflabel{patExpands}
      \item $\uPhi = \uPhi', \uPhyp{\tsmv}{a}{\tau}{\eparse}$ \BY{assumption} \pflabel{uPhidef}
      \item $\patType{\pctx}{p}{\tau}$ \BY{Theorem \ref{thm:typed-pattern-expansion} on \pfref{patExpands}} \pflabel{patType}
      \item $\encodeBody{b}{\ebody}$ \BY{assumption} \pflabel{encodeBody}
      \item $\evalU{\ap{\eparse}{\ebody}}{{\lbltxt{SuccessP}}\cdot{\ecand}}$ \BY{assumption} \pflabel{evalU}
      \item $\decodeCEPat{\ecand}{\cpv}$ \BY{assumption} \pflabel{decodeCEPat}
      \item $\segOK{\segof{\cpv}}{b}$ \BY{assumption} \pflabel{segOK}
      \item $\cvalidP{\upctx}{\pscene{\uDelta}{\uPhi}{b}}{\cpv}{p}{\tau}$ \BY{assumption} \pflabel{cvalidP}
      \item $\summaryOf{\cpv} = \sseq{\acesplicedt{m'_i}{n'_i}}{\nty} \cup \sseq{\acesplicedp{m_i}{n_i}}{\npat}$ \BY{definition}\pflabel{summaryOf}
      \item $\sseq{
            \expandsTU{\uDelta}
            {
              \parseUTypF{\bsubseq{b}{m'_i}{n'_i}}
            }{\tau'_i}
          }{\nty}$ \BY{Lemma \ref{lemma:proto-pattern-expansion-decomposition-S} on \pfref{cvalidP} and \pfref{summaryOf}}\pflabel{c1}
      \item $\sseq{\istypeU{\Delta}{\tau'_i}}{\nty}$ \BY{Lemma \ref{lemma:type-expansion-U}, part 1 over \pfref{c1}}\pflabel{t1}
      \item $\sseq{
        \cvalidT{\emptyset}{
          \tsceneUP
            {\uDelta}{b}
        }{
          \ctau_i
        }{\tau_i}
      }{\npat}$  \BY{Lemma \ref{lemma:proto-pattern-expansion-decomposition-S} on \pfref{cvalidP} and \pfref{summaryOf}}\pflabel{c2}
      \item $\sseq{\istypeU{\Delta}{\tau_i}}{\npat}$ \BY{Lemma \ref{lemma:type-expansion-U}, part 2 over \pfref{c2}}\pflabel{t2}
      \item $\sseq{
        \patExpands
          {\upctx_i}
          {\uPhi}
          {\parseUPatF{\bsubseq{b}{m_i}{n_i}}}
          {p_i}
          {\tau_i}
      }{\npat}$ \BY{Lemma \ref{lemma:proto-pattern-expansion-decomposition-S} on \pfref{cvalidP} and \pfref{summaryOf}}\pflabel{c3}
    \item $\sseq{\patType{\upctx_i}{p_i}{\tau_i}}{\npat}$ \BY{Theorem \ref{thm:typed-pattern-expansion} over \pfref{c3}}\pflabel{t3}
    \item $\upctx = \biguplus_{0 \leq i < \npat} \upctx_i$ \pflabel{c4}
    \end{pfsteps*}
    The conclusions hold as follows:
    \begin{enumerate}
      \item \pfref{uPhidef} and \pfref{patType}
      \item \pfref{encodeBody}
      \item \pfref{evalU}
      \item \pfref{decodeCEPat}
      \item \pfref{segOK}
      \item \pfref{summaryOf}
      \item \pfref{c1} and \pfref{t1}
      \item \pfref{c2} and \pfref{t2}
      \item \pfref{c3} and \pfref{t3}
      \item \pfref{c4}
    \end{enumerate}
    \resetpfcounter
\end{byCases}
\end{proof}
\end{grayparbox}

% \inferrule{
%   \uPhi = \uPhi', \uPhyp{\tsmv}{a}{\tau}{\eparse}\\\\
%   \encodeBody{b}{\ebody}\\
%   \evalU{\ap{\eparse}{\ebody}}{{\lbltxt{SuccessP}}\cdot{\ecand}}\\
%   \decodeCEPat{\ecand}{\cpv}\\\\
%     \segOK{\segof{\cpv}}{b}\\
%   \cvalidP{\upctx}{\pscene{\uDelta}{\uPhi}{b}}{\cpv}{p}{\tau}
% }{
%   \patExpands{\upctx}{\uPhi}{\utsmap{\tsmv}{b}}{p}{\tau}
% }}

\chapter{$\miniVerseParam$}\label{appendix:miniVerseParam}

\clearpage
\section{Expanded Language (XL)}
\subsection{Syntax}
\subsubsection{Signatures and Module Expressions}
\[\begin{array}{lllllll}
\textbf{Sort} & & & \textbf{Operational Form} 
%& \textbf{Stylized Form} 
& \textbf{Description}\\
\mathsf{Sig} & \sigma & ::= & \asignature{\kappa}{u}{\tau} 
%& \signature{u}{\kappa}{\tau} 
& \text{signature}\\
\mathsf{Mod} & M & ::= & X 
%& X 
& \text{module variable}\\
&&& \astruct{c}{e} 
%& \struct{c}{e} 
& \text{structure}\\
&&& \aseal{\sigma}{M} 
%& \seal{M}{\sigma} 
& \text{seal}\\
&&& \amlet{\sigma}{M}{X}{M} %& \mlet{X}{M}{M}{\sigma} 
& \text{definition}
\end{array}\]

\subsubsection{Kinds and Constructors}
\[\begin{array}{lrlllll}
\textbf{Sort} & & & \textbf{Operational Form} 
%& \textbf{Stylized Form} 
& \textbf{Description}\\
\mathsf{Kind} & \kappa & ::= & \akdarr{\kappa}{u}{\kappa} 
%& \kdarr{u}{\kappa}{\kappa} 
& \text{dependent function}\\
&&& \akunit 
%& \kunit 
& \text{nullary product}\\
&&& \akdbprod{\kappa}{u}{\kappa} 
%& \kdbprod{u}{\kappa}{\kappa} 
& \text{dependent product}\\
%&&& \akdprodstd & \kdprodstd & \text{labeled dependent product}\\
&&& \akty 
%& \kty
& \text{types}\\
&&& \aksing{\tau} 
%& \ksing{\tau} 
& \text{singleton}\\
\mathsf{Con} & c, \tau & ::= & u 
%& u 
& \text{constructor variable}\\
&&& t 
%& t 
& \text{type variable}
\\
&&& \acabs{u}{c} 
%& \cabs{u}{c} 
& \text{abstraction}\\
&&& \acapp{c}{c} 
%& \capp{c}{c} 
& \text{application}\\
&&& \actriv 
%& \ctriv 
& \text{trivial}\\
&&& \acpair{c}{c}
% & \cpair{c}{c} 
& \text{pair}\\
&&& \acprl{c} 
%& \cprl{c} 
& \text{left projection}\\
&&& \acprr{c} 
%& \cprr{c} 
& \text{right projection}\\
%&&& \adtplX & \dtplX & \text{labeled dependent tuple}\\
%&&& \adprj{\ell}{c} & \prj{c}{\ell} & \text{projection}\\
&&& \aparr{\tau}{\tau} 
%& \parr{\tau}{\tau} 
& \text{partial function}\\
&&& \aallu{\kappa}{u}{\tau} 
%& \forallu{u}{\kappa}{\tau} 
& \text{polymorphic}\\
&&& \arec{t}{\tau} 
%& \rect{t}{\tau} 
& \text{recursive}\\
&&& \aprod{\labelset}{\mapschema{\tau}{i}{\labelset}} 
%& \prodt{\mapschema{\tau}{i}{\labelset}} 
& \text{labeled product}\\
&&& \asum{\labelset}{\mapschema{\tau}{i}{\labelset}} 
%& \sumt{\mapschema{\tau}{i}{\labelset}} 
& \text{labeled sum}\\
&&& \amcon{M} 
%& \mcon{M} 
& \text{constructor component}
\end{array}\]
\clearpage

\subsubsection{Expressions, Rules and Patterns}
\[\begin{array}{lllllll}
\textbf{Sort} & & & \textbf{Operational Form} 
%& \textbf{Stylized Form} 
& \textbf{Description}\\
\mathsf{Exp} & e & ::= & x 
%& x 
& \text{variable}\\
&&& \aelam{\tau}{x}{e} 
%& \lam{x}{\tau}{e} 
& \text{abstraction}\\
&&& \aeap{e}{e} 
%& \ap{e}{e} 
& \text{application}\\
&&& \aeclam{\kappa}{u}{e} %& \clam{u}{\kappa}{e} 
& \text{constructor abstraction}\\
&&& \aecap{e}{\kappa} %& \cAp{e}{\kappa} 
& \text{constructor application}\\
&&& \aefold{e} %& \fold{e} 
& \text{fold}\\
&&& \aeunfold{e} %& \unfold{e} 
& \text{unfold}\\
&&& \aetpl{\labelset}{\mapschema{e}{i}{\labelset}} 
%& \tpl{\mapschema{e}{i}{\labelset}} 
& \text{labeled tuple}\\
&&& \aepr{\ell}{e} 
%& \prj{e}{\ell} 
& \text{projection}\\
&&& \aein{\ell}{e} 
%& \inj{\ell}{e} 
& \text{injection}\\
&&& \aematchwith{n}{e}{\seqschemaX{r}} 
%& \matchwith{e}{\seqschemaX{r}} 
& \text{match}\\
&&& \amval{M} 
%& \mval{M} 
& \text{value component}\\
\mathsf{Rule} & r & ::= & \aematchrule{p}{e} 
%& \matchrule{p}{e} 
& \text{rule}\\
\mathsf{Pat} & p & ::= & x 
%& x 
& \text{variable pattern}\\
&&& \aewildp 
%& \wildp 
& \text{wildcard pattern}\\
&&& \aefoldp{p} 
%& \foldp{p} 
& \text{fold pattern}\\
&&& \aetplp{\labelset}{\mapschema{p}{i}{\labelset}} 
%& \tplp{\mapschema{p}{i}{\labelset}} 
& \text{labeled tuple pattern}\\
&&& \aeinjp{\ell}{p} 
%& \injp{\ell}{p} 
& \text{injection pattern}
\end{array}\]

\subsection{Statics}\label{appendix:P-statics}
\subsubsection{Unified Contexts}
A \emph{unified context}, $\Omega$, is an ordered finite function. 
We write
\begin{itemize}
\item $\Omega, X : \sigma$ when $X \notin \domof{\Omega}$ and $\issigX{\sigma}$ for the extension of $\Omega$ with a mapping from $X$ to the hypothesis $X : \sigma$.
\item $\Omega, x : \tau$ when $x \notin \domof{\Omega}$ and $\haskindX{\tau}{\akty}$ for the extension of $\Omega$ with a mapping from $x$ to the hypothesis $x : \tau$
\item $\Omega, u :: \kappa$ when $u \notin \domof{\Omega}$ and $\iskindX{\kappa}$ for the extension of $\Omega$ with a mapping from $u$ to the hypothesis $u :: \kappa$
\end{itemize}
A well-formed unified context is one that can be constructed by some sequence of such extensions, starting from the empty context, $\emptyset$. We identify unified contexts up to exchange and contraction in the usual manner.

% \begin{definition}[Unified Context Formation] $\isctxU{\Omega}$ iff:
% \begin{enumerate}
% \item $\Omega = \emptyset$; or
% \item $\Omega = \Omega', X : \sigma$ and $\issig{\Omega'}{\sigma}$; or 
% \item $\Omega = \Omega', u :: \kappa$ and $\iskind{\Omega'}{\kappa}$; or 
% \item $\Omega = \Omega', x : \tau$ and $\isTypeP{\Omega'}{\tau}$.
% \end{enumerate}
% \end{definition}

\subsubsection{Signatures and Structures}
\noindent\fbox{$\strut\issigX{\sigma}$}~~$\sigma$ is a signature
\begin{equation}\label{rule:issig}
\inferrule{
  \iskindX{\kappa}\\
  \haskind{\Omega, u :: \kappa}{\tau}{\akty}
}{
  \issigX{\asignature{\kappa}{u}{\tau}}
}
\end{equation}

\noindent\fbox{$\strut\sigequalX{\sigma}{\sigma'}$}~~$\sigma$ and $\sigma'$ are definitionally equal
\begin{equation}\label{rule:sigequal}
\inferrule{
  \kequalX{\kappa}{\kappa'}\\
  \cequal{\Omega, u :: \kappa}{\tau}{\tau'}{\akty}
}{
  \sigequalX{\asignature{\kappa}{u}{\tau}}{\asignature{\kappa'}{u}{\tau'}}
}
\end{equation}

\noindent\fbox{$\strut\sigsubX{\sigma}{\sigma'}$}~~$\sigma$ is a subsignature of $\sigma'$
\begin{equation}\label{rule:sigsub}
\inferrule{
  \ksubX{\kappa}{\kappa'}\\
  \issubtypeP{\Omega, u :: \kappa}{\tau}{\tau'}
}{
  \sigsubX{\asignature{\kappa}{u}{\tau}}{\asignature{\kappa'}{u}{\tau'}}
}
\end{equation}

\noindent\fbox{$\strut\hassigX{M}{\sigma}$}~~$M$ matches $\sigma$
\begin{subequations}\label{rules:hassig}
\begin{equation}\label{rule:hassig-subsume}
\inferrule{
  \hassigX{M}{\sigma}\\
  \sigsubX{\sigma}{\sigma'}
}{
  \hassigX{M}{\sigma'}
}
\end{equation}
\begin{equation}\label{rule:hassig-var}
\inferrule{ }{
  \hassig{\Omega, X : \sigma}{X}{\sigma}
}
\end{equation}
\begin{equation}\label{rule:hassig-struct}
\inferrule{
  \haskindX{c}{\kappa}\\
  \hastypeP{\Omega}{e}{[c/u]\tau}
}{
  \hassigX{\astruct{c}{e}}{\asignature{\kappa}{u}{\tau}}
}
\end{equation}
\begin{equation}\label{rule:hassig-seal}
\inferrule{
  \issigX{\sigma}\\
  \hassigX{M}{\sigma}
}{
  \hassigX{\aseal{\sigma}{M}}{\sigma}
}
\end{equation}
\begin{equation}\label{rule:hassig-let}
\inferrule{
  \hassigX{M}{\sigma}\\
  \issigX{\sigma'}\\
  \hassig{\Omega, X : \sigma}{M'}{\sigma'}  
}{
  \hassigX{\amlet{\sigma'}{M}{X}{M'}}{\sigma'}
}
\end{equation}
\end{subequations}

\noindent\fbox{$\strut\ismvalX{M}$}~~$M$ is, or stands for, a module value
\begin{subequations}\label{rules:ismval}
\begin{equation}\label{rule:ismval-struct}
\inferrule{ }{
  \ismvalX{\astruct{c}{e}}
}
\end{equation}
\begin{equation}\label{rule:ismval-var}
\inferrule{ }{
  \ismval{\Omega, X : \sigma}{X}
}
\end{equation}
\end{subequations}

\subsubsection{Kinds and Constructors}
\noindent\fbox{$\strut\iskindX{\kappa}$}~~$\kappa$ is a kind
\begin{subequations}\label{rules:iskind}
\begin{equation}\label{rule:iskind-darr}
\inferrule{
  \iskindX{\kappa_1}\\
  \iskind{\Omega, u :: \kappa_1}{\kappa_2}
}{
  \iskindX{\akdarr{\kappa_1}{u}{\kappa_2}}
}
\end{equation}
\begin{equation}\label{rule:iskind-unit}
\inferrule{ }{
  \iskindX{\akunit}
}
\end{equation}
\begin{equation}\label{rule:iskind-dprod}
\inferrule{
  \iskindX{\kappa_1}\\
  \iskind{\Omega, u :: \kappa_1}{\kappa_2}
}{
  \iskindX{\akdbprod{\kappa_1}{u}{\kappa_2}}
}
\end{equation}
\begin{equation}\label{rule:iskind-ty}
\inferrule{ }{
  \iskindX{\akty}
}
\end{equation}
\begin{equation}\label{rule:iskind-sing}
\inferrule{
  \haskindX{\tau}{\akty}
}{
  \iskindX{\aksing{\tau}}
}
\end{equation}
\end{subequations}

\noindent\fbox{$\strut\kequalX{\kappa}{\kappa'}$}~~$\kappa$ and $\kappa'$ are definitionally equal
\begin{subequations}\label{rules:kequal}
\begin{equation}\label{rule:kequal-refl}
\inferrule{
  \iskindX{\kappa}
}{
  \kequalX{\kappa}{\kappa}
}
\end{equation}
\begin{equation}\label{rule:kequal-sym}
\inferrule{
  \kequalX{\kappa}{\kappa'}
}{
  \kequalX{\kappa'}{\kappa}
}
\end{equation}
\begin{equation}\label{rule:kequal-trans}
\inferrule{
  \kequalX{\kappa}{\kappa'}\\
  \kequalX{\kappa'}{\kappa''}
}{
  \kequalX{\kappa}{\kappa''}
}
\end{equation}
\begin{equation}\label{rule:kequal-darr}
\inferrule{
  \kequalX{\kappa_1}{\kappa_1'}\\
  \kequal{\Omega, u :: \kappa_1}{\kappa_2}{\kappa_2'}
}{
  \kequalX{\akdarr{\kappa_1}{u}{\kappa_2}}{\akdarr{\kappa_1'}{u}{\kappa_2'}}
}
\end{equation}
\begin{equation}\label{rule:kequal-dprod}
\inferrule{
  \kequalX{\kappa_1}{\kappa'_1}\\
  \kequal{\Omega, u :: \kappa_1}{\kappa_2}{\kappa'_2}
}{
  \kequalX{\akdbprod{\kappa_1}{u}{\kappa_2}}{\akdbprod{\kappa'_1}{u}{\kappa'_2}}  
}
\end{equation}
\begin{equation}\label{rule:kequal-sing}
\inferrule{
  \cequalX{c}{c'}{\akty}
}{
  \kequalX{\aksing{c}}{\aksing{c'}}
}
\end{equation}
\end{subequations}

\noindent\fbox{$\strut\ksubX{\kappa}{\kappa'}$}~~$\kappa$ is a subkind of $\kappa'$
\begin{subequations}\label{rules:ksub}
\begin{equation}\label{rule:ksub-equal}
\inferrule{
  \kequalX{\kappa}{\kappa'}
}{
  \ksubX{\kappa}{\kappa'}
}
\end{equation}
\begin{equation}\label{rule:ksub-trans}
\inferrule{
  \ksubX{\kappa}{\kappa'}\\
  \ksubX{\kappa'}{\kappa''}
}{
  \ksubX{\kappa}{\kappa''}
}
\end{equation}
\begin{equation}\label{rule:ksub-darr}
\inferrule{
  \ksubX{\kappa'_1}{\kappa_1}\\
  \ksub{\Omega, u :: \kappa'_1}{\kappa_2}{\kappa'_2}  
}{
  \ksubX{\akdarr{\kappa_1}{u}{\kappa_2}}{\akdarr{\kappa'_1}{u}{\kappa'_2}}
}
\end{equation}
\begin{equation}\label{rule:ksub-dprod}
\inferrule{
  \ksubX{\kappa_1}{\kappa'_1}\\
  \ksub{\Omega, u :: \kappa_1}{\kappa_2}{\kappa'_2}
}{
  \ksubX{\akdbprod{\kappa_1}{u}{\kappa_2}}{\akdbprod{\kappa'_1}{u}{\kappa'_2}}
}
\end{equation}
\begin{equation}\label{rule:ksub-sing}
\inferrule{
  \haskindX{\tau}{\akty}
}{
  \ksubX{\aksing{\tau}}{\akty}
}
\end{equation}
\begin{equation}\label{rule:ksub-sing-2}
\inferrule{
  \issubtypePX{\tau}{\tau'}
}{
  \ksubX{\aksing{\tau}}{\aksing{\tau'}}
}
\end{equation}
\end{subequations}

\noindent\fbox{$\strut\haskindX{c}{\kappa}$}~~$c$ has kind $\kappa$
\begin{subequations}\label{rules:haskind}
\begin{equation}\label{rule:haskind-subsume}
\inferrule{
  \haskindX{c}{\kappa_1}\\
  \ksubX{\kappa_1}{\kappa_2}
}{
  \haskindX{c}{\kappa_2}
}
\end{equation}
\begin{equation}\label{rule:haskind-var}
\inferrule{ }{\haskind{\Omega, u :: \kappa}{u}{\kappa}}
\end{equation}
\begin{equation}\label{rule:haskind-abs}
\inferrule{
  \haskind{\Omega, u :: \kappa_1}{c_2}{\kappa_2}
}{
  \haskindX{\acabs{u}{c_2}}{\akdarr{\kappa_1}{u}{\kappa_2}}
}
\end{equation}
\begin{equation}\label{rule:haskind-app}
\inferrule{
  \haskindX{c_1}{\akdarr{\kappa_2}{u}{\kappa}}\\
  \haskindX{c_2}{\kappa_2}
}{
  \haskindX{\acapp{c_1}{c_2}}{[c_1/u]\kappa}
}
\end{equation}
\begin{equation}\label{rule:haskind-unit}
\inferrule{ }{
  \haskindX{\actriv}{\akunit}
}
\end{equation}
\begin{equation}\label{rule:haskind-pair}
\inferrule{
  \haskindX{c_1}{\kappa_1}\\
  \haskindX{c_2}{[c_1/u]\kappa_2}
}{
  \haskindX{\acpair{c_1}{c_2}}{\akdbprod{\kappa_1}{u}{\kappa_2}}
}
\end{equation}
\begin{equation}\label{rule:haskind-prl}
\inferrule{
  \haskindX{c}{\akdbprod{\kappa_1}{u}{\kappa_2}}
}{
  \haskindX{\acprl{c}}{\kappa_1}
}
\end{equation}
\begin{equation}\label{rule:haskind-prr}
\inferrule{
  \haskindX{c}{\akdbprod{\kappa_1}{u}{\kappa_2}}
}{
  \haskindX{\acprr{c}}{[\acprl{c}/u]\kappa_2}
}
\end{equation}
\begin{equation}\label{rule:haskind-parr}
\inferrule{
  \haskindX{\tau_1}{\akty}\\
  \haskindX{\tau_2}{\akty}
}{
  \haskindX{\aparr{\tau_1}{\tau_2}}{\akty}
}
\end{equation}
\begin{equation}\label{rule:haskind-all}
\inferrule{
  \iskindX{\kappa}\\
  \haskind{\Omega, u :: \kappa}{\tau}{\akty}
}{
  \haskindX{\aallu{\kappa}{u}{\tau}}{\akty}
}
\end{equation}
\begin{equation}\label{rule:haskind-rec}
\inferrule{
  \haskind{\Omega, t :: \akty}{\tau}{\akty}
}{
  \haskindX{\arec{t}{\tau}}{\akty}
}
\end{equation}
\begin{equation}\label{rule:haskind-prod}
\inferrule{
  \{\haskindX{\tau_i}{\akty}\}_{1 \leq i \leq n}
}{
  \haskindX{\aprod{\labelset}{\mapschema{\tau}{i}{\labelset}}}{\akty}
}
\end{equation}
\begin{equation}\label{rule:haskind-sum}
\inferrule{
  \{\haskindX{\tau_i}{\akty}\}_{1 \leq i \leq n}
}{
  \haskindX{\asum{\labelset}{\mapschema{\tau}{i}{\labelset}}}{\akty}
}
\end{equation}
\begin{equation}\label{rule:haskind-sing}
\inferrule{
  \haskindX{c}{\akty}
}{
  \haskindX{c}{\aksing{c}}
}
\end{equation}
\begin{equation}\label{rule:haskind-stat}
\inferrule{
  \ismvalX{M}\\
  \hassigX{M}{\asignature{\kappa}{u}{\tau}}
}{
  \haskindX{\amcon{M}}{\kappa}
}
\end{equation}
\end{subequations}

\noindent\fbox{$\strut\cequalX{c}{c'}{\kappa}$}~~$c$ and $c'$ are definitionally equal as constructors of kind $\kappa$
\begin{subequations}\label{rules:cequal}
\begin{equation}\label{rule:cequal-refl}
\inferrule{
  \haskindX{c}{\kappa}
}{
  \cequalX{c}{c}{\kappa}
}
\end{equation}
\begin{equation}\label{rule:cequal-sym}
\inferrule{
  \cequalX{c}{c'}{\kappa}
}{
  \cequalX{c'}{c}{\kappa}
}
\end{equation}
\begin{equation}\label{rule:cequal-trans}
\inferrule{
  \cequalX{c}{c'}{\kappa}\\
  \cequalX{c'}{c''}{\kappa}
}{
  \cequalX{c}{c''}{\kappa}
}
\end{equation}
\begin{equation}\label{rule:cequal-lam}
\inferrule{
  \cequal{\Omega, u :: \kappa_1}{c}{c'}{\kappa_2}
}{
  \cequalX{\acabs{u}{c}}{\acabs{u}{c'}}{\akdarr{\kappa_1}{u}{\kappa_2}}
}
\end{equation}
\begin{equation}\label{rule:cequal-app-1}
\inferrule{
  \cequalX{c_1}{c_1'}{\akdarr{\kappa_2}{u}{\kappa}}\\
  \cequalX{c_2}{c_2'}{\kappa_2}
}{
  \cequalX{\acapp{c_1}{c_2}}{\acapp{c'_1}{c'_2}}{\kappa}
}
\end{equation}
\begin{equation}\label{rule:cequal-app-2}
\inferrule{
  \haskindX{\acabs{u}{c}}{\akdarr{\kappa_2}{u}{\kappa}}\\
  \haskindX{c_2}{\kappa_2}
}{
  \cequalX{\acapp{\acabs{u}{c}}{c_2}}{[c_2/u]c}{[c_2/u]\kappa}
}
\end{equation}
\begin{equation}\label{rule:cequal-pair}
\inferrule{
  \cequalX{c_1}{c'_1}{\kappa_1}\\
  \cequalX{c_2}{c'_2}{[c_1/u]\kappa_2}
}{
  \cequalX{\acpair{c_1}{c_2}}{\acpair{c'_1}{c'_2}}{\akdbprod{\kappa_1}{u}{\kappa_2}}
}
\end{equation}
\begin{equation}\label{rule:cequal-prl-1}
\inferrule{
  \cequalX{c}{c'}{\akdbprod{\kappa_1}{u}{\kappa_2}}
}{
  \cequalX{\acprl{c}}{\acprl{c'}}{\kappa_1}
}
\end{equation}
\begin{equation}\label{rule:cequal-prl-2}
\inferrule{
  \haskindX{c_1}{\kappa_1}\\
  \haskindX{c_2}{\kappa_2}
}{
  \cequalX{\acprl{\acpair{c_1}{c_2}}}{c_1}{\kappa_1}
}
\end{equation}
\begin{equation}\label{rule:cequal-prr-1}
\inferrule{
  \cequalX{c}{c'}{\akdbprod{\kappa_1}{u}{\kappa_2}}
}{
  \cequalX{\acprr{c}}{\acprr{c'}}{[\acprl{c}/u]\kappa_2}
}
\end{equation}
\begin{equation}\label{rule:cequal-prr-2}
\inferrule{
  \haskindX{c_1}{\kappa_1}\\
  \haskindX{c_2}{\kappa_2}
}{
  \cequalX{\acprr{\acpair{c_1}{c_2}}}{c_2}{\kappa_2}
}
\end{equation}
\begin{equation}\label{rule:cequal-parr}
\inferrule{
  \cequalX{\tau_1}{\tau'_1}{\akty}\\
  \cequalX{\tau_2}{\tau'_2}{\akty}
}{
  \cequalX{\aparr{\tau_1}{\tau_2}}{\aparr{\tau'_1}{\tau'_2}}{\akty}
}
\end{equation}
\begin{equation}\label{rule:cequal-all}
\inferrule{
  \kequalX{\kappa}{\kappa'}\\
  \cequal{\Omega, u :: \kappa}{\tau}{\tau'}{\akty}
}{
  \cequalX{\aallu{\kappa}{u}{\tau}}{\aallu{\kappa'}{u}{\tau'}}{\akty}
}
\end{equation}
\begin{equation}\label{rule:cequal-rec}
\inferrule{
  \cequal{\Omega, t :: \akty}{\tau}{\tau'}{\akty}
}{
  \cequalX{\arec{t}{\tau}}{\arec{t}{\tau'}}{\akty}
}
\end{equation}
\begin{equation}\label{rule:cequal-prod}
\inferrule{
  \{\cequalX{\tau_i}{\tau'_i}{\akty}\}_{1 \leq i \leq n}
}{
  \cequalX{\aprod{\labelset}{\mapschema{\tau}{i}{\labelset}}}{\aprod{\labelset}{\mapschema{\tau'}{i}{\labelset}}}{\akty}
}
\end{equation}
\begin{equation}\label{rule:cequal-sum}
\inferrule{
  \{\cequalX{\tau_i}{\tau'_i}{\akty}\}_{1 \leq i \leq n}
}{
  \cequalX{\asum{\labelset}{\mapschema{\tau}{i}{\labelset}}}{\asum{\labelset}{\mapschema{\tau'}{i}{\labelset}}}{\akty}
}
\end{equation}
\begin{equation}\label{rule:cequal-sing}
\inferrule{
  \haskindX{c}{\aksing{c'}}
}{
  \cequalX{c}{c'}{\akty}
}
\end{equation}
\begin{equation}\label{rule:cequal-stat}
\inferrule{
  % \ismvalX{\astruct{c}{e}}\\
  \hassigX{\astruct{c}{e}}{\asignature{\kappa}{u}{\tau}}
}{
  \cequalX{\amcon{\astruct{c}{e}}}{c}{\kappa}
}
\end{equation}
\end{subequations}
\subsubsection{Expressions, Rules and Patterns}
% \noindent\fbox{$\strut\istypeP{\Omega}{\tau}$}~~$\tau$ is a type

% \vspace{6px}\noindent Types, $\tau$, classify expressions. The constructors of kind $\akty$ coincide with the types of $\miniVerseParam$.
% \begin{equation}\label{rule:istypeP}
% \inferrule{
%   \haskindX{\tau}{\akty}
% }{
%   \istypeP{\Omega}{\tau}
% }
% \end{equation}

% \noindent\fbox{$\strut\tequalPX{\tau}{\tau'}$}~~$\tau$ and $\tau'$ are definitionally equal types

% \vspace{6px}\noindent Type equality then coincides with constructor equality at kind $\akty$.
% \begin{equation}\label{rule:tequalP}
% \inferrule{
%   \cequalX{\tau}{\tau}{\akty}
% }{
%   \tequalPX{\tau}{\tau'}
% }
% \end{equation}


\noindent\fbox{$\strut\issubtypePX{\tau}{\tau'}$}~~$\tau$ is a subtype of $\tau'$

\begin{subequations}\label{rules:issubtypeP}  
\begin{equation}\label{rule:issubtypeP-equal}
\inferrule{
  \cequalX{\tau_1}{\tau_2}{\akty}
}{
  \issubtypePX{\tau_1}{\tau_2}
}
\end{equation}
\begin{equation}\label{rule:issubtypeP-trans}
\inferrule{
  \issubtypePX{\tau}{\tau'}\\
  \issubtypePX{\tau'}{\tau''}
}{
  \issubtypePX{\tau}{\tau''}
}
\end{equation}
\begin{equation}\label{rule:issubtypeP-parr}
\inferrule{
  \issubtypePX{\tau_1'}{\tau_1}\\
  \issubtypePX{\tau_2}{\tau_2'}
}{
  \issubtypePX{\aparr{\tau_1}{\tau_2}}{\aparr{\tau_1'}{\tau_2'}}
}
\end{equation}
\begin{equation}\label{rule:issubtypeP-all}
\inferrule{
  \ksubX{\kappa'}{\kappa}\\
  \issubtypeP{\Omega, u :: \kappa'}{\tau}{\tau'}
}{
  \issubtypePX{\aallu{\kappa}{u}{\tau}}{\aallu{\kappa'}{u}{\tau'}}
}
\end{equation}
\begin{equation}\label{rule:issubtypeP-prod}
\inferrule{
  \{\issubtypePX{\tau_i}{\tau'_i}\}_{i \in \labelset}
}{
  \issubtypePX{\aprod{\labelset}{\mapschema{\tau}{i}{\labelset}}}{\aprod{\labelset}{\mapschema{\tau'}{i}{\labelset}}}
}
\end{equation}
\begin{equation}\label{rule:issubtypeP-sum}
\inferrule{
  \{\issubtypePX{\tau_i}{\tau'_i}\}_{i \in \labelset}
}{
  \issubtypePX{\asum{\labelset}{\mapschema{\tau}{i}{\labelset}}}{\asum{\labelset}{\mapschema{\tau'}{i}{\labelset}}}
}
\end{equation}
\end{subequations}

\noindent\fbox{$\strut\hastypeP{\Omega}{e}{\tau}$}~~$e$ has type $\tau$
\begin{subequations}\label{rules:hastypeP}
\begin{equation}\label{rule:hastypeP-subsume}
\inferrule{
  \hastypeP{\Omega}{e}{\tau}\\
  \issubtypePX{\tau}{\tau'}
}{
  \hastypeP{\Omega}{e}{\tau'}
}
\end{equation}
\begin{equation}\label{rule:hastypeP-var}
  \inferrule{ }{
    \hastypeP{\Omega, \Ghyp{x}{\tau}}{x}{\tau}
  }
\end{equation}
\begin{equation}\label{rule:hastypeP-lam}
  \inferrule{
    \haskind{\Omega}{\tau}{\akty}\\
    \hastypeP{\Omega, \Ghyp{x}{\tau}}{e}{\tau'}
  }{
    \hastypeP{\Omega}{\aelam{\tau}{x}{e}}{\aparr{\tau}{\tau'}}
  }
\end{equation}
\begin{equation}\label{rule:hastypeP-ap}
  \inferrule{
    \hastypeP{\Omega}{e_1}{\aparr{\tau}{\tau'}}\\
    \hastypeP{\Omega}{e_2}{\tau}
  }{
    \hastypeP{\Omega}{\aeap{e_1}{e_2}}{\tau'}
  }
\end{equation}
\begin{equation}\label{rule:hastypeP-clam}
  \inferrule{
    \iskindX{\kappa}\\
    \hastypeP{\Omega, u :: \kappa}{e}{\tau}
  }{
    \hastypeP{\Omega}{\aeclam{\kappa}{u}{e}}{\aallu{\kappa}{u}{\tau}}
  }
\end{equation}
\begin{equation}\label{rule:hastypeP-cap}
  \inferrule{
    \hastypeP{\Omega}{e}{\aallu{\kappa}{u}{\tau}}\\
    \haskindX{c}{\kappa}
  }{
    \hastypeP{\Omega}{\aecap{e}{c}}{[c/u]\tau}
  }
\end{equation}
\begin{equation}\label{rule:hastypeP-fold}
  \inferrule{\
    % \haskind{\Omega, t :: \akty}{\tau}{\akty}\\
    \hastypeP{\Omega}{e}{[\arec{t}{\tau}/t]\tau}
  }{
    \hastypeP{\Omega}{\aefold{e}}{\arec{t}{\tau}}
  }
\end{equation}
\begin{equation}\label{rule:hastypeP-unfold}
  \inferrule{
    \hastypeP{\Omega}{e}{\arec{t}{\tau}}
  }{
    \hastypeP{\Omega}{\aeunfold{e}}{[\arec{t}{\tau}/t]\tau}
  }
\end{equation}
\begin{equation}\label{rule:hastypeP-tpl}
  \inferrule{
    \{\hastypeP{\Omega}{e_i}{\tau_i}\}_{i \in \labelset}
  }{
    \hastypeP{\Omega}{\aetpl{\labelset}{\mapschema{e}{i}{\labelset}}}{\aprod{\labelset}{\mapschema{\tau}{i}{\labelset}}}
  }
\end{equation}
\begin{equation}\label{rule:hastypeP-pr}
  \inferrule{
    \hastypeP{\Omega}{e}{\aprod{\labelset, \ell}{\mapschema{\tau}{i}{\labelset}; \ell \hookrightarrow \tau}}
  }{
    \hastypeP{\Omega}{\aepr{\ell}{e}}{\tau}
  }
\end{equation}
\begin{equation}\label{rule:hastypeP-in}
  \inferrule{
    % \{\haskind{\Omega}{\tau_i}{\akty}\}_{i \in \labelset}\\
    % \haskind{\Omega}{\tau}{\akty}\\
    \hastypeP{\Omega}{e}{\tau}
  }{
    \hastypeP{\Omega}{\aein{\ell}{e}}{\asum{\labelset, \ell}{\mapschema{\tau}{i}{\labelset}; \ell \hookrightarrow \tau}}
  }
\end{equation}
\begin{equation}\label{rule:hastypeP-match}
\inferrule{
  \hastypeP{\Omega}{e}{\tau}\\
  % \haskind{\Omega}{\tau'}{\akty}\\
  \{\ruleTypeP{\Omega}{r_i}{\tau}{\tau'}\}_{1 \leq i \leq n}\\
}{\hastypeP{\Omega}{\aematchwith{n}{e}{\seqschemaX{r}}}{\tau'}}
\end{equation}
\begin{equation}\label{rule:hastypeP-dyn}
\inferrule{
  \ismvalX{M}\\
  \hassigX{M}{\asignature{\kappa}{u}{\tau}}
}{
  \hastypeP{\Omega}{\amval{M}}{[\amcon{M}/u]\tau}
}
\end{equation}
\end{subequations}
\noindent\fbox{$\strut\ruleTypeP{\Omega}{r}{\tau}{\tau'}$}~~$r$ takes values of type $\tau$ to values of type $\tau'$
\begin{equation}\label{rule:ruleTypeP}
\inferrule{
  \patTypeP{\Omega'}{p}{\tau}\\
  \hastypeP{\Gcons{\Omega}{\Omega'}}{e}{\tau'}
}{
  \ruleTypeP{\Omega}{\aematchrule{p}{e}}{\tau}{\tau'}
}
\end{equation}

\noindent\fbox{$\strut\patTypeP{\Omega'}{p}{\tau}$}~~$p$ matches values of type $\tau$ generating hypotheses $\Omega'$

\begin{subequations}\label{rules:patTypeP}
\begin{equation}\label{rule:patTypeP-subsume}
\inferrule{
  \patTypeP{\Omega'}{p}{\tau}\\
  \issubtypePX{\tau}{\tau'}
}{
  \patTypeP{\Omega'}{p}{\tau'}
}
\end{equation}
\begin{equation}\label{rule:patTypeP-var}
\inferrule{ }{\patTypeP{\Ghyp{x}{\tau}}{x}{\tau}}
\end{equation}
\begin{equation}\label{rule:patTypeP-wild}
\inferrule{ }{\patTypeP{\emptyset}{\aewildp}{\tau}}
\end{equation}
\begin{equation}\label{rule:patTypeP-fold}
\inferrule{
  \patTypeP{\Omega'}{p}{[\arec{t}{\tau}/t]\tau}
}{
  \patTypeP{\Omega'}{\aefoldp{p}}{\arec{t}{\tau}}
}
\end{equation}
\begin{equation}\label{rule:patTypeP-tpl}
\inferrule{
  \{\patTypeP{\Omega_i}{p_i}{\tau_i}\}_{i \in \labelset}
}{
  \patTypeP{\Gconsi{i \in \labelset}{\Omega_i}}{\aetplp{\labelset}{\mapschema{p}{i}{\labelset}}}{\aprod{\labelset}{\mapschema{\tau}{i}{\labelset}}}
}
\end{equation}
\begin{equation}\label{rule:patTypeP-inj}
\inferrule{
  \patTypeP{\Omega'}{p}{\tau}
}{
  \patTypeP{\Omega'}{\aeinjp{\ell}{p}}{\asum{\labelset, \ell}{\mapschema{\tau}{i}{\labelset}; \mapitem{\ell}{\tau}}}
}
\end{equation}
\end{subequations}

\subsubsection{Metatheory}
The rules above are syntax-directed, so we assume an inversion lemma for each rule as needed without stating it separately or proving it explicitly. The following standard lemmas also hold, for all basic judgements $J$ above.

\begin{lemma}[Weakening]\label{lemma:weakening-P}  If $\Omega \vdash J$ then $\Omega \cup \Omega' \vdash J$.
% \begin{enumerate}
% \item \begin{enumerate}
%   \item If $\issigX{\sigma}$ then $\issig{\Omega \cup \Omega'}{\sigma}$.
%   \item If $\sigequal{\Omega}{\sigma}{\sigma'}$ then $\sigequal{\Omega \cup \Omega'}{\sigma}{\sigma'}$.
%   \item If $\sigsub{\Omega}{\sigma}{\sigma'}$ then $\sigsub{\Omega \cup \Omega'}{\sigma}{\sigma'}$.
%   \item If $\hassigX{M}{\sigma}$ then $\hassig{\Omega \cup \Omega'}{M}{\sigma}$.
%   \item If $\ismvalX{M}$ then $\ismval{\Omega \cup \Omega'}{M}$.
%   \end{enumerate}
% \item \begin{enumerate}
% \item If $\iskindX{\kappa}$ then $\iskind{\Omega \cup \Omega'}{\kappa}$.
% \item If $\kequalX{\kappa}{\kappa'}$ then $\kequal{\Omega \cup \Omega'}{\kappa}{\kappa'}$.
% \item If $\ksubX{\kappa}{\kappa'}$ then $\ksub{\Omega \cup \Omega'}{\kappa}{\kappa'}$.
% \item If $\haskindX{c}{\kappa}$ then $\haskind{\Omega \cup \Omega'}{c}{\kappa}$.
% \item If $\cequalX{c}{c'}{\kappa}$ then $\cequal{\Omega \cup \Omega'}{c}{c'}{\kappa}$.
% \end{enumerate}
% \item \begin{enumerate}
% \item If $\istypeP{\Omega}{\tau}$ then $\istypeP{\Omega \cup \Omega'}{\tau}$.
% \item If $\tequalPX{\tau}{\tau'}$ then $\tequalP{\Omega \cup \Omega'}{\tau}{\tau'}$.
% \item If $\issubtypePX{\tau}{\tau'}$ then $\issubtypeP{\Omega \cup \Omega'}{\tau}{\tau'}$.
% \item If $\hastypeP{\Omega}{e}{\tau}$ then $\hastypeP{\Omega \cup \Omega'}{e}{\tau}$.
% \item If $\ruleTypeP{\Omega}{r}{\tau}{\tau'}$ then $\ruleTypeP{\Omega \cup \Omega'}{r}{\tau}{\tau'}$.
% \item If $\patTypePC{\Omega}{\Omega''}{p}{\tau}$ then $\patTypePC{\Omega \cup \Omega'}{\Omega''}{p}{\tau}$.
% \end{enumerate}
% \end{enumerate}
\end{lemma}
\begin{proof-sketch} By straightforward mutual rule induction.
\end{proof-sketch}

A \emph{substitution}, $\omega$, is a finite function that maps:
\begin{itemize}
\item each $X \in \domof{\omega}$ to a module expression subtitution, $M/X$; 
\item each $u \in \domof{\omega}$ to a constructor substitution, $c/u$; and 
\item each $x \in \domof{\omega}$ to an expression substitution, $e/x$.
\end{itemize}

We write $\hastypeP{\Omega}{\omega}{\Omega'}$ iff $\domof{\omega}=\domof{\Omega'}$ and:
\begin{itemize}
\item for each $M/X \in \omega$, we have $X : \sigma \in \Omega'$ and $\hassigX{M}{\sigma}$ and $\ismvalX{M}$; and
\item for each $c/u \in \omega$, we have $u :: \kappa \in \Omega'$ and $\haskindX{c}{\kappa}$; and 
\item for each $e/x \in \omega$, we have $x : \tau \in \Omega'$ and $\hastypeP{\Omega}{e}{\tau}$.
\end{itemize}

We simultaneously apply a substitution by placing it in prefix position. For example, $[\omega]e$ applies the substitutions $\omega$ simultaneously to $e$.

\begin{lemma}[Substitution]\label{lemma:substitution-P} If $\Omega \cup \Omega' \cup \Omega'' \vdash J$ and $\hastypeP{\Omega}{\omega}{\Omega'}$ then $\Omega \cup [\omega]\Omega'' \vdash [\omega]J$.
\end{lemma}
\begin{proof-sketch} By straightforward rule induction. 
\end{proof-sketch}

\begin{lemma}[Decomposition]\label{lemma:decomposition-P} 
If $\Omega \cup [\omega]\Omega'' \vdash [\omega]J$ and $\hastypeP{\Omega}{\omega}{\Omega'}$ then $\Omega \cup \Omega' \cup \Omega'' \vdash J$.
\end{lemma}
\begin{proof-sketch} By straightforward rule induction.
\end{proof-sketch}

% \begin{lemma}[Regularity]\label{lemma:regularity-P} ~
% \begin{enumerate}
% \item ...
% \end{enumerate}
% \end{lemma}

\subsection{Structural Dynamics}
The structural dynamics of modules is defined as a transition system, and is organized around judgements of the following form:

\vspace{10px}
$\begin{array}{ll}
\textbf{Judgement Form} & \textbf{Description}\\
\stepsU{M}{M'} & \text{$M$ transitions to $M'$}\\
\isvalP{M} & \text{$M$ is a module value}\\
\matchfail{M} & \text{$M$ raises match failure}
\end{array}$
\vspace{10px}

The structural dynamics of expressions is also defined as a transition system, and is organized around judgements of the following form:

\vspace{10px}
$\begin{array}{ll}
\textbf{Judgement Form} & \textbf{Description}\\
\stepsU{e}{e'} & \text{$e$ transitions to $e'$}\\
\isvalP{e} & \text{$e$ is a value}\\
\matchfail{e} & \text{$e$ raises match failure}
\end{array}$
\vspace{10px}

We also define auxiliary judgements for \emph{iterated transition}, $\multistepU{e}{e'}$, and \emph{evaluation}, $\evalU{e}{e'}$ of expressions.

\begin{definition}[Iterated Transition]\label{defn:iterated-transition-P} Iterated transition, $\multistepU{e}{e'}$, is the reflexive, transitive closure of the transition judgement, $\stepsU{e}{e'}$.\end{definition}

\begin{definition}[Evaluation]\label{defn:evaluation-P} $\evalU{e}{e'}$ iff $\multistepU{e}{e'}$ and $\isvalU{e'}$. \end{definition}

Similarly, we lift these definitions to the level of module expressions as well.

\begin{definition}[Iterated Module Transition]\label{defn:iterated-transition-modules-P} Iterated transition, $\multistepU{M}{M'}$, is the reflexive, transitive closure of the transition judgement, $\stepsU{M}{M'}$.\end{definition}

\begin{definition}[Module Evaluation]\label{defn:evaluation-modules-P} $\evalU{M}{M'}$ iff $\multistepU{M}{M'}$ and $\isvalU{M'}$. \end{definition}



As in $\miniVersePat$, our subsequent developments do not make mention of particular rules in the dynamics, nor do they make mention of other judgements, not listed above, that are used only for defining the dynamics of the match operator, so we do not produce these details here. Instead, it suffices to state the following conditions.

The Preservation condition ensures that evaluation preserves typing.
\begin{condition}[Preservation]\label{condition:preservation-P} ~
\begin{enumerate}
\item If $\hassig{}{M}{\sigma}$ and $\stepsU{M}{M'}$ then $\hassig{}{M}{\sigma}$.
\item If $\hastypeUC{e}{\tau}$ and $\stepsU{e}{e'}$ then $\hastypeUC{e'}{\tau}$.
\end{enumerate}
\end{condition}

The Progress condition ensures that evaluation of a well-typed expanded expression cannot ``get stuck''. We must consider the possibility of match failure in this condition.
\begin{condition}[Progress]\label{condition:progress-P} ~
\begin{enumerate}
\item If $\hassig{}{M}{\sigma}$ then either $\isvalU{M}$ or $\matchfail{M}$ or there exists an $M'$ such that $\stepsU{M}{M'}$.
\item If $\hastypeUC{e}{\tau}$ then either $\isvalU{e}$ or $\matchfail{e}$ or there exists an $e'$ such that $\stepsU{e}{e'}$.
\end{enumerate}
\end{condition}

\section{Unexpanded Language (UL)}
\subsection{Syntax}
\subsubsection{Stylized Syntax -- Unexpanded Signatures and Modules}
\[\begin{array}{lllllll}
\textbf{Sort} & & 
%& \textbf{Operational Form} 
& \textbf{Stylized Form} & \textbf{Description}\\
\mathsf{USig} & \usigma & ::= 
%& \ausignature{\ukappa}{\uu}{\utau} 
& \signature{\uu}{\ukappa}{\utau} & \text{signature}\\
\mathsf{UMod} & \uM & ::= 
%& \uX 
& \uX & \text{module identifier}\\
&&
%& \austruct{\uc}{\ue} 
& \struct{\uc}{\ue} & \text{structure}\\
&&
%& \auseal{\usigma}{\uM} 
& \seal{\uM}{\usigma} & \text{seal}\\
&&
%& \aumlet{\usigma}{\uM}{\uX}{\uM} 
& \mlet{\uX}{\uM}{\uM}{\usigma} & \text{definition}\\
% \LCC &&
%& \lightgray 
% & \color{Yellow} & \color{Yellow}\\
&&
%& \aumdefpetsm{\urho}{e}{\tsmv}{\uM} 
& \defpetsm{\tsmv}{\urho}{e}{\uM} & \text{peTSM definition}\\
%&&&                                    & \texttt{expressions}~\{e\}~\texttt{in}~\uM\\
&&
%& \aumletpetsm{\uepsilon}{\tsmv}{\uM} 
& \uletpetsm{\tsmv}{\uepsilon}{\uM} & \text{peTSM binding}\\
% &&&                                  & \texttt{expressions}~\texttt{in}~\uM\\
% &&& ... & ... & \text{peTSM designation}\\
&&
%& \audefpptsm{\urho}{e}{\tsmv}{\uM} 
& \defpptsm{\tsmv}{\urho}{e}{\uM} & \text{ppTSM definition}\\
% &&&                                    & \texttt{patterns}~\{e\}~\texttt{in}~\uM\\
&&
%& \auletpptsm{\uepsilon}{\tsmv}{\uM} 
& \uletpptsm{\tsmv}{\uepsilon}{\uM} & \text{ppTSM binding}%\ECC%
% &&& & \texttt{patterns}~\texttt{in}~\uM\\
% &&& ... & ... & \text{ppTSM designation}\ECC
\end{array}\]%\vspace{-15px}
% \caption[Syntax of unexpanded module expressions and signatures in $\miniVerseParam$]{Syntax of unexpanded module expressions and signatures in $\miniVerseParam$.}\vspace{-5px}
% \label{fig:P-unexpanded-modules-signatures}
% \end{figure}
% \begin{figure}[p] \vspace{-10px}

\subsubsection{Stylized Syntax -- Unexpanded Kinds and Constructors}
\[\begin{array}{lrlllll}
\textbf{Sort} & & 
%& \textbf{Operational Form} 
& \textbf{Stylized Form} & \textbf{Description}\\
\mathsf{UKind} & \ukappa & ::= 
%& \aukdarr{\ukappa}{\uu}{\ukappa} 
& \kdarr{\uu}{\ukappa}{\ukappa} & \text{dependent function}\\
&&
%& \aukunit 
& \kunit & \text{nullary product}\\
&&
%& \aukdbprod{\ukappa}{\uu}{\ukappa} 
& \kdbprod{\uu}{\ukappa}{\ukappa} & \text{dependent product}\\
%&&& \akdprodstd & \kdprodstd & \text{labeled dependent product}\\
&&
%& \aukty 
& \kty & \text{types}\\
&&
%& \auksing{\utau} 
& \ksing{\utau} & \text{singleton}\\
\mathsf{UCon} & \uc, \utau & ::= 
%& \uu 
& \uu & \text{constructor identifier}\\
&&
%& \ut 
& \ut & \\
&&
%& \aucasc{\ukappa}{\uc} 
& \casc{\uc}{\ukappa} & \text{ascription}\\
&&
%& \aucabs{\uu}{\uc} 
& \cabs{\uu}{\uc} & \text{abstraction}\\
&&
%& \aucapp{c}{c} 
& \capp{c}{c} & \text{application}\\
&&
%& \auctriv 
& \ctriv & \text{trivial}\\
&&
%& \aucpair{\uc}{\uc} 
& \cpair{\uc}{\uc} & \text{pair}\\
&&
%& \aucprl{\uc} 
& \cprl{\uc} & \text{left projection}\\
&&
%& \aucprr{\uc} 
& \cprr{\uc} & \text{right projection}\\
%&&& \adtplX & \dtplX & \text{labeled dependent tuple}\\
%&&& \adprj{\ell}{c} & \prj{c}{\ell} & \text{projection}\\
&&
%& \auparr{\utau}{\utau} 
& \parr{\utau}{\utau} & \text{partial function}\\
&&
%& \auallu{\ukappa}{\uu}{\utau} 
& \forallu{\uu}{\ukappa}{\utau} & \text{polymorphic}\\
&&
%& \aurec{\ut}{\utau} 
& \rect{\ut}{\utau} & \text{recursive}\\
&&
%& \auprod{\labelset}{\mapschema{\utau}{i}{\labelset}} 
& \prodt{\mapschema{\utau}{i}{\labelset}} & \text{labeled product}\\
&&
%& \ausum{\labelset}{\mapschema{\utau}{i}{\labelset}} 
& \sumt{\mapschema{\utau}{i}{\labelset}} & \text{labeled sum}\\
&&
%& \aumcon{\uX} 
& \mcon{\uX} & \text{constructor component}
\end{array}\]%\vspace{-15px}
% \caption[Syntax of unexpanded kinds and constructors in $\miniVerseParam$]{Syntax of unexpanded kinds and constructors in $\miniVerseParam$.}\vspace{-10px}
% \label{fig:P-unexpanded-kinds-constructors}
% \end{figure}
\clearpage

\subsubsection{Stylized Syntax -- Unexpanded Expressions, Rules and Patterns}
% \clearpage
% \begin{figure}[p]
\[\begin{array}{lllllll}
\textbf{Sort} & & 
%& \textbf{Operational Form} 
& \textbf{Stylized Form} & \textbf{Description}\\
\mathsf{UExp} & \ue & ::= 
%& \ux 
& \ux & \text{identifier}\\
&&
% & \auasc{\utau}{\ue} 
& \asc{\ue}{\utau} & \text{ascription}\\
&&
% & \auletsyn{\ux}{\ue}{\ue} 
& \letsyn{\ux}{\ue}{\ue} & \text{value binding}\\
% &&
%& \auanalam{\ux}{\ue} 
% & \analam{\ux}{\ue} & \text{abstraction (unannotated)}\\
&&
%& \aulam{\utau}{\ux}{\ue} 
& \lam{\ux}{\utau}{\ue} & \text{abstraction}\\
&&
%& \auap{\ue}{\ue} 
& \ap{\ue}{\ue} & \text{application}\\
&&
%& \auclam{\ukappa}{\uu}{\ue} 
& \clam{\uu}{\ukappa}{\ue} & \text{constructor abstraction}\\
&&
%& \aucap{\ue}{\uc} 
& \cAp{\ue}{\uc} & \text{constructor application}\\
&&
%& \auanafold{\ue} 
& \fold{\ue} & \text{fold}\\
&&
%& \auunfold{\ue} 
& \unfold{\ue} & \text{unfold}\\
&&
%& \autpl{\labelset}{\mapschema{\ue}{i}{\labelset}} 
& \tpl{\mapschema{\ue}{i}{\labelset}} & \text{labeled tuple}\\
&&
%& \aupr{\ell}{\ue} 
& \prj{\ue}{\ell} & \text{projection}\\
&&
%& \auanain{\ell}{\ue} 
& \inj{\ell}{\ue} & \text{injection}\\
&&
%& \aumatchwithb{n}{\ue}{\seqschemaX{\urv}} 
& \matchwith{\ue}{\seqschemaX{\urv}} & \text{match}\\
&&
%& \aumval{\uX} 
& \mval{\uX} & \text{value component}\\
% \LCC &&
% %& \color{Yellow} 
% & \color{Yellow} & \color{Yellow} \\
% &&& \audefpetsm{\urho}{e}{\tsmv}{\ue} & \texttt{syntax}~\tsmv~\texttt{at}~\urho~\texttt{for} & \text{peTSM definition}\\
% &&&                                    & \texttt{expressions}~\{e\}~\texttt{in}~\ue\\
% &&& \auletpetsm{\uepsilon}{\tsmv}{\ue} & \texttt{let}~\texttt{syntax}~\tsmv=\uepsilon~\texttt{for} & \text{peTSM binding}\\
% &&&                                  & \texttt{expressions}~\texttt{in}~\ue\\
% &&& ... & ... & \text{peTSM designation}\\
&&
%& \auappetsm{b}{\uepsilon} 
& \utsmap{\uepsilon}{b} & \text{peTSM application}\\%\ECC\\%\ECC
% &&& \auelit{b} & {\lit{b}}  & \text{peTSM unadorned literal}\\
% &&& \audefpptsm{\urho}{e}{\tsmv}{\ue} & \texttt{syntax}~\tsmv~\texttt{at}~\urho~\texttt{for} & \text{ppTSM definition}\\
% &&&                                    & \texttt{patterns}~\{e\}~\texttt{in}~\ue\\
% &&& \auletpptsm{\uepsilon}{\tsmv}{\ue} & \texttt{let}~\texttt{syntax}~\tsmv=\uepsilon~\texttt{for} & \text{ppTSM binding}\\
% &&& & \texttt{patterns}~\texttt{in}~\ue\\
% &&& ... & ... & \text{ppTSM designation}\\\ECC
\mathsf{URule} & \urv & ::= 
%& \aumatchrule{\upv}{\ue} 
& \matchrule{\upv}{\ue} & \text{match rule}\\
\mathsf{UPat} & \upv & ::= 
%& \ux 
& \ux & \text{identifier pattern}\\
&&
%& \auwildp 
& \wildp & \text{wildcard pattern}\\
&&
%& \aufoldp{\upv} 
& \foldp{\upv} & \text{fold pattern}\\
&&
%& \autplp{\labelset}{\mapschema{\upv}{i}{\labelset}} 
& \tplp{\mapschema{\upv}{i}{\labelset}} & \text{labeled tuple pattern}\\
&&
%& \auinjp{\ell}{\upv} 
& \injp{\ell}{\upv} 
& \text{injection pattern}\\
% \LCC &&
%& \lightgray 
% & \color{Yellow} & \color{Yellow}\\
&&
%& \auappptsm{b}{\uepsilon} 
& \utsmap{\uepsilon}{b} & \text{ppTSM application}%\ECC
% &&& \auplit{b} & \lit{b} & \text{ppTSM unadorned literal}\ECC
\end{array}\]
% \caption[Syntax of unexpanded expressions, rules and patterns in $\miniVerseParam$]{Syntax of unexpanded expressions, rules and patterns in $\miniVerseParam$.}
% \label{fig:P-unexpanded-terms}
% \end{figure}

\subsubsection{Stylized Syntax -- Unexpanded TSM Types and Expressions}
% \begin{figure}[p]
\[\begin{array}{lllllll}
\textbf{Sort} & & 
%& \textbf{Operational Form} 
& \textbf{Stylized Form} 
& \textbf{Description}\\
% \LCC \color{Yellow}&\color{Yellow}& \color{Yellow}
%& \lightgray 
% & \color{Yellow} & \color{Yellow}\\
\mathsf{UMType} & \urho & ::= 
%& \autype{\utau} 
& \utau & \text{type annotation}\\
&&
%& \aualltypes{\ut}{\urho} 
& \alltypes{\ut}{\urho} & \text{type parameterization}\\
&&
%& \auallmods{\usigma}{\uX}{\urho} 
& \allmods{\uX}{\usigma}{\urho} & \text{module parameterization}\\
\mathsf{UMExp} & \uepsilon & ::= 
%& \abindref{\tsmv} 
& \tsmv & \text{TSM binding reference}\\
&&
%& \auabstype{\ut}{\uepsilon} 
& \abstype{\ut}{\uepsilon} & \text{type abstraction}\\
&&
%& \auabsmod{\usigma}{\uX}{\uepsilon} 
& \absmod{\uX}{\usigma}{\uepsilon} & \text{module abstraction}\\
&&
%& \auaptype{\utau}{\uepsilon} 
& \aptype{\uepsilon}{\utau} & \text{type application}\\
&&
%& \auapmod{\uM}{\uepsilon} 
& \apmod{\uepsilon}{\uX} & \text{module application}%\ECC
\end{array}
\]
% \caption{Syntax of unexpanded TSM types and expressions.}
% \label{fig:P-macro-expressions-types-u}
% \end{figure}

\subsubsection{Stylized Syntax -- TSM Types and Expressions}

% \clearpage
% \begin{figure}[p]
\[\begin{array}{lllllll}
\textbf{Sort} & & & \textbf{Operational Form} 
%& \textbf{Stylized Form} 
& \textbf{Description}\\
% \LCC \color{Yellow}&\color{Yellow}& \color{Yellow}
%& \lightgray 
% & \color{Yellow} & \color{Yellow}\\
\mathsf{MType} & \rho & ::= & \aetype{\tau} 
%& \tau 
& \text{type annotation}\\
&&& \aealltypes{t}{\rho} 
%& \alltypes{t}{\rho} 
& \text{type parameterization}\\
&&& \aeallmods{\sigma}{X}{\rho} 
%& \allmods{X}{\sigma}{\rho} 
& \text{module parameterization}\\
\mathsf{MExp} & \epsilon & ::= & \adefref{a} 
%& a 
& \text{TSM definition reference}\\
&&& \aeabstype{t}{\epsilon} 
%& \abstype{t}{\epsilon} 
& \text{type abstraction}\\
&&& \aeabsmod{\sigma}{X}{\epsilon} 
%& \absmod{X}{\sigma}{\epsilon} 
& \text{module abstraction}\\
&&& \aeaptype{\tau}{\epsilon} 
%& \aptype{\epsilon}{\tau} 
& \text{type application}\\
&&& \aeapmod{M}{\epsilon} 
%& \aptype{\epsilon}{M} 
& \text{module application}%\ECC
\end{array}\]
% \caption[Syntax of TSM types and expressions in $\miniVerseParam$]{Syntax of TSM types and expressions.}
% \label{fig:P-macro-expressions-types}
% \end{figure}

\subsubsection{Body Lengths}
We write $\sizeof{b}$ for the length of $b$. 
The metafunction $\sizeof{\uM}$ computes the sum of the lengths of expression literal bodies in $\uM$:
\[
\begin{array}{ll}
\sizeof{\uX} & = 0\\
\sizeof{\struct{\uc}{\ue}} & = \sizeof{\ue}\\
\sizeof{\seal{\uM}{\usigma}} & = \sizeof{\uM}\\
\sizeof{\mlet{\uX}{\uM}{\uM'}{\usigma}} & = \sizeof{\uM} + \sizeof{\uM'}\\
\sizeof{\defpetsm{\tsmv}{\urho}{e}{\uM}} & = \sizeof{\uM}\\
\sizeof{\uletpetsm{\tsmv}{\uepsilon}{\uM}} & = \sizeof{\uM}\\
\sizeof{\defpptsm{\tsmv}{\urho}{e}{\uM}} & = \sizeof{\uM}\\
\sizeof{\uletpptsm{\tsmv}{\uepsilon}{\uM}} & = \sizeof{\uM}
\end{array}
\]
and $\sizeof{\ue}$ computes the sum of the lengths of expression literal bodies in $\ue$:
\[
\begin{array}{ll}
\sizeof{\ux} & = 0\\
\sizeof{\lam{\ux}{\utau}{\ue}} &= \sizeof{\ue}\\
\sizeof{\ap{\ue_1}{\ue_2}} & = \sizeof{\ue_1} + \sizeof{\ue_2}\\
\sizeof{\clam{\uu}{\ukappa}{\ue}} & = \sizeof{\ue}\\
\sizeof{\cAp{\ue}{\uc}} & = \sizeof{\ue}\\
\sizeof{\fold{\ue}} & = \sizeof{\ue}\\
\sizeof{\unfold{\ue}} & = \sizeof{\ue}\\
%\end{align*}
%\begin{align*}
\sizeof{\tpl{\mapschema{\ue}{i}{\labelset}}} & = \sum_{i \in \labelset} \sizeof{\ue_i}\\
\sizeof{\prj{\ell}{\ue}} & = \sizeof{\ue}\\
\sizeof{\inj{\ell}{\ue}} & = \sizeof{\ue}\\
\sizeof{\matchwith{\ue}{\seqschemaX{\urv}}} & = \sizeof{\ue} + \sum_{1 \leq i \leq n} \sizeof{r_i}\\
\sizeof{\mval{\uX}} & = 0\\
% \sizeof{\caseof{\ue}{\mapschemab{\ux}{\ue}{i}{\labelset}}} & = \sizeof{\ue} + \sum_{i \in \labelset} \sizeof{\ue_i}\\
% \sizeof{\uesyntax{\tsmv}{\utau}{\eparse}{\ue}} & = \sizeof{\ue}\\
\sizeof{\utsmap{\uepsilon}{b}} & = \sizeof{b}
\end{array}
\]
and $\sizeof{\urv}$ computes the sum of the lengths of expression literal bodies in $\urv$:
\begin{align*}
\sizeof{\matchrule{\upv}{\ue}} & = \sizeof{\ue}
\end{align*}
% and $\sizeof{\uepsilon}$ computes the sum of the lengths of expression literal bodies in $\uepsilon$:
% \begin{align*}
% \sizeof{\tsmv} & = 0\\
% \sizeof{\abstype{\ut}{\uepsilon}} & = \sizeof{\uepsilon}\\
% \sizeof{\absmod{\uX}{\usigma}{\uepsilon}} & = \sizeof{\uepsilon}\\
% \sizeof{\aptype{\uepsilon}{\utau}} & = 0\\
% \sizeof{\apmod{\uepsilon}{\uM}} & = \sizeof{\uM}
% \end{align*}

Similarly, the metafunction $\sizeof{\upv}$ computes the sum of the lengths of the pattern literal bodies in $\upv$:
\begin{align*}
\sizeof{\ux} & = 0\\
\sizeof{\foldp{\upv}} & = \sizeof{\upv}\\
\sizeof{\tplp{\mapschema{\upv}{i}{\labelset}}} & = \sum_{i \in \labelset} \sizeof{\upv_i}\\
\sizeof{\injp{\ell}{\upv}} & = \sizeof{\upv}\\
\sizeof{\utsmap{\uepsilon}{b}} & = \sizeof{b}
\end{align*}

\subsubsection{Common Unexpanded Forms}\label{appendix:P-shared-forms}
Each expanded form, with a few minor exceptions noted below, maps onto an unexpanded form. We refer to these as the \emph{common forms}. In particular:
\begin{itemize}
\item Each module variable, $X$, maps onto a unique module identifier, written $\sigilof{X}$.
\item Each signature, $\sigma$, maps onto an unexpanded signature, $\Uof{\sigma}$, as follows:
\begin{align*}
\Uof{\asignature{\kappa}{u}{c}} & = \signature{\sigilof{u}}{\Uof{\kappa}}{\Uof{c}}
\end{align*}
\item Each module expression, $M$, maps onto an unexpanded module expression, $\uM$, as follows:
\begin{align*}
\Uof{X} & = \sigilof{X}\\
\Uof{\astruct{\uc}{\ue}} & = \struct{\Uof{\uc}}{\Uof{\ue}}\\
\Uof{\aseal{\sigma}{M}} & = \seal{\Uof{M}}{\Uof{\sigma}}\\
\Uof{\amlet{\sigma}{M}{X}{M'}} & = \mlet{\sigilof{X}}{\Uof{M}}{\Uof{M'}}{\Uof{\sigma}}
\end{align*}
\item Each constructor variable, $u$, maps onto a unique {type identifier}, written $\sigilof{u}$.
\item Each kind, $\kappa$, maps onto an unexpanded kind, $\Uof{\kappa}$, as follows:
\begin{align*}
\Uof{\akdarr{\kappa}{u}{\kappa'}} & = \kdarr{\sigilof{u}}{\Uof{\kappa}}{\Uof{\kappa'}}\\
\Uof{\akunit} & = \kunit\\
\Uof{\akdbprod{\kappa}{u}{\kappa'}} & = \kdbprod{\sigilof{u}}{\Uof{\kappa}}{\Uof{\kappa'}}\\
\Uof{\akty} & = \kty\\
\Uof{\aksing{\tau}} & = \ksing{\Uof{\tau}}
\end{align*}
\item Each constructor, $c$, except for constructors of the form $\amcon{M}$ where $M$ is not a module variable, maps onto an unexpanded type, $\Uof{c}$, as follows: 
  \begin{align*}
  \Uof{u} &= \sigilof{u}\\
  \Uof{\acabs{u}{c}} & = \cabs{\sigilof{u}}{\Uof{c}}\\
  \Uof{\acapp{c}{c'}} & = \capp{\Uof{c}}{\Uof{c'}}\\
  \Uof{\actriv} & = \ctriv\\
  \Uof{\acpair{c}{c'}} & = \cpair{\Uof{c}}{\Uof{c'}}\\
  \Uof{\acprl{c}} & = \cprl{\Uof{c}}\\
  \Uof{\acprr{c}} & = \cprr{\Uof{c}}\\
  \Uof{\aparr{\tau_1}{\tau_2}} & = \parr{\Uof{\tau_1}}{\Uof{\tau_2}}\\
  \Uof{\aallu{\kappa}{u}{\tau}} & = \forallu{\sigilof{u}}{\Uof{\kappa}}{\Uof{\tau}}\\
  \Uof{\arec{t}{\tau}} & = \rect{\sigilof{t}}{\Uof{\tau}}\\
  \Uof{\aprod{\labelset}{\mapschema{\tau}{i}{\labelset}}} & = \prodt{\mapschemax{\Uofv}{\tau}{i}{\labelset}}\\
  \Uof{\asum{\labelset}{\mapschema{\tau}{i}{\labelset}}} & = \sumt{\mapschemax{\Uofv}{\tau}{i}{\labelset}}\\
  \Uof{\amcon{X}} & = \mcon{\sigilof{X}}
  \end{align*}
\item Each expression variable, $x$, maps onto a unique expression identifier, written $\sigilof{x}$.
\item Each expanded expression, $e$, except expressions of the form $\amval{M}$ where $M$ is not a module variable, maps onto an unexpanded expression, $\Uof{e}$, as follows:
\begin{align*}
\Uof{x} & = \sigilof{x}\\
\Uof{\aelam{\tau}{x}{e}} & = \lam{\sigilof{x}}{\Uof{\tau}}{\Uof{e}}\\
\Uof{\aeap{e_1}{e_2}} & = \ap{\Uof{e_1}}{\Uof{e_2}}\\
\Uof{\aeclam{\kappa}{u}{e}} & = \clam{\sigilof{u}}{\Uof{\kappa}}{\Uof{e}}\\
\Uof{\aecap{e}{c}} & = \cAp{\Uof{e}}{\Uof{c}}\\
\Uof{\aefold{e}} & = \fold{\Uof e}\\
\Uof{\aeunfold{e}} & = \unfold{\Uof{e}}\\
\Uof{\aetpl{\labelset}{\mapschema{e}{i}{\labelset}}} & = \tpl{\mapschemax{\Uofv}{e}{i}{\labelset}}\\
\Uof{\aepr{\ell}{e}} & = \prj{\Uof{e}}{\ell}\\
\Uof{\aein{\ell}{e}} &= \inj{\ell}{\Uof{e}}\\
\Uof{\aematchwith{n}{e}{\seqschemaX{r}}} & = \matchwith{\Uof{e}}{\seqschemaXx{\Uofv}{r}}\\
\Uof{\amval{X}} & = \mval{\sigilof{X}}
\end{align*}
\end{itemize}
\begin{itemize}
\item Each expanded rule, $r$, maps onto an unexpanded rule, $\Uof{r}$, as follows:
\begin{align*}
\Uof{\aematchrule{p}{e}} & = \aumatchrule{\Uof{p}}{\Uof{e}}
\end{align*}
\item Each expanded pattern, $p$, maps onto an unexpanded pattern, $\Uof{p}$, as follows:
\begin{align*}
\Uof{x} & = \sigilof{x}\\
\Uof{\aewildp} &= \auwildp\\
\Uof{\aefoldp{p}} &= \aufoldp{\Uof{p}}\\
\Uof{\aetplp{\labelset}{\mapschema{p}{i}{\labelset}}} & = \autplp{\labelset}{\mapschemax{\Uofv}{p}{i}{\labelset}}\\
\Uof{\aeinjp{\ell}{p}} & = \auinjp{\ell}{\Uof{p}}
\end{align*}
\end{itemize}

\subsubsection{Textual Syntax}
There is also a context-free textual syntax for the UL. We need only posit the existence of partial metafunctions that satisfy the following condition. 
\begin{condition}[Textual Representability]\label{condition:textual-representability-P} ~
\begin{enumerate}
% \item For each $\usigma$, there exists $b$ such that $\parseUSig{b}{\usigma}$.
% \item For each $\uM$, there exists $b$ such that $\parseUMod{b}{\uM}$.
\item For each $\ukappa$, there exists $b$ such that $\parseUKind{b}{\ukappa}$.
\item For each $\uc$, there exists $b$ such that $\parseUCon{b}{\uc}$.
\item For each $\ue$, there exists $b$ such that $\parseUExp{b}{\ue}$.
\item For each $\upv$, there exists $b$ such that $\parseUPat{b}{\upv}$.
\end{enumerate}
\end{condition}

\begin{condition}[Expression Parsing Monotonicity]\label{condition:body-parsing-P} If $\parseUExp{b}{\ue}$ then $\sizeof{\ue} < \sizeof{b}$.\end{condition}

\begin{condition}[Pattern Parsing Monotonicity]\label{condition:pattern-parsing-P} If $\parseUPat{b}{\upv}$ then $\sizeof{\upv} < \sizeof{b}$.\end{condition}

\subsection{Typed Expansion}\label{appendix:typed-expansion-P}
\subsubsection{Unexpanded Unified Contexts}\label{appendix:u-unified-ctxs}
A \emph{unexpanded unified context}, $\uOmega$, takes the form $\uOmegaEx{\uD}{\uG}{\uMctx}{\Omega}$, where $\uMctx$ is a \emph{module identifier expansion context}, $\uD$ is a \emph{constructor identifier expansion context}, $\uG$ is an \emph{expression identifier expansion context}, and $\Omega$ is a unified context.

% \subsubsection{Identifier Expansion Contexts}
A module identifier expansion context, $\uMctx$, is a finite function that maps each module identifier $\uX \in \domof{\uMctx}$ to the module identifier expansion $\vExpands{\uX}{X}$. We write $\uOmega, \uMhyp{\uX}{X}{\sigma}$ when $\uOmega=\uOmegaEx{\uD}{\uG}{\uMctx}{\Omega}$ as an abbreviation of \[\uOmegaEx{\uD}{\uG}{\uMctx \uplus \vExpands{\uX}{X}}{\Omega, X : \sigma}\]

A constructor identifier expansion context, $\uD$, is a finite function that maps each constructor identifier $\uu \in \domof{\uD}$ to the constructor identifier expansion $\vExpands{\uu}{u}$. We write $\uOmega, \uKhyp{\uu}{u}{\kappa}$ when $\uOmega=\uOmegaEx{\uD}{\uG}{\uMctx}{\Omega}$ as an abbreviation of \[\uOmegaEx{\uD \uplus \vExpands{\uu}{u}}{\uG}{\uMctx}{\Omega, u :: \kappa}\]

An expression identifier expansion context, $\uG$, is a finite function that maps each expression identifier $\ux \in \domof{\uG}$ to the expression identifier expansion $\vExpands{\ux}{x}$. We write $\uOmega, \uGhyp{\ux}{x}{\tau}$ when $\uOmega=\uOmegaEx{\uD}{\uG}{\uMctx}{\Omega}$ as an abbreviation of \[\uOmegaEx{\uD}{\uG \uplus \vExpands{\ux}{x}}{\uMctx}{\Omega, x : \tau}\]

\subsubsection{Body Encoding and Decoding}
An assumed type abbreviated $\tBody$ classifies encodings of literal bodies, $b$. The mapping from literal bodies to values of type $\tBody$ is defined by the \emph{body encoding judgement} $\encodeBody{b}{\ebody}$. An inverse mapping is defined   by the \emph{body decoding judgement} $\decodeBody{\ebody}{b}$.
\[\begin{array}{ll}
\textbf{Judgement Form} & \textbf{Description}\\
\encodeBody{b}{e} & \text{$b$ has encoding $e$}\\
\decodeBody{e}{b} & \text{$e$ has decoding $b$}
\end{array}\]
The following condition establishes an isomorphism between literal bodies and values of type $\tBody$ mediated by the judgements above.
\begin{condition}[Body Isomorphism]\label{condition:body-isomorphism-P} ~
\begin{enumerate}
\item For every literal body $b$, we have that $\encodeBody{b}{\ebody}$ for some $\ebody$ such that $\hastypeUC{\ebody}{\tBody}$ and $\isvalU{\ebody}$.
\item If $\hastypeUC{\ebody}{\tBody}$ and $\isvalU{\ebody}$ then $\decodeBody{\ebody}{b}$ for some $b$.
\item If $\encodeBody{b}{\ebody}$ then $\decodeBody{\ebody}{b}$.
\item If $\hastypeUC{\ebody}{\tBody}$ and $\isvalU{\ebody}$ and $\decodeBody{\ebody}{b}$ then $\encodeBody{b}{\ebody}$. 
\item If $\encodeBody{b}{\ebody}$ and $\encodeBody{b}{\ebody'}$ then $\ebody = \ebody'$.
\item If $\hastypeUC{\ebody}{\tBody}$ and $\isvalU{\ebody}$ and $\decodeBody{\ebody}{b}$ and $\decodeBody{\ebody}{b'}$ then $b=b'$.
\end{enumerate}
\end{condition}
We also assume a partial metafunction, $\bsubseq{b}{m}{n}$, which extracts a subsequence of $b$ starting at position $m$ and ending at position $n$, inclusive, where $m$ and $n$ are natural numbers. The following condition is technically necessary.
\begin{condition}[Body Subsequencing]\label{condition:body-subsequences-P} If $\bsubseq{b}{m}{n}=b'$ then $\sizeof{b'} \leq \sizeof{b}$. \end{condition}

\subsubsection{Parse Results}
 The type function abbreviated $\tParseResultF$, and auxiliary abbreviations used below, is defined as follows:
\begin{align*}
L_\mathtt{P} & \defeq \lbltxt{ParseError}, \lbltxt{Success}\\
\tParseResultF & \defeq \acabs{t}{\asum{L_\mathtt{P}}{
  \mapitem{\lbltxt{ParseError}}{\prodt{}}, 
  \mapitem{\lbltxt{Success}}{t}
}}\\
\tParseResult{\tau} & \defeq \acapp{\tParseResultF}{\tau}\\
\lbltxt{SuccessE}\cdot e & \defeq \aein{L_\mathtt{P}}{\mathtt{Success}}{\mapitem{\mathtt{ParseError}}{\tpl{}}, \mapitem{\mathtt{Success}}{\tPProtoExpr}}{e}\\
\lbltxt{SuccessP}\cdot e & \defeq \aein{L_\mathtt{P}}{\mathtt{Success}}{\mapitem{\mathtt{ParseError}}{\tpl{}}, \mapitem{\mathtt{Success}}{\tCEPat}}{e}
\end{align*} %[\mapitem{\lbltxt{ParseError}}{\prodt{}}, \mapitem{\lbltxt{SuccessE}}{\tCEExp}]

\subsubsection{TSM Contexts}
\emph{peTSM contexts}, $\uPsi$, are of the form $\uAS{\uA}{\Psi}$, where $\uA$ is a \emph{TSM identifier expansion context} and $\Psi$ is a \emph{peTSM definition context}.

\emph{ppTSM contexts}, $\uPhi$, are of the form $\uAS{\uA}{\Phi}$, where $\uA$ is a TSM identifier expansion context and $\Phi$ is a \emph{ppTSM definition context}.

A \emph{TSM identifier expansion context}, $\uA$, is a finite function mapping each TSM identifier $\tsmv \in \domof{\uA}$ to the \emph{TSM identifier expansion}, $\vExpands{\tsmv}{\epsilon}$, for some \emph{TSM expression}, $\epsilon$. We write $\ctxUpdate{\uA}{\tsmv}{\epsilon}$ for the TSM identifier expansion context that maps $\tsmv$ to $\vExpands{\tsmv}{\epsilon}$, and defers to $\uA$ for all other TSM identifiers (i.e. the previous mapping is \emph{updated}.)

A \emph{peTSM definition context}, $\Psi$, is a finite function mapping each TSM name $a \in \domof{\Psi}$ to an \emph{expanded peTSM definition}, $\petsmdefn{a}{\rho}{\eparse}$, where $\rho$ is the peTSM's type annotation, and $\eparse$ is its parse function. We write $\Psi, \petsmdefn{a}{\rho}{\eparse}$ when $a \notin \domof{\Psi}$ for the extension of $\Psi$ that maps $a$ to $\petsmdefn{a}{\rho}{\eparse}$. We write $\petsmenv{\Omega}{\Psi}$  when all the TSM type annotations in $\Psi$ are well-formed assuming $\Omega$, and the parse functions in $\Psi$ are closed and of the appropriate type.


\begin{definition}[peTSM Definition Context Formation]\label{def:peTSM-def-ctx-formation} $\petsmenv{\Omega}{\Psi}$ iff for each ${\petsmdefn{a}{\rho}{\eparse}} \in \Psi$, we have $\istsmty{\Omega}{\rho}$ and \[\hastypeP{\emptyset}{\eparse}{\aparr{\tBody}{\tParseResultPCEExp}}\]\end{definition}

\begin{definition}[peTSM Context Formation] $\petsmctx{\Omega}{\uAS{\uA}{\Psi}}$ iff $\petsmenv{\Omega}{\Psi}$ and for each $\vExpands{\tsmv}{\epsilon} \in \uA$ we have $\hastsmtypeExp{\Omega}{\Psi}{\epsilon}{\rho}$ for some $\rho$.
\end{definition}

A \emph{ppTSM definition context}, $\Phi$, is a finite function mapping each TSM name $a \in \domof{\Phi}$ to an \emph{expanded ppTSM definition}, $\pptsmdefn{a}{\rho}{\eparse}$, where $\rho$ is the ppTSM's type annotation, and $\eparse$ is its parse function. We write $\Phi, \pptsmdefn{a}{\rho}{\eparse}$ when $a \notin \domof{\Phi}$ for the extension of $\Phi$ that maps $a$ to $\pptsmdefn{a}{\rho}{\eparse}$. We write $\pptsmenv{\Omega}{\Phi}$  when all the type annotations in $\Phi$ are well-formed assuming $\Omega$, and the parse functions in $\Phi$ are closed and of the appropriate type.

\begin{definition}[ppTSM Definition Context Formation]\label{def:ppTSM-def-ctx-formation} $\pptsmenv{\Omega}{\Phi}$ iff for each $\pptsmdefn{\tsmv}{\rho}{\eparse} \in \Phi$, we have $\istsmty{\Omega}{\rho}$ and \[\hastypeP{\emptyset}{\eparse}{\aparr{\tBody}{\tParseResultCEPat}}\]\end{definition}

\begin{definition}[ppTSM Context Formation] $\pptsmctx{\Omega}{\uAS{\uA}{\Phi}}$ iff $\pptsmenv{\Omega}{\Phi}$ and for each $\vExpands{\tsmv}{\epsilon} \in \uA$ we have $\hastsmtypePat{\Omega}{\Phi}{\epsilon}{\rho}$ for some $\rho$.
\end{definition}

\subsubsection{Signature and Module Expansion}
\noindent\fbox{$\strut\sigExpandsPX{\usigma}{\sigma}$}~~$\usigma$ has well-formed expansion $\sigma$
\begin{equation}\label{rule:sigExpandsP}
\inferrule{
  \kExpandsX{\ukappa}{\kappa}\\
  \cExpands{\uOmega, \uKhyp{\uu}{u}{\kappa}}{\utau}{\tau}{\akty}
}{
  \sigExpandsPX{\signature{\uu}{\ukappa}{\utau}}{\asignature{\kappa}{u}{\tau}}
}
\end{equation}

\noindent\fbox{$\strut\mExpandsPX{\uM}{M}{\sigma}$}~~$\uM$ has expansion $M$ matching $\sigma$
\begin{subequations}\label{rules:mExpandsP}
\begin{equation}\label{rule:mExpandsP-subsumes}
\inferrule{
  \mExpandsPX{\uM}{M}{\sigma}\\
  \sigsub{\uOmega}{\sigma}{\sigma'}
}{
  \mExpandsPX{\uM}{M}{\sigma'}
}
\end{equation}
\begin{equation}\label{rule:mExpands-var}
\inferrule{ }{
  \mExpandsP{\uOmega, \uMhyp{\uX}{X}{\sigma}}{\uPsi}{\uPhi}{\uX}{X}{\sigma}
}
\end{equation}
\begin{equation}\label{rule:mExpandsP-struct}
\inferrule{
  \kanaX{\uc}{c}{\kappa}\\
  \eanaPX{\ue}{e}{[c/u]\tau}
}{
  \mExpandsPX{\struct{\uc}{\ue}}{\astruct{c}{e}}{\asignature{\kappa}{u}{\tau}}
}
\end{equation}
\begin{equation}\label{rule:mExpandsP-seal}
\inferrule{
  \sigExpandsPX{\usigma}{\sigma}\\
  \mExpandsPX{\uM}{M}{\sigma}
}{
  \mExpandsPX{\seal{\uM}{\usigma}}{\aseal{\sigma}{M}}{\sigma} 
}
\end{equation}
\begin{equation}\label{rule:mExpandsP-mlet}
\inferrule{
  \mExpandsPX{\uM}{M}{\sigma}\\
  \sigExpandsPX{\usigma'}{\sigma'}\\\\
  \mExpandsP{\uOmega, \uMhyp{\uX}{X}{\sigma}}{\uPsi}{\uPhi}{\uM'}{M'}{\sigma'}
}{
  \mExpandsPX{\mlet{\uX}{\uM}{\uM'}{\usigma'}}{\amlet{\sigma'}{M}{X}{M'}}{\sigma'}
}
\end{equation}
\begin{equation}\label{rule:mExpandsP-syntaxpe}
\inferrule{
  \tsmtyExpands{\uOmega}{\urho}{\rho}\\
  \hastypeP{\emptyset}{\eparse}{\aparr{\tBody}{\tParseResultPCEExp}}\\\\
  \evalU{\eparse}{\eparse'}\\
  \mExpandsP{\uOmega}{\uAS{\uA \uplus \mapitem{\tsmv}{\adefref{a}}}{\Psi, \petsmdefn{a}{\rho}{\eparse'}}}{\uPhi}{\uM}{M}{\sigma}
}{
  \mExpandsP{\uOmega}{\uAS{\uA}{\Psi}}{\uPhi}{\defpetsm{\tsmv}{\urho}{\eparse}{\uM}}{M}{\sigma}
}
\end{equation}
\begin{equation}\label{rule:mExpandsP-letpetsm}
\inferrule{
  % \uOmega = \uOmegaEx{\uD}{\uG}{\uMctx}{\Omega}\\
  \tsmexpExpandsExp{\uOmega}{\uAS{\uA}{\Psi}}{\uepsilon}{\epsilon}{\rho}\\
  % \tsmexpEvalsExp{\Omega}{\Psi}{\epsilon}{\epsilon_\text{normal}}\\\\
  \mExpandsP{\uOmega}{\uAS{\uA\uplus\mapitem{\tsmv}{\epsilon_\text{normal}}}{\Psi}}{\uPhi}{\uM}{M}{\sigma}
}{
  \mExpandsP{\uOmega}{\uAS{\uA}{\Psi}}{\uPhi}{\uletpetsm{\tsmv}{\uepsilon}{\uM}}{M}{\sigma}
}
\end{equation}
\begin{equation}\label{rule:mExpandsP-syntaxpp}
\inferrule{ 
  \tsmtyExpands{\uOmega}{\urho}{\rho}\\
  \hastypeP{\emptyset}{\eparse}{\aparr{\tBody}{\tParseResultCEPat }}\\\\
  \evalU{\eparse}{\eparse'}\\
  \mExpandsP{\uOmega}{\uPsi}{\uAS{\uA \uplus \mapitem{\tsmv}{\adefref{a}}}{\Phi, \pptsmdefn{a}{\rho}{\eparse'}}}{\uM}{M}{\sigma}
}{
  \mExpandsP{\uOmega}{\uPsi}{\uAS{\uA}{\Phi}}{\defpptsm{\tsmv}{\urho}{\eparse}{\uM}}{M}{\sigma}
}
\end{equation}
\begin{equation}\label{rule:mExpandsP-letpptsm}
\inferrule{
  \tsmexpExpandsPat{\uOmega}{\uAS{\uA}{\Phi}}{\uepsilon}{\epsilon}{\rho}\\
  \mExpandsP{\uOmega}{\uPsi}{\uAS{\uA\uplus\mapitem{\tsmv}{\epsilon}}{\Phi}}{\uM}{M}{\sigma}
}{
  \mExpandsP{\uOmega}{\uPsi}{\uAS{\uA}{\Phi}}{\uletpptsm{\tsmv}{\uepsilon}{\uM}}{M}{\sigma}
}
\end{equation}
\end{subequations}

\subsubsection{Kind and Constructor Expansion}
\noindent\fbox{$\strut\kExpandsX{\ukappa}{\kappa}$}~~$\ukappa$ has well-formed expansion $\kappa$
\begin{subequations}\label{rules:kExpands-B}
\begin{equation}\label{rule:kExpands-B-darr}
\inferrule{
  \kExpandsX{\ukappa_1}{\kappa_1}\\
  \kExpands{\uOmega, \uKhyp{\uu}{u}{\kappa_1}}{\ukappa_2}{\kappa_2}
}{
  \kExpandsX{\kdarr{\uu}{\ukappa_1}{\ukappa_2}}{\akdarr{\kappa_1}{u}{\kappa_2}}
}
\end{equation}
\begin{equation}\label{rule:kExpands-B-unit}
\inferrule{ }{
  \kExpandsX{\kunit}{\akunit}
}
\end{equation}
\begin{equation}\label{rule:kExpands-B-dprod}
\inferrule{
  \kExpandsX{\ukappa_1}{\kappa_1}\\
  \kExpands{\uOmega, \uKhyp{\uu}{u}{\kappa_1}}{\ukappa_2}{\kappa_2}
}{
  \kExpandsX{\kdbprod{\uu}{\ukappa_1}{\ukappa_2}}{\akdbprod{\kappa_1}{u}{\kappa_2}}
}
\end{equation}
\begin{equation}\label{rule:kExpands-B-ty}
\inferrule{ }{
  \kExpandsX{\kty}{\akty}
}
\end{equation}
\begin{equation}\label{rule:kExpands-B-sing}
\inferrule{
  \cExpandsX{\utau}{\tau}{\akty}
}{
  \kExpandsX{\ksing{\utau}}{\aksing{\tau}}
}
\end{equation}
\end{subequations}

\noindent\fbox{$\strut\cExpandsX{\uc}{c}{\kappa}$}~~$\uc$ has expansion $c$ of kind $\kappa$
\begin{subequations}\label{rules:cExpands}
\begin{equation}\label{rule:cExpands-subsume}
\inferrule{
  \cExpandsX{\uc}{c}{\kappa_1}\\
  \ksubX{\kappa_1}{\kappa_2}
}{
  \cExpandsX{\uc}{c}{\kappa_2}
}
\end{equation}
\begin{equation}\label{rule:cExpands-var}
\inferrule{ }{\cExpands{\uOmega, \uKhyp{\uu}{u}{\kappa}}{\uu}{u}{\kappa}}
\end{equation}
\begin{equation}\label{rule:cExpands-abs}
\inferrule{
  \cExpands{\uOmega, \uKhyp{\uu}{u}{\kappa_1}}{\uc_2}{c_2}{\kappa_2}
}{
  \cExpandsX{\cabs{\uu}{\uc_2}}{\acabs{u}{c_2}}{\akdarr{\kappa_1}{u}{\kappa_2}}
}
\end{equation}
\begin{equation}\label{rule:cExpands-app}
\inferrule{
  \cExpandsX{\uc_1}{c_1}{\akdarr{\kappa_2}{u}{\kappa}}\\
  \cExpandsX{\uc_2}{c_2}{\kappa_2}
}{
  \cExpandsX{\capp{\uc_1}{\uc_2}}{\acapp{c_1}{c_2}}{[c_1/u]\kappa}
}
\end{equation}
\begin{equation}\label{rule:cExpands-unit}
\inferrule{ }{
  \cExpandsX{\ctriv}{\actriv}{\akunit}
}
\end{equation}
\begin{equation}\label{rule:cExpands-pair}
\inferrule{
  \cExpandsX{\uc_1}{c_1}{\kappa_1}\\
  \cExpandsX{\uc_2}{c_2}{[c_1/u]\kappa_2}
}{
  \cExpandsX{\cpair{\uc_1}{\uc_2}}{\acpair{c_1}{c_2}}{\akdbprod{\kappa_1}{u}{\kappa_2}}
}
\end{equation}
\begin{equation}\label{rule:cExpands-prl}
\inferrule{
  \cExpandsX{\uc}{c}{\akdbprod{\kappa_1}{u}{\kappa_2}}
}{
  \cExpandsX{\cprl{\uc}}{\acprl{c}}{\kappa_1}
}
\end{equation}
\begin{equation}\label{rule:cExpands-prr}
\inferrule{
  \cExpandsX{\uc}{c}{\akdbprod{\kappa_1}{u}{\kappa_2}}
}{
  \cExpandsX{\cprr{\uc}}{\acprr{c}}{[\acprl{c}/u]\kappa_2}
}
\end{equation}
\begin{equation}\label{rule:cExpands-parr}
\inferrule{
  \cExpandsX{\utau_1}{\tau_1}{\akty}\\
  \cExpandsX{\utau_2}{\tau_2}{\akty}
}{
  \cExpandsX{\parr{\utau_1}{\utau_2}}{\aparr{\tau_1}{\tau_2}}{\akty}
}
\end{equation}
\begin{equation}\label{rule:cExpands-all}
\inferrule{
  \kExpandsX{\ukappa}{\kappa}\\
  \cExpands{\uOmega, \uKhyp{\uu}{u}{\kappa}}{\utau}{\tau}{\akty}
}{
  \cExpandsX{\forallu{\uu}{\ukappa}{\utau}}{\aallu{\kappa}{u}{\tau}}{\akty}
}
\end{equation}
\begin{equation}\label{rule:cExpands-rec}
\inferrule{
  \cExpands{\uOmega, \uKhyp{\ut}{t}{\akty}}{\utau}{\tau}{\akty}
}{
  \cExpandsX{\rect{\ut}{\utau}}{\arec{t}{\tau}}{\akty}
}
\end{equation}
\begin{equation}\label{rule:cExpands-prod}
\inferrule{
  \{\cExpandsX{\utau_i}{\tau_i}{\akty}\}_{1 \leq i \leq n}
}{
  \cExpandsX{\prodt{\mapschema{\utau}{i}{\labelset}}}{\aprod{\labelset}{\mapschema{\tau}{i}{\labelset}}}{\akty}
}
\end{equation}
\begin{equation}\label{rule:cExpands-sum}
\inferrule{
  \{\cExpandsX{\utau_i}{\tau_i}{\akty}\}_{1 \leq i \leq n}
}{
  \cExpandsX{\sumt{\mapschema{\utau}{i}{\labelset}}}{\asum{\labelset}{\mapschema{\tau}{i}{\labelset}}}{\akty}
}
\end{equation}
\begin{equation}\label{rule:cExpands-sing}
\inferrule{
  \cExpandsX{\uc}{c}{\akty}
}{
  \cExpandsX{\uc}{c}{\aksing{c}}
}
\end{equation}
\begin{equation}\label{rule:cExpands-stat}
\inferrule{ }{
  \cExpands{\uOmega, \uMhyp{\uX}{X}{\asignature{\kappa}{u}{\tau}}}{\mcon{\uX}}{\amcon{X}}{\kappa}
}
\end{equation}
\end{subequations}

\subsubsection{Type, Expression, Rule and Pattern Expansion}
% \noindent\fbox{$\strut\tExpandsPX{\utau}{\tau}$}~~$\utau$ has well-formed expansion $\tau$
% \begin{equation}\label{rule:tExpandsP}
% \inferrule{
%   \cExpandsX{\utau}{\tau}{\akty}
% }{
%   \tExpandsPX{\utau}{\tau}
% }
% \end{equation}

\noindent\fbox{$\strut\expandsPX{\ue}{e}{\tau}$}~~$\ue$ has expansion $e$ of type $\tau$
\begin{subequations}\label{rules:expandsP}
\begin{equation}\label{rule:expandsP-subsume}
  \inferrule{
    \expandsPX{\ue}{e}{\tau}\\
    \issubtypePX{\tau}{\tau'}
  }{
    \expandsPX{\ue}{e}{\tau'}
  }
\end{equation}

\begin{equation}\label{rule:expandsP-var}
  \inferrule{ }{ 
    \expandsP{\uOmega, \uGhyp{\ux}{x}{\tau}}{\uPsi}{\uPhi}{\ux}{x}{\tau}
  }
\end{equation}
\begin{equation}\label{rule:expandsP-asc}
  \inferrule{
    \cExpands{\uOmega}{\utau}{\tau}{\akty}\\
    \expandsP{\uOmega}{\uPsi}{\uPhi}{\ue}{e}{\tau}
  }{
    \expandsP{\uOmega}{\uPsi}{\uPhi}{\asc{\ue}{\utau}}{e}{\tau}
  }
\end{equation}
\begin{equation}\label{rule:expandsP-letsyn}
  \inferrule{
    \expandsP{\uOmega}{\uPsi}{\uPhi}{\ue_1}{e_1}{\tau_1}\\
    \expandsP{\uOmega, \uGhyp{\ux}{x}{\tau_1}}{\uPsi}{\uPhi}{\ue_2}{e_2}{\tau_2}
  }{
    \expandsP{\uOmega}{\uPsi}{\uPhi}{\letsyn{\ux}{\ue_1}{\ue_2}}{
      \aeap{\aelam{\tau_1}{x}{e_2}}{e_1}
    }{\tau_2}
  }
\end{equation}
%Functions with an argument type annotation can appear in synthetic position.
\begin{equation}\label{rule:expandsP-lam}
  \inferrule{
    \cExpands{\uOmega}{\utau_1}{\tau_1}{\akty}\\
    \expandsP{\uOmega, \uGhyp{\ux}{x}{\tau_1}}{\uPsi}{\uPhi}{\ue}{e}{\tau_2}
  }{
    \expandsPX{\lam{\ux}{\utau_1}{\ue}}{\aelam{\tau_1}{x}{e}}{\aparr{\tau_1}{\tau_2}}
  }
\end{equation}

%Function applications can appear in synthetic position. The argument is analyzed against the argument type synthesized by the function.
\begin{equation}\label{rule:expandsP-ap}
  \inferrule{
    \expandsPX{\ue_1}{e_1}{\aparr{\tau_2}{\tau}}\\
    \expandsPX{\ue_2}{e_2}{\tau_2}
  }{
    \expandsPX{\ap{\ue_1}{\ue_2}}{\aeap{e_1}{e_2}}{\tau}
  }
\end{equation}

%Type lambdas and type applications can appear in synthetic position.
\begin{equation}\label{rule:expandsP-tlam}
  \inferrule{
    \kExpandsX{\ukappa}{\kappa}\\
    \expandsP{\uOmega, \uKhyp{\uu}{u}{\kappa}}{\uPsi}{\uPhi}{\ue}{e}{\tau}
  }{
    \expandsPX{\clam{\uu}{\ukappa}{\ue}}{\aeclam{\kappa}{u}{e}}{\aallu{\kappa}{u}{\tau}}
  }
\end{equation}
\begin{equation}\label{rule:expandsP-tap}
  \inferrule{
    \expandsPX{\ue}{e}{\aallu{\kappa}{u}{\tau}}\\
    \ksynX{\uc}{c}{\kappa}
  }{
    \expandsPX{\cAp{\ue}{\uc}}{\aecap{e}{c}}{[c/t]\tau}
  }
\end{equation}
% Values of recursive types can be introduced only in analytic position.
\begin{equation}\label{rule:expandsP-fold}
  \inferrule{
    \expandsPX{\ue}{e}{[\arec{t}{\tau}/t]\tau}
  }{
    \expandsPX{\fold{\ue}}{\aefold{e}}{\arec{t}{\tau}}
  }
\end{equation}

%Unfoldings can appear in synthetic position.
\begin{equation}\label{rule:expandsP-unfold}
  \inferrule{
    \expandsPX{\ue}{e}{\arec{t}{\tau}}
  }{
    \expandsPX{\unfold{\ue}}{\aeunfold{e}}{[\arec{t}{\tau}/t]\tau}
  }
\end{equation}

%Labeled tuples can appear in synthetic position. Each of the field values are then in synthetic position. 
\begin{equation}\label{rule:expandsP-tpl}
  \inferrule{
    \{\expandsPX{\ue_i}{e_i}{\tau_i}\}_{i \in \labelset}
  }{
    \expandsPX{\tpl{\mapschema{\ue}{i}{\labelset}}}{\aetpl{\labelset}{\mapschema{e}{i}{\labelset}}}{\aprod{\labelset}{\mapschema{\tau}{i}{\labelset}}}
  }
\end{equation}

%Fields can be projected out of a labeled tuple in synthetic position.
\begin{equation}\label{rule:expandsP-pr}
  \inferrule{
    \expandsPX{\ue}{e}{\aprod{\labelset, \ell}{\mapschema{\tau}{i}{\labelset}; \mapitem{\ell}{\tau}}}
  }{
    \expandsPX{\prj{\ue}{\ell}}{\aepr{\ell}{e}}{\tau}
  }
\end{equation}

% Values of labeled sum type can appear only in analytic position.
\begin{equation}\label{rule:expandsP-in}
  \inferrule{
    \expandsPX{\ue'}{e'}{\tau'}
  }{
    \expandsPX{\inj{\ell}{\ue}}{\aein{\ell}{e'}}{\asum{\labelset, \ell}{\mapschema{\tau}{i}{\labelset}; \mapitem{\ell}{\tau'}}}
    % \uOmega \vdash_{\uPsi; \uPhi} \left(\shortstack{$\ue \leadsto $\\$\Leftarrow$\vspace{-1.2em}}\right)
    %\expandsPX{\auanain{\ell}{\ue}}{\aein{\ell}}{\asum{\labelset, \ell}{\mapschema{\tau}{i}{\labelset}; \mapitem{\ell}{\tau}}}
  }
\end{equation}

%Match expressions can appear in synthetic position.
\begin{equation}\label{rule:expandsP-match}
  \inferrule{
    % \uOmega = \uOmegaEx{\uD}{\uG}{\uM}{\Omega}\\\\
    \expandsPX{\ue}{e}{\tau}\\
    % \haskind{\Omega}{\tau'}{\akty}\\
    \{\rsynPX{\urv_i}{r_i}{\tau}{\tau'}\}_{1 \leq i \leq n}
  }{
    \expandsPX{\matchwith{\ue}{\seqschemaX{\urv}}}{\aematchwith{n}{e}{\seqschemaX{r}}}{\tau'}
  }
\end{equation}

\begin{equation}\label{rule:expandsP-mval}
  \inferrule{ }{
    \expandsP{\uOmega, \uMhyp{\uX}{X}{\asignature{\kappa}{u}{\tau}}}{\uPsi}{\uPhi}{\mval{\uX}}{\amval{X}}{[\amcon{X}/u]\tau}
  }
\end{equation}

% ueTSMs can be defined and applied in synthetic position.
% \begin{equation}\label{rule:expandsP-defpetsm}
% \inferrule{
%   \tsmtyExpands{\uOmega}{\urho}{\rho}\\
%   \hastypeP{\emptyset}{\eparse}{\aparr{\tBody}{\tParseResultPCEExp}}\\\\
%   \expandsP{\uOmega}{\uASI{\uA \uplus \mapitem{\tsmv}{\adefref{a}}}{\Psi, \petsmdefn{a}{\rho}{\eparse}}{\uI}}{\uPhi}{\ue}{e}{\tau}
% }{
%   \expandsP{\uOmega}{\uASI{\uA}{\Psi}{\uI}}{\uPhi}{\usyntaxueP{\tsmv}{\urho}{\eparse}{\ue}}{e}{\tau}
% }
% \end{equation}

% \begin{equation}\label{rule:expandsP-letpetsm}
% \inferrule{
%   \tsmexpExpandsExp{\uOmega}{\uASI{\uA}{\Psi}{\uI}}{\uepsilon}{\epsilon}{\rho}\\
%   \expandsP{\uOmega}{\uASI{\uA\uplus\mapitem{\tsmv}{\epsilon}}{\Psi}{\uI}}{\uPhi}{\ue}{e}{\tau}
% }{
%   \expandsP{\uOmega}{\uASI{\uA}{\Psi}{\uI}}{\uPhi}{\uletpetsm{\tsmv}{\uepsilon}{\ue}}{e}{\tau}
% }
% \end{equation}

\begin{equation}\label{rule:expandsP-apuetsm}
\inferrule{
  \uOmega = \uOmegaEx{\uD}{\uG}{\uMctx}{\Omega_\text{app}}\\
  \uPsi=\uAS{\uA}{\Psi}\\\\
  \tsmexpExpandsExp{\uOmega}{\uPsi}{\uepsilon}{\epsilon}{\aetype{\tau_\text{final}}}\\
  \tsmexpEvalsExp{\Omega_\text{app}}{\Psi}{\epsilon}{\epsilon_\text{normal}}\\\\
  \tsmdefof{\epsilon_\text{normal}}=a\\
  \Psi = \Psi', \petsmdefn{a}{\rho}{\eparse}\\\\
  \encodeBody{b}{\ebody}\\
  \evalU{\ap{\eparse}{\ebody}}{{\lbltxt{SuccessE}}\cdot{e_\text{pproto}}}\\
  \decodePCEExp{e_\text{pproto}}{\pce}\\\\
  \prepce{\Omega_\text{app}}{\Psi}{\pce}{\ce}{\epsilon_\text{normal}}{\aetype{\tau_\text{proto}}}{\omega}{\Omega_\text{params}}\\\\
  \segOK{\segof{\ce}}{b}\\
  \cvalidEP{\Omega_\text{params}}{\esceneP{\OParams}{\uOmega}{\uPsi}{\uPhi}{b}}{\ce}{e}{\tau_\text{proto}}
}{
  \expandsP{\uOmega}{\uPsi}{\uPhi}{\utsmap{\uepsilon}{b}}{[\omega]e}{[\omega]\tau_\text{proto}}
}
\end{equation}

% These rules are nearly identical to Rules (\ref{rule:expandsUP-syntax}) and (\ref{rule:expandsUP-tsmap}), differing only in that the typed expansion premises have been replaced by corresponding synthetic typed expansion premises. The premises of these rules can be understood as described in Sections \ref{sec:U-uetsm-definition} and \ref{sec:U-uetsm-application}. The body encoding judgement and candidate expansion expression decoding judgements were characterized in Sec. \ref{sec:typed-expansion-UP}. We discuss candidate expansion validation in Sec. \ref{sec:ce-validation-B} below.

% To support ueTSM implicits, ueTSM contexts, $\uPsi$, are redefined to take the form $\uASI{\uA}{\Psi}{\uI}$. TSM naming contexts, $\uA$, and ueTSM definition contexts, $\Psi$, were defined in Sec. \ref{sec:typed-expansion-UP}. We write $\uPsi, \uShyp{\tsmv}{a}{\tau}{\eparse}$ when $\uPsi=\uASI{\uA}{\Psi}{\uI}$ as shorthand for \[\uASI{\ctxUpdate{\uA}{\tsmv}{a}}{\Psi, \xuetsmbnd{a}{\tau}{\eparse}}{\uI}\]

% \emph{TSM designation contexts}, $\uI$, are finite functions that map each type $\tau \in \domof{\uI}$ to the \emph{TSM designation} $\designate{\tau}{a}$, for some symbol $a$. We write $\uI \uplus \designate{\tau}{a}$ for the TSM designation context that maps $\tau$ to $\designate{\tau}{a}$ and defers to $\uI$ for all other types (i.e. the previous designation, if any, is updated). 

% The TSM designation context in the ueTSM context is updated by expressions of ueTSM designation form. Such expressions can appear in synthetic position, where they are governed by the following rule:% We write $\uIOK{\Delta}{\uI}$ when each type in $\uI$ is well-formed assuming $\Delta$.
%\begin{definition}[TSM Designation Context Well-Formedness] $\uIOK{\Delta}{{\uI}$ iff for each $\designate{\tau}{a}$ we have $\istypeU{\Delta}{\tau}$.\end{definition}

% \todo{peTSM implicit designation}
% \begin{equation}\label{rule:expandsP-implicite}
%   \inferrule{
%     \esyn{\uDelta}{\uGamma}{\uASI{\uA \uplus \vExpands{\tsmv}{a}}{\Psi, \xuetsmbnd{a}{\tau}{\eparse}}{\uI \uplus \designate{\tau}{a}}}{\uPhi}{\ue}{e}{\tau'}
%   }{
%     \esyn{\uDelta}{\uGamma}{\uASI{\uA \uplus \vExpands{\tsmv}{a}}{\Psi, \xuetsmbnd{a}{\tau}{\eparse}}{\uI}}{\uPhi}{\implicite{\tsmv}{\ue}}{e}{\tau'}
%   }
% \end{equation}

% % Like ueTSMs, upTSMs can be defined in synthetic position.
% \begin{equation}\label{rule:expandsP-syntaxup}
% \inferrule{
%   \tsmtyExpands{\uOmega}{\urho}{\rho}\\
%   \hastypeP{\emptyset}{\eparse}{\aparr{\tBody}{\tParseResultCEPat}}\\\\
%   \expandsP{\uOmega}{\uPsi}{\uASI{\uA \uplus \mapitem{\tsmv}{\adefref{a}}}{\Phi, \pptsmdefn{a}{\rho}{\eparse}}{\uI}}{\ue}{e}{\tau}
% }{
%   \expandsP{\uOmega}{\uPsi}{\uASI{\uA}{\Phi}{\uI}}{\usyntaxup{\tsmv}{\urho}{\eparse}{\ue}}{e}{\tau}
% }
% \end{equation}


% \begin{equation}\label{rule:expandsP-letpptsm}
% \inferrule{
%   \tsmexpExpandsPat{\uOmega}{\uASI{\uA}{\Phi}{\uI}}{\uepsilon}{\epsilon}{\rho}\\
%   \expandsP{\uOmega}{\uPsi}{\uASI{\uA\uplus\mapitem{\tsmv}{\epsilon}}{\Phi}{\uI}}{\ue}{e}{\tau}
% }{
%   \expandsP{\uOmega}{\uPsi}{\uASI{\uA}{\Phi}{\uI}}{\uletpptsm{\tsmv}{\uepsilon}{\ue}}{e}{\tau}
% }
% \end{equation}

% % This rule is nearly identical to Rule (\ref{rule:expandsUP-defuptsm}), differing only in that the typed expansion premise has been replaced by the corresponding synthetic typed expansion premise. The premises can be understood as described in Section \ref{sec:uptsm-definition}.

% % To support upTSM implicits, upTSM contexts, $\uPhi$, are redefined to take the form $\uASI{\uA}{\Phi}{\uI}$. upTSM definition contexts, $\Phi$, were defined in Sec. \ref{sec:uptsm-definition}. We write $\uPhi, \uPhyp{\tsmv}{a}{\tau}{\eparse}$ when $\uPhi=\uASI{\uA}{\Phi}{\uI}$ as shorthand for \[\uASI{\ctxUpdate{\uA}{\tsmv}{a}}{\Phi, \xuptsmbnd{a}{\tau}{\eparse}}{\uI}\]

% % The TSM designation context in the upTSM context is updated by expressions of upTSM designation form. Such expressions can appear in synthetic position, where they are governed by the following rule:% We write $\uIOK{\Delta}{\uI}$ when each type in $\uI$ is well-formed assuming $\Delta$.
% %\begin{definition}[TSM Designation Context Well-Formedness] $\uIOK{\Delta}{{\uI}$ iff for each $\designate{\tau}{a}$ we have $\istypeU{\Delta}{\tau}$.\end{definition}
% \todo{ppTSM implicit designation}
% \begin{equation}\label{rule:expandsP-implicitp}
%   \inferrule{
%     \esyn{\uDelta}{\uGamma}{\uPsi}{\uASI{\uA\uplus\vExpands{\tsmv}{a}}{\Phi, \xuptsmbnd{a}{\tau}{\eparse}}{\uI \uplus \designate{\tau}{a}}}{\ue}{e}{\tau'}
%   }{
%     \esyn{\uDelta}{\uGamma}{\uPsi}{\uASI{\uA\uplus\vExpands{\tsmv}{a}}{\Phi, \xuetsmbnd{a}{\tau}{\eparse}}{\uI}}{\implicitp{\tsmv}{\ue}}{e}{\tau'}
%   }
% \end{equation}
\end{subequations}

% \begin{subequations}\label{rules:expandsP}
% Type analysis subsumes type synthesis, in that when a type can be synthesized for an unexpanded expression, that unexpanded expression can also be analyzed against that type, producing the same expansion. This is expressed by the following \emph{subsumption rule} for unexpanded expressions.

% Additional rules are needed for certain forms in order to propagate types for analysis into subexpressions, and for forms that can appear only in analytic position.



% Rule (\ref{rule:esyn-tpl}) governed labeled tuples in synthetic position. The following rule governs labeled tuples in analytic position.


% Rule (\ref{rule:esyn-match}) governed match expressions in synthetic position. The following rule governs match expressions in analytic position.

% Rule (\ref{rule:esyn-defuetsm}) governed ueTSM definitions in synthetic position. The following rule governs ueTSM definitions in analytic position.
% \begin{equation}\label{rule:expandsP-defpetsm}
% \inferrule{
%   \tsmtyExpands{\uOmega}{\urho}{\rho}\\
%   \hastypeP{\emptyset}{\eparse}{\aparr{\tBody}{\tParseResultPCEExp}}\\\\
%   \expandsP{\uOmega}{\uASI{\uA \uplus \mapitem{\tsmv}{\adefref{a}}}{\Psi, \petsmdefn{a}{\rho}{\eparse}}{\uI}}{\uPhi}{\ue}{e}{\tau}
% }{
%   \expandsP{\uOmega}{\uASI{\uA}{\Psi}{\uI}}{\uPhi}{\usyntaxueP{\tsmv}{\urho}{\eparse}{\ue}}{e}{\tau}
% }
% \end{equation}

% \begin{equation}\label{rule:expandsP-letpetsm}
% \inferrule{
%   \tsmexpExpandsExp{\uOmega}{\uASI{\uA}{\Psi}{\uI}}{\uepsilon}{\epsilon}{\rho}\\
%   \expandsP{\uOmega}{\uASI{\uA\uplus\mapitem{\tsmv}{\epsilon}}{\Psi}{\uI}}{\uPhi}{\ue}{e}{\tau}
% }{
%   \expandsP{\uOmega}{\uASI{\uA}{\Psi}{\uI}}{\uPhi}{\uletpetsm{\tsmv}{\uepsilon}{\ue}}{e}{\tau}
% }
% \end{equation}

% \todo{peTSM implicit designation}
% Rule (\ref{rule:esyn-implicite}) governed ueTSM designations in synthetic position. The following rule governs ueTSM designations in analytic position.
% \begin{equation}\label{rule:expandsP-implicite}
%   \inferrule{
%     \eana{\uDelta}{\uGamma}{\uASI{\uA \uplus \vExpands{\tsmv}{a}}{\Psi, \xuetsmbnd{a}{\tau}{\eparse}}{\uI \uplus \designate{\tau}{a}}}{\uPhi}{\ue}{e}{\tau'}
%   }{
%     \eana{\uDelta}{\uGamma}{\uASI{\uA \uplus \vExpands{\tsmv}{a}}{\Psi, \xuetsmbnd{a}{\tau}{\eparse}}{\uI}}{\uPhi}{\implicite{\tsmv}{\ue}}{e}{\tau'}
%   }
% \end{equation}

% \todo{peTSM implicit application}
% % An expression of unadorned literal form can appear only in analytic position. The following rule extracts the TSM designated at the type that the expression is being analyzed against from the TSM designation context in the ueTSM context and applies it implicitly, i.e. the premises correspond to those of Rule (\ref{rule:esyn-apuetsm}).


% Rule (\ref{rule:esyn-defuptsm}) governed upTSM definitions in synthetic position. The following rule governs upTSM definitions in analytic position.
% \begin{equation}\label{rule:expandsP-syntaxup}
% \inferrule{
%   \tsmtyExpands{\uOmega}{\urho}{\rho}\\
%   \hastypeP{\emptyset}{\eparse}{\aparr{\tBody}{\tParseResultCEPat}}\\\\
%   \expandsP{\uOmega}{\uPsi}{\uASI{\uA \uplus \mapitem{\tsmv}{\adefref{a}}}{\Phi, \pptsmdefn{a}{\rho}{\eparse}}{\uI}}{\ue}{e}{\tau}
% }{
%   \expandsP{\uOmega}{\uPsi}{\uASI{\uA}{\Phi}{\uI}}{\usyntaxup{\tsmv}{\urho}{\eparse}{\ue}}{e}{\tau}
% }
% \end{equation}


% \begin{equation}\label{rule:expandsP-letpptsm}
% \inferrule{
%   \tsmexpExpandsPat{\uOmega}{\uASI{\uA}{\Phi}{\uI}}{\uepsilon}{\epsilon}{\rho}\\
%   \expandsP{\uOmega}{\uPsi}{\uASI{\uA\uplus\mapitem{\tsmv}{\epsilon}}{\Phi}{\uI}}{\ue}{e}{\tau}
% }{
%   \expandsP{\uOmega}{\uPsi}{\uASI{\uA}{\Phi}{\uI}}{\uletpptsm{\tsmv}{\uepsilon}{\ue}}{e}{\tau}
% }
% \end{equation}


% \todo{ppTSM implicit designation}
% % Rule (\ref{rule:esyn-implicitp}) governed upTSM designations in synthetic position. The following rule governs upTSM designations in analytic position.
% \begin{equation}\label{rule:expandsP-implicitp}
%   \inferrule{
%     \eana{\uDelta}{\uGamma}{\uPsi}{\uASI{\uA\uplus\vExpands{\tsmv}{a}}{\Phi, \xuptsmbnd{a}{\tau}{\eparse}}{\uI \uplus \designate{\tau}{a}}}{\ue}{e}{\tau'}
%   }{
%     \eana{\uDelta}{\uGamma}{\uPsi}{\uASI{\uA\uplus\vExpands{\tsmv}{a}}{\Phi, \xuetsmbnd{a}{\tau}{\eparse}}{\uI}}{\implicitp{\tsmv}{\ue}}{e}{\tau'}
%   }
% \end{equation}

% \end{subequations}

\noindent\fbox{$\strut\rExpandsSP{\uOmega}{\uPsi}{\uPhi}{\urv}{r}{\tau}{\tau'}$}~~$\urv$ has expansion $r$ taking values of type $\tau$ to values of type $\tau'$
\begin{equation}\label{rule:rExpandsP}
  \inferrule{
    \uOmega=\uOmegaEx{\uD}{\uG}{\uMctx}{\Omega}\\
    \patExpandsP{\uOmegaEx{\emptyset}{\uG'}{\emptyset}{\Omega'}}{\uPhi}{\upv}{p}{\tau}\\
    \expandsP{\uOmegaEx{\uD}{\uG \uplus \uG'}{\uMctx}{\Omega \cup \Omega'}}{\uPsi}{\uPhi}{\ue}{e}{\tau'}
  }{
    \rExpandsSP{\uOmega}{\uPsi}{\uPhi}{\matchrule{\upv}{\ue}}{\aematchrule{p}{e}}{\tau}{\tau'}
  }
\end{equation}


\noindent\fbox{$\strut\patExpandsP{\uOmega'}{\uPhi}{\upv}{p}{\tau}$}~~$\upv$ has expansion $p$ matching against $\tau$ generating hypotheses $\uOmega'$
% The typed pattern expansion judgement is inductively defined by Rules (\ref{rules:patExpandsP}) as follows. %As in $\miniVersePat$, \emph{unexpanded pattern typing contexts}, $\upctx$, are defined identically to unexpanded typing contexts (i.e. we only use a distinct metavariable to emphasize their distinct roles in the judgements above). 

% The following rules are written identically to the typed pattern expansion rules for shared pattern forms in $\miniVersePat$, i.e. Rules (\ref{rule:patExpands-var}) through (\ref{rule:patExpands-in}).
\begin{subequations}\label{rules:patExpandsP}
\begin{equation}\label{rule:patExpandsP-subsume}
\inferrule{
  \uOmega=\uOmegaEx{\uD}{\uG}{\uMctx}{\Omega}\\\\
  \patExpandsP{\uOmega'}{\uPhi}{\upv}{p}{\tau}\\
  \issubtypeP{\Omega}{\tau}{\tau'}
}{
  \patExpandsP{\uOmega'}{\uPhi}{\upv}{p}{\tau'}
}
\end{equation}
\begin{equation}\label{rule:patExpandsP-var}
\inferrule{ }{
  \patExpandsP{\uOmegaEx{\emptyset}{\vExpands{\ux}{x}}{\emptyset}{\Ghyp{x}{\tau}}}{\uPhi}{\ux}{x}{\tau}
}
\end{equation}
\begin{equation}\label{rule:patExpandsP-wild}
\inferrule{ }{
  \patExpandsP{\uOmegaEx{\emptyset}{\emptyset}{\emptyset}{\emptyset}}{\uPhi}{\wildp}{\aewildp}{\tau}
}
\end{equation}
\begin{equation}\label{rule:patExpandsP-fold}
\inferrule{ 
  \patExpandsP{\uOmega'}{\uPhi}{\upv}{p}{[\arec{t}{\tau}/t]\tau}
}{
  \patExpandsP{\uOmega'}{\uPhi}{\foldp{\upv}}{\aefoldp{p}}{\arec{t}{\tau}}
}
\end{equation}
\begin{equation}\label{rule:patExpandsP-tpl}
\inferrule{
  \tau=\aprod{\labelset}{\mapschema{\tau}{i}{\labelset}}\\\\
  \{\patExpandsP{{\uOmega_i}}{\uPhi}{\upv_i}{p_i}{\tau_i}\}_{i \in \labelset}
}{
  %\patExpandsP{\Gconsi{i \in \labelset}{\upctx_i}}{A}{B}{C}
  \patExpandsP{\Gconsi{i \in \labelset}{\uOmega_i}}{\uPhi}{\tplp{\mapschema{\upv}{i}{\labelset}}}{\aetplp{\labelset}{\mapschema{p}{i}{\labelset}}}{\tau}
  % \patExpands{\Gconsi{i \in \labelset}{\pctx_i}}{\Phi}{
  %   \autplp{\labelset}{\mapschema{\upv}{i}{\labelset}}
  % }{
  %   \aetplp{\labelset}{\mapschema{p}{i}{\labelset}}
  % }{
  %   \aprod{\labelset}{\mapschema{\tau}{i}{\labelset}}
  % } %{\autplp{\labelset}{\mapschema{\upv}{i}{\labelset}}}{\aetplp{\labelset}{\mapschema}{p}{i}{\labelset}}{...}
  %\left(\shortstack{$\Delta \vdash_{\uPhi} \autplp{\labelset}{\mapschema{\upv}{i}{\labelset}}$\\$\leadsto$\\$\aetplp{\labelset}{\mapschema{p}{i}{\labelset}} : \aprod{\labelset}{\mapschema{\tau}{i}{\labelset}} \dashV \Gconsi{i \in \labelset}{\upctx_i}$\vspace{-1.2em}}\right)
}
\end{equation}
\begin{equation}\label{rule:patExpandsP-in}
\inferrule{
  \patExpandsP{\uOmega'}{\uPhi}{\upv}{p}{\tau}
}{
  \patExpandsP{\uOmega'}{\uPhi}{\injp{\ell}{\upv}}{\aeinjp{\ell}{p}}{\asum{\labelset, \ell}{\mapschema{\tau}{i}{\labelset}; \mapitem{\ell}{\tau}}}
}
\end{equation}

\begin{equation}\label{rule:patExpandsP-apuptsm}
\inferrule{
  \uOmega=\uOmegaEx{\uD}{\uG}{\uMctx}{\Omega_\text{app}}\\
  \uPhi=\uAS{\uA}{\Phi}\\\\
  \tsmexpExpandsPat{\uOmega}{\uPhi}{\uepsilon}{\epsilon}{\aetype{\tau_\text{final}}}\\
  \tsmexpEvalsPat{\Omega_\text{app}}{\Phi}{\epsilon}{\epsilon_\text{normal}}\\\\
  \tsmdefof{\epsilon_\text{normal}}=a\\
  \Phi = \Phi', \pptsmdefn{a}{\rho}{\eparse}\\\\
  \encodeBody{b}{\ebody}\\
  \evalU{\ap{\eparse}{\ebody}}{{\lbltxt{SuccessP}}\cdot{e_\text{pproto}}}\\
  \decodePCEPat{e_\text{pproto}}{\pcp}\\\\
  \prepcp{\Omega_\text{app}}{\Phi}{\pcp}{\cpv}{\epsilon_\text{normal}}{\aetype{\tau_\text{proto}}}{\omega}{\Omega_\text{params}}\\\\
      \segOK{\segof{\cpv}}{b}\\
  \cvalidPP{\uOmega'}{\psceneP{\uOmega}{\uPhi}{b}}{\cpv}{p}{\tau_\text{proto}}
}{
  \patExpandsP{\uOmega'}{\uPhi}{\utsmap{\uepsilon}{b}}{p}{[\omega]\tau_\text{proto}}
}
\end{equation}
\end{subequations}

\subsubsection{TSM Types and Expressions}
\noindent\fbox{$\strut\istsmty{\Omega}{\rho}$}~~$\rho$ is a TSM type
\begin{subequations}\label{rules:istsmty}
\begin{equation}\label{rule:istsmty-type}
\inferrule{
  \haskindX{\tau}{\akty}
}{
  \istsmty{\Omega}{\aetype{\tau}}
}
\end{equation}
\begin{equation}\label{rule:istsmty-alltypes}
\inferrule{
  \istsmty{\Omega, t :: \akty}{\rho}
}{
  \istsmty{\Omega}{\aealltypes{t}{\rho}}
}
\end{equation}
\begin{equation}\label{rule:istsmty-allmods}
\inferrule{
  \issig{\Omega}{\sigma}\\
  \istsmty{\Omega, X : \sigma}{\rho}
}{
  \istsmty{\Omega}{\aeallmods{\sigma}{X}{\rho}}
}
\end{equation}
\end{subequations}

\noindent\fbox{$\strut\tsmtyExpands{\uOmega}{\urho}{\rho}$}~~$\urho$ has well-formed expansion $\rho$
\begin{subequations}\label{rules:tsmtyExpands}
\begin{equation}\label{rule:tsmtyExpands-type}
\inferrule{
  \cExpandsX{\utau}{\tau}{\akty}
}{
  \tsmtyExpands{\uOmega}{{\utau}}{\aetype{\tau}}
}
\end{equation}
\begin{equation}\label{rule:tsmtyExpands-alltypes}
\inferrule{
  \tsmtyExpands{\uOmega, \uKhyp{\ut}{t}{\akty}}{\urho}{\rho}
}{
  \tsmtyExpands{\uOmega}{\alltypes{\ut}{\urho}}{\aealltypes{t}{\rho}}
}
\end{equation}
\begin{equation}\label{rule:tsmtyExpands-allmods}
\inferrule{
  \sigExpandsPX{\usigma}{\sigma}\\
  \tsmtyExpands{\uOmega, \uMhyp{\uX}{X}{\sigma}}{\urho}{\rho}
}{
  \tsmtyExpands{\uOmega}{\allmods{\uX}{\usigma}{\urho}}{\aeallmods{\sigma}{X}{\rho}}
}
\end{equation}
\end{subequations}

\noindent\fbox{$\strut\hastsmtypeExp{\Omega}{\Psi}{\epsilon}{\rho}$}~~$\epsilon$ is a peTSM expression at $\rho$
\begin{subequations}\label{rules:hastsmtypeExp}
\begin{equation}\label{rule:hastsmtypeExp-defref}
\inferrule{ }{
  \hastsmtypeExp{\Omega}{\Psi, \petsmdefn{a}{\rho}{\eparse}}{\adefref{a}}{\rho}
}
\end{equation}
\begin{equation}\label{rule:hastsmtypeExp-abstype}
\inferrule{
  \hastsmtypeExp{\Omega, t :: \akty}{\Psi}{\epsilon}{\rho}
}{
  \hastsmtypeExp{\Omega}{\Psi}{\aeabstype{t}{\epsilon}}{\aealltypes{t}{\rho}}
}
\end{equation}
\begin{equation}\label{rule:hastsmtypeExp-absmod}
\inferrule{
  \issigX{\sigma}\\
  \hastsmtypeExp{\Omega, X : \sigma}{\Psi}{\epsilon}{\rho}
}{
  \hastsmtypeExp{\Omega}{\Psi}{\aeabsmod{\sigma}{X}{\epsilon}}{\aeallmods{\sigma}{X}{\rho}}
}
\end{equation}
\begin{equation}\label{rule:hastsmtypeExp-aptype}
\inferrule{
  \hastsmtypeExp{\Omega}{\Psi}{\epsilon}{\aealltypes{t}{\rho}}\\
  \haskindX{\tau}{\akty}
}{
  \hastsmtypeExp{\Omega}{\Psi}{\aeaptype{\tau}{\epsilon}}{[\tau/t]\rho}
}
\end{equation}
\begin{equation}\label{rule:hastsmtypeExp-apmod}
\inferrule{
  \hastsmtypeExp{\Omega}{\Psi}{\epsilon}{\aeallmods{\sigma}{X'}{\rho}}\\
  \hassig{\Omega}{X}{\sigma}
}{
  \hastsmtypeExp{\Omega}{\Psi}{\aeapmod{X}{\epsilon}}{[X/X']\rho}
}
\end{equation}
\end{subequations}

\noindent\fbox{$\strut\hastsmtypePat{\Omega}{\Phi}{\epsilon}{\rho}$}~~$\epsilon$ is a ppTSM expression at $\rho$
\begin{subequations}\label{rules:hastsmtypePat}
\begin{equation}\label{rule:hastsmtypePat-defref}
\inferrule{ }{
  \hastsmtypePat{\Omega}{\Phi, \pptsmdefn{a}{\rho}{\eparse}}{\adefref{a}}{\rho}
}
\end{equation}
\begin{equation}\label{rule:hastsmtypePat-abstype}
\inferrule{
  \hastsmtypePat{\Omega, t :: \akty}{\Phi}{\epsilon}{\rho}
}{
  \hastsmtypePat{\Omega}{\Phi}{\aeabstype{t}{\epsilon}}{\aealltypes{t}{\rho}}
}
\end{equation}
\begin{equation}\label{rule:hastsmtypePat-absmod}
\inferrule{
  \issigX{\sigma}\\
  \hastsmtypePat{\Omega, X : \sigma}{\Phi}{\epsilon}{\rho}
}{
  \hastsmtypePat{\Omega}{\Phi}{\aeabsmod{\sigma}{X}{\epsilon}}{\aeallmods{\sigma}{X}{\rho}}
}
\end{equation}
\begin{equation}\label{rule:hastsmtypePat-aptype}
\inferrule{
  \hastsmtypePat{\Omega}{\Phi}{\epsilon}{\aealltypes{t}{\rho}}\\
  \haskindX{\tau}{\akty}
}{
  \hastsmtypePat{\Omega}{\Phi}{\aeaptype{\tau}{\epsilon}}{[\tau/t]\rho}
}
\end{equation}
\begin{equation}\label{rule:hastsmtypePat-apmod}
\inferrule{
  \hastsmtypePat{\Omega}{\Phi}{\epsilon}{\aeallmods{\sigma}{X'}{\rho}}\\
  \hassig{\Omega}{X}{\sigma}
}{
  \hastsmtypePat{\Omega}{\Phi}{\aeapmod{X}{\epsilon}}{[X/X']\rho}
}
\end{equation}

\end{subequations}

\noindent\fbox{$\strut\tsmexpExpandsExp{\uOmega}{\uPsi}{\uepsilon}{\epsilon}{\rho}$}~~$\uepsilon$ has peTSM expression expansion $\epsilon$ at $\rho$
\begin{subequations}\label{rules:tsmexpExpandsExp}
\begin{equation}\label{rule:tsmexpExpandsExp-bindref}
\inferrule{
  \hastsmtypeExp{\Omega}{\Psi}{\epsilon}{\rho}  
}{
  \tsmexpExpandsExp{\uOmegaEx{\uD}{\uG}{\uMctx}{\Omega}}{\uAS{\uA, \mapitem{\tsmv}{\epsilon}}{\Psi}}{{\tsmv}}{\epsilon}{\rho}
}
\end{equation}
\begin{equation}\label{rule:tsmexpExpandsExp-abstype}
\inferrule{
  \tsmexpExpandsExp{\uOmega, \uKhyp{\ut}{t}{\akty}}{\uPsi}{\uepsilon}{\epsilon}{\rho}
}{
  \tsmexpExpandsExp{\uOmega}{\uPsi}{\abstype{\ut}{\uepsilon}}{\aeabstype{t}{\epsilon}}{\aealltypes{t}{\rho}}
}
\end{equation}
\begin{equation}\label{rule:tsmexpExpandsExp-absmod}
\inferrule{
  \sigExpandsPX{\usigma}{\sigma}\\
  \tsmexpExpandsExp{\uOmega, \uMhyp{\uX}{X}{\sigma}}{\uPsi}{\uepsilon}{\epsilon}{\rho}
}{
  \tsmexpExpandsExp{\uOmega}{\uPsi}{\absmod{\uX}{\usigma}{\uepsilon}}{\aeabsmod{\sigma}{X}{\epsilon}}{\aeallmods{\sigma}{X}{\rho}}
}
\end{equation}
\begin{equation}\label{rule:tsmexpExpandsExp-aptype}
\inferrule{
  \tsmexpExpandsExp{\uOmega}{\uPsi}{\uepsilon}{\epsilon}{\aealltypes{t}{\rho}}\\
  \cExpandsX{\utau}{\tau}{\akty}
}{
  \tsmexpExpandsExp{\uOmega}{\uPsi}{\aptype{\uepsilon}{\utau}}{\aeaptype{\tau}{\epsilon}}{[\tau/t]\rho} 
}
\end{equation}
\begin{equation}\label{rule:tsmexpExpandsExp-apmod}
\inferrule{
  \tsmexpExpandsExp{\uOmega}{\uPsi}{\uepsilon}{\epsilon}{\aeallmods{\sigma}{X'}{\rho}}\\
  \mExpandsPX{\uX}{X}{\sigma}
}{
  \tsmexpExpandsExp{\uOmega}{\uPsi}{\apmod{\uepsilon}{\uX}}{\aeapmod{X}{\epsilon}}{[X/X']\rho}
}
\end{equation}
\end{subequations}

\noindent\fbox{$\strut\tsmexpExpandsPat{\uOmega}{\uPsi}{\uepsilon}{\epsilon}{\rho}$}~~$\uepsilon$ has ppTSM expression expansion $\epsilon$ at $\rho$
\begin{subequations}\label{rules:tsmexpExpandsPat}
\begin{equation}\label{rule:tsmexpExpandsPat-bindref}
\inferrule{
  \hastsmtypePat{\Omega}{\Phi}{\epsilon}{\rho}  
}{
  \tsmexpExpandsPat{\uOmegaEx{\uD}{\uG}{\uMctx}{\Omega}}{\uAS{\uA, \mapitem{\tsmv}{\epsilon}}{\Phi}}{{\tsmv}}{\epsilon}{\rho}
}
\end{equation}
\begin{equation}\label{rule:tsmexpExpandsPat-abstype}
\inferrule{
  \tsmexpExpandsPat{\uOmega, \uKhyp{\ut}{t}{\akty}}{\uPhi}{\uepsilon}{\epsilon}{\rho}
}{
  \tsmexpExpandsPat{\uOmega}{\uPhi}{\abstype{\ut}{\uepsilon}}{\aeabstype{t}{\epsilon}}{\aealltypes{t}{\rho}}
}
\end{equation}
\begin{equation}\label{rule:tsmexpExpandsPat-absmod}
\inferrule{
  \sigExpandsPX{\usigma}{\sigma}\\
  \tsmexpExpandsPat{\uOmega, \uMhyp{\uX}{X}{\sigma}}{\uPhi}{\uepsilon}{\epsilon}{\rho}
}{
  \tsmexpExpandsPat{\uOmega}{\uPhi}{\absmod{\uX}{\usigma}{\uepsilon}}{\aeabsmod{\sigma}{X}{\epsilon}}{\aeallmods{\sigma}{X}{\rho}}
}
\end{equation}
\begin{equation}\label{rule:tsmexpExpandsPat-aptype}
\inferrule{
  \tsmexpExpandsPat{\uOmega}{\uPhi}{\uepsilon}{\epsilon}{\aealltypes{t}{\rho}}\\
  \cExpandsX{\utau}{\tau}{\akty}
}{
  \tsmexpExpandsPat{\uOmega}{\uPhi}{\aptype{\uepsilon}{\utau}}{\aeaptype{\tau}{\epsilon}}{[\tau/t]\rho} 
}
\end{equation}
\begin{equation}\label{rule:tsmexpExpandsPat-apmod}
\inferrule{
  \tsmexpExpandsPat{\uOmega}{\uPhi}{\uepsilon}{\epsilon}{\aeallmods{\sigma}{X'}{\rho}}\\
  \mExpandsPX{\uX}{X}{\sigma}
}{
  \tsmexpExpandsPat{\uOmega}{\uPhi}{\apmod{\uepsilon}{\uX}}{\aeapmod{X}{\epsilon}}{[X/X']\rho}
}
\end{equation}
\end{subequations}

\noindent\fbox{$\tsmexpNormalExp{\Omega}{\Psi}{\epsilon}$}~~$\epsilon$ is a normal peTSM expression
\begin{subequations}\label{rules:tsmexpNormalExp}
\begin{equation}\label{rule:tsmexpNormalExp-defref}
\inferrule{ }{
  \tsmexpNormalExp{\Omega}{\Psi, \petsmdefn{a}{\rho}{\eparse}}{\adefref{a}}
}
\end{equation}
\begin{equation}\label{rule:tsmexpNormalExp-abstype}
\inferrule{ }{
  \tsmexpNormalExp{\Omega}{\Psi}{\aeabstype{t}{\epsilon}}
}
\end{equation}
\begin{equation}\label{rule:tsmexpNormalExp-absmod}
\inferrule{ }{
  \tsmexpNormalExp{\Omega}{\Psi}{\aeabsmod{\sigma}{X}{\epsilon}}
}
\end{equation}
\begin{equation}\label{rule:tsmexpNormalExp-aptype}
\inferrule{
  \epsilon \neq \aeabstype{t}{\epsilon'}\\
  \tsmexpNormalExp{\Omega}{\Psi}{\epsilon}
}{
  \tsmexpNormalExp{\Omega}{\Psi}{\aeaptype{\tau}{\epsilon}}
}
\end{equation}
\begin{equation}\label{rule:tsmexpNormalExp-apmod}
\inferrule{
  \epsilon \neq \aeabsmod{\sigma}{X'}{\epsilon'}\\
  \tsmexpNormalExp{\Omega}{\Psi}{\epsilon}
}{
  \tsmexpNormalExp{\Omega}{\Psi}{\aeapmod{X}{\epsilon}}
}
\end{equation}
\end{subequations}

\noindent\fbox{$\tsmexpNormalPat{\Omega}{\Psi}{\epsilon}$}~~$\epsilon$ is a normal ppTSM expression
\begin{subequations}\label{rules:tsmexpNormalPat}
\begin{equation}\label{rule:tsmexpNormalPat-defref}
\inferrule{ }{
  \tsmexpNormalPat{\Omega}{\Psi, \petsmdefn{a}{\rho}{\eparse}}{\adefref{a}}
}
\end{equation}
\begin{equation}\label{rule:tsmexpNormalPat-abstype}
\inferrule{ }{
  \tsmexpNormalPat{\Omega}{\Psi}{\aeabstype{t}{\epsilon}}
}
\end{equation}
\begin{equation}\label{rule:tsmexpNormalPat-absmod}
\inferrule{ }{
  \tsmexpNormalPat{\Omega}{\Psi}{\aeabsmod{\sigma}{X}{\epsilon}}
}
\end{equation}
\begin{equation}\label{rule:tsmexpNormalPat-aptype}
\inferrule{
  \epsilon \neq \aeabstype{t}{\epsilon'}\\
  \tsmexpNormalPat{\Omega}{\Psi}{\epsilon}
}{
  \tsmexpNormalPat{\Omega}{\Psi}{\aeaptype{\tau}{\epsilon}}
}
\end{equation}
\begin{equation}\label{rule:tsmexpNormalPat-apmod}
\inferrule{
  \epsilon \neq \aeabsmod{\sigma}{X'}{\epsilon'}\\
  \tsmexpNormalPat{\Omega}{\Psi}{\epsilon}
}{
  \tsmexpNormalPat{\Omega}{\Psi}{\aeapmod{X}{\epsilon}}
}
\end{equation}
\end{subequations}

\noindent\fbox{$\strut\tsmexpStepsExp{\Omega}{\Psi}{\epsilon}{\epsilon'}$}~~peTSM expression $\epsilon$ transitions to $\epsilon'$
\begin{subequations}\label{rules:tsmexpStepsExp}
\begin{equation}\label{rule:tsmexpStepsExp-aptype-1}
\inferrule{
  \tsmexpStepsExp{\Omega}{\Psi}{\epsilon}{\epsilon'}
}{
  \tsmexpStepsExp{\Omega}{\Psi}{\aeaptype{\tau}{\epsilon}}{\aeaptype{\tau}{\epsilon'}}
}
\end{equation}
\begin{equation}\label{rule:tsmexpStepsExp-aptype-2}
\inferrule{ }{
  \tsmexpStepsExp{\Omega}{\Psi}{\aeaptype{\tau}{\aeabstype{t}{\epsilon}}}{[\tau/t]\epsilon}
}
\end{equation}
\begin{equation}\label{rule:tsmexpStepsExp-apmod-1}
\inferrule{
  \tsmexpStepsExp{\Omega}{\Psi}{\epsilon}{\epsilon'}
}{
  \tsmexpStepsExp{\Omega}{\Psi}{\aeapmod{X}{\epsilon}}{\aeapmod{X}{\epsilon'}}
}
\end{equation}
\begin{equation}\label{rule:tsmexpStepsExp-apmod-2}
\inferrule{ }{
  \tsmexpStepsExp{\Omega}{\Psi}{\aeapmod{X}{\aeabsmod{\sigma}{X'}{\epsilon}}}{[X/X']\epsilon}
}
\end{equation}
\end{subequations}

\noindent\fbox{$\strut\tsmexpStepsPat{\Omega}{\Psi}{\epsilon}{\epsilon'}$}~~peTSM expression $\epsilon$ transitions to $\epsilon'$
\begin{subequations}\label{rules:tsmexpStepsPat}
\begin{equation}\label{rule:tsmexpStepsPat-aptype-1}
\inferrule{
  \tsmexpStepsPat{\Omega}{\Psi}{\epsilon}{\epsilon'}
}{
  \tsmexpStepsPat{\Omega}{\Psi}{\aeaptype{\tau}{\epsilon}}{\aeaptype{\tau}{\epsilon'}}
}
\end{equation}
\begin{equation}\label{rule:tsmexpStepsPat-aptype-2}
\inferrule{ }{
  \tsmexpStepsPat{\Omega}{\Psi}{\aeaptype{\tau}{\aeabstype{t}{\epsilon}}}{[\tau/t]\epsilon}
}
\end{equation}
\begin{equation}\label{rule:tsmexpStepsPat-apmod-1}
\inferrule{
  \tsmexpStepsPat{\Omega}{\Psi}{\epsilon}{\epsilon'}
}{
  \tsmexpStepsPat{\Omega}{\Psi}{\aeapmod{X}{\epsilon}}{\aeapmod{X}{\epsilon'}}
}
\end{equation}
\begin{equation}\label{rule:tsmexpStepsPat-apmod-2}
\inferrule{ }{
  \tsmexpStepsPat{\Omega}{\Psi}{\aeapmod{X}{\aeabsmod{\sigma}{X'}{\epsilon}}}{[X/X']\epsilon}
}
\end{equation}
\end{subequations}

\noindent\fbox{$\strut\tsmexpMultistepsExp{\Omega}{\Psi}{\epsilon}{\epsilon'}$}~~peTSM expression $\epsilon$ transitions in multiple steps to $\epsilon'$
\begin{subequations}\label{rules:tsmexpMultistepsExp}
\begin{equation}\label{rule:tsmexpMultistepsExp-refl}
\inferrule{ }{
  \tsmexpMultistepsExp{\Omega}{\Psi}{\epsilon}{\epsilon}
}
\end{equation}
\begin{equation}\label{rule:tsmexpMultistepsExp-steps}
\inferrule{
  \tsmexpStepsExp{\Omega}{\Psi}{\epsilon}{\epsilon'}
}{
  \tsmexpMultistepsExp{\Omega}{\Psi}{\epsilon}{\epsilon'}
}
\end{equation}
\begin{equation}\label{rule:tsmexpMultistepsExp-trans}
\inferrule{
  \tsmexpMultistepsExp{\Omega}{\Psi}{\epsilon}{\epsilon'}\\
  \tsmexpMultistepsExp{\Omega}{\Psi}{\epsilon'}{\epsilon''}
}{
  \tsmexpMultistepsExp{\Omega}{\Psi}{\epsilon}{\epsilon''}
}
\end{equation}
\end{subequations}

\noindent\fbox{$\strut\tsmexpMultistepsPat{\Omega}{\Psi}{\epsilon}{\epsilon'}$}~~peTSM expression $\epsilon$ transitions in multiple steps to $\epsilon'$
\begin{subequations}\label{rules:tsmexpMultistepsPat}
\begin{equation}\label{rule:tsmexpMultistepsPat-refl}
\inferrule{ }{
  \tsmexpMultistepsPat{\Omega}{\Psi}{\epsilon}{\epsilon}
}
\end{equation}
\begin{equation}\label{rule:tsmexpMultistepsPat-steps}
\inferrule{
  \tsmexpStepsExp{\Omega}{\Psi}{\epsilon}{\epsilon'}
}{
  \tsmexpMultistepsPat{\Omega}{\Psi}{\epsilon}{\epsilon'}
}
\end{equation}
\begin{equation}\label{rule:tsmexpMultistepsPat-trans}
\inferrule{
  \tsmexpMultistepsPat{\Omega}{\Psi}{\epsilon}{\epsilon'}\\
  \tsmexpMultistepsPat{\Omega}{\Psi}{\epsilon'}{\epsilon''}
}{
  \tsmexpMultistepsPat{\Omega}{\Psi}{\epsilon}{\epsilon''}
}
\end{equation}
\end{subequations}

\noindent\fbox{$\strut\tsmexpEvalsExp{\Omega}{\Psi}{\epsilon}{\epsilon'}$}~~peTSM expression $\epsilon$ normalizes to $\epsilon'$
\begin{equation}\label{rule:tsmexpEvalsExp}
\inferrule{
  \tsmexpMultistepsExp{\Omega}{\Psi}{\epsilon}{\epsilon'}\\
  \tsmexpNormalExp{\Omega}{\Psi}{\epsilon'}
}{
  \tsmexpEvalsExp{\Omega}{\Psi}{\epsilon}{\epsilon'}
}
\end{equation}


\noindent\fbox{$\strut\tsmexpEvalsPat{\Omega}{\Psi}{\epsilon}{\epsilon'}$}~~peTSM expression $\epsilon$ normalizes to $\epsilon'$
\begin{equation}\label{rule:tsmexpEvalsPat}
\inferrule{
  \tsmexpMultistepsExp{\Omega}{\Psi}{\epsilon}{\epsilon'}\\
  \tsmexpNormalExp{\Omega}{\Psi}{\epsilon'}
}{
  \tsmexpEvalsPat{\Omega}{\Psi}{\epsilon}{\epsilon'}
}
\end{equation}

The following metafunction extracts the TSM name from a TSM expression.
\begin{subequations}
\begin{align}
\tsmdefof{\adefref{a}} & = a \label{eqn:tsmdefof-adefref}\\
\tsmdefof{\aeabstype{t}{\epsilon}} & = \tsmdefof{\epsilon} \label{eqn:tsmdefof-abstype}\\
\tsmdefof{\aeabsmod{\sigma}{X}{\epsilon}} & = \tsmdefof{\epsilon} \label{eqn:tsmdefof-absmod}\\
\tsmdefof{\aeaptype{\tau}{\epsilon}} & = \tsmdefof{\epsilon} \label{eqn:tsmdefof-aptype}\\
\tsmdefof{\aeapmod{X}{\epsilon}} & = \tsmdefof{\epsilon} \label{eqn:tsmdefof-apmod}
\end{align}
\end{subequations}
\section{Proto-Expansion Validation}\label{appendix:P-proto-expansion-validation}
\subsection{Syntax of Proto-Expansions}
\subsubsection{Syntax -- Parameterized Proto-Expressions}
\[\begin{array}{lllllll}
\textbf{Sort} & & & \textbf{Operational Form} & \textbf{Stylized Form} & \textbf{Description}\\
% \LCC \color{Yellow}&\color{Yellow}&\color{Yellow}& \color{Yellow} & \color{Yellow} & \color{Yellow}\\
\mathsf{PPrExpr} & \pce & ::= & \apceexp{\ce} & \pceexp{\ce} & \text{proto-expression}\\
&&& \apcebindtype{t}{\pce} & \pcebindtype{t}{\pce} & \text{type binding}\\
&&& \apcebindmod{X}{\pce} & \pcebindmod{X}{\pce} & \text{module binding}%\ECC
\end{array}\]

\subsubsection{Syntax -- Parameterized Proto-Patterns}
\[\begin{array}{lllllll}
\textbf{Sort} & & & \textbf{Operational Form} & \textbf{Stylized Form} & \textbf{Description}\\
% \LCC \color{Yellow}&\color{Yellow}&\color{Yellow}& \color{Yellow} & \color{Yellow} & \color{Yellow}\\
\mathsf{PPrPat} & \pcp & ::= & \apcepat{\cpv} & {\cpv} & \text{proto-pattern}\\
&&& \apcebindtype{t}{\pcp} & \pcebindtype{t}{\pcp} & \text{type binding}\\
&&& \apcebindmod{X}{\pcp} & \pcebindmod{X}{\pcp} & \text{module binding}%\ECC
\end{array}\]

\subsubsection{Syntax -- Proto-Kinds and Proto-Constructors}
\[\begin{array}{lrlllll}
\textbf{Sort} & & & \textbf{Operational Form} & \textbf{Stylized Form} & \textbf{Description}\\
\mathsf{PrKind} & \cekappa & ::= & \acekdarr{\cekappa}{u}{\cekappa} & \kdarr{u}{\cekappa}{\cekappa} & \text{dependent function}\\
&&& \acekunit & \kunit & \text{nullary product}\\
&&& \acekdbprod{\cekappa}{u}{\cekappa} & \kdbprod{u}{\cekappa}{\cekappa} & \text{dependent product}\\
%&&& \akdprodstd & \kdprodstd & \text{labeled dependent product}\\
&&& \acekty & \kty & \text{types}\\
&&& \aceksing{\ctau} & \ksing{\ctau} & \text{singleton}\\
% \LCC &&& \color{Yellow} & \color{Yellow} & \color{Yellow}\\
&&& \acesplicedk{m}{n} & \splicedk{m}{n} & \text{spliced kind}\\%\ECC\\
\mathsf{PrCon} & \cec, \ctau & ::= & u & u & \text{constructor variable}\\
&&& t & t & \text{type variable}\\
% &&& \acecasc{\cekappa}{\cec} & \casc{\cec}{\cekappa} & \text{ascription}\\
&&& \acecabs{u}{\cec} & \cabs{u}{\cec} & \text{abstraction}\\
&&& \acecapp{\cec}{\cec} & \capp{\cec}{\cec} & \text{application}\\
&&& \acectriv & \ctriv & \text{trivial}\\
&&& \acecpair{\cec}{\cec} & \cpair{\cec}{\cec} & \text{pair}\\
&&& \acecprl{\cec} & \cprl{\cec} & \text{left projection}\\
&&& \acecprr{\cec} & \cprr{\cec} & \text{right projection}\\
%&&& \adtplX & \dtplX & \text{labeled dependent tuple}\\
%&&& \adprj{\ell}{c} & \prj{c}{\ell} & \text{projection}\\
&&& \aceparr{\ctau}{\ctau} & \parr{\ctau}{\ctau} & \text{partial function}\\
&&& \aceallu{\cekappa}{u}{\ctau} & \forallu{u}{\cekappa}{\ctau} & \text{polymorphic}\\
&&& \acerec{t}{\ctau} & \rect{t}{\ctau} & \text{recursive}\\
&&& \aceprod{\labelset}{\mapschema{\ctau}{i}{\labelset}} & \prodt{\mapschema{\ctau}{i}{\labelset}} & \text{labeled product}\\
&&& \acesum{\labelset}{\mapschema{\ctau}{i}{\labelset}} & \sumt{\mapschema{\ctau}{i}{\labelset}} & \text{labeled sum}\\
&&& \acemcon{X} & \mcon{X} & \text{constructor component}\\
% \LCC &&& \color{Yellow} & \color{Yellow} & \color{Yellow}\\
&&& \acesplicedc{m}{n}{\cekappa} & \splicedc{m}{n}{\cekappa} & \text{spliced constructor}%\ECC
\end{array}\]

\subsubsection{Syntax -- Proto-Expressions and Proto-Rules}
\[\arraycolsep=4pt\begin{array}{lllllll}
\textbf{Sort} & & & \textbf{Operational Form} & \textbf{Stylized Form} & \textbf{Description}\\
\mathsf{PrExp} & \ce & ::= & x & x & \text{variable}\\
&&& \aceasc{\ctau}{\ce} & \asc{\ce}{\ctau} & \text{ascription}\\
&&& \aceletsyn{x}{\ce}{\ce} & \letsyn{x}{\ce}{\ce} & \text{value binding}\\
% &&& \aceanalam{x}{\ce} & \analam{x}{\ce} & \text{abstraction (unannotated)}\\
&&& \acelam{\ctau}{x}{\ce} & \lam{x}{\ctau}{\ce} & \text{abstraction}\\
&&& \aceap{\ce}{\ce} & \ap{\ce}{\ce} & \text{application}\\
&&& \aceclam{\cekappa}{u}{\ce} & \clam{u}{\cekappa}{\ce} & \text{constructor abstraction}\\
&&& \acecap{\ce}{\cec} & \cAp{\ce}{\cec} & \text{constructor application}\\
&&& \acefold{\ce} & \fold{\ce} & \text{fold}\\
&&& \aceunfold{\ce} & \unfold{\ce} & \text{unfold}\\
&&& \acetpl{\labelset}{\mapschema{\ce}{i}{\labelset}} & \tpl{\mapschema{\ce}{i}{\labelset}} & \text{labeled tuple}\\
&&& \acepr{\ell}{\ce} & \prj{\ce}{\ell} & \text{projection}\\
&&& \aceanain{\ell}{\ce} & \inj{\ell}{\ce} & \text{injection}\\
&&& \acematchwith{n}{\ce}{\seqschemaX{\crv}} & \matchwith{\ce}{\seqschemaX{\crv}} & \text{match}\\
&&& \acemval{X} & \mval{X} & \text{value component}\\
% \LCC &&& \color{Yellow} & \color{Yellow} & \color{Yellow}\\
&&& \acesplicede{m}{n}{\ctau} & \splicede{m}{n}{\ctau} & \text{spliced expression}\\%\ECC\\
\mathsf{PrRule} & \crv & ::= & \acematchrule{p}{\ce} & \matchrule{p}{\ce} & \text{rule}\end{array}\]

\subsubsection{Syntax -- Proto-Patterns}
\[\begin{array}{lllllll}
\mathsf{PrPat} & \cpv & ::= & \acewildp & \wildp & \text{wildcard pattern}\\
&&& \acefoldp{p} & \foldp{p} & \text{fold pattern}\\
&&& \acetplp{\labelset}{\mapschema{\cpv}{i}{\labelset}} & \tplp{\mapschema{\cpv}{i}{\labelset}} & \text{labeled tuple pattern}\\
&&& \aceinjp{\ell}{\cpv} & \injp{\ell}{\cpv} & \text{injection pattern}\\
% \LCC &&& \color{Yellow} & \color{Yellow} & \color{Yellow}\\
&&& \acesplicedp{m}{n}{\ctau} & \splicedp{m}{n}{\ctau} & \text{spliced pattern} %\ECC
\end{array}\]

\subsubsection{Common Proto-Expansion Terms}
Each expanded term, with a few exceptions noted below, maps onto a proto-expansion term. We refer to these as the \emph{common proto-expansion terms}. In particular:
\begin{itemize}
  \item Each kind, $\kappa$, maps onto a proto-kind, $\Cof{\kappa}$, as follows:
  \[\arraycolsep=1pt\begin{array}{rl}
  \Cof{\akdarr{\kappa_1}{u}{\kappa_2}} & = \acekdarr{\Cof{\kappa_1}}{u}{\Cof{\kappa_2}}\\
  \Cof{\akunit} & = \acekunit\\
  \Cof{\akdbprod{\kappa_1}{u}{\kappa_2}} & = \acekdbprod{\Cof{\kappa_1}}{u}{\Cof{\kappa_2}}\\
  \Cof{\akty} & = \acekty\\
  \Cof{\aksing{\tau}} & = \aceksing{\Cof{\tau}}
  \end{array}\]
  \item Each constructor, $c$, maps onto a proto-constructor, $\Cof{c}$, as follows:
  \[\arraycolsep=1pt\begin{array}{rl}
  \Cof{u} & = u\\
  \Cof{\acabs{u}{c}} & = \acecabs{u}{\Cof{c}}\\
  \Cof{\acapp{c_1}{c_2}} & = \acecapp{\Cof{c_1}}{\Cof{c_2}}\\
  \Cof{\actriv} & = \acectriv\\
  \Cof{\acpair{c_1}{c_2}} & = \acecpair{\Cof{c_1}}{\Cof{c_2}}\\
  \Cof{\acprl{c}} & = \acecprl{\Cof{c}}\\
  \Cof{\acprr{c}} & = \acecprr{\Cof{c}}\\
  \Cof{\aparr{\tau_1}{\tau_2}} & = \aceparr{\Cof{\tau_1}}{\Cof{\tau_2}}\\
  \Cof{\aall{t}{\tau}} & = \aceall{t}{\Cof{\tau}}\\
  \Cof{\arec{t}{\tau}} & = \acerec{t}{\Cof{\tau}}\\
  \Cof{\aprod{\labelset}{\mapschema{\tau}{i}{\labelset}}} & = \aceprod{\labelset}{\mapschemax{\Cofv}{\tau}{i}{\labelset}}\\
  \Cof{\asum{\labelset}{\mapschema{\tau}{i}{\labelset}}} & = \acesum{\labelset}{\mapschemax{\Cofv}{\tau}{i}{\labelset}}\\
  \Cof{\amcon{X}} & = \acemcon{X}
  \end{array}\]
  \item Each expanded expression, $e$, except for the value projection of a module expression that is not of module variable form, maps onto a proto-expression, $\Cof{e}$, as follows:
  \[\arraycolsep=1pt\begin{array}{rl}
  \Cof{x} & = x\\
  \Cof{\aelam{\tau}{x}{e}} & = \acelam{\Cof{\tau}}{x}{\Cof{e}}\\
  \Cof{\aeap{e_1}{e_2}} & = \aceap{\Cof{e_1}}{\Cof{e_2}}\\
  \Cof{\aeclam{\kappa}{u}{e}} & = \aceclam{\Cof{\kappa}}{u}{\Cof{e}}\\
  \Cof{\aecap{e}{c}} & = \acecap{\Cof{e}}{\Cof{c}}\\
  \Cof{\aefold{e}} & = \acefold{\Cof e}\\
  \Cof{\aeunfold{e}} & = \aceunfold{\Cof{e}}\\
  \Cof{\aetpl{\labelset}{\mapschema{e}{i}{\labelset}}} & = \acetpl{\labelset}{\mapschemax{\Cofv}{e}{i}{\labelset}}\\
  \Cof{\aein{\ell}{e}} &= \acein{\ell}{\Cof{e}}\\
  \Cof{\aematchwith{n}{e}{\seqschemaX{r}}} & = \acematchwith{n}{\Cof{e}}{\seqschemaXx{\Cofv}{r}}\\
  \Cof{\amval{X}} & = \acemval{X}
  \end{array}\]
  \item Each expanded rule, $r$, maps onto the proto-rule, $\Cof{r}$, as follows:
  \begin{align*}
  \Cof{\aematchrule{p}{e}} & = \acematchrule{p}{\Cof{e}}
  \end{align*}
  Notice that proto-rules bind expanded patterns, not proto-patterns. This is because proto-rules appear in proto-expressions, which are generated by peTSMs. It would not be sensible for an peTSM to splice a pattern out of a literal body.
  \item Each expanded pattern, $p$, except for the variable patterns, maps onto a proto-pattern, $\Cof{p}$, as follows:
  \begin{align*}
  \Cof{\aewildp} & = \acewildp\\
  \Cof{\aefoldp{p}} & = \acefoldp{\Cof{p}}\\
  \Cof{\aetplp{\labelset}{\mapschema{p}{i}{\labelset}}} & = \acetplp{\labelset}{\mapschemax{\Cofv}{p}{i}{\labelset}}\\
  \Cof{\aeinjp{\ell}{p}} & = \aceinjp{\ell}{\Cof{p}}
  \end{align*}
\end{itemize}

\subsubsection{Parameterized Proto-Expression Encoding and Decoding}
The type abbreviated $\tPProtoExpr$ classifies encodings of \emph{parameterized proto-expressions}. The mapping from parameterized proto-expressions to values of type $\tPProtoExpr$ is defined by the \emph{parameterized proto-expression encoding judgement}, $\encodePCEExp{\pce}{e}$. An inverse mapping is defined by the \emph{parameterized proto-expression decoding judgement}, $\decodePCEExp{e}{\pce}$.

\[\begin{array}{ll}
\textbf{Judgement Form} & \textbf{Description}\\
\encodePCEExp{\pce}{e} & \text{$\pce$ has encoding $e$}\\
\decodePCEExp{e}{\pce} & \text{$e$ has decoding $\pce$}
\end{array}\]

Rather than picking a particular definition of $\tPProtoExpr$ and defining the judgements above inductively against it, we only state the following condition, which establishes an isomorphism between values of type $\tPProtoExpr$ and parameterized proto-expressions.

\begin{condition}[Parameterized Proto-Expression Isomorphism]\label{condition:parameterized-proto-expression-isomorphism} ~
\begin{enumerate}
\item For every $\pce$, we have $\encodePCEExp{\pce}{\ecand}$ for some $\ecand$ such that $\hastypeUC{\ecand}{\tPProtoExpr}$ and $\isvalU{\ecand}$.
\item If $\hastypeUC{\ecand}{\tPProtoExpr}$ and $\isvalU{\ecand}$ then $\decodePCEExp{\ecand}{\pce}$ for some $\pce$.
\item If $\encodePCEExp{\pce}{\ecand}$ then $\decodePCEExp{\ecand}{\pce}$.
\item If $\hastypeUC{\ecand}{\tPProtoExpr}$ and $\isvalU{\ecand}$ and $\decodePCEExp{\ecand}{\pce}$ then $\encodePCEExp{\pce}{\ecand}$.
\item If $\encodePCEExp{\pce}{\ecand}$ and $\encodePCEExp{\pce}{\ecand'}$ then $\ecand=\ecand'$.
\item If $\hastypeUC{\ecand}{\tPProtoExpr}$ and $\isvalU{\ecand}$ and $\decodePCEExp{\ecand}{\pce}$ and $\decodePCEExp{\ecand}{\pce'}$ then $\pce=\pce'$.
\end{enumerate}
\end{condition}

\subsubsection{Parameterized Proto-Pattern Encoding and Decoding}
The type abbreviated $\tPCEPat$ classifies encodings of \emph{parameterized proto-patterns}. The mapping from parameterized proto-patterns to values of type $\tPCEPat$ is defined by the \emph{parameterized proto-pattern encoding judgement}, $\encodePCEPat{\pcp}{p}$. An inverse mapping is defined by the \emph{parameterized proto-expression decoding judgement}, $\decodePCEPat{p}{\pcp}$.

\[\begin{array}{ll}
\textbf{Judgement Form} & \textbf{Description}\\
\encodePCEPat{\pcp}{p} & \text{$\pcp$ has encoding $p$}\\
\decodePCEPat{p}{\pcp} & \text{$p$ has decoding $\pcp$}
\end{array}\]

Again, rather than picking a particular definition of $\tPCEPat$ and defining the judgements above inductively against it, we only state the following condition, which establishes an isomorphism between values of type $\tPCEPat$ and parameterized proto-patterns.

\begin{condition}[Parameterized Proto-Pattern Isomorphism]\label{condition:proto-pattern-isomorphism-P} ~
\begin{enumerate}
\item For every $\pcp$, we have $\encodePCEPat{\pcp}{\ecand}$ for some $\ecand$ such that $\hastypeUC{\ecand}{\tPCEPat}$ and $\isvalU{\ecand}$.
\item If $\hastypeUC{\ecand}{\tPCEPat}$ and $\isvalU{\ecand}$ then $\decodePCEPat{\ecand}{\pcp}$ for some $\pcp$.
\item If $\encodePCEPat{\pcp}{\ecand}$ then $\decodePCEPat{\ecand}{\pcp}$.
\item If $\hastypeUC{\ecand}{\tPCEPat}$ and $\isvalU{\ecand}$ and $\decodePCEPat{\ecand}{\pcp}$ then $\encodePCEPat{\pcp}{\ecand}$.
\item If $\encodePCEPat{\pcp}{\ecand}$ and $\encodePCEPat{\pcp}{\ecand'}$ then $\ecand=\ecand'$.
\item If $\hastypeUC{\ecand}{\tPCEPat}$ and $\isvalU{\ecand}$ and $\decodePCEPat{\ecand}{\pcp}$ and $\decodePCEPat{\ecand}{\pcp'}$ then $\pcp=\pcp'$.
\end{enumerate}
\end{condition}

\subsubsection{Splice Summaries}
The \emph{splice summary} of a proto-expression, $\summaryOf{\ce}$, or proto-pattern, $\summaryOf{\cpv}$, is the finite set of references to spliced kinds, constructors, expressions and patterns that it mentions.

\subsubsection{Segmentations}
A \emph{segment set}, $\psi$, is a finite set of pairs of natural numbers indicating the locations of spliced terms. The \emph{segmentation} of a proto-expression, $\segof{\ce}$, or proto-pattern, $\segof{\cpv}$, is the segment set implied by the splice summary.

The predicate $\segOK{\psi}{b}$ checks that each segment in $\psi$, has non-negative length and is within bounds of $b$, and that the segments in $\psi$ do not overlap.

% \subsubsection{Segmentations}
% A \emph{segmentation}, $\psi$, is a finite set of \emph{segments}. Segments consist of two natural numbers and a sort, i.e. segments are of the form $\segKind{m}{n}$ or $\segCon{m}{n}$ or $\segExp{m}{n}$ or $\segPat{m}{n}$.

% The metafunction $\segof{\ce}$ determines the segmentation of $\ce$ by generating one segment for each reference to a spliced term. More specifically:
% \begin{itemize}
% \item We define $\segof{\cekappa}$ as follows:
% \[\arraycolsep=1pt\begin{array}{rl}
%   \segof{\acekdarr{\cekappa_1}{u}{\cekappa_2}} & = \segof{\cekappa_1} \cup \segof{\cekappa_2}\\
%   \segof{\acekunit} & = \emptyset\\
%   \segof{\acekdbprod{\cekappa_1}{u}{\cekappa_2}} & = \segof{\cekappa_1} \cup \segof{\cekappa_2}\\
%   \segof{\acekty} & = \emptyset\\
%   \segof{\aceksing{\ctau}} & = \segof{\ctau}\\
%   \segof{\acesplicedk{m}{n}} & = \{ \segKind{m}{n} \}
% \end{array}\]
% \item We define $\segof{\cec}$ as follows:
% \[\arraycolsep=1pt\begin{array}{rl}
%   \segof{u} & = \emptyset\\
%   \segof{\acecabs{u}{\cec}} & = \segof{\cec}\\
%   \segof{\acecapp{\cec_1}{\cec_2}} & = \segof{\cec_1} \cup \segof{\cec_2}\\
%   \segof{\acectriv} & = \emptyset\\
%   \segof{\acecpair{\cec_1}{\cec_2}} & = \segof{\cec_1} \cup \segof{\cec_2}\\
%   \segof{\acecprl{\cec}} & = \segof{\cec}\\
%   \segof{\acecprr{\cec}} & = \segof{\cec}\\
%   \segof{\aceparr{\ctau_1}{\ctau_2}} & = \segof{\ctau_1} \cup \segof{\ctau_2}\\
%   \segof{\aceallu{u}{\cekappa}{\ctau}} &= \segof{\cekappa} \cup \segof{\ctau}\\
%   \segof{\acerec{t}{\ctau}} & = \segof{\ctau}\\
%   \segof{\aceprod{\labelset}{\mapschema{\ctau}{i}{\labelset}}} & = \cup_{i \in \labelset} \segof{\ctau_i}\\
%   \segof{\acesum{\labelset}{\mapschema{\ctau}{i}{\labelset}}} & = \cup_{i \in \labelset} \segof{\ctau_i}\\
%   \segof{\acemcon{X}} & = \emptyset\\
%   \segof{\acesplicedc{m}{n}} & = \{ \segCon{m}{n} \}
%   \end{array}\]
% \item We define $\segof{\ce}$ as follows:
% \[\arraycolsep=1pt\begin{array}{rl} 

% \segof{x} & = \emptyset\\
% \segof{\acelam{\ctau}{x}{\ce}} & = \segof{\ctau} \cup \segof{\ce} \\
% \segof{\aceclam{\cekappa}{u}{\ce}} & = \segof{\cekappa} \cup \segof{\ce}\\
% \segof{\acecap{\ce}{\cec}} & = \segof{\cec} \cup \segof{\ce}\\
% \segof{\acefold{\ce}} & = \segof{\ce}\\
% \segof{\aceunfold{\ce}} & = \segof{\ce}\\
% \segof{\acetpl{\labelset}{\mapschema{\ce}{i}{\labelset}}} & = \cup_{i \in \labelset} \segof{\ce_i}\\
% \segof{\acepr{\ell}{\ce}} & = \segof{\ce}\\
% \segof{\acein{\ell}{\ce}} & = \segof{\ce}\\
% \segof{\acematchwith{n}{\ce}{\seqschemaX{\crv}}} & = \segof{\ce} \cup_{1 \leq i \leq n} \segof{\crv_i}\\
% \segof{\acemval{X}} & = \emptyset\\
% \segof{\acesplicede{m}{n}{\ctau}} & = \{ \segExp{m}{n} \} \cup \segof{\ctau}\\
% \end{array}\]
% \item We define $\segof{\crv}$ as follows:
% \[\arraycolsep=1pt\begin{array}{rl} 

% \segof{\acematchrule{p}{\ce}} & = \segof{\ce}
% \end{array}\]
% \end{itemize}

% The metafunction $\segof{\cpv}$ determines the segmentation of $\cpv$ by generating one segment for each reference to a spliced type or pattern:
% \[
% \arraycolsep=1pt\begin{array}{rl}

% \segof{\acewildp} & = \emptyset\\
% \segof{\acefoldp{\cpv}} & = \segof{\cpv}\\
% \segof{\acetplp{\labelset}{\mapschema{\cpv}{i}{\labelset}}} & = \cup_{i \in \labelset} \segof{\cpv_i}\\
% \segof{\aceinjp{\ell}{\cpv}} & = \segof{\cpv}\\
% \segof{\acesplicedp{m}{n}{\ctau}} & = \{ \segPat{m}{n} \} \cup \segof{\ctau}
% \end{array}
% \]

% The predicate $\segOK{\psi}{b}$ checks that each segment in $\psi$, has non-negative length and is within bounds of $b$, and that the segments in $\psi$ do not overlap.

\subsection{Deparameterization}
\begin{minipage}{0.42\textwidth}
\noindent\fbox{$\strut\prepce{\Omega_\text{app}}{\Psi}{\pce}{\ce}{\epsilon}{\rho}{\omega}{\Omega_\text{params}}$}\end{minipage}
\begin{minipage}{0.58\textwidth}
When applying peTSM $\epsilon$ at $\rho$, $\pce$ has deparameterization $\ce$ with parameter substitution $\omega$\end{minipage}
\begin{subequations}\label{rules:prepce}
\begin{equation}\label{rule:prepce-ceexp}
\inferrule{ }{
  \prepce{\Omega_\text{app}}{\Psi, \petsmdefn{a}{\rho}{\eparse}}{\apceexp{\ce}}{\ce}{\adefref{a}}{\rho}{\emptyset}{\emptyset}
}
\end{equation}
\begin{equation}\label{rule:prepce-alltypes}
\inferrule{
  \prepce{\Omega_\text{app}}{\Psi}{\pce}{\ce}{\epsilon}{\aealltypes{t}{\rho}}{\omega}{\Omega}\\
  t \notin \domof{\Omega_\text{app}}
}{
  \prepce{\Omega_\text{app}}{\Psi}{\apcebindtype{t}{\pce}}{\ce}{\aeaptype{\tau}{\epsilon}}{\rho}{\omega, \tau/t}{\Omega, t :: \akty}
}
\end{equation}
\begin{equation}\label{rule:prepce-allmods}
\inferrule{
  \prepce{\Omega_\text{app}}{\Psi}{\pce}{\ce}{\epsilon}{\aeallmods{\sigma}{X}{\rho}}{\omega}{\Omega}\\
  X \notin \domof{\Omega_\text{app}}
}{
  \prepce{\Omega_\text{app}}{\Psi}{\apcebindmod{X}{\pce}}{\ce}{\aeapmod{X'}{\epsilon}}{\rho}{\omega, X'/X}{\Omega, X : \sigma}
}
\end{equation}
\end{subequations}

\noindent\begin{minipage}{0.42\textwidth}
\fbox{$\strut\prepcp{\Omega_\text{app}}{\Phi}{\pcp}{\cpv}{\epsilon}{\rho}{\omega}{\Omega_\text{params}}$}\end{minipage}
\begin{minipage}{0.58\textwidth}
When applying ppTSM $\epsilon$ at $\rho$, $\pcp$ has deparameterization $\cpv$ with parameter substitution $\omega$\end{minipage}
\begin{subequations}\label{rules:prepcp}
\begin{equation}\label{rule:prepcp-cepat}
\inferrule{ }{
  \prepcp{\Omega_\text{app}}{\Phi, \pptsmdefn{a}{\rho}{\eparse}}{\apcepat{\cpv}}{\cpv}{\adefref{a}}{\rho}{\emptyset}{\emptyset}
}
\end{equation}
\begin{equation}\label{rule:prepcp-alltypes}
\inferrule{
  \prepcp{\Omega_\text{app}}{\Phi}{\pcp}{\cpv}{\epsilon}{\aealltypes{t}{\rho}}{\omega}{\Omega}\\
  t \notin \domof{\Omega_\text{app}}
}{
  \prepcp{\Omega_\text{app}}{\Phi}{\apcebindtype{t}{\pcp}}{\cpv}{\aeaptype{\tau}{\epsilon}}{\rho}{\omega, \tau/t}{\Omega, t :: \akty}
}
\end{equation}
\begin{equation}\label{rule:prepcp-allmods}
\inferrule{
  \prepce{\Omega_\text{app}}{\Phi}{\pcp}{\cpv}{\epsilon}{\aeallmods{\sigma}{X}{\rho}}{\omega}{\Omega}\\
  X \notin \domof{\Omega_\text{app}}
}{
  \prepcp{\Omega_\text{app}}{\Phi}{\apcebindmod{X}{\pcp}}{\cpv}{\aeapmod{X'}{\epsilon}}{\rho}{\omega, X'/X}{\Omega, X : \sigma}
}
\end{equation}
% \begin{equation}\label{rule:prepcp-ceexp}
% \inferrule{ }{
%   \prepcp{\Omega_\text{app}}{\Phi, \pptsmdefn{a}{\rho}{\eparse}}{\adefref{a}}{\rho}{\emptyset}{\emptyset}
% }
% \end{equation}
% \begin{equation}\label{rule:prepcp-alltypes}
% \inferrule{
%   \prepcp{\Omega_\text{app}}{\Phi}{\epsilon}{\aealltypes{t}{\rho}}{\omega}{\Omega}\\
%   t \notin \domof{\Omega_\text{app}}
% }{
%   \prepcp{\Omega_\text{app}}{\Phi}{\aeaptype{\tau}{\epsilon}}{\rho}{\omega, \tau/t}{\Omega, t :: \akty}
% }
% \end{equation}
% \begin{equation}\label{rule:prepcp-allmods}
% \inferrule{
%   \prepcp{\Omega_\text{app}}{\Phi}{\epsilon}{\aeallmods{\sigma}{X}{\rho}}{\omega}{\Omega}\\
%   X \notin \domof{\Omega_\text{app}}
% }{
%   \prepcp{\Omega_\text{app}}{\Phi}{\aeapmod{X'}{\epsilon}}{\rho}{\omega, X'/X}{\Omega, X : \sigma}
% }
% \end{equation}
\end{subequations}

\subsection{Proto-Expansion Validation}
\subsubsection{Splicing Scenes}
\emph{Expression splicing scenes}, $\escenev$, are of the form $\esceneP{\Omega}{\uOmega}{\uPsi}{\uPhi}{b}$, \emph{constructor splicing scenes}, $\cscenev$, are of the form $\csceneP{\Omega}{\uOmega}{b}$, and \emph{pattern splicing scenes}, $\pscenev$, are of the form $\psceneP{\uOmega}{\uPhi}{b}$. We write $\csfrom{\escenev}$ for the constructor splicing scene constructed by dropping the TSM contexts from $\escenev$:
\[\csfrom{\esceneP{\OParams}{\uOmega}{\uPsi}{\uPhi}{b}} = \csceneP{\OParams}{\uOmega}{b}\]

\subsubsection{Proto-Kind and Proto-Constructor Validation}
\noindent\fbox{$\strut\cvalidKX{\cekappa}{\kappa}$}~~$\cekappa$ has well-formed expansion $\kappa$
\begin{subequations}\label{rules:cvalidK}
\begin{equation}\label{rule:cvalidK-darr}
\inferrule{
  \cvalidKX{\cekappa_1}{\kappa_1}\\
  \cvalidK{\Omega, u :: \kappa_1}{\cscenev}{\cekappa_2}{\kappa_2}
}{
  \cvalidKX{\acekdarr{\cekappa_1}{u}{\cekappa_2}}{\akdarr{\kappa_1}{u}{\kappa_2}}
}
\end{equation}
\begin{equation}\label{rule:cvalidK-unit}
\inferrule{ }{
  \cvalidKX{\acekunit}{\akunit}
}
\end{equation}
\begin{equation}\label{rule:cvalidK-dprod}
\inferrule{
  \cvalidKX{\cekappa_1}{\kappa_1}\\
  \cvalidK{\Omega, u :: \kappa_1}{\cscenev}{\cekappa_2}{\kappa_2}
}{
  \cvalidKX{\acekdbprod{\cekappa_1}{u}{\cekappa_2}}{\akdbprod{\kappa_1}{u}{\kappa_2}}
}
\end{equation}
\begin{equation}\label{rule:cvalidK-ty}
\inferrule{ }{
  \cvalidKX{\acekty}{\akty}
}
\end{equation}
\begin{equation}\label{rule:cvalidK-sing}
\inferrule{
  \cvalidCX{\ctau}{\tau}{\akty}
}{
  \cvalidKX{\aceksing{\ctau}}{\aksing{\tau}}
}
\end{equation}
\begin{equation}\label{rule:cvalidK-spliced}
\inferrule{
  \parseUKind{\bsubseq{b}{m}{n}}{\ukappa}\\
  \kExpands{\uOmega}{\ukappa}{\kappa}\\\\
  \uOmega=\uOmegaEx{\uD}{\uG}{\uMctx}{\Omega_\text{app}}\\
  \domof{\Omega} \cap \domof{\Omega_\text{app}} = \emptyset
}{
  \cvalidK{\Omega}{\csceneP{\OParams}{\uOmega}{b}}{\acesplicedk{m}{n}}{\kappa}
}
\end{equation}
\end{subequations}

\noindent\fbox{$\strut\cvalidCX{\cec}{c}{\kappa}$}~~$\cec$ has expansion $c$ of kind $\kappa$
\begin{subequations}\label{rules:cvalidC}
\begin{equation}\label{rule:cvalidC-subsume}
\inferrule{
  \cvalidCX{\cec}{c}{\kappa_1}\\
  \ksubX{\kappa_1}{\kappa_2}
}{
  \cvalidCX{\cec}{c}{\kappa_2}
}
\end{equation}
\begin{equation}\label{rule:cvalidC-var}
\inferrule{ }{\cvalidC{\Omega, {u} :: {\kappa}}{\cscenev}{u}{u}{\kappa}}
\end{equation}
\begin{equation}\label{rule:cvalidC-abs}
\inferrule{
  \cvalidC{\Omega, u :: \kappa_1}{\cscenev}{\cec_2}{c_2}{\kappa_2}
}{
  \cvalidCX{\acecabs{u}{\cec_2}}{\acabs{u}{c_2}}{\akdarr{\kappa_1}{u}{\kappa_2}}
}
\end{equation}
\begin{equation}\label{rule:cvalidC-app}
\inferrule{
  \cvalidCX{\cec_1}{c_1}{\akdarr{\kappa_2}{u}{\kappa}}\\
  \cvalidCX{\cec_2}{c_2}{\kappa_2}
}{
  \cvalidCX{\acecapp{\cec_1}{\cec_2}}{\acapp{c_1}{c_2}}{[c_1/u]\kappa}
}
\end{equation}
\begin{equation}\label{rule:cvalidC-unit}
\inferrule{ }{
  \cvalidCX{\acectriv}{\actriv}{\akunit}
}
\end{equation}
\begin{equation}\label{rule:cvalidC-pair}
\inferrule{
  \cvalidCX{\cec_1}{c_1}{\kappa_1}\\
  \cvalidCX{\cec_2}{c_2}{[c_1/u]\kappa_2}
}{
  \cvalidCX{\acecpair{\cec_1}{\cec_2}}{\acpair{c_1}{c_2}}{\akdbprod{\kappa_1}{u}{\kappa_2}}
}
\end{equation}
\begin{equation}\label{rule:cvalidC-prl}
\inferrule{
  \cvalidCX{\cec}{c}{\akdbprod{\kappa_1}{u}{\kappa_2}}
}{
  \cvalidCX{\acecprl{\cec}}{\acprl{c}}{\kappa_1}
}
\end{equation}
\begin{equation}\label{rule:cvalidC-prr}
\inferrule{
  \cvalidCX{\cec}{c}{\akdbprod{\kappa_1}{u}{\kappa_2}}
}{
  \cvalidCX{\acecprr{\cec}}{\acprr{c}}{[\acprl{c}/u]\kappa_2}
}
\end{equation}
\begin{equation}\label{rule:cvalidC-parr}
\inferrule{
  \cvalidCX{\ctau_1}{\tau_1}{\akty}\\
  \cvalidCX{\ctau_2}{\tau_2}{\akty}
}{
  \cvalidCX{\aceparr{\ctau_1}{\ctau_2}}{\aparr{\tau_1}{\tau_2}}{\akty}
}
\end{equation}
\begin{equation}\label{rule:cvalidC-all}
\inferrule{
  \cvalidKX{\cekappa}{\kappa}\\
  \cvalidC{\Omega, u :: \kappa}{\cscenev}{\ctau}{\tau}{\akty}
}{
  \cvalidCX{\aceallu{\cekappa}{u}{\ctau}}{\aallu{\kappa}{u}{\tau}}{\akty}
}
\end{equation}
\begin{equation}\label{rule:cvalidC-rec}
\inferrule{
  \cvalidC{\Omega, t :: \akty}{\cscenev}{\ctau}{\tau}{\akty}
}{
  \cvalidCX{\acerec{t}{\ctau}}{\arec{t}{\tau}}{\akty}
}
\end{equation}
\begin{equation}\label{rule:cvalidC-prod}
\inferrule{
  \{\cvalidCX{\ctau_i}{\tau_i}{\akty}\}_{1 \leq i \leq n}
}{
  \cvalidCX{\aceprod{\labelset}{\mapschema{\ctau}{i}{\labelset}}}{\aprod{\labelset}{\mapschema{\tau}{i}{\labelset}}}{\akty}
}
\end{equation}
\begin{equation}\label{rule:cvalidC-sum}
\inferrule{
  \{\cvalidCX{\ctau_i}{\tau_i}{\akty}\}_{1 \leq i \leq n}
}{
  \cvalidCX{\acesum{\labelset}{\mapschema{\ctau}{i}{\labelset}}}{\asum{\labelset}{\mapschema{\tau}{i}{\labelset}}}{\akty}
}
\end{equation}
\begin{equation}\label{rule:cvalidC-sing}
\inferrule{
  \cvalidCX{\cec}{c}{\akty}
}{
  \cvalidCX{\cec}{c}{\aksing{c}}
}
\end{equation}
\begin{equation}\label{rule:cvalidC-stat}
\inferrule{ }{
  \cvalidC{\Omega, X : {\asignature{\kappa}{u}{\tau}}}{\cscenev}{\acemcon{X}}{\amcon{X}}{\kappa}
}
\end{equation}
\begin{equation}\label{rule:cvalidC-spliced}
\inferrule{
  \cscenev=\csceneP{\OParams}{\uOmega}{b}\\
  \cvalidK{\OParams}{\cscenev}{\cekappa}{\kappa}\\
  \parseUCon{\bsubseq{b}{m}{n}}{\uc}\\
  \cExpands{\uOmega}{\uc}{c}{\kappa}\\\\
  \uOmega=\uOmegaEx{\uD}{\uG}{\uMctx}{\Omega_\text{app}}\\
  \domof{\Omega} \cap \domof{\Omega_\text{app}} = \emptyset
}{
  \cvalidC{\Omega}{\cscenev}{\acesplicedc{m}{n}{\cekappa}}{c}{\kappa}
}
\end{equation}
\end{subequations}
\subsubsection{Proto-Expression and Proto-Rule Validation}
% \begin{equation}
% \inferrule{
%   \ccanaX{\ctau}{\tau}{\akty}
% }{
%   \cvalidTP{\Omega}{\cscenev}{\ctau}{\tau}
% }
% \end{equation}
\noindent\fbox{$\strut\cvalidEPX{\ce}{e}{\tau}$}~~$\ce$ has expansion $e$ of type $\tau$
\begin{subequations}\label{rules:cvalidE-P}
\begin{equation}\label{rule:cvalidE-P-subsume}
  \inferrule{
    \cvalidEPX{\ce}{e}{\tau}\\
    \issubtypePX{\tau}{\tau'}
  }{
    \cvalidEPX{\ce}{e}{\tau'}
  }
\end{equation}
\begin{equation}\label{rule:cvalidE-P-var}
  \inferrule{ }{ 
    \cvalidEP{\Omega, \Ghyp{x}{\tau}}{\escenev}{x}{x}{\tau}
  }
\end{equation}
\begin{equation}\label{rule:cvalidE-P-asc}
\inferrule{
  \cvalidC{\Omega}{\csfrom{\escenev}}{\ctau}{\tau}{\akty}\\
  \cvalidEP{\Omega}{\escenev}{\ce}{e}{\tau}
}{
  \cvalidEP{\Omega}{\escenev}{\aceasc{\ctau}{\ce}}{e}{\tau}
}
\end{equation}
\begin{equation}\label{rule:cvalidE-P-letsyn}
  \inferrule{
    \cvalidEP{\Omega}{\escenev}{\ce_1}{e_1}{\tau_1}\\
    \cvalidEP{\Omega, x : \tau_1}{\ce_2}{e_2}{\tau_2}
  }{
    \cvalidEP{\Omega}{\escenev}{\aceletsyn{x}{\ce_1}{\ce_2}}{
      \aeap{\aelam{\tau_1}{x}{e_2}}{e_1}
    }{\tau_2}
  }
\end{equation}
\begin{equation}\label{rule:cvalidE-P-lam}
  \inferrule{
    \cvalidC{\Omega}{\csfrom{\escenev}}{\ctau_1}{\tau_1}{\akty}\\
    \cvalidEP{\Omega, \Ghyp{x}{\tau_1}}{\escenev}{\ce}{e}{\tau_2}
  }{
    \cvalidEPX{\acelam{\ctau_1}{x}{\ce}}{\aelam{\tau_1}{x}{e}}{\aparr{\tau_1}{\tau_2}}
  }
\end{equation}
\begin{equation}\label{rule:cvalidE-P-ap}
  \inferrule{
    \cvalidEPX{\ce_1}{e_1}{\aparr{\tau_2}{\tau}}\\
    \cvalidEPX{\ce_2}{e_2}{\tau_2}
  }{
    \cvalidEPX{\aceap{\ce_1}{\ce_2}}{\aeap{e_1}{e_2}}{\tau}
  }
\end{equation}
\begin{equation}\label{rule:cvalidE-P-clam}
  \inferrule{
    \cvalidK{\Omega}{\csfrom{\escenev}}{\cekappa}{\kappa}\\
    \cvalidEP{\Omega, u :: \kappa}{\escenev}{\ce}{e}{\tau}
  }{
    \csynX{\aceclam{\cekappa}{u}{\ce}}{\aeclam{\kappa}{u}{e}}{\aallu{\kappa}{u}{\tau}}
  }
\end{equation}
\begin{equation}\label{rule:cvalidE-P-cap}
  \inferrule{
    \cvalidEPX{\ce}{e}{\aallu{\kappa}{u}{\tau}}\\
    \cvalidC{\Omega}{\csfrom{\escenev}}{\cec}{c}{\kappa}
  }{
    \cvalidEPX{\acecap{\ce}{\cec}}{\aecap{e}{c}}{[c/u]\tau}
  }
\end{equation}
\begin{equation}\label{rule:cvalidE-P-fold}
  \inferrule{
    \cvalidEPX{\ce}{e}{[\arec{t}{\tau}/t]\tau}
  }{
    \cvalidEPX{\aceanafold{\ce}}{\aefold{e}}{\arec{t}{\tau}}
  }
\end{equation}
\begin{equation}\label{rule:cvalidE-P-unfold}
  \inferrule{
    \cvalidEPX{\ce}{e}{\arec{t}{\tau}}
  }{
    \cvalidEPX{\aceunfold{\ce}}{\aeunfold{e}}{[\arec{t}{\tau}/t]\tau}
  }
\end{equation}
\begin{equation}\label{rule:cvalidE-P-tpl}
  \inferrule{
    \tau=\aprod{\labelset}{\mapschema{\tau}{i}{\labelset}}\\\\    
    \{\cvalidEPX{\ce_i}{e_i}{\tau_i}\}_{i \in \labelset}
  }{
    \cvalidEPX{\acetpl{\labelset}{\mapschema{\ce}{i}{\labelset}}}{\aetpl{\labelset}{\mapschema{e}{i}{\labelset}}}{\tau}
  }
\end{equation}
\begin{equation}\label{rule:cvalidE-P-pr}
  \inferrule{
    \cvalidEPX{\ce}{e}{\aprod{\labelset, \ell}{\mapschema{\tau}{i}{\labelset}; \mapitem{\ell}{\tau}}}
  }{
    \cvalidEPX{\acepr{\ell}{\ce}}{\aepr{\ell}{e}}{\tau}
  }
\end{equation}
\begin{equation}\label{rule:cvalidE-P-in}
  \inferrule{
    \asum{\labelset, \ell}{\mapschema{\tau}{i}{\labelset}; \mapitem{\ell}{\tau'}}\\\\
    \cvalidEPX{\ce'}{e'}{\tau'}
  }{
    \cvalidEPX{\aceanain{\ell}{\ce'}}{\aein{\ell}{e'}}{\tau}
    %\left(\shortstack{$\Delta~\Gamma \vdash^{\escenev} $\\$\leadsto$\\$ \Leftarrow $\vspace{-1.2em}}\right)
    %\eanaX{\auanain{\ell}{\ue}}{\aein{\ell}}{\asum{\labelset, \ell}{\mapschema{\tau}{i}{\labelset}; \mapitem{\ell}{\tau}}}
  }
\end{equation}
\begin{equation}\label{rule:cvalidE-P-match}
  \inferrule{
    % n > 0\\
    \cvalidEPX{\ce}{e}{\tau}\\
    \{\cvalidRP{\Omega}{\escenev}{\crv_i}{r_i}{\tau}{\tau'}\}_{1 \leq i \leq n}
  }{
    \cvalidEPX{\acematchwithb{n}{\ce}{\seqschemaX{\crv}}}{\aematchwith{n}{e}{\seqschemaX{r}}}{\tau'}
  }
\end{equation}
\begin{equation}\label{rule:cvalidE-P-mval}
\inferrule{ }{
  \cvalidEP{\Omega, X : \asignature{\kappa}{u}{\tau}}{\escenev}{\acemval{X}}{\amval{X}}{[\amcon{X}/u]\tau}
}
\end{equation}
\begin{equation}\label{rule:cvalidE-P-splicede}
\inferrule{
  \escenev = \esceneP{\OParams}{\uOmega}{\uPsi}{\uPhi}{b}\\
  \cvalidC{\OParams}{\csfrom{\escenev}}{\ctau}{\tau}{\akty}\\\\
  \parseUExp{\bsubseq{b}{m}{n}}{\ue}\\
  \expandsP{\uOmega}{\uPsi}{\uPhi}{\ue}{e}{\tau}\\\\
  \uOmega=\uOmegaEx{\uD}{\uG}{\uMctx}{\Omega_\text{app}}\\
  \domof{\Omega} \cap \domof{\Omega_\text{app}} = \emptyset
}{
  \cvalidEP{\Omega}{\escenev}{\acesplicede{m}{n}{\ctau}}{e}{\tau}
}
\end{equation}
\end{subequations}

\noindent\fbox{$\strut\cvalidRP{\Omega}{\escenev}{\crv}{r}{\tau}{\tau'}$}~~$\crv$ has expansion $r$ taking values of type $\tau$ to values of type $\tau'$
\begin{equation}\label{rule:cvalidR-P}
\inferrule{
  \patTypeP{\Omega'}{p}{\tau}\\
  \cvalidEP{\Gcons{\Omega}{\Omega'}}{\escenev}{\ce}{e}{\tau'}
}{
  \cvalidRP{\Omega}{\escenev}{\acematchrule{p}{\ce}}{\aematchrule{p}{e}}{\tau}{\tau'}
}
\end{equation}

\subsubsection{Proto-Pattern Validation}
\noindent\fbox{$\strut\cvalidPPE{\uOmega}{\pscenev}{\cpv}{p}{\tau}$}~~$\cpv$ has expansion $p$ matching against $\tau$ generating hypotheses $\uOmega$
\begin{subequations}\label{rules:cvalidPP}
\begin{equation}\label{rule:cvalidPP-wild}
\inferrule{ }{
  \cvalidPP{\uOmegaEx{\emptyset}{\emptyset}{\emptyset}{\emptyset}}{\pscenev}{\acewildp}{\aewildp}{\tau}
}
\end{equation}
\begin{equation}\label{rule:cvalidPP-fold}
\inferrule{
  \cvalidPP{\uOmega}{\pscenev}{\cpv}{p}{[\arec{t}{\tau}/t]\tau}
}{
  \cvalidPP{\uOmega}{\pscenev}{\acefoldp{\cpv}}{\aefoldp{p}}{\arec{t}{\tau}}
}
\end{equation}
\begin{equation}\label{rule:cvalidPP-tpl}
\inferrule{
  \cpv=\acetplp{\labelset}{\mapschema{\cpv}{i}{\labelset}}\\
  p=\aetplp{\labelset}{\mapschema{p}{i}{\labelset}}\\\\
  \{\cvalidPP{\upctx_i}{\pscenev}{\cpv_i}{p_i}{\tau_i}\}_{i \in \labelset}
}{
  \cvalidPP{\GIconsi{i \in \labelset}{\uOmega_i}}{\pscenev}{\cpv}{p}{\aprod{\labelset}{\mapschema{\tau}{i}{\labelset}}}
  %\cvalidPP{}{\cpv}{p}{}
%\left(\shortstack{$\vdash^{\pscenev} $\\$\leadsto$\\$ :~\dashVx^{\,\Gconsi{i \in \labelset}{\upctx_i}}$\vspace{-1.2em}}\right)
}
\end{equation}
\begin{equation}\label{rule:cvalidPP-in}
\inferrule{
  \cvalidPP{\uOmega}{\pscenev}{\cpv}{p}{\tau}
}{
  \cvalidPP{\uOmega}{\pscenev}{\aceinjp{\ell}{\cpv}}{\aeinjp{\ell}{p}}{\asum{\labelset, \ell}{\mapschema{\tau}{i}{\labelset}; \mapitem{\ell}{\tau}}}
}
\end{equation}
\begin{equation}\label{rule:cvalidPP-spliced}
\inferrule{
  \cvalidC{\OParams}{\csceneP{\OParams}{\uOmega}{b}}{\ctau}{\tau}{\akty}\\
  \parseUPat{\bsubseq{b}{m}{n}}{\upv}\\
  \patExpandsP{\uOmega'}{\uPhi}{\upv}{p}{\tau}
}{
  \cvalidPP{\uOmega'}{\pscene{\uOmega}{\uPhi}{b}}{\acesplicedp{m}{n}{\ctau}}{p}{\tau}
}
\end{equation}
\end{subequations}


\section{Metatheory}\label{appendix:metatheory-P}
\subsection{TSM Expression Evaluation}
\begin{theorem}[peTSM Preservation]
\label{thm:peTSM-preservation}
If $\hastsmtypeExp{\Omega}{\Psi}{\epsilon}{\rho}$ and $\tsmexpStepsExp{\Omega}{\Psi}{\epsilon}{\epsilon'}$ then $\hastsmtypeExp{\Omega}{\Psi}{\epsilon'}{\rho}$.
\end{theorem}
\begin{proof}\todo{proof}\end{proof}

\begin{theorem}[peTSM Preservation (Multistep)]
\label{thm:peTSM-preservation-multistep}
If $\hastsmtypeExp{\Omega}{\Psi}{\epsilon}{\rho}$ and $\tsmexpMultistepsExp{\Omega}{\Psi}{\epsilon}{\epsilon'}$ then $\hastsmtypeExp{\Omega}{\Psi}{\epsilon'}{\rho}$.
\end{theorem}
\begin{proof}\todo{proof}\end{proof}

\begin{theorem}[peTSM Preservation (Evaluation)]
\label{thm:peTSM-preservation-evaluation}
If $\hastsmtypeExp{\Omega}{\Psi}{\epsilon}{\rho}$ and $\tsmexpEvalsExp{\Omega}{\Psi}{\epsilon}{\epsilon'}$ then $\hastsmtypeExp{\Omega}{\Psi}{\epsilon'}{\rho}$.
\end{theorem}
\begin{proof}\todo{proof}\end{proof}

\begin{theorem}[ppTSM Preservation]
\label{thm:ppTSM-preservation}
If $\hastsmtypePat{\Omega}{\Phi}{\epsilon}{\rho}$ and $\tsmexpStepsPat{\Omega}{\Phi}{\epsilon}{\epsilon'}$ then $\hastsmtypePat{\Omega}{\Phi}{\epsilon'}{\rho}$.
\end{theorem}
\begin{proof}\todo{proof}\end{proof}

\begin{theorem}[ppTSM Preservation (Multistep)]
\label{thm:ppTSM-preservation-multistep}
If $\hastsmtypePat{\Omega}{\Phi}{\epsilon}{\rho}$ and $\tsmexpMultistepsPat{\Omega}{\Phi}{\epsilon}{\epsilon'}$ then $\hastsmtypePat{\Omega}{\Phi}{\epsilon'}{\rho}$.
\end{theorem}
\begin{proof}\todo{proof}\end{proof}

\begin{theorem}[ppTSM Preservation (Evaluation)]
\label{thm:ppTSM-preservation-evaluation}
If $\hastsmtypePat{\Omega}{\Phi}{\epsilon}{\rho}$ and $\tsmexpEvalsPat{\Omega}{\Phi}{\epsilon}{\epsilon'}$ then $\hastsmtypePat{\Omega}{\Phi}{\epsilon'}{\rho}$.
\end{theorem}
\begin{proof}\todo{proof}\end{proof}

\begin{theorem}[peTSM Progress]
\label{thm:peTSM-progress}
If $\hastsmtypeExp{\Omega}{\Psi}{\epsilon}{\rho}$ then either $\tsmexpStepsExp{\Omega}{\Psi}{\epsilon}{\epsilon'}$ for some $\epsilon'$ or $\tsmexpNormalExp{\Omega}{\Psi}{\epsilon}$.
\end{theorem}
\begin{proof}\todo{proof}\end{proof}

\begin{theorem}[ppTSM Progress]
\label{thm:ppTSM-progress}
If $\hastsmtypePat{\Omega}{\Phi}{\epsilon}{\rho}$ then either $\tsmexpStepsPat{\Omega}{\Phi}{\epsilon}{\epsilon'}$ for some $\epsilon'$ or $\tsmexpNormalPat{\Omega}{\Phi}{\epsilon}$.
\end{theorem}
\begin{proof}\todo{proof}\end{proof}

\subsection{Typed Expansion}
\subsubsection{Kinds and Constructors}
\begin{theorem}[Kind and Constructor Expansion]
\label{thm:kind-and-constructor-expansion-P}
~
\begin{enumerate}
\item If $\kExpands{\uOmegaEx{\uD}{\uG}{\uMctx}{\Omega}}{\ukappa}{\kappa}$ then $\iskind{\Omega}{\kappa}$.
\item If $\cExpands{\uOmegaEx{\uD}{\uG}{\uMctx}{\Omega}}{\uc}{c}{\kappa}$ then $\haskind{\Omega}{c}{\kappa}$.
\end{enumerate}
\end{theorem}
\begin{proof}\todo{proof}\end{proof}

\subsubsection{TSM Types and Expressions}
\begin{theorem}[TSM Type Expansion]
\label{thm:tsm-type-expansion-P}
If $\tsmtyExpands{\uOmegaEx{\uD}{\uG}{\uMctx}{\Omega}}{\urho}{\rho}$ then $\istsmty{\Omega}{\rho}$.
\end{theorem}
\begin{proof}\todo{proof}\end{proof}

\begin{theorem}[peTSM Expression Expansion]
\label{thm:peTSM-expression-expansion}
If $\tsmexpExpandsExp{\uOmegaEx{\uD}{\uG}{\uMctx}{\Omega}}{\uAS{\uA}{\Psi}}{\uepsilon}{\epsilon}{\rho}$ then $\hastsmtypeExp{\Omega}{\Psi}{\epsilon}{\rho}$.
\end{theorem}
\begin{proof}\todo{proof}\end{proof}

\begin{theorem}[ppTSM Expression Expansion]
\label{thm:ppTSM-expression-expansion}
If $\tsmexpExpandsPat{\uOmegaEx{\uD}{\uG}{\uMctx}{\Omega}}{\uAS{\uA}{\Phi}}{\uepsilon}{\epsilon}{\rho}$ then $\hastsmtypePat{\Omega}{\Phi}{\epsilon}{\rho}$.
\end{theorem}
\begin{proof}\todo{proof}\end{proof}

\subsubsection{Patterns}
\begin{lemma}[Pattern Deparameterization]
\label{lemma:pattern-deparameterization-P}
If $\prepcp{\Omega_\text{app}}{\Phi}{\pcp}{\cpv}{\epsilon}{\rho}{\omega}{\Omega_\text{params}}$ and $\hastsmtypePat{\Omega}{\Phi}{\epsilon}{\rho'}$ then $\domof{\Omega_\text{app}} \cap \domof{\Omega_\text{params}} = \emptyset$ and $\hastypeP{\Omega_\text{app}}{\omega}{\Omega_\text{params}}$.
\end{lemma}
\begin{proof} By rule induction over Rules (\ref{rules:prepcp}).
\begin{byCases}
  \item[\text{(\ref{rule:prepcp-ceexp})}] We have:
    \begin{pfsteps*}
      \item $\omega=\emptyset$ \BY{assumption}
      \item $\Omega_\text{params}=\emptyset$ \BY{assumption}
      \item $\domof{\Omega_\text{app}} \cap \domof{\emptyset} = \emptyset$ \BY{definition}
      \item $\hastypeP{\Omega_\text{app}}{\emptyset}{\emptyset}$ \BY{definition}
    \end{pfsteps*}
    \resetpfcounter
  \item[\text{(\ref{rule:prepcp-alltypes})}] We have:
    \begin{pfsteps*}
      \item $\epsilon=\aeaptype{\tau}{\epsilon'}$ \BY{assumption}
      \item $\prepcp{\Omega_\text{app}}{\Phi}{\pcp}{\cpv}{\epsilon'}{\aealltypes{t}{\rho}}{\omega'}{\Omega'}$ \BY{assumption} \pflabel{prepcp}
      \item $t \notin \domof{\Omega_\text{app}}$ \BY{assumption} \pflabel{notin}
      \item $\omega=\omega', \tau/t$ \BY{assumption}
      \item $\Omega_\text{params} = \Omega', t :: \akty$ \BY{assumption}
      \item $\domof{\Omega_\text{app}} \cap \domof{\Omega'} = \emptyset$ \BY{IH on \pfref{prepcp}} \pflabel{IH1}
      \item $\hastypeP{\Omega_\text{app}}{\omega'}{\Omega'}$ \BY{IH on \pfref{prepcp}} \pflabel{IH2}
      \item $\domof{\Omega_\text{app}} \cap \domof{\Omega', t :: \akty}$ \BY{\pfref{notin} and \pfref{IH2} and definition of finite set intersection}
      \item $\hastypeP{\Omega_\text{app}}{\omega', \tau/t}{\Omega', t :: \akty}$ \BY{\todo{definition of omega type} and \todo{assumption that epsilon is well-typed}}
    \end{pfsteps*}
  \item[\text{(\ref{rule:prepcp-allmods})}] \todo{this case is analagous}
\end{byCases}
\end{proof}

\begin{theorem}[Typed Pattern Expansion]\label{thm:typed-pattern-expansion-P} ~
\begin{enumerate}
  \item If $\pExpandsSP{\uOmegaEx{\uD}{\uG}{\uMctx}{\Omega_\text{app}}}{\uPhi}{\upv}{p}{\tau}{\uOmegaEx{\uD'}{\uG'}{\uMctx'}{\Omega'}}$ then $\uMctx' = \emptyset$ and $\uD' = \emptyset$ and $\patTypePC{\Omega_\text{app}}{\Omega'}{p}{\tau}$.
  \item If $\cvalidPP{\uOmegaEx{\uD'}{\uG'}{\uMctx'}{\Omega'}}{\pscene{\uOmegaEx{\uD}{\uG}{\uMctx}{\Omega_\text{app}}}{\uPhi}{b}}{\cpv}{p}{\tau}$ and $\domof{\Omega_\text{params}} \cap \domof{\Omega_\text{app}} = \emptyset$ then $\uMctx' = \emptyset$ and $\uD' = \emptyset$ and $\patTypePC{\Omega_\text{params} \cup \Omega_\text{app}}{\Omega'}{p}{\tau}$.
\end{enumerate}
\end{theorem}
\begin{proof} My mutual rule induction over Rules (\ref{rules:patExpandsP}) and Rules (\ref{rules:cvalidPP}).
\begin{enumerate}
\item In the following, let $\uOmega = \uOmegaEx{\uD}{\uG}{\uMctx}{\Omega_\text{app}}$ and $\uOmega' = \uOmegaEx{\uD'}{\uG'}{\uMctx'}{\Omega'}$.
  \begin{byCases}
    \item[\text{(\ref{rule:patExpandsP-subsume}) \textbf{through} (\ref{rule:patExpandsP-in})}] These cases follow by applying the IH, part 1 and applying the corresponding pattern typing rule in Rules (\ref{rules:patTypeP}).

    \item[\text{(\ref{rule:patExpandsP-apuptsm})}] We have:
    \begin{pfsteps*}
    \item $\upv=\utsmap{\uepsilon}{b}$ \BY{assumption}
    \item $\uPhi=\uAS{\uA}{\Phi}$ \BY{assumption}
    \item $\tsmexpExpandsPat{\uOmega}{\uPhi}{\uepsilon}{\epsilon}{\aetype{\tau_\text{final}}}$ \BY{assumption}
    \item $\tsmexpEvalsPat{\Omega_\text{app}}{\Phi}{\epsilon}{\epsilon_\text{normal}}$ \BY{assumption}
    \item $\tsmdefof{\epsilon_\text{normal}}=a$ \BY{assumption}
    \item $\Phi = \Phi', \pptsmdefn{a}{\rho}{\eparse}$ \BY{assumption}
    \item $\encodeBody{b}{\ebody}$ \BY{assumption}
    \item $\evalU{\ap{\eparse}{\ebody}}{{\lbltxt{SuccessP}}\cdot{\ecand}}$ \BY{assumption}
    \item $\decodeCEPat{\ecand}{\cpv}$ \BY{assumption}
    \item $\prepcp{\Omega_\text{app}}{\Phi}{\pcp}{\cpv}{\epsilon}{\aetype{\tau_\text{proto}}}{\omega}{\Omega_\text{params}}$ \BY{assumption}
    \item $\cvalidPP{\uOmega'}{\psceneP{\uOmega}{\uPhi}{b}}{\cpv}{p}{\tau_\text{proto}}$ \BY{assumption}
    \item $\tau = [\omega]\tau_\text{proto}$ \BY{assumption}
    \item $\domof{\Omega_\text{params}} \cap \domof{\Omega_\text{app}} = \emptyset$ \BY{\todo{lemma}}
    \item $\hastypeP{\Omega_\text{app}}{\omega}{\Omega_\text{params}}$ \BY{\todo{lemma}}
    \item $\uMctx' = \emptyset$ \BY{\todo{IH part 2}}
    \item $\uD' = \emptyset$
    \item $\patTypePC{\Omega_\text{params} \cup \Omega_\text{app}}{\Omega'}{p}{\tau_\text{proto}}$
    \item $\patTypePC{\Omega_\text{app}}{\Omega'}{p}{[\omega]\tau_\text{proto}}$ \BY{\todo{substitution}}
    \end{pfsteps*}

    % \item[\text{(\ref{rule:patExpandsP-subsume})}] We have:
    %   \begin{pfsteps*}
    %   \item $\patExpandsP{\uOmega'}{\uPhi}{\upv}{p}{\tau'}$ \BY{assumption} \pflabel{patExpandsP}
    %   \item $\issubtypeP{\Omega}{\tau'}{\tau}$ \BY{assumption} \pflabel{issubtypeP}
    %   \item $\uMctx' = \emptyset$  \BY{IH, part 1 on \pfref{patExpandsP}}
    %   \item $\uD' = \emptyset$   \BY{IH, part 1 on \pfref{patExpandsP}}
    %   \item $\patTypeP{\Omega'}{p}{\tau'}$  \BY{IH, part 1 on \pfref{patExpandsP}}
    %   \item $\patTypeP{\Omega'}{p}{\tau}$ \BY{Rule (\ref{rule:patTypeP-subsume}) on \pfref{patExpandsP} and \pfref{issubtypeP}}
    %   \end{pfsteps*}
    %   \resetpfcounter
    % \item[\text{(\ref{rule:patExpandsP-var})}] We have:
    %   \begin{pfsteps*}
    %   \item $\upv=x$ \BY{assumption}
    %   \item $p=x$ \BY{assumption}
    %   \item $\uMctx' = \emptyset$ \BY{assumption}
    %   \item $\uD' = \emptyset$ \BY{assumption}
    %   \item $\Omega' = x : \tau$ \BY{assumption}
    %   \item $\patTypeP{\Omega'}{x}{\tau}$ \BY{Rule (\ref{rule:patTypeP-var})}
    %   \end{pfsteps*}
    %    \resetpfcounter
    % \item[\text{(\ref{rule:patExpandsP-wild})}] We have:
    %   \begin{pfsteps*}
    %   \item $\upv=\wildp$ \BY{assumption}
    %   \item $p = \aewildp$ \BY{assumption}
    %   \item $\uMctx' = \emptyset$ \BY{assumption}
    %   \item $\uD' = \emptyset$ \BY{assumption}
    %   \item $\Omega' = \emptyset$ \BY{assumption}
    %   \item $\patTypeP{\Omega'}{\aewildp}{\tau}$ \BY{Rule (\ref{rule:patTypeP-wild})}
    %   \end{pfsteps*}
    %   \resetpfcounter
    % \item[\text{(\ref{rule:patExpandsP-fold})}] We have:
    %   \begin{pfsteps*}
    %   \item $\upv = \foldp{\upv'}$ \BY{assumption}
    %   \item $p = \aefoldp{p'}$ \BY{assumption}
    %   \item $\tau=\arec{t}{\tau'}$ \BY{assumption}
    %   \item $\patExpandsP{\uOmega'}{\uPhi}{\upv'}{p'}{[\arec{t}{\tau'}/t]\tau'}$ \BY{assumption}\pflabel{patExpandsP}
    %   \item $\uMctx' = \emptyset$ \BY{IH, part 1 on \pfref{patExpandsP}}
    %   \item $\uD' = \emptyset$ \BY{IH, part 1 on \pfref{patExpandsP}}
    %   \item $\patTypeP{\Omega'}{p'}{[\arec{t}{\tau'}/t]\tau'}$ \BY{IH, part 1 on \pfref{patExpandsP}} \pflabel{patTypeP}
    %   \item $\patTypeP{\Omega'}{\aefoldp{p'}}{\arec{t}{\tau'}}$ \BY{Rule (\ref{rule:patTypeP-fold}) on \pfref{patTypeP}}
    %   \end{pfsteps*}
    %   \resetpfcounter
  \end{byCases}
\item We induct on the premise. In the following, let $\upctx=\uGG{\uG}{\pctx}$ and $\uPhi=\uASI{\uA}{\Phi}{\uI}$.
  \begin{byCases}
    \item[\text{(\ref{rule:cvalidP-B-wild}) through (\ref{rule:cvalidP-B-spliced})}] In each case, the proof is written identically to the proof of the corresponding case in the proof of Theorem \ref{thm:typed-pattern-expansion}.
  \end{byCases}
\end{enumerate}
The mutual induction can be shown to be well-founded by showing that the following numeric metric on the judgements that we induct on is decreasing:
\begin{align*}
\sizeof{\patExpands{\upctx}{\uPhi}{\upv}{p}{\tau}} & = \sizeof{\upv}\\
\sizeof{{\cvalidP{\upctx}{\pscene{\uDelta}{\uPhi}{b}}{\cpv}{p}{\tau}}} & = \sizeof{b}
\end{align*}
where $\sizeof{b}$ is the length of $b$ and $\sizeof{\upv}$ is the sum of the lengths of the literal bodies in $\upv$,
\begin{align*}
\sizeof{\ux} & = 0\\
\sizeof{\aufoldp{\upv}} & = \sizeof{\upv}\\
\sizeof{\autplp{\labelset}{\mapschema{\upv}{i}{\labelset}}} & = \sum_{i \in \labelset} \sizeof{\upv_i}\\
\sizeof{\auinjp{\ell}{\upv}} & = \sizeof{\upv}\\
\sizeof{\auapuptsm{b}{\tsmv}} & = \sizeof{b}\\
\sizeof{\auplit{b}} & = \sizeof{b}
\end{align*}

The only case in the proof of part 1 that invokes part 2 are Case (\ref{rule:patExpands-B-apuptsm}) and (\ref{rule:patExpands-B-lit}). There, we have that the metric remains stable: \begin{align*}
 & \sizeof{\patExpands{\upctx}{\uPhi, \uShyp{\tsmv}{a}{\tau}{\eparse}}{\auapuptsm{b}{\tsmv}}{p}{\tau}}\\
=& \sizeof{\patExpands{\upctx}{\uASI{\uA}{\Phi', \xuptsmbnd{a}{\tau}{\eparse}}{\uI', \designate{\tau}{a}}}{\auplit{b}}{p}{\tau}}\\
=& \sizeof{{\cvalidP{\upctx}{\pscene{\uDelta}{\uPhi, \uShyp{\tsmv}{a}{\tau}{\eparse}}{b}}{\cpv}{p}{\tau}}}\\
=&\sizeof{b}\end{align*}

The only case in the proof of part 2 that invokes part 1 is Case (\ref{rule:cvalidP-B-spliced}). There, we have that $\parseUPat{\bsubseq{b}{m}{n}}{\upv}$ and the IH is applied to the judgement $\patExpands{\upctx}{\uPhi}{\upv}{p}{\tau}$. Because the metric is stable when passing from part 1 to part 2, we must have that it is strictly decreasing in the other direction:
\[\sizeof{\patExpands{\upctx}{\uPhi}{\upv}{p}{\tau}} < \sizeof{{\cvalidP{\upctx}{\pscene{\uDelta}{\uPhi}{b}}{\acesplicedp{m}{n}{\ctau}}{p}{\tau}}}\]
i.e. by the definitions above, 
\[\sizeof{\upv} < \sizeof{b}\]

This is established by appeal to Condition \ref{condition:body-subsequences}, which states that subsequences of $b$ are no longer than $b$, and the following condition, which states that an unexpanded pattern constructed by parsing a textual sequence $b$ is strictly smaller, as measured by the metric defined above, than the length of $b$, because some characters must necessarily be used to delimit each literal body.
% \begin{condition}[Pattern Parsing Monotonicity]\label{condition:pattern-parsing-B} If $\parseUPat{b}{\upv}$ then $\sizeof{\upv} < \sizeof{b}$.\end{condition}

Combining Conditions \ref{condition:body-subsequences} and \ref{condition:pattern-parsing-B}, we have that $\sizeof{\ue} < \sizeof{b}$ as needed.
\end{proof}
\subsubsection{Expressions and Rules}
\begin{theorem}[Typed Expression and Rule Expansion]
\label{thm:typed-expression-expansion-P}
~
\begin{enumerate}
\item \begin{enumerate}
  \item If $\expandsP{\uOmegaEx{\uD}{\uG}{\uMctx}{\Omega}}{\uPsi}{\uPhi}{\ue}{e}{\tau}$ then $\hastypeP{\Omega}{e}{\tau}$.
  \item If $\rExpandsSP{\uOmegaEx{\uD}{\uG}{\uMctx}{\Omega}}{\uPsi}{\uPhi}{\urv}{r}{\tau}{\tau'}$ then $\ruleTypeP{\Omega}{r}{\tau}{\tau'}$.
  \end{enumerate}
\item \begin{enumerate}
  \item If $\cvalidEP{\Omega_\text{params}}{\esceneP{\OParams}{\uOmegaEx{\uD}{\uG}{\uMctx}{\Omega_\text{app}}}{\uPsi}{\uPhi}{b}}{\ce}{e}{\tau}$ and $\domof{\Omega_\text{params}} \cap \domof{\Omega_\text{app}} = \emptyset$ then $\hastypeP{\Omega_\text{params} \cup \Omega_\text{app}}{e}{\tau}$.
  \item If $\cvalidRP{\Omega_\text{params}}{\esceneP{\OParams}{\uOmegaEx{\uD}{\uG}{\uMctx}{\Omega_\text{app}}}{\uPsi}{\uPhi}{b}}{\crv}{r}{\tau}{\tau'}$ and $\domof{\Omega_\text{params}} \cap \domof{\Omega_\text{app}} = \emptyset$ then $\ruleTypeP{\Omega_\text{params} \cup \Omega_\text{app}}{r}{\tau}{\tau'}$.
  \end{enumerate}
\end{enumerate}
\end{theorem}
\begin{proof} \todo{proof} \end{proof}

\subsubsection{Signatures and Modules}
\begin{theorem}[Signature Expansion]
\label{thm:signature-expansion-P}
If $\sigExpandsP{\uOmegaEx{\uD}{\uG}{\uMctx}{\Omega}}{\usigma}{\sigma}$ then $\issig{\Omega}{\sigma}$.
\end{theorem}
\begin{proof} \todo{proof} \end{proof}

\begin{theorem}[Module Expansion]
\label{thm:module-expansion-P}
If $\mExpandsP{\uOmegaEx{\uD}{\uG}{\uMctx}{\Omega}}{\uPsi}{\uPhi}{\uM}{M}{\sigma}$ then $\hassig{\Omega}{M}{\sigma}$.
\end{theorem}
\begin{proof} \todo{proof} \end{proof}

\subsection{Reasoning Principles}

\begin{theorem}[peTSM Segmentation, Context Independence and Typing]
\label{thm:petsm-reasoning-principles}
If $\expandsP{\uOmega}{\uPsi}{\uPhi}{\utsmap{\uepsilon}{b}}{e}{\tau}$ then:
\begin{enumerate}
  \item $\uOmega = \uOmegaEx{\uD}{\uG}{\uMctx}{\Omega_\text{app}}$
  \item $\uPsi=\uAS{\uA}{\Psi}$
  \item $\tsmexpExpandsExp{\uOmega}{\uPsi}{\uepsilon}{\epsilon}{\aetype{\tau_\text{final}}}$
  \item $\tsmexpEvalsExp{\Omega_\text{app}}{\Psi}{\epsilon}{\epsilon_\text{normal}}$
  \item $\tsmdefof{\epsilon_\text{normal}}=a$
  \item $\Psi = \Psi', \petsmdefn{a}{\rho}{\eparse}$
  \item $\encodeBody{b}{\ebody}$
  \item $\evalU{\ap{\eparse}{\ebody}}{{\lbltxt{SuccessE}}\cdot{e_\text{pproto}}}$
  \item $\decodePCEExp{e_\text{pproto}}{\pce}$
  \item $\prepce{\Omega_\text{app}}{\Psi}{\pce}{\ce}{\epsilon_\text{normal}}{\aetype{\tau_\text{proto}}}{\omega}{\Omega_\text{params}}$
  \item (\textbf{Segmentation}) $\segOK{\segof{\ce}}{b}$
  \item $\cvalidEP{\Omega_\text{params}}{\esceneP{\OParams}{\uOmega}{\uPsi}{\uPhi}{b}}{\ce}{e'}{\tau_\text{proto}}$
  \item $e = [\omega]e'$
  \item $\tau = [\omega]\tau_\text{proto}$
  \item (\textbf{Context Independence}) $\domof{\Omega_\text{app}} \cap \domof{\Omega_\text{params}} = \emptyset$
  \item (\textbf{Typing}) $\tau_\text{final} = [\omega]\tau_\text{proto}$
\end{enumerate}
\end{theorem}
\begin{proof} \todo{proof} \end{proof}

Similarly, ppTSM application is guaranteed to produce a segmentation of the literal body and respect the type annotation on the ppTSM definition.
\begin{theorem}[ppTSM Segmentation and Typing]
\label{thm:pptsm-reasoning-principles}
If $\patExpandsP{\uOmega'}{\uPhi}{\utsmap{\uepsilon}{b}}{p}{\tau}$ then:
\begin{enumerate}
  \item $\uOmega=\uOmegaEx{\uD}{\uG}{\uMctx}{\Omega_\text{app}}$
  \item $\uPhi=\uAS{\uA}{\Phi}$
  \item $\tsmexpExpandsPat{\uOmega}{\uPhi}{\uepsilon}{\epsilon}{\aetype{\tau_\text{final}}}$
  \item $\tsmexpEvalsPat{\Omega_\text{app}}{\Phi}{\epsilon}{\epsilon_\text{normal}}$
  \item $\tsmdefof{\epsilon_\text{normal}}=a$
  \item $\Phi = \Phi', \pptsmdefn{a}{\rho}{\eparse}$
  \item $\encodeBody{b}{\ebody}$
  \item $\evalU{\ap{\eparse}{\ebody}}{{\lbltxt{SuccessP}}\cdot{\ecand}}$
  \item $\decodeCEPat{\ecand}{\cpv}$
  \item $\prepcp{\Omega_\text{app}}{\Phi}{\pcp}{\cpv}{\epsilon_\text{normal}}{\aetype{\tau_\text{proto}}}{\omega}{\Omega_\text{params}}$
  \item (\textbf{Segmentation}) $\segOK{\segof{\cpv}}{b}$
  \item $\cvalidPP{\uOmega'}{\psceneP{\uOmega}{\uPhi}{b}}{\cpv}{p}{\tau_\text{proto}}$
  \item $\tau = [\omega]\tau_\text{proto}$
  \item (\textbf{Typing}) $\tau_\text{final} = [\omega]\tau_\text{proto}$
\end{enumerate}
\end{theorem}
\begin{proof} \todo{proof} \end{proof}

Spliced terms have access only to the bindings at the application site.
\begingroup
\begin{theorem}[peTSM Shadowing Prohibition]
\label{thm:petsm-shadowing-prohibition}
~
\begin{enumerate}
\item If $\cvalidK{\Omega}{\csceneP{\OParams}{\uOmegaEx{\uD}{\uG}{\uMctx}{\Omega_\text{app}}}{b}}{\acesplicedk{m}{n}}{\kappa}$ then:
  \begin{enumerate}
    \item $\parseUKind{\bsubseq{b}{m}{n}}{\ukappa}$
    \item $\kExpands{\uOmega}{\ukappa}{\kappa}$
    \item $\domof{\Omega} \cap \domof{\Omega_\text{app}} = \emptyset$
  \end{enumerate}
\item If $\cvalidC{\Omega}{\csceneP{\OParams}{\uOmegaEx{\uD}{\uG}{\uMctx}{\Omega_\text{app}}}{b}}{\acesplicedc{m}{n}{\cekappa}}{c}{\kappa}$ then:
  \begin{enumerate}
    \item $\parseUCon{\bsubseq{b}{m}{n}}{\uc}$
    \item $\cExpands{\uOmega}{\uc}{c}{\kappa}$
    \item $\domof{\Omega} \cap \domof{\Omega_\text{app}} = \emptyset$
  \end{enumerate}
\item If $\cvalidEP{\Omega}{\esceneP{\OParams}{\uOmegaEx{\uD}{\uG}{\uMctx}{\Omega_\text{app}}}{\uPsi}{\uPhi}{b}}{\acesplicede{m}{n}{\ctau}}{e}{\tau}$ then:\todo{revise this}
  \begin{enumerate}
    \item $\parseUExp{\bsubseq{b}{m}{n}}{\ue}$
    \item $\expandsP{\uOmega}{\uPsi}{\uPhi}{\ue}{e}{\tau}$
    \item $\domof{\Omega} \cap \domof{\Omega_\text{app}} = \emptyset$
  \end{enumerate}
\end{enumerate}
\end{theorem}
\begin{proof} \todo{proof} \end{proof}

\chapter{Bidirectional $\miniVersePat$}\label{appendix:simple-implicits}

\section{Expanded Language (XL)}
The Bidirectional $\miniVersePat$ expanded language (XL) is the same as the  $\miniVersePat$ XL, which was detailed in Appendix \ref{appendix:SES-XL}. %It consists of types, $\tau$, expanded expressions, $e$, expanded rules, $r$, and expanded patterns, $p$.

\section{Unexpanded Language (UL)}
\subsection{Syntax}
\subsubsection{Stylized Syntax}
The stylized syntax extends the stylized syntax of the $\miniVersePat$ UL given in Sec. \ref{appendix:SES-syntax}.

\[\begin{array}{lllllll}
\textbf{Sort} & & 
%& \textbf{Operational Form} 
& \textbf{Stylized Form} & \textbf{Description}\\
\mathsf{UTyp} & \utau & ::= 
%& \cdots 
& \cdots & \text{(as in $\miniVersePat$)}\\
\mathsf{UExp} & \ue & ::= 
%& \cdots 
& \cdots & \text{(as in $\miniVersePat$)}\\
% &&
%& \auasc{\utau}{\ue} 
% & \asc{\ue}{\utau} & \text{ascription}\\
% &&
%& \auletsyn{\ux}{\ue}{\ue} 
% & \letsyn{\ux}{\ue}{\ue} & \text{value binding}\\
&&& \implicite{\tsmv}{\ue} & \text{seTSM designation}\\
&&& \implicitp{\tsmv}{\ue} & \text{spTSM designation}\\
&&& \lit{b} & \text{seTSM unadorned literal}\\
% &&& \auanalam{\ux}{\ue} & \analam{\ux}{\ue} & \text{abstraction (unannotated)}\\
% &&& \aulam{\utau}{\ux}{\ue} & \lam{\ux}{\utau}{\ue} & \text{abstraction (annotated)}\\
% &&& \auap{\ue}{\ue} & \ap{\ue}{\ue} & \text{application}\\
% &&& \autlam{\ut}{\ue} & \Lam{\ut}{\ue} & \text{type abstraction}\\
% &&& \autap{\ue}{\utau} & \App{\ue}{\utau} & \text{type application}\\
% &&& \auanafold{\ue} & \fold{\ue} & \text{fold}\\
% &&& \auunfold{\ue} & \unfold{\ue} & \text{unfold}\\
% &&& \autpl{\labelset}{\mapschema{\ue}{i}{\labelset}} & \tpl{\mapschema{\ue}{i}{\labelset}} & \text{labeled tuple}\\
% &&& \aupr{\ell}{\ue} & \prj{\ue}{\ell} & \text{projection}\\
% &&& \auanain{\ell}{\ue} & \inj{\ell}{\ue} & \text{injection}\\
% &&& \aumatchwithb{n}{\ue}{\seqschemaX{\urv}} & \matchwith{\ue}{\seqschemaX{\urv}} & \text{match}\\
% &&& \audefuetsm{\utau}{e}{\tsmv}{\ue} & \texttt{syntax}~\tsmv~\texttt{at}~\utau~\texttt{for} & \text{ueTSM definition}\\
% &&&                                    & \texttt{expressions}~\{e\}~\texttt{in}~\ue\\
% \LCC &&& \lightgray & \lightgray & \lightgray \\
% &&& \auimplicite{\tsmv}{\ue} & \texttt{implicit\,syntax}~\tsmv~\texttt{for} & \text{ueTSM designation}\\
% &&&                          & \texttt{expressions\,in}~\ue\\ \ECC
% &&& \autsmap{b}{\tsmv} & \utsmap{\tsmv}{b} & \text{ueTSM application}\\%\ECC
% \LCC &&& \lightgray & \lightgray & \lightgray \\
% &&& \auelit{b} & {\lit{b}}  & \text{ueTSM unadorned literal}\\\ECC
% &&& \audefuptsm{\utau}{e}{\tsmv}{\ue} & \texttt{syntax}~\tsmv~\texttt{at}~\utau~\texttt{for} & \text{upTSM definition}\\
% &&&                                    & \texttt{patterns}~\{e\}~\texttt{in}~\ue\\
% \LCC &&& \lightgray & \lightgray & \lightgray \\
% &&& \auimplicitp{\tsmv}{\ue} & \texttt{implicit\,syntax}~\tsmv~\texttt{for} & \text{upTSM designation}\\
% &&&                          & \texttt{patterns\,in}~\ue\\ \ECC
\mathsf{URule} & \urv & ::= 
%& \aumatchrule{\upv}{\ue} 
& \cdots & \text{(as in $\miniVersePat$)}\\
\mathsf{UPat} & \upv & ::= 
%& \ux 
& \cdots & \text{(as in $\miniVersePat$)}\\
&&& \lit{b} & \text{spTSM unadorned literal}
% &&& \auwildp & \wildp & \text{wildcard pattern}\\
% &&& \aufoldp{\upv} & \foldp{\upv} & \text{fold pattern}\\
% &&& \autplp{\labelset}{\mapschema{\upv}{i}{\labelset}} & \tplp{\mapschema{\upv}{i}{\labelset}} & \text{labeled tuple pattern}\\
% &&& \auinjp{\ell}{\upv} & \injp{\ell}{\upv} & \text{injection pattern}\\
% &&& \auapuptsm{b}{\tsmv} & \utsmap{\tsmv}{b} & \text{upTSM application}\\
% \LCC &&& \lightgray & \lightgray & \lightgray\\
% &&& \auplit{b} & \lit{b} & \text{upTSM unadorned literal}\ECC
\end{array}\]
\subsubsection{Body Lengths}
We write $\sizeof{b}$ for the length of $b$. The metafunction $\sizeof{\ue}$ computes the sum of the lengths of expression literal bodies in $\ue$. It is defined by extending the definition given in Sec. \ref{appendix:SES-syntax} with the following additional cases:
\[
\begin{array}{ll}
% \sizeof{\asc{\ue}{\utau}} & = \sizeof{\ue}\\
% \sizeof{\letsyn{\ux}{\ue_1}{\ue_2}} & = \sizeof{\ue_1} + \sizeof{\ue_2}\\
\sizeof{\implicite{\tsmv}{\ue}} & = \sizeof{\ue}\\
\sizeof{\implicitp{\tsmv}{\ue}} & = \sizeof{\ue}\\
\sizeof{\lit{b}} & = \sizeof{b}
\end{array}
\]

Similarly, the metafunction $\sizeof{\upv}$ computes the sum of the lengths of the pattern literal bodies in $\upv$. It is defined by extending the definition given in Sec. \ref{appendix:SES-syntax} with the following additional case:
\begin{align*}
\sizeof{\lit{b}} & = \sizeof{b}
\end{align*}

\subsubsection{Textual Syntax}
In addition to the stylized syntax, there is also a context-free textual syntax for the UL. We need only posit the existence of the following partial metafunctions.

\begin{condition}[Textual Representability]\label{condition:textual-representability-BS} ~
\begin{enumerate}
\item For each $\utau$, there exists $b$ such that $\parseUTyp{b}{\utau}$. 
\item For each $\ue$, there exists $b$ such that $\parseUExp{b}{\ue}$.
% \item For each $\urv$, there exists $b$ such that $\parseURule{b}{\urv}$.
\item {For each $\upv$, there exists $b$ such that $\parseUPat{b}{\upv}$.}
\end{enumerate}
\end{condition}

We also impose the following technical conditions.

\begin{condition}[Expression Parsing Monotonicity]\label{condition:body-parsing-BS} If $\parseUExp{b}{\ue}$ then $\sizeof{\ue} < \sizeof{b}$.\end{condition}

\begin{condition}[Pattern Parsing Monotonicity]\label{condition:pattern-parsing-BS} If $\parseUPat{b}{\upv}$ then $\sizeof{\upv} < \sizeof{b}$.\end{condition}

\subsection{Bidirectionally Typed Expansion}
\subsubsection{Contexts}
\emph{Unexpanded type formation contexts}, $\uDelta$, and \emph{unexpanded typing contexts}, $\uGamma$, were defined in Sec. \ref{appendix:typed-expression-expansion-S}.

\subsubsection{Body Encoding and Decoding}
The type $\tBody$ and the judgements $\encodeBody{b}{e}$ and $\decodeBody{e}{b}$ are characterized in Sec. \ref{appendix:typed-expression-expansion-S}.

\subsubsection{Parse Results}
The types $\tParseResultExp$ and $\tParseResultPat$ are defined as in Sec. \ref{appendix:typed-expression-expansion-S}.

\subsubsection{TSM Contexts}

\emph{seTSM contexts}, $\uPsi$, are of the form $\uASI{\uA}{\Psi}{\uI}$, where $\uA$ is a \emph{TSM identifier expansion context}, $\Psi$ is a \emph{seTSM definition context} and $\uI$ is a \emph{TSM implicit designation context}.

\emph{spTSM contexts}, $\uPhi$, are of the form $\uASI{\uA}{\Phi}{\uI}$, where $\uA$ is a {TSM identifier expansion context}, defined above, and $\Phi$ is a \emph{spTSM definition context}. 

A \emph{TSM identifier expansion context}, $\uA$, is a finite function mapping each TSM identifier $\tsmv \in \domof{\uA}$ to the \emph{TSM identifier expansion}, $\vExpands{\tsmv}{a}$, for some \emph{TSM name}, $a$. We write $\ctxUpdate{\uA}{\tsmv}{a}$ for the TSM identifier expansion context that maps $\tsmv$ to $\vExpands{\tsmv}{a}$, and defers to $\uA$ for all other TSM identifiers (i.e. the previous mapping is \emph{updated}.)

An \emph{seTSM definition context}, $\Psi$, is a finite function mapping each TSM name $a \in \domof{\Psi}$ to an \emph{expanded seTSM definition}, $\xuetsmbnd{a}{\tau}{\eparse}$, where $\tau$ is the seTSM's type annotation, and $\eparse$ is its parse function. We write $\Psi, \xuetsmbnd{a}{\tau}{\eparse}$ when $a \notin \domof{\Psi}$ for the extension of $\Psi$ that maps $a$ to $\xuetsmbnd{a}{\tau}{\eparse}$. We write $\uetsmenv{\Delta}{\Psi}$  when all the type annotations in $\Psi$ are well-formed assuming $\Delta$, and the parse functions in $\Psi$ are closed and of the appropriate type.

\begin{definition}[seTSM Definition Context Formation]\label{def:seTSM-def-ctx-formation} $\uetsmenv{\Delta}{\Psi}$ iff for each $\xuetsmbnd{a}{\tau}{\eparse} \in \Psi$, we have $\istypeU{\Delta}{\tau}$ and $\hastypeU{\emptyset}{\emptyset}{\eparse}{\aparr{\tBody}{\tParseResultExp}}$.\end{definition}

An \emph{spTSM definition context}, $\Phi$, is a finite function mapping each TSM name $a \in \domof{\Phi}$ to an \emph{expanded seTSM definition}, $\xuptsmbnd{a}{\tau}{\eparse}$, where $\tau$ is the spTSM's type annotation, and $\eparse$ is its parse function. We write $\Phi, \xuptsmbnd{a}{\tau}{\eparse}$ when $a \notin \domof{\Phi}$ for the extension of $\Phi$ that maps $a$ to $\xuptsmbnd{a}{\tau}{\eparse}$. We write $\uptsmenv{\Delta}{\Phi}$  when all the type annotations in $\Phi$ are well-formed assuming $\Delta$, and the parse functions in $\Phi$ are closed and of the appropriate type.

\begin{definition}[spTSM Definition Context Formation]\label{def:spTSM-def-ctx-formation} $\uptsmenv{\Delta}{\Phi}$ iff for each $\xuptsmbnd{a}{\tau}{\eparse} \in \Phi$, we have $\istypeU{\Delta}{\tau}$ and $\hastypeU{\emptyset}{\emptyset}{\eparse}{\aparr{\tBody}{\tParseResultPat}}$.\end{definition}

A \emph{TSM implicit designation context}, $\uI$, is a finite function that maps each type $\tau \in \domof{\uI}$ to the \emph{TSM designation} $\designate{\tau}{a}$, for some TSM name $a$. We write $\uI \uplus \designate{\tau}{a}$ for the TSM implicit designation context that maps $\tau$ to $\designate{\tau}{a}$ and defers to $\uI$ for all other types (i.e. the previous designation, if any, is updated.)

\begin{definition}[TSM Implicit Designation Context Formation]\label{def:implicit-designation-ctx-formation-S} $\uIOK{\Delta}{\uI}$ iff for each $\designate{\tau}{a} \in \uI$, we have $\istypeU{\Delta}{\tau}$.
\end{definition}

\begin{definition}[seTSM Context Formation] $\uetsmctx{\Delta}{\uASI{\uA}{\Psi}{\uI}}$ iff
\begin{enumerate}
\item $\uetsmenv{\Delta}{\Psi}$; and
\item for each $\vExpands{\tsmv}{a} \in \uA$ we have $a \in \domof{\Psi}$; and
\item $\uIOK{\Delta}{\uI}$; and
\item for each $\designate{\tau}{a} \in \uI$, we have $a \in \domof{\Psi}$.
\end{enumerate}
\end{definition}

\begin{definition}[spTSM Context Formation] $\uptsmctx{\Delta}{\uASI{\uA}{\Phi}{\uI}}$ iff 
\begin{enumerate}
\item $\uptsmenv{\Delta}{\Phi}$; and
\item for each $\vExpands{\tsmv}{a} \in \uA$ we have $a \in \domof{\Phi}$; and
\item $\uIOK{\Delta}{\uI}$; and
\item for each $\designate{\tau}{a} \in \uI$ we have $a \in \domof{\Phi}$.
\end{enumerate}
\end{definition}

We define $\uPsi, \uShyp{\tsmv}{a}{\tau}{\eparse}$, when $\uPsi=\uASI{\uA}{\Phi}{\uI}$, as an abbreviation of \[\uASI{\ctxUpdate{\uA}{\tsmv}{a}}{\Psi, \xuetsmbnd{a}{\tau}{\eparse}}{\uI}\]
%\vspace{10px}

We define $\uPhi, \uPhyp{\tsmv}{a}{\tau}{\eparse}$, when $\uPhi=\uASI{\uA}{\Phi}{\uI}$, as an abbreviation of \[\uASI{\ctxUpdate{\uA}{\tsmv}{a}}{\Phi, \xuptsmbnd{a}{\tau}{\eparse}}{\uI}\]

\subsubsection{Type Expansion}
The \emph{type expansion judgement}, $\expandsTU{\uDelta}{\utau}{\tau}$, is inductively defined as in $\miniVersePat$ by Rules (\ref{rules:expandsTU}).

\subsubsection{Bidirectionally Typed Expression and Rule Expansion}
\noindent\fbox{$\esyn{\uDelta}{\uGamma}{\uPsi}{\uPhi}{\ue}{e}{\tau}$}~~$\ue$ has expansion $e$ synthesizing type $\tau$

\begin{subequations}\label{rules:esyn-S}
\begin{equation}\label{rule:esyn-S-var}
  \inferrule{ }{ 
    \esyn{\uDelta}{\uGamma, \uGhyp{\ux}{x}{\tau}}{\uPsi}{\uPhi}{\ux}{x}{\tau}
  }
\end{equation}
\begin{equation}\label{rule:esyn-S-asc}
  \inferrule{
    \expandsTU{\uDelta}{\utau}{\tau}\\
    \eanaX{\ue}{e}{\tau}
  }{
    \esynX{\asc{\ue}{\utau}}{e}{\tau}
  }
\end{equation}
\begin{equation}\label{rule:esyn-S-let}
  \inferrule{
    \esynX{\ue}{e}{\tau}\\
    \esyn{\uDelta}{\uGamma, \uGhyp{\ux}{x}{\tau}}{\uPsi}{\uPhi}{\ue'}{e'}{\tau'}
  }{
    \esynX{\letsyn{\ux}{\ue}{\ue'}}{\aeap{\aelam{\tau}{x}{e'}}{e}}{\tau'}
  }
\end{equation}
\begin{equation}\label{rule:esyn-S-lam}
  \inferrule{
    \expandsTU{\uDelta}{\utau_1}{\tau_1}\\
    \esyn{\uDelta}{\uGamma, \uGhyp{\ux}{x}{\tau_1}}{\uPsi}{\uPhi}{\ue}{e}{\tau_2}
  }{
    \esynX{\lam{\ux}{\utau_1}{\ue}}{\aelam{\tau_1}{x}{e}}{\aparr{\tau_1}{\tau_2}}
  }
\end{equation}
\begin{equation}\label{rule:esyn-S-ap}
  \inferrule{
    \esynX{\ue_1}{e_1}{\aparr{\tau_2}{\tau}}\\
    \eanaX{\ue_2}{e_2}{\tau_2}
  }{
    \esynX{\ap{\ue_1}{\ue_2}}{\aeap{e_1}{e_2}}{\tau}
  }
\end{equation}
\begin{equation}\label{rule:esyn-S-tlam}
  \inferrule{
    \esyn{\uDelta, \uDhyp{\ut}{t}}{\uGamma}{\uPsi}{\uPhi}{\ue}{e}{\tau}
  }{
    \esynX{\Lam{\ut}{\ue}}{\aetlam{t}{e}}{\aall{t}{\tau}}
  }
\end{equation}
\begin{equation}\label{rule:esyn-S-tap}
  \inferrule{
    \esynX{\ue}{e}{\aall{t}{\tau}}\\
    \expandsTU{\uDelta}{\utau'}{\tau'}
  }{
    \esynX{\App{\ue}{\utau'}}{\aetap{e}{\tau'}}{[\tau'/t]\tau}
  }
\end{equation}
\begin{equation}\label{rule:esyn-S-unfold}
  \inferrule{
    \esynX{\ue}{e}{\arec{t}{\tau}}
  }{
    \esynX{\unfold{\ue}}{\aeunfold{e}}{[\arec{t}{\tau}/t]\tau}
  }
\end{equation}
\begin{equation}\label{rule:esyn-S-tpl}
  \inferrule{
    \{\esynX{\ue_i}{e_i}{\tau_i}\}_{i \in \labelset}
  }{
    \esynX{\tpl{\mapschema{\ue}{i}{\labelset}}}{\aetpl{\labelset}{\mapschema{e}{i}{\labelset}}}{\aprod{\labelset}{\mapschema{\tau}{i}{\labelset}}}
  }
\end{equation}
\begin{equation}\label{rule:esyn-S-pr}
  \inferrule{
    \esynX{\ue}{e}{\aprod{\labelset, \ell}{\mapschema{\tau}{i}{\labelset}; \mapitem{\ell}{\tau}}}
  }{
    \esynX{\prj{\ue}{\ell}}{\aepr{\ell}{e}}{\tau}
  }
\end{equation}
\begin{equation}\label{rule:esyn-defuetsm}
\inferrule{
  \expandsTU{\uDelta}{\utau}{\tau}\\
  \hastypeU{\emptyset}{\emptyset}{\eparse}{\aparr{\tBody}{\tParseResultExp}}\\\\
  \evalU{\eparse}{\eparse'}\\
  \esyn{\uDelta}{\uGamma}{\uPsi, \uShyp{\tsmv}{a}{\tau}{\eparse'}}{\uPhi}{\ue}{e}{\tau'}
}{
  \esynX{\usyntaxueP{\tsmv}{\utau}{\eparse}{\ue}}{e}{\tau'}
}
\end{equation}
\begin{equation}\label{rule:esyn-S-apuetsm}
\inferrule{
  \uPsi = \uPsi', \uShyp{\tsmv}{a}{\tau}{\eparse}\\\\
  \encodeBody{b}{\ebody}\\
  \evalU{\ap{\eparse}{\ebody}}{\lbltxt{SuccessE}\cdot\ecand}\\
  \decodeCondE{\ecand}{\ce}\\\\
    \segOK{\segof{\ce}}{b}\\
  \cana{\emptyset}{\emptyset}{\esceneUP{\uDelta}{\uGamma}{\uPsi}{\uPhi}{b}}{\ce}{e}{\tau}
}{
  \esyn{\uDelta}{\uGamma}{\uPsi}{\uPhi}{\utsmap{\tsmv}{b}}{e}{\tau}
}
\end{equation}
\begin{equation}\label{rule:esyn-S-implicite}
  \inferrule{
    \uPsi = \uASI{\uA \uplus \vExpands{\tsmv}{a}}{\Psi, \xuetsmbnd{a}{\tau}{\eparse}}{\uI}\\\\
    \esyn{\uDelta}{\uGamma}{\uASI{\uA \uplus \vExpands{\tsmv}{a}}{\Psi, \xuetsmbnd{a}{\tau}{\eparse}}{\uI \uplus \designate{\tau}{a}}}{\uPhi}{\ue}{e}{\tau'}
  }{
    \esyn{\uDelta}{\uGamma}{\uPsi}{\uPhi}{\implicite{\tsmv}{\ue}}{e}{\tau'}
  }
\end{equation}
\begin{equation}\label{rule:esyn-S-defuptsm}
\inferrule{
  \expandsTU{\uDelta}{\utau}{\tau}\\
  \hastypeU{\emptyset}{\emptyset}{\eparse}{\aparr{\tBody}{\tParseResultPat}}\\\\
  \evalU{\eparse}{\eparse'}\\
  \esyn{\uDelta}{\uGamma}{\uPsi}{\uPhi, \uPhyp{\tsmv}{a}{\tau}{\eparse'}}{\ue}{e}{\tau'}
}{
  \esynX{\usyntaxup{\tsmv}{\utau}{\eparse}{\ue}}{e}{\tau'}
}
\end{equation}
\begin{equation}\label{rule:esyn-S-implicitp}
  \inferrule{
    \uPhi = \uASI{\uA\uplus\vExpands{\tsmv}{a}}{\Phi, \xuptsmbnd{a}{\tau}{\eparse}}{\uI}\\\\
    \esyn{\uDelta}{\uGamma}{\uPsi}{\uASI{\uA\uplus\vExpands{\tsmv}{a}}{\Phi, \xuptsmbnd{a}{\tau}{\eparse}}{\uI \uplus \designate{\tau}{a}}}{\ue}{e}{\tau'}
  }{
    \esyn{\uDelta}{\uGamma}{\uPsi}{\uPhi}{\implicitp{\tsmv}{\ue}}{e}{\tau'}
  }
\end{equation}
\end{subequations}

\noindent\fbox{$\eana{\uDelta}{\uGamma}{\uPsi}{\uPhi}{\ue}{e}{\tau}$}~~$\ue$ has expansion $e$ when analyzed against type $\tau$

\begin{subequations}\label{rules:eana-S}
\begin{equation}\label{rule:eana-S-subsume}
  \inferrule{
    \esynX{\ue}{e}{\tau}
  }{
    \eanaX{\ue}{e}{\tau}
  }
\end{equation}
\begin{equation}\label{rule:eana-S-let}
  \inferrule{
    \esynX{\ue}{e}{\tau}\\
    \eana{\uDelta}{\uGamma, \uGhyp{\ux}{x}{\tau}}{\uPsi}{\uPhi}{\ue'}{e'}{\tau'}
  }{
    \eanaX{\letsyn{\ux}{\ue}{\ue'}}{\aeap{\aelam{\tau}{x}{e'}}{e}}{\tau'}
  }
\end{equation}
\begin{equation}\label{rule:eana-S-tlam}
  \inferrule{
    \eana{\uDelta, \uDhyp{\ut}{t}}{\uGamma}{\uPsi}{\uPhi}{\ue}{e}{\tau}
  }{
    \eanaX{\Lam{\ut}{\ue}}{\aetlam{t}{e}}{\aall{t}{\tau}}
  }
\end{equation}
\begin{equation}\label{rule:eana-S-fold}
  \inferrule{
    \eanaX{\ue}{e}{[\arec{t}{\tau}/t]\tau}
  }{
    \eanaX{\fold{\ue}}{\aefold{e}}{\arec{t}{\tau}}
  }
\end{equation}
\begin{equation}\label{rule:eana-S-tpl}
  \inferrule{
    \{\eanaX{\ue_i}{e_i}{\tau_i}\}_{i \in \labelset}
  }{
    \eanaX{\tpl{\mapschema{\ue}{i}{\labelset}}}{\aetpl{\labelset}{\mapschema{e}{i}{\labelset}}}{\aprod{\labelset}{\mapschema{\tau}{i}{\labelset}}}
  }
\end{equation}
\begin{equation}\label{rule:eana-S-in}
  \inferrule{
    \eanaX{\ue}{e}{\tau'}
  }{
    \eanaX{\inj{\ell}{\ue}}{\aein{\ell}{e}}{\asum{\labelset, \ell}{\mapschema{\tau}{i}{\labelset}; \mapitem{\ell}{\tau'}}}
  }
\end{equation}
\begin{equation}\label{rule:eana-S-match}
  \inferrule{
    \esynX{\ue}{e}{\tau}\\
    \{\ranaX{\urv_i}{r_i}{\tau}{\tau'}\}_{1 \leq i \leq n}
  }{
    \eanaX{\matchwith{\ue}{\seqschemaX{\urv}}}{\aematchwith{n}{e}{\seqschemaX{r}}}{\tau'}
  }
\end{equation}
\begin{equation}\label{rule:eana-S-defuetsm}
\inferrule{
  \expandsTU{\uDelta}{\utau}{\tau}\\
  \hastypeU{\emptyset}{\emptyset}{\eparse}{\aparr{\tBody}{\tParseResultExp}}\\\\
  \evalU{\eparse}{\eparse'}\\
  \eana{\uDelta}{\uGamma}{\uPsi, \uShyp{\tsmv}{a}{\tau}{\eparse'}}{\uPhi}{\ue}{e}{\tau'}
}{
  \eanaX{\usyntaxueP{\tsmv}{\utau}{\eparse}{\ue}}{e}{\tau'}
}
\end{equation}
\begin{equation}\label{rule:eana-S-implicite}
  \inferrule{
    \uPsi = \uASI{\uA \uplus \vExpands{\tsmv}{a}}{\Psi, \xuetsmbnd{a}{\tau}{\eparse}}{\uI}\\\\
    \eana{\uDelta}{\uGamma}{\uASI{\uA \uplus \vExpands{\tsmv}{a}}{\Psi, \xuetsmbnd{a}{\tau}{\eparse}}{\uI \uplus \designate{\tau}{a}}}{\uPhi}{\ue}{e}{\tau'}
  }{
    \eana{\uDelta}{\uGamma}{\uPsi}{\uPhi}{\implicite{\tsmv}{\ue}}{e}{\tau'}
  }
\end{equation}
\begin{equation}\label{rule:eana-S-lit}
  \inferrule{
    \uPsi = \uASI{\uA}{\Psi, \xuetsmbnd{a}{\tau}{\eparse}}{\uI \uplus \designate{\tau}{a}}\\\\
  \encodeBody{b}{\ebody}\\
  \evalU{\ap{\eparse}{\ebody}}{\lbltxt{SuccessE}\cdot\ecand}\\
  \decodeCondE{\ecand}{\ce}\\\\
    \segOK{\segof{\ce}}{b}\\
  \cana{\emptyset}{\emptyset}{\esceneUP{\uDelta}{\uGamma}{\uPsi}{\uPhi}{b}}{\ce}{e}{\tau}
  }{
    \eana{\uDelta}{\uGamma}{\uPsi}{\uPhi}{\lit{b}}{e}{\tau}
  }
\end{equation}
\begin{equation}\label{rule:eana-S-defuptsm}
\inferrule{
  \expandsTU{\uDelta}{\utau}{\tau}\\
  \hastypeU{\emptyset}{\emptyset}{\eparse}{\aparr{\tBody}{\tParseResultPat}}\\\\
  \evalU{\eparse}{\eparse'}\\
  \eana{\uDelta}{\uGamma}{\uPsi}{\uPhi, \uPhyp{\tsmv}{a}{\tau}{\eparse'}}{\ue}{e}{\tau'}
}{
  \eanaX{\usyntaxup{\tsmv}{\utau}{\eparse}{\ue}}{e}{\tau'}
}
\end{equation}
\begin{equation}\label{rule:eana-S-implicitp}
  \inferrule{
    \uPhi = \uASI{\uA\uplus\vExpands{\tsmv}{a}}{\Phi, \xuptsmbnd{a}{\tau}{\eparse}}{\uI}\\\\
    \eana{\uDelta}{\uGamma}{\uPsi}{\uASI{\uA\uplus\vExpands{\tsmv}{a}}{\Phi, \xuptsmbnd{a}{\tau}{\eparse}}{\uI \uplus \designate{\tau}{a}}}{\ue}{e}{\tau'}
  }{
    \eana{\uDelta}{\uGamma}{\uPsi}{\uPhi}{\implicitp{\tsmv}{\ue}}{e}{\tau'}
  }
\end{equation}
\end{subequations}

\noindent\fbox{$\rana{\uDelta}{\uGamma}{\uPsi}{\uPhi}{\urv}{r}{\tau}{\tau'}$}~~$\urv$ has expansion $r$ taking values of type $\tau$ to values of type $\tau'$

\begin{equation}\label{rule:rana-S}
  \inferrule{
    \patExpands{\uGG{\uG'}{\Gamma'}}{\uPhi}{\upv}{p}{\tau}\\
    \eana{\uDD{\uD}{\Delta}}{\uGG{\uG \uplus \uG'}{\Gamma \cup \Gamma'}}{\uPsi}{\uPhi}{\ue}{e}{\tau'}
  }{
    \rana{\uDD{\uD}{\Delta}}{\uGG{\uG}{\Gamma}}{\uPsi}{\uPhi}{\matchrule{\upv}{\ue}}{\aematchrule{p}{e}}{\tau}{\tau'}
  }
\end{equation}

\subsubsection{Pattern Expansion}
\noindent\fbox{$\patExpands{\upctx}{\uPhi}{\upv}{p}{\tau}$}~~$\upv$ has expansion $p$ matching against $\tau$ generating hypotheses $\upctx$

\begin{subequations}\label{rules:patExpands-B}
\begin{equation}\label{rule:patExpands-B-var}
\inferrule{ }{
  \patExpands{\uGG{\vExpands{\ux}{x}}{\Ghyp{x}{\tau}}}{\uPhi}{\ux}{x}{\tau}
}
\end{equation}
\begin{equation}\label{rule:patExpands-B-wild}
\inferrule{ }{
  \patExpands{\uGG{\emptyset}{\emptyset}}{\uPhi}{\wildp}{\aewildp}{\tau}
}
\end{equation}
\begin{equation}\label{rule:patExpands-B-fold}
\inferrule{ 
  \patExpands{\upctx}{\uPhi}{\upv}{p}{[\arec{t}{\tau}/t]\tau}
}{
  \patExpands{\upctx}{\uPhi}{\foldp{\upv}}{\aefoldp{p}}{\arec{t}{\tau}}
}
\end{equation}
\begin{equation}\label{rule:patExpands-B-tpl}
\inferrule{
    \tau = \aprod{\labelset}{\mapschema{\tau}{i}{\labelset}}\\\\
  \{\patExpands{{\upctx_i}}{\uPhi}{\upv_i}{p_i}{\tau_i}\}_{i \in \labelset}\\
}{
    \patExpands{\GIconsi{i \in \labelset}{\upctx_i}}{\uPhi}{\tplp{\mapschema{\upv}{i}{\labelset}}}{\aetplp{\labelset}{\mapschema{p}{i}{\labelset}}}{\tau}
}
\end{equation}
\begin{equation}\label{rule:patExpands-B-in}
\inferrule{
  \patExpands{\upctx}{\uPhi}{\upv}{p}{\tau}
}{
  \patExpands{\upctx}{\uPhi}{\injp{\ell}{\upv}}{\aeinjp{\ell}{p}}{\asum{\labelset, \ell}{\mapschema{\tau}{i}{\labelset}; \mapitem{\ell}{\tau}}}
}
\end{equation}
\begin{equation}\label{rule:patExpands-B-apuptsm}
\inferrule{
  \uPhi = \uPhi', \uPhyp{\tsmv}{a}{\tau}{\eparse}\\\\
  \encodeBody{b}{\ebody}\\
  \evalU{\ap{\eparse}{\ebody}}{{\lbltxt{SuccessP}}\cdot{\ecand}}\\
  \decodeCEPat{\ecand}{\cpv}\\\\
    \segOK{\segof{\cpv}}{b}\\
  \cvalidP{\upctx}{\pscene{\uDelta}{\uPhi}{b}}{\cpv}{p}{\tau}
}{
  \patExpands{\upctx}{\uPhi}{\utsmap{\tsmv}{b}}{p}{\tau}
}
\end{equation}
\begin{equation}\label{rule:patExpands-B-lit}
\inferrule{
  \uPhi = \uASI{\uA}{\Phi, \xuptsmbnd{a}{\tau}{\eparse}}{\uI, \designate{\tau}{a}}\\\\
  \encodeBody{b}{\ebody}\\
  \evalU{\ap{\eparse}{\ebody}}{{\lbltxt{SuccessP}}\cdot{\ecand}}\\
  \decodeCEPat{\ecand}{\cpv}\\\\
    \segOK{\segof{\cpv}}{b}\\
  \cvalidP{\upctx}{\pscene{\uDelta}{\uPhi}{b}}{\cpv}{p}{\tau}
}{
  \patExpands{\upctx}{\uPhi}{\lit{b}}{p}{\tau}
}
\end{equation}
\end{subequations}

\section{Proto-Expansion Validation}
\subsection{Syntax of Proto-Expansions}
The syntax of proto-expansions was defined in Sec. \ref{appendix:proto-expansions-SES}.

% \[\begin{array}{lllllll}
% \textbf{Sort} & & & \textbf{Operational Form} & \textbf{Stylized Form} & \textbf{Description}\\
% \mathsf{PrTyp} & \ctau & ::= & \cdots & \cdots & \text{(as in $\miniVersePat$)}\\
% % &&& \aceparr{\ctau}{\ctau} & \parr{\ctau}{\ctau} & \text{partial function}\\
% % &&& \aceall{t}{\ctau} & \forallt{t}{\ctau} & \text{polymorphic}\\
% % &&& \acerec{t}{\ctau} & \rect{t}{\ctau} & \text{recursive}\\
% % &&& \aceprod{\labelset}{\mapschema{\ctau}{i}{\labelset}} & \prodt{\mapschema{\ctau}{i}{\labelset}} & \text{labeled product}\\
% % &&& \acesum{\labelset}{\mapschema{\ctau}{i}{\labelset}} & \sumt{\mapschema{\ctau}{i}{\labelset}} & \text{labeled sum}\\
% %\LCC &&& \gray & \gray & \gray\\
% % &&& \acesplicedt{m}{n} & \splicedt{m}{n} & \text{spliced}\\%\ECC
% \mathsf{PrExp} & \ce & ::= & \cdots & \cdots & \text{(as in $\miniVersePat$)}\\
% &&& \aceasc{\ctau}{\ce} & \asc{\ce}{\ctau} & \text{ascription}\\
% &&& \aceletsyn{x}{\ce}{\ce} & \letsyn{x}{\ce}{\ce} & \text{value binding}\\
% % &&& \aceanalam{x}{\ce} & \analam{x}{\ce} & \text{abstraction (unannotated)}\\
% % &&& \acelam{\ctau}{x}{\ce} & \lam{x}{\ctau}{\ce} & \text{abstraction (annotated)}\\
% % &&& \aceap{\ce}{\ce} & \ap{\ce}{\ce} & \text{application}\\
% % &&& \acetlam{t}{\ce} & \Lam{t}{\ce} & \text{type abstraction}\\
% % &&& \acetap{\ce}{\ctau} & \App{\ce}{\ctau} & \text{type application}\\
% % &&& \aceanafold{\ce} & \fold{\ce} & \text{fold}\\
% % &&& \aceunfold{\ce} & \unfold{\ce} & \text{unfold}\\
% % &&& \acetpl{\labelset}{\mapschema{\ce}{i}{\labelset}} & \tpl{\mapschema{\ce}{i}{\labelset}} & \text{labeled tuple}\\
% % &&& \acepr{\ell}{\ce} & \prj{\ce}{\ell} & \text{projection}\\
% % &&& \aceanain{\ell}{\ce} & \inj{\ell}{\ce} & \text{injection}\\
% % &&& \acematchwithb{n}{\ce}{\seqschemaX{\urv}} & \matchwith{\ce}{\seqschemaX{\crv}} & \text{match}\\%\LCC &&& \gray & \gray & \gray\\
% % &&& \acesplicede{m}{n} & \splicede{m}{n} & \text{spliced}\\%\ECC
% % &&& \acesplicedet{m}{n}{\ctau} & \splicedet{m}{n}{\ctau} & \text{spliced (analytic)}\\
% \mathsf{PrRule} & \crv & ::= & \cdots & \cdots & \text{(as in $\miniVersePat$)}\\
% \mathsf{PrPat} & \cpv & ::= & \cdots & \cdots & \text{(as in $\miniVersePat$)}\\
% % &&& \acefoldp{p} & \foldp{p} & \text{fold pattern}\\
% % &&& \acetplp{\labelset}{\mapschema{\cpv}{i}{\labelset}} & \tplp{\mapschema{\cpv}{i}{\labelset}} & \text{labeled tuple pattern}\\
% % &&& \aceinjp{\ell}{\cpv} & \injp{\ell}{\cpv} & \text{injection pattern}\\
% % &&& \acesplicedp{m}{n} & \splicedp{m}{n} & \text{spliced}
% \end{array}\]

\subsubsection{Common Proto-Expansion Terms}
Each expanded term, except variable patterns, maps onto a proto-expansion term. We refer to these as the \emph{common proto-expansion terms}. In particular:
\begin{itemize}
  \item Each type, $\tau$, maps onto a proto-type, $\Cof{\tau}$, as follows:
  \[\arraycolsep=1pt\begin{array}{rl}
  \Cof{t} & = t\\
  \Cof{\aparr{\tau_1}{\tau_2}} & = \aceparr{\Cof{\tau_1}}{\Cof{\tau_2}}\\
  \Cof{\aall{t}{\tau}} & = \aceall{t}{\Cof{\tau}}\\
  \Cof{\arec{t}{\tau}} & = \acerec{t}{\Cof{\tau}}\\
  \Cof{\aprod{\labelset}{\mapschema{\tau}{i}{\labelset}}} & = \aceprod{\labelset}{\mapschemax{\Cofv}{\tau}{i}{\labelset}}\\
  \Cof{\asum{\labelset}{\mapschema{\tau}{i}{\labelset}}} & = \acesum{\labelset}{\mapschemax{\Cofv}{\tau}{i}{\labelset}}
  \end{array}\]
  \item Each expanded expression, $e$, maps onto a proto-expression, $\Cof{e}$, as follows:
  \[\arraycolsep=1pt\hspace{-15px}\begin{array}{rl}
  \Cof{x} & = x\\
  \Cof{\aelam{\tau}{x}{e}} & = \acelam{\Cof{\tau}}{x}{\Cof{e}}\\
  \Cof{\aeap{e_1}{e_2}} & = \aceap{\Cof{e_1}}{\Cof{e_2}}\\
  \Cof{\aetlam{t}{e}} & = \acetlam{t}{\Cof{e}}\\
  \Cof{\aetap{e}{\tau}} & = \acetap{\Cof{e}}{\Cof{\tau}}\\
  \Cof{\aefold{e}} & = \aceasc{\acerec{t}{\Cof{\tau}}}{\acefold{\Cof e}}\\
  \Cof{\aeunfold{e}} & = \aceunfold{\Cof{e}}\\
  \Cof{\aetpl{\labelset}{\mapschema{e}{i}{\labelset}}} & = \acetpl{\labelset}{\mapschemax{\Cofv}{e}{i}{\labelset}}\\
  \Cof{\aein{\ell}{e}} &= \aceasc
    {
      \acesum{\labelset}{\mapschemax{\Cofv}{\tau}{i}{\labelset}}
    }{\acein{\ell}{\Cof{e}}}\\
  \Cof{\aematchwith{n}{e}{\seqschemaX{r}}} & = \aceasc{\Cof{\tau}}{\acematchwith{n}{\Cof{e}}{\seqschemaXx{\Cofv}{r}}}
  \end{array}\]
  \item Each expanded rule, $r$, maps onto the proto-rule, $\Cof{r}$, as follows:
  \begin{align*}
  \Cof{\aematchrule{p}{e}} & = \acematchrule{p}{\Cof{e}}
  \end{align*}
  \item Each expanded pattern, $p$, except for the variable patterns, maps onto a proto-pattern, $\Cof{p}$, as follows:
  \begin{align*}
  \Cof{\aewildp} & = \acewildp\\
  \Cof{\aefoldp{p}} & = \acefoldp{\Cof{p}}\\
  \Cof{\aetplp{\labelset}{\mapschema{p}{i}{\labelset}}} & = \acetplp{\labelset}{\mapschemax{\Cofv}{p}{i}{\labelset}}\\
  \Cof{\aeinjp{\ell}{p}} & = \aceinjp{\ell}{\Cof{p}}
  \end{align*}
\end{itemize}

These definitions differ from those given in Sec. \ref{appendix:proto-expansions-SES} in that they include the type information necessary for bidirectional typechecking.

\subsubsection{Proto-Expression Encoding and Decoding}
The type $\tCEExp$ and the judgements $\encodeCondE{\ce}{e}$ and $\decodeCondE{e}{\ce}$ are characterized as described in Sec. \ref{appendix:proto-expansions-SES}.

\subsubsection{Proto-Pattern Encoding and Decoding}
The type $\tCEPat$ and the judgements $\encodeCEPat{\cpv}{e}$ and $\decodeCEPat{e}{\cpv}$ are characterized as described in Sec. \ref{appendix:proto-expansions-SES}.

\subsubsection{Splice Summaries}
The \emph{splice summary} of a proto-expression, $\summaryOf{\ce}$, or proto-pattern, $\summaryOf{\cpv}$, is the finite set of references to spliced types, expressions {and patterns} that it mentions.

\subsubsection{Segmentations}
A \emph{segment set}, $\psi$, is a finite set of pairs of natural numbers indicating the locations of spliced terms. The \emph{segmentation} of a proto-expression, $\segof{\ce}$, or proto-pattern, $\segof{\cpv}$, is the segment set implied by its splice summary.

\subsection{Proto-Expansion Validation}\label{appendix:proto-expansion-validation-BS}
\subsubsection{Proto-Type Validation}
The \emph{proto-type validation judgement}, $\cvalidT{\Delta}{\tscenev}{\ctau}{\tau}$, is inductively defined by Rules (\ref{rules:cvalidT-U}), which were defined in Sec. \ref{appendix:proto-type-validation-SES}.

\subsubsection{Bidirectional Proto-Expression and Proto-Rule Validation}
\emph{Expression splicing scenes}, $\escenev$, are of the form $\esceneUP{\uDelta}{\uGamma}{\uPsi}{\uPhi}{b}$. We write $\tsfrom{\escenev}$ for the type splicing scene constructed by dropping unnecessary contexts from $\escenev$:
\[\tsfrom{\esceneUP{\uDelta}{\uGamma}{\uPsi}{\uPhi}{b}} = \tsceneUP{\uDelta}{b}\]

\noindent\fbox{$\strut\csynX{\ce}{e}{\tau}$}~~$\ce$ has expansion $e$ synthesizing type $\tau$
\begin{subequations}\label{rules:csyn}
\begin{equation}\label{rule:csyn-var}
  \inferrule{ }{ 
    \csyn{\Delta}{\Gamma, \Ghyp{x}{\tau}}{\escenev}{x}{x}{\tau}
  }
\end{equation}
\begin{equation}\label{rule:csyn-asc}
  \inferrule{
    \cvalidT{\Delta}{\tsfrom{\escenev}}{\ctau}{\tau}\\
    \canaX{\ce}{e}{\tau}
  }{
    \csynX{\aceasc{\ctau}{\ce}}{e}{\tau}
  }
\end{equation}
\begin{equation}\label{rule:csyn-let}
  \inferrule{
    \csynX{\ce}{e}{\tau}\\
    \csyn{\Delta}{\Gamma, \Ghyp{x}{\tau}}{\escenev}{\ce'}{e'}{\tau'}
  }{
    \csynX{\aceletsyn{x}{\ce}{\ce'}}{\aeap{\aelam{\tau}{x}{e'}}{e}}{\tau'}
  }
\end{equation}
\begin{equation}\label{rule:csyn-lam}
  \inferrule{
    \cvalidT{\Delta}{\tsfrom{\escenev}}{\ctau_1}{\tau_1}\\
    \csyn{\Delta}{\Gamma, \Ghyp{x}{\tau_1}}{\escenev}{\ce}{e}{\tau_2}
  }{
    \csynX{\acelam{\ctau_1}{x}{\ce}}{\aelam{\tau_1}{x}{e}}{\aparr{\tau_1}{\tau_2}}
  }
\end{equation}
\begin{equation}\label{rule:csyn-ap}
  \inferrule{
    \csynX{\ce_1}{e_1}{\aparr{\tau_2}{\tau}}\\
    \canaX{\ce_2}{e_2}{\tau_2}
  }{
    \csynX{\aceap{\ce_1}{\ce_2}}{\aeap{e_1}{e_2}}{\tau}
  }
\end{equation}
\begin{equation}\label{rule:csyn-tlam}
  \inferrule{
    \csyn{\Delta, \Dhyp{t}}{\Gamma}{\escenev}{\ce}{e}{\tau}
  }{
    \csynX{\acetlam{t}{\ce}}{\aetlam{t}{e}}{\aall{t}{\tau}}
  }
\end{equation}
\begin{equation}\label{rule:csyn-tap}
  \inferrule{
    \csynX{\ce}{e}{\aall{t}{\tau}}\\
    \cvalidT{\Delta}{\tsfrom{\escenev}}{\ctau'}{\tau'}
  }{
    \csynX{\acetap{\ce}{\ctau'}}{\aetap{e}{\tau'}}{[\tau'/t]\tau}
  }
\end{equation}
\begin{equation}\label{rule:csyn-unfold}
  \inferrule{
    \csynX{\ce}{e}{\arec{t}{\tau}}
  }{
    \csynX{\aceunfold{\ce}}{\aeunfold{e}}{[\arec{t}{\tau}/t]\tau}
  }
\end{equation}
\begin{equation}\label{rule:csyn-tpl}
  \inferrule{
    \tau = \aprod{\labelset}{\mapschema{\tau}{i}{\labelset}}\\\\
    \{\csynX{\ce_i}{e_i}{\tau_i}\}_{i \in \labelset}
  }{
    \csynX{\acetpl{\labelset}{\mapschema{\ce}{i}{\labelset}}}{\aetpl{\labelset}{\mapschema{e}{i}{\labelset}}}{\tau}
  }
\end{equation}
\begin{equation}\label{rule:csyn-pr}
  \inferrule{
    \csynX{\ce}{e}{\aprod{\labelset, \ell}{\mapschema{\tau}{i}{\labelset}; \mapitem{\ell}{\tau}}}
  }{
    \csynX{\acepr{\ell}{\ce}}{\aepr{\ell}{e}}{\tau}
  }
\end{equation}
% \begin{equation}\label{rule:csyn-match}
%   \inferrule{
%     n > 0\\
%     \csynX{\ce}{e}{\tau}\\
%     \{\crsynX{\crv_i}{r_i}{\tau}{\tau'}\}_{1 \leq i \leq n}
%   }{
%     \csynX{\acematchwithb{n}{\ce}{\seqschemaX{\crv}}}{\aematchwith{n}{e}{\seqschemaX{r}}}{\tau'}
%   }
% \end{equation}
\begin{equation}\label{rule:csyn-splicede}
\inferrule{
  \cvalidT{\emptyset}{\tsfrom{\escenev}}{\ctau}{\tau}\\
  \escenev=\esceneUP{\uDD{\uD}{\Delta_\text{app}}}{\uGG{\uG}{\Gamma_\text{app}}}{\uPsi}{\uPhi}{b}\\
  \parseUExp{\bsubseq{b}{m}{n}}{\ue}\\
  \eana{\uDD{\uD}{\Delta_\text{app}}}{\uGG{\uG}{\Gamma_\text{app}}}{\uPsi}{\uPhi}{\ue}{e}{\tau}\\\\
  \Delta \cap \Delta_\text{app} = \emptyset\\
  \domof{\Gamma} \cap \domof{\Gamma_\text{app}} = \emptyset
}{
  \csyn{\Delta}{\Gamma}{\escenev}{\acesplicede{m}{n}{\ctau}}{e}{\tau}
}
\end{equation}
\end{subequations}

\noindent\fbox{$\strut\canaX{\ce}{e}{\tau}$}~~$\ce$ has expansion $e$ when analyzed against type $\tau$
\begin{subequations}\label{rules:cana}
\begin{equation}\label{rule:cana-subsume}
  \inferrule{
    \csynX{\ce}{e}{\tau}
  }{
    \canaX{\ce}{e}{\tau}
  }
\end{equation}
\begin{equation}\label{rule:cana-let}
  \inferrule{
    \csynX{\ce}{e}{\tau}\\
    \cana{\Delta}{\Gamma, \Ghyp{x}{\tau}}{\escenev}{\ce'}{e'}{\tau'}
  }{
    \canaX{\aceletsyn{x}{\ce}{\ce'}}{\aeap{\aelam{\tau}{x}{e'}}{e}}{\tau'}
  }
\end{equation}
% \begin{equation}\label{rule:cana-analam}
%   \inferrule{
%     \cana{\Delta}{\Gamma, \Ghyp{x}{\tau_1}}{\escenev}{\ce}{e}{\tau_2}
%   }{
%     \canaX{\aceanalam{x}{\ue}}{\aelam{\tau_1}{x}{e}}{\aparr{\tau_1}{\tau_2}}
%   }
% \end{equation}
\begin{equation}\label{rule:cana-tlam}
  \inferrule{
    \cana{\Delta, \Dhyp{t}}{\Gamma}{\escenev}{\ce}{e}{\tau}
  }{
    \canaX{\acetlam{t}{\ce}}{\aetlam{t}{e}}{\aall{t}{\tau}}
  }
\end{equation}
\begin{equation}\label{rule:cana-fold}
  \inferrule{
    \canaX{\ce}{e}{[\arec{t}{\tau}/t]\tau}
  }{
    \canaX{\aceanafold{\ce}}{\aefold{e}}{\arec{t}{\tau}}
  }
\end{equation}
\begin{equation}\label{rule:cana-tpl}
  \inferrule{
    \tau = \aprod{\labelset}{\mapschema{\tau}{i}{\labelset}}\\\\
    \{\canaX{\ce_i}{e_i}{\tau_i}\}_{i \in \labelset}
  }{
    \canaX{\acetpl{\labelset}{\mapschema{\ce}{i}{\labelset}}}{\aetpl{\labelset}{\mapschema{e}{i}{\labelset}}}{\tau}
  }
\end{equation}
\begin{equation}\label{rule:cana-in}
  \inferrule{
    \canaX{\ce}{e}{\tau'}
  }{
    % \left(\shortstack{$\Delta~\Gamma \vdash^{\escenev} \aceanain{\ell}{\ce}$\\$\leadsto$\\$\aein{\ell} \Leftarrow \asum{\labelset, \ell}{\mapschema{\tau}{i}{\labelset}; \mapitem{\ell}{\tau}}$\vspace{-1.2em}}\right)
    \canaX{\acein{\ell}{\ce}}{\aein{\ell}{e}}{\asum{\labelset, \ell}{\mapschema{\tau}{i}{\labelset}; \mapitem{\ell}{\tau'}}}
  }
\end{equation}
\begin{equation}\label{rule:cana-match}
  \inferrule{
    \csynX{\ce}{e}{\tau}\\
    \{\cranaX{\crv_i}{r_i}{\tau}{\tau'}\}_{1 \leq i \leq n}
  }{
    \canaX{\acematchwithb{n}{\ce}{\seqschemaX{\crv}}}{\aematchwith{n}{e}{\seqschemaX{r}}}{\tau'}
  }
\end{equation}
\end{subequations}

\noindent\fbox{$\strut\crana{\Delta}{\Gamma}{\escenev}{\crv}{r}{\tau}{\tau'}$}~~$\crv$ has expansion $r$ taking values of type $\tau$ to values of type $\tau'$
\begin{equation}\label{rule:crana}
\inferrule{
  \patType{\pctx}{p}{\tau}\\
  \cana{\Delta}{\Gcons{\Gamma}{\pctx}}{\escenev}{\ce}{e}{\tau'}
}{
  \crana{\Delta}{\Gamma}{\escenev}{\acematchrule{p}{\ce}}{\aematchrule{p}{e}}{\tau}{\tau'}
}
\end{equation}

\subsubsection{Proto-Pattern Validation}
\emph{Pattern splicing scenes}, $\pscenev$, are of the form $\pscene{\uDelta}{\uPhi}{b}$.

\vspace{10px}\noindent\fbox{\strut$\cvalidP{\upctx}{\pscenev}{\cpv}{p}{\tau}$}~~$\cpv$ has expansion $p$ matching against $\tau$ generating hypotheses $\upctx$
\begin{subequations}\label{rules:cvalidP-B}
\begin{equation}\label{rule:cvalidP-B-wild}
\inferrule{ }{
  \cvalidP{\uGG{\emptyset}{\emptyset}}{\pscenev}{\acewildp}{\aewildp}{\tau}
}
\end{equation}
\begin{equation}\label{rule:cvalidP-B-fold}
\inferrule{
  \cvalidP{\upctx}{\pscenev}{\cpv}{p}{[\arec{t}{\tau}/t]\tau}
}{
  \cvalidP{\upctx}{\pscenev}{\acefoldp{\cpv}}{\aefoldp{p}}{\arec{t}{\tau}}
}
\end{equation}
\begin{equation}\label{rule:cvalidP-B-tpl}
\inferrule{
  \tau = \aprod{\labelset}{\mapschema{\tau}{i}{\labelset}}\\\\
  \{\cvalidP{\upctx_i}{\pscenev}{\cpv_i}{p_i}{\tau_i}\}_{i \in \labelset}
}{
% \left(\shortstack{$\vdash^{\pscenev} \acetplp{\labelset}{\mapschema{\cpv}{i}{\labelset}}$\\$\leadsto$\\$\aetplp{\labelset}{\mapschema{p}{i}{\labelset}} : \aprod{\labelset}{\mapschema{\tau}{i}{\labelset}}~\dashVx^{\,\Gconsi{i \in \labelset}{\upctx_i}}$\vspace{-1.2em}}\right)
  \cvalidP{\GIconsi{i \in \labelset}{\upctx_i}}{\pscenev}{\acetplp{\labelset}{\mapschema{\cpv}{i}{\labelset}}}{\aetplp{\labelset}{\mapschema{p}{i}{\labelset}}}{\tau}
}
\end{equation}
\begin{equation}\label{rule:cvalidP-B-in}
\inferrule{
  \cvalidP{\upctx}{\pscenev}{\cpv}{p}{\tau}
}{
  \cvalidP{\upctx}{\pscenev}{\aceinjp{\ell}{\cpv}}{\aeinjp{\ell}{p}}{\asum{\labelset, \ell}{\mapschema{\tau}{i}{\labelset}; \mapitem{\ell}{\tau}}}
}
\end{equation}
\begin{equation}\label{rule:cvalidP-B-spliced}
\inferrule{
  \cvalidT{\emptyset}{\tsceneUP{\uDelta}{b}}{\ctau}{\tau}\\
  \parseUPat{\bsubseq{b}{m}{n}}{\upv}\\
  \patExpands{\upctx}{\uPhi}{\upv}{p}{\tau}
}{
  \cvalidP{\upctx}{\pscene{\uDelta}{\uPhi}{b}}{\acesplicedp{m}{n}{\ctau}}{p}{\tau}
}
\end{equation}
\end{subequations}
\section{Metatheory}
\subsection{Typed Pattern Expansion}\label{appendix:SES-typed-pattern-expansion}
\begin{theorem}[Typed Pattern Expansion]\label{thm:typed-pattern-expansion-B} ~
\begin{enumerate}
  \item If $\pExpandsSP{\uDD{\uD}{\Delta}}{\uASI{\uA}{\Phi}{\uI}}{\upv}{p}{\tau}{\uGG{\uG}{\pctx}}$ then $\patType{\pctx}{p}{\tau}$.
  \item If $\cvalidP{\uGG{\uG}{\pctx}}{\pscene{\uDD{\uD}{\Delta}}{\uAP{\uA}{\Phi}}{b}}{\cpv}{p}{\tau}$ then $\patType{\pctx}{p}{\tau}$.
\end{enumerate}
\end{theorem}
\begin{proof}
  By mutual rule induction over Rules (\ref{rules:patExpands-B}) and Rules (\ref{rules:cvalidP-B}
  \begin{enumerate}
  \item We induct on the premise. In the following, let $\uDelta=\uDD{\uD}{\Delta}$ and $\upctx=\uGG{\uG}{\pctx}$ and $\uPhi=\uASI{\uA}{\Phi}{\uI}$.
  \begin{byCases}
    \item[\text{(\ref{rule:patExpands-B-var}) \textbf{through} (\ref{rule:patExpands-B-apuptsm})}] These cases follow by identical argument to the corresponding cases of Theorem \ref{thm:typed-pattern-expansion}.

    \item[\text{(\ref{rule:patExpands-B-lit})}]~
    \resetpfcounter
    \begin{pfsteps}
      \item \upv = \lit{b} \BY{assumption}
      \item \Phi = \Phi', \xuptsmbnd{a}{\tau}{\eparse} \BY{assumption}
      \item \uI = \uI', \designate{\tau}{a} \BY{assumption}
      \item   \encodeBody{b}{\ebody} \BY{assumption}
      \item   \evalU{\ap{\eparse}{\ebody}}{{\lbltxt{SuccessP}}\cdot{\ecand}} \BY{assumption}
      \item  \decodeCEPat{\ecand}{\cpv} \BY{assumption}
      \item  \cvalidP{\upctx}{\pscene{\uDelta}{\uPhi}{b}}{\cpv}{p}{\tau} \BY{assumption} \pflabel{cvalidP}
      \item \patType{\pctx}{p}{\tau} \BY{IH, part 2 on \pfref{cvalidP}}
    \end{pfsteps}
    \resetpfcounter
%     \item[\text{(\ref{rule:patExpands-var})}] ~
%       \begin{pfsteps*}
%         \item $\upv=\ux$ \BY{assumption}
%         \item $p=x$ \BY{assumption}
%         \item $\pctx=\Ghyp{x}{\tau}$ \BY{assumption}
%         \item $\patType{\Ghyp{x}{\tau}}{x}{\tau}$ \BY{Rule (\ref{rule:patType-var})}
%       \end{pfsteps*}
%       \resetpfcounter
%     \item[\text{(\ref{rule:patExpands-wild})}] ~
%       \begin{pfsteps*}
%         \item $p=\aewildp$ \BY{assumption}
%         \item $\pctx = \emptyset$ \BY{assumption}
%         \item $\patType{\emptyset}{\aewildp}{\tau}$ \BY{Rule (\ref{rule:patType-wild})}
%       \end{pfsteps*}
%       \resetpfcounter
%     \item[\text{(\ref{rule:patExpands-fold})}] ~
%       \begin{pfsteps*}
%         \item $\upv=\foldp{\upv'}$ \BY{assumption}
%         \item $p=\aefoldp{p'}$ \BY{assumption}
%         \item $\tau=\arec{t}{\tau'}$ \BY{assumption}
%         %\item $\uptsmenv{\Delta}{\Phi}$ \BY{assumption} \pflabel{env}
%         \item $\patExpands{\upctx}{\uPhi}{\upv'}{p'}{[\arec{t}{\tau'}/t]\tau'}$ \BY{assumption} \pflabel{patExpands}
%         \item $\patType{\pctx}{p'}{[\arec{t}{\tau'}/t]\tau'}$ \BY{IH, part 1 on \pfref{patExpands}} \pflabel{patType}
%         \item $\patType{\pctx}{\aefoldp{p'}}{\arec{t}{\tau'}}$ \BY{Rule (\ref{rule:patType-fold}) on \pfref{patType}}
%       \end{pfsteps*}
%       \resetpfcounter
%     \item[\text{(\ref{rule:patExpands-tpl})}] ~
%       \begin{pfsteps*}
%         \item $\upv=\tplp{\mapschema{\upv}{i}{\labelset}}$ \BY{assumption}
%         \item $p=\aetplp{\labelset}{\mapschema{p}{i}{\labelset}}$ \BY{assumption}
%         \item $\tau=\aprod{\labelset}{\mapschema{\tau}{i}{\labelset}}$ \BY{assumption}
%         \item $\{\patExpands{\uGG{\uG_i}{\pctx_i}}{\uPhi}{\upv_i}{p_i}{\tau_i}\}_{i \in \labelset}$ \BY{assumption} \pflabel{patExpands}
%         \item $\pctx = \Gconsi{i \in \labelset}{\pctx_i}$ \BY{assumption}
%         %\item $\uptsmenv{\Delta}{\Phi}$ \BY{assumption} \pflabel{env}
%         \item $\{\patType{\pctx_i}{p_i}{\tau_i}\}_{i \in \labelset}$ \BY{IH, part 1 over \pfref{patExpands}}\pflabel{patType}
%         \item $\patType{\Gconsi{i \in \labelset}{\pctx_i}}{\aetplp{\labelset}{\mapschema{p}{i}{\labelset}}}{\aprod{\labelset}{\mapschema{\tau}{i}{\labelset}}}$ \BY{Rule (\ref{rule:patType-tpl}) on \pfref{patType}}
%       \end{pfsteps*}
%       \resetpfcounter
%     \item[\text{(\ref{rule:patExpands-in})}] ~
%       \begin{pfsteps*}
%         \item $\upv=\injp{\ell}{\upv'}$ \BY{assumption}
%         \item $p=\aeinjp{\ell}{p'}$ \BY{assumption}
%         \item $\tau=\asum{\labelset, \ell}{\mapschema{\tau}{i}{\labelset}; \mapitem{\ell}{\tau'}}$ \BY{assumption}
%         \item $\patExpands{\upctx}{\uPhi}{\upv'}{p'}{\tau'}$ \BY{assumption} \pflabel{patExpands}
% %        \item $\uptsmenv{\Delta}{\Phi}$ \BY{assumption} \pflabel{env}
%         \item $\patType{\pctx}{p'}{\tau'}$ \BY{IH, part 1 on \pfref{patExpands}} \pflabel{patType}
%         \item $\patType{\pctx}{\aeinjp{\ell}{p'}}{\asum{\labelset, \ell}{\mapschema{\tau}{i}{\labelset}; \mapitem{\ell}{\tau'}}}$ \BY{Rule (\ref{rule:patType-inj}) on \pfref{patType}}
%       \end{pfsteps*}
%       \resetpfcounter
%     \item[\text{(\ref{rule:patExpands-apuptsm})}] ~
%       \begin{pfsteps*}
%         \item $\upv=\utsmap{\tsmv}{b}$ \BY{assumption}
%         \item $\uA=\uA', \vExpands{\tsmv}{a}$ \BY{assumption}
%         \item $\Phi=\Phi', \xuptsmbnd{a}{\tau}{\eparse}$ \BY{assumption}
%         \item $\encodeBody{b}{\ebody}$ \BY{assumption}
%         \item $\evalU{\eparse(\ebody)}{{\lbltxt{SuccessP}}\cdot{\ecand}}$ \BY{assumption}
%         \item $\decodeCEPat{\ecand}{\cpv}$ \BY{assumption}
%         \item $\cvalidP{\uGG{\uG}{\pctx}}{\pscene{\uDelta}{\uAP{\uA}{\Phi}}{b}}{\cpv}{p}{\tau}$ \BY{assumption} \pflabel{cvalidP}
% %        \item $\uptsmenv{\Delta}{\Phi', \xuptsmbnd{a}{\tau}{\eparse}}$ \BY{assumption} \pflabel{env}
%         \item $\patType{\pctx}{p}{\tau}$ \BY{IH, part 2 on \pfref{cvalidP}}
%       \end{pfsteps*}
%       \resetpfcounter
  \end{byCases}

  \item We induct on the premise. All cases follow by identical argument to the corresponding cases of Theorem \ref{thm:typed-pattern-expansion}.
%   \begin{byCases}
%     \item[\text{(\ref{rule:cvalidP-B-wild})}] ~
%       \begin{pfsteps*}
%         \item $p=\aewildp$ \BY{assumption}
%         \item $\pctx=\emptyset$ \BY{assumption}
%         \item $\patType{\emptyset}{\aewildp}{\tau}$ \BY{Rule (\ref{rule:patType-wild})}
%       \end{pfsteps*}
%       \resetpfcounter
%     \item[\text{(\ref{rule:cvalidP-UP-fold})}] ~
%       \begin{pfsteps*}
%         \item $\cpv=\acefoldp{\cpv'}$ \BY{assumption}
%         \item $p=\aefoldp{p'}$ \BY{assumption}
%         \item $\tau=\arec{t}{\tau'}$ \BY{assumption}
%         % \item $\uptsmenv{\Delta}{\Phi}$ \BY{assumption} \pflabel{env}
%         \item $\cvalidP{\upctx}{\pscene{\uDelta}{\uPhi}{b}}{\cpv'}{p'}{[\arec{t}{\tau'}/t]\tau'}$ \BY{assumption} \pflabel{cvalidP}
%         \item $\patType{\pctx}{p'}{[\arec{t}{\tau'}/t]\tau'}$ \BY{IH, part 2 on \pfref{cvalidP}} \pflabel{patType}
%         \item $\patType{\pctx}{\aefoldp{p'}}{\arec{t}{\tau'}}$ \BY{Rule (\ref{rule:patType-fold}) on \pfref{patType}}
%       \end{pfsteps*}
%       \resetpfcounter
%     \item[\text{(\ref{rule:cvalidP-UP-tpl})}] ~
%       \begin{pfsteps*}
%         \item $\cpv=\acetplp{\labelset}{\mapschema{\cpv}{i}{\labelset}}$ \BY{assumption}
%         \item $p=\aetplp{\labelset}{\mapschema{p}{i}{\labelset}}$ \BY{assumption}
%         \item $\tau=\aprod{\labelset}{\mapschema{\tau}{i}{\labelset}}$ \BY{assumption}
%         \item $\{\cvalidP{\uGG{\uG_i}{\pctx_i}}{\pscene{\uDelta}{\uPhi}{b}}{\cpv_i}{p_i}{\tau_i}\}_{i \in \labelset}$ \BY{assumption} \pflabel{cvalidP}
%         \item $\pctx = \Gconsi{i \in \labelset}{\pctx_i}$ \BY{assumption}
%         %\item $\uptsmenv{\Delta}{\Phi}$ \BY{assumption} \pflabel{env}
%         \item $\{\patType{\pctx_i}{p_i}{\tau_i}\}_{i \in \labelset}$ \BY{IH, part 2 over \pfref{cvalidP}}\pflabel{patType}
%         \item $\patType{\Gconsi{i \in \labelset}{\pctx_i}}{\aetplp{\labelset}{\mapschema{p}{i}{\labelset}}}{\aprod{\labelset}{\mapschema{\tau}{i}{\labelset}}}$ \BY{Rule (\ref{rule:patType-tpl}) on \pfref{patType}}
%       \end{pfsteps*}
%       \resetpfcounter
%     \item[\text{(\ref{rule:cvalidP-UP-in})}] ~
%       \begin{pfsteps*}
%         \item $\cpv=\aceinjp{\ell}{\cpv'}$ \BY{assumption}
%         \item $p=\aeinjp{\ell}{p'}$ \BY{assumption}
%         \item $\tau=\asum{\labelset, \ell}{\mapschema{\tau}{i}{\labelset}; \mapitem{\ell}{\tau'}}$ \BY{assumption}
%         \item $\cvalidP{\upctx}{\pscene{\uDelta}{\uPhi}{b}}{\cpv'}{p'}{\tau'}$ \BY{assumption} \pflabel{cvalidP}
% %        \item $\uptsmenv{\Delta}{\Phi}$ \BY{assumption} \pflabel{env}
%         \item $\patType{\pctx}{p'}{\tau'}$ \BY{IH, part 2 on \pfref{cvalidP}} \pflabel{patType}
%         \item $\patType{\pctx}{\aeinjp{\ell}{p'}}{\asum{\labelset, \ell}{\mapschema{\tau}{i}{\labelset}; \mapitem{\ell}{\tau'}}}$ \BY{Rule (\ref{rule:patType-inj}) on \pfref{patType}}
%       \end{pfsteps*}
%       \resetpfcounter
%     \item[\text{(\ref{rule:cvalidP-UP-spliced})}] ~
%       \begin{pfsteps*}
%         \item $\cpv=\acesplicedp{m}{n}{\ctau}$ \BY{assumption}
%         \item $\cvalidT{\emptyset}{\tsceneUP{\uDelta}{b}}{\ctau}{\tau}$ \BY{assumption}
%         \item $\parseUExp{\bsubseq{b}{m}{n}}{\upv}$ \BY{assumption}
%         \item $\patExpands{\upctx}{\uPhi}{\upv}{p}{\tau}$ \BY{assumption} \pflabel{patExpands}
%         \item $\patType{\pctx}{p}{\tau}$ \BY{IH, part 1 on \pfref{patExpands}}
%       \end{pfsteps*}
%       \resetpfcounter
%   \end{byCases}
  \end{enumerate}
The mutual induction can be shown to be well-founded by an argument nearly identical to that that given in the proof of Theorem \ref{thm:typed-pattern-expansion}, differing only in that the appeal to Condition \ref{condition:pattern-parsing} is replaced by an appeal to the analagous Condition \ref{condition:pattern-parsing-BS}.

% showing that the following numeric metric on the judgements that we induct on is decreasing:
% \begin{align*}
% \sizeof{\patExpands{\upctx}{\uPhi}{\upv}{p}{\tau}} & = \sizeof{\upv}\\
% \sizeof{{\cvalidP{\upctx}{\pscene{\uDelta}{\uPhi}{b}}{\cpv}{p}{\tau}}} & = \sizeof{b}
% \end{align*}
% where $\sizeof{b}$ is the length of $b$ and $\sizeof{\upv}$ is the sum of the lengths of the literal bodies in $\upv$, as defined in Sec. \ref{appendix:SES-syntax}.

% The only case in the proof of part 1 that invokes part 2 is Case (\ref{rule:patExpands-apuptsm}). There, we have that the metric remains stable: \begin{align*}
%  & \sizeof{\patExpands{\upctx}{\uPhi}{\utsmap{\tsmv}{b}}{p}{\tau}}\\
% =& \sizeof{{\cvalidP{\upctx}{\pscene{\uDelta}{\uPhi}{b}}{\cpv}{p}{\tau}}}\\
% =&\sizeof{b}\end{align*}

% The only case in the proof of part 2 that invokes part 1 is Case (\ref{rule:cvalidP-UP-spliced}). There, we have that $\parseUPat{\bsubseq{b}{m}{n}}{\upv}$ and the IH is applied to the judgement $\patExpands{\upctx}{\uPhi}{\upv}{p}{\tau}$. Because the metric is stable when passing from part 1 to part 2, we must have that it is strictly decreasing in the other direction:
% \[\sizeof{\patExpands{\upctx}{\uPhi}{\upv}{p}{\tau}} < \sizeof{{\cvalidP{\upctx}{\pscene{\uDelta}{\uPhi}{b}}{\acesplicedp{m}{n}{\ctau}}{p}{\tau}}}\]
% i.e. by the definitions above, 
% \[\sizeof{\upv} < \sizeof{b}\]

% This is established by appeal to Condition \ref{condition:body-subsequences}, which states that subsequences of $b$ are no longer than $b$, and the Condition \ref{condition:pattern-parsing}, which states that an unexpanded pattern constructed by parsing a textual sequence $b$ is strictly smaller, as measured by the metric defined above, than the length of $b$, because some characters must necessarily be used to apply the pattern TSM and delimit each literal body. Combining Conditions \ref{condition:body-subsequences} and \ref{condition:pattern-parsing}, we have that $\sizeof{\upv} < \sizeof{b}$ as needed.
\end{proof}
\subsection{Typed Expression Expansion}\label{appendix:SES-typed-expression-expansion}
\begin{theorem}[Typed Expansion (Full)]\label{thm:typed-expansion-full-U} ~
\begin{enumerate}
  \item \begin{enumerate}
    \item If $\esyn{\uDD{\uD}{\Delta}}{\uGG{\uG}{\Gamma}}{\uPsi}{\uPhi}{\ue}{e}{\tau}$ then $\hastypeU{\Delta}{\Gamma}{e}{\tau}$.
    \item If $\eana{\uDD{\uD}{\Delta}}{\uGG{\uG}{\Gamma}}{\uPsi}{\uPhi}{\ue}{e}{\tau}$ and $\istypeU{\Delta}{\tau}$ then $\hastypeU{\Delta}{\Gamma}{e}{\tau}$.
    \item If $\rana{\uDD{\uD}{\Delta}}{\uGG{\uG}{\Gamma}}{\uPsi}{\uPhi}{\urv}{r}{\tau}{\tau'}$ and $\istypeU{\Delta}{\tau'}$ then $\ruleType{\Delta}{\Gamma}{r}{\tau}{\tau'}$.
  \end{enumerate}
  \item \begin{enumerate}
    \item If $\csyn{\Delta}{\Gamma}{\esceneUP{\uDD{\uD}{\Delta_\text{app}}}{\uGG{\uG}{\Gamma_\text{app}}}{\uPsi}{\uPhi}{b}}{\ce}{e}{\tau}$ and $\Delta \cap \Delta_\text{app}=\emptyset$ and $\domof{\Gamma} \cap \domof{\Gamma_\text{app}}=\emptyset$ then $\hastypeU{\Dcons{\Delta}{\Delta_\text{app}}}{\Gcons{\Gamma}{\Gamma_\text{app}}}{e}{\tau}$. 
    \item If $\cana{\Delta}{\Gamma}{\esceneUP{\uDD{\uD}{\Delta_\text{app}}}{\uGG{\uG}{\Gamma_\text{app}}}{\uPsi}{\uPhi}{b}}{\ce}{e}{\tau}$ and $\istypeU{\Delta}{\tau}$ and $\Delta \cap \Delta_\text{app}=\emptyset$ and $\domof{\Gamma} \cap \domof{\Gamma_\text{app}}=\emptyset$ then $\hastypeU{\Dcons{\Delta}{\Delta_\text{app}}}{\Gcons{\Gamma}{\Gamma_\text{app}}}{e}{\tau}$. 
    \item If $\crana{\Delta}{\Gamma}{\esceneUP{\uDD{\uD}{\Delta_\text{app}}}{\uGG{\uG}{\Gamma_\text{app}}}{\uPsi}{\uPhi}{b}}{\crv}{r}{\tau}{\tau'}$ and $\istypeU{\Delta}{\tau'}$ and $\Delta \cap \Delta_\text{app}=\emptyset$ and $\domof{\Gamma} \cap \domof{\Gamma_\text{app}}=\emptyset$ then $\ruleType{\Dcons{\Delta}{\Delta_\text{app}}}{\Gcons{\Gamma}{\Gamma_\text{app}}}{r}{\tau}{\tau'}$.
  \end{enumerate}
\end{enumerate}
\end{theorem}
\begin{proof}
By mutual rule induction over Rules (\ref{rules:esyn-S}), Rules (\ref{rules:eana-S}), Rule (\ref{rule:rana-S}), Rules (\ref{rules:csyn}), Rules (\ref{rules:cana}) and Rule (\ref{rule:crana}).

\begin{enumerate}
\item In the following, let $\uDelta=\uDD{\uD}{\Delta}$ and $\uGamma=\uGG{\uG}{\Gamma}$.
  \begin{enumerate}
    \item \todo{1a}
    \item \todo{1b}
    \item \todo{1c}
  \end{enumerate}
\item \todo{2}
  \begin{enumerate}
    \item \todo{2a}
    \item \todo{2b}
    \item \todo{2c}
  \end{enumerate}
\end{enumerate}
\end{proof}

% % \subsection{Expressibility}
% The following lemma establishes that each type can be expressed as a well-formed proto-type, under the same type formation context and any type splicing scene.
% \begin{lemma}[Proto-Expansion Type Expressibility]\label{lemma:proto-type-expressibility-U} If $\istypeU{\Delta}{\tau}$ then $\cvalidT{\Delta}{\tscenev}{\Cof{\tau}}{\tau}$. \end{lemma}
% \begin{proof}
% By rule induction over Rules (\ref{rules:istypeU}). In each case, we apply the IH on or over each premise, then apply the corresponding proto-type validation rule in Rules (\ref{rules:cvalidT-U}).
% \end{proof}

% The Type Expressibility Lemma establishes that every well-formed type, $\tau$, can be expressed as a well-formed unexpanded type, $\Uof{\tau}$. This requires defining the metafunction $\Uof{\Delta}$ which maps $\Delta$ onto an unexpanded type formation context as follows:
% \begin{align*}
% \Uof{\emptyset} &= \uDD{\emptyset}{\emptyset}\\
% \Uof{\Delta, \Dhyp{t}} &= \Uof{\Delta}, \uDhyp{\sigilof{t}}{t}
% \end{align*}
% \begin{lemma}[Type Expressibility]\label{lemma:type-expressibility} If $\istypeU{\Delta}{\tau}$ then $\expandsTU{\Uof{\Delta}}{\Uof{\tau}}{\tau}$.\end{lemma}
% \begin{proof} By rule induction over Rules (\ref{rules:istypeU}) using the definitions of $\Uof{\tau}$ and $\Uof{\Delta}$ above. In each case, we apply the IH to or over each premise, then apply the corresponding type expansion rule in Rules (\ref{rules:expandsTU}).\end{proof}


% The following lemma establishes that each well-typed expanded expression, $e$, can be expressed as a valid proto-expression, $\Cof{e}$, that is assigned the same type under any expression splicing scene.
% \begin{theorem}[Proto-Expansion Expression Expressibility]\label{theorem:proto-expressions-expressibility-U} If $\hastypeU{\Delta}{\Gamma}{e}{\tau}$ then $\cvalidE{\Delta}{\Gamma}{\escenev}{\Cof{e}}{e}{\tau}$.\end{theorem}
% \begin{proof} By rule induction over Rules (\ref{rules:hastypeU}). The rule transformation above guarantees that this lemma holds by construction. In particular, in each case, we apply Lemma \ref{lemma:proto-type-expressibility-U} to or over each type formation premise, the IH to or over each typing premise, then apply the corresponding proto-expression validation rule in Rules (\ref{rule:cvalidE-U-var}) through (\ref{rule:cvalidE-U-case}).
% \end{proof}

% The following lemma establishes that each well-typed expanded expression, $e$, can be expressed as a valid ce-expression, $\Cof{e}$, that is assigned the same type under any expression splicing scene.
% \begin{theorem}[Candidate Expansion Expression Expressibility]\label{lemma:ce-expressions-expressibility-UP} Both of the following hold:
% \begin{enumerate}
% \item If $\hastypeU{\Delta}{\Gamma}{e}{\tau}$ then $\cvalidE{\Delta}{\Gamma}{\escenev}{\Cof{e}}{e}{\tau}$.
% \item If $\ruleType{\Delta}{\Gamma}{r}{\tau}{\tau'}$ then $\cvalidR{\Delta}{\Gamma}{\escenev}{\Cof{r}}{r}{\tau}{\tau'}$.
% \end{enumerate}
% \end{theorem}
% \begin{proof} By mutual rule induction over Rules (\ref{rules:hastypeUP}) and Rule (\ref{rule:ruleType}). 

% For part 1, we induct on the assumption. 
% \begin{byCases}
% \item[\text{(\ref{rule:hastypeUP-var}) through (\ref{rule:hastypeUP-in})}] In each of these cases, we apply Lemma \ref{lemma:ce-type-expressibility-U} to or over each type formation premise, the IH (part 1) to or over each typing premise, then apply the corresponding ce-expression validation rule in Rules (\ref{rule:cvalidE-UP-var}) through (\ref{rule:cvalidE-UP-in}).
% \item[\text{(\ref{rule:hastypeUP-match})}] ~
%   \begin{pfsteps}
%   \item e = \aematchwith{n}{e'}{\seqschemaX{r}} \BY{assumption}
%   \item \Cof{e} = \acematchwith{n}{\Cof{\tau}}{\Cof{e'}}{\seqschemaXx{\Cofv}{r}} \BY{definition of $\Cof{e}$}
%   \item \hastypeU{\Delta}{\Gamma}{e'}{\tau'} \BY{assumption} \pflabel{hasType}
%   \item \istypeU{\Delta}{\tau} \BY{assumption} \pflabel{isType}
%   \item \{\ruleType{\Delta}{\Gamma}{r_i}{\tau'}{\tau}\}_{1 \leq i \leq n} \BY{assumption} \pflabel{ruleType}
%   \item \cvalidE{\Delta}{\Gamma}{\escenev}{\Cof{e'}}{e'}{\tau'} \BY{IH, part 1 on \pfref{hasType}} \pflabel{cvalidE}
%   \item \cvalidT{\Delta}{\tsfrom{\escenev}}{\Cof{\tau}}{\tau} \BY{Lemma \ref{lemma:candidate-expansion-type-validation} on \pfref{isType}} \pflabel{cvalidT}
%   \item \{\cvalidR{\Delta}{\Gamma}{\escenev}{\Cof{r_i}}{r_i}{\tau'}{\tau}\}_{1 \leq i \leq n} \BY{IH, part 2 over \pfref{ruleType}} \pflabel{cvalidR}
%   \item \cvalidE{\Delta}{\Gamma}{\escenev}{\acematchwith{n}{\Cof{\tau}}{\Cof{e'}}{\seqschemaXx{\Cofv}{r}}}{\aematchwith{n}{e'}{\seqschemaX{r}}}{\tau} \BY{Rule (\ref{rule:cvalidE-UP-match}) on \pfref{cvalidE}, \pfref{cvalidT} and \pfref{cvalidR}}
%   \end{pfsteps}
% \end{byCases}
% \resetpfcounter

% For part 2, we induct on the assumption. There is only one case.
% \begin{byCases}
% \item[\text{(\ref{rule:ruleType})}] ~
%   \begin{pfsteps}
%     \item r = \aematchrule{p}{e} \BY{assumption}
%     \item \Cof{r} = \acematchrule{p}{\Cof{e}} \BY{definition of $\Cof{r}$}
%     \item \patType{\pctx}{p}{\tau} \BY{assumption} \pflabel{patType}
%     \item \hastypeU{\Delta}{\Gcons{\Gamma}{\pctx}}{e}{\tau'} \BY{assumption} \pflabel{hasType}
%     \item \cvalidE{\Delta}{\Gcons{\Gamma}{\pctx}}{\escenev}{\Cof{e}}{e}{\tau'} \BY{IH, part 1 on \pfref{hasType}} \pflabel{cvalidE}
%     \item \cvalidR{\Delta}{\Gamma}{\escenev}{\acematchrule{p}{\Cof{e}}}{\aematchrule{p}{e}}{\tau}{\tau'} \BY{Rule (\ref{rule:cvalidR-UP}) on \pfref{patType} and \pfref{cvalidE}}
%   \end{pfsteps}
%   \resetpfcounter
% \end{byCases}
% \end{proof}

% The following lemma establishes that every well-typed expanded pattern that generates no hypotheses can be expressed as a ce-pattern.
% \begin{lemma}[Candidate Expansion Pattern Expressibility]\label{lemma:ce-pattern-expressibility-U} If $\patType{\emptyset}{p}{\tau}$ then $\cvalidP{\uGG{\emptyset}{\emptyset}}{\pscene{\uDelta}{\uPhi}{b}}{\Cof{p}}{p}{\tau}$.\end{lemma}
% \begin{proof} By rule induction over Rules (\ref{rules:patType}).
% \begin{byCases}
% \item[\text{(\ref{rule:patType-var})}] This case does not apply.
% \item[\text{(\ref{rule:patType-wild})}] ~
%   \begin{pfsteps*}
%     \item $p=\aewildp$ \BY{assumption}
%     \item $\Cof{p}=\acewildp$ \BY{definition of $\Cof{p}$}
%     \item $\cvalidP{\uGG{\emptyset}{\emptyset}}{\pscene{\uDelta}{\uPhi}{b}}{\acewildp}{\aewildp}{\tau}$ \BY{Rule (\ref{rule:cvalidP-UP-wild})}
%   \end{pfsteps*}
%   \resetpfcounter
% \item[\text{(\ref{rule:patType-fold})}] ~
%   \begin{pfsteps*}
%     \item $p=\aefoldp{p'}$ \BY{assumption}
%     \item $\Cof{p}=\acefoldp{\Cof{p'}}$ \BY{definition of $\Cof{p}$}
%     \item $\tau=\arec{t}{\tau'}$ \BY{assumption}
%     \item $\patType{\emptyset}{p'}{[\arec{t}{\tau'}/t]\tau'}$ \BY{assumption} \pflabel{patType}
%     \item $\cvalidP{\uGG{\emptyset}{\emptyset}}{\pscene{\uDelta}{\uPhi}{b}}{\Cof{p'}}{p}{[\arec{t}{\tau'}/t]\tau'}$ \BY{IH on \pfref{patType}} \pflabel{cvalidP}
%     \item $\cvalidP{\uGG{\emptyset}{\emptyset}}{\pscene{\uDelta}{\uPhi}{b}}{\acefoldp{\Cof{p'}}}{\aefoldp{p'}}{\arec{t}{\tau'}}$ \BY{Rule (\ref{rule:cvalidP-UP-fold}) on \pfref{cvalidP}}
%   \end{pfsteps*}
%   \resetpfcounter
% \item[\text{(\ref{rule:patType-tpl})}] ~
%   \begin{pfsteps*}
%     \item $p=\aetplp{\labelset}{\mapschema{p}{i}{\labelset}}$ \BY{assumption}
%     \item $\Cof{p}=\acetpl{\labelset}{\mapschemax{\Cofv}{p}{i}{\labelset}}$ \BY{definition of $\Cof{p}$}
%     \item $\tau=\aprod{\labelset}{\mapschema{\tau}{i}{\labelset}}$ \BY{assumption}
%     \item $\{\patType{\emptyset}{p_i}{\tau_i}\}_{i \in \labelset}$ \BY{assumption} \pflabel{patType}
%     \item $\{\cvalidP{\uGG{\emptyset}{\emptyset}}{\pscene{\uDelta}{\uPhi}{b}}{\Cof{p_i}}{p_i}{\tau_i}\}_{i \in \labelset}$ \BY{IH over \pfref{patType}} \pflabel{cvalidP}
%     \item $\cvalidP{\uGG{\emptyset}{\emptyset}}{\pscene{\uDelta}{\uPhi}{b}}{\acetpl{\labelset}{\mapschemax{\Cofv}{p}{i}{\labelset}}}{\aetplp{\labelset}{\mapschema{p}{i}{\labelset}}}{\aprod{\labelset}{\mapschema{\tau}{i}{\labelset}}}$ \BY{Rule (\ref{rule:cvalidP-UP-tpl}) on \pfref{cvalidP}}
%   \end{pfsteps*}
%   \resetpfcounter
% \item[\text{(\ref{rule:patType-inj})}] ~
%   \begin{pfsteps*}
%     \item $p=\aeinjp{\ell}{p'}$ \BY{assumption}
%     \item $\Cof{p}=\aceinjp{\ell}{\Cof{p'}}$ \BY{definition of $\Cof{p}$}
%     \item $\tau=\asum{\labelset, \ell}{\mapschema{\tau}{i}{\labelset}; \mapitem{\ell}{\tau'}}$ \BY{assumption}
%     \item $\patType{\emptyset}{p'}{\tau'}$ \BY{assumption}\pflabel{patType}
%     \item $\cvalidP{\uGG{\emptyset}{\emptyset}}{\pscene{\uDelta}{\uPhi}{b}}{\Cof{p'}}{p'}{\tau'}$ \BY{IH on \pfref{patType}}\pflabel{cvalidP}
%     \item $\cvalidP{\uGG{\emptyset}{\emptyset}}{\pscene{\uDelta}{\uPhi}{b}}{\aceinjp{\ell}{\Cof{p'}}}{\aeinjp{\ell}{p'}}{\asum{\labelset, \ell}{\mapschema{\tau}{i}{\labelset}; \mapitem{\ell}{\tau'}}}$ \BY{Rule (\ref{rule:cvalidP-UP-in}) on \pfref{cvalidP}}
%   \end{pfsteps*}
%   \resetpfcounter
% \end{byCases}
% \end{proof}

% \subsubsection{Expressibility}
% The following lemma establishes that each well-typed expanded pattern can be expressed as an unexpanded pattern matching values of the same type and generating the same hypotheses and corresponding identifier updates. The metafunction $\Uof{\pctx}$ maps $\pctx$ to an unexpanded typing context as follows:
% \begin{align*}
% \Uof{\emptyset} & = \uGG{\emptyset}{\emptyset}\\
% \Uof{\pctx, x : \tau} & = \Uof{\pctx}, \uGhyp{\sigilof{x}}{x}{\tau}\\
% \Uof{\Gconsi{i \in \labelset}{\pctx_i}} & = \Gconsi{i \in \labelset}{\Uof{\pctx_i}}
% \end{align*}
% \begin{lemma}[Pattern Expressibility]\label{lemma:pattern-expressibility} If $\patType{\pctx}{p}{\tau}$ then $\patExpands{\Uof{\pctx}}{\uPhi}{\Uof{p}}{p}{\tau}$.\end{lemma}
% \begin{proof} By rule induction over Rules (\ref{rules:patType}), using the definitions of $\Uof{\pctx}$ and $\Uof{p}$ given above. In each case, we can apply the IH to or over each premise, then apply the corresponding rule in Rules (\ref{rules:patExpands}).\end{proof}

% We can now establish the Expressibility Theorem -- that each well-typed expanded expression, $e$, can be expressed as an unexpanded expression, $\ue$, and assigned the same type under the corresponding contexts.

% \begin{theorem}[Expressibility] Both of the following hold:
% \begin{enumerate}
% \item If $\hastypeU{\Delta}{\Gamma}{e}{\tau}$ then $\expandsUP{\Uof{\Delta}}{\Uof{\Gamma}}{\uPsi}{\uPhi}{\Uof{e}}{e}{\tau}$.
% \item If $\ruleType{\Delta}{\Gamma}{r}{\tau}{\tau'}$ then $\ruleExpands{\Uof{\Delta}}{\Uof{\Gamma}}{\uPsi}{\uPhi}{\Uof{r}}{r}{\tau}{\tau'}$.
% \end{enumerate}
% \end{theorem}
% \begin{proof} By mutual rule induction over Rules (\ref{rules:hastypeUP}) and Rule (\ref{rule:ruleType}). 

% For part 1, we induct on the assumption. The rule transformation defined above guarantees that this part holds by its construction. In particular, in each case, we can apply Lemma \ref{lemma:type-expressibility} to or over each type formation premise, the IH (part 1) to or over each typing premise, the IH (part 2) over each rule typing premise, then apply the corresponding rule in Rules (\ref{rules:expandsUP}).

% For part 2, we induct on the assumption. There is only one case:
% \begin{byCases}
% \item[(\ref{rule:ruleType})] ~
% \begin{pfsteps*}
% \item $r = \aematchrule{p}{e}$ \BY{assumption}
% \item $\patType{\pctx}{p}{\tau}$ \BY{assumption} \pflabel{patType}
% \item $\hastypeU{\Delta}{\Gamma \cup \pctx}{e}{\tau'}$ \BY{assumption} \pflabel{hasType}
% \item $\Uof{\Gamma}=\uGG{\uG}{\Gamma}$, for some $\uG$ \BY{definition of $\Uof{\Gamma}$}
% \item $\Uof{\pctx} =\uGG{\uG'}{\pctx}$, for some $\uG'$ \BY{definition of $\Uof{\pctx}$}
% \item $\Uof{\Gamma \cup \pctx} = \uGG{\uG \uplus \uG'}{\Gamma \cup \pctx}$ \BY{definition of $\Uof{\pctx}$}
% \item $\Uof{r} = \aumatchrule{\Uof{p}}{\Uof{e}}$ \BY{definition of $\Uof{r}$}
% \item $\patExpands{\uGG{\uG'}{\pctx}}{\uPhi}{\Uof{p}}{p}{\tau}$ \BY{Lemma \ref{lemma:pattern-expressibility} on \pfref{patType}} \pflabel{patExpands}
% \item $\expandsUP{\uDelta}{\uGG{\uGcons{\uG}{\uG'}}{\Gcons{\Gamma}{\pctx}}}{\uPsi}{\uPhi}{\Uof{e}}{e}{\tau'}$ \BY{IH, part 1 on \pfref{hasType}} \pflabel{expandsUP}
% \item $\ruleExpands{\Uof{\Delta}}{\uGG{\uG}{\Gamma}}{\uPsi}{\uPhi}{\aumatchrule{\Uof{p}}{\Uof{e}}}{\aematchrule{p}{e}}{\tau}{\tau'}$ \BY{Rule (\ref{rule:ruleExpands}) on \pfref{patExpands} and \pfref{expandsUP}}
% \end{pfsteps*}
% \resetpfcounter
% \end{byCases}
% \end{proof}

\subsection{Reasoning Principles}
The following theorem, together with Theorem \ref{thm:typed-expansion-short-U}, establishes \textbf{Typing}, \textbf{Segmentation} and \textbf{Context Independence} as discussed in Sec. \ref{sec:uetsms-validation}.

\begin{theorem}[Typing, Segmentation and Context Independence]
\label{thm:tsc-B}
If $\expandsSG{\uDelta}{\uGamma}{\uPsi}{\uPhi}{\utsmap{\tsmv}{b}}{e}{\tau}$ then:
\begin{enumerate}
\item (\textbf{Typing}) $\uPsi = \uPsi', \uShyp{\tsmv}{a}{\tau}{\eparse}$
\item $\encodeBody{b}{\ebody}$
\item $\evalU{\ap{\eparse}{\ebody}}{\lbltxt{SuccessE}\cdot\ecand}$
\item $\decodeCondE{\ecand}{\ce}$
\item (\textbf{Segmentation}) $\segOK{\segof{\ce}}{b}$
\item (\textbf{Context Independence}) $\cvalidE{\emptyset}{\emptyset}{\esceneSG{\uDelta}{\uGamma}{\uPsi}{\uPhi}{b}}{\ce}{e}{\tau}$ 
\end{enumerate}
\end{theorem}
\begin{proof} By rule induction over Rules (\ref{rules:expandsU}). The only rule that applies is Rule (\ref{rule:expandsU-tsmap}). The conclusions of the theorem are the premises of this rule.
\end{proof}

The following theorem establishes a prohibition on \textbf{Shadowing} as discussed in Sec. \ref{sec:uetsms-validation}.

\begin{theorem}[Shadowing Prohibition]
\label{thm:shadowing-prohibition-SES} ~
\begin{enumerate}
\item If $\cvalidT{\Delta}{\tsceneU{\uDD{\uD}{\Delta_\text{app}}}{b}}{\acesplicedt{m}{n}}{\tau}$ then:\begin{enumerate}
\item $\parseUTyp{\bsubseq{b}{m}{n}}{\utau}$
\item $\expandsTU{\uDD{\uD}{\Delta_\text{app}}}{\utau}{\tau}$
\item $\Delta \cap \Delta_\text{app} = \emptyset$
\end{enumerate}
\item If $\cvalidE{\Delta}{\Gamma}{\escenev}{\acesplicede{m}{n}{\ctau}}{e}{\tau}$ then:
\begin{enumerate}
\item $\cvalidT{\emptyset}{\tsfrom{\escenev}}{\ctau}{\tau}$
\item $  \escenev=\esceneU{\uDD{\uD}{\Delta_\text{app}}}{\uGG{\uG}{\Gamma_\text{app}}}{\uPsi}{b}$
\item $\parseUExp{\bsubseq{b}{m}{n}}{\ue}$
\item $\expandsU{\uDD{\uD}{\Delta_\text{app}}}{\uGG{\uG}{\Gamma_\text{app}}}{\uPsi}{\ue}{e}{\tau}$
\item $\Delta \cap \Delta_\text{app} = \emptyset$
\item $\domof{\Gamma} \cap \domof{\Gamma_\text{app}} = \emptyset$
\end{enumerate}
\end{enumerate}
\end{theorem}
\begin{proof} ~
\begin{enumerate}
\item By rule induction over Rules (\ref{rules:cvalidT-U}). The only rule that applies is Rule (\ref{rule:cvalidT-U-splicedt}). The conclusions are the premises of tihs rule.
\item By rule induction over Rules (\ref{rules:cvalidE-U}). The only rule that applies is Rule (\ref{rule:cvalidE-U-splicede}). The conclusions are the premises of tihs rule.
\end{enumerate}
\end{proof}

\begin{grayparbox}
The following theorem, together with Theorem \ref{thm:typed-pattern-expansion} part 1, establishes \textbf{Typing} and \textbf{Segmentation}, as discussed in Sec. \ref{sec:ptsms-validation}.

\begin{theorem}[spTSM Typing and Segmentation]
\label{thm:spTSM-Typing-Segmentation-B}
If $\patExpands{\upctx}{\uPhi}{\utsmap{\tsmv}{b}}{p}{\tau}$ then 
\begin{enumerate}
        \item (\textbf{Typing}) $\uPhi=\uPhi', \uPhyp{\tsmv}{a}{\tau}{\eparse}$
        \item $\encodeBody{b}{\ebody}$
        \item $\evalU{\eparse(\ebody)}{{\lbltxt{SuccessP}}\cdot{\ecand}}$
        \item $\decodeCEPat{\ecand}{\cpv}$
        \item (\textbf{Segmentation}) $\segOK{\segof{\cpv}}{b}$
\end{enumerate}
\end{theorem}
\begin{proof} By rule induction over Rules (\ref{rules:patExpands}). The only rule that applies is Rule (\ref{rule:patExpands-apuptsm}). The conclusions are premises of this rule.
\end{proof}
\end{grayparbox}

% \subsubsection{Candidate Expansion Expressibility}
% The following lemma establishes that each well-typed expanded expression, $e$, can be expressed as a valid ce-expression, $\Cof{e}$, that synthesizes the same type under the same contexts and any expression splicing scene.
% \begin{theorem}[Candidate Expansion Expression Expressibility]\label{lemma:ce-expressions-expressibility-B} Both of the following hold:
% \begin{enumerate}
% \item If $\hastypeU{\Delta}{\Gamma}{e}{\tau}$ then $\csyn{\Delta}{\Gamma}{\escenev}{\Cof{e}}{e}{\tau}$.
% \item If $\ruleType{\Delta}{\Gamma}{r}{\tau}{\tau'}$ then $\crsyn{\Delta}{\Gamma}{\escenev}{\Cof{r}}{r}{\tau}{\tau'}$.
% \end{enumerate}
% \end{theorem}
% \begin{proof} By mutual rule induction over Rules (\ref{rules:hastypeUP}) and Rule (\ref{rule:ruleType}). In each case, we apply the IH, part 1 to or over each typing premise, the IH, part 2 over each rule typing premise, Lemma \ref{lemma:ce-type-expressibility-U} to or over each type formation premise and then derive the conclusion by applying Rules (\ref{rules:csyn}) and Rule (\ref{rule:crsyn}) as needed.
% \end{proof}

% The following lemma establishes that every well-typed expanded pattern that generates no hypotheses can be expressed as a ce-pattern.
% \begin{lemma}[Candidate Expansion Pattern Expressibility]\label{lemma:ce-pattern-expressibility-B} If $\patType{\emptyset}{p}{\tau}$ then $\cvalidP{\uGG{\emptyset}{\emptyset}}{\pscene{\Delta}{\uPhi}{b}}{\Cof{p}}{p}{\tau}$.\end{lemma}
% \begin{proof} The proof is nearly identical to the proof of Lemma \ref{lemma:ce-pattern-expressibility-U}, differing only in that each mention of a rule in Rules (\ref{rules:cvalidP-UP}) is replaced by a mention of the corresponding rule in Rules (\ref{rules:cvalidP-B}).
% \end{proof}

% \subsubsection{Outer Surface Expressibility}
% The following lemma establishes that each well-typed expanded pattern can be expressed as an unexpanded pattern matching values of the same type and generating the same hypotheses and corresponding sigil updates. The metafunction $\Uof{\pctx}$ was defined in \ref{sec:typed-expansion-UP}.
% \begin{lemma}[Pattern Expressibility]\label{lemma:pattern-expressibility-B} If $\patType{\pctx}{p}{\tau}$ then $\patExpands{\Uof{\pctx}}{\uPhi}{\Uof{p}}{p}{\tau}$.\end{lemma}
% \begin{proof} By rule induction over Rules (\ref{rules:patType}), using the definitions of $\Uof{\pctx}$ and $\Uof{p}$. In each case, we can apply the IH to or over each premise, then apply the corresponding rule in Rules (\ref{rules:patExpands-B}).\end{proof}

% We can now establish the Expressibility Theorem -- that each well-typed expanded expression, $e$, can be expressed as an unexpanded expression, $\ue$, which synthesizes the same type under the corresponding contexts.

% \begin{theorem}[Expressibility] Both of the following hold:
% \begin{enumerate}
% \item If $\hastypeU{\Delta}{\Gamma}{e}{\tau}$ then $\esyn{\Uof{\Delta}}{\Uof{\Gamma}}{\uPsi}{\uPhi}{\Uof{e}}{e}{\tau}$.
% \item If $\ruleType{\Delta}{\Gamma}{r}{\tau}{\tau'}$ then $\rsyn{\Uof{\Delta}}{\Uof{\Gamma}}{\uPsi}{\uPhi}{\Uof{r}}{r}{\tau}{\tau'}$.
% \end{enumerate}
% \end{theorem}
% \begin{proof} By mutual rule induction over Rules (\ref{rules:hastypeUP}) and Rule (\ref{rule:ruleType}) using the definitions of $\Uof{\Delta}$, $\Uof{\Gamma}$, $\Uof{e}$ and $\Uof{r}$. In each case, we apply the IH, part 1 to or over each typing premise, the IH, part 2 over each rule typing premise, Lemma \ref{lemma:type-expressibility} to or over each type formation premise, Lemma \ref{lemma:pattern-expressibility-B} to each pattern typing premise, then derive the conclusion by applying Rules (\ref{rules:esyn}) and Rule (\ref{rule:rsyn}).  
% \end{proof} 

% \chapter{Parametric Implicits}
% ...

% \subsubsection{Kinds and Constructors}
% Kind expansion

% \begin{subequations}\label{rules:kExpands}
% \begin{equation}\label{rule:kExpands-darr}
% \inferrule{
%   \kExpandsX{\ukappa_1}{\kappa_1}\\
%   \kExpands{\uOmega, \uKhyp{\uu}{u}{\kappa_1}}{\ukappa_2}{\kappa_2}
% }{
%   \kExpandsX{\kdarr{\uu}{\ukappa_1}{\ukappa_2}}{\akdarr{\kappa_1}{u}{\kappa_2}}
% }
% \end{equation}
% \begin{equation}\label{rule:kExpands-unit}
% \inferrule{ }{
%   \kExpandsX{\kunit}{\akunit}
% }
% \end{equation}
% \begin{equation}\label{rule:kExpands-dprod}
% \inferrule{
%   \kExpandsX{\ukappa_1}{\kappa_1}\\
%   \kExpands{\uOmega, \uKhyp{\uu}{u}{\kappa_1}}{\ukappa_2}{\kappa_2}
% }{
%   \kExpandsX{\kdbprod{\uu}{\ukappa_1}{\ukappa_2}}{\akdbprod{\kappa_1}{u}{\kappa_2}}
% }
% \end{equation}
% \begin{equation}\label{rule:kExpands-ty}
% \inferrule{ }{
%   \kExpandsX{\kty}{\akty}
% }
% \end{equation}
% \begin{equation}\label{rule:kExpands-sing}
% \inferrule{
%   \kanaX{\utau}{\tau}{\akty}
% }{
%   \kExpandsX{\ksing{\utau}}{\aksing{\tau}}
% }
% \end{equation}
% \end{subequations}

% Synthetic constructor expansion
% \begin{subequations}\label{rules:ksyn}
% \begin{equation}\label{rule:ksyn-var}
% \inferrule{ }{\ksyn{\uOmega, \uKhyp{\uu}{u}{\kappa}}{\uu}{u}{\kappa}}
% \end{equation}
% \begin{equation}\label{rule:ksyn-asc}
% \inferrule{
%   \kExpandsX{\ukappa}{\kappa}\\
%   \kanaX{\uc}{c}{\kappa}
% }{
%   \ksynX{\casc{\uc}{\ukappa}}{c}{\kappa}
% }
% \end{equation}
% \begin{equation}\label{rule:ksyn-app}
% \inferrule{
%   \ksynX{\uc_1}{c_1}{\akdarr{\kappa_2}{u}{\kappa}}\\
%   \kanaX{\uc_2}{c_2}{\kappa_2}
% }{
%   \ksynX{\capp{\uc_1}{\uc_2}}{\acapp{c_1}{c_2}}{[c_1/u]\kappa}
% }
% \end{equation}
% \begin{equation}\label{rule:ksyn-unit}
% \inferrule{ }{
%   \ksynX{\ctriv}{\actriv}{\akunit}
% }
% \end{equation}
% \begin{equation}\label{rule:ksyn-prl}
% \inferrule{
%   \ksynX{\uc}{c}{\akdbprod{\kappa_1}{u}{\kappa_2}}
% }{
%   \ksynX{\cprl{\uc}}{\acprl{c}}{\kappa_1}
% }
% \end{equation}
% \begin{equation}\label{rule:ksyn-prr}
% \inferrule{
%   \ksynX{\uc}{c}{\akdbprod{\kappa_1}{u}{\kappa_2}}
% }{
%   \ksynX{\cprr{\uc}}{\acprr{c}}{[\acprl{c}/u]\kappa_2}
% }
% \end{equation}
% \begin{equation}\label{rule:ksyn-parr}
% \inferrule{
%   \kanaX{\utau_1}{\tau_1}{\akty}\\
%   \kanaX{\utau_2}{\tau_2}{\akty}
% }{
%   \ksynX{\parr{\utau_1}{\utau_2}}{\aparr{\tau_1}{\tau_2}}{\akty}
% }
% \end{equation}
% \begin{equation}\label{rule:ksyn-all}
% \inferrule{
%   \kExpandsX{\ukappa}{\kappa}\\
%   \kana{\uOmega, \uKhyp{\uu}{u}{\kappa}}{\utau}{\tau}{\akty}
% }{
%   \ksynX{\forallu{\uu}{\ukappa}{\utau}}{\aallu{\kappa}{u}{\tau}}{\akty}
% }
% \end{equation}
% \begin{equation}\label{rule:ksyn-rec}
% \inferrule{
%   \kana{\uOmega, \uKhyp{\ut}{t}{\akty}}{\utau}{\tau}{\akty}
% }{
%   \ksynX{\rect{\ut}{\utau}}{\arec{t}{\tau}}{\akty}
% }
% \end{equation}
% \begin{equation}\label{rule:ksyn-prod}
% \inferrule{
%   \{\kanaX{\utau_i}{\tau_i}{\akty}\}_{1 \leq i \leq n}
% }{
%   \ksynX{\prodt{\mapschema{\utau}{i}{\labelset}}}{\aprod{\labelset}{\mapschema{\tau}{i}{\labelset}}}{\akty}
% }
% \end{equation}
% \begin{equation}\label{rule:ksyn-sum}
% \inferrule{
%   \{\kanaX{\utau_i}{\tau_i}{\akty}\}_{1 \leq i \leq n}
% }{
%   \ksynX{\sumt{\labelset}{\mapschema{\utau}{i}{\labelset}}}{\asum{\labelset}{\mapschema{\tau}{i}{\labelset}}}{\akty}
% }
% \end{equation}
% \begin{equation}\label{rule:ksyn-stat}
% \inferrule{ }{
%   \ksyn{\uOmega, \uMhyp{\uX}{X}{\asignature{\kappa}{u}{\tau}}}{\mcon{\uX}}{\amcon{X}}{\kappa}
% }
% \end{equation}
% \end{subequations}

% Analytic constructor expansion
% \begin{subequations}\label{rules:kana}
% \begin{equation}\label{rule:kana-subsume}
% \inferrule{
%   \ksynX{\uc}{c}{\kappa_1}\\
%   \ksubX{\kappa_1}{\kappa_2}
% }{
%   \kanaX{\uc}{c}{\kappa_2}
% }
% \end{equation}
% \begin{equation}\label{rule:kana-sing}
% \inferrule{
%   \kanaX{\uc}{c}{\akty}
% }{
%   \kanaX{\uc}{c}{\aksing{c}}
% }
% \end{equation}
% \begin{equation}\label{rule:kana-abs}
% \inferrule{
%   \kana{\uOmega, \uKhyp{\uu}{u}{\kappa_1}}{\uc_2}{c_2}{\kappa_2}
% }{
%   \kanaX{\cabs{\uu}{\uc_2}}{\acabs{u}{c_2}}{\akdarr{\kappa_1}{u}{\kappa_2}}
% }
% \end{equation}
% \begin{equation}\label{rule:kana-pair}
% \inferrule{
%   \kanaX{\uc_1}{c_1}{\kappa_1}\\
%   \kanaX{\uc_2}{c_2}{[c_1/u]\kappa_2}
% }{
%   \kanaX{\cpair{\uc_1}{\uc_2}}{\acpair{c_1}{c_2}}{\akdbprod{\kappa_1}{u}{\kappa_2}}
% }
% \end{equation}
% \end{subequations}


% \subsubsection{Types, Expressions, Rules and Patterns}
% Type expansion
% \begin{equation}\label{rule:tExpandsP-B}
% \inferrule{
%   \kanaX{\utau}{\tau}{\akty}
% }{
%   \cExpandsX{\utau}{\tau}{\akty}
% }
% \end{equation}

% Synthetic typed expression expansion
% \begin{subequations}\label{rules:esynP}
% \begin{equation}\label{rule:esynP-var}
%   \inferrule{ }{ 
%     \esynP{\uOmega, \uGhyp{\ux}{x}{\tau}}{\uPsi}{\uPhi}{\ux}{x}{\tau}
%   }
% \end{equation}

% %A \emph{type ascription} can be placed on an unexpanded expression to specify the type that it should be analyzed against. The ascribed type is synthesized if type analysis succeeds.
% \begin{equation}\label{rule:esynP-asc}
%   \inferrule{
%     \cExpandsX{\utau}{\tau}{\akty}\\
%     \eanaPX{\ue}{e}{\tau}
%   }{
%     \esynPX{\asc{\ue}{\utau}}{e}{\tau}
%   }
% \end{equation}

% %We define let-binding of a value in synthetic position primitively in $\miniVerseUB$. The following rule governs such bindings in synthetic position.
% \begin{equation}\label{rule:esynP-let}
%   \inferrule{
%     \esynPX{\ue}{e}{\tau}\\
%     \esynP{\uOmega, \uGhyp{\ux}{x}{\tau}}{\uPsi}{\uPhi}{\ue'}{e'}{\tau'}
%   }{
%     \esynPX{\letsyn{\ux}{\ue}{\ue'}}{\aeap{\aelam{\tau}{x}{e'}}{e}}{\tau'}
%   }
% \end{equation}

% %Functions with an argument type annotation can appear in synthetic position.
% \begin{equation}\label{rule:esynP-lam}
%   \inferrule{
%     \cExpandsX{\utau_1}{\tau_1}{\akty}\\
%     \esynP{\uOmega, \uGhyp{\ux}{x}{\tau_1}}{\uPsi}{\uPhi}{\ue}{e}{\tau_2}
%   }{
%     \esynPX{\lam{\ux}{\utau_1}{\ue}}{\aelam{\tau_1}{x}{e}}{\aparr{\tau_1}{\tau_2}}
%   }
% \end{equation}

% %Function applications can appear in synthetic position. The argument is analyzed against the argument type synthesized by the function.
% \begin{equation}\label{rule:esynP-ap}
%   \inferrule{
%     \esynPX{\ue_1}{e_1}{\aparr{\tau_2}{\tau}}\\
%     \eanaPX{\ue_2}{e_2}{\tau_2}
%   }{
%     \esynPX{\ap{\ue_1}{\ue_2}}{\aeap{e_1}{e_2}}{\tau}
%   }
% \end{equation}

% %Type lambdas and type applications can appear in synthetic position.
% \begin{equation}\label{rule:esynP-tlam}
%   \inferrule{
%     \kExpandsX{\ukappa}{\kappa}\\
%     \esynP{\uOmega, \uKhyp{\uu}{u}{\kappa}}{\uPsi}{\uPhi}{\ue}{e}{\tau}
%   }{
%     \esynPX{\clam{\uu}{\ukappa}{\ue}}{\aeclam{\kappa}{u}{e}}{\aallu{\kappa}{u}{\tau}}
%   }
% \end{equation}
% \begin{equation}\label{rule:esynP-tap}
%   \inferrule{
%     \esynPX{\ue}{e}{\aallu{\kappa}{u}{\tau}}\\
%     \ksynX{\uc}{c}{\kappa}
%   }{
%     \esynPX{\cAp{\ue}{\uc}}{\aecap{e}{c}}{[c/t]\tau}
%   }
% \end{equation}

% %Unfoldings can appear in synthetic position.
% \begin{equation}\label{rule:esynP-unfold}
%   \inferrule{
%     \esynPX{\ue}{e}{\arec{t}{\tau}}
%   }{
%     \esynPX{\unfold{\ue}}{\aeunfold{e}}{[\arec{t}{\tau}/t]\tau}
%   }
% \end{equation}

% %Labeled tuples can appear in synthetic position. Each of the field values are then in synthetic position. 
% \begin{equation}\label{rule:esynP-tpl}
%   \inferrule{
%     \{\esynPX{\ue_i}{e_i}{\tau_i}\}_{i \in \labelset}
%   }{
%     \esynPX{\tpl{\mapschema{\ue}{i}{\labelset}}}{\aetpl{\labelset}{\mapschema{e}{i}{\labelset}}}{\aprod{\labelset}{\mapschema{\tau}{i}{\labelset}}}
%   }
% \end{equation}

% %Fields can be projected out of a labeled tuple in synthetic position.
% \begin{equation}\label{rule:esynP-pr}
%   \inferrule{
%     \esynPX{\ue}{e}{\aprod{\labelset, \ell}{\mapschema{\tau}{i}{\labelset}; \mapitem{\ell}{\tau}}}
%   }{
%     \esynPX{\prj{\ue}{\ell}}{\aepr{\ell}{e}}{\tau}
%   }
% \end{equation}

% %Match expressions can appear in synthetic position.
% \begin{equation}\label{rule:esynP-match}
%   \inferrule{
%     n > 0\\
%     \esynPX{\ue}{e}{\tau}\\
%     \{\rsynPX{\urv_i}{r_i}{\tau}{\tau'}\}_{1 \leq i \leq n}
%   }{
%     \esynPX{\matchwith{\ue}{\seqschemaX{\urv}}}{\aematchwith{n}{e}{\seqschemaX{r}}}{\tau'}
%   }
% \end{equation}

% \begin{equation}\label{rule:esynP-mval}
%   \inferrule{ }{
%     \esynP{\uOmega, \uMhyp{\uX}{X}{\asignature{\kappa}{u}{\tau}}}{\uPsi}{\uPhi}{\mval{\uX}}{\amval{X}}{[\amcon{X}/u]\tau}
%   }
% \end{equation}

% % ueTSMs can be defined and applied in synthetic position.
% % \begin{equation}\label{rule:esynP-defpetsm}
% % \inferrule{
% %   \tsmtyExpands{\uOmega}{\urho}{\rho}\\
% %   \hastypeP{\emptyset}{\eparse}{\aparr{\tBody}{\tParseResultPCEExp}}\\\\
% %   \esynP{\uOmega}{\uASI{\uA \uplus \mapitem{\tsmv}{\adefref{a}}}{\Psi, \petsmdefn{a}{\rho}{\eparse}}{\uI}}{\uPhi}{\ue}{e}{\tau}
% % }{
% %   \esynP{\uOmega}{\uASI{\uA}{\Psi}{\uI}}{\uPhi}{\usyntaxueP{\tsmv}{\urho}{\eparse}{\ue}}{e}{\tau}
% % }
% % \end{equation}

% % \begin{equation}\label{rule:esynP-letpetsm}
% % \inferrule{
% %   \tsmexpExpandsExp{\uOmega}{\uASI{\uA}{\Psi}{\uI}}{\uepsilon}{\epsilon}{\rho}\\
% %   \esynP{\uOmega}{\uASI{\uA\uplus\mapitem{\tsmv}{\epsilon}}{\Psi}{\uI}}{\uPhi}{\ue}{e}{\tau}
% % }{
% %   \esynP{\uOmega}{\uASI{\uA}{\Psi}{\uI}}{\uPhi}{\uletpetsm{\tsmv}{\uepsilon}{\ue}}{e}{\tau}
% % }
% % \end{equation}

% \begin{equation}\label{rule:esynP-apuetsm}
% \inferrule{
%   \uOmega = \uOmegaEx{\uD}{\uG}{\uMctx}{\Omega_\text{app}}\\
%   \uPsi=\uAS{\uA}{\Psi, \petsmdefn{a}{\rho}{\eparse}}{\uI}\\\\
%   \tsmexpExpandsExp{\uOmega}{\uPsi}{\uepsilon}{\epsilon}{\aetype{\tau_\text{final}}}\\
%   \tsmexpEvalsExp{\Omega_\text{app}}{\Psi}{\epsilon}{\epsilon_\text{normal}}\\\\
%   \tsmdefof{\epsilon_\text{normal}}=a\\
%   \encodeBody{b}{\ebody}\\
%   \evalU{\ap{\eparse}{\ebody}}{{\lbltxt{SuccessE}}\cdot{e_\text{pproto}}}\\\\
%   \decodePCEExp{e_\text{pproto}}{\pce}\\\\
%   \prepce{\Omega_\text{app}}{\Psi, \petsmdefn{a}{\rho}{\eparse}}{\pce}{\ce}{\epsilon_\text{normal}}{\aetype{\tau_\text{proto}}}{\omega}{\Omega_\text{params}}\\\\
%   \segOK{\segof{\ce}}{b}\\
%   \canaP{\Omega_\text{params}}{\esceneP{\OParams}{\uOmega}{\uPsi}{\uPhi}{b}}{\ce}{e}{\tau_\text{proto}}
% }{
%   \esynP{\uOmega}{\uPsi}{\uPhi}{\utsmap{\uepsilon}{b}}{[\omega]e}{[\omega]\tau_\text{proto}}
% }
% \end{equation}

% % These rules are nearly identical to Rules (\ref{rule:expandsUP-syntax}) and (\ref{rule:expandsUP-tsmap}), differing only in that the typed expansion premises have been replaced by corresponding synthetic typed expansion premises. The premises of these rules can be understood as described in Sections \ref{sec:U-uetsm-definition} and \ref{sec:U-uetsm-application}. The body encoding judgement and candidate expansion expression decoding judgements were characterized in Sec. \ref{sec:typed-expansion-UP}. We discuss candidate expansion validation in Sec. \ref{sec:ce-validation-B} below.

% % To support ueTSM implicits, ueTSM contexts, $\uPsi$, are redefined to take the form $\uASI{\uA}{\Psi}{\uI}$. TSM naming contexts, $\uA$, and ueTSM definition contexts, $\Psi$, were defined in Sec. \ref{sec:typed-expansion-UP}. We write $\uPsi, \uShyp{\tsmv}{a}{\tau}{\eparse}$ when $\uPsi=\uASI{\uA}{\Psi}{\uI}$ as shorthand for \[\uASI{\ctxUpdate{\uA}{\tsmv}{a}}{\Psi, \xuetsmbnd{a}{\tau}{\eparse}}{\uI}\]

% % \emph{TSM designation contexts}, $\uI$, are finite functions that map each type $\tau \in \domof{\uI}$ to the \emph{TSM designation} $\designate{\tau}{a}$, for some symbol $a$. We write $\uI \uplus \designate{\tau}{a}$ for the TSM designation context that maps $\tau$ to $\designate{\tau}{a}$ and defers to $\uI$ for all other types (i.e. the previous designation, if any, is updated). 

% % The TSM designation context in the ueTSM context is updated by expressions of ueTSM designation form. Such expressions can appear in synthetic position, where they are governed by the following rule:% We write $\uIOK{\Delta}{\uI}$ when each type in $\uI$ is well-formed assuming $\Delta$.
% %\begin{definition}[TSM Designation Context Well-Formedness] $\uIOK{\Delta}{{\uI}$ iff for each $\designate{\tau}{a}$ we have $\istypeU{\Delta}{\tau}$.\end{definition}

% % \todo{peTSM implicit designation}
% % \begin{equation}\label{rule:esynP-implicite}
% %   \inferrule{
% %     \esyn{\uDelta}{\uGamma}{\uASI{\uA \uplus \vExpands{\tsmv}{a}}{\Psi, \xuetsmbnd{a}{\tau}{\eparse}}{\uI \uplus \designate{\tau}{a}}}{\uPhi}{\ue}{e}{\tau'}
% %   }{
% %     \esyn{\uDelta}{\uGamma}{\uASI{\uA \uplus \vExpands{\tsmv}{a}}{\Psi, \xuetsmbnd{a}{\tau}{\eparse}}{\uI}}{\uPhi}{\implicite{\tsmv}{\ue}}{e}{\tau'}
% %   }
% % \end{equation}

% % % Like ueTSMs, upTSMs can be defined in synthetic position.
% % \begin{equation}\label{rule:esynP-syntaxup}
% % \inferrule{
% %   \tsmtyExpands{\uOmega}{\urho}{\rho}\\
% %   \hastypeP{\emptyset}{\eparse}{\aparr{\tBody}{\tParseResultCEPat}}\\\\
% %   \esynP{\uOmega}{\uPsi}{\uASI{\uA \uplus \mapitem{\tsmv}{\adefref{a}}}{\Phi, \pptsmdefn{a}{\rho}{\eparse}}{\uI}}{\ue}{e}{\tau}
% % }{
% %   \esynP{\uOmega}{\uPsi}{\uASI{\uA}{\Phi}{\uI}}{\usyntaxup{\tsmv}{\urho}{\eparse}{\ue}}{e}{\tau}
% % }
% % \end{equation}


% % \begin{equation}\label{rule:esynP-letpptsm}
% % \inferrule{
% %   \tsmexpExpandsPat{\uOmega}{\uASI{\uA}{\Phi}{\uI}}{\uepsilon}{\epsilon}{\rho}\\
% %   \esynP{\uOmega}{\uPsi}{\uASI{\uA\uplus\mapitem{\tsmv}{\epsilon}}{\Phi}{\uI}}{\ue}{e}{\tau}
% % }{
% %   \esynP{\uOmega}{\uPsi}{\uASI{\uA}{\Phi}{\uI}}{\uletpptsm{\tsmv}{\uepsilon}{\ue}}{e}{\tau}
% % }
% % \end{equation}

% % % This rule is nearly identical to Rule (\ref{rule:expandsUP-defuptsm}), differing only in that the typed expansion premise has been replaced by the corresponding synthetic typed expansion premise. The premises can be understood as described in Section \ref{sec:uptsm-definition}.

% % % To support upTSM implicits, upTSM contexts, $\uPhi$, are redefined to take the form $\uASI{\uA}{\Phi}{\uI}$. upTSM definition contexts, $\Phi$, were defined in Sec. \ref{sec:uptsm-definition}. We write $\uPhi, \uPhyp{\tsmv}{a}{\tau}{\eparse}$ when $\uPhi=\uASI{\uA}{\Phi}{\uI}$ as shorthand for \[\uASI{\ctxUpdate{\uA}{\tsmv}{a}}{\Phi, \xuptsmbnd{a}{\tau}{\eparse}}{\uI}\]

% % % The TSM designation context in the upTSM context is updated by expressions of upTSM designation form. Such expressions can appear in synthetic position, where they are governed by the following rule:% We write $\uIOK{\Delta}{\uI}$ when each type in $\uI$ is well-formed assuming $\Delta$.
% % %\begin{definition}[TSM Designation Context Well-Formedness] $\uIOK{\Delta}{{\uI}$ iff for each $\designate{\tau}{a}$ we have $\istypeU{\Delta}{\tau}$.\end{definition}
% % \todo{ppTSM implicit designation}
% % \begin{equation}\label{rule:esynP-implicitp}
% %   \inferrule{
% %     \esyn{\uDelta}{\uGamma}{\uPsi}{\uASI{\uA\uplus\vExpands{\tsmv}{a}}{\Phi, \xuptsmbnd{a}{\tau}{\eparse}}{\uI \uplus \designate{\tau}{a}}}{\ue}{e}{\tau'}
% %   }{
% %     \esyn{\uDelta}{\uGamma}{\uPsi}{\uASI{\uA\uplus\vExpands{\tsmv}{a}}{\Phi, \xuetsmbnd{a}{\tau}{\eparse}}{\uI}}{\implicitp{\tsmv}{\ue}}{e}{\tau'}
% %   }
% % \end{equation}
% \end{subequations}


% Analytic typed expression expansion
% \begin{subequations}\label{rules:eanaP}
% % Type analysis subsumes type synthesis, in that when a type can be synthesized for an unexpanded expression, that unexpanded expression can also be analyzed against that type, producing the same expansion. This is expressed by the following \emph{subsumption rule} for unexpanded expressions.
% \begin{equation}\label{rule:eanaP-subsume}
%   \inferrule{
%     \esynPX{\ue}{e}{\tau}\\
%     \issubtypePX{\tau}{\tau'}
%   }{
%     \eanaPX{\ue}{e}{\tau'}
%   }
% \end{equation}

% % Additional rules are needed for certain forms in order to propagate types for analysis into subexpressions, and for forms that can appear only in analytic position.

% % Rule (\ref{rule:esyn-let}) governed value bindings in synthetic position. The following rule governs value bindings in analytic position.
% \begin{equation}\label{rule:eanaP-let}
%   \inferrule{
%     \esynPX{\ue}{e}{\tau}\\
%     \eanaP{\uOmega, \uGhyp{\ux}{x}{\tau}}{\uPsi}{\uPhi}{\ue'}{e'}{\tau'}
%   }{
%     \eanaPX{\letsyn{\ux}{\ue}{\ue'}}{\aeap{\aelam{\tau}{x}{e'}}{e}}{\tau'}
%   }
% \end{equation}

% % An unannotated function can appear only in analytic position. The argument type is determined from the type that the unannotated function is being analyzed against. 
% \begin{equation}\label{rule:eanaP-analam}
%   \inferrule{
%     \eanaP{\uOmega, \uGhyp{\ux}{x}{\tau_1}}{\uPsi}{\uPhi}{\ue}{e}{\tau_2}
%   }{
%     \eanaPX{\analam{\ux}{\ue}}{\aelam{\tau_1}{x}{e}}{\aparr{\tau_1}{\tau_2}}
%   }
% \end{equation}

% % Rule (\ref{rule:esyn-tlam}) governed type lambdas in synthetic position. The following rule governs type lambdas in analytic position.
% % \begin{equation}\label{rule:eanaP-tlam}
% %   \inferrule{
% %     \eana{\uDelta, \uDhyp{\ut}{t}}{\uGamma}{\uPsi}{\uPhi}{\ue}{e}{\tau}
% %   }{
% %     \eanaPX{\clam{\uu}{\ue}}{\aetlam{t}{e}}{\aall{t}{\tau}}
% %   }
% % \end{equation}

% % Values of recursive types can be introduced only in analytic position.
% \begin{equation}\label{rule:eanaP-fold}
%   \inferrule{
%     \eanaPX{\ue}{e}{[\arec{t}{\tau}/t]\tau}
%   }{
%     \eanaPX{\fold{\ue}}{\aefold{e}}{\arec{t}{\tau}}
%   }
% \end{equation}

% % Rule (\ref{rule:esyn-tpl}) governed labeled tuples in synthetic position. The following rule governs labeled tuples in analytic position.
% \begin{equation}\label{rule:eanaP-tpl}
%   \inferrule{
%     \{\eanaPX{\ue_i}{e_i}{\tau_i}\}_{i \in \labelset}
%   }{
%     \eanaPX{\tpl{\mapschema{\ue}{i}{\labelset}}}{\aetpl{\labelset}{\mapschema{e}{i}{\labelset}}}{\aprod{\labelset}{\mapschema{\tau}{i}{\labelset}}}
%   }
% \end{equation}

% % Values of labeled sum type can appear only in analytic position.
% \begin{equation}\label{rule:eanaP-in}
%   \inferrule{
%     \tau = \asum{\labelset, \ell}{\mapschema{\tau}{i}{\labelset}; \mapitem{\ell}{\tau'}}\\\\
%     \eanaPX{\ue'}{e'}{\tau'}
%   }{
%     \eanaPX{\inj{\ell}{\ue}}{\aein{\ell}{e'}}{\tau}
%     % \uOmega \vdash_{\uPsi; \uPhi} \left(\shortstack{$\ue \leadsto $\\$\Leftarrow$\vspace{-1.2em}}\right)
%     %\eanaPX{\auanain{\ell}{\ue}}{\aein{\ell}}{\asum{\labelset, \ell}{\mapschema{\tau}{i}{\labelset}; \mapitem{\ell}{\tau}}}
%   }
% \end{equation}

% % Rule (\ref{rule:esyn-match}) governed match expressions in synthetic position. The following rule governs match expressions in analytic position.
% \begin{equation}\label{rule:eanaP-match}
%   \inferrule{
%     \esynPX{\ue}{e}{\tau}\\
%     \{\ranaPX{\urv_i}{r_i}{\tau}{\tau'}\}_{1 \leq i \leq n}
%   }{
%     \eanaPX{\matchwith{\ue}{\seqschemaX{\urv}}}{\aematchwith{n}{e}{\seqschemaX{r}}}{\tau'}
%   }
% \end{equation}

% % Rule (\ref{rule:esyn-defuetsm}) governed ueTSM definitions in synthetic position. The following rule governs ueTSM definitions in analytic position.
% % \begin{equation}\label{rule:eanaP-defpetsm}
% % \inferrule{
% %   \tsmtyExpands{\uOmega}{\urho}{\rho}\\
% %   \hastypeP{\emptyset}{\eparse}{\aparr{\tBody}{\tParseResultPCEExp}}\\\\
% %   \eanaP{\uOmega}{\uASI{\uA \uplus \mapitem{\tsmv}{\adefref{a}}}{\Psi, \petsmdefn{a}{\rho}{\eparse}}{\uI}}{\uPhi}{\ue}{e}{\tau}
% % }{
% %   \eanaP{\uOmega}{\uASI{\uA}{\Psi}{\uI}}{\uPhi}{\usyntaxueP{\tsmv}{\urho}{\eparse}{\ue}}{e}{\tau}
% % }
% % \end{equation}

% % \begin{equation}\label{rule:eanaP-letpetsm}
% % \inferrule{
% %   \tsmexpExpandsExp{\uOmega}{\uASI{\uA}{\Psi}{\uI}}{\uepsilon}{\epsilon}{\rho}\\
% %   \eanaP{\uOmega}{\uASI{\uA\uplus\mapitem{\tsmv}{\epsilon}}{\Psi}{\uI}}{\uPhi}{\ue}{e}{\tau}
% % }{
% %   \eanaP{\uOmega}{\uASI{\uA}{\Psi}{\uI}}{\uPhi}{\uletpetsm{\tsmv}{\uepsilon}{\ue}}{e}{\tau}
% % }
% % \end{equation}

% % \todo{peTSM implicit designation}
% % Rule (\ref{rule:esyn-implicite}) governed ueTSM designations in synthetic position. The following rule governs ueTSM designations in analytic position.
% % \begin{equation}\label{rule:eanaP-implicite}
% %   \inferrule{
% %     \eana{\uDelta}{\uGamma}{\uASI{\uA \uplus \vExpands{\tsmv}{a}}{\Psi, \xuetsmbnd{a}{\tau}{\eparse}}{\uI \uplus \designate{\tau}{a}}}{\uPhi}{\ue}{e}{\tau'}
% %   }{
% %     \eana{\uDelta}{\uGamma}{\uASI{\uA \uplus \vExpands{\tsmv}{a}}{\Psi, \xuetsmbnd{a}{\tau}{\eparse}}{\uI}}{\uPhi}{\implicite{\tsmv}{\ue}}{e}{\tau'}
% %   }
% % \end{equation}

% % \todo{peTSM implicit application}
% % % An expression of unadorned literal form can appear only in analytic position. The following rule extracts the TSM designated at the type that the expression is being analyzed against from the TSM designation context in the ueTSM context and applies it implicitly, i.e. the premises correspond to those of Rule (\ref{rule:esyn-apuetsm}).
% \begin{equation}\label{rule:eanaP-lit}
%   \inferrule{
%     \encodeBody{b}{\ebody}\\
%     \evalU{\ap{\eparse}{\ebody}}{\inj{\lbltxt{Success}}{\ecand}}\\
%     \decodeCondE{\ecand}{\ce}\\\\
%     \cana{\emptyset}{\emptyset}{\esceneUP{\uDelta}{\uGamma}{\uASI{\uA}{\Psi, \xuetsmbnd{a}{\tau}{\eparse}}{\uI \uplus \designate{\tau}{a}}}{\uPhi}{b}}{\ce}{e}{\tau}
%   }{
%     \eana{\uDelta}{\uGamma}{\uASI{\uA}{\Psi, \xuetsmbnd{a}{\tau}{\eparse}}{\uI \uplus \designate{\tau}{a}}}{\uPhi}{\auelit{b}}{e}{\tau}
%   }
% \end{equation}

% % Rule (\ref{rule:esyn-defuptsm}) governed upTSM definitions in synthetic position. The following rule governs upTSM definitions in analytic position.
% % \begin{equation}\label{rule:eanaP-syntaxup}
% % \inferrule{
% %   \tsmtyExpands{\uOmega}{\urho}{\rho}\\
% %   \hastypeP{\emptyset}{\eparse}{\aparr{\tBody}{\tParseResultCEPat}}\\\\
% %   \eanaP{\uOmega}{\uPsi}{\uASI{\uA \uplus \mapitem{\tsmv}{\adefref{a}}}{\Phi, \pptsmdefn{a}{\rho}{\eparse}}{\uI}}{\ue}{e}{\tau}
% % }{
% %   \eanaP{\uOmega}{\uPsi}{\uASI{\uA}{\Phi}{\uI}}{\usyntaxup{\tsmv}{\urho}{\eparse}{\ue}}{e}{\tau}
% % }
% % \end{equation}


% % \begin{equation}\label{rule:eanaP-letpptsm}
% % \inferrule{
% %   \tsmexpExpandsPat{\uOmega}{\uASI{\uA}{\Phi}{\uI}}{\uepsilon}{\epsilon}{\rho}\\
% %   \eanaP{\uOmega}{\uPsi}{\uASI{\uA\uplus\mapitem{\tsmv}{\epsilon}}{\Phi}{\uI}}{\ue}{e}{\tau}
% % }{
% %   \eanaP{\uOmega}{\uPsi}{\uASI{\uA}{\Phi}{\uI}}{\uletpptsm{\tsmv}{\uepsilon}{\ue}}{e}{\tau}
% % }
% % \end{equation}


% % \todo{ppTSM implicit designation}
% % % Rule (\ref{rule:esyn-implicitp}) governed upTSM designations in synthetic position. The following rule governs upTSM designations in analytic position.
% % \begin{equation}\label{rule:eanaP-implicitp}
% %   \inferrule{
% %     \eana{\uDelta}{\uGamma}{\uPsi}{\uASI{\uA\uplus\vExpands{\tsmv}{a}}{\Phi, \xuptsmbnd{a}{\tau}{\eparse}}{\uI \uplus \designate{\tau}{a}}}{\ue}{e}{\tau'}
% %   }{
% %     \eana{\uDelta}{\uGamma}{\uPsi}{\uASI{\uA\uplus\vExpands{\tsmv}{a}}{\Phi, \xuetsmbnd{a}{\tau}{\eparse}}{\uI}}{\implicitp{\tsmv}{\ue}}{e}{\tau'}
% %   }
% % \end{equation}

% \end{subequations}

% Synthetic rule expansion
% %The synthetic typed rule expansion judgement is invoked iteratively by Rule (\ref{rule:esyn-match}) to synthesize a type, $\tau'$, from the branch expressions in the rule sequence. This judgement is defined mutually inductively with Rules (\ref{rules:esyn}) and Rules (\ref{rules:eana}) by the following rule. 
% \begin{equation}\label{rule:rsynP}
%   \inferrule{
%     \uOmega=\uOmegaEx{\uD}{\uG}{\uMctx}{\Omega}\\
%     \patExpandsP{\uOmegaEx{\emptyset}{\uG'}{\emptyset}{\Omega'}}{\uPhi}{\upv}{p}{\tau}\\
%     \esynP{\uOmegaEx{\uD}{\uG \uplus \uG'}{\uMctx}{\Omega \cup \Omega'}}{\uPsi}{\uPhi}{\ue}{e}{\tau'}
%   }{
%     \rsynP{\uOmega}{\uPsi}{\uPhi}{\matchrule{\upv}{\ue}}{\aematchrule{p}{e}}{\tau}{\tau'}
%   }
% \end{equation}

% Analytic rule expansion
% %The analytic typed rule expansion judgement is invoked iteratively by Rule (\ref{rule:eana-match}). This judgement is defined mutually inductively with Rules (\ref{rules:esyn}), Rules (\ref{rules:eana}), and Rule (\ref{rule:rsyn}) by the following rule, which is the analytic analag of Rule (\ref{rule:rsyn}).
% \begin{equation}\label{rule:ranaP}
%   \inferrule{
%     \uOmega=\uOmegaEx{\uD}{\uG}{\uMctx}{\Omega}\\
%     \patExpandsP{\uOmegaEx{\emptyset}{\uG'}{\emptyset}{\Omega'}}{\uPhi}{\upv}{p}{\tau}\\
%     \eanaP{\uOmegaEx{\uD}{\uG \uplus \uG'}{\uMctx}{\Omega \cup \Omega'}}{\uPsi}{\uPhi}{\ue}{e}{\tau'}
%   }{
%     \ranaP{\uOmega}{\uPsi}{\uPhi}{\matchrule{\upv}{\ue}}{\aematchrule{p}{e}}{\tau}{\tau'}
%   }
% \end{equation}

% %The premises of these rules can be understood as described in Sec. \ref{sec:typed-expansion-UP}.% We will define typed pattern expansion below.

% Typed pattern expansion
% % The typed pattern expansion judgement is inductively defined by Rules (\ref{rules:patExpandsP}) as follows. %As in $\miniVersePat$, \emph{unexpanded pattern typing contexts}, $\upctx$, are defined identically to unexpanded typing contexts (i.e. we only use a distinct metavariable to emphasize their distinct roles in the judgements above). 

% % The following rules are written identically to the typed pattern expansion rules for shared pattern forms in $\miniVersePat$, i.e. Rules (\ref{rule:patExpands-var}) through (\ref{rule:patExpands-in}).
% \begin{subequations}\label{rules:patExpandsP-B}
% \begin{equation}\label{rule:patExpandsP-B-subsume}
% \inferrule{
%   \uOmega=\uOmegaEx{\uD}{\uG}{\uMctx}{\Omega}\\\\
%   \patExpandsP{\uOmega'}{\uPhi}{\upv}{p}{\tau}\\
%   \issubtypeP{\Omega}{\tau}{\tau'}
% }{
%   \patExpandsP{\uOmega'}{\uPhi}{\upv}{p}{\tau'}
% }
% \end{equation}
% \begin{equation}\label{rule:patExpandsP-B-var}
% \inferrule{ }{
%   \patExpandsP{\uOmegaEx{\emptyset}{\vExpands{\ux}{x}}{\emptyset}{\Ghyp{x}{\tau}}}{\uPhi}{\ux}{x}{\tau}
% }
% \end{equation}
% \begin{equation}\label{rule:patExpandsP-B-wild}
% \inferrule{ }{
%   \patExpandsP{\uOmegaEx{\emptyset}{\emptyset}{\emptyset}{\emptyset}}{\uPhi}{\wildp}{\aewildp}{\tau}
% }
% \end{equation}
% \begin{equation}\label{rule:patExpandsP-B-fold}
% \inferrule{ 
%   \patExpandsP{\uOmega'}{\uPhi}{\upv}{p}{[\arec{t}{\tau}/t]\tau}
% }{
%   \patExpandsP{\uOmega'}{\uPhi}{\foldp{\upv}}{\aefoldp{p}}{\arec{t}{\tau}}
% }
% \end{equation}
% \begin{equation}\label{rule:patExpandsP-B-tpl}
% \inferrule{
%   \tau=\aprod{\labelset}{\mapschema{\tau}{i}{\labelset}}\\\\
%   \{\patExpandsP{{\uOmega_i}}{\uPhi}{\upv_i}{p_i}{\tau_i}\}_{i \in \labelset}
% }{
%   %\patExpandsP{\Gconsi{i \in \labelset}{\upctx_i}}{A}{B}{C}
%   \patExpandsP{\Gconsi{i \in \labelset}{\uOmega_i}}{\uPhi}{\tplp{\mapschema{\upv}{i}{\labelset}}}{\aetplp{\labelset}{\mapschema{p}{i}{\labelset}}}{\tau}
%   % \patExpands{\Gconsi{i \in \labelset}{\pctx_i}}{\Phi}{
%   %   \autplp{\labelset}{\mapschema{\upv}{i}{\labelset}}
%   % }{
%   %   \aetplp{\labelset}{\mapschema{p}{i}{\labelset}}
%   % }{
%   %   \aprod{\labelset}{\mapschema{\tau}{i}{\labelset}}
%   % } %{\autplp{\labelset}{\mapschema{\upv}{i}{\labelset}}}{\aetplp{\labelset}{\mapschema}{p}{i}{\labelset}}{...}
%   %\left(\shortstack{$\Delta \vdash_{\uPhi} \autplp{\labelset}{\mapschema{\upv}{i}{\labelset}}$\\$\leadsto$\\$\aetplp{\labelset}{\mapschema{p}{i}{\labelset}} : \aprod{\labelset}{\mapschema{\tau}{i}{\labelset}} \dashV \Gconsi{i \in \labelset}{\upctx_i}$\vspace{-1.2em}}\right)
% }
% \end{equation}
% \begin{equation}\label{rule:patExpandsP-B-in}
% \inferrule{
%   \patExpandsP{\uOmega'}{\uPhi}{\upv}{p}{\tau}
% }{
%   \patExpandsP{\uOmega'}{\uPhi}{\injp{\ell}{\upv}}{\aeinjp{\ell}{p}}{\asum{\labelset, \ell}{\mapschema{\tau}{i}{\labelset}; \mapitem{\ell}{\tau}}}
% }
% \end{equation}

% \begin{equation}\label{rule:patExpandsP-B-apuptsm}
% \inferrule{
%   \uOmega=\uOmegaEx{\uD}{\uG}{\uMctx}{\Omega_\text{app}}\\
%   \uPhi=\uASI{\uA}{\Phi, \pptsmdefn{a}{\rho}{\eparse}}{\uI}\\\\
%   \tsmexpExpandsPat{\uOmega}{\uPhi}{\uepsilon}{\epsilon}{\aetype{\tau_\text{final}}}\\
%   \tsmdefof{\epsilon}=a\\\\
%   \encodeBody{b}{\ebody}\\
%   \evalU{\ap{\eparse}{\ebody}}{{\lbltxt{Success}}\cdot{\ecand}}\\
%   \decodeCEPat{\ecand}{\cpv}\\\\
%   \prepcp{\Omega_\text{app}}{\Phi, \pptsmdefn{a}{\rho}{\eparse}}{\pcp}{\cpv}{\epsilon}{\aetype{\tau_\text{cand}}}{\omega}{\Omega_\text{params}}\\\\
%   \cvalidPP{\uOmega'}{\psceneP{\uOmega}{\uPhi}{b}}{\cpv}{p}{\tau_\text{cand}}
% }{
%   \patExpandsP{\uOmega'}{\uPhi}{\utsmap{\uepsilon}{b}}{p}{[\omega]\tau_\text{cand}}
% }
% \end{equation}

% \todo{ppTSM implicit application}
% % Unexpanded patterns of unadorned literal form are governed by the following rule, which extracts the designated upTSM from the upTSM context and applies it implicitly, i.e. the premises correspond to those of Rule (\ref{rule:patExpandsP-apuptsm}).
% \begin{equation}\label{rule:patExpandsP-B-lit}
% \inferrule{
%   \encodeBody{b}{\ebody}\\
%   \evalU{\ap{\eparse}{\ebody}}{\inj{\lbltxt{Success}}{\ecand}}\\
%   \decodeCEPat{\ecand}{\cpv}\\\\
%   \cvalidPP{\uOmega}{\pscene{\uDelta}{\uASI{\uA}{\Phi, \xuptsmbnd{a}{\tau}{\eparse}}{\uI, \designate{\tau}{a}}}{b}}{\cpv}{p}{\tau}
% }{
%   \patExpands{\uOmega}{\uASI{\uA}{\Phi, \xuptsmbnd{a}{\tau}{\eparse}}{\uI, \designate{\tau}{a}}}{\lit{b}}{p}{\tau}
% }
% \end{equation}

% \end{subequations}


% \subsubsection{Unexpanded Signatures and Module Expressions}
% Signature expansion
% \begin{equation}\label{rule:sigExpandsP-B}
% \inferrule{
%   \kExpandsX{\ukappa}{\kappa}\\
%   \cExpands{\uOmega, \uKhyp{\uu}{u}{\kappa}}{\utau}{\tau}{\akty}
% }{
%   \sigExpandsPX{\signature{\uu}{\ukappa}{\utau}}{\asignature{\kappa}{u}{\tau}}
% }
% \end{equation}

% Synthetic module expression expansion
% \begin{subequations}\label{rules:msyn}
% \begin{equation}\label{rule:msyn-var}
% \inferrule{ }{
%   \msyn{\uOmega, \uMhyp{\uX}{X}{\sigma}}{\uPsi}{\uPhi}{\uX}{X}{\sigma}
% }
% \end{equation}
% \begin{equation}\label{rule:msyn-seal}
% \inferrule{
%   \sigExpandsPX{\usigma}{\sigma}\\
%   \manaX{\uM}{M}{\sigma}
% }{
%   \msynX{\seal{\uM}{\usigma}}{\aseal{\sigma}{M}}{\sigma} 
% }
% \end{equation}
% \begin{equation}\label{rule:msyn-mlet}
% \inferrule{
%   \msynX{\uM}{M}{\sigma}\\
%   \sigExpandsPX{\usigma'}{\sigma'}\\\\
%   \mana{\uOmega, \uMhyp{\uX}{X}{\sigma}}{\uPsi}{\uPhi}{\uM'}{M'}{\sigma'}
% }{
%   \msynX{\mlet{\uX}{\uM}{\uM'}{\usigma'}}{\amlet{\sigma'}{M}{X}{M'}}{\sigma'}
% }
% \end{equation}
% \begin{equation}\label{rule:msyn-syntaxpe}
% \inferrule{
%   \tsmtyExpands{\uOmega}{\urho}{\rho}\\
%   \hastypeP{\emptyset}{\eparse}{\aparr{\tBody}{\tParseResultPCEExp}}\\\\
%   \msyn{\uOmega}{\uASI{\uA \uplus \mapitem{\tsmv}{\adefref{a}}}{\Psi, \petsmdefn{a}{\rho}{\eparse}}{\uI}}{\uPhi}{\uM}{M}{\sigma}
% }{
%   \msyn{\uOmega}{\uASI{\uA}{\Psi}{\uI}}{\uPhi}{\usyntaxueP{\tsmv}{\urho}{\eparse}{\uM}}{M}{\sigma}
% }
% \end{equation}
% \begin{equation}\label{rule:msyn-letpetsm}
% \inferrule{
%   \tsmexpExpandsExp{\uOmega}{\uASI{\uA}{\Psi}{\uI}}{\uepsilon}{\epsilon}{\rho}\\
%   \msyn{\uOmega}{\uASI{\uA\uplus\mapitem{\tsmv}{\epsilon}}{\Psi}{\uI}}{\uPhi}{\uM}{M}{\sigma}
% }{
%   \msyn{\uOmega}{\uASI{\uA}{\Psi}{\uI}}{\uPhi}{\uletpetsm{\tsmv}{\uepsilon}{\uM}}{M}{\sigma}
% }
% \end{equation}
% \todo{peTSM implicit designation at module level}
% \begin{equation}\label{rule:msyn-implicitpe}
% \inferrule{
%   ...
% }{
%   ...
% }
% \end{equation}
% \begin{equation}\label{rule:msyn-syntaxpp}
% \inferrule{
%   \tsmtyExpands{\uOmega}{\urho}{\rho}\\
%   \hastypeP{\emptyset}{\eparse}{\aparr{\tBody}{\tParseResultCEPat}}\\\\
%   \msyn{\uOmega}{\uPsi}{\uASI{\uA \uplus \mapitem{\tsmv}{\adefref{a}}}{\Phi, \pptsmdefn{a}{\rho}{\eparse}}{\uI}}{\uM}{M}{\sigma}
% }{
%   \msyn{\uOmega}{\uPsi}{\uASI{\uA}{\Phi}{\uI}}{\usyntaxup{\tsmv}{\urho}{\eparse}{\uM}}{M}{\sigma}
% }
% \end{equation}
% \begin{equation}\label{rule:msyn-letpptsm}
% \inferrule{
%   \tsmexpExpandsPat{\uOmega}{\uASI{\uA}{\Phi}{\uI}}{\uepsilon}{\epsilon}{\rho}\\
%   \msyn{\uOmega}{\uPsi}{\uASI{\uA\uplus\mapitem{\tsmv}{\epsilon}}{\Phi}{\uI}}{\uM}{M}{\sigma}
% }{
%   \msyn{\uOmega}{\uPsi}{\uASI{\uA}{\Phi}{\uI}}{\uletpptsm{\tsmv}{\uepsilon}{\uM}}{M}{\sigma}
% }
% \end{equation}
% \todo{ppTSM implicit designation at module level}
% \begin{equation}\label{rule:msyn-implicitpp}
% \inferrule{
%   ...
% }{
%   ...
% }
% \end{equation}
% \end{subequations}

% Analytic module expression expansion
% \begin{subequations}\label{rules:mana}
% \begin{equation}\label{rule:mana-subsumes}
% \inferrule{
%   \msynX{\uM}{M}{\sigma}\\
%   \sigsub{\uOmega}{\sigma}{\sigma'}
% }{
%   \manaX{\uM}{M}{\sigma'}
% }
% \end{equation}
% \begin{equation}\label{rule:mana-struct}
% \inferrule{
%   \kanaX{\uc}{c}{\kappa}\\
%   \eanaPX{\ue}{e}{[c/u]\tau}
% }{
%   \manaX{\struct{\uc}{\ue}}{\astruct{c}{e}}{\asignature{\kappa}{u}{\tau}}
% }
% \end{equation}
% \begin{equation}\label{rule:mana-syntaxpe}
% \inferrule{
%   \tsmtyExpands{\uOmega}{\urho}{\rho}\\
%   \hastypeP{\emptyset}{\eparse}{\aparr{\tBody}{\tParseResultPCEExp}}\\\\
%   \mana{\uOmega}{\uASI{\uA \uplus \mapitem{\tsmv}{\adefref{a}}}{\Psi, \petsmdefn{a}{\rho}{\eparse}}{\uI}}{\uPhi}{\uM}{M}{\sigma}
% }{
%   \mana{\uOmega}{\uASI{\uA}{\Psi}{\uI}}{\uPhi}{\usyntaxueP{\tsmv}{\urho}{\eparse}{\uM}}{M}{\sigma}
% }
% \end{equation}
% \begin{equation}\label{rule:mana-letpetsm}
% \inferrule{
%   \tsmexpExpandsExp{\uOmega}{\uASI{\uA}{\Psi}{\uI}}{\uepsilon}{\epsilon}{\rho}\\
%   \mana{\uOmega}{\uASI{\uA\uplus\mapitem{\tsmv}{\epsilon}}{\Psi}{\uI}}{\uPhi}{\uM}{M}{\sigma}
% }{
%   \mana{\uOmega}{\uASI{\uA}{\Psi}{\uI}}{\uPhi}{\uletpetsm{\tsmv}{\uepsilon}{\uM}}{M}{\sigma}
% }
% \end{equation}
% \todo{peTSM implicit designation at module level}
% \begin{equation}\label{rule:mana-implicitpe}
% \inferrule{
%   ...
% }{
%   ...
% }
% \end{equation}
% \begin{equation}\label{rule:mana-syntaxpp}
% \inferrule{
%   \tsmtyExpands{\uOmega}{\urho}{\rho}\\
%   \hastypeP{\emptyset}{\eparse}{\aparr{\tBody}{\tParseResultCEPat}}\\\\
%   \mana{\uOmega}{\uPsi}{\uASI{\uA \uplus \mapitem{\tsmv}{\adefref{a}}}{\Phi, \pptsmdefn{a}{\rho}{\eparse}}{\uI}}{\uM}{M}{\sigma}
% }{
%   \mana{\uOmega}{\uPsi}{\uASI{\uA}{\Phi}{\uI}}{\usyntaxup{\tsmv}{\urho}{\eparse}{\uM}}{M}{\sigma}
% }
% \end{equation}
% \begin{equation}\label{rule:mana-letpptsm}
% \inferrule{
%   \tsmexpExpandsPat{\uOmega}{\uASI{\uA}{\Phi}{\uI}}{\uepsilon}{\epsilon}{\rho}\\
%   \mana{\uOmega}{\uPsi}{\uASI{\uA\uplus\mapitem{\tsmv}{\epsilon}}{\Phi}{\uI}}{\uM}{M}{\sigma}
% }{
%   \mana{\uOmega}{\uPsi}{\uASI{\uA}{\Phi}{\uI}}{\uletpptsm{\tsmv}{\uepsilon}{\uM}}{M}{\sigma}
% }
% \end{equation}
% \todo{ppTSM implicit designation at module level}
% \begin{equation}\label{rule:mana-implicitpp}
% \inferrule{
%   ...
% }{
%   ...
% }
% \end{equation}
% \end{subequations}

% \subsubsection{TSM Types and Expressions}
% TSM Expression Typing

% \vspace{10px}
% $\begin{array}{ll}
% \textbf{Judgement Form} & \textbf{Description}\\
% \istsmty{\Omega}{\rho} & \text{$\rho$ is a well-formed TSM type}\\
% \hastsmtypeExp{\Omega}{\Psi}{\epsilon}{\rho} & \text{peTSM expression $\epsilon$ has TSM type $\rho$}\\
% \hastsmtypePat{\Omega}{\Phi}{\epsilon}{\rho} & \text{ppTSM expression $\epsilon$ has TSM type $\rho$}
% \end{array}$
% \vspace{10px}

% peTSM Expression Evaluation

% \vspace{10px}
% $\begin{array}{ll}
% \textbf{Judgement Form} & \textbf{Description}\\
% \tsmexpNormalExp{\Omega}{\Psi}{\epsilon} & \text{peTSM expression $\epsilon$ is in normal form}\\
% \tsmexpStepsExp{\Omega}{\Psi}{\epsilon}{\epsilon'} & \text{peTSM expression $\epsilon$ transitions to $\epsilon'$}\\
% \end{array}$
% \vspace{10px}

% + auxiliary judgements for multi-step transitions and evaluation

% unexpanded TSM types and expressions

% \vspace{10px}
% $\begin{array}{ll}
% \textbf{Judgement Form} & \textbf{Description}\\
% \tsmtyExpands{\uOmega}{\urho}{\rho} & \text{$\urho$ has expansion $\rho$}\\
% \tsmexpExpandsExp{\uOmega}{\uPsi}{\uepsilon}{\epsilon}{\rho} & \text{unexpanded peTSM expression $\uepsilon$ has expansion $\epsilon$ and type $\rho$}\\
% \tsmexpExpandsPat{\uOmega}{\uPhi}{\uepsilon}{\epsilon}{\rho} & \text{unexpanded ppTSM expression $\uepsilon$ has expansion $\epsilon$ and type $\rho$}
% \end{array}$
% \vspace{10px}

% TSM type formation
% \begin{subequations}\label{rules:istsmty-B}
% \begin{equation}\label{rule:istsmty-B-type}
% \inferrule{
%   \haskindX{\tau}{\akty}
% }{
%   \istsmty{\Omega}{\aetype{\tau}}
% }
% \end{equation}
% \begin{equation}\label{rule:istsmty-B-alltypes}
% \inferrule{
%   \istsmty{\Omega, t :: \akty}{\rho}
% }{
%   \istsmty{\Omega}{\aealltypes{t}{\rho}}
% }
% \end{equation}
% \begin{equation}\label{rule:istsmty-B-allmods}
% \inferrule{
%   \issig{\Omega}{\sigma}\\
%   \istsmty{\Omega, X : \sigma}{\rho}
% }{
%   \istsmty{\Omega}{\aeallmods{\sigma}{X}{\rho}}
% }
% \end{equation}
% \end{subequations}

% Unexpanded TSM type expansion
% \begin{subequations}\label{rules:tsmtyExpands-B}
% \begin{equation}\label{rule:tsmtyExpands-B-type}
% \inferrule{
%   \cExpandsX{\utau}{\tau}{\akty}
% }{
%   \tsmtyExpands{\uOmega}{{\utau}}{\aetype{\tau}}
% }
% \end{equation}
% \begin{equation}\label{rule:tsmtyExpands-B-alltypes}
% \inferrule{
%   \tsmtyExpands{\uOmega, \uKhyp{\ut}{t}{\akty}}{\urho}{\rho}
% }{
%   \tsmtyExpands{\uOmega}{\alltypes{\ut}{\urho}}{\aealltypes{t}{\rho}}
% }
% \end{equation}
% \begin{equation}\label{rule:tsmtyExpands-B-allmods}
% \inferrule{
%   \sigExpandsPX{\usigma}{\sigma}\\
%   \tsmtyExpands{\uOmega, \uMhyp{\uX}{X}{\sigma}}{\urho}{\rho}
% }{
%   \tsmtyExpands{\uOmega}{\allmods{\uX}{\usigma}{\urho}}{\aeallmods{\sigma}{X}{\rho}}
% }
% \end{equation}
% \end{subequations}
% peTSM Expression Typing
% \begin{subequations}\label{rules:hastsmtypeExp-B}
% \begin{equation}\label{rule:hastsmtypeExp-B-defref}
% \inferrule{ }{
%   \hastsmtypeExp{\Omega}{\Psi, \petsmdefn{a}{\rho}{\eparse}}{\adefref{a}}{\rho}
% }
% \end{equation}
% \begin{equation}\label{rule:hastsmtypeExp-B-abstype}
% \inferrule{
%   \hastsmtypeExp{\Omega, t :: \akty}{\Psi}{\epsilon}{\rho}
% }{
%   \hastsmtypeExp{\Omega}{\Psi}{\aeabstype{t}{\epsilon}}{\aealltypes{t}{\rho}}
% }
% \end{equation}
% \begin{equation}\label{rule:hastsmtypeExp-B-absmod}
% \inferrule{
%   \issigX{\sigma}\\
%   \hastsmtypeExp{\Omega, X : \sigma}{\Psi}{\epsilon}{\rho}
% }{
%   \hastsmtypeExp{\Omega}{\Psi}{\aeabsmod{\sigma}{X}{\epsilon}}{\aeallmods{\sigma}{X}{\rho}}
% }
% \end{equation}
% \begin{equation}\label{rule:hastsmtypeExp-B-aptype}
% \inferrule{
%   \hastsmtypeExp{\Omega}{\Psi}{\epsilon}{\aealltypes{t}{\rho}}\\
%   \haskindX{\tau}{\akty}
% }{
%   \hastsmtypeExp{\Omega}{\Psi}{\aeaptype{\tau}{\epsilon}}{[\tau/t]\rho}
% }
% \end{equation}
% \begin{equation}\label{rule:hastsmtypeExp-B-apmod}
% \inferrule{
%   \hastsmtypeExp{\Omega}{\Psi}{\epsilon}{\aeallmods{\sigma}{X'}{\rho}}\\
%   \hassig{\Omega}{X}{\sigma}
% }{
%   \hastsmtypeExp{\Omega}{\Psi}{\aeapmod{X}{\epsilon}}{[X/X']\rho}
% }
% \end{equation}
% \end{subequations}

% ppTSM Expression Typing
% \begin{subequations}\label{rules:hastsmtypePat-B}
% \begin{equation}\label{rule:hastsmtypePat-B-defref}
% \inferrule{ }{
%   \hastsmtypePat{\Omega}{\Phi, \pptsmdefn{a}{\rho}{\eparse}}{\adefref{a}}{\rho}
% }
% \end{equation}
% \begin{equation}\label{rule:hastsmtypePat-B-abstype}
% \inferrule{
%   \hastsmtypePat{\Omega, t :: \akty}{\Phi}{\epsilon}{\rho}
% }{
%   \hastsmtypePat{\Omega}{\Phi}{\aeabstype{t}{\epsilon}}{\aealltypes{t}{\rho}}
% }
% \end{equation}
% \begin{equation}\label{rule:hastsmtypePat-B-absmod}
% \inferrule{
%   \issigX{\sigma}\\
%   \hastsmtypePat{\Omega, X : \sigma}{\Phi}{\epsilon}{\rho}
% }{
%   \hastsmtypePat{\Omega}{\Phi}{\aeabsmod{\sigma}{X}{\epsilon}}{\aeallmods{\sigma}{X}{\rho}}
% }
% \end{equation}
% \begin{equation}\label{rule:hastsmtypePat-B-aptype}
% \inferrule{
%   \hastsmtypePat{\Omega}{\Phi}{\epsilon}{\aealltypes{t}{\rho}}\\
%   \haskindX{\tau}{\akty}
% }{
%   \hastsmtypePat{\Omega}{\Phi}{\aeaptype{\tau}{\epsilon}}{[\tau/t]\rho}
% }
% \end{equation}
% \begin{equation}\label{rule:hastsmtypePat-B-apmod}
% \inferrule{
%   \hastsmtypePat{\Omega}{\Phi}{\epsilon}{\aeallmods{\sigma}{X'}{\rho}}\\
%   \hassig{\Omega}{X}{\sigma}
% }{
%   \hastsmtypePat{\Omega}{\Phi}{\aeapmod{X}{\epsilon}}{[X/X']\rho}
% }
% \end{equation}

% \end{subequations}

% peTSM Expression Expansion
% \begin{subequations}\label{rules:tsmexpExpandsExp-B}
% \begin{equation}\label{rule:tsmexpExpandsExp-B-bindref}
% \inferrule{
%   \hastsmtypeExp{\Omega}{\Psi}{\epsilon}{\rho}  
% }{
%   \tsmexpExpandsExp{\uOmegaEx{\uD}{\uG}{\uMctx}{\Omega}}{\uASI{\uA, \mapitem{\tsmv}{\epsilon}}{\Psi}{\uI}}{{\tsmv}}{\epsilon}{\rho}
% }
% \end{equation}
% \begin{equation}\label{rule:tsmexpExpandsExp-B-abstype}
% \inferrule{
%   \tsmexpExpandsExp{\uOmega, \uKhyp{\ut}{t}{\akty}}{\uPsi}{\uepsilon}{\epsilon}{\rho}
% }{
%   \tsmexpExpandsExp{\uOmega}{\uPsi}{\abstype{\ut}{\uepsilon}}{\aeabstype{t}{\epsilon}}{\aealltypes{t}{\rho}}
% }
% \end{equation}
% \begin{equation}\label{rule:tsmexpExpandsExp-B-absmod}
% \inferrule{
%   \sigExpandsPX{\usigma}{\sigma}\\
%   \tsmexpExpandsExp{\uOmega, \uMhyp{\uX}{X}{\sigma}}{\uPsi}{\uepsilon}{\epsilon}{\rho}
% }{
%   \tsmexpExpandsExp{\uOmega}{\uPsi}{\absmod{\uX}{\usigma}{\uepsilon}}{\aeabsmod{\sigma}{X}{\epsilon}}{\aeallmods{\sigma}{X}{\rho}}
% }
% \end{equation}
% \begin{equation}\label{rule:tsmexpExpandsExp-B-aptype}
% \inferrule{
%   \tsmexpExpandsExp{\uOmega}{\uPsi}{\uepsilon}{\epsilon}{\aealltypes{t}{\rho}}\\
%   \cExpandsX{\utau}{\tau}{\akty}
% }{
%   \tsmexpExpandsExp{\uOmega}{\uPsi}{\aptype{\uepsilon}{\utau}}{\aeaptype{\tau}{\epsilon}}{[\tau/t]\rho} 
% }
% \end{equation}
% \begin{equation}\label{rule:tsmexpExpandsExp-B-apmod}
% \inferrule{
%   \tsmexpExpandsExp{\uOmega}{\uPsi}{\uepsilon}{\epsilon}{\aeallmods{\sigma}{X'}{\rho}}\\
%   \manaX{\uX}{X}{\sigma}
% }{
%   \tsmexpExpandsExp{\uOmega}{\uPsi}{\apmod{\uepsilon}{\uX}}{\aeapmod{X}{\epsilon}}{[X/X']\rho}
% }
% \end{equation}
% \end{subequations}

% ppTSM expression expansion
% \begin{subequations}\label{rules:tsmexpExpandsPat-B}
% \begin{equation}\label{rule:tsmexpExpandsPat-B-bindref}
% \inferrule{
%   \hastsmtypePat{\Omega}{\Phi}{\epsilon}{\rho}  
% }{
%   \tsmexpExpandsPat{\uOmegaEx{\uD}{\uG}{\uMctx}{\Omega}}{\uASI{\uA, \mapitem{\tsmv}{\epsilon}}{\Phi}{\uI}}{{\tsmv}}{\epsilon}{\rho}
% }
% \end{equation}
% \begin{equation}\label{rule:tsmexpExpandsPat-B-abstype}
% \inferrule{
%   \tsmexpExpandsPat{\uOmega, \uKhyp{\ut}{t}{\akty}}{\uPhi}{\uepsilon}{\epsilon}{\rho}
% }{
%   \tsmexpExpandsPat{\uOmega}{\uPhi}{\abstype{\ut}{\uepsilon}}{\aeabstype{t}{\epsilon}}{\aealltypes{t}{\rho}}
% }
% \end{equation}
% \begin{equation}\label{rule:tsmexpExpandsPat-B-absmod}
% \inferrule{
%   \sigExpandsPX{\usigma}{\sigma}\\
%   \tsmexpExpandsPat{\uOmega, \uMhyp{\uX}{X}{\sigma}}{\uPhi}{\uepsilon}{\epsilon}{\rho}
% }{
%   \tsmexpExpandsPat{\uOmega}{\uPhi}{\absmod{\uX}{\usigma}{\uepsilon}}{\aeabsmod{\sigma}{X}{\epsilon}}{\aeallmods{\sigma}{X}{\rho}}
% }
% \end{equation}
% \begin{equation}\label{rule:tsmexpExpandsPat-B-aptype}
% \inferrule{
%   \tsmexpExpandsPat{\uOmega}{\uPhi}{\uepsilon}{\epsilon}{\aealltypes{t}{\rho}}\\
%   \cExpandsX{\utau}{\tau}{\akty}
% }{
%   \tsmexpExpandsPat{\uOmega}{\uPhi}{\aptype{\uepsilon}{\utau}}{\aeaptype{\tau}{\epsilon}}{[\tau/t]\rho} 
% }
% \end{equation}
% \begin{equation}\label{rule:tsmexpExpandsPat-B-apmod}
% \inferrule{
%   \tsmexpExpandsPat{\uOmega}{\uPhi}{\uepsilon}{\epsilon}{\aeallmods{\sigma}{X'}{\rho}}\\
%   \manaX{\uX}{X}{\sigma}
% }{
%   \tsmexpExpandsPat{\uOmega}{\uPhi}{\apmod{\uepsilon}{\uX}}{\aeapmod{X}{\epsilon}}{[X/X']\rho}
% }
% \end{equation}
% \end{subequations}

% peTSM expression normal forms
% \begin{subequations}\label{rules:tsmexpNormalExp-B}
% \begin{equation}\label{rule:tsmexpNormalExp-B-defref}
% \inferrule{ }{
%   \tsmexpNormalExp{\Omega}{\Psi, \petsmdefn{a}{\rho}{\eparse}}{\adefref{a}}
% }
% \end{equation}
% \begin{equation}\label{rule:tsmexpNormalExp-B-abstype}
% \inferrule{ }{
%   \tsmexpNormalExp{\Omega}{\Psi}{\aeabstype{t}{\epsilon}}
% }
% \end{equation}
% \begin{equation}\label{rule:tsmexpNormalExp-B-absmod}
% \inferrule{ }{
%   \tsmexpNormalExp{\Omega}{\Psi}{\aeabsmod{\sigma}{X}{\epsilon}}
% }
% \end{equation}
% \begin{equation}\label{rule:tsmexpNormalExp-B-aptype}
% \inferrule{
%   \epsilon \neq \aeabstype{t}{\epsilon'}\\
%   \tsmexpNormalExp{\Omega}{\Psi}{\epsilon}
% }{
%   \tsmexpNormalExp{\Omega}{\Psi}{\aeaptype{\tau}{\epsilon}}
% }
% \end{equation}
% \begin{equation}\label{rule:tsmexpNormalExp-B-apmod}
% \inferrule{
%   \epsilon \neq \aeabsmod{\sigma}{X'}{\epsilon'}\\
%   \tsmexpNormalExp{\Omega}{\Psi}{\epsilon}
% }{
%   \tsmexpNormalExp{\Omega}{\Psi}{\aeapmod{X}{\epsilon}}
% }
% \end{equation}
% \end{subequations}

% peTSM transitions
% \begin{subequations}\label{rules:tsmexpStepsExp-B}
% \begin{equation}\label{rule:tsmexpStepsExp-B-aptype-1}
% \inferrule{
%   \tsmexpStepsExp{\Omega}{\Psi}{\epsilon}{\epsilon'}
% }{
%   \tsmexpStepsExp{\Omega}{\Psi}{\aeaptype{\tau}{\epsilon}}{\aeaptype{\tau}{\epsilon'}}
% }
% \end{equation}
% \begin{equation}\label{rule:tsmexpStepsExp-B-aptype-2}
% \inferrule{ }{
%   \tsmexpStepsExp{\Omega}{\Psi}{\aeaptype{\tau}{\aeabstype{t}{\epsilon}}}{[\tau/t]\epsilon}
% }
% \end{equation}
% \begin{equation}\label{rule:tsmexpStepsExp-B-apmod-1}
% \inferrule{
%   \tsmexpStepsExp{\Omega}{\Psi}{\epsilon}{\epsilon'}
% }{
%   \tsmexpStepsExp{\Omega}{\Psi}{\aeapmod{X}{\epsilon}}{\aeapmod{X}{\epsilon'}}
% }
% \end{equation}
% \begin{equation}\label{rule:tsmexpStepsExp-B-apmod-2}
% \inferrule{ }{
%   \tsmexpStepsExp{\Omega}{\Psi}{\aeapmod{X}{\aeabsmod{\sigma}{X'}{\epsilon}}}{[X/X']\epsilon}
% }
% \end{equation}
% \end{subequations}

% peTSM reflexive, transitive transitions
% \begin{subequations}\label{rules:tsmexpMultistepsExp-B}
% \begin{equation}\label{rule:tsmexpMultistepsExp-B-refl}
% \inferrule{ }{
%   \tsmexpMultistepsExp{\Omega}{\Psi}{\epsilon}{\epsilon}
% }
% \end{equation}
% \begin{equation}\label{rule:tsmexpMultistepsExp-B-steps}
% \inferrule{
%   \tsmexpStepsExp{\Omega}{\Psi}{\epsilon}{\epsilon'}
% }{
%   \tsmexpMultistepsExp{\Omega}{\Psi}{\epsilon}{\epsilon'}
% }
% \end{equation}
% \begin{equation}\label{rule:tsmexpMultistepsExp-B-trans}
% \inferrule{
%   \tsmexpMultistepsExp{\Omega}{\Psi}{\epsilon}{\epsilon'}\\
%   \tsmexpMultistepsExp{\Omega}{\Psi}{\epsilon'}{\epsilon''}
% }{
%   \tsmexpMultistepsExp{\Omega}{\Psi}{\epsilon}{\epsilon''}
% }
% \end{equation}
% \end{subequations}

% peTSM normalization
% \begin{equation}\label{rule:tsmexpEvalsExp-B}
% \inferrule{
%   \tsmexpMultistepsExp{\Omega}{\Psi}{\epsilon}{\epsilon'}\\
%   \tsmexpNormalExp{\Omega}{\Psi}{\epsilon'}
% }{
%   \tsmexpEvalsExp{\Omega}{\Psi}{\epsilon}{\epsilon'}
% }
% \end{equation}

% TSM expression definition extraction

% \begin{subequations}
% \begin{align}
% \tsmdefof{\adefref{a}} & = a\\
% \tsmdefof{\aeabstype{t}{\epsilon}} & = \tsmdefof{\epsilon}\\
% \tsmdefof{\aeabsmod{\sigma}{X}{\epsilon}} & = \tsmdefof{\epsilon}\\
% \tsmdefof{\aeaptype{\tau}{\epsilon}} & = \tsmdefof{\epsilon}\\
% \tsmdefof{\aeapmod{X}{\epsilon}} & = \tsmdefof{\epsilon}
% \end{align}
% \end{subequations}

% \subsubsection{Candidate Expansion Kind and Constructor Validation}
% %The \emph{ce-type validation judgement}, $\cvalidT{\Delta}{\tscenev}{\ctau}{\tau}$, is inductively defined by Rules (\ref{rules:cvalidT-U}), which were defined in Sec. \ref{sec:ce-validation-U}.

% ce-kind validation
% \begin{subequations}\label{rules:cvalidK-B}
% \begin{equation}\label{rule:cvalidK-B-darr}
% \inferrule{
%   \cvalidKX{\cekappa_1}{\kappa_1}\\
%   \cvalidK{\Omega, u :: \kappa_1}{\cscenev}{\cekappa_2}{\kappa_2}
% }{
%   \cvalidKX{\acekdarr{\cekappa_1}{u}{\cekappa_2}}{\akdarr{\kappa_1}{u}{\kappa_2}}
% }
% \end{equation}
% \begin{equation}\label{rule:cvalidK-B-unit}
% \inferrule{ }{
%   \cvalidKX{\acekunit}{\akunit}
% }
% \end{equation}
% \begin{equation}\label{rule:cvalidK-B-dprod}
% \inferrule{
%   \cvalidKX{\cekappa_1}{\kappa_1}\\
%   \cvalidK{\Omega, u :: \kappa_1}{\cscenev}{\cekappa_2}{\kappa_2}
% }{
%   \cvalidKX{\acekdbprod{\cekappa_1}{u}{\cekappa_2}}{\akdbprod{\kappa_1}{u}{\kappa_2}}
% }
% \end{equation}
% \begin{equation}\label{rule:cvalidK-B-ty}
% \inferrule{ }{
%   \cvalidKX{\acekty}{\akty}
% }
% \end{equation}
% \begin{equation}\label{rule:cvalidK-B-sing}
% \inferrule{
%   \ccanaX{\ctau}{\tau}{\akty}
% }{
%   \cvalidKX{\aceksing{\ctau}}{\aksing{\tau}}
% }
% \end{equation}
% \begin{equation}\label{rule:cvalidK-B-spliced}
% \inferrule{
%   \parseUKind{\bsubseq{b}{m}{n}}{\ukappa}\\
%   \kExpands{\uOmega}{\ukappa}{\kappa}\\\\
%   \uOmega=\uOmegaEx{\uD}{\uG}{\uMctx}{\Omega_\text{app}}\\
%   \domof{\Omega} \cap \domof{\Omega_\text{app}} = \emptyset
% }{
%   \cvalidK{\Omega}{\tsceneP{\uOmega}{b}}{\acesplicedk{m}{n}}{\kappa}
% }
% \end{equation}
% \end{subequations}

% Synthetic ce-constructor validation
% \begin{subequations}\label{rules:ccsyn}
% \begin{equation}\label{rule:ccsyn-var}
% \inferrule{ }{\ccsyn{\Omega, {u} :: {\kappa}}{\cscenev}{u}{u}{\kappa}}
% \end{equation}
% \begin{equation}\label{rule:ccsyn-asc}
% \inferrule{
%   \cvalidKX{\cekappa}{\kappa}\\
%   \ccanaX{\cec}{c}{\kappa}
% }{
%   \ccsynX{\acecasc{\cekappa}{\cec}}{c}{\kappa}
% }
% \end{equation}
% \begin{equation}\label{rule:ccsyn-app}
% \inferrule{
%   \ccsynX{\cec_1}{c_1}{\akdarr{\kappa_2}{u}{\kappa}}\\
%   \ccsynX{\cec_2}{c_2}{\kappa_2}
% }{
%   \ccsynX{\acecapp{\cec_1}{\cec_2}}{\acapp{c_1}{c_2}}{[c_1/u]\kappa}
% }
% \end{equation}
% \begin{equation}\label{rule:ccsyn-unit}
% \inferrule{ }{
%   \ccsynX{\acectriv}{\actriv}{\akunit}
% }
% \end{equation}
% \begin{equation}\label{rule:ccsyn-prl}
% \inferrule{
%   \ccsynX{\cec}{c}{\akdbprod{\kappa_1}{u}{\kappa_2}}
% }{
%   \ccsynX{\acecprl{\cec}}{\acprl{c}}{\kappa_1}
% }
% \end{equation}
% \begin{equation}\label{rule:ccsyn-prr}
% \inferrule{
%   \ccsynX{\cec}{c}{\akdbprod{\kappa_1}{u}{\kappa_2}}
% }{
%   \ccsynX{\acecprr{\cec}}{\acprr{c}}{[\acprl{c}/u]\kappa_2}
% }
% \end{equation}
% \begin{equation}\label{rule:ccsyn-parr}
% \inferrule{
%   \ccanaX{\ctau_1}{\tau_1}{\akty}\\
%   \ccanaX{\ctau_2}{\tau_2}{\akty}
% }{
%   \ccsynX{\aceparr{\ctau_1}{\ctau_2}}{\aparr{\tau_1}{\tau_2}}{\akty}
% }
% \end{equation}
% \begin{equation}\label{rule:ccsyn-all}
% \inferrule{
%   \cvalidKX{\cekappa}{\kappa}\\
%   \ccana{\Omega, u :: \kappa}{\cscenev}{\ctau}{\tau}{\akty}
% }{
%   \ccsynX{\aceallu{\cekappa}{u}{\ctau}}{\aallu{\kappa}{u}{\tau}}{\akty}
% }
% \end{equation}
% \begin{equation}\label{rule:ccsyn-rec}
% \inferrule{
%   \ccana{\Omega, t :: \akty}{\cscenev}{\ctau}{\tau}{\akty}
% }{
%   \ccsynX{\acerec{t}{\ctau}}{\arec{t}{\tau}}{\akty}
% }
% \end{equation}
% \begin{equation}\label{rule:ccsyn-prod}
% \inferrule{
%   \{\ccanaX{\ctau_i}{\tau_i}{\akty}\}_{1 \leq i \leq n}
% }{
%   \ccsynX{\aceprod{\labelset}{\mapschema{\ctau}{i}{\labelset}}}{\aprod{\labelset}{\mapschema{\tau}{i}{\labelset}}}{\akty}
% }
% \end{equation}
% \begin{equation}\label{rule:ccsyn-sum}
% \inferrule{
%   \{\ccanaX{\ctau_i}{\tau_i}{\akty}\}_{1 \leq i \leq n}
% }{
%   \ccsynX{\acesum{\labelset}{\mapschema{\ctau}{i}{\labelset}}}{\asum{\labelset}{\mapschema{\tau}{i}{\labelset}}}{\akty}
% }
% \end{equation}
% \begin{equation}\label{rule:ccsyn-stat}
% \inferrule{ }{
%   \ccsyn{\Omega, X : {\asignature{\kappa}{u}{\tau}}}{\cscenev}{\acemcon{X}}{\amcon{X}}{\kappa}
% }
% \end{equation}
% \begin{equation}\label{rule:ccsyn-spliced}
% \inferrule{
%   \parseUCon{\bsubseq{b}{m}{n}}{\uc}\\
%   \ksyn{\uOmega}{\uc}{c}{\kappa}\\\\
%   \uOmega=\uOmegaEx{\uD}{\uG}{\uMctx}{\Omega_\text{app}}\\
%   \domof{\Omega} \cap \domof{\Omega_\text{app}} = \emptyset
% }{
%   \ccsyn{\Omega}{\tsceneP{\uOmega}{b}}{\acesplicedc{m}{n}{\cekappa}}{c}{\kappa}
% }
% \end{equation}
% \end{subequations}

% Analytic constructor expansion
% \begin{subequations}\label{rules:ccana}
% \begin{equation}\label{rule:ccana-subsume}
% \inferrule{
%   \ccsynX{\cec}{c}{\kappa_1}\\
%   \ksubX{\kappa_1}{\kappa_2}
% }{
%   \ccanaX{\cec}{c}{\kappa_2}
% }
% \end{equation}
% \begin{equation}\label{rule:ccana-sing}
% \inferrule{
%   \kanaX{\cec}{c}{\akty}
% }{
%   \kanaX{\cec}{c}{\aksing{c}}
% }
% \end{equation}
% \begin{equation}\label{rule:ccana-abs}
% \inferrule{
%   \ccana{\Omega, u :: \kappa_1}{\cscenev}{\cec_2}{c_2}{\kappa_2}
% }{
%   \ccanaX{\acecabs{u}{\cec_2}}{\acabs{u}{c_2}}{\akdarr{\kappa_1}{u}{\kappa_2}}
% }
% \end{equation}
% \begin{equation}\label{rule:ccana-pair}
% \inferrule{
%   \ccanaX{\cec_1}{c_1}{\kappa_1}\\
%   \ccanaX{\cec_2}{c_2}{[c_1/u]\kappa_2}
% }{
%   \ccanaX{\acecpair{\cec_1}{\cec_2}}{\acpair{c_1}{c_2}}{\akdbprod{\kappa_1}{u}{\kappa_2}}
% }
% \end{equation}
% \begin{equation}\label{rule:ccana-spliced}
% \inferrule{
%   \parseUCon{\bsubseq{b}{m}{n}}{\uc}\\
%   \kana{\uOmega}{\uc}{c}{\kappa}\\\\
%   \uOmega=\uOmegaEx{\uD}{\uG}{\uMctx}{\Omega_\text{app}}\\
%   \domof{\Omega} \cap \domof{\Omega_\text{app}} = \emptyset
% }{
%   \ccana{\Omega}{\tsceneP{\uOmega}{b}}{\acesplicedc{m}{n}{\cekappa}}{c}{\kappa}
% }
% \end{equation}
% \end{subequations}

% \subsubsection{Bidirectional Candidate Expansion Expression Validation}
% Like the bidirectionally typed expression expansion judgements, the bidirectional ce-expression validation judgements distinguish type synthesis from type analysis. The \emph{synthetic ce-expression validation judgement}, $\csynX{\ce}{e}{\tau}$, and the \emph{analytic ce-expression validation judgement}, $\canaX{\ce}{e}{\tau}$, are defined mutually inductively with Rules (\ref{rules:esyn}) and Rules (\ref{rules:eana}) by Rules (\ref{rules:csyn}) and Rules (\ref{rules:cana}), respectively, as follows.


% \begin{equation}
% \inferrule{
%   \ccanaX{\ctau}{\tau}{\akty}
% }{
%   \cvalidTP{\Omega}{\cscenev}{\ctau}{\tau}
% }
% \end{equation}

% \paragraph{Type Synthesis} \begin{subequations}\label{rules:csynP}
% Synthetic ce-expression validation is governed by the following rules.
% \begin{equation}\label{rule:csynP-var}
%   \inferrule{ }{ 
%     \csynP{\Omega, \Ghyp{x}{\tau}}{\escenev}{x}{x}{\tau}
%   }
% \end{equation}
% \begin{equation}\label{rule:csynP-asc}
%   \inferrule{
%     \cvalidTP{\Omega}{\csfrom{\escenev}}{\ctau}{\tau}\\
%     \canaPX{\ce}{e}{\tau}
%   }{
%     \csynPX{\aceasc{\ctau}{\ce}}{e}{\tau}
%   }
% \end{equation}
% \begin{equation}\label{rule:csynP-let}
%   \inferrule{
%     \csynPX{\ce}{e}{\tau}\\
%     \csynP{\Omega, \Ghyp{x}{\tau}}{\escenev}{\ce'}{e'}{\tau'}
%   }{
%     \csynPX{\aceletsyn{x}{\ce}{\ce'}}{\aeap{\aelam{\tau}{x}{e'}}{e}}{\tau'}
%   }
% \end{equation}
% \begin{equation}\label{rule:csynP-lam}
%   \inferrule{
%     \cvalidTP{\Omega}{\csfrom{\escenev}}{\ctau_1}{\tau_1}\\
%     \csynP{\Omega, \Ghyp{x}{\tau_1}}{\escenev}{\ce}{e}{\tau_2}
%   }{
%     \csynPX{\acelam{\ctau_1}{x}{\ce}}{\aelam{\tau_1}{x}{e}}{\aparr{\tau_1}{\tau_2}}
%   }
% \end{equation}
% \begin{equation}\label{rule:csynP-ap}
%   \inferrule{
%     \csynPX{\ce_1}{e_1}{\aparr{\tau_2}{\tau}}\\
%     \canaPX{\ce_2}{e_2}{\tau_2}
%   }{
%     \csynPX{\aceap{\ce_1}{\ce_2}}{\aeap{e_1}{e_2}}{\tau}
%   }
% \end{equation}
% \begin{equation}\label{rule:csynP-clam}
%   \inferrule{
%     \cvalidK{\Omega}{\csfrom{\escenev}}{\cekappa}{\kappa}\\
%     \csynP{\Omega, u :: \kappa}{\escenev}{\ce}{e}{\tau}
%   }{
%     \csynX{\aceclam{\cekappa}{u}{\ce}}{\aeclam{\kappa}{u}{e}}{\aallu{\kappa}{u}{\tau}}
%   }
% \end{equation}
% \begin{equation}\label{rule:csynP-cap}
%   \inferrule{
%     \csynPX{\ce}{e}{\aallu{\kappa}{u}{\tau}}\\
%     \ccana{\Omega}{\csfrom{\escenev}}{\cec}{c}{\kappa}
%   }{
%     \csynPX{\acecap{\ce}{\cec}}{\aecap{e}{c}}{[c/u]\tau}
%   }
% \end{equation}
% \begin{equation}\label{rule:csynP-unfold}
%   \inferrule{
%     \csynPX{\ce}{e}{\arec{t}{\tau}}
%   }{
%     \csynPX{\aceunfold{\ce}}{\aeunfold{e}}{[\arec{t}{\tau}/t]\tau}
%   }
% \end{equation}
% \begin{equation}\label{rule:csynP-tpl}
%   \inferrule{
%     \ce=\acetpl{\labelset}{\mapschema{\ce}{i}{\labelset}}\\
%     e=\aetpl{\labelset}{\mapschema{e}{i}{\labelset}}\\\\
%     \{\csynPX{\ce_i}{e_i}{\tau_i}\}_{i \in \labelset}
%   }{
%     \csynPX{\ce}{e}{\aprod{\labelset}{\mapschema{\tau}{i}{\labelset}}}
%   }
% \end{equation}
% \begin{equation}\label{rule:csynP-pr}
%   \inferrule{
%     \csynPX{\ce}{e}{\aprod{\labelset, \ell}{\mapschema{\tau}{i}{\labelset}; \mapitem{\ell}{\tau}}}
%   }{
%     \csynPX{\acepr{\ell}{\ce}}{\aepr{\ell}{e}}{\tau}
%   }
% \end{equation}
% \begin{equation}\label{rule:csynP-match}
%   \inferrule{
%     n > 0\\
%     \csynPX{\ce}{e}{\tau}\\
%     \{\crsynPX{\crv_i}{r_i}{\tau}{\tau'}\}_{1 \leq i \leq n}
%   }{
%     \csynPX{\acematchwithb{n}{\ce}{\seqschemaX{\crv}}}{\aematchwith{n}{e}{\seqschemaX{r}}}{\tau'}
%   }
% \end{equation}
% \begin{equation}\label{rule:csynP-mval}
% \inferrule{ }{
%   \csynP{\Omega, X : \asignature{\kappa}{u}{\tau}}{\escenev}{\acemval{X}}{\amval{X}}{[\amcon{X}/u]\tau}
% }
% \end{equation}
% \begin{equation}\label{rule:csynP-splicede}
% \inferrule{
%   \parseUExp{\bsubseq{b}{m}{n}}{\ue}\\
%   \esynP{\uOmega}{\uPsi}{\uPhi}{\ue}{e}{\tau}\\\\
%   \uOmega=\uOmegaEx{\uD}{\uG}{\uMctx}{\Omega_\text{app}}\\
%   \domof{\Omega} \cap \domof{\Omega_\text{app}} = \emptyset
% }{
%   \csynP{\Omega}{\esceneP{\OParams}{\uOmega}{\uPsi}{\uPhi}{b}}{\acesplicede{m}{n}{\ctau}}{e}{\tau}
% }
% \end{equation}
% \end{subequations}

% Rules (\ref{rule:csyn-var}) through (\ref{rule:csyn-match}) are analagous to Rules (\ref{rule:esyn-var}) through (\ref{rule:esyn-match}). Rule (\ref{rule:csyn-splicede}) governs references to spliced unexpanded expressions in synthetic position, and can be understood as described in Sec. \ref{sec:ce-validation-U}.


% \paragraph{Type Analysis} \begin{subequations}\label{rules:canaP}
% Analytic ce-expression validation is governed by the following rules.
% \begin{equation}\label{rule:canaP-subsume}
%   \inferrule{
%     \csynPX{\ce}{e}{\tau}\\
%     \issubtypePX{\tau}{\tau'}
%   }{
%     \canaPX{\ce}{e}{\tau'}
%   }
% \end{equation}
% \begin{equation}\label{rule:canaP-let}
%   \inferrule{
%     \csynPX{\ce}{e}{\tau}\\
%     \canaP{\Omega, \Ghyp{x}{\tau}}{\escenev}{\ce'}{e'}{\tau'}
%   }{
%     \canaPX{\aceletsyn{x}{\ce}{\ce'}}{\aeap{\aelam{\tau}{x}{e'}}{e}}{\tau'}
%   }
% \end{equation}
% \begin{equation}\label{rule:canaP-analam}
%   \inferrule{
%     \canaP{\Gamma, \Ghyp{x}{\tau_1}}{\escenev}{\ce}{e}{\tau_2}
%   }{
%     \canaPX{\aceanalam{x}{\ue}}{\aelam{\tau_1}{x}{e}}{\aparr{\tau_1}{\tau_2}}
%   }
% \end{equation}
% \begin{equation}\label{rule:canaP-clam}
%   \inferrule{
%     \cvalidKX{\cekappa}{\kappa}\\
%     \canaP{\Omega, u :: \kappa}{\escenev}{\ce}{e}{\tau}
%   }{
%     \canaPX{\aceclam{\cekappa}{u}{\ce}}{\aeclam{\kappa}{u}{e}}{\aallu{\kappa}{u}{\tau}}
%   }
% \end{equation}
% \begin{equation}\label{rule:canaP-fold}
%   \inferrule{
%     \canaPX{\ce}{e}{[\arec{t}{\tau}/t]\tau}
%   }{
%     \canaPX{\aceanafold{\ce}}{\aefold{e}}{\arec{t}{\tau}}
%   }
% \end{equation}
% \begin{equation}\label{rule:canaP-tpl}
%   \inferrule{
%     \ce=\acetpl{\labelset}{\mapschema{\ce}{i}{\labelset}}\\
%     e=\aetpl{\labelset}{\mapschema{e}{i}{\labelset}}\\\\
%     \{\canaPX{\ce_i}{e_i}{\tau_i}\}_{i \in \labelset}
%   }{
%     \canaPX{\ce}{e}{\aprod{\labelset}{\mapschema{\tau}{i}{\labelset}}}
%   }
% \end{equation}
% \begin{equation}\label{rule:canaP-in}
%   \inferrule{
%     \ce=\aceanain{\ell}{\ce'}\\
%     e=\aein{\labelset, \ell}{\ell}{\mapschema{\tau}{i}{\labelset}; \mapitem{\ell}{\tau}}{e'}\\\\
%     \canaX{\ce'}{e'}{\tau}
%   }{
%     \canaPX{\ce}{e}{\asum{\labelset, \ell}{\mapschema{\tau}{i}{\labelset}; \mapitem{\ell}{\tau}}}
%     %\left(\shortstack{$\Delta~\Gamma \vdash^{\escenev} $\\$\leadsto$\\$ \Leftarrow $\vspace{-1.2em}}\right)
%     %\eanaX{\auanain{\ell}{\ue}}{\aein{\ell}}{\asum{\labelset, \ell}{\mapschema{\tau}{i}{\labelset}; \mapitem{\ell}{\tau}}}
%   }
% \end{equation}
% \begin{equation}\label{rule:canaP-match}
%   \inferrule{
%     \csynPX{\ce}{e}{\tau}\\
%     \{\cranaPX{\crv_i}{r_i}{\tau}{\tau'}\}_{1 \leq i \leq n}
%   }{
%     \canaPX{\acematchwithb{n}{\ce}{\seqschemaX{\crv}}}{\aematchwith{n}{e}{\seqschemaX{r}}}{\tau'}
%   }
% \end{equation}
% \begin{equation}\label{rule:canaP-splicede}
% \inferrule{
%   \parseUExp{\bsubseq{b}{m}{n}}{\ue}\\
%   \eanaP{\uOmega}{\uPsi}{\uPhi}{\ue}{e}{\tau}\\\\
%   \uOmega=\uOmegaEx{\uD}{\uG}{\uMctx}{\Omega_\text{app}}\\
%   \domof{\Omega} \cap \domof{\Omega_\text{app}} = \emptyset
% }{
%   \canaP{\Omega}{\esceneP{\OParams}{\uOmega}{\uPsi}{\uPhi}{b}}{\acesplicede{m}{n}{\ctau}}{e}{\tau}
% }
% \end{equation}
% \end{subequations}

% Rules (\ref{rule:cana-subsume}) through (\ref{rule:cana-match}) are analagous to Rules (\ref{rule:eana-subsume}) through (\ref{rule:eana-match}). Rule (\ref{rule:cana-splicede}) governs references to spliced unexpanded expressions in analytic position. 

% \subsubsection{Bidirectional Candidate Expansion Rule Validation}
% The \emph{synthetic ce-rule validation judgement} is defined mutually inductively with Rules (\ref{rules:esyn}) by the following rule.
% \begin{equation}\label{rule:crsynP}
% \inferrule{
%   \patTypeP{\Omega'}{p}{\tau}\\
%   \csynP{\Gcons{\Omega}{\Omega'}}{\escenev}{\ce}{e}{\tau'}
% }{
%   \crsynPX{\acematchrule{p}{\ce}}{\aematchrule{p}{e}}{\tau}{\tau'}
% }
% \end{equation}

% The \emph{analytic ce-rule validation judgement} is defined mutually inductively with Rules (\ref{rules:eana}) by the following rule.
% \begin{equation}\label{rule:cranaP}
% \inferrule{
%   \patType{\Omega'}{p}{\tau}\\
%   \canaP{\Gcons{\Omega}{\Omega'}}{\escenev}{\ce}{e}{\tau'}
% }{
%   \cranaPX{\acematchrule{p}{\ce}}{\aematchrule{p}{e}}{\tau}{\tau'}
% }
% \end{equation}

% \subsubsection{Candidate Expansion Pattern Validation}
% The \emph{ce-pattern validation judgement} is inductively defined by the following rules, which are written identically to Rules (\ref{rules:cvalidP-UP}).
% \begin{subequations}\label{rules:cvalidP-P}
% \begin{equation}\label{rule:cvalidP-P-wild}
% \inferrule{ }{
%   \cvalidPP{\uOmegaEx{\emptyset}{\emptyset}{\emptyset}{\emptyset}}{\pscenev}{\acewildp}{\aewildp}{\tau}
% }
% \end{equation}
% \begin{equation}\label{rule:cvalidP-P-fold}
% \inferrule{
%   \cvalidPP{\uOmega}{\pscenev}{\cpv}{p}{[\arec{t}{\tau}/t]\tau}
% }{
%   \cvalidPP{\uOmega}{\pscenev}{\acefoldp{\cpv}}{\aefoldp{p}}{\arec{t}{\tau}}
% }
% \end{equation}
% \begin{equation}\label{rule:cvalidP-P-tpl}
% \inferrule{
%   \cpv=\acetplp{\labelset}{\mapschema{\cpv}{i}{\labelset}}\\
%   p=\aetplp{\labelset}{\mapschema{p}{i}{\labelset}}\\\\
%   \{\cvalidPP{\upctx_i}{\pscenev}{\cpv_i}{p_i}{\tau_i}\}_{i \in \labelset}
% }{
%   \cvalidPP{\Gconsi{i \in \labelset}{\uOmega_i}}{\pscenev}{\cpv}{p}{\aprod{\labelset}{\mapschema{\tau}{i}{\labelset}}}
%   %\cvalidPP{}{\cpv}{p}{}
% %\left(\shortstack{$\vdash^{\pscenev} $\\$\leadsto$\\$ :~\dashVx^{\,\Gconsi{i \in \labelset}{\upctx_i}}$\vspace{-1.2em}}\right)
% }
% \end{equation}
% \begin{equation}\label{rule:cvalidP-P-in}
% \inferrule{
%   \cvalidPP{\uOmega}{\pscenev}{\cpv}{p}{\tau}
% }{
%   \cvalidPP{\uOmega}{\pscenev}{\aceinjp{\ell}{\cpv}}{\aeinjp{\ell}{p}}{\asum{\labelset, \ell}{\mapschema{\tau}{i}{\labelset}; \mapitem{\ell}{\tau}}}
% }
% \end{equation}
% \begin{equation}\label{rule:cvalidP-P-spliced}
% \inferrule{
%   \parseUPat{\bsubseq{b}{m}{n}}{\upv}\\
%   \patExpandsP{\uOmega'}{\uPhi}{\upv}{p}{\tau}
% }{
%   \cvalidPP{\uOmega'}{\pscene{\uOmega}{\uPhi}{b}}{\acesplicedp{m}{n}{\ctau}}{p}{\tau}
% }
% \end{equation}
% \end{subequations}

% Finally, the following theorem establishes that bidirectionally typed expression and rule expansion produces expanded expressions and rules of the appropriate type under the appropriate contexts. These statements must be stated mutually with the corresponding statements about birectional ce-expression and ce-rule validation because the judgements are mutually defined. 

% \begin{theorem}[Typed Expansion] Letting $\uPsi=\uASI{\uA}{\Psi}{\uI}$, if $\uetsmenv{\Delta}{\Psi}$ then all of the following hold:
% \begin{enumerate}
%   \item \begin{enumerate}
%     \item \begin{enumerate}
%       \item If $\esyn{\uDD{\uD}{\Delta}}{\uGG{\uG}{\Gamma}}{\uPsi}{\uPhi}{\ue}{e}{\tau}$ then $\hastypeU{\Delta}{\Gamma}{e}{\tau}$.
%       \item If $\rsyn{\uDD{\uD}{\Delta}}{\uGG{\uG}{\Gamma}}{\uPsi}{\uPhi}{\urv}{r}{\tau}{\tau'}$  then $\ruleType{\Delta}{\Gamma}{r}{\tau}{\tau'}$.
%     \end{enumerate}
%     \item \begin{enumerate}
%       \item If $\eana{\uDD{\uD}{\Delta}}{\uGG{\uG}{\Gamma}}{\uPsi}{\uPhi}{\ue}{e}{\tau}$ and $\istypeU{\Delta}{\tau}$ then $\hastypeU{\Delta}{\Gamma}{e}{\tau}$.
%       \item If $\rana{\uDD{\uD}{\Delta}}{\uGG{\uG}{\Gamma}}{\uPsi}{\uPhi}{\urv}{r}{\tau}{\tau'}$ and $\istypeU{\Delta}{\tau'}$ then $\ruleType{\Delta}{\Gamma}{r}{\tau}{\tau'}$.
%     \end{enumerate}
%   \end{enumerate}
%   \item \begin{enumerate}
%     \item \begin{enumerate}
%       \item If $\csyn{\Delta}{\Gamma}{\esceneUP{\uDD{\uD}{\Delta_\text{app}}}{\uGG{\uG}{\Gamma_\text{app}}}{\uPsi}{\uPhi}{b}}{\ce}{e}{\tau}$ and $\Delta \cap \Delta_\text{app}=\emptyset$ and $\domof{\Gamma} \cap \domof{\Gamma_\text{app}}=\emptyset$ then $\hastypeU{\Dcons{\Delta}{\Delta_\text{app}}}{\Gcons{\Gamma}{\Gamma_\text{app}}}{e}{\tau}$. 
%       \item If $\crsyn{\Delta}{\Gamma}{\esceneUP{\uDD{\uD}{\Delta_\text{app}}}{\uGG{\uG}{\Gamma_\text{app}}}{\uPsi}{\uPhi}{b}}{\crv}{r}{\tau}{\tau'}$ and $\Delta \cap \Delta_\text{app}=\emptyset$ and $\domof{\Gamma} \cap \domof{\Gamma_\text{app}}=\emptyset$ then $\ruleType{\Dcons{\Delta}{\Delta_\text{app}}}{\Gcons{\Gamma}{\Gamma_\text{app}}}{r}{\tau}{\tau'}$.
%     \end{enumerate}
%     \item \begin{enumerate}
%       \item If $\cana{\Delta}{\Gamma}{\esceneUP{\uDD{\uD}{\Delta_\text{app}}}{\uGG{\uG}{\Gamma_\text{app}}}{\uPsi}{\uPhi}{b}}{\ce}{e}{\tau}$ and $\Delta \cap \Delta_\text{app}=\emptyset$ and $\domof{\Gamma} \cap \domof{\Gamma_\text{app}}=\emptyset$ and $\istypeU{\Dcons{\Delta}{\Delta_\text{app}}}{\tau}$ then $\hastypeU{\Dcons{\Delta}{\Delta_\text{app}}}{\Gcons{\Gamma}{\Gamma_\text{app}}}{e}{\tau}$. 
%       \item If $\crana{\Delta}{\Gamma}{\esceneUP{\uDD{\uD}{\Delta_\text{app}}}{\uGG{\uG}{\Gamma_\text{app}}}{\uPsi}{\uPhi}{b}}{\crv}{r}{\tau}{\tau'}$ and $\Delta \cap \Delta_\text{app}=\emptyset$ and $\domof{\Gamma} \cap \domof{\Gamma_\text{app}}=\emptyset$ and $\istypeU{\Dcons{\Delta}{\Delta_\text{app}}}{\tau'}$ then $\ruleType{\Dcons{\Delta}{\Delta_\text{app}}}{\Gcons{\Gamma}{\Gamma_\text{app}}}{r}{\tau}{\tau'}$.
%     \end{enumerate}
%   \end{enumerate}
% \end{enumerate}
% \end{theorem}
% \begin{proof} By mutual rule induction over Rules (\ref{rules:esyn}), Rules (\ref{rules:eana}), Rule (\ref{rule:rsyn}), Rule (\ref{rule:rana}), Rules (\ref{rules:csyn}), Rules (\ref{rules:cana}), Rule (\ref{rule:crsyn}) and Rule (\ref{rule:crana}). In the following, we refer to the induction hypothesis applied to the assumption $\uetsmenv{\Delta}{\Psi}$ as simply the ``IH''. When we apply the induction hypothesis to a different argument, we refer to it as the ``Outer IH''.

% \begin{enumerate}
%   \item In the following, let $\uDelta=\uDD{\uD}{\Delta}$ and $\uGamma=\uGG{\uG}{\Gamma}$. We have:
%   \begin{enumerate}
%     \item \begin{enumerate}
%       \item We induct on the assumption.
%         \begin{byCases}
%           \item[\text{(\ref{rule:esyn-var})}] We have:
%             \begin{pfsteps*}
%               \item $e=x$ \BY{assumption}
%               \item $\Gamma=\Gamma', \Ghyp{x}{\tau}$ \BY{assumption}
%               \item $\hastypeU{\Delta}{\Gamma', \Ghyp{x}{\tau}}{x}{\tau}$ \BY{Rule (\ref{rule:hastypeUP-var})}
%             \end{pfsteps*}
%             \resetpfcounter
%           \item[\text{(\ref{rule:esyn-asc})}] We have:
%             \begin{pfsteps*}
%                \item $\ue=\auasc{\utau}{\ue'}$ \BY{assumption}
%                \item $\expandsTU{\uDelta}{\utau}{\tau}$ \BY{assumption}\pflabel{expandsTU}
%                \item $\eanaX{\ue'}{e}{\tau}$ \BY{assumption}\pflabel{eanaX}
%                \item $\istypeU{\Delta}{\tau}$ \BY{Lemma \ref{lemma:type-expansion-U} on \pfref{expandsTU}}\pflabel{istype}
%                \item $\hastypeU{\Delta}{\Gamma}{e}{\tau}$ \BY{IH, part 1(b)(i) to \pfref{eanaX} and \pfref{istype}}
%              \end{pfsteps*}
%              \resetpfcounter
%           \item[\text{(\ref{rule:esyn-let}) through (\ref{rule:esyn-match})}] In each of these cases, we apply:
%             \begin{itemize}
%               \item Lemma \ref{lemma:type-expansion-U} to or over all type expansion premises.
%               \item The IH, part 1(a)(i) to or over all synthetic typed expression expansion premises.
%               \item The IH, part 1(a)(ii) to or over all synthetic rule expansion premises.
%               \item The IH, part 1(b)(i) to or over all analytic typed expression expansion premises.
%             \end{itemize}
%             We then derive the conclusion by applying Rules (\ref{rules:hastypeUP}) and Rule (\ref{rule:ruleType}) as needed.
%           \item[\text{(\ref{rule:esyn-defuetsm})}] We have:
%             \begin{pfsteps*}
%               \item $\ue=\audefuetsm{\utau'}{\eparse}{\tsmv}{\ue'}$ \BY{assumption}
%               \item $\expandsTU{\uDelta}{\utau'}{\tau'}$ \BY{assumption} \pflabel{expandsTU}
%               \item $\hastypeU{\emptyset}{\emptyset}{\eparse}{\aparr{\tBody}{\tParseResultExp}}$ \BY{assumption}\pflabel{eparse}
%               \item $\esyn{\uDelta}{\uGamma}{\uASI{\ctxUpdate{\uA}{\tsmv}{a}}{\Psi, \xuetsmbnd{a}{\tau'}{\eparse}}{\uI}}{\uPhi}{\ue'}{e}{\tau}$ \BY{assumption}\pflabel{expandsU}
%               \item $\uetsmenv{\Delta}{\Psi}$ \BY{assumption}\pflabel{uetsmenv1}
%               \item $\istypeU{\Delta}{\tau'}$ \BY{Lemma \ref{lemma:type-expansion-U} to \pfref{expandsTU}} \pflabel{istype}
%               \item $\uetsmenv{\Delta}{\Psi, \xuetsmbnd{\tsmv}{\tau'}{\eparse}}$ \BY{Definition \ref{def:ueTSM-def-ctx-formation-UP} on \pfref{uetsmenv1}, \pfref{istype} and \pfref{eparse}}\pflabel{uetsmenv3}
%               \item $\hastypeU{\Delta}{\Gamma}{e}{\tau}$ \BY{Outer IH, part 1(a)(i) on \pfref{uetsmenv3} and \pfref{expandsU}}
%             \end{pfsteps*}
%             \resetpfcounter
%           \item[\text{(\ref{rule:esyn-apuetsm})}] We have:
%             \begin{pfsteps*}
%               \item $\ue=\autsmap{b}{\tsmv}$ \BY{assumption}
%               \item $\uPsi = \uASI{\ctxUpdate{\uA'}{\tsmv}{a}}{\Psi', \xuetsmbnd{a}{\tau}{\eparse}}{\uI}$ \BY{assumption}
%               \item $\encodeBody{b}{\ebody}$ \BY{assumption}
%               \item $\evalU{\eparse(\ebody)}{\inj{\lbltxt{Success}}{\ecand}}$ \BY{assumption}
%               \item $\decodeCondE{\ecand}{\ce}$ \BY{assumption}
%               \item $\cana{\emptyset}{\emptyset}{\esceneUP{\uDelta}{\uGamma}{\uPsi}{\uPhi}{b}}{\ce}{e}{\tau}$ \BY{assumption}\pflabel{cvalidE}
%               \item $\uetsmenv{\Delta}{\Psi}$ \BY{assumption} \pflabel{uetsmenv}
%               \item $\istypeU{\Delta}{\tau}$ \BY{Definition \ref{def:ueTSM-def-ctx-formation-UP} on \pfref{uetsmenv}} \pflabel{istype}
%               \item $\emptyset \cap \Delta = \emptyset$ \BY{finite set intersection identity} \pflabel{delta-cap}
%               \item ${\emptyset} \cap \domof{\Gamma} = \emptyset$ \BY{finite set intersection identity} \pflabel{gamma-cap}
%               \item $\hastypeU{\emptyset \cup \Delta}{\emptyset \cup \Gamma}{e}{\tau}$ \BY{IH, part 2(a)(i) on \pfref{cvalidE}, \pfref{delta-cap}, \pfref{gamma-cap} and \pfref{istype}} \pflabel{penultimate}
%               \item $\hastypeU{\Delta}{\Gamma}{e}{\tau}$ \BY{definition of finite set and finite function union over \pfref{penultimate}}               
%              \end{pfsteps*} 
%              \resetpfcounter
%           \item[\text{(\ref{rule:esyn-implicite})}] We have:
%             \begin{pfsteps*}
%               \item $\ue=\auimplicite{\tsmv}{\ue}$ \BY{assumption}
%               \item $\uPsi=\uASI{\uA' \uplus \vExpands{\tsmv}{a}}{\Psi', \xuetsmbnd{a}{\tau'}{\eparse}}{\uI}$ \BY{assumption}
%               \item $\esyn{\uDelta}{\uGamma}{\uASI{\uA' \uplus \vExpands{\tsmv}{a}}{\Psi', \xuetsmbnd{a}{\tau'}{\eparse}}{\uI \uplus \designate{\tau}{a}}}{\uPhi}{\ue}{e}{\tau}$ \BY{assumption} \pflabel{esyn}
%               \item $\hastypeU{\Delta}{\Gamma}{e}{\tau}$ \BY{IH, part 1(a)(i) on \pfref{esyn}}
%             \end{pfsteps*}
%             \resetpfcounter
%           \item[\text{(\ref{rule:esyn-defuptsm})}] We have:
%             \begin{pfsteps*}
%               \item $\ue=\audefuptsm{\utau'}{\eparse}{\tsmv}{\ue'}$ \BY{assumption}
%               \item $\expandsTU{\uDelta}{\utau'}{\tau'}$ \BY{assumption} \pflabel{expandsTU}
%             %  \item \hastypeU{\emptyset}{\emptyset}{\eparse}{\aparr{\tBody}{\tParseResultExp}} \BY{assumption}\pflabel{eparse}
%               \item $\esyn{\uDelta}{\uGamma}{\uPsi}{\uPhi, \uPhyp{\tsmv}{a}{\tau'}{\eparse}}{\ue'}{e}{\tau}$ \BY{assumption}\pflabel{expandsU}
%             %  \item \uetsmenv{\Delta}{\Psi} \BY{assumption}\pflabel{uetsmenv1}
%             %  \item \istypeU{\Delta}{\tau'} \BY{Lemma \ref{lemma:type-expansion-U} to \pfref{expandsTU}} \pflabel{istype}
%             %  \item \uetsmenv{\Delta}{\Psi, \xuetsmbnd{\tsmv}{\tau'}{\eparse}} \BY{Definition \ref{def:ueTSM-def-ctx-formation} on \pfref{uetsmenv1}, \pfref{istype} and \pfref{eparse}}\pflabel{uetsmenv3}
%               \item $\hastypeU{\Delta}{\Gamma}{e}{\tau}$ \BY{IH, part 1(a)(i) on \pfref{expandsU}}
%             \end{pfsteps*}
%             \resetpfcounter
%           \item[\text{(\ref{rule:esyn-implicitp})}] We have:
%             \begin{pfsteps*}
%               \item $\ue=\auimplicitp{\tsmv}{\ue}$ \BY{assumption}
%               \item $\uPhi=\uASI{\uA \uplus \vExpands{\tsmv}{a}}{\Phi, \xuptsmbnd{a}{\tau'}{\eparse}}{\uI}$ \BY{assumption}
%               \item $\esyn{\uDelta}{\uGamma}{\uPsi}{\uASI{\uA \uplus \vExpands{\tsmv}{a}}{\Phi, \xuptsmbnd{a}{\tau'}{\eparse}}{\uI \uplus \designate{\tau}{a}}}{\ue}{e}{\tau}$ \BY{assumption} \pflabel{esyn}
%               \item $\hastypeU{\Delta}{\Gamma}{e}{\tau}$ \BY{IH, part 1(a)(i) on \pfref{esyn}}
%             \end{pfsteps*}
%             \resetpfcounter
%         \end{byCases}
%       \item We induct on the assumption. There is one case.
%         \begin{byCases}
%           \item[\text{(\ref{rule:rsyn})}] We have:
%             \begin{pfsteps*}
%               \item $\urv=\aumatchrule{\upv}{\ue}$ \BY{assumption}
%               \item $r=\aematchrule{p}{e}$ \BY{assumption}
%               \item $\patExpands{\uGG{\uA'}{\pctx}}{\uPhi}{\upv}{p}{\tau}$ \BY{assumption} \pflabel{patExpands}
%               \item $\esyn{\uDelta}{\uGG{{\uA}\uplus{\uA'}}{\Gcons{\Gamma}{\pctx}}}{\uPsi}{\uPhi}{\ue}{e}{\tau'}$ \BY{assumption} \pflabel{expandsUP}
%               \item $\patType{\pctx}{p}{\tau}$ \BY{Theorem \ref{thm:typed-pattern-expansion-B}, part 1 on \pfref{patExpands}}\pflabel{patType}
%               \item $\hastypeU{\Delta}{\Gcons{\Gamma}{\pctx}}{e}{\tau'}$ \BY{IH, part 1(a)(i) on \pfref{expandsUP}} \pflabel{hasType}
%               \item $\ruleType{\Delta}{\Gamma}{\aematchrule{p}{e}}{\tau}{\tau'}$ \BY{Rule (\ref{rule:ruleType}) on \pfref{patType} and \pfref{hasType}}
%             \end{pfsteps*}
%             \resetpfcounter
%         \end{byCases}
%     \end{enumerate}
%     \item \begin{enumerate}
%       \item We induct on the assumption.
%         \begin{byCases}
%           \item[\text{(\ref{rule:eana-subsume})}] We have:
%             \begin{pfsteps*}
%               \item $\esynX{\ue}{e}{\tau}$ \BY{assumption} \pflabel{esyn}
%               \item $\hastypeU{\Delta}{\Gamma}{e}{\tau}$ \BY{IH, part 1(a)(i) on \pfref{esyn}}
%             \end{pfsteps*}
%           \item[\text{(\ref{rule:eana-let}) through (\ref{rule:eana-match})}] In each of these cases, we apply:
%             \begin{itemize}
%               \item Lemma \ref{lemma:type-expansion-U} to or over all type expansion premises.
%               \item The IH, part 1(a)(i) to or over all synthetic typed expression expansion premises.
%               \item The IH, part 1(a)(ii) to or over all synthetic rule expansion premises.
%               \item The IH, part 1(b)(i) to or over all analytic typed expression expansion premises.
%             \end{itemize}
%             We then derive the conclusion by applying Rules (\ref{rules:hastypeUP}) and Rule (\ref{rule:ruleType}) as needed. 
%           \item[\text{(\ref{rule:eana-defuetsm})}] We have:
%             \begin{pfsteps*}
%               \item $\ue=\audefuetsm{\utau'}{\eparse}{\tsmv}{\ue'}$ \BY{assumption}
%               \item $\expandsTU{\uDelta}{\utau'}{\tau'}$ \BY{assumption} \pflabel{expandsTU}
%               \item $,$ \BY{assumption}\pflabel{eparse}
%               \item $\eana{\uDelta}{\uGamma}{\uPsi, \uShyp{\tsmv}{a}{\tau'}{\eparse}}{\uPhi}{\ue'}{e}{\tau}$ \BY{assumption}\pflabel{expandsU}
%               \item $\uetsmenv{\Delta}{\Psi}$ \BY{assumption}\pflabel{uetsmenv1}
%               \item $\istypeU{\Delta}{\tau'}$ \BY{Lemma \ref{lemma:type-expansion-U} to \pfref{expandsTU}} \pflabel{istype}
%               \item $\uetsmenv{\Delta}{\Psi, \xuetsmbnd{\tsmv}{\tau'}{\eparse}}$ \BY{Definition \ref{def:ueTSM-def-ctx-formation-UP} on \pfref{uetsmenv1}, \pfref{istype} and \pfref{eparse}}\pflabel{uetsmenv3}
%             %  \item \uetsmenv{\Delta}{\Psi} \BY{assumption}\pflabel{uetsmenv1}
%             %  \item \istypeU{\Delta}{\tau'} \BY{Lemma \ref{lemma:type-expansion-U} to \pfref{expandsTU}} \pflabel{istype}
%             %  \item \uetsmenv{\Delta}{\Psi, \xuetsmbnd{\tsmv}{\tau'}{\eparse}} \BY{Definition \ref{def:ueTSM-def-ctx-formation} on \pfref{uetsmenv1}, \pfref{istype} and \pfref{eparse}}\pflabel{uetsmenv3}
%               \item $\hastypeU{\Delta}{\Gamma}{e}{\tau}$ \BY{IH, part 1(b)(i) on \pfref{expandsU}}
%             \end{pfsteps*}
%             \resetpfcounter
%           \item[\text{(\ref{rule:eana-implicite})}] We have:
%             \begin{pfsteps*}
%               \item $\ue=\autsmap{b}{\tsmv}$ \BY{assumption}
%               \item $\uPsi = \uPsi', \uShyp{\tsmv}{a}{\tau}{\eparse}$ \BY{assumption}
%               \item $\encodeBody{b}{\ebody}$ \BY{assumption}
%               \item $\evalU{\eparse(\ebody)}{\inj{\lbltxt{Success}}{\ecand}}$ \BY{assumption}
%               \item $\decodeCondE{\ecand}{\ce}$ \BY{assumption}
%               \item $\cana{\emptyset}{\emptyset}{\esceneUP{\uDelta}{\uGamma}{\uPsi}{\uPhi}{b}}{\ce}{e}{\tau}$ \BY{assumption}\pflabel{cvalidE}
%             %  \item \uetsmenv{\Delta}{\Psi} \BY{assumption} \pflabel{uetsmenv}
%               \item $\emptyset \cap \Delta = \emptyset$ \BY{finite set intersection identity} \pflabel{delta-cap}
%               \item ${\emptyset} \cap \domof{\Gamma} = \emptyset$ \BY{finite set intersection identity} \pflabel{gamma-cap}
%               \item $\hastypeU{\emptyset \cup \Delta}{\emptyset \cup \Gamma}{e}{\tau}$ \BY{IH, part 2(b)(i) on \pfref{cvalidE}, \pfref{delta-cap}, and \pfref{gamma-cap}} \pflabel{penultimate}
%               \item $\hastypeU{\Delta}{\Gamma}{e}{\tau}$ \BY{definition of finite set union over \pfref{penultimate}}               
%              \end{pfsteps*} 
%              \resetpfcounter
%           \item[\text{(\ref{rule:eana-lit})}] We have:
%             \begin{pfsteps*}
%               \item $\ue=\auelit{b}$ \BY{assumption}
%               \item $\uPsi=\uASI{\uA}{\Psi, \xuetsmbnd{a}{\tau}{\eparse}}{\uI \uplus \designate{\tau}{a}}$ \BY{assumption}
%               \item $\encodeBody{b}{\ebody}$ \BY{assumption}
%               \item $\evalU{\ap{\eparse}{\ebody}}{\inj{\lbltxt{Success}}{\ecand}}$ \BY{assumption}
%               \item $\decodeCondE{\ecand}{\ce}$ \BY{assumption}
%               \item $\cana{\emptyset}{\emptyset}{\esceneUP{\uDelta}{\uGamma}{\uASI{\uA}{\Psi, \xuetsmbnd{a}{\tau}{\eparse}}{\uI \uplus \designate{\tau}{a}}}{\uPhi}{b}}{\ce}{e}{\tau}$ \BY{assumption} \pflabel{cvalidE}
%               \item $\emptyset \cap \Delta = \emptyset$ \BY{finite set intersection identity} \pflabel{delta-cap}
%               \item ${\emptyset} \cap \domof{\Gamma} = \emptyset$ \BY{finite set intersection identity} \pflabel{gamma-cap}
%               \item $\hastypeU{\emptyset \cup \Delta}{\emptyset \cup \Gamma}{e}{\tau}$ \BY{IH, part 2(a)(i) on \pfref{cvalidE}, \pfref{delta-cap}, and \pfref{gamma-cap}} \pflabel{penultimate}
%               \item $\hastypeU{\Delta}{\Gamma}{e}{\tau}$ \BY{definition of finite set union over \pfref{penultimate}}
%             \end{pfsteps*}
%             \resetpfcounter
%           \item[\text{(\ref{rule:eana-defuptsm})}] We have:
%             \begin{pfsteps*}
%               \item $\ue=\audefuptsm{\utau'}{\eparse}{\tsmv}{\ue'}$ \BY{assumption}
%               \item $\expandsTU{\uDelta}{\utau'}{\tau'}$ \BY{assumption} \pflabel{expandsTU}
%             %  \item \hastypeU{\emptyset}{\emptyset}{\eparse}{\aparr{\tBody}{\tParseResultExp}} \BY{assumption}\pflabel{eparse}
%               \item $\eana{\uDelta}{\uGamma}{\uPsi}{\uPhi, \uPhyp{\tsmv}{a}{\tau'}{\eparse}}{\ue'}{e}{\tau}$ \BY{assumption}\pflabel{expandsU}
%             %  \item \uetsmenv{\Delta}{\Psi} \BY{assumption}\pflabel{uetsmenv1}
%             %  \item \istypeU{\Delta}{\tau'} \BY{Lemma \ref{lemma:type-expansion-U} to \pfref{expandsTU}} \pflabel{istype}
%             %  \item \uetsmenv{\Delta}{\Psi, \xuetsmbnd{\tsmv}{\tau'}{\eparse}} \BY{Definition \ref{def:ueTSM-def-ctx-formation} on \pfref{uetsmenv1}, \pfref{istype} and \pfref{eparse}}\pflabel{uetsmenv3}
%               \item $\hastypeU{\Delta}{\Gamma}{e}{\tau}$ \BY{IH, part 1(b)(i) on \pfref{expandsU}}
%             \end{pfsteps*}
%             \resetpfcounter
%           \item[\text{(\ref{rule:eana-implicitp})}] We have:
%             \begin{pfsteps*}
%               \item $\ue=\auimplicitp{\tsmv}{\ue}$ \BY{assumption}
%               \item $\uPhi=\uASI{\uA \uplus \vExpands{\tsmv}{a}}{\Phi, \xuptsmbnd{a}{\tau'}{\eparse}}{\uI}$ \BY{assumption}
%               \item $\eana{\uDelta}{\uGamma}{\uPsi}{\uASI{\uA \uplus \vExpands{\tsmv}{a}}{\Phi, \xuptsmbnd{a}{\tau'}{\eparse}}{\uI \uplus \designate{\tau}{a}}}{\ue}{e}{\tau}$ \BY{assumption} \pflabel{esyn}
%               \item $\hastypeU{\Delta}{\Gamma}{e}{\tau}$ \BY{IH, part 1(b)(i) on \pfref{esyn}}
%             \end{pfsteps*}
%             \resetpfcounter
%         \end{byCases}
%       \item We induct on the assumption. There is one case.
%         \begin{byCases}
%           \item[\text{(\ref{rule:rana})}] We have:
%             \begin{pfsteps*}
%               \item $\urv=\aumatchrule{\upv}{\ue}$ \BY{assumption}
%               \item $r=\aematchrule{p}{e}$ \BY{assumption}
%               \item $\patExpands{\uGG{\uA'}{\pctx}}{\uPhi}{\upv}{p}{\tau}$ \BY{assumption} \pflabel{patExpands}
%               \item $\eana{\uDelta}{\uGG{{\uA}\uplus{\uA'}}{\Gcons{\Gamma}{\pctx}}}{\uPsi}{\uPhi}{\ue}{e}{\tau'}$ \BY{assumption} \pflabel{expandsUP}
%               \item $\patType{\pctx}{p}{\tau}$ \BY{Theorem \ref{thm:typed-pattern-expansion-B}, part 1 on \pfref{patExpands}}\pflabel{patType}
%               \item $\hastypeU{\Delta}{\Gcons{\Gamma}{\pctx}}{e}{\tau'}$ \BY{IH, part 1(b)(i) on \pfref{expandsUP}} \pflabel{hasType}
%               \item $\ruleType{\Delta}{\Gamma}{\aematchrule{p}{e}}{\tau}{\tau'}$ \BY{Rule (\ref{rule:ruleType}) on \pfref{patType} and \pfref{hasType}}
%             \end{pfsteps*}
%             \resetpfcounter
%         \end{byCases}
%     \end{enumerate}
%   \end{enumerate}
%   \item In the following, let $\uDelta=\uDD{\uD}{\Delta_\text{app}}$ and $\uGamma=\uGG{\uG}{\Gamma_\text{app}}$ and $\escenev=\esceneUP{\uDelta}{\uGamma}{\uPsi}{\uPhi}{b}$.
%   \begin{enumerate}
%     \item \begin{enumerate}
%       \item We induct on the assumption.
%         \begin{byCases}
%           \item[\text{(\ref{rule:csyn-var})}] We have:
%             \begin{pfsteps*}
%               \item $e=x$ \BY{assumption}
%               \item $\Gamma=\Gamma', \Ghyp{x}{\tau}$ \BY{assumption}
%               \item $\hastypeU{\Delta}{\Gamma', \Ghyp{x}{\tau}}{x}{\tau}$ \BY{Rule (\ref{rule:hastypeUP-var})}
%             \end{pfsteps*}
%             \resetpfcounter 
%           \item[\text{(\ref{rule:csyn-asc})}] We have:
%             \begin{pfsteps*}
%                \item $\ce=\aceasc{\ctau}{\ce'}$ \BY{assumption}
%                \item $\Delta \cap \Delta_\text{app}=\emptyset$ \BY{assumption} \pflabel{delta-disjoint}
%                \item $\domof{\Gamma} \cap \domof{\Gamma_\text{app}}=\emptyset$ \BY{assumption} \pflabel{gamma-disjoint}
%                \item $\cvalidT{\Delta}{\tsfrom{\escenev}}{\ctau}{\tau}$ \BY{assumption}\pflabel{expandsTU}
%                \item $\canaX{\ce'}{e}{\tau}$ \BY{assumption}\pflabel{eanaX}
%                \item $\istypeU{\Delta \cup \Delta_\text{app}}{\tau}$ \BY{Lemma \ref{lemma:candidate-expansion-type-validation} on \pfref{expandsTU}}\pflabel{istype}
%                \item $\hastypeU{\Delta}{\Gamma}{e}{\tau}$ \BY{IH, part 2(b)(i) to \pfref{eanaX}, \pfref{delta-disjoint}, \pfref{gamma-disjoint} and  \pfref{istype}}
%              \end{pfsteps*}
%              \resetpfcounter
%           \item[\text{(\ref{rule:csyn-let}) through (\ref{rule:csyn-match})}] In each of these cases, we apply:
%             \begin{itemize}
%               \item Lemma \ref{lemma:candidate-expansion-type-validation} to or over all ce-type validation premises.
%               \item The IH, part 2(a)(i) to or over all synthetic ce-expression validation premises.
%               \item The IH, part 2(a)(ii) to or over all synthetic ce-rule validation premises.
%               \item The IH, part 2(b)(i) to or over all analytic ce-expression validation premises.
%             \end{itemize}
%             We then derive the conclusion by applying Rules (\ref{rules:hastypeUP}), Rule (\ref{rule:ruleType}), Lemma \ref{lemma:weakening-UP},  the identification convention and exchange as needed.
%           \item[\text{(\ref{rule:csyn-splicede})}] We have:
%             \begin{pfsteps*}
%               \item $\ce=\acesplicede{m}{n}{\ctau}$ \BY{assumption}
%               \item $\parseUExp{\bsubseq{b}{m}{n}}{\ue}$ \BY{assumption}
%               \item $\esyn{\uDelta}{\uGamma}{\uPsi}{\uPhi}{\ue}{e}{\tau}$ \BY{assumption} \pflabel{expands}
%             %  \item $\uetsmenv{\Delta_\text{app}}{\Psi}$ \BY{assumption} \pflabel{uetsmenv}
%               \item $\Delta \cap \Delta_\text{app}=\emptyset$ \BY{assumption} \pflabel{delta-disjoint}
%               \item $\domof{\Gamma} \cap \domof{\Gamma_\text{app}}=\emptyset$ \BY{assumption} \pflabel{gamma-disjoint}
%               \item $\hastypeU{\Delta_\text{app}}{\Gamma_\text{app}}{e}{\tau}$ \BY{IH, part 1(a)(i) on \pfref{expands}} \pflabel{hastype}
%               \item $\hastypeU{\Dcons{\Delta}{\Delta_\text{app}}}{\Gcons{\Gamma}{\Gamma_\text{app}}}{e}{\tau}$ \BY{Lemma \ref{lemma:weakening-UP} over $\Delta$ and $\Gamma$ and exchange on \pfref{hastype}}
%             \end{pfsteps*}
%             \resetpfcounter
%         \end{byCases}
%       \item We induct on the assumption. There is one case.
%         \begin{byCases}
%           \item[\text{(\ref{rule:crsyn})}] We have:
%             \begin{pfsteps*}
%               \item $\crv=\acematchrule{p}{\ce}$ \BY{assumption}
%               \item $r=\aematchrule{p}{e}$ \BY{assumption}
%               \item $\patType{\pctx}{p}{\tau}$ \BY{assumption} \pflabel{patType}
%               \item $\csyn{\Delta}{\Gcons{\Gamma}{\pctx}}{\esceneUP{\uDelta}{\uGamma}{\uPsi}{\uPhi}{b}}{\ce}{e}{\tau'}$ \BY{assumption} \pflabel{cvalidE}
%               \item $\Delta \cap \Delta_\text{app} = \emptyset$ \BY{assumption}\pflabel{delta-disjoint}
%               \item $\domof{\Gamma} \cap \domof{\pctx} = \emptyset$ \BY{identification convention}\pflabel{gamma-disjoint1}
%               \item $\domof{\Gamma_\text{app}} \cap \domof{\pctx} = \emptyset$ \BY{identification convention}\pflabel{gamma-disjoint2}
%               \item $\domof{\Gamma} \cap \domof{\Gamma_\text{app}} = \emptyset$ \BY{assumption}\pflabel{gamma-disjoint3}
%               \item $\domof{\Gcons{\Gamma}{\pctx}} \cap \domof{\Gamma_\text{app}} = \emptyset$ \BY{standard finite set definitions and identities on \pfref{gamma-disjoint1}, \pfref{gamma-disjoint2} and \pfref{gamma-disjoint3}}\pflabel{gamma-disjoint4}
%               \item $\hastypeU{\Dcons{\Delta}{\Delta_\text{app}}}{\Gcons{\Gcons{\Gamma}{\pctx}}{\Gamma_\text{app}}}{e}{\tau'}$ \BY{IH, part 2(a)(i) on \pfref{cvalidE}, \pfref{delta-disjoint} and \pfref{gamma-disjoint4}}\pflabel{hastype}
%               \item $\hastypeU{\Dcons{\Delta}{\Delta_\text{app}}}{\Gcons{\Gcons{\Gamma}{\Gamma_\text{app}}}{\pctx}}{e}{\tau'}$ \BY{exchange of $\pctx$ and $\Gamma_\text{app}$ on \pfref{hastype}}\pflabel{hastype2}
%               \item $\ruleType{\Dcons{\Delta}{\Delta_\text{app}}}{\Gcons{\Gamma}{\Gamma_\text{app}}}{\aematchrule{p}{e}}{\tau}{\tau'}$ \BY{Rule (\ref{rule:ruleType}) on \pfref{patType} and \pfref{hastype2}}
%             \end{pfsteps*}
%             \resetpfcounter
%         \end{byCases}
%     \end{enumerate}
%     \item  \begin{enumerate}
%       \item We induct on the assumption.
%         \begin{byCases}
%           \item[\text{(\ref{rule:cana-subsume})}] We have:
%             \begin{pfsteps*}
%               \item $\csynX{\ce}{e}{\tau}$ \BY{assumption} \pflabel{esyn}
%               \item $\hastypeU{\Delta}{\Gamma}{e}{\tau}$ \BY{IH, part 2(a)(i) on \pfref{esyn}}
%             \end{pfsteps*}
%           \item[\text{(\ref{rule:cana-let}) through (\ref{rule:eana-match})}] In each of these cases, we apply:
%             \begin{itemize}
%               \item Lemma \ref{lemma:candidate-expansion-type-validation} to or over all ce-type validation premises.
%               \item The IH, part 2(a)(i) to or over all synthetic ce-expression validation premises.
%               \item The IH, part 2(a)(ii) to or over all synthetic ce-rule validation premises.
%               \item The IH, part 2(b)(i) to or over all analytic ce-expression validation premises.
%             \end{itemize}
%             We then derive the conclusion by applying Rules (\ref{rules:hastypeUP}), Rule (\ref{rule:ruleType}), Lemma \ref{lemma:weakening-UP},  the identification convention and exchange as needed.
%           \item[\text{(\ref{rule:cana-splicede})}] We have:
%             \begin{pfsteps*}
%               \item $\ce=\acesplicede{m}{n}{\ctau}$ \BY{assumption}
%               \item $\parseUExp{\bsubseq{b}{m}{n}}{\ue}$ \BY{assumption}
%               \item $\eana{\uDelta}{\uGamma}{\uPsi}{\uPhi}{\ue}{e}{\tau}$ \BY{assumption} \pflabel{expands}
%               \item $\istypeU{\Delta \cup \Delta_\text{app}}{\tau}$ \BY{assumption} \pflabel{istype}
%             %  \item $\uetsmenv{\Delta_\text{app}}{\Psi}$ \BY{assumption} \pflabel{uetsmenv}
%               \item $\Delta \cap \Delta_\text{app}=\emptyset$ \BY{assumption} \pflabel{delta-disjoint}
%               \item $\domof{\Gamma} \cap \domof{\Gamma_\text{app}}=\emptyset$ \BY{assumption} \pflabel{gamma-disjoint}
%               \item $\hastypeU{\Delta_\text{app}}{\Gamma_\text{app}}{e}{\tau}$ \BY{IH, part 1(b)(i) on \pfref{expands}, \pfref{delta-disjoint}, \pfref{gamma-disjoint} and \pfref{istype}} \pflabel{hastype}
%               \item $\hastypeU{\Dcons{\Delta}{\Delta_\text{app}}}{\Gcons{\Gamma}{\Gamma_\text{app}}}{e}{\tau}$ \BY{Lemma \ref{lemma:weakening-UP} over $\Delta$ and $\Gamma$ and exchange on \pfref{hastype}}
%             \end{pfsteps*}
%             \resetpfcounter
%         \end{byCases}
%       \item We induct on the assumption. There is one case.
%         \begin{byCases}
%           \item[\text{(\ref{rule:crana})}] We have:    
%             \begin{pfsteps*}
%                 \item $\crv=\acematchrule{p}{\ce}$ \BY{assumption}
%                 \item $r=\aematchrule{p}{e}$ \BY{assumption}
%                 \item $\patType{\pctx}{p}{\tau}$ \BY{assumption} \pflabel{patType}
%                 \item $\cana{\Delta}{\Gcons{\Gamma}{\pctx}}{\esceneUP{\uDelta}{\uGamma}{\uPsi}{\uPhi}{b}}{\ce}{e}{\tau'}$ \BY{assumption} \pflabel{cvalidE}
%                 \item $\istypeU{\Delta \cup \Delta_\text{app}}{\tau'}$ \BY{assumption} \pflabel{istype}
%                 \item $\domof{\Gamma} \cap \domof{\Gamma_\text{app}} = \emptyset$ \BY{assumption}\pflabel{gamma-disjoint3}
%                 \item $\Delta \cap \Delta_\text{app} = \emptyset$ \BY{assumption}\pflabel{delta-disjoint}
%                 \item $\domof{\Gamma} \cap \domof{\pctx} = \emptyset$ \BY{identification convention}\pflabel{gamma-disjoint1}
%                 \item $\domof{\Gamma_\text{app}} \cap \domof{\pctx} = \emptyset$ \BY{identification convention}\pflabel{gamma-disjoint2}
%                 \item $\domof{\Gcons{\Gamma}{\pctx}} \cap \domof{\Gamma_\text{app}} = \emptyset$ \BY{standard finite set definitions and identities on \pfref{gamma-disjoint1}, \pfref{gamma-disjoint2} and \pfref{gamma-disjoint3}}\pflabel{gamma-disjoint4}
%                 \item $\hastypeU{\Dcons{\Delta}{\Delta_\text{app}}}{\Gcons{\Gcons{\Gamma}{\pctx}}{\Gamma_\text{app}}}{e}{\tau'}$ \BY{IH, part 2(b)(i) on \pfref{cvalidE}, \pfref{delta-disjoint}, \pfref{gamma-disjoint4} and \pfref{istype}}\pflabel{hastype}
%                 \item $\hastypeU{\Dcons{\Delta}{\Delta_\text{app}}}{\Gcons{\Gcons{\Gamma}{\Gamma_\text{app}}}{\pctx}}{e}{\tau'}$ \BY{exchange of $\pctx$ and $\Gamma_\text{app}$ on \pfref{hastype}}\pflabel{hastype2}
%                 \item $\ruleType{\Dcons{\Delta}{\Delta_\text{app}}}{\Gcons{\Gamma}{\Gamma_\text{app}}}{\aematchrule{p}{e}}{\tau}{\tau'}$ \BY{Rule (\ref{rule:ruleType}) on \pfref{patType} and \pfref{hastype2}}
%               \end{pfsteps*}
%               \resetpfcounter

%         \end{byCases}
%     \end{enumerate}
%   \end{enumerate}
% \end{enumerate}

% We must now show that the induction is well-founded. All applications of the IH are on subterms except the following.  

% \begin{itemize}
% \item The only cases in the proof of part 1 that invoke the IH, part 2 are Case (\ref{rule:esyn-apuetsm}) in the proof of part 1(a)(i) and Case (\ref{rule:eana-lit}) in the proof of part 1(b)(i). The only cases in the proof of part 2 that invoke the IH, part 1 are Case (\ref{rule:csyn-splicede}) in the proof of part 2(a)(i) and Case (\ref{rule:cana-splicede}) in the proof of part 2(b)(i). We can show that the following metric on the judgements that we induct on is stable in one direction and strictly decreasing in the other direction:
% \begin{align*}
% \sizeof{\esyn{\uDelta}{\uGamma}{\uPsi}{\uPhi}{\ue}{e}{\tau}} & = \sizeof{\ue}\\
% \sizeof{\eana{\uDelta}{\uGamma}{\uPsi}{\uPhi}{\ue}{e}{\tau}} & = \sizeof{\ue}\\
% \sizeof{\csyn{\Delta}{\Gamma}{\esceneUP{\uDelta}{\uGamma}{\uPsi}{\uPhi}{b}}{\ce}{e}{\tau}} & = \sizeof{b}\\
% \sizeof{\cana{\Delta}{\Gamma}{\esceneUP{\uDelta}{\uGamma}{\uPsi}{\uPhi}{b}}{\ce}{e}{\tau}} & = \sizeof{b}
% \end{align*}
% where $\sizeof{b}$ is the length of $b$ and $\sizeof{\ue}$ is the sum of the lengths of the ueTSM literal bodies in $\ue$,
% \begin{align*}
% \sizeof{\ux} & = 0\\
% \sizeof{\auasc{\utau}{\ue}} & = \sizeof{\ue}\\
% \sizeof{\auletsyn{\ux}{\ue}{\ue'}} & = \sizeof{\ue} + \sizeof{\ue'}\\
% \sizeof{\auanalam{\ux}{\ue}} & = \sizeof{\ue}\\
% \sizeof{\aulam{\utau}{\ux}{\ue}} &= \sizeof{\ue}\\
% \sizeof{\auap{\ue_1}{\ue_2}} & = \sizeof{\ue_1} + \sizeof{\ue_2}\\
% \sizeof{\autlam{\ut}{\ue}} & = \sizeof{\ue}\\
% \sizeof{\autap{\ue}{\utau}} & = \sizeof{\ue}\\
% \sizeof{\auanafold{\ue}} & = \sizeof{\ue}\\
% \sizeof{\auunfold{\ue}} & = \sizeof{\ue}\\
% %\end{align*}
% %\begin{align*}
% \sizeof{\autpl{\labelset}{\mapschema{\ue}{i}{\labelset}}} & = \sum_{i \in \labelset} \sizeof{\ue_i}\\
% \sizeof{\aupr{\ell}{\ue}} & = \sizeof{\ue}\\
% \sizeof{\auanain{\ell}{\ue}} & = \sizeof{\ue}\\
% %\sizeof{\aucase{\labelset}{\utau}{\ue}{\mapschemab{\ux}{\ue}{i}{\labelset}}} & = \sizeof{\ue} + \sum_{i \in \labelset} \sizeof{\ue_i}\\
% \sizeof{\aumatchwithb{n}{\ue}{\seqschemaX{\urv}}} & = \sizeof{\ue} + \sum_{1 \leq i \leq n} \sizeof{r_i}\\
% \sizeof{\audefuetsm{\utau}{\eparse}{\tsmv}{\ue}} & = \sizeof{\ue}\\
% \sizeof{\auimplicite{\tsmv}{\ue}} & = \sizeof{\ue}\\
% \sizeof{\autsmap{b}{\tsmv}} & = \sizeof{b}\\
% \sizeof{\auelit{b}} & = \sizeof{b}\\
% \sizeof{\audefuptsm{\utau}{\eparse}{\tsmv}{\ue}} & = \sizeof{\ue}\\
% \sizeof{\auimplicitp{\tsmv}{\ue}} & = \sizeof{\ue}
% \end{align*}
% and $\sizeof{r}$ is defined as follows:
% \begin{align*}
% \sizeof{\aumatchrule{\upv}{\ue}} & = \sizeof{\ue}
% \end{align*}

% Going from part 1 to part 2, the metric remains stable:
% \begin{align*}
%  & \sizeof{\esyn{\uDelta}{\uGamma}{\uPsi}{\uPhi}{\autsmap{b}{\tsmv}}{e}{\tau}}\\
% =& \sizeof{\eana{\uDelta}{\uGamma}{\uPsi}{\uPhi}{\auelit{b}}{e}{\tau}}\\
% =& \sizeof{\cana{\emptyset}{\emptyset}{\esceneUP{\uDelta}{\uGamma}{\uPsi}{\uPhi}{b}}{\ce}{e}{\tau}}\\
% =&\sizeof{b}\end{align*}

% Going from part 2 to part 1, in each case we have that $\parseUExp{\bsubseq{b}{m}{n}}{\ue}$ and the IH is applied to the judgements $\esyn{\uDelta}{\uGamma}{\uPsi}{\uPhi}{\ue}{e}{\tau}$ and $\eana{\uDelta}{\uGamma}{\uPsi}{\uPhi}{\ue}{e}{\tau}$, respectively. Because the metric is stable when passing from part 1 to part 2, we must have that it is strictly decreasing in the other direction:
% \[\sizeof{\esyn{\uDelta}{\uGamma}{\uPsi}{\uPhi}{\ue}{e}{\tau}} < \sizeof{\csyn{\Delta}{\Gamma}{\esceneUP{\uDelta}{\uGamma}{\uPsi}{\uPhi}{b}}{\acesplicede{m}{n}{\ctau}}{e}{\tau}}\]
% and
% \[\sizeof{\eana{\uDelta}{\uGamma}{\uPsi}{\uPhi}{\ue}{e}{\tau}} < \sizeof{\cana{\Delta}{\Gamma}{\esceneUP{\uDelta}{\uGamma}{\uPsi}{\uPhi}{b}}{\acesplicede{m}{n}{\ctau}}{e}{\tau}}\]
% i.e. by the definitions above, 
% \[\sizeof{\ue} < \sizeof{b}\]

% This is established by appeal to Condition \ref{condition:body-subsequences}, which states that subsequences of $b$ are no longer than $b$, and the following condition, which states that an unexpanded expression constructed by parsing a textual sequence $b$ is strictly smaller, as measured by the metric defined above, than the length of $b$, because some characters must necessarily be used to delimit each literal body.
% \begin{condition}[Expression Parsing Monotonicity]\label{condition:body-parsing-B} If $\parseUExp{b}{\ue}$ then $\sizeof{\ue} < \sizeof{b}$.\end{condition}

% Combining Conditions \ref{condition:body-subsequences} and \ref{condition:body-parsing-B}, we have that $\sizeof{\ue} < \sizeof{b}$ as needed.
% \item In Case (\ref{rule:eana-subsume}) of the proof of part 1(b)(i), we apply the IH, part 1(a)(i), with $\ue=\ue$. This is well-founded because all applications of the IH, part 1(b)(i) elsewhere in the proof are on strictly smaller terms.
% \item Similarly, in Case (\ref{rule:cana-subsume}) of the proof of part 2(b)(i), we apply the IH, part 2(a)(i), with $\ce=\ce$. This is well-founded because all applications of the IH, part 2(b)(i) elsewhere in the proof are on strictly smaller terms.
% \end{itemize}
% \end{proof} 



%\renewcommand{\baselinestretch}{1.0}\normalsize
% By default \bibsection is \chapter*, but we really want this to show
% up in the table of contents and pdf bookmarks.
\renewcommand{\bibsection}{\chapter*{\bibname}\addcontentsline{toc}{chapter}{Bibliography}}
% \renewcommand{\bibpreamble}{\todolater{List conference abbreviations.}\\
% \todolater{Remove extraneous nonsense from entries.}}
\bibliographystyle{plainnat}
\bibliography{../papers/research} %your bib file


\end{document}

% !TEX root = omar-thesis.tex

\section{Motivating Examples}\label{sec:motivating-examples}
To further motivate our work, we will now provide a number of examples of derived syntactic forms that decrease the syntactic cost of working with various data structures. We cover the first two examples -- regular expressions and lists -- in substantial detail. We will refer back to these examples throughout this work. We then more concisely survey a number of other examples, grouped into  categories, to establish the broad applicability of our contributions. %We assume that the reader is familiar with a typed functional language like Standard ML (cf. \cite{harper1997programming}).

\subsection{Regular Expressions}\label{sec:syntax-examples-regexps}
Let us take the perspective of a regular expression library provider (we assume the reader has some familiarity with regular expressions \cite{Thompson:1968:PTR:363347.363387}). The abstract syntax of {regexes}, $r$, over strings, $s$, is specified below:\[r ::= \textbf{empty} ~|~ \textbf{str}(s) ~|~ \textbf{seq}(r; r) ~|~ \textbf{or}(r; r) ~|~ \textbf{star}(r)\]

\paragraph{Recursive Sums}
One way to express this abstract syntax is by defining a recursive sum type \cite{pfpl}. In VerseML, a labeled recursive sum type can be defined like this:

\begin{figure}[ht]
\begin{lstlisting}[numbers=none]
type Rx = Empty | Str of string | Seq of Rx * Rx | 
          Or of Rx * Rx | Star of Rx | Group of Rx
\end{lstlisting}
\caption{Definition of the recursive sum type \li{Rx}}
\label{fig:datatype-rx}
\end{figure}

The abstract syntax of regexes is too verbose to be practical  in all but the most trivial examples, so the POSIX standard specifies a more concise concrete syntax \cite{STD95954}. A number of programming languages support derived syntax for regular expressions based on this standard, e.g. Perl \cite{books/daglib/0028711}. Let us consider a hypothetical dialect of ML called ML+Rx (perhaps constructed using a tool like Camlp4, discussed in Sec. \ref{sec:direct-syntax-extension}) that similarly builds in derived forms for regexes (we will compare VerseML to ML+Rx in later chapters). ML+Rx supports \emph{regex literals}, e.g.
\begin{lstlisting}[numbers=none]
/SURLA|T|G|CEURL/
\end{lstlisting}
desugars to:
\begin{lstlisting}[numbers=none]
Or(Str "SSTRAESTR", Or(Str "SSTRTESTR", Or(Str "SSTRGESTR", Str "SSTRCESTR")))
\end{lstlisting}

ML+Rx also supports \emph{spliced subexpressions} in regex literals. For example, the function \li{example_rx} shown below constructs a regex by splicing in a string, \li{name}, and another regex, \li{ssn}:

\begin{figure}[ht]
\begin{lstlisting}[numbers=none]
let ssn = /SURL\d\d\d-\d\d-\d\d\d\dEURL/
fun example_rx(name : string) => /SURL@EURLnameSURL: %EURLssn/
\end{lstlisting}
\caption{An example of syntax that supports spliced subexpressions.}
\label{fig:derived-spliced-subexpressions}
\end{figure}
The prefix \li{@} indicates that \lstinline{name} should be spliced in as a string, and the prefix \li. 

To splice in an expression that does not take the form of a variable, e.g. a function call, we can delimit it with parentheses:
\begin{lstlisting}[numbers=none]
/SURL@(EURLcapitalize nameSURL): %EURLssn/
\end{lstlisting}

Finally, ML+Rx allows us to pattern match on a value of type \li{Rx} using derived pattern syntax. For example:
\begin{figure}[ht]
\begin{lstlisting}[numbers=none]
fun read_example_rx(r : Rx) => 
  match r with 
    /SURL@EURLnameSURL: %EURLssn/ => Some (name, ssn)
  | _ => None\end{lstlisting}
\caption{An example of derived pattern syntax.}
\label{fig:derived-pattern-syntax}
\end{figure}

This expression desugars to:
\begin{lstlisting}[numbers=none]
fun read_example_rx(r : Rx) => 
  match r with
    Seq(Str(name), Seq(Str "SSTR: ESTR", ssn)) => Some (name, ssn)
  | _ => None
\end{lstlisting}
\paragraph{Abstract Types} Encoding regexes as values of type \li{Rx} is straightforward, but there are reasons why one might not wish to expose this encoding to clients directly. First, regexes are usually identified up to their reduction to a normal form. For example, $\textbf{seq}(\textbf{empty}, r)$ has normal form $r$. It can be useful for regexes with the same normal form to be  indistinguishable from the perspective of client code. Second, it can be useful for performance reasons to maintain additional data alongside regexes (e.g. a corresponding finite automaton), but one would not want to expose this ``implementation detail'' to clients. In fact, there may be many ways to represent regular expression patterns, each with different performance trade-offs, so we would like to provide clients with a choice of implementations. For these reasons, another approach in VerseML, as in ML, is to abstract over the choice of representation using  the module system's support for abstract types. In particular, we can define the \emph{module signature} \li{RX} where the type of patterns, \lstinline{t}, is held abstract:
%Notice that it exposes an interface otherwise  to the one available using a case type:

\begin{figure}[ht]
\begin{lstlisting}[deletekeywords={case},numbers=none]
signature RX = sig {
  type t
  val Empty : t
  val Str : string -> t
  val Seq : t * t -> t
  val Or : t * t -> t
  val Star : t -> t
  val Group : t -> t
  val case : (
    t -> {
    	Empty : 'a,
    	Str : string -> 'a,
    	Seq : t * t -> 'a,
    	Or : t * t -> 'a,
    	Star : t -> 'a,
    	Group : t -> 'a
    } -> 'a
}
\end{lstlisting}
\caption{Definition of the \lstinline{RX} signature.}
\label{fig:signature-RX}
\end{figure}

 Clients of any module \lstinline{R} that has been sealed against \lstinline{RX}, written \lstinline{R :> RX}, manipulate patterns as values of the type \li{R.t} using the interface described by this signature. The identity of the type \lstinline{R.t} is held abstract outside the module during typechecking (i.e. it acts as a newly generated type). As a result, the burden of proving that there is no way to use the case analysis function to distinguish patterns with the same normal form is local to the module, and implementation details do not escape (and can thus evolve freely). %The details are standard and not particularly relevant for our purposes, so we omit them here.

 \todo{talk about module-parameterized derived syntactic forms for this}

 \todo{talk about pattern matching over values of abstract type}

\subsection{Lists}\label{sec:syntax-examples-lists}
\todo{write this (Spring 2016)}
\subsection{Sets, Maps, Vectors and Other Containers}\label{sec:syntax-examples-containers}
\todo{write this (Spring 2016)}
\subsection{HTML and Other Web Languages}\label{sec:syntax-examples-html}
\todo{write this; cite Ur/Web (Spring 2016)}
\subsection{Dates, URLs and Other Standardized Formats}\label{sec:syntax-examples-dates}
\todo{write this (Spring 2016)}
\subsection{Query Languages} The language of regular expressions can be considered a query language over strings. There are many other query languages that focus on different types of data, e.g. XQuery for XML trees, or that are associated with different database technologies, e.g. SQL for relational databases. \todo{finish this (Spring 2016)} 
\subsection{Monadic Commands}\label{sec:syntax-examples-monads}
\todo{write this; cite Bob's blog (Spring 2016)}
\subsection{Quasiquotation and Object Language Syntax}\label{sec:syntax-examples-quasiquotation}
\todo{write this (Spring 2016)}
\subsection{Grammars}\label{sec:syntax-examples-grammars}
\todo{write this (Spring 2016)}
\subsection{Mathematical and Scientific Notations}\label{sec:syntax-examples-math-science}
\subsubsection{SMILES: Chemical Notation}
\todo{write this; cite SMILES \url{https://en.wikipedia.org/wiki/Simplified_molecular-input_line-entry_system} (Spring 2016)}
\subsubsection{\TeX~Mathematical Formula Notation}
\todo{write this (Spring 2016)}
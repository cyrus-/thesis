% !TEX root = omar-thesis.tex

\section{Preliminaries}\label{sec:preliminaries}
This work is rooted in the tradition of full-scale functional languages with non-trivial type structure like ML and Haskell (as might have been obvious from Chapter \ref{chap:intro}). Familiarity with basic concepts in these languages, e.g. variables, types, functions, tuples, records,  recursive datatypes and structural pattern matching, is assumed throughout this work. Readers who are not familiar with these concepts are encouraged to consult the early chapters of an introductory text like Harper's \emph{Programming in Standard ML} \cite{harper1997programming} (a working draft can be found online). We briefly discuss integration of TSMs into languages from other language design traditions in Sec. \ref{sec:integration}.

Chapter \ref{chap:ptsms} and some of the motivating examples given below, consider questions of integration with an ML-style module system, so readers with experience in a language without such a module system (e.g. Haskell) are also advised to review the relevant chapters in \emph{Programming in Standard ML} \cite{harper1997programming} before delving into these sections.

The formal systems that we will construct are specified within the metatheoretic framework of type theory. More specifically, we assume that abstract syntax trees (ASTs), abstract binding trees (which enrich ASTs with the notions of binding and scope), substitution, implicit identification of ABTs up to $\alpha$-equivalence (i.e. the \emph{identification convention}), structural induction and rule induction are defined as described in Harper's \emph{Practical Foundations for Programming Languages, Second Edition} (\emph{PFPL}) \cite{pfpl} (a working draft can be found online), except as otherwise stated. Familiarity with other formal accounts of type systems, e.g. Pierce's \emph{Types and Programming Languages} (\emph{TAPL}) \cite{tapl}, should also suffice. This document is organized so as to be readable even if the sections defining formal systems are skipped (although much precision will, of course, be lost).


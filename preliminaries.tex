% !TEX root = omar-thesis.tex

\section{Preliminaries}\label{sec:preliminaries}
\vspace{-3px}
This work is rooted in the tradition of full-scale functional languages like Standard ML, OCaml and Haskell (as might have been obvious from Chapter \ref{chap:intro}.) Familiarity with basic concepts in these languages, e.g. variables, types, polymorphic and recursive functions, tuples, records, recursive datatypes and structural pattern matching, is assumed throughout this work. Readers who are not familiar with these concepts are encouraged to consult the early chapters of an introductory text like Harper's \emph{Programming in Standard ML} \cite{harper1997programming} (a working draft can be found online.) We discuss integrating TSMs into languages from other design traditions in Sec. \ref{sec:integration}.

In Chapter \ref{chap:ptsms} and onward, as well as in some of the motivating examples below, we also assume basic familiarity with ML-style module systems. Readers with experience in a language without such a module system (e.g. Haskell) are also advised to consult the relevant chapters in \emph{Programming in Standard ML} \cite{harper1997programming} as needed. We distinguish \emph{modules}, which are language constructs, from \emph{libraries}, which are extralinguistic packaging constructs managed by some implementation-defined compilation manager (e.g. \li{CM}, distributed with Standard ML of New Jersey (SML/NJ) \cite{DBLP:conf/plilp/AppelM91}.) A library can export any number of modules, signatures and TSM definitions.

The formal systems that we will consider are defined within the metatheoretic framework of type theory. More specifically, we will assume that abstract binding trees (ABTs, which enrich abstract syntax trees with the notions of binding and scope, as discussed in Chapter \ref{chap:intro}), renaming, alpha-equivalence, substitution, structural induction and rule induction are defined as described in Harper's \emph{Practical Foundations for Programming Languages, Second Edition} (\emph{PFPL}) \cite{pfpl}. Familiarity with other formal accounts of type systems, e.g. Pierce's \emph{Types and Programming Languages} (\emph{TAPL}) \cite{tapl}, should also suffice.% This document is organized so as to be readable even if the sections defining formal systems are skipped entirely, although much precision will, of course, be lost.


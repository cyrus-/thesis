% !TEX root = omar-thesis.tex

\section{Preliminaries}
This work is rooted in the tradition of full-scale functional languages with non-trivial type structure like ML and Haskell (as might have been obvious from the exposition in Chapter \ref{chap:intro}). Familiarity with basic concepts in these languages, e.g. variables, types, functions, tuples, records,  recursive datatypes and nested pattern matching, is assumed throughout this work. Readers who are not familiar with these concepts are encouraged to consult the early chapters of an introductory text like Harper's \emph{Programming in Standard ML} \cite{harper1997programming} (a working draft can be found online). We briefly discuss integration of TSMs and TSLs into languages from other language design traditions in Sec. \ref{sec:integration}.

Chapter \ref{sec:tsms-parameterized} and Chapter \ref{chap:tsls}, and some of the motivating examples given below, consider questions of integration with an ML-style module system, so readers with experience in a language without such a module system (e.g. Haskell) are also advised to review the relevant chapters in \emph{Programming in Standard ML} \cite{harper1997programming} before delving into these portions of these chapters.

The formal systems that we will construct in later chapters are specified within the metatheoretic framework of type theory. More specifically, we assume familiarity with fundamental background concepts (e.g. abstract binding trees, substitution, implicit identification of terms up to $\alpha$-equivalence, structural induction and rule induction) covered in detail in Harper's \emph{Practical Foundations for Programming Languages, Second Edition} (\emph{PFPL}) \cite{pfpl} (a working draft can be found online). Familiarity with other formal accounts of type systems, e.g. Pierce's \emph{Types and Programming Languages} (\emph{TAPL}) \cite{tapl}, should also suffice. This document is organized so as to be readable even if the sections describing formal systems are skipped (although some precision will, of course, be lost).


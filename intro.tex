% !TEX root = omar-thesis.tex
\chapter{Introduction}\label{chap:intro}
%\vspace{-5px}
% \begin{quote}\textit{The recent development of programming languages suggests that the simul\-taneous achievement of simplicity 
% and generality in language design is a serious unsolved 
% problem.}\begin{flushright}--- John Reynolds (1970) \cite{Reynolds70}\end{flushright}
% \end{quote}
%\begin{quote}
%\textit{Try to imagine that you are a tree. How do you want to look out here?}
%\textit{You want your tree to have some character.}
%\begin{flushright} --- Bob Ross, \emph{The Joy of Painting}\end{flushright}
%\end{quote}

%\vspace{-5px}
\section{Motivation}\label{sec:intro-motivation}

%Programming languages come in many sizes. The smallest languages -- for example, the various ``lambda calculi'' -- isolate language primitives of interest for the benefit of students, researchers and language designers interested in studying their mathematical properties. These studies inform the design of ``full-scale'' programming 
%\footnote{Throughout this work, words and phrases that should be read as having an intuitive or informal meaning, rather than a strict mathematical meaning, will be introduced with quotation marks.} 
% languages, which combine several such primitives, or generalizations thereof. Full-scale languages are interesting objects of formal study in their own right. They also serve as useful tools for software developers, allowing them to construct, reason about and modularly organize large software systems.

There are many ways to draw a tree. For example, consider these three drawings:


\begin{subequations}
\begin{equation}\label{simple-example-txt-form}
\texttt{1 / 2\textasciicircum3}
\end{equation}
\begin{equation}\label{simple-example-sty-form}
\frac{\numintro{1}}{{\numintro{2}^{\numintro{3}}}}
\end{equation}

\begin{equation}\label{simple-example-op-form}\adiv{\anumintro{1}}{
	\apow{\anumintro{2}}{\anumintro{3}}
}\end{equation}
\end{subequations}

\noindent
These are different drawings of a single tree -- an \emph{abstract syntax tree (AST)}, in particular, of the sort defined by the syntax chart in Figure \ref{fig:simple-example}.\footnote{Some familiarity with ASTs and inductively defined judgements is preliminary to this work. See Sec. \ref{sec:preliminaries} for references and a more thorough discussion of necessary preliminaries.} ASTs of this sort are the expressions of a programming language, $\simplelang$, that can be used to perform simple arithmetic calculations with numbers. 
As the syntax chart suggests, every $\simplelang$ expression can be drawn in \emph{textual form}, like Drawing (\ref{simple-example-txt-form}), \emph{stylized form}, like Drawing (\ref{simple-example-sty-form}), or \emph{operational form}, like Drawing (\ref{simple-example-op-form}).


% In particular, let us consider a simple programming language, $\simplelang$, for performing arithmetic calculations with numbers. The expressions of $\simplelang$ are \emph{abstract syntax trees (ASTs)} of a sort defined by the syntax chart in Figure \ref{fig:simple-example}.\footnote{Familiarity with abstract syntax trees is preliminary to this work (see Sec. \ref{sec:preliminaries} for other preliminaries.)}  For example, the following expression is drawn in stylized form:
% The same expression is drawn in textual form as follows:


% \noindent
% and in operational form as follows:

 The semantics of $\simplelang$ identifies ASTs {structurally}, i.e. independently of how they are drawn. For instance, the semantics defines a judgement $\isvalU{e}$, which characterizes certain expressions as \emph{values} (as distinct from  expressions that can be further simplified.) This judgement is defined by a single inference rule, which establishes that only the number expressions are values in $\simplelang$:
\begin{mathpar}
\inferrule{ }{
	\isvalU{\anumintro{n}}
}
\end{mathpar}
Although this inference rule is drawn using an operational form, we can nevertheless apply it to derive that $\isvalU{\numintro{2}}$, because $\numintro{2}$ and $\anumintro{2}$ identify the same AST. In other words, ``syntax doesn't matter'' from the perspective of the semantics of our language.


\begin{figure}
\hspace{-5px}$\begin{array}{lrlllll}
\textbf{Sort} & & & \textbf{Operational Form} & \textbf{Typeset Form} & \textbf{Textual Form} & \textbf{Description}\\
\mathsf{Exp} & e & ::= & \anumintro{n} & \numintro{n} & \numintro{n} & \text{numbers}\\
&&& \aplus{e}{e} & e + e & e\texttt{ + }e & \text{addition} \\
&&& \aminus{e}{e} & e - e & e\texttt{ - }e & \text{subtraction}\\
&&& \amult{e}{e} & e \times e & e\texttt{ * }e & \text{multiplication}\\
&&& \adiv{e}{e} & \frac{e}{e} & e\texttt{ / }e & \text{division}\\
&&& \apow{e}{e} & {e}^{e} & e\verb|^|e & \text{exponentiation}
\end{array}$
\caption[Syntax of $\simplelang$]{Syntax of $\simplelang$. Metavariable $n$ ranges over mathematical numbers (defined in some suitable manner) and $\numintro{n}$ abbreviates the numeral forms (one for each number $n$, drawn in our examples in \texttt{typewriter} font.) A complete definition of the stylized and textual syntax of $\simplelang$ would require 1) defining these numeral forms explicitly; 2) defining a form for parenthesized expressions; 3) defining the precedence and associativity of each infix operator; and 4) defining whitespace conventions. These details are not relevant to the present discussion, so we assume the usual conventions without stating them explicitly (the definition of ALGOL 60 introduced many of the conventions that persist today \cite{naur1963revised}.)}
\label{fig:simple-example}
\end{figure}

That being said, most humans would nevertheless distinguish the examples drawn in stylized and textual form above, perhaps by characterizing them as more ``'readable'' than the example drawn in operational form. Novice programmers in particular would likely be able to grasp the intended meaning more quickly when shown the first two drawings, because they closely follow the widely-known arithmetic conventions. Similarly, the first two drawings  require less effort to produce than the drawing in operational form, given typical tools. %Mistakes may also be less frequent when producing drawings in stylized or textual form (for $\simplelang$ expressions, perhaps only because operational forms use more parentheses). 
So it seems that different drawings of an AST, though semantically indistinguishable, must in fact be distinguished by the \emph{cognitive costs}  that human programmers incur when examining or producing them. We consider various operational definitions of this broad notion in Sec. \ref{sec:syntactic-properties}. %Regardless, it is  apparent that although syntax doesn't matter (semantically), different drawings of .  %that the popular refrain amongst language researchers that syntax ``doesn't matter'' is of rather limited.   %that intuitively,  drawings in stylized or textual form are of lower syntactic cost than those in operational form. This is important not only because aesthetics, but also because syntactic cost relates to measures of programmer productivity and other quantities and qualities of extrinsic interest (we discuss these and other measures in more detail in Sec. \ref{sec:syntactic-properties}). 

%The forms defined by the syntax chart in Figure \ref{fig:simple-example} suffice to allow programmers to draw any $\simplelang$ expression. However, 
In seeking to reduce the cognitive costs of common idioms, syntax designers often include additional \emph{derived forms} (colloquially, ``syntactic sugar'') in the syntax definition.  Derived forms are defined by context-independent \emph{desugaring} (a.k.a. \emph{rewriting} or \emph{redrawing}) \emph{rules}. For example, we can define a derived stylized form for calculating the square root of a $\simplelang$ expression by including the following rule in the definition of the stylized syntax of expressions:
\begin{subequations}
\begin{equation}\label{rule:simplelang-sqrt}
\sqrt{e} \rightarrowtriangle e^{\frac{\numintro{1}}{\numintro{2}}}
\end{equation}
Similarly, we can define a derived textual form for negating an expression by including the following rule in the definition of the textual syntax of expressions:\footnote{For $\footnotesimplelang$, it is not necessary to restrict, for example, textual and operational forms from being interspersed -- no ambiguities can arise. In richer languages, this may no longer be the case. The desugaring process must then be modified to first convert the pattern on the righthand side of a desugaring rule like Rule (\ref{rule:simplelang-negate}) to the desired form variant before it is applied.}
\begin{equation}\label{rule:simplelang-negate}
\texttt{-}e \rightarrowtriangle \amult{e}{\anumintro{-1}}
\end{equation}
\end{subequations}
\noindent 
When we encounter a drawing of a $\simplelang$ expression, we desugar  it by first recursively desugaring its subexpressions, then, if the drawing is in derived form, applying the corresponding rule above. If this process succeeds, the resulting drawing will identify an AST, i.e. it will use the forms in Figure \ref{fig:simple-example} (which we refer to collectively as \emph{primitive forms}, to distinguish them from derived forms.)
%Similarly, we might define a derived form for taking an arbitrary root of an expression as follows:
% \begin{align*}
% \sqrt[e']{e} & \rightarrowtriangle e^{\frac{\numintro{1}}{e'}}
% \end{align*}

$\simplelang$ expressions are rather limited -- they can express only simple calculations of a single type -- so we should not expect to need more than a few more derived forms like these to satisfyingly capture the  idioms that would arise in common human usage of $\simplelang$. %Consequently, there is little opportunity to go beyond simple derived forms like these. 
But the study of programming languages (and logics more generally) has produced many other sorts of trees, suitable for expressing substantially richer types of computations.  As humans have used these rich languages across a variety of problem domains, more idioms -- and with them, more derived forms -- have naturally emerged.  For example, Standard ML (a ``general-purpose''  language in the functional tradition \cite{mthm97-for-dart,harper1997programming}) defines derived textual forms for working with lists. In SML, \lstinline{[1, 2, 3, 4, 5]} is the same as writing: 
\begin{lstlisting}[numbers=none]
Cons(1, Cons(2, Cons(3, Cons(4, Cons(5, Nil)))))
\end{lstlisting}
if \li{Nil} and \li{Cons} are bound to the constructors of the parameterized recursive datatype \li{list} defined in the SML Basis library (i.e. SML's ``standard library''.)\footnote{The desugaring actually uses unforgeable identifiers bound permanently to these constructors.} Many other languages similarly define lists or list-like data structures in their standard libraries, and corresponding derived forms in their language definitions, anticipating that programmers are likely to use these data structures pervasively, across problem domains.

Other types of computations may be common only in more niche problem domains, which are served by third-party libraries. Language designers cannot reasonably anticipate and accomodate idioms that are specific to these libraries, so library providers (or other interested parties) sometimes define a \emph{syntax dialect} -- a new syntax definition constructed by extending an existing syntax definition with new derived forms. For example, Ur/Web extends Ur's textual syntax with derived forms for SQL queries, XHTML elements and other datatypes defined by a web programming library \cite{conf/popl/Chlipala15}. For example, the following program fragment shows how XHTML expressions that contain strings and other XHTML expressions can be drawn in Ur/Web: % Such dialects are sometimes qualitatively taxonomized as amongst the ``domain-specific language'' for this reason \cite{fowler2010domain}. %Syntactic cost is often assessed qualitatively \cite{green1996usability}, though quantitative metrics can be defined. 
\begin{lstlisting}[numbers=none]
val b = SURL<xml><p>Hello, {EURLcase name of
                          None => SURL<xml>World</xml>EURL
                        | Some s => SURL<xml>{[EURLsSURL]}</xml>}!</p></xml>EURL
\end{lstlisting}                              
The desugaring of this form is substantially more verbose.

We will consider other examples of data structures for which derived forms can substantially lower syntactic cost in Sec. \ref{sec:motivating-examples}. In Sec. \ref{sec:syntax-dialects}, we will describe systems like Camlp4 \cite{ocaml-manual}, Copper \cite{conf/gpce/WykS07}, SugarJ/Sugar* \cite{erdweg2011sugarj,erdweg2013framework} and Racket's preprocessor \cite{Flatt:2012:CLR:2063176.2063195} that have lowered the costs of defining and implementing syntax dialects, and thereby contributed to their ongoing proliferation.


%Full-scale languages are also interesting objects of mathematical study. Uniquely, however, they are also designed for use by humans. Consequently, their designers  typically define both an abstract syntax and a textual syntax. This textual syntax serves as the primary interface between human programmers and the language, so it is common to define various \emph{derived forms}, i.e. forms defined by a context-independent \emph{desugaring} to a set of \emph{base forms}. These serve to decrease the \emph{syntactic cost} or \emph{cognitive cost} of selected idioms. 
%In some cases, a derived form is designed to capture an idiom77Gu that involves only the primitive constructs of the language. 

%The hope amongst some language designers is that a limited number of derived forms like these will suffice to produce a ``general-purpose'' textual syntax, i.e. one that is accepted as suitable for use across a wide variety of application domains. Alas, a stable design that fully achieves this ideal has yet to emerge, as evidenced by the diverse array of \emph{syntax dialects} -- dialects that introduce only new derived forms -- that continue to proliferate around all major contemporary languages. 

%In fact, tools that aid in the construction of so-called  ``domain-specific'' language dialects (DSLs)\footnote{In some parts of the literature, such dialects are called ``external DSLs'', to distinguish them from  ``internal'' or ``embedded DSLs'', which are actually  library interfaces that only ``resemble'' distinct dialects \cite{fowler2010domain}.} seem only to be becoming more prominent over time. 

%\subsection{Why are there so many language dialects?}
%{This calls for an investigation}: why is it that programmers and researchers are still so often unable to satisfyingly express the constructs that they seek in libraries, as modes of use of the ``general-purpose'' primitives already available in major languages today, and instead see a need for new language dialects?

%Perhaps the most common sort of dialect is the \emph{syntax dialect} -- a dialect that introduces only new derived syntactic forms, motivated by a desire to decrease the {syntactic cost} of working with one or more library constructs of interest. 
%Put another way, syntax dialects can be specified by a context-independent expansion to the existing language that they are based on. 
%For example, Ur/Web is a syntax dialect of Ur (a language that itself descends from ML \cite{conf/pldi/Chlipala10}) that builds in derived forms for SQL queries, HTML elements and other datatypes used in the domain of web programming \cite{conf/popl/Chlipala15}. %Syntactic cost is often assessed qualitatively \cite{green1996usability}, though quantitative metrics can be defined. 
%This is not an isolated example -- we will consider a number of additional types of data that similarly stand to benefit from the availability of specialized derived forms in Sec. \ref{sec:motivating-examples}. 
%Tools like Camlp4 \cite{ocaml-manual}, Sugar* \cite{erdweg2011sugarj,erdweg2013framework} and Racket \cite{Flatt:2012:CLR:2063176.2063195}, which we will discuss in Sec. \ref{sec:existing-approaches}, have lowered the engineering costs of constructing syntax dialects in such situations, further contributing to their proliferation. 

%More advanced dialects introduce new type structure, going beyond what is possible with only new derived forms. As a simple example, the static and dynamic semantics of records cannot be expressed by context-independent expansion to a language with only nullary and binary products. Various languages have explored ``record-like'' primitives that go further, supporting functional update operators, width and depth coercions (sometimes implicit)%\cite{Cardelli:1984:SMI:1096.1098}
%, methods, prototypic dispatch and other such ``semantic embellishments'' that in turn cannot be expressed by context-independent expansion to a language with only standard record types (we will detail an  example in Sec. \ref{sec:metamodules-motivating-examples}). OCaml primitively builds in the type structure of polymorphic variants, open datatypes and  operations that use format strings like $\mathtt{sprintf}$ \cite{ocaml-manual}. ReactiveML builds in primitives for functional reactive programming \cite{mandel2005reactiveml}. ML5 builds in high-level primitives for distributed programming based on a modal lambda calculus \cite{Murphy:2007:TDP:1793574.1793585}. Manticore \cite{conf/popl/FluetRRSX07} and AliceML  \cite{AliceLookingGlass} build in parallel programming primitives with a more elaborate type structure than is found in simpler accounts of parallelism. 
%MLj builds in the type structure of the Java object system (motivated by a desire to interface safely and naturally with Java libraries) \cite{Benton:1999:IWW:317636.317791}. Other dialects do the same for other foreign languages, e.g. Furr and Foster describe a dialect of OCaml that builds in the type structure of C \cite{Furr:2005:CTS:1065010.1065019}. Tools like proof assistants and logical frameworks are used to specify and reason metatheoretically about dialects like these, and tools like compiler generators and language frameworks \cite{erdweg2013state} lower their implementation cost, again contributing to their proliferation. 

\vspace{-5px}
\section{Problems with the Dialect-Oriented Approach}\label{sec:problems-with-dialects}
Some  view the proliferation of syntax dialects as harmless or even as desirable, arguing that programmers can simply choose the right dialect for the job at hand \cite{journals/stp/Ward94}. However, this ``dialect-oriented'' approach is difficult to reconcile with the best practices of ``programming in the large'' \cite{DeRemer76}, i.e. developing large programs ``consisting of many small programs (modules), possibly written by different people'' whose interactions are mediated by an expressive type and binding discipline.


\subsection{Conservatively Combining Syntax Dialects}
% express record types as syntactic sugar over the simply-typed lambda calculus with  binary product types.\footnote{Pairs can of course be expressed as syntactic sugar atop records, though one could argue that using binary products as the more primitive concept is simpler.} The static semantics need to be extended with new type and term operators. However, the simplest way to express the dynamic semantics of the newly introduced term operators is by translation to nested binary products, so we can leave the operational semantics alone. \todo{fill this out} %For example, there are dozens of constructs that go by the name of ``records'' in various languages, each defined by a slightly different collection of primitive operations. \todo{examples} %, encouraged  historically  by the availability of tools like compiler generators and,  more recently, language workbenches \cite{workbenches} and DSL frameworks \cite{dsl}. Unfortunately, taking this approach makes it substantially more difficult for clients to import high-level abstractions orthogonally. 
% test 

The first problem is that programmers cannot always ``combine'' different dialects when they want to use the derived forms that they define together in a single program \cite{DeRemer76}.

For example, consider a syntax dialect, $\mathcal{H}$, defining derived forms for working with encodings of HTML elements, and another syntax dialect, $\mathcal{R}$,  defining derived forms for working with encodings of regular expressions. In domains like bioinformatics, both HTML elements and regular expressions are common, so it would be useful to construct a ``combined dialect'' where all of these derived forms are defined. 

For this notion of ``dialect combination'' to be well-defined at all, we must first have that $\mathcal{H}$ and $\mathcal{R}$ are defined using the same syntax definition system. In practice, there are many different syntax definition systems, each of which operate on subtly different classes of grammars, or that do not use grammars at all. If the dialect designers  have not  picked the same system, ``dialect combination'' remains only an informal notion.%$\mathcal{H} \cup \mathcal{R}$ is simply undefined.% (e.g. parser combinator libraries like Haskell's \li{parsec} \cite{parsec}.)

If $\mathcal{H}$ and $\mathcal{R}$ are coincidentally defined in the same system, we must also have that this system operationalizes the notion of ``dialect combination''. Not all syntax definition systems do so (e.g. Racket's preprocessor is \emph{monolithic} \cite{Flatt:2012:CLR:2063176.2063195}.) Clients can sometimes manually  ``copy-and-paste'' portions of the constituent dialect definitions to construct the ``combined'' dialect, but this is not systematic and, in practice, quite error-prone.%In both this and the previous case, ``dialect combination'' is a strictly informal notion, left to library clients to operationalize through manual labor (hence the quotes).

If we further restrict our interest  to dialects specified using a single system that does operationalize some notion of dialect combination (or equivalently one that allows programmers to systematically combine \emph{dialect fragments}), there is likely still a problem: there is no guarantee that the combined dialect will conserve important properties that can be established about the constituent dialects in isolation (i.e. \emph{modularly}.) In other words, establishing $P(\mathcal{H})$ and $P(\mathcal{R})$ is not sufficient to establish $P(\mathcal{H} \cup \mathcal{R})$. Clients must re-establish these properties for each combined dialect that they construct.%In other words, any putative ``combined language'' must formally be considered a  distinct system for which one must derive essentially all metatheorems of interest anew, guided only informally by those derived for the dialects individually. %There is no well-defined mechanism for constructing such a ``combined language'' in general. 

For example, consider two syntax dialects defined using the formalism of Camlp4: $\mathcal{D}_1$ defines derived forms for sets, and $\mathcal{D}_2$ defines derived forms for finite maps, both delimited by \verb~{|~ and \verb~|}~.\footnote{In OCaml, simple curly braces are already reserved by the language for record types and values.} Though each dialect defines a deterministic grammar, i.e. $\mathrm{det}(\mathcal{D}_1)$ and $\mathrm{det}(\mathcal{D}_2)$, when the grammars are na\"ively combined by Camlp4, it is not the case that $\mathrm{det}(\mathcal{D}_1 \cup \mathcal{D}_2)$ (i.e. syntactic ambiguities can arise in the combined dialect.) In particular, the empty set and the empty dictionary are both drawn \verb~{||}~. A third syntax dialect might use the same forms that $\mathcal{D}_2$ defines, but for ordered finite maps.

Schwerdfeger and Van Wyk describe a modular analysis, implemented in Copper \cite{conf/gpce/WykS07}, that ``nearly'' guarantees that determinism is conserved when syntax dialects defined by LALR(1) grammars are combined \cite{conf/pldi/SchwerdfegerW09}, the caveat being that newly introduced derived forms must be prefixed by distinct starting tokens (which, by nature, must be verbose if they are to be predictably unique.) We will return to this requirement (and some other subtle requirements of this approach) in Sec. \ref{sec:syntax-dialects}.


\subsection{Abstract Reasoning About Derived Forms}\label{sec:abs-reasoning-intro}
Even putting aside the difficulties of conservatively combining syntax dialects, there are questions about how \emph{reasonable} sprinkling library-specific derived forms throughout a large software system might be. 
For example, consider the perspective of a programmer attempting to comprehend (i.e. reason about) the following program fragment drawn in a syntax dialect constructed by combining SML's textual syntax with some number of other syntax dialects:
\begin{lstlisting}[numbers=none]
    val a = get_a()
    val w = get_w()
    val x = read_data(a)
    val y = {|(!R)@&{&/x!/:2_!x}'!R}|}
\end{lstlisting}

If the programmer happens to be familiar with the intentionally terse syntax of the stack-based database query processing language K, then this might pose no difficulties. If the programmer does not recognize this syntax, there is no simple, definitive protocol for answering questions like:

\begin{enumerate}
\item Which of the constituent dialects defined the derived form that appears on Line 4?
\item Is the character \li{x} inside this derived form parsed as a ``spliced'' expression, \li{x}, or parsed in some other way peculiar to this derived form?
\item If \li{x} is the spliced expression \li{x}, does it refer to the binding on the previous line? Or was that binding shadowed by an unseen binding in the desugaring?
\item If I rename \li{w} or move its binding down past the binding of \li{y}, could that possibly break the program, or change its meaning? In other words, does the desugaring rule assume that some variable identified as \li{w} is in scope?
\item What type does \li{y} have?
\end{enumerate}

%In other words, encountering an unfamiliar derived form has made it difficult for the programmer to maintain the usual \emph{type discipline} and \emph{binding discipline}. %Compelling the programmer to examine the desugaring directly defeat the purpose of defining the derived form -- decreasing cognitive cost. Indeed, it substantially increases cognitive cost.

In contrast, when a programmer encounters, for example, a function call (e.g. the call to \li{read_data} above), in a setting where the syntax has not been extended, the analagous questions can be answered by following clear protocols that become ``cognitive reflexes'' after sufficient experience with the language:
\begin{enumerate}
\item The language's syntax definition determines how \li{read_data(a)} should be parsed.
\item According to the syntax definition, \li{read_data} and \li{a} are parsed as variables.
\item The variable \li{a} on Line 3 can only refer to the binding of \li{a} on Line 1.
\item \li{read_data(a)} does not mention \li{w}, so the binding of \li{w} can be renamed, moved or removed without changing the meaning of Line 3.
\item The type of \li{x} can be determined to be \li{B} by first determining that the type of \li{read_data} is \li{A -> B} for some \li{A}, and then checking that \li{a} has type \li{A}. Nothing else needs to be known about the values of \li{read_data} or \li{a}. 
\end{enumerate}

In other words, languages like ML are structured so as to provide programmers with powerful abstract reasoning principles. In the words of Reynolds \cite{B304}:
\begin{quote}
\emph{Type structure is a syntactic discipline for enforcing levels of abstraction.}
\end{quote}

Syntax dialects, on the other hand, do not maintain \emph{syntactic abstraction} -- if the desugaring of a program is held abstract, programmers can no longer reason about types and binding in the usual disciplined manner.

%A related issue arises when one works within a language with a module system, i.e. a system that supports interacting through a defined interface with various implementations of that interface. For example, consider different regular expression engines that differ only with regard to their performance in various circumstances, or different parser generators that accept the same class of grammar. Ideally, one would like to be able to define derived forms once such that they operate only through the common interface. To do so today requires both an awkward syntactic trick and coordination between library providers, as we will discuss in Sec. \ref{sec:syntax-examples-regexps}. Ideally, this would not be necessary.

%It is thus infeasible to simply allow different contributors to a software system to choose their own favorite dialect for each component they are responsible for. 
%It it clear that dialects are better rhetorical devices than practical engineering artifacts. 

%Due to this paucity of modular reasoning principles, the ``dialect-oriented'' approach is problematic for software development ``in the large''. %Large software projects and software ecosystems must pick a single language that does provide powerful modular reasoning principles and, to benefit from them, stay inside it.

% \subsection{Central Planning Considered Harmful}
% Dialects do sometimes have a less direct influence on large-scale software development: they can help convince the designers in control of comparatively popular languages, like OCaml and Scala, to include some variant of the primitives that they feature into backwards-compatible language revisions. %These decisions are increasingly influenced by community processes, e.g. the Scala Improvement Process.  %This approach concentrates power as well as responsibility over maintaining metatheoretic guarantees in the hands of a small group of language designers, though increasingly influenced by various community processes (e.g. the Scala Improvement Process). 
% %Dialects thus serve the role of rhetorical vehicles for new ideas, rather than direct artifacts. 
% %Over time, accepting such extensions has caused these languages to balloon in size. 
% This \emph{ad hoc} approach is unsustainable, for three main reasons. First, as we will demonstrate in Sec. \ref{sec:motivating-examples}, there are simply too  many potentially useful such primitives, and many of these capture idioms common only in relatively narrow application domains. It is unreasonable to expect language designers to be able to evaluate all of these use cases in a timely and informed manner. Second, primitives introduced earlier in a language's lifespan can end up monopolizing finite ``syntactic resources'', forcing subsequent primitives to use ever more esoteric forms. And third, primitives that prove after some time to be flawed in some way cannot be removed or modified without breaking backwards compatibility. For these reasons, language designers are justifiably reticent to add new primitives to major languages.%Because there is often no empirical data about how useful a construct is in practice until it is available in a major language, decisions about which constructs to include are often informed only by intuition (and are thus)
% %Recalling the words of  Reynolds, which are clearly as relevant today as they were almost half a century ago \cite{Reynolds70}: %This approach is antithetical to the ideal of a truly \emph{general-purpose language} described at the beginning of this section.
% %\newpage

%\subsection{Toward More Reasonable Primitives}
%These 
%This leaves two possible paths forward. One is to simply eschew ``niche'' derived forms and settle on the existing designs, which might be considered to sit at a ``sweet spot'' in the overall language design space (accepting that in some circumstances, this leads to  high cognitive cost). 


%Similarly, it recently introduced ``open datatypes'', which subsume its previous more specialized exception type, and captures many use cases for .

%Viewed ``dually'', one might equivalently ask for a language that builds in a core that is as small as possible, but provides expressive power comparable to languages with much larger cores. This is our goal in the work being proposed

%Similarly, it recently introduced ``open datatypes'', which subsume its previous more specialized exception type, and captures many use cases for .

%Viewed ``dually'', one might equivalently ask for a language that builds in a core that is as small as possible, but provides expressive power comparable to languages with much larger cores. This is our goal in the work being proposed. 

%\vspace{-10px}
\section{Contributions}\label{sec:contributions}
%%Our broad aim in the work being proposed is to introduce primitive language mechanisms that give library providers the ability to  express new syntactic expansions as well as new types and operators in a safe and modularly composable manner. 
Our aim in this work is to introduce primitive language constructs that reduce the need for syntax dialects. In particular, we 
introduce \textbf{typed syntax macros}, or \textbf{TSMs}. TSMs are applied to trees of {generalized literal form}, which are definitively delimited but whose bodies are initially left unparsed, to programmatically control their expansion to trees that do not contain such forms (i.e. trees in \emph{expanded form}.) 

Syntactic conflicts between TSMs are impossible by construction, because the syntax of the language is never directly modified -- only contextually repurposed. Expansion occurs simultaneously with typing, in a phase that we call \emph{typed expansion}. As such, the semantics can take the type and binding structure of the surrounding program into account to impose constraints that serve to ensure that clients can maintain a reasonable type discipline and binding discipline, i.e. that questions like those above can be answered without examining the expansion itself. More specifically, TSMs maintain a \emph{hygienic binding discipline}, meaning that the class of perverse expansions that Questions 4 and 5 above were concerned with are disallowed entirely. We will, of course, make these notions more technically precise as we continue.

We  introduce TSMs first in a simple language of expressions and types in Chapter \ref{chap:uetsms}, then add support for pattern matching  in Chapter \ref{chap:uptsms}. In Chapter \ref{chap:ptsms}, we add an ML-style module system, and define \emph{parametric TSMs}, i.e. TSMs that take type and module parameters. This allows the expansions that parametric TSMs generate to interact with the surrounding context in a controlled manner, and also allows library providers to define TSMs that operate not just at a single type, but uniformly over a type- and module-parameterized family of types.

In the chapters just mentioned, we make a simplifying assumption: that each TSM definition is fully self-contained, i.e. that TSM definitions do not need access to libraries. This allows us to focus on the fundamental contributions of this work, but it is, of course, an unrealistic assumption in practice. We relax this assumption in Chapter \ref{chap:static-eval}.

%\item \textbf{Type-specific languages}, or \textbf{TSLs}. TSLs, described 
In Chapter \ref{chap:tsls}, we show how library clients can contextually designate, for any type, a privileged TSM at that type, and then rely on a bidirectional type system to invoke that TSM implicitly. This method of \emph{TSM implicits} can reduce the cognitive cost of an idiom to very nearly the same extent that a special-purpose dialect can, while still maintaining the reasoning principles just summarized.
%\item \textbf{Metamodules}, introduced in Sec. \ref{sec:metamodules}, reduce the need to primitively build in the type structure of constructs like records (and variants thereof),  labeled sums and other interesting constructs that we will introduce later by giving library providers programmatic ``hooks'' directly into the semantics, which are specified as a \emph{type-directed translation semantics} targeting a small \emph{typed internal language} (introduced in Sec. \ref{sec:VerseML}). %For example, a library provider can implement the type structure of records with a metamodule that:
%\begin{enumerate}
%\item introduces a type constructor, \lstinline{record}, parameterized by finite mappings from labels to types, and defines, programmatically, a translation to unary and binary product types (which are built in to the internal language); and 
%\item introduces operators used to work with records, minimally record introduction and elimination (but perhaps also various functional update operators), and directly implements the logic governing their typechecking and translation to the IL (which builds in only nullary and binary products). 
%\end{enumerate}
%We will see direct analogies between ML-style modules (which our mechanisms also support) and metamodules later.
%\end{enumerate} 


As vehicles for this work, we define a small programming language in each of the chapters just mentioned, each building upon the previous one. All formal claims made in this work involve these small languages.

In examples, we assume a full-scale functional language called VerseML.\footnote{We distinguish VerseML from Wyvern, which is the language described in our prior publications about some of the work that we will describe, because Wyvern is a group effort evolving independently.} VerseML is the language of Chapter \ref{chap:tsls}  extended with a few additional conveniences that are commonly found in other functional languages and semantically orthogonal to TSMs (e.g. higher-rank polymorphism \cite{conf/icfp/DunfieldK13}, signature abbreviations, and a number of standard derived forms that are not library-specific, e.g. for curried functions.) %VerseML is, as its name suggests, a conceptual descendent of ML. It diverges from other dialects of ML that have a similar type structure in that it has a bidirectional type system \cite{Pierce:2000:LTI:345099.345100} (like, for example, Scala \cite{OdeZenZen01}) for reasons that have to do with the mechanism of TSM implicits described in Chapters \ref{chap:tsls} and \ref{chap:ptsms}. 
%The reason we will not follow Standard ML \cite{mthm97-for-dart} in giving a complete formal definition of VerseML in this work is both to emphasize that the primitives we introduce are ``insensitive'' to the details of the underlying type structure of the language (so TSMs can be considered for inclusion in a variety of languages, not only dialects of ML), and to avoid distracting the reader (and the author) with definitions that are already well-understood in the literature and that are orthogonal to those that are the focus of this work. 
We do not formally define these features to avoid unnecessarily complicating our presentation with details that are not essential to the ideas presented herein -- our purpose is only to introduce new language constructs, not to define a full-scale programming language. As such, all examples involving VerseML should be understood to be informal motivating material for the subsequent formal material. %We anticipate that future full-scale language specifications will be able to combine the ideas  in the proposed work without trouble. %The purpose of the work being proposed is to serve as a reference for those interested in the new constructs we introduce, not to serve as a language specification. 
%We will give a brief overview of these languages are organized in Sec. \ref{sec:VerseML}.

%TSMs, like other macro systems, perform \emph{static code generation} (also sometimes called \emph{static} or \emph{compile-time metaprogramming}), meaning that the relevant rules in the static semantics of the language call for the evaluation of \emph{static functions} that generate term encodings. Static functions are functions that are evaluated statically, i.e. during typing. %Library providers write these static functions using the VerseML \emph{static language} (SL).  
%Maintaining a separation between the static (or ``compile-time'') phase and the dynamic (or ``run-time'') phase is an important facet of VerseML's design. % static code generation. %We will  also introduce a simple variant of each of these primitives that leverages VerseML's support for local type inference to further reduce syntactic cost in certain common situations. 


\subsection*{Thesis Statement}
In summary, this work defends the following statement:

\begin{quote}
A functional programming language can give library providers the ability to %meta\-pro\-gram\-matic\-ally 
programmatically control the parsing and expansion of expressions and patterns of generalized literal form while maintaining a type discipline, a hygienic binding discipline and modular reasoning principles. %These  primitives are  expressive enough to subsume the need for a variety of primitives that are, or would need to be, built in to comparable contemporary languages.
\end{quote}
\section{Disclaimers}
Before we continue, it may be prudent to explicitly acknowledge that eliminating the need for dialects would indeed be asking for too much: certain syntax design decisions are fundamentally incompatible with others or require coordination across a language design. We aim only to decrease the need for syntax dialects in this work. We summarize some of the situations that we explicitly do not consider when we discuss future work in Sec. \ref{sec:future-work}. % out a larger design space within a single language, VerseML.%a subset of constructs that can be specified by a semantics of a certain ``shape'' specified by VerseML (we will make this more specific later). %There is nothing ``universal'' about VerseML.

It may also be useful to explicitly acknowledge that library providers could use TSMs  to define constructs that are in ``poor taste''. We  expect that in practice, VerseML will come with a standard library defining an expertly curated collection of TSMs, as well as guidelines for advanced users regarding when it would be sensible to define their own TSMs (following the example of languages that support operator overloading or \emph{ad hoc} polymorphism using type classes \cite{Hall:1996:TCH:227699.227700,conf/popl/DreyerHCK07}, which also have some potential for ``abuse'' or ``overuse''.) %For most programmers, using VerseML will not require explicitly defining a TSM on their own.%be substantially different from using a language like ML or one of its dialects. 
The majority of programmers would not be expected to define TSMs themselves.

%Finally, VerseML is not designed as a dependently-typed language like Coq, Agda or Idris. %because these languages do not maintain a phase separation between ``compile-time'' and ``run-time.'' This phase separation is useful for programming tasks (where one would like to be able to discover errors before running a program, particularly programs that may have an effect) but less so for theorem proving tasks (where it is mainly the fact that a pure expression is well-typed that is of interest, by the propositions-as-types principle). 

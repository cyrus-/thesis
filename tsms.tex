% !TEX root = omar-thesis.tex
\chapter{Unparameterized Expression TSMs}\label{chap:tsms}
We now introduce a new primitive -- the \textbf{typed syntax macro} (TSM). TSMs, like term-rewriting macros (Sec. \ref{sec:term-rewriting}), generate expansions. Unlike term-rewriting macros, TSMs are applied to unparsed \emph{generalized literal forms}, which gives them substantially more syntactic flexibility. This chapter considers perhaps the simplest manifestation of TSMs: \textbf{unparameterized expression TSMs} (ueTSMs), which generate expressions of a single specified type. We will consider unparameterized pattern TSMs (upTSMs) in Chapter \ref{sec:pattern-tsms} and parameterized TSMs (pTSMs) in Chapter \ref{sec:tsms-parameterized}.

%Like the term-rewriting macros just described, TSMs can be parameterized by modules, so they can be used to define syntax valid at any abstract type defined by a module satisfying a specified signature. As we will discuss in the remainder of this section, this addresses all of the problems brought up above, at moderate syntactic cost.

\section{Expression TSMs By Example}\label{sec:tsms-by-example}
%A typed syntax macro is invoked by applying it to a \emph{delimited form}, which can contain  arbitrary syntax in its \emph{body}.  
We begin in this section with a ``tutorial-style'' introduction to ueTSMs in VerseML. In particular, we discuss a ueTSM for constructing values of the recursive labeled sum type \li{Rx} that was defined in Figure \ref{fig:datatype-rx}. We then formally specify ueTSMs with a reduced calculus, $\miniVerseUE$, in Sec. \ref{sec:tsms-minimal-formalism}. %We conclude in Sec. \ref{sec:uetsms-discussion} 

\subsection{Usage}\label{sec:uetsms-usage}
In the following concrete VerseML expression, we apply a TSM named \li{#\dolla#rx} to the \emph{generalized literal form} \li{/SURLA|T|G|CEURL/}:
\begin{lstlisting}[numbers=none,mathescape=|]
$rx /SURLA|T|G|CEURL/
\end{lstlisting}
Generalized literal forms are left unparsed when concrete expressions are first parsed. It is only during the subsequent \emph{typed expansion} process that the TSM parses the \emph{body} of the provided literal form, i.e. the characters between forward slashes in blue here, to generate a \emph{candidate expansion}. The language then \emph{validates} the candidate expansion according to criteria that we will establish in Sec. \ref{sec:uetsms-validation}. If candidate expansion validation succeeds, the language generates the \emph{final expansion} (or more concisely, simply the \emph{expansion}) of the expression. The program will behave as if the expression above has been replaced by its expansion. The expansion of the expression above, written concretely, is:
\begin{lstlisting}[numbers=none]
Or(Str "SSTRAESTR", Or(Str "SSTRTESTR", Or(Str "SSTRGESTR", Str "SSTRCESTR")))
\end{lstlisting}
%The constructors above are those of the type \li{Rx} that was defined in Figure \ref{fig:datatype-rx}.

A number of literal forms, shown in Figure \ref{fig:literal-forms},  are available in VerseML's concrete syntax. Any literal form can be used with any TSM, e.g. we could have equivalently written the example above as \li{#\dolla#rx `SURLA|T|G|CEURL`} (in fact, this would be convenient if we had wanted to express a regex containing forward slashes but not backticks). TSMs have access only to the literal bodies. Because TSMs do not extend the concrete syntax of the language directly, there cannot be syntactic conflicts between TSMs.

 %The form does not directly determine the expansion. 

\begin{figure}
\begin{lstlisting}
'SURLbody cannot contain an apostropheEURL'
`SURLbody cannot contain a backtickEURL`
[SURLbody cannot contain unmatched square bracketsEURL]
{SURLbody cannot contain an unmatched curly braceEURL}
/SURLbody cannot contain a forward slashEURL/
\SURLbody cannot contain a backslashEURL\
\end{lstlisting}
%SURL<tag>body includes enclosing tags</tag>EURL
\caption[Available Generalized Literal Forms]{Generalized literal forms available for use in VerseML's concrete syntax. The characters in blue indicate where the literal bodies are located within each form. In this figure, each line describes how the literal body is constrained by the form shown on that line. The Wyvern language specifies additional forms, including whitespace-delimited forms \cite{TSLs} and multipart forms \cite{sac15}, but for simplicity we leave these out of VerseML.}
\label{fig:literal-forms}
\end{figure}
\subsection{Definition}\label{sec:uetsms-definition}
%The original expression, above, is statically rewritten to this expression.
Let us now take the perspective of the library provider. The definition of the TSM \lstinline{#\dolla#rx} shown being applied above has the following form:
\begin{lstlisting}[numbers=none,mathescape=|]
syntax $rx at Rx {
  static fn(body : Body) : CEExp ParseResult => 
    (* regex literal parser here *)
}
\end{lstlisting}
This {TSM definition} first names the TSM. 
 TSM names must begin with the dollar symbol (\li{#\dolla#}) to clearly distinguish them from variables (and thereby clearly distinguish TSM application from function application). This is inspired by a similar convention enforced by the Rust macro system \cite{Rust/Macros}.

The TSM definition then specifies a \emph{type annotation}, \lstinline{at Rx}, and a \emph{parse function} within curly braces. 
The {parse function} is a \emph{static function} responsible for parsing the literal body when the TSM is applied to generate an encoding of the candidate expansion, or an indication of an error if one cannot be generated (e.g. when the body is ill-formed according to the syntactic specification that the TSM implements). Static functions are functions that are applied during the typed expansion process. For this reason, they do not have access to surrounding variable bindings (because those variables stand in for dynamic values). For now, let us simply assume that static functions are closed (we discuss introducing a distinct class of static bindings so that static values can be shared between TSM definitions in Sec. \ref{sec:uetsms-static-language}).

The parse function must have type \li{Body -> CEExp ParseResult}. These types are defined in the VerseML \emph{prelude}, which is a collection of definitions available ambiently. The input type, \lstinline{Body}, gives the parse function access to the {body} of the provided literal form. For our purposes, it suffices to define \li{Body} as an abbreviation for the \li{string} type:
\begin{lstlisting}[numbers=none]
type Body = string
\end{lstlisting} 

The output type, \li{CEExp ParseResult}, is a labeled sum type that distinguishes between successful parses and parse errors. The parameterized type \li{'a ParseResult} is defined in Figure \ref{fig:indexrange-and-parseresult}.

If parsing succeeds, the parse function returns a value of the form \li{Success(#$\ecand$#)}, where $\ecand$ is the \emph{encoding of the candidate expansion}. Encodings of candidate expansions are, for expression TSMs, values of the type \lstinline{CEExp} defined in Figure \ref{fig:candidate-exp-verseml} (in Chapter \ref{sec:pattern-tsms}, we will introduce pattern TSMs, which generate patterns rather than expressions; this is why \li{ParseResult} is defined as a parameterized type). Expressions can mention types, so we also need to define a type \li{CETyp} in Figure \ref{fig:candidate-exp-verseml}. 
\begin{figure}
\begin{lstlisting}[numbers=none]
type IndexRange = {startIndex : nat, endIndex : nat}

type 'a ParseResult = Success of 'a 
                    | ParseError of {
                        msg : string, loc : IndexRange
                      }
\end{lstlisting}
\caption[Definitions of \li{IndexRange} and \li{ParseResult} in the VerseML prelude.]{Definitions of \li{IndexRange} and \li{ParseResult} in the VerseML prelude.}
\label{fig:indexrange-and-parseresult}
\end{figure}
\begin{figure}
\begin{lstlisting}[numbers=none]
type CETyp = TyVar of var_t 
           | Arrow of CETyp * CETyp 
           | (* ... *) 
           | Spliced of IndexRange

type CEExp = Var of var_t 
           | Fn of var_t * CETyp * CEExp
           | App of CEExp * CEExp
           | (* ... *) 
           | Spliced of IndexRange
\end{lstlisting}
\caption[Abbreviated definitions of \li{CETyp} and \li{CEExp} in the VerseML prelude]{Abbreviated definitions \li{CETyp} and \li{CEExp} in the VerseML prelude. We assume some suitable type \li{var_t} exists, not shown.}
\label{fig:candidate-exp-verseml}
\end{figure}
% We will show a complete encoding when we describe our reduced formal system $\miniVerseUE$ in Sec. \ref{sec:tsms-minimal-formalism}. 
We discuss the constructors labeled \li{Spliced} in Sec. \ref{sec:splicing-and-hygiene}; the remaining constructors (some of which are elided for concision) encode the abstract syntax of VerseML expressions and types. To decrease the syntactic cost of working with the types defined in Figure \ref{fig:candidate-exp-verseml}, the prelude provides \emph{quasiquotation syntax} at these types, which is itself implemented using TSMs. We will discuss these TSMs in more detail in Sec. \ref{sec:tsms-for-tsms}. The definitions in Figure \ref{fig:candidate-exp-verseml} are recursive labeled sum types to simplify our exposition, but we could have chosen alternative encodings of terms, e.g. based on abstract binding trees \cite{pfpl}, with only minor modification to the semantics. % It is extended with one additional form used to handled spliced subexpressions, 

If the parse function determines that a candidate expansion cannot be generated, i.e. there is a parse error in the literal body, it returns a value labeled by \li{ParseError}. It must provide an error message and indicate the location of the error within the body of the literal form as a value of type \li{IndexRange}, also defined in Figure \ref{fig:indexrange-and-parseresult}. This information can be used by VerseML compilers when reporting errors to the programmer.

%Notice that the types just described are those that one would expect to find in a typical parser.

%One would find types analagous to those just described in any parser, so for concision, we elide the details of \li{#\dolla#rx}'s parse function.
%The parse function must treat the TSM parameters parametrically, i.e. it does not have access to any values in the supplied module parameter. Only the expansion the parse function generates can refer to module parameters. 
%For example, the following definition is ill-sorted:
%\begin{lstlisting}[numbers=none]
%syntax pattern_bad[Q : PATTERN] at Q.t {
%  static fn (body : Body) : Exp => 
%    if Q.flag then (* ... *) else (* ... *)
%}
%\end{lstlisting}%So the parse function parses the body of the delimited form to generate an encoding of the elaboration.

\subsection{Splicing}\label{sec:splicing-and-hygiene}
To support splicing syntax, like that described in Sec. \ref{sec:syntax-examples-regexps}, the parse function must be able to parse subexpressions out of the supplied literal body. For example, consider the code snippet in Figure \ref{fig:derived-spliced-subexpressions}, expressed instead using the \li{#\dolla#rx} TSM:
\begin{lstlisting}[numbers=none]
val ssn = $rx /SURL\d\d\d-\d\d-\d\d\d\dEURL/
fun example_rx_tsm(name: string) => $rx /SURL@EURLnameSURL: %EURLssn/
\end{lstlisting}
The subexpressions \lstinline{name} and \lstinline{ssn} on the second line appear directly in the body of the literal form, so we call them \emph{spliced subexpressions} (and color them black when typesetting them in this document). When the parse function determines that a subsequence of the literal body should be treated as a spliced subexpression (here, by recognizing the characters \lstinline{@} or \lstinline{%} followed by a variable or parenthesized expression), 
it can refer to it within the candidate expansion it generates using the \li{Spliced} constructor of the \li{CEExp} type shown in Figure \ref{fig:candidate-exp-verseml}. The \li{Spliced} constructor requires a value of type \li{IndexRange} because spliced subexpressions are referred to indirectly by their position within the literal body. This prevents TSMs from ``forging'' a spliced subexpression (i.e. claiming that an expression is a spliced subexpression, even though it does not appear in the body of the literal form). Expressions can also contain types, so one can also mark spliced types in an analagous manner using the \li{Spliced} constructor of the \li{CETyp} type. %In particular, the parse function must provide the index range of spliced subexpressions to the \li{Spliced} constructor of the type \li{MarkedExp}. %Only subexpressions that actually appear in the body of the literal form can be marked as spliced subexpressions.

The candidate expansion generated by \li{#\dolla#rx} for the body of \lstinline{example_rx_tsm}, if written in a hypothetical concrete syntax for candidate expansions where references to spliced subexpressions are written \li{spliced<startIdx, endIndex>}, is:
\begin{lstlisting}[numbers=none]
Seq(Str(spliced<1, 4>), Seq(Str "SSTR: ESTR", spliced<8, 10>))
\end{lstlisting}
Here, \li{spliced<1, 4>} refers to the subexpression \li{name} by position and \li{spliced<8, 10>} refers to the subexpression \li{ssn} by position. 

%For example, had the  would not be a valid expansion, because the  that are not inside spliced subexpressions:
%\begin{lstlisting}[numbers=none]
%Q.Seq(Q.Str(name), Q.Seq(Q.Str ": ", ssn))
%\end{lstlisting}

\subsection{Typing}\label{sec:uetsms-validation}
The language \emph{validates} candidate expansions before a final expansion is generated. One aspect of candidate expansion validation is checking  the candidate expansion against the type annotation specified by the TSM, e.g. the type \li{Rx} in the example above. This maintains a \emph{type discipline}: if a programmer sees a TSM being applied when examining a well-typed program, they need only look up the TSM's type annotation to determine the type of the generated expansion. Determining the type does not require examine the expansion directly.


\subsection{Hygiene}
The spliced subexpressions that the candidate expansion refers to (by their position within the literal body, cf. above) must be parsed, typed and expanded during the candidate expansion validation process (otherwise, the language would not be able to check the type of the candidate expansion). To maintain a useful \emph{binding discipline}, i.e. to allow programmers to reason also about variable binding without examining expansions directly, the validation process maintains two additional properties related to spliced subexpressions: \textbf{context independent expansion} and \textbf{expansion independent splicing}. These are collectively referred to as the \emph{hygiene properties} (because they are conceptually related to the concept of hygiene in term rewriting macro systems, cf. Sec. \ref{sec:term-rewriting}.) 

\paragraph{Context Independent Expansion} Programmers expect to be able to choose variable and symbol names freely, i.e. without needing to satisfy ``hidden assumptions'' made by the TSMs that are applied in scope of a binding. For this reason, context-dependent candidate expansions, i.e. those with free variables or symbols, are deemed invalid (even at application sites where those variables happen to be bound). An example of a TSM that generates context-dependent candidate expansions is shown below:
\begin{lstlisting}[numbers=none]
syntax $bad1 at Rx {
	static fn(body : Body) : ParseResultExp => Success (Var 'SSTRxESTR')
}
\end{lstlisting}
The candidate expansion this TSM generates would be well-typed only when there is an assumption \li{x : Rx} in the application site typing context. This ``hidden assumption'' makes reasoning about binding and renaming especially difficult, so this candidate expansion is deemed invalid (even when \li{#\dolla#bad1} is applied in a context where \li{x} happens to be bound).

Of course, this prohibition does not extend into the spliced subexpressions referred to in a candidate expansion because spliced subexpressions are authored by the TSM client and appear at the application site, and so can justifiably refer to application site bindings. We saw examples of spliced subexpressions that referred to variables bound at the application site in Sec. \ref{sec:splicing-and-hygiene}. Because candidate expansions refer to spliced subexpressions indirectly, checking this property is straightforward -- we only allow access to the application site typing context when typing spliced subexpressions. In the next section, we will formalize this intuition. % The TSM provider can only refer to them opaquely.

In the examples in Sec. \ref{sec:uetsms-usage} and Sec. \ref{sec:splicing-and-hygiene}, the expansion used constructors associated with the \li{Rx} type, e.g. \li{Seq} and \li{Str}. This might appear to violate our prohibition on context-dependent expansions. This is not the case only because in VerseML, constructor labels are not variables or scoped symbols. Syntactically, they must begin with a capital letter (like Haskell's datatype constructors). Different labeled sum types can use common constructor labels without conflict because the type the term is being checked against -- e.g. \li{Rx}, due to the type ascription on \li{#\dolla#rx} -- determines which type of value will be constructed. For dialects of ML where datatype definitions do introduce new variables or scoped symbols, we need parameterized TSMs. We will return to this topic in Chapter \ref{sec:tsms-parameterized}. % Indeed, we used the label \li{Spliced} for two different recursive labeled sum types in Figure \ref{fig:candidate-exp-verseml}.

\paragraph{Expansion Independent Splicing} Spliced subexpressions, as just described, must be given access to application site bindings. The \emph{expansion independent splicing} property ensures that spliced subexpressions have access to \emph{only} those bindings, i.e. a TSM cannot introduce new bindings into spliced subexpressions. For example, consider the following hypothetical candidate expansion (written concretely as above):
\begin{lstlisting}[numbers=none]
fn(x : Rx) => spliced<0, 4>
\end{lstlisting}
The variable \li{x} is not available when typing the indicated spliced subexpression, nor can it shadow any bindings of \li{x} that might appear at the application site.

For TSM providers, the benefit of this property is that they can choose the names of variables used internally within expansions freely, without worrying about whether they might shadow those that a client might have defined at the application site.

TSM clients can, in turn, determine exactly which bindings are available in a spliced subexpression without examining the expansion it appears within. In other words, there can be no ``hidden variables''. 

The trade-off is that this prevents library providers from defining  alternative binding forms. For example, Haskell's derived form for monadic commands (i.e. \li{do}-notation) supports binding the result of executing a command to a variable that is then available in the subsequent commands in a command sequence. In VerseML, this cannot be expressed in the same way. We will show an alternative formulation of Haskell's syntax for monadic commands that uses VerseML's anonymous function syntax to bind variables in Sec. \ref{sec:application-monadic-commands}. We will discuss mechanisms that would allow us to relax this restriction while retaining client control over variable names as future work in Sec. \ref{sec:controlled-binding}.

%These properties suffice to ensure that programmers and tools can freely rename a variable without changing the meaning of the program. The only information that is necessary to perform such a \emph{rename refactoring} is the locations of spliced subexpressions within all the literal forms for which the variable being renamed is in scope; the expansions need not otherwise be examined. It would be straightforward to develop a tool and/or editor plugin to indicate the locations of spliced subexpressions to the user, like we do in this document (by coloring spliced subexpressions black). We discuss tool support as future work in Sec. \ref{sec:interaction-with-tools}.

\subsection{Final Expansion}
If validation succeeds, the language generates the \emph{final expansion} from the candidate expansion by replacing references to spliced subexpressions with their final expansions. The final expansion of the body of \li{example_rx_tsm} is:
\begin{lstlisting}[numbers=none]
Seq(Str(name), Seq(Str "SSTR: ESTR", ssn))
\end{lstlisting}

\subsection{Scoping}
A benefit of specifying TSMs as a language primitive, rather than relying on extralinguistic mechanisms to manipulate the concrete syntax of our language directly, is that TSMs follow standard scoping rules. 

For example, we can define a TSM that is visible only to a single expression like this:
\begin{lstlisting}[numbers=none]
let x = let 
    syntax $rx at Rx { (* ... *) }
  in (* $rx is in scope here *) end 
in (* $rx is no longer in scope *) end
\end{lstlisting}

We will consider the question of how TSM definitions can be exported from within modules in Sec. \ref{sec:tsm-packaging}.

\subsection{Comparison to ML+Rx}
Let us compare the VerseML TSM \li{#\dolla#rx} to ML+Rx, the hypothetical syntactic dialect of ML with support for derived forms for regular expressions described in Sec. \ref{sec:syntax-examples-regexps}.

Both ML+Rx and \li{#\dolla#rx} give programmers the ability to use the same standard syntax for constructing regexes, including syntax for splicing in other strings and regexes. In VerseML, however, we incur the additional syntactic cost of explicitly applying the \li{#\dolla#rx} TSM each time we wish to use regex syntax. This cost does not grow with the size of the regex, so it would only be significant in programs that involve a large number of small regexes (which do, of course, exist). In Chapter \ref{chap:tsls} we will consider a design where even this syntactic cost can be eliminated in certain situations.

The benefit of this approach is that we can easily define other TSMs to use alongside the \li{#\dolla#rx} TSM without needing to consider the possibility of syntactic conflict. Furthermore, programmers can rely on the typing discipline and the hygienic binding discipline described above to reason about programs, including those that contain unfamiliar forms. Put pithily, VerseML helps programmers avoid ``conflict and confusion''. 


\section{\texorpdfstring{$\miniVerseUE$}{miniVerseUE}}\label{sec:tsms-minimal-formalism}\label{sec:miniVerseU}

% \begin{figure}[p!]
% $\begin{array}{lllllll}
% \textbf{variables} & \textbf{type variables} & \textbf{labels} & \textbf{label sets} & \textbf{TSM variables} & \textbf{literal bodies} & \textbf{nats}\\
% x & t & \ell & \labelset & \tsmv & b & n\\~\end{array}$\\
% $\begin{array}{ll}
% \textbf{type formation contexts} & \textbf{typing contexts}\\
% \Delta ::= \emptyset ~\vert~ \Delta, t & \Gamma ::= \emptyset ~\vert~ \Gamma, x : \tau\\
% ~
% \end{array}$\\
% ~\\
% $\begin{array}{lcl}
% \gheading{types}\\
% \tau & ::= & t ~\vert~ \parr{\tau}{\tau} ~\vert~ \forallt{t}{\tau} ~\vert~ \rect{t}{\tau} ~\vert~  \prodt{\mapschema{\tau}{i}{\labelset}} ~\vert~ \sumt{\mapschema{\tau}{i}{\labelset}}\\
% ~\\
% \gheading{expanded expressions}\\
% e & ::= & x ~\vert~ \lam{x}{\tau}{e} ~\vert~ \app{e}{e} ~\vert~ \Lam{t}{e} ~\vert~ \App{e}{\tau} ~\vert~ \fold{t}{\tau}{e} ~\vert~ \unfold{e} ~\vert~ \tpl{\mapschema{e}{i}{\labelset}} ~\vert~ \prj{e}{\ell} \\
% & \vert & \inj{\ell}{e} ~\vert~ \caseof{e}{\mapschemab{x}{e}{i}{\labelset}}\\
% ~\\
% \gheading{TSM expressions}\\
% \tsme & ::= & \tsmv ~\vert~ \utsmdef{\tau}{\ue}\\
% ~\\
% \gheading{unexpanded expressions}\\
% \ue & ::= & {x} ~\vert~ \lam{x}{\tau}{\ue} ~\vert~ \ue(\ue) ~\vert~ \Lam{t}{\ue} ~\vert~ \App{\ue}{\tau} ~\vert~ \fold{t}{\tau}{\ue} ~\vert~ \unfold{\ue} ~\vert~ \tpl{\mapschema{\ue}{i}{\labelset}} ~\vert~ \prj{\ue}{\ell} \\
% & \vert & \inj{\ell}{\ue} ~\vert~ \caseof{\ue}{\mapschemab{x}{\ue}{i}{\labelset}}\\
% & \vert & \uesyntax{\tsmv}{\tsme}{\ue} ~\vert~ \utsmapp{\eta}{b}\\
% ~\\
% \gheading{candidate expansion types}\\
% \mtau & ::= & t ~\vert~ \parr{\mtau}{\mtau} ~\vert~ \forallt{t}{\mtau} ~\vert~ \rect{t}{\mtau} ~\vert~ \prodt{\mapschema{\tau}{i}{\labelset}} ~\vert~ \sumt{\mapschema{\mtau}{i}{\labelset}} \\
% & \vert & \mtspliced{\tau}\\
% ~\\
% \gheading{candidate expansion expressions}\\
% \me & ::= & x ~\vert~ \lam{x}{\mtau}{\me} ~\vert~ \app{\me}{\me} ~\vert~ \Lam{t}{\me} ~|~ \App{\me}{\mtau} ~\vert~ \fold{t}{\mtau}{\me} ~\vert~ \unfold{\me} ~\vert~ \tpl{\mapschema{\me}{i}{\labelset}} ~\vert~ \prj{\me}{\ell} \\
% & \vert & \inj{\ell}{\me} ~\vert~ \caseof{\me}{\mapschemab{x}{\me}{i}{\labelset}}\\
% & \vert & \mspliced{e}
% % \\~
% \end{array}$
% \todo{finish breaking this up into syntax tables}
% \caption[Syntax of $\miniVerseUE$]{Syntax of $\miniVerseUE$. The forms $\mapschema{V}{i}{\labelset}$ and $\mapschemab{x}{V}{i}{\labelset}$ where $V$ is a metavariable indicate finite mappings from each label $i \in \labelset$ to a term, $V_i$, or binder, $x_i.V_i$, respectively.}
% \label{fig:lambda-tsm-syntax}
% \end{figure}


To make the intuitions developed in the previous section mathematically precise, we will now introduce a reduced calculus with support for ueTSMs called $\miniVerseUE$. 
%For reference, the syntax of $\miniVerseUE$ is specified in Figure \ref{fig:lambda-tsm-syntax}. We will reproduce relevant portions of this specification inline (in tabular form) as we continue. 
%We specify all formal systems in this document within the metatheoretic framework detailed in \emph{PFPL} \cite{pfpl}, and assume familiarity of fundamental background concepts (e.g. abstract binding trees, substitution, implicit identification of terms up to $\alpha$-equivalence, structural induction and rule induction) covered therein. %Familiarity with other accounts of typed lambda calculi should also suffice to understand the formal systems in this document. 



\subsection{Types and Expanded Expressions}\label{sec:U-expanded-terms}

\begin{figure}
$\begin{array}{lllllll}
\textbf{Sort} & & & \textbf{Uniform Form} & \textbf{Stylized Form} & \textbf{Description}\\
\mathsf{Typ} & \tau & ::= & t & t & \text{variable}\\
&&& \aparr{\tau}{\tau} & \parr{\tau}{\tau} & \text{partial function}\\
&&& \aall{t}{\tau} & \forallt{t}{\tau} & \text{polymorphic}\\
&&& \arec{t}{\tau} & \rect{t}{\tau} & \text{recursive}\\
&&& \aprod{\labelset}{\mapschema{\tau}{i}{\labelset}} & \prodt{\mapschema{\tau}{i}{\labelset}} & \text{labeled product}\\
&&& \asum{\labelset}{\mapschema{\tau}{i}{\labelset}} & \sumt{\mapschema{\tau}{i}{\labelset}} & \text{labeled sum}\\
\mathsf{EExp} & e & ::= & x & x & \text{variable}\\
&&& \aelam{\tau}{x}{e} & \lam{x}{\tau}{e} & \text{abstraction}\\
&&& \aeap{e}{e} & \ap{e}{e} & \text{application}\\
&&& \aetlam{t}{e} & \Lam{t}{e} & \text{type abstraction}\\
&&& \aetap{e}{\tau} & \App{e}{\tau} & \text{type application}\\
&&& \aefold{t}{\tau}{e} & \fold{e} & \text{fold}\\
&&& \aeunfold{e} & \unfold{e} & \text{unfold}\\
&&& \aetpl{\labelset}{\mapschema{e}{i}{\labelset}} & \tpl{\mapschema{e}{i}{\labelset}} & \text{labeled tuple}\\
&&& \aepr{\ell}{e} & \prj{e}{\ell} & \text{projection}\\
&&& \aein{\labelset}{\ell}{\mapschema{\tau}{i}{\labelset}}{e} & \inj{\ell}{e} & \text{injection}\\
&&& \aecase{\labelset}{e}{\mapschemab{x}{e}{i}{\labelset}} & \caseof{e}{\mapschemab{x}{e}{i}{\labelset}} & \text{case analysis}
\end{array}$
\caption[Syntax of types and expanded expressions in $\miniVerseUE$.]{Abstract syntax of types and expanded expressions, which form the inner core of $\miniVerseUE$. Metavariable $x$ ranges over variables, $t$ ranges over type variables, $\ell$ ranges over labels and $\labelset$ ranges over finite sets of label. We write $\mapschema{\tau}{i}{\labelset}$ for a sequence of operator arguments $\tau_i$, one for each $i\in \labelset$, and similarly for other argument arities. Operations parameterized by label sets are identified up to mutual reordering of the label set and the corresponding argument sequence. %When using stylized forms, the label set is omitted when it can be inferred, e.g. the labeled product type $\prodt{\finmap{\mapitem{\ell_1}{e_1}, \mapitem{\ell_2}{e_2}}}$ leaves the label set $\{\ell_1, \ell_2\}$ implicit. 
When we use the stylized forms $\fold{e}$ and $\inj{\ell}{e}$, we assume that the reader can infer the missing parameters and arguments from the surrounding context.}
\label{fig:U-expanded-terms}
\end{figure}

The \emph{inner core} of $\miniVerseUE$ consists of \emph{types}, $\tau$, and \emph{expanded expressions}, $e$. The syntax of the inner core is specified by the syntax chart in Figure \ref{fig:U-expanded-terms}. 
The {inner core} forms a purely functional language with support for partial functions, quantification over types, recursive types, labeled product types and labeled sum types. The reader can consult \emph{PFPL} \cite{pfpl} (or another text on type systems, e.g. \emph{TAPL} \cite{tapl}) for an introductory account of these constructs, or closely related variants thereof. For our purposes, it suffices to recall the following facts and definitions.

\subsection{Statics of the Inner Core}
The \emph{statics of the inner core} is specified by hypothetical judgements of the following form:

\[\begin{array}{ll}
\textbf{Judgement Form} & \textbf{Description}\\
\istypeU{\Delta}{\tau} & \text{$\tau$ is a well-formed type assuming $\Delta$}\\
%\isctxU{\Delta}{\Gamma} & \text{$\Gamma$ is a well-formed typing context assuming $\Delta$}\\
\hastypeU{\Delta}{\Gamma}{e}{\tau} & \text{$e$ is assigned type $\tau$ assuming $\Delta$ and $\Gamma$}
\end{array}\]
\noindent
\emph{Type formation contexts}, $\Delta$, consist of hypotheses of the form $\Dhyp{t}$ and \emph{typing contexts}, $\Gamma$, consist of hypotheses of the form $\Ghyp{x}{\tau}$. Syntactically, we write contexts as comma-separated sequences of {hypotheses} identified up to exchange and contraction. We write empty contexts $\emptyset$, or omit their mention entirely.

The type formation judgement, $\istypeU{\Delta}{\tau}$, ensures that all free type variables in $\tau$ are tracked by $\Delta$. It is inductively defined by the following rules:
\begin{subequations}\label{rules:istypeU}
\begin{equation}\label{rule:istypeU-var}
\inferrule{ }{\istypeU{\Delta, \Dhyp{t}}{t}}
\end{equation}
\begin{equation}\label{rule:istypeU-parr}
\inferrule{
  \istypeU{\Delta}{\tau_1}\\
  \istypeU{\Delta}{\tau_2}
}{\istypeU{\Delta}{\aparr{\tau_1}{\tau_2}}}
\end{equation}
\begin{equation}\label{rule:istypeU-all}
  \inferrule{
    \istypeU{\Delta, \Dhyp{t}}{\tau}
  }{
    \istypeU{\Delta}{\aall{t}{\tau}}
  }
\end{equation}
\begin{equation}\label{rule:istypeU-rec}
  \inferrule{
    \istypeU{\Delta, \Dhyp{t}}{\tau}
  }{
    \istypeU{\Delta}{\arec{t}{\tau}}
  }
\end{equation}
\begin{equation}\label{rule:istypeU-prod}
  \inferrule{
    \{\istypeU{\Delta}{\tau_i}\}_{i \in \labelset}
  }{
    \istypeU{\Delta}{\aprod{\labelset}{\mapschema{\tau}{i}{\labelset}}}
  }
\end{equation}
\begin{equation}\label{rule:istypeU-sum}
  \inferrule{
    \{\istypeU{\Delta}{\tau_i}\}_{i \in \labelset}
  }{
    \istypeU{\Delta}{\asum{\labelset}{\mapschema{\tau}{i}{\labelset}}}
  }
\end{equation}
\end{subequations}
Premises of the form $\{{J}_i\}_{i \in \labelset}$ mean that for each $i \in \labelset$, the judgement ${J}_i$ must hold. 

The typing judgement, $\hastypeU{\Delta}{\Gamma}{e}{\tau}$, assigns types to expressions. It is inductively defined by the following rules:
\begin{subequations}\label{rules:hastypeU}
\begin{equation}\label{rule:hastypeU-var}
  \inferrule{ }{
    \hastypeU{\Delta}{\Gamma, \Ghyp{x}{\tau}}{x}{\tau}
  }
\end{equation}
\begin{equation}\label{rule:hastypeU-lam}
  \inferrule{
    \istypeU{\Delta}{\tau}\\
    \hastypeU{\Delta}{\Gamma, \Ghyp{x}{\tau}}{e}{\tau'}
  }{
    \hastypeU{\Delta}{\Gamma}{\aelam{\tau}{x}{e}}{\aparr{\tau}{\tau'}}
  }
\end{equation}
\begin{equation}\label{rule:hastypeU-ap}
  \inferrule{
    \hastypeU{\Delta}{\Gamma}{e_1}{\aparr{\tau}{\tau'}}\\
    \hastypeU{\Delta}{\Gamma}{e_2}{\tau}
  }{
    \hastypeU{\Delta}{\Gamma}{\aeap{e_1}{e_2}}{\tau'}
  }
\end{equation}
\begin{equation}\label{rule:hastypeU-tlam}
  \inferrule{
    \hastypeU{\Delta, \Dhyp{t}}{\Gamma}{e}{\tau}
  }{
    \hastypeU{\Delta}{\Gamma}{\aetlam{t}{e}}{\aall{t}{\tau}}
  }
\end{equation}
\begin{equation}\label{rule:hastypeU-tap}
  \inferrule{
    \hastypeU{\Delta}{\Gamma}{e}{\aall{t}{\tau}}\\
    \istypeU{\Delta}{\tau'}
  }{
    \hastypeU{\Delta}{\Gamma}{\aetap{e}{\tau'}}{[\tau'/t]\tau}
  }
\end{equation}
\begin{equation}\label{rule:hastypeU-fold}
  \inferrule{\
    \istypeU{\Delta, \Dhyp{t}}{\tau}\\
    \hastypeU{\Delta}{\Gamma}{e}{[\arec{t}{\tau}/t]\tau}
  }{
    \hastypeU{\Delta}{\Gamma}{\aefold{t}{\tau}{e}}{\arec{t}{\tau}}
  }
\end{equation}
\begin{equation}\label{rule:hastypeU-unfold}
  \inferrule{
    \hastypeU{\Delta}{\Gamma}{e}{\arec{t}{\tau}}
  }{
    \hastypeU{\Delta}{\Gamma}{\aeunfold{e}}{[\arec{t}{\tau}/t]\tau}
  }
\end{equation}
\begin{equation}\label{rule:hastypeU-tpl}
  \inferrule{
    \{\hastypeU{\Delta}{\Gamma}{e_i}{\tau_i}\}_{i \in \labelset}
  }{
    \hastypeU{\Delta}{\Gamma}{\aetpl{\labelset}{\mapschema{e}{i}{\labelset}}}{\aprod{\labelset}{\mapschema{\tau}{i}{\labelset}}}
  }
\end{equation}
\begin{equation}\label{rule:hastypeU-pr}
  \inferrule{
    \hastypeU{\Delta}{\Gamma}{e}{\aprod{\labelset, \ell}{\mapschema{\tau}{i}{\labelset}; \ell \hookrightarrow \tau}}
  }{
    \hastypeU{\Delta}{\Gamma}{\aepr{\ell}{e}}{\tau}
  }
\end{equation}
\begin{equation}\label{rule:hastypeU-in}
  \inferrule{
    \{\istypeU{\Delta}{\tau_i}\}_{i \in \labelset}\\
    \istypeU{\Delta}{\tau}\\
    \hastypeU{\Delta}{\Gamma}{e}{\tau}
  }{
    \hastypeU{\Delta}{\Gamma}{\aein{\labelset, \ell}{\ell}{\mapschema{\tau}{i}{\labelset}; \ell \hookrightarrow \tau}{e}}{\asum{\labelset, \ell}{\mapschema{\tau}{i}{\labelset}; \ell \hookrightarrow \tau}}
  }
\end{equation}
\begin{equation}\label{rule:hastypeU-case}
  \inferrule{
    \hastypeU{\Delta}{\Gamma}{e}{\asum{\labelset}{\mapschema{\tau}{i}{\labelset}}}\\
    \{\hastypeU{\Delta}{\Gamma, x_i : \tau_i}{e_i}{\tau}\}_{i \in \labelset}
  }{
    \hastypeU{\Delta}{\Gamma}{\aecase{\labelset}{e}{\mapschemab{x}{e}{i}{\labelset}}}{\tau}
  }
\end{equation}
\end{subequations}
Rules (\ref{rules:istypeU}) and (\ref{rules:hastypeU}) are syntax-directed, so we assume an inversion lemma for each rule as needed without stating or proving it explicitly. The statics of the inner core also obeys the following standard lemmas. 

The Weakening Lemma expresses the intuition that extending a context with unused variables preserves well-formedness and typing.
\begin{lemma}[Weakening]\label{lemma:weakening-U} All of the following hold: 
\begin{enumerate} 
\item If $\istypeU{\Delta}{\tau}$ then $\istypeU{\Delta, \Dhyp{t}}{\tau}$.
%\item If $\isctxU{\Delta}{\Gamma}$ then $\isctxU{\Delta, \Dhyp{t}}{\Gamma}$.
\item If $\hastypeU{\Delta}{\Gamma}{e}{\tau}$ then $\hastypeU{\Delta, \Dhyp{t}}{\Gamma}{e}{\tau}$.
\item If $\hastypeU{\Delta}{\Gamma}{e}{\tau}$ and $\istypeU{\Delta}{\tau'}$ then $\hastypeU{\Delta}{\Gamma, \Ghyp{x}{\tau'}}{e}{\tau}$.
\end{enumerate}
\end{lemma}
\begin{proof-sketch} For each part, by rule induction on the derivation of the assumption. 
%\begin{enumerate} 
%\item By rule induction on Rules (\ref{rules:istypeU}).
%\item By rule induction on Rules (\ref{rules:isctxU}).
%\item By rule induction on Rules (\ref{rules:hastypeU}).
%\item By rule induction on Rules (\ref{rules:hastypeU}).
%\end{enumerate}
\end{proof-sketch}

The Substitution Lemma expresses the intuition that substitution of a well-formed type for a type variable, or an expression of the appropriate type for an expression variable, preserves well-formedness and typing. 
\begin{lemma}[Substitution]\label{lemma:substitution-U} All of the following hold:
\begin{enumerate}
\item If $\istypeU{\Delta, \Dhyp{t}}{\tau}$ and $\istypeU{\Delta}{\tau'}$ then $\istypeU{\Delta}{[\tau'/t]\tau}$.
%\item If $\isctxU{\Delta, \Dhyp{t}}{\Gamma}$ and $\istypeU{\Delta}{\tau'}$ then $\isctxU{\Delta}{[\tau'/t]\Gamma}$.
\item If $\hastypeU{\Delta, \Dhyp{t}}{\Gamma}{e}{\tau}$ and $\istypeU{\Delta}{\tau'}$ then $\hastypeU{\Delta}{[\tau'/t]\Gamma}{[\tau'/t]e}{[\tau'/t]\tau}$.
\item If $\hastypeU{\Delta}{\Gamma, \Ghyp{x}{\tau'}}{e}{\tau}$ and $\hastypeU{\Delta}{\Gamma}{e'}{\tau'}$ then $\hastypeU{\Delta}{\Gamma}{[e'/x]e}{\tau}$.
\end{enumerate}\end{lemma}
\begin{proof-sketch}
For each part, by rule induction on the derivation of the first assumption.
\end{proof-sketch}

The Decomposition Lemma is the converse of the Substitution Lemma.
\begin{lemma}[Decomposition]\label{lemma:decomposition-U} All of the following hold:
\begin{enumerate}
\item If $\istypeU{\Delta}{[\tau'/t]\tau}$ and $\istypeU{\Delta}{\tau'}$ then $\istypeU{\Delta, \Dhyp{t}}{\tau}$.
%\item If $\isctxU{\Delta}{[\tau'/t]\Gamma}$ and $\istypeU{\Delta}{\tau'}$ then $\isctxU{\Delta, \Dhyp{t}}{\Gamma}$.
\item If $\hastypeU{\Delta}{[\tau'/t]\Gamma}{[\tau'/t]e}{[\tau'/t]\tau}$ and $\istypeU{\Delta}{\tau'}$ then $\hastypeU{\Delta, \Dhyp{t}}{\Gamma}{e}{\tau}$.
\item If $\hastypeU{\Delta}{\Gamma}{[e'/x]e}{\tau}$ and $\hastypeU{\Delta}{\Gamma}{e'}{\tau'}$ then $\hastypeU{\Delta}{\Gamma, \Ghyp{x}{\tau'}}{e}{\tau}$.
\end{enumerate}\end{lemma}
\begin{proof-sketch}
\begin{enumerate}
\item By rule induction on Rules (\ref{rules:istypeU}) and case analysis on the definition of substitution. In all cases, the derivation of $\istypeU{\Delta}{[\tau'/t]\tau}$ does not depend on the structure of $\tau'$.
%\item Context formation of $[\tau'/t]\Gamma$ does not depend on the structure of $\tau'$.
\item By rule induction on Rules (\ref{rules:hastypeU}) and case analysis on the definition of substitution. In all cases, the derivation of $\hastypeU{\Delta}{[\tau'/t]\Gamma}{[\tau'/t]e}{[\tau'/t]\tau}$ does not depend on the structure of $\tau'$.
\item By rule induction on Rules (\ref{rules:hastypeU}) and case analysis on the definition of substitution. In all cases, the derivation of $\hastypeU{\Delta}{\Gamma}{[e'/x]e}{\tau}$ does not depend on the structure of $e'$.
\end{enumerate}
\end{proof-sketch}

The Regularity Lemma expresses the intuition that the type assigned to an expression under a well-formed typing context is well-formed. 
\begin{lemma}[Regularity]\label{lemma:regularity-U} If $\hastypeU{\Delta}{\Gamma}{e}{\tau}$ and $\istypeU{\Delta}{\tau_i}$ for each assumption $x_i : \tau_i$ in $\Gamma$ then $\istypeU{\Delta}{\tau}$.\end{lemma}
\begin{proof-sketch}
By rule induction on Rules (\ref{rules:hastypeU}) and Lemma \ref{lemma:substitution-U}.
\end{proof-sketch}
\subsection{Structural Dynamics}\label{sec:dynamics-U}
The \emph{structural dynamics of }$\miniVerseUE$ is specified by judgements of the following form:
\[\begin{array}{ll}
\textbf{Judgement Form} & \textbf{Description}\\
\stepsU{e}{e'} & \text{$e$ transitions to $e'$}\\
\isvalU{e} & \text{$e$ is a value}
\end{array}\]
We also define auxiliary judgements for \emph{iterated transition}, $\multistepU{e}{e'}$, and \emph{evaluation}, $\evalU{e}{e'}$.

\begin{definition}[Iterated Transition]\label{defn:iterated-transition-U} Iterated transition, $\multistepU{e}{e'}$, is the reflexive, transitive closure of the transition judgement.\end{definition}

\begin{definition}[Evaluation]\label{defn:evaluation-U}  $\evalU{e}{e'}$ iff $\multistepU{e}{e'}$ and $\isvalU{e'}$. \end{definition}

Our subsequent developments do not require making reference to particular rules in the structural dynamics (because TSMs operate statically), so we do not reproduce the rules here. Instead, it suffices to state the following conditions.

The Canonical Forms condition characterizes well-typed values. We assume an \emph{eager} (i.e. \emph{by-value}) formulation of the dynamics. 
\begin{condition}[Canonical Forms]\label{condition:canonical-forms-U} If $\hastypeUC{e}{\tau}$ and $\isvalU{e}$ then:
\begin{enumerate}
\item If $\tau=\aparr{\tau_1}{\tau_2}$ then $e=\aelam{\tau_1}{x}{e'}$ and $\hastypeUCO{\Ghyp{x}{\tau_1}}{e'}{\tau_2}$.
\item If $\tau=\aall{t}{\tau'}$ then $e=\aetlam{t}{e'}$ and $\hastypeUCO{\Dhyp{t}}{e'}{\tau'}$.
\item If $\tau=\arec{t}{\tau'}$ then $e=\aefold{t}{\tau'}{e'}$ and $\hastypeUC{e'}{[\abop{rec}{t.\tau'}/t]\tau'}$ and $\isvalU{e'}$. 
\item If $\tau=\aprod{\labelset}{\mapschema{\tau}{i}{\labelset}}$ then $e=\aetpl{\labelset}{\mapschema{e}{i}{\labelset}}$ and $\hastypeUC{e_i}{\tau_i}$ and $\isvalU{e_i}$ for each $i \in \labelset$.
\item If $\tau=\asum{\labelset}{\mapschema{\tau}{i}{\labelset}}$ then for some label set $L'$ and label $\ell$, we have that $\labelset=\labelset', \ell$ and $\tau=\asum{\labelset', \ell}{\mapschema{\tau}{i}{\labelset'}; \mapitem{\ell}{\tau_\ell}}$ and $e=\aein{\labelset', \ell}{\ell}{\mapschema{\tau}{i}{\labelset'}; \ell \hookrightarrow \tau_\ell}{e'}$ and $\hastypeUC{e'}{\tau_\ell}$ and $\isvalU{e'}$.
\end{enumerate}\end{condition}

The Preservation and Progress conditions (together, the {Type Safety} condition) ensure that the statics of the inner core and the structural dynamics are coherent, i.e. evaluation of a well-typed expression preserves typing and does not get stuck.
\begin{condition}[Preservation]\label{condition:preservation-U} If $\hastypeUC{e}{\tau}$ and $\stepsU{e}{e'}$ then $\hastypeUC{e'}{\tau}$. \end{condition}
\begin{condition}[Progress]\label{condition:progress-U} If $\hastypeUC{e}{\tau}$ then either $\isvalU{e}$ or there exists an $e'$ such that $\stepsU{e}{e'}$. \end{condition}


\subsection{Typed Expansion}\label{sec:typed-expansion-U}
Programs ultimately evaluate as expanded expressions, but programmers author \emph{unex\-panded expressions}, $\ue$. The syntax of unexpanded expressions is specified in Figure \ref{fig:U-unexpanded-terms}. % Let us define this correspondence by the metafunction $\Uof{e}$:
%\[
%\begin{split}
%\Uof{x} & = x\\
%\Uof{\aelam{\tau}{x}{e}} & = \aulam{\tau}{x}{\Uof{e}}\\
%\Uof{\aeap{e_1}{e_2}} & = \auap{\Uof{e_1}}{\Uof{e_2}}
%\end{split}
%\] and so on for the remaining expanded expression forms.

Unexpanded expressions are typed and expanded simultaneously according to the \emph{typed expansion judgement}:
\[\begin{array}{ll}
\textbf{Judgement Form} & \textbf{Description}\\
\expandsUX{\ue}{e}{\tau} & \text{$\ue$ expands to $e$ at type $\tau$ assuming $\Delta$ and $\Gamma$ under ueTSM}\\
& \text{environment $\Sigma$}
\end{array}\]

\begin{figure}
\hspace{-5px}$\begin{array}{lllllll}
\textbf{Sort} & & & \textbf{Uniform Form} & \textbf{Stylized Form} & \textbf{Description}\\
\mathsf{UExp} & \ue & ::= & x & x & \text{variable}\\
&&& \aulam{\tau}{x}{\ue} & \lam{x}{\tau}{\ue} & \text{abstraction}\\
&&& \auap{\ue}{\ue} & \ap{\ue}{\ue} & \text{application}\\
&&& \autlam{t}{\ue} & \Lam{t}{\ue} & \text{type abstraction}\\
&&& \autap{\ue}{\tau} & \App{\ue}{\tau} & \text{type application}\\
&&& \aufold{t}{\tau}{\ue} & \fold{\ue} & \text{fold}\\
&&& \auunfold{\ue} & \unfold{\ue} & \text{unfold}\\
&&& \autpl{\labelset}{\mapschema{\ue}{i}{\labelset}} & \tpl{\mapschema{\ue}{i}{\labelset}} & \text{labeled tuple}\\
&&& \aupr{\ell}{\ue} & \prj{\ue}{\ell} & \text{projection}\\
&&& \auin{\labelset}{\ell}{\mapschema{\tau}{i}{\labelset}}{\ue} & \inj{\ell}{\ue} & \text{injection}\\
&&& \aucase{\labelset}{\ue}{\mapschemab{x}{\ue}{i}{\labelset}} & \caseof{\ue}{\mapschemab{x}{\ue}{i}{\labelset}} & \text{case analysis}\\
&&& \audefuetsm{\tau}{\ue}{\tsmv}{\ue} & \uesyntax{\tsmv}{\tau}{\ue}{\ue} & \text{ueTSM definition}\\
&&& \autsmap{b}{\tsmv} & \utsmap{\tsmv}{b} & \text{ueTSM application}
\end{array}$
\caption[Syntax of unexpanded expressions in $\miniVerseUE$]{Abstract syntax of unexpanded expressions in $\miniVerseUE$. Metavariable $\tsmv$ ranges over TSM names and $b$ ranges over literal bodies. Literal bodies might contain unparsed subexpressions, so variable renaming and substitution cannot be defined in the usual manner over unexpanded expressions (i.e. unexpanded expressions should be considered ``partially parsed'' abstract syntax trees, not abstract binding trees.)}
\label{fig:U-unexpanded-terms}
\end{figure}
\noindent
The typed expansion judgement is inductively defined as follows:
\begin{itemize}
\item For each expanded expression form specified in Figure \ref{fig:U-expanded-terms}, there is a corresponding unexpanded expression form in Figure \ref{fig:U-unexpanded-terms}. The corresponding typed expansion rules are derived from the typing rules in Rules (\ref{rules:hastypeU}). For example, the rules for typed expansion of unexpanded expressions of variable, function and function application form are below: \begin{subequations}\label{rules:expandsU}
\begin{equation}\label{rule:expandsU-var}
  \inferrule{ }{\expandsU{\Delta}{\Gamma, x : \tau}{\Sigma}{x}{x}{\tau}}
\end{equation}
\begin{equation}\label{rule:expandsU-lam}
  \inferrule{
    \istypeU{\Delta}{\tau}\\
    \expandsU{\Delta}{\Gamma, x : \tau}{\Sigma}{\ue}{e}{\tau'}
  }{\expandsUX{\aulam{\tau}{x}{\ue}}{\aelam{\tau}{x}{e}}{\aparr{\tau}{\tau'}}}
\end{equation}
\begin{equation}\label{rule:expandsU-ap}
  \inferrule{
    \expandsUX{\ue_1}{e_1}{\aparr{\tau}{\tau'}}\\
    \expandsUX{\ue_2}{e_2}{\tau}
  }{
    \expandsUX{\auap{\ue_1}{\ue_2}}{\aeap{e_1}{e_2}}{\tau'}
  }
\end{equation}
Generically, for each rule $\rulemv$ in Rules (\ref{rules:hastypeU}),
\begin{mathpar}
\inferrule{J_1\\ \cdots \\ J_k}{J}
\end{mathpar}
we derive a typed expansion rule $\Uof{\rulemv}$ as follows:
\begin{mathpar}
\inferrule{
  \Uof{J_1} \\
  \cdots\\
  \Uof{J_k}
}{
  \Uof{J}
}
\end{mathpar}
where
\[\begin{split}
\Uof{\istypeU{\Delta}{\tau}} & = \istypeU{\Delta}{\tau} \\
\Uof{\hastypeU{\Gamma}{\Delta}{e}{\tau}} & = \expandsU{\Gamma}{\Delta}{\Sigma}{\Uof{e}}{e}{\tau}\\
\Uof{\{J_i\}_{i \in \labelset}} & = \{\Uof{J_i}\}_{i \in \labelset}
\end{split}\]
and where $\Uof{e}$ is defined as follows:
\begin{itemize}
\item When $e$ is of indefinite form, $\Uof{e}$ is a uniquely corresponding metavariable of indefinite form ranging over unexpanded expressions.
\item When $e$ is of definite form, $\Uof{e}$ is defined as follows:
\[\begin{split}
\Uof{x} & = x\\
\Uof{\aelam{\tau}{x}{e}} & = \aulam{\tau}{x}{\Uof{e}}\\
\Uof{\aeap{e_1}{e_2}} & = \auap{\Uof{e_1}}{\Uof{e_2}}
\end{split}\]
and so on for the remaining expanded expression forms.
\end{itemize}

By giving the rules for corresponding forms generically like this, our specification is robust to changes to the structure of the inner core (as long as these changes do not require changes to the judgement forms in the statics of the inner core).% Even if changes to the judgement forms in the statics of the inner core are needed (e.g. the addition of a symbol context), it is easy to see would correspond to changes in the generic specification above.
% \begin{subequations}\label{rules:expandsU}
% \begin{equation}\label{rule:expandsU-var}
%   \inferrule{ }{\expandsU{\Delta}{\Gamma, x : \tau}{\Sigma}{x}{x}{\tau}}
% \end{equation}
% \begin{equation}\label{rule:expandsU-lam}
%   \inferrule{
%     \istypeU{\Delta}{\tau}\\
%     \expandsU{\Delta}{\Gamma, x : \tau}{\Sigma}{\ue}{e}{\tau'}
%   }{\expandsUX{\aulam{\tau}{x}{\ue}}{\aelam{\tau}{x}{e}}{\aparr{\tau}{\tau'}}}
% \end{equation}
% \begin{equation}\label{rule:expandsU-ap}
%   \inferrule{
%     \expandsUX{\ue_1}{e_1}{\aparr{\tau}{\tau'}}\\
%     \expandsUX{\ue_2}{e_2}{\tau}
%   }{
%     \expandsUX{\auap{\ue_1}{\ue_2}}{\aeap{e_1}{e_2}}{\tau'}
%   }
% \end{equation}
% \begin{equation}\label{rule:expandsU-tlam}
%   \inferrule{
%     \expandsU{\Delta, \Dhyp{t}}{\Gamma}{\Sigma}{\ue}{e}{\tau}
%   }{
%     \expandsUX{\autlam{t}{\ue}}{\aetlam{t}{e}}{\aall{t}{\tau}}
%   }
% \end{equation}
% \begin{equation}\label{rule:expandsU-tap}
%   \inferrule{
%     \expandsUX{\ue}{e}{\aall{t}{\tau}}\\
%     \istypeU{\Delta}{\tau'}
%   }{
%     \expandsUX{\autap{\ue}{\tau'}}{\aetap{e}{\tau'}}{[\tau'/t]\tau}
%   }
% \end{equation}
% \begin{equation}\label{rule:expandsU-fold}
%   \inferrule{
%     \istypeU{\Delta, \Dhyp{t}}{\tau}\\
%     \expandsUX{\ue}{e}{[\arec{t}{\tau}/t]\tau}
%   }{
%     \expandsUX{\aufold{t}{\tau}{\ue}}{\aefold{t}{\tau}{e}}{\arec{t}{\tau}}
%   }
% \end{equation}
% \begin{equation}\label{rule:expandsU-unfold}
%   \inferrule{
%     \expandsUX{\ue}{e}{\arec{t}{\tau}}
%   }{
%     \expandsUX{\auunfold{\ue}}{\aeunfold{e}}{[\arec{t}{\tau}/t]\tau}
%   }
% \end{equation}
% \begin{equation}\label{rule:expandsU-tpl}
%   \inferrule{
%     \{\expandsUX{\ue_i}{e_i}{\tau_i}\}_{i \in \labelset}
%   }{
%     \expandsUX{\autpl{\labelset}{\mapschema{\ue}{i}{\labelset}}}{\aetpl{\labelset}{\mapschema{e}{i}{\labelset}}}{\aprod{\labelset}{\mapschema{\tau}{i}{\labelset}}}
%   }
% \end{equation}
% \begin{equation}\label{rule:expandsU-pr}
%   \inferrule{
%     \expandsUX{\ue}{e}{\aprod{\labelset, \ell}{\mapschema{\tau}{i}{\labelset}; \mapitem{\ell}{\tau}}}
%   }{
%     \expandsUX{\aupr{\ell}{\ue}}{\aepr{\ell}{e}}{\tau}
%   }
% \end{equation}
% \begin{equation}\label{rule:expandsU-in}
%   \inferrule{
%     \{\istypeU{\Delta}{\tau_i}\}_{i \in \labelset}\\
%     \istypeU{\Delta}{\tau}\\
%     \expandsUX{\ue}{e}{\tau}
%   }{
%     \left\{\shortstack{$\Delta~\Gamma \vdash_\Sigma \auin{\labelset, \ell}{\ell}{\mapschema{\tau}{i}{\labelset}; \mapitem{\ell}{\tau}}{\ue}$\\$\leadsto$\\$\aein{\labelset, \ell}{\ell}{\mapschema{\tau}{i}{\labelset}; \mapitem{\ell}{\tau}}{e} : \asum{\labelset, \ell}{\mapschema{\tau}{i}{\labelset}; \mapitem{\ell}{\tau}}$\vspace{-1.2em}}\right\}
%   }
% \end{equation}
% \begin{equation}\label{rule:expandsU-case}
%   \inferrule{
%     \expandsUX{\ue}{e}{\asum{\labelset}{\mapschema{\tau}{i}{\labelset}}}\\
%     \{\expandsU{\Delta}{\Gamma, \Ghyp{x_i}{\tau_i}}{\Sigma}{\ue_i}{e_i}{\tau}\}_{i \in \labelset}
%   }{
%     \expandsUX{\aucase{\labelset}{\ue}{\mapschemab{x}{\ue}{i}{\labelset}}}{\aecase{\labelset}{e}{\mapschemab{x}{e}{i}{\labelset}}}{\tau}
%   }
% \end{equation}
\item There are two unexpanded expression forms that do not correspond to expanded expression forms: the ueTSM definition form and the ueTSM application form. The rules governing these two forms are given below and discussed in the next two subsections, respectively. 
\setcounter{equation}{11}
\begin{equation}\label{rule:expandsU-syntax}
\inferrule{
  \istypeU{\Delta}{\tau}\\
  \expandsU{\emptyset}{\emptyset}{\emptyset}{\ueparse}{\eparse}{\aparr{\tBody}{\tParseResultExp}}\\\\
  a \notin \domof{\Sigma}\\
  \expandsU{\Delta}{\Gamma}{\Sigma, \xuetsmbnd{a}{\tau}{\eparse}}{\ue}{e}{\tau'}
}{
  \expandsUX{\audefuetsm{\tau}{\ueparse}{a}{\ue}}{e}{\tau'}
}
\end{equation}
\begin{equation}\label{rule:expandsU-tsmap}
\inferrule{
  \encodeBody{b}{\ebody}\\
  \evalU{\ap{\eparse}{\ebody}}{\inj{\lbltxt{Success}}{\ecand}}\\
  \decodeCondE{\ecand}{\ce}\\\\
  \cvalidE{\emptyset}{\emptyset}{\esceneU{\Delta}{\Gamma}{\Sigma, \xuetsmbnd{a}{\tau}{\eparse}}{b}}{\ce}{e}{\tau}
}{
  \expandsU{\Delta}{\Gamma}{\Sigma, \xuetsmbnd{a}{\tau}{\eparse}}{\autsmap{b}{a}}{e}{\tau}
}
\end{equation}
\end{subequations}
\end{itemize}
%Notice that each form of expanded expression (Figure \ref{fig:U-expanded-terms}) corresponds to a form of unexpanded expression (Figure \ref{fig:U-unexpanded-terms}). For each typing rule in Rules (\ref{rules:hastypeU}), there is a corresponding typed expansion rule -- Rules (\ref{rule:expandsU-var}) through (\ref{rule:expandsU-case}) -- where the unexpanded and expanded forms correspond. The premises also correspond -- if a typing judgement appears as a premise of a typing rule, then the corresponding premise in the corresponding typed expansion rule is the corresponding typed expansion judgement. The ueTSM environment is not extended or inspected by these rules (it is only ``threaded through'' them opaquely).

%There are two unexpanded expression forms that do not correspond to an expanded expression form: the ueTSM definition form, and the ueTSM application form. The rules governing these two forms interact with the ueTSM environment, and are the topics of the next two subsections, respectively.

\subsection{ueTSM Definition}\label{sec:U-uetsm-definition}
The \emph{ueTSM definition form}: $$\uesyntax{a}{\tau}{\ueparse}{\ue}$$ allows the programmer to introduce a {ueTSM} named $a$ at type $\tau$ with \emph{unexpanded parse function} $\ueparse$ into the ueTSM environment of $\ue$. The uniform form corresponding to this stylized form is $\audefuetsm{\tau}{\ueparse}{a}{\ue}$. 
The premises of Rule (\ref{rule:expandsU-syntax}), which governs this form, can be understood as follows, in order:
\begin{enumerate}
\item The first premise ensures that the type annotation is well-formed.
\item The second premise types and expands the {unexpanded parse function}, $\ueparse$, to produce the \emph{expanded parse function}, $\eparse$. Parse functions are applied statically when the TSM that they are associated with is applied (as we will discuss in the next subsection), so they cannot be given access to the surrounding bindings. %Notice that this occurs under empty contexts, i.e. parse functions cannot refer to the surrounding bindings. 
The parse function must be of type \[\aparr{\tBody}{\tParseResultExp}\] where 
$\tParseResultExp$ abbreviates the following labeled sum type\footnote{In VerseML, the \li{ParseError} constructor of \li{ParseResult} required an error message and an error location, but we omit these in our formalization for simplicity}:
\[
\tParseResultExp \triangleq [\mapitem{\lbltxt{Success}}{\tCEExp}, \mapitem{\lbltxt{ParseError}}{\prodt{}}]
\] and 
 $\tBody$ and $\tCEExp$ abbreviate types that we will characterize below. 
\item The third premise of Rule (\ref{rule:expandsU-syntax}) checks that there is not already a ueTSM named $a$ in the ueTSM environment, $\Sigma$. 

ueTSM environments are finite mappings from TSM names, $a$, to \emph{expanded ueTSM definitions}, $\xuetsmdef{\tau}{\eparse}$, where $\tau$ is the ueTSM's {type annotation} and $\eparse$ is the ueTSM's {expanded parse function}. The \emph{ueTSM environment formation judgement}, $\uetsmenv{\Delta}{\Sigma}$, ensures that the type annotations in $\Sigma$ are well-formed assuming $\Delta$, and that the parse functions in $\Sigma$ are  of type $\aparr{\tBody}{\tParseResultExp}$.
\[\begin{array}{ll}
\textbf{Judgement Form} & \textbf{Description}\\
\uetsmenv{\Delta}{\Sigma} & \text{ueTSM environment $\Sigma$ is well-formed assuming $\Delta$}\end{array}\]
This judgement is inductively defined by the following rules:
\begin{subequations}\label{rules:uetsmenv-U}
\begin{equation}\label{rule:uetsmenv-empty}
\inferrule{ }{\uetsmenv{\Delta}{\emptyset}}
\end{equation}
\begin{equation}\label{rule:uetsmenv-ext}
\inferrule{
  \uetsmenv{\Delta}{\Sigma}\\
  \istypeU{\Delta}{\tau}\\
  \hastypeU{\emptyset}{\emptyset}{\eparse}{\aparr{\tBody}{\tParseResultExp}}
}{
  \uetsmenv{\Delta}{\Sigma, \xuetsmbnd{a}{\tau}{\eparse}}
}
\end{equation}
\end{subequations}

\item 
The fourth premise of Rule (\ref{rule:expandsU-syntax}) extends the ueTSM environment with the newly determined expanded ueTSM definition and proceeds to assign a type, $\tau'$, and expansion, $e$, for $\ue$.
\end{enumerate}
The conclusion of Rule (\ref{rule:expandsU-syntax}) specifies $\tau'$ and $e$ as the type and expansion of the expression as a whole, i.e. TSM definitions ``disappear'' when an unexpanded expression is expanded (because they specify behavior that is relevant only during typed expansion). 

\subsection{ueTSM Application}\label{sec:U-uetsm-application}
The last form of unexpanded expression is the form for applying a ueTSM named $a$ to a literal form with literal body $b$:
\[
\utsmap{a}{b}
\] 
The stylized form for ueTSM application, shown above, uses forward slashes as delimiters, though stylized variants of any of the literal forms specified in Figure \ref{fig:literal-forms} would be straightforward to add to the syntax table in Figure \ref{fig:U-unexpanded-terms} (we omit them for simplicity).

The uniform form corresponding to this stylized form is $\autsmap{b}{a}$, i.e. there is an operator $\texttt{uapuetsm}[b]$ for each literal body $b$. This operator is indexed by the TSM name $a$ and takes no arguments. TSM names are symbolic, i.e. they can be compared for equality and disequality, but unlike variables, they do not stand for terms. Instead, they are simply references into the ueTSM environment. 

The premises of Rule (\ref{rule:expandsU-tsmap}), which governs the ueTSM application form, can be understood as follows, in order:
\begin{enumerate}
\item The \emph{body encoding judgement} $\encodeBody{b}{\ebody}$ specifies a mapping from the literal body, $b$, to an expanded value, $\ebody$, of type $\tBody$. An inverse mapping is specified by the \emph{body decoding judgement} $\decodeBody{\ebody}{b}$.
\[\begin{array}{ll}
\textbf{Judgement Form} & \textbf{Description}\\
\encodeBody{b}{e} & \text{$b$ encodes to $e$}\\
\decodeBody{e}{b} & \text{$e$ decodes to $b$}
\end{array}\]
Picking a particular definition of $\tBody$ and inductively defining these judgements against it would be tedious and uninteresting. Moreover, there are many definitions of $\tBody$ that would work within our system. For these reasons, we instead simply state  the following conditions, which establish an isomorphism between literal bodies and values of type $\tBody$.
\begin{condition}[Body Encoding and Decoding] All of the following hold:
\begin{enumerate}
\item For every literal body $b$, we have that $\encodeBody{b}{\ebody}$ and $\hastypeUC{\ebody}{\tBody}$ and $\isvalU{\ebody}$.
\item If $\hastypeUC{\ebody}{\tBody}$ and $\isvalU{\ebody}$ then $\decodeBody{\ebody}{b}$ for some $b$.
\item If $\encodeBody{b}{\ebody}$ then $\decodeBody{\ebody}{b}$.
\item If $\hastypeUC{\ebody}{\tBody}$ and $\isvalU{\ebody}$ and $\decodeBody{\ebody}{b}$ then $\encodeBody{b}{\ebody}$. 
\item If $\encodeBody{b}{\ebody}$ and $\encodeBody{b}{\ebody'}$ then $\ebody = \ebody'$.
\item If $\hastypeUC{\ebody}{\tBody}$ and $\isvalU{\ebody}$ and $\decodeBody{\ebody}{b}$ and $\decodeBody{\ebody}{b'}$ then $b=b'$.
\end{enumerate}
\end{condition}
\item The second premise applies the expanded parse function $\eparse$ associated with $a$ in the ueTSM environment to $\ebody$. If parsing succeeds, i.e. a value of the (stylized) form $\inj{\lbltxt{Success}}{\ecand}$ results from evaluation, then $\ecand$ will be a value of type $\tCEExp$ (assuming a well-formed ueTSM environment and transitive application of Assumption \ref{condition:preservation-U}). We call $\ecand$ the \emph{encoding of the candidate expansion}.

If the parse function produces a value labeled $\lbltxt{ParseError}$, then typed expansion fails. No rule is necessary to handle this case. 

\item The third premise of Rule (\ref{rule:expandsU-tsmap}), $\decodeCondE{\ecand}{\ce}$, decodes the encoding of the candidate expansion, i.e. it maps $\ecand$ onto a \emph{candidate expansion expression}, $\ce$. The syntax of candidate expansion expressions and \emph{candidate expansion types}, $\ctau$, which can occur within candidate expansion expressions, is specified in Figure \ref{fig:U-candidate-terms}. The inverse mapping is specified by the judgement $\encodeCondE{\ce}{\ecand}$. 
\[\begin{array}{ll}
\textbf{Judgement Form} & \textbf{Description}\\
\decodeCondE{e}{\ce} & \text{$e$ decodes to $\ce$}\\
\encodeCondE{\ce}{e} & \text{$\ce$ encodes to $e$}
\end{array}\]

Again, defining the type $\tCEExp$ (and other associated types) explicitly, and these judgements inductively against these definitions, would be tedious and uninteresting, and would constitute an overly specific specification. Instead, we simply state the following conditions, which establish an isomorphism between values of type $\tCEExp$ and candidate expansion expressions.

\begin{condition}[Candidate Expansion Expression Encoding and Decoding] All of the following hold:
\begin{enumerate}
\item If $\hastypeUC{\ecand}{\tCEExp}$ and $\isvalU{\ecand}$ then $\decodeCondE{\ecand}{\ce}$ for some $\ce$.
\item For every $\ce$, we have $\encodeCondE{\ce}{\ecand}$ such that $\hastypeUC{\ecand}{\tCEExp}$ and $\isvalU{\ecand}$.
\item If $\hastypeUC{\ecand}{\tCEExp}$ and $\isvalU{\ecand}$ and $\decodeCondE{\ecand}{\ce}$ then $\encodeCondE{\ce}{\ecand}$.
\item If $\encodeCondE{\ce}{\ecand}$ then $\decodeCondE{\ce}{\ecand}$.
\item If $\hastypeUC{\ecand}{\tCEExp}$ and $\isvalU{\ecand}$ and $\decodeCondE{\ecand}{\ce}$ and $\decodeCondE{\ecand}{\ce'}$ then $\ce=\ce'$.
\item If $\encodeCondE{\ce}{\ecand}$ and $\encodeCondE{\ce}{\ecand'}$ then $\ecand=\ecand'$.
\end{enumerate}
\end{condition}

\item The final premise of Rule (\ref{rule:expandsU-tsmap}) \emph{validates} the candidate expansion and simultaneously generates the \emph{final expansion}, $e$. This is the topic of the next subsection.
\end{enumerate}

\subsection{Candidate Expansion Validation and Final Expansion}\label{sec:ce-validation-U}
\begin{figure}
$\begin{array}{lllllll}
\textbf{Sort} & & & \textbf{Uniform Form} & \textbf{Stylized Form} & \textbf{Description}\\
\mathsf{CETyp} & \ctau & ::= & t & t & \text{variable}\\
&&& \aceparr{\ctau}{\ctau} & \parr{\ctau}{\ctau} & \text{partial function}\\
&&& \aceall{t}{\ctau} & \forallt{t}{\ctau} & \text{polymorphic}\\
&&& \acerec{t}{\ctau} & \rect{t}{\ctau} & \text{recursive}\\
&&& \aceprod{\labelset}{\mapschema{\ctau}{i}{\labelset}} & \prodt{\mapschema{\ctau}{i}{\labelset}} & \text{labeled product}\\
&&& \acesum{\labelset}{\mapschema{\ctau}{i}{\labelset}} & \sumt{\mapschema{\ctau}{i}{\labelset}} & \text{labeled sum}\\
&&& \acesplicedt{m}{n} & \splicedt{m}{n} & \text{spliced}\\
\mathsf{CEExp} & \ce & ::= & x & x & \text{variable}\\
&&& \acelam{\ctau}{x}{\ce} & \lam{x}{\ctau}{\ce} & \text{abstraction}\\
&&& \aceap{\ce}{\ce} & \ap{\ce}{\ce} & \text{application}\\
&&& \acetlam{t}{\ce} & \Lam{t}{\ce} & \text{type abstraction}\\
&&& \acetap{\ce}{\ctau} & \App{\ce}{\ctau} & \text{type application}\\
&&& \acefold{t}{\ctau}{\ce} & \fold{\ce} & \text{fold}\\
&&& \aceunfold{\ce} & \unfold{\ce} & \text{unfold}\\
&&& \acetpl{\labelset}{\mapschema{\ce}{i}{\labelset}} & \tpl{\mapschema{\ce}{i}{\labelset}} & \text{labeled tuple}\\
&&& \acepr{\ell}{\ce} & \prj{\ce}{\ell} & \text{projection}\\
&&& \acein{\labelset}{\ell}{\mapschema{\ctau}{i}{\labelset}}{\ce} & \inj{\ell}{\ce} & \text{injection}\\
&&& \acecase{\labelset}{\ce}{\mapschemab{x}{\ce}{i}{\labelset}} & \caseof{\ce}{\mapschemab{x}{\ce}{i}{\labelset}} & \text{case analysis}\\
&&& \acesplicede{m}{n} & \splicede{m}{n} & \text{spliced}
\end{array}$
\caption[Syntax of candidate expansion types and expressions in $\miniVerseUE$]{Abstract syntax of candidate expansion types, $\ctau$ and expressions, $\ce$ (pro\-nounced ``grave $\tau$'' and ``grave $e$'', respectively), in $\miniVerseUE$. Metavariables $m$ and $n$ range over natural numbers.}
\label{fig:U-candidate-terms}
\end{figure}

The \emph{candidate expansion validation judgements} validate candidate expansion types, $\ctau$, and candidate expansion expressions, $\ce$, and simultaneously generate their final expansions.
\[\begin{array}{ll}
\textbf{Judgement Form} & \textbf{Description}\\
\cvalidT{\Delta}{\tscenev}{\ctau}{\tau} & \text{Candidate expansion type $\ctau$ expands to $\tau$ assuming $\Delta$ and }\\
& \text{type splicing scene $\tscenev$.}\\
\cvalidE{\Delta}{\Gamma}{\escenev}{\ce}{e}{\tau} & \text{Candidate expansion expression $\ce$ expands to $e$ at type $\tau$}\\
& \text{assuming $\Delta$ and $\Gamma$ and with expression splicing scene $\escenev$.}
\end{array}\]
\emph{Expression splicing scenes}, $\escenev$, are of the form $\esceneU{\Delta}{\Gamma}{\Sigma}{b}$. They consist of the type formation context, $\Delta$, the typing context, $\Gamma$, the ueTSM environment, $\Sigma$, and the literal body, $b$, from the ueTSM application site (cf. Rule (\ref{rule:expandsU-tsmap})). \emph{Type splicing scenes}, $\tscenev$, are of the form $\tsceneU{\Delta}{b}$. We write $\tsfrom{\escenev}$ for the type splicing scene constructed by dropping the typing context and ueTSM environment from $\escenev$:
\[\tsfrom{\esceneU{\Delta}{\Gamma}{\Sigma}{b}} = \tsceneU{\Delta}{b}\]

\subsubsection{Candidate Expansion Type Validation}
The \emph{candidate expansion type validation judgement}, $\cvalidT{\Delta}{\tscenev}{\ctau}{\tau}$, is inductively defined as follows:
\begin{itemize}
\item For each type form specified in Figure \ref{fig:U-expanded-terms}, there is a corresponding candidate expansion type form in Figure \ref{fig:U-candidate-terms}. The corresponding candidate expansion type validation rules are derived from Rules (\ref{rules:istypeU}). For example, the validation rules for candidate expansion types of variable, partial function and polymorphic form are derived from Rules (\ref{rule:istypeU-var}) through (\ref{rule:istypeU-all}):
\begin{subequations}\label{rules:cvalidT-U}
\begin{equation}\label{rule:cvalidT-U-tvar}
\inferrule{ }{
  \cvalidT{\Delta, \Dhyp{t}}{\tscenev}{t}{t}
}
\end{equation}
\begin{equation}\label{rule:cvalidT-U-parr}
  \inferrule{
    \cvalidT{\Delta}{\tscenev}{\ctau_1}{\tau_1}\\
    \cvalidT{\Delta}{\tscenev}{\ctau_2}{\tau_2}
  }{
    \cvalidT{\Delta}{\tscenev}{\aceparr{\ctau_1}{\ctau_2}}{\aparr{\tau_1}{\tau_2}}
  }
\end{equation}
\begin{equation}\label{rule:cvalidT-U-all}
  \inferrule {
    \cvalidT{\Delta, \Dhyp{t}}{\tscenev}{\ctau}{\tau}
  }{
    \cvalidT{\Delta}{\tscenev}{\aceall{t}{\ctau}}{\aall{t}{\tau}}
  }
\end{equation}
Generically, for each rule $\rulemv$ in Rules (\ref{rules:istypeU}), 
\begin{mathpar}
\inferrule{J_1\\\cdots\\J_k}{J}
\end{mathpar}
we derive the candidate expansion type validation rule $\VTypof{\rulemv}$ as follows:
\begin{mathpar}
\inferrule{
  \VTypof{J_1}\\
  \cdots\\
  \VTypof{J_k}
}{
  \VTypof{J}
}
\end{mathpar}
where 
\[\begin{split}
\VTypof{\istypeU{\Delta}{\tau}} & = \cvalidT{\Delta}{\tscenev}{\VTypof{\tau}}{\tau}\\
\VTypof{\{J_i\}_{i \in \labelset}} & = \{\VTypof{J_i}\}_{i \in \labelset}
\end{split}\]
and where $\VTypof{\tau}$ is defined as follows:
\begin{itemize}
\item When $\tau$ is of indefinite form, $\VTypof{\tau}$ is a uniquely corresponding metavariable of indefinite form ranging over candidate expansion types.
\item When $\tau$ is of definite form, $\VTypof{\tau}$ is defined as follows:
\[\begin{split}
\VTypof{t} & = t\\
\VTypof{\aparr{\tau_1}{\tau_2}} & = \aceparr{\VTypof{\tau_1}}{\VTypof{\tau_2}}\\
\VTypof{\aall{t}{\tau}} & = \aceall{t}{\VTypof{\tau}} 
\end{split}\]
and so on for the remaining type forms.
\end{itemize}

Notice that in Rule (\ref{rule:cvalidT-U-tvar}), only hypotheses in the candidate expansion's type formation context are available to the candidate expansion. In particular, the hypotheses in the application site type formation context, tracked by the type splicing scene $\tscenev$, are not available to the candidate expansion. Indeed, $\tscenev$ is not inspected by any of the rules above. This achieves \emph{context-independent expansion} as described in Sec. \ref{sec:splicing-and-hygiene} for type expansions -- ueTSMs cannot impose ``hidden constraints'' on the application site type formation context.

\item The only form of candidate expansion type that does not correspond to a form of type is $\acesplicedt{m}{n}$, which is a \emph{reference to a spliced type}, i.e. it indicates that a type should be parsed out from the literal body, $b$, beginning at position $m$ and ending at position $n$. Rule (\ref{rule:cvalidT-U-splicedt}) governs this form.
\setcounter{equation}{6}
\begin{equation}\label{rule:cvalidT-U-splicedt}
  \inferrule{
    \parseTyp{\bsubseq{b}{m}{n}}{\tau}\\
    \istypeU{\Delta_S}{\tau}\\
    \Delta \cap \Delta_S = \emptyset
  }{
    \cvalidT{\Delta}{\tsceneU{\Delta_S}{b}}{\acesplicedt{m}{n}}{\tau}
  }
\end{equation}
The first premise extracts the indicated subsequence of $b$ (using the metafunction $\bsubseq{b}{m}{n}$) and parses it (using the metafunction $\mathsf{parseTyp}(b)$) to produce the spliced type, $\tau$. Again, it would be tedious and uninteresting to specify these metafunctions explicitly, so instead, we only specify certain necessary conditions in the next subsection. %We assume sensible definitions of these metafunctions.

The second premise of Rule (\ref{rule:cvalidT-U-splicedt}) checks that $\tau$ is well-formed under the application site type formation context, $\Delta_S$. Notice that the hypotheses in the candidate expansion's type formation context, $\Delta$, are not made available to $\tau$. %This enforces \emph{expansion independent splicing} as described in Sec. \ref{sec:splicing-and-hygiene} for type variables that appear in candidate expansion types. 

The third premise of Rule (\ref{rule:cvalidT-U-splicedt}) requires that the candidate expansion's type formation context, $\Delta$, be disjoint from the application site type formation context, $\Delta_S$. This premise can always be discharged by alpha-varying the candidate expansion that the reference to the spliced type appears within. 

This achieves \emph{expansion-independent splicing} as described in Sec. \ref{sec:splicing-and-hygiene} for type variables -- the TSM provider can choose type variable names freely within a candidate expansion, because the language keeps them from shadowing type variables at the application site (by alpha-varying the candidate expansion as needed).%Such a change in bound variable names is possible again because variables bound by the ueTSM provider in a candidate expansion cannot ``leak into'' spliced terms because the hypotheses in $\Delta$ are not made available to the spliced type, $\tau$. 
\end{subequations}
\end{itemize}
% \begin{subequations}\label{rules:cvalidT-U}
% \begin{equation}\label{rule:cvalidT-U-tvar}
% \inferrule{ }{
%   \cvalidT{\Delta, \Dhyp{t}}{\tscenev}{t}{t}
% }
% \end{equation}
% \begin{equation}\label{rule:cvalidT-U-parr}
%   \inferrule{
%     \cvalidT{\Delta}{\tscenev}{\ctau_1}{\tau_1}\\
%     \cvalidT{\Delta}{\tscenev}{\ctau_2}{\tau_2}
%   }{
%     \cvalidT{\Delta}{\tscenev}{\aceparr{\ctau_1}{\ctau_2}}{\aparr{\tau_1}{\tau_2}}
%   }
% \end{equation}
% \begin{equation}\label{rule:cvalidT-U-all}
%   \inferrule {
%     \cvalidT{\Delta, \Dhyp{t}}{\tscenev}{\ctau}{\tau}
%   }{
%     \cvalidT{\Delta}{\tscenev}{\aceall{t}{\ctau}}{\aall{t}{\tau}}
%   }
% \end{equation}
% \begin{equation}\label{rule:cvalidT-U-rec}
%   \inferrule{
%     \cvalidT{\Delta, \Dhyp{t}}{\tscenev}{\ctau}{\tau}
%   }{
%     \cvalidT{\Delta}{\tscenev}{\acerec{t}{\ctau}}{\arec{t}{\tau}}
%   }
% \end{equation}
% \begin{equation}\label{rule:cvalidT-U-prod}
%   \inferrule{
%     \{\cvalidT{\Delta}{\tscenev}{\ctau_i}{\tau_i}\}_{i \in \labelset}
%   }{
%     \cvalidT{\Delta}{\tscenev}{\aceprod{\labelset}{\mapschema{\ctau}{i}{\labelset}}}{\aprod{\labelset}{\mapschema{\tau}{i}{\labelset}}}
%   }
% \end{equation}
% \begin{equation}\label{rule:cvalidT-U-sum}
%   \inferrule{
%     \{\cvalidT{\Delta}{\tscenev}{\ctau_i}{\tau_i}\}_{i \in \labelset}
%   }{
%     \cvalidT{\Delta}{\tscenev}{\acesum{\labelset}{\mapschema{\ctau}{i}{\labelset}}}{\asum{\labelset}{\mapschema{\tau}{i}{\labelset}}}
%   }
% \end{equation}
% \end{subequations}
%Each form of type, $\tau$, corresponds to a form of candidate expansion type, $\ctau$ (compare Figures \ref{fig:U-expanded-terms} and \ref{fig:U-candidate-terms}). For each type formation rule in Rules (\ref{rules:istypeU}), there is a corresponding candidate expansion type validation rule -- Rules (\ref{rule:cvalidT-U-tvar}) to (\ref{rule:cvalidT-U-sum}) -- where the candidate expansion type and the final expansion correspond. The premises also correspond. 



\subsubsection{Candidate Expansion Expression Validation}
The \emph{candidate expansion expression validation judgement}, $\cvalidE{\Delta}{\Gamma}{\escenev}{\ce}{e}{\tau}$, is defined mutually inductively with Rules (\ref{rules:expandsU}) as follows.% This is necessary because a typed expansion judgement appears as a premise in Rule (\ref{rule:cvalidE-U-splicede}) below, and a candidate expansion expression validation judgement appears as a premise in Rule (\ref{rule:expandsU-tsmap}) above.
\begin{itemize}
\item For each expression form specified in Figure \ref{fig:U-expanded-terms}, there is a corresponding candidate expansion expression form in Figure \ref{fig:U-candidate-terms}. The corresponding candidate expansion expression validation rules are derived from Rules (\ref{rules:hastypeU}). For example, the validation rules for candidate expansion expressions of variable, function and function application form are derived from Rules (\ref{rule:hastypeU-var}) through (\ref{rule:hastypeU-ap}):
\begin{subequations}\label{rules:cvalidE-U}
\begin{equation}\label{rule:cvalidE-U-var}
\inferrule{ }{
  \cvalidE{\Delta}{\Gamma, \Ghyp{x}{\tau}}{\escenev}{x}{x}{\tau}
}
\end{equation}
\begin{equation}\label{rule:cvalidE-U-lam}
\inferrule{
  \cvalidT{\Delta}{\tsfrom{\escenev}}{\ctau}{\tau}\\
  \cvalidE{\Delta}{\Gamma, \Ghyp{x}{\tau}}{\escenev}{\ce}{e}{\tau'}
}{
  \cvalidE{\Delta}{\Gamma}{\escenev}{\acelam{\ctau}{x}{\ce}}{\aelam{\tau}{x}{e}}{\aparr{\tau}{\tau'}}
}
\end{equation}
\begin{equation}\label{rule:cvalidE-U-ap}
  \inferrule{
    \cvalidE{\Delta}{\Gamma}{\escenev}{\ce_1}{e_1}{\aparr{\tau}{\tau'}}\\
    \cvalidE{\Delta}{\Gamma}{\escenev}{\ce_2}{e_2}{\tau}
  }{
    \cvalidE{\Delta}{\Gamma}{\escenev}{\aceap{\ce_1}{\ce_2}}{\aeap{e_1}{e_2}}{\tau'}
  }
\end{equation}
Generically, for each rule $\rulemv$ in Rules (\ref{rules:hastypeU}),
\begin{mathpar}
  \inferrule{
    J_1\\
    \cdots\\
    J_k
  }{
    J
  }
\end{mathpar}
we derive the candidate expansion expression validation rule $\VExpof{\rulemv}$ as follows:
\begin{mathpar}
  \inferrule{
    \VExpof{J_1}\\
    \cdots\\
    \VExpof{J_k}
  }{
    \VExpof{J}
  }
\end{mathpar}
where 
\[\begin{split}
  \VExpof{\istypeU{\Delta}{\tau}} & = \cvalidT{\Delta}{\tsfrom{\escenev}}{\VTypof{\tau}}{\tau'}\\
  \VExpof{\hastypeU{\Delta}{\Gamma}{e}{\tau}} & = \cvalidE{\Delta}{\Gamma}{\escenev}{\VExpof{e}}{e}{\tau}\\
  \VExpof{\{J_i\}_{i \in \labelset}} & = \{\VExpof{J_i}\}_{i \in \labelset}
\end{split}\]
and where $\VExpof{e}$ is defined as follows:
\begin{itemize}
  \item When $e$ is of indefinite form, $\VExpof{e}$ is a uniquely corresponding metavariable of indefinite form ranging over candidate expansion expressions.
  \item When $e$ is of definite form, $\VExpof{e}$ is defined as follows:
  \[\begin{split}
    \VExpof{x} & = x\\
    \VExpof{\aelam{\tau}{x}{e}} & = \acelam{\VTypof{\tau}}{x}{\VExpof{e}}\\
    \VExpof{\aeap{e_1}{e_2}} & = \aceap{\VExpof{e_1}}{\VExpof{e_2}}
  \end{split}\]
  and so on for the remaining expression forms.
\end{itemize}

Notice that in Rule (\ref{rule:cvalidE-U-var}), only hypotheses in the candidate expansion's typing context are available to the candidate expansion. Hypotheses in the application site typing context, tracked by the expression splicing scene $\escenev$ are not available to the candidate expansion. Indeed, $\escenev$ is not inspected by any of the rules above. This achieves \emph{context-independent expansion} as described in Sec. \ref{sec:splicing-and-hygiene} for expression expansions -- ueTSMs cannot impose ``hidden constraints'' on the application site typing context.

\item The only candidate expansion expression form that does not correspond to an expanded expression form is $\acesplicede{m}{n}$. Candidate expansion expressions of this form are \emph{references to a spliced unexpanded expression}, i.e. they indicate that an unexpanded expression should be parsed out from the literal body, $b$, beginning at position $m$ and ending at position $n$. Rule (\ref{rule:cvalidE-U-splicede}) governs this form. 
\setcounter{equation}{11}
\begin{equation}\label{rule:cvalidE-U-splicede}
\inferrule{
  \parseUExp{\bsubseq{b}{m}{n}}{\ue}\\\\
  \expandsU{\Delta_S}{\Gamma_S}{\Sigma_S}{\ue}{e}{\tau}\\
  \Delta \cap \Delta_S = \emptyset\\
  \domof{\Gamma} \cap \domof{\Gamma_S} = \emptyset
}{
  \cvalidE{\Delta}{\Gamma}{\esceneU{\Delta_S}{\Gamma_S}{\Sigma_S}{b}}{\acesplicede{m}{n}}{e}{\tau}
}
\end{equation}
The first premise extracts the indicated subsequence of $b$ (using the metafunction $\bsubseq{b}{m}{n}$) and parses it (using the metafunction $\mathsf{parseUExp}(b)$) to produce the spliced unexpanded expression, $\ue$. Again, we do not explicitly specify these metafunctions for simplicity. Instead, we will give certain necessary conditions in the next subsection.

The second premise of Rule (\ref{rule:cvalidE-U-splicede}) types and expands the spliced unexpanded expression $\ue$ assuming the application site contexts, $\Delta_S$ and $\Gamma_S$, and ueTSM environment, $\Sigma_S$. 

The third premise requires that the candidate expansion's type formation context, $\Delta$, be disjoint from the application site type formation context, $\Delta_S$. Similarly, the fourth premise requires that the candidate expansion's typing context, $\Gamma$, be disjoint from the application site typing context, $\Gamma_S$. These two premises can always be discharged by alpha-varying the candidate expansion expression that the reference to the spliced unexpanded expression appears within. 

This achieves \emph{expansion-independent splicing} as described in Sec. \ref{sec:splicing-and-hygiene} -- the TSM provider can choose variable and type variable names freely within a candidate expansion, because the language keeps them from shadowing those at the application site (by alpha-varying the candidate expansion as needed).
\end{subequations}
\end{itemize}
% \begin{subequations}\label{rules:cvalidE-U}
% \begin{equation}\label{rule:cvalidE-U-var}
% \inferrule{ }{
%   \cvalidE{\Delta}{\Gamma, \Ghyp{x}{\tau}}{\escenev}{x}{x}{\tau}
% }
% \end{equation}
% \begin{equation}\label{rule:cvalidE-U-lam}
% \inferrule{
%   \cvalidT{\Delta}{\tsfrom{\escenev}}{\ctau}{\tau}\\
%   \cvalidE{\Delta}{\Gamma, \Ghyp{x}{\tau}}{\escenev}{\ce}{e}{\tau'}
% }{
%   \cvalidE{\Delta}{\Gamma}{\escenev}{\acelam{\ctau}{x}{\ce}}{\aelam{\tau}{x}{e}}{\aparr{\tau}{\tau'}}
% }
% \end{equation}
% \begin{equation}\label{rule:cvalidE-U-ap}
%   \inferrule{
%     \cvalidE{\Delta}{\Gamma}{\escenev}{\ce_1}{e_1}{\aparr{\tau}{\tau'}}\\
%     \cvalidE{\Delta}{\Gamma}{\escenev}{\ce_2}{e_2}{\tau}
%   }{
%     \cvalidE{\Delta}{\Gamma}{\escenev}{\aceap{\ce_1}{\ce_2}}{\aeap{e_1}{e_2}}{\tau'}
%   }
% \end{equation}
% \begin{equation}\label{rule:cvalidE-U-tlam}
%   \inferrule{
%     \cvalidE{\Delta, \Dhyp{t}}{\Gamma}{\escenev}{\ce}{e}{\tau}
%   }{
%     \cvalidEX{\acetlam{t}{\ce}}{\aetlam{t}{e}}{\aall{t}{\tau}}
%   }
% \end{equation}
% \begin{equation}\label{rule:cvalidE-U-tap}
%   \inferrule{
%     \cvalidEX{\ce}{e}{\aall{t}{\tau}}\\
%     \cvalidT{\Delta}{\tsfrom{\escenev}}{\ctau'}{\tau'}
%   }{
%     \cvalidEX{\acetap{\ce}{\ctau'}}{\aetap{e}{\tau'}}{[\tau'/t]\tau}
%   }
% \end{equation}
% \begin{equation}\label{rule:cvalidE-U-fold}
%   \inferrule{
%     \cvalidT{\Delta, \Dhyp{t}}{\escenev}{\ctau}{\tau}\\
%     \cvalidEX{\ce}{e}{[\arec{t}{\tau}/t]\tau}
%   }{
%     \cvalidEX{\acefold{t}{\ctau}{\ce}}{\aefold{t}{\tau}{e}}{\arec{t}{\tau}}
%   }
% \end{equation}
% \begin{equation}\label{rule:cvalidE-U-unfold}
%   \inferrule{
%     \cvalidEX{\ce}{e}{\arec{t}{\tau}}
%   }{
%     \cvalidEX{\aceunfold{\ce}}{\aeunfold{e}}{[\arec{t}{\tau}/t]\tau}
%   }
% \end{equation}
% \begin{equation}\label{rule:cvalidE-U-tpl}
%   \inferrule{
%     \{\cvalidEX{\ce_i}{e_i}{\tau_i}\}_{i \in \labelset}
%   }{
%     \cvalidEX{\acetpl{\labelset}{\mapschema{\ce}{i}{\labelset}}}{\aetpl{\labelset}{\mapschema{e}{i}{\labelset}}}{\aprod{\labelset}{\mapschema{\tau}{i}{\labelset}}}
%   }
% \end{equation}
% \begin{equation}\label{rule:cvalidE-U-pr}
%   \inferrule{
%     \cvalidEX{\ce}{e}{\aprod{\labelset, \ell}{\mapschema{\tau}{i}{\labelset}; \mapitem{\ell}{\tau}}}
%   }{
%     \cvalidEX{\acepr{\ell}{\ce}}{\aepr{\ell}{e}}{\tau}
%   }
% \end{equation}
% \begin{equation}\label{rule:cvalidE-U-in}
%   \inferrule{
%     \{\cvalidT{\Delta}{\tsfrom{\escenev}}{\ctau_i}{\tau_i}\}_{i \in \labelset}\\
%     \cvalidT{\Delta}{\tsfrom{\escenev}}{\ctau}{\tau}\\
%     \cvalidEX{\ce}{e}{\tau}
%   }{
%     \left\{\shortstack{$\Delta~\Gamma \vdash_\Sigma \acein{\labelset, \ell}{\ell}{\mapschema{\ctau}{i}{\labelset}; \mapitem{\ell}{\ctau}}{\ce}$\\$\leadsto$\\$\aein{\labelset, \ell}{\ell}{\mapschema{\tau}{i}{\labelset}; \mapitem{\ell}{\tau}}{e} : \asum{\labelset, \ell}{\mapschema{\tau}{i}{\labelset}; \mapitem{\ell}{\tau}}$\vspace{-1.2em}}\right\}
%   }
% \end{equation}
% \begin{equation}\label{rule:cvalidE-U-case}
%   \inferrule{
%     \cvalidEX{\ce}{e}{\asum{\labelset}{\mapschema{\tau}{i}{\labelset}}}\\
%     \{\cvalidE{\Delta}{\Gamma, \Ghyp{x_i}{\tau_i}}{\escenev}{\ue_i}{e_i}{\tau}\}_{i \in \labelset}
%   }{
%     \cvalidEX{\acecase{\labelset}{\ce}{\mapschemab{x}{\ce}{i}{\labelset}}}{\aecase{\labelset}{e}{\mapschemab{x}{e}{i}{\labelset}}}{\tau}
%   }
% \end{equation}
% \begin{equation}\label{rule:cvalidE-U-splicede}
% \inferrule{
%   \parseUExp{\bsubseq{b}{m}{n}}{\ue}\\\\
%   \Delta \cap \Delta_S = \emptyset\\
%   \domof{\Gamma} \cap \domof{\Gamma_S} = \emptyset\\
%   \expandsU{\Delta_S}{\Gamma_S}{\Sigma_S}{\ue}{e}{\tau}
% }{
%   \cvalidE{\Delta}{\Gamma}{\esceneU{\Delta_S}{\Gamma_S}{\Sigma_S}{b}}{\acesplicede{m}{n}}{e}{\tau}
% }
% \end{equation}
% \end{subequations}

% Each form of expanded expression, $e$, corresponds to a form of candidate expansion expression, $\ce$ (compare Figure \ref{fig:U-expanded-terms} and Figure \ref{fig:U-candidate-terms}). For each typing rule in Rules \ref{rules:hastypeU}, there is a corresponding candidate expansion expression validation rule -- Rules (\ref{rule:cvalidE-U-var}) to (\ref{rule:cvalidE-U-case}) -- where the candidate expansion expression and expanded expression correspond. The premises also correspond.


%Candidate expansions cannot themselves define or apply TSMs. This simplifies our metatheory, though it can be inconvenient at times for TSM providers. We discuss adding the ability to use TSMs within candidate expansions in Sec. \ref{sec:tsms-in-expansions}.


\subsection{Metatheory}
For the judgements we have specified to form a sensible language, we must have that typed expansion of unexpanded expressions be consistent with typing of expanded expressions. Formally, this can be expressed as the following theorem.

\begin{theorem}[Typed Expansion]\label{thm:typed-expansion-U}
If $\expandsU{\Delta}{\Gamma}{\Sigma}{\ue}{e}{\tau}$ and $\uetsmenv{\Delta}{\Sigma}$ then $\hastypeU{\Delta}{\Gamma}{e}{\tau}$. 
\end{theorem}
\begin{proof}
By rule induction over Rules (\ref{rules:expandsU}).
\begin{byCases}
\item[\text{(\ref{rule:expandsU-var})}] We have
\begin{pfsteps}
  \item \ue=x \BY{assumption}
  \item e=x \BY{assumption}
  \item \Gamma=\Gamma', \Ghyp{x}{\tau} \BY{assumption}
  \item \hastypeU{\Delta}{\Gamma', \Ghyp{x}{\tau}}{x}{\tau} \BY{Rule (\ref{rule:hastypeU-var})}
\end{pfsteps}
\resetpfcounter

\item[\text{(\ref{rule:expandsU-lam})}] We have 
\begin{pfsteps}
  \item \ue=\aulam{\tau_1}{x}{\ue'} \BY{assumption}
  \item e=\aelam{\tau_1}{x}{e'} \BY{assumption}
  \item \tau=\aparr{\tau_1}{\tau_2} \BY{assumption}
  \item \istypeU{\Delta}{\tau_1} \BY{assumption} \pflabel{istype}
  \item \expandsU{\Delta}{\Gamma, \Ghyp{x}{\tau_1}}{\Sigma}{\ue'}{e'}{\tau_2} \BY{assumption} \pflabel{expandsU}
  \item \uetsmenv{\Delta}{\Sigma} \BY{assumption} \pflabel{uetsmenv}
  \item \hastypeU{\Delta}{\Gamma, \Ghyp{x}{\tau_1}}{e'}{\tau_2} \BY{IH on \pfref{expandsU} and \pfref{uetsmenv}} \pflabel{hastypeU}
  \item \hastypeU{\Delta}{\Gamma}{\aelam{\tau_1}{x}{e'}}{\aparr{\tau_1}{\tau_2}} \BY{Rule (\ref{rule:hastypeU-lam}) on \pfref{istype} and \pfref{hastypeU}}
\end{pfsteps}
\resetpfcounter

\item[\text{(\ref{rule:expandsU-ap})}] We have
\begin{pfsteps}
  \item \ue=\auap{\ue_1}{\ue_2} \BY{assumption}
  \item e=\aeap{e_1}{e_2} \BY{assumption}
  \item \expandsU{\Delta}{\Gamma}{\Sigma}{\ue_1}{e_1}{\aparr{\tau_1}{\tau}} \BY{assumption}\pflabel{expandsU1}
  \item \expandsU{\Delta}{\Gamma}{\Sigma}{\ue_2}{e_2}{\tau_1} \BY{assumption}\pflabel{expandsU2}
  \item \uetsmenv{\Delta}{\Sigma} \BY{assumption} \pflabel{uetsmenv}
  \item \hastypeU{\Delta}{\Gamma}{e_1}{\aparr{\tau_1}{\tau}} \BY{IH on \pfref{expandsU1} and \pfref{uetsmenv}}\pflabel{hastypeU1}
  \item \hastypeU{\Delta}{\Gamma}{e_2}{\tau_1} \BY{IH on \pfref{expandsU2} and \pfref{uetsmenv}}\pflabel{hastypeU2}
  \item \hastypeU{\Delta}{\Gamma}{\aeap{e_1}{e_2}}{\tau} \BY{Rule (\ref{rule:hastypeU-ap}) on \pfref{hastypeU1} and \pfref{hastypeU2}}
\end{pfsteps}
\resetpfcounter

\item[\Uof{\text{\ref{rule:hastypeU-tlam}}}~\text{through}~\Uof{\text{\ref{rule:hastypeU-case}}}] These cases follow analagously, i.e. we apply the IH to all typed expansion premises and then apply the corresponding typing rule.
\\

\item[\text{(\ref{rule:expandsU-syntax})}] We have 
\begin{pfsteps}
  \item \ue=\audefuetsm{\tau'}{\ueparse}{a}{\ue'} \BY{assumption}
  \item \istypeU{\Delta}{\tau'} \BY{assumption}\pflabel{istype}
  \item \expandsU{\emptyset}{\emptyset}{\emptyset}{\ueparse}{\eparse}{\aparr{\tBody}{\tParseResultExp}} \BY{assumption}\pflabel{ueparse}
  \item \expandsU{\Delta}{\Gamma}{\Sigma, \xuetsmbnd{a}{\tau'}{\eparse}}{\ue'}{e}{\tau} \BY{assumption}\pflabel{expandsU}
  \item \uetsmenv{\Delta}{\Sigma} \BY{assumption}\pflabel{uetsmenv1}
  \item \uetsmenv{\emptyset}{\emptyset} \BY{Rule (\ref{rule:uetsmenv-empty})}\pflabel{uetsmenv2}
  \item \hastypeU{\emptyset}{\emptyset}{\eparse}{\aparr{\tBody}{\tParseResultExp}} \BY{IH on \pfref{ueparse} and \pfref{uetsmenv2}}\pflabel{eparse2}
  \item \uetsmenv{\Delta}{\Sigma, \xuetsmbnd{a}{\tau'}{\eparse}} \BY{Rule (\ref{rule:uetsmenv-ext}) on \pfref{uetsmenv1}, \pfref{istype} and \pfref{eparse2}}\pflabel{uetsmenv3}
  \item \hastypeU{\Delta}{\Gamma}{e}{\tau} \BY{IH on \pfref{expandsU} and \pfref{uetsmenv3}}
\end{pfsteps}
\resetpfcounter 

\item[\text{(\ref{rule:expandsU-tsmap})}] We have 
\begin{pfsteps}
  \item \ue=\autsmap{b}{a} \BY{assumption}
  \item \Sigma=\Sigma', \xuetsmbnd{a}{\tau}{\eparse} \BY{assumption}
  \item \encodeBody{b}{\ebody} \BY{assumption}
  \item \evalU{\eparse(\ebody)}{\inj{\lbltxt{Success}}{\ecand}} \BY{assumption}
  \item \decodeCondE{\ecand}{\ce} \BY{assumption}
  \item \cvalidE{\emptyset}{\emptyset}{\esceneU{\Delta}{\Gamma}{\Sigma}{b}}{\ce}{e}{\tau} \BY{assumption}\pflabel{cvalidE}
  \item \uetsmenv{\Delta}{\Sigma} \BY{assumption} \pflabel{uetsmenv}
  \item \hastypeU{\Delta}{\Gamma}{e}{\tau} \BY{Theorem \ref{thm:candidate-expansion-validation-U} on \pfref{cvalidE} and \pfref{uetsmenv}}
\end{pfsteps}
\resetpfcounter
\end{byCases}
\end{proof}
We need to define the following theorem about candidate expansion expression validation mutually with Theorem \ref{thm:typed-expansion-U}. 
\begin{theorem}[Candidate Expansion Expression Validation]\label{thm:candidate-expansion-validation-U}
If $\cvalidE{\Delta}{\Gamma}{\esceneU{\Delta_S}{\Gamma_S}{\Sigma_S}{b}}{\ce}{e}{\tau}$ and $\uetsmenv{\Delta_S}{\Sigma_S}$ then $\hastypeU{\Dcons{\Delta}{\Delta_S}}{\Gcons{\Gamma}{\Gamma_S}}{e}{\tau}$.
\end{theorem}
\begin{proof} By rule induction over Rules (\ref{rules:cvalidE-U}).
\begin{byCases}
\item[\text{(\ref{rule:cvalidE-U-var})}] We have
\begin{pfsteps*}
  \item $\ce=x$ \BY{assumption}
  \item $e=x$ \BY{assumption}
  \item $\Gamma=\Gamma', \Ghyp{x}{\tau}$ \BY{assumption}
  \item $\hastypeU{\Dcons{\Delta}{\Delta_S}}{\Gamma', \Ghyp{x}{\tau}}{x}{\tau}$ \BY{Rule (\ref{rule:hastypeU-var})} \pflabel{hastypeU}
  \item $\hastypeU{\Dcons{\Delta}{\Delta_S}}{\Gcons{\Gamma', \Ghyp{x}{\tau}}{\Gamma_S}}{x}{\tau}$ \BY{Lemma \ref{lemma:weakening-U} over $\Gamma_S$ to \pfref{hastypeU}}
\end{pfsteps*}
\resetpfcounter

\item[\text{(\ref{rule:cvalidE-U-lam})}] We have
\begin{pfsteps*}
  \item $\ce=\acelam{\ctau_1}{x}{\ce'}$ \BY{assumption}
  \item $e=\aelam{\tau_1}{x}{e'}$ \BY{assumption}
  \item $\tau=\aparr{\tau_1}{\tau_2}$ \BY{assumption}
  \item $\cvalidT{\Delta}{\esceneU{\Delta_S}{\Gamma_S}{\Sigma_S}{b}}{\ctau_1}{\tau_1}$ \BY{assumption} \pflabel{cvalidT}
  \item $\cvalidE{\Delta}{\Gamma, \Ghyp{x}{\tau_1}}{\esceneU{\Delta_S}{\Gamma_S}{\Sigma_S}{b}}{\ce'}{e'}{\tau_2}$ \BY{assumption} \pflabel{cvalidE}
  \item $\uetsmenv{\Delta_S}{\Sigma_S}$ \BY{assumption} \pflabel{uetsmenv}
  \item $\istypeU{\Dcons{\Delta}{\Delta_S}}{\tau_1}$ \BY{Lemma \ref{lemma:candidate-expansion-type-validation} on \pfref{cvalidT}} \pflabel{istype}
  \item $\hastypeU{\Dcons{\Delta}{\Delta_S}}{\Gcons{\Gamma, \Ghyp{x}{\tau_1}}{\Gamma_S}}{e'}{\tau_2}$ \BY{IH on \pfref{cvalidE} and \pfref{uetsmenv}} \pflabel{hastype1}
  \item $\hastypeU{\Dcons{\Delta}{\Delta_S}}{\Gcons{\Gamma}{\Gamma_S}, \Ghyp{x}{\tau_1}}{e'}{\tau_2}$ \BY{exchange over $\Gamma_S$ on \pfref{hastype1}} \pflabel{hastype2}
  \item $\hastypeU{\Dcons{\Delta}{\Delta_S}}{\Gcons{\Gamma}{\Gamma_S}}{\aelam{\tau_1}{x}{e'}}{\aparr{\tau_1}{\tau_2}}$ \BY{Rule (\ref{rule:hastypeU-lam}) on \pfref{istype} and \pfref{hastype2}}
\end{pfsteps*}
\resetpfcounter

\item[\text{(\ref{rule:cvalidE-U-ap})}] We have
\begin{pfsteps*}
  \item $\ce=\aceap{\ce_1}{\ce_2}$ \BY{assumption}
  \item $e=\aeap{e_1}{e_2}$ \BY{assumption}
  \item $\cvalidE{\Delta}{\Gamma}{\esceneU{\Delta_S}{\Gamma_S}{\Sigma_S}{b}}{\ce_1}{e_1}{\aparr{\tau_1}{\tau}}$ \BY{assumption} \pflabel{cvalidE1}
  \item $\cvalidE{\Delta}{\Gamma}{\esceneU{\Delta_S}{\Gamma_S}{\Sigma_S}{b}}{\ce_2}{e_2}{\tau_1}$ \BY{assumption} \pflabel{cvalidE2}
  \item $\uetsmenv{\Delta_S}{\Sigma_S}$ \BY{assumption} \pflabel{uetsmenv}
  \item $\hastypeU{\Dcons{\Delta}{\Delta_S}}{\Gcons{\Gamma}{\Gamma_S}}{e_1}{\aparr{\tau_1}{\tau}}$ \BY{IH on \pfref{cvalidE1} and \pfref{uetsmenv}} \pflabel{hastypeU1}
  \item $\hastypeU{\Dcons{\Delta}{\Delta_S}}{\Gcons{\Gamma}{\Gamma_S}}{e_2}{\tau_1}$ \BY{IH on \pfref{cvalidE2} and \pfref{uetsmenv}} \pflabel{hastypeU2}
  \item $\hastypeU{\Dcons{\Delta}{\Delta_S}}{\Gcons{\Gamma}{\Gamma_S}}{\aeap{e_1}{e_2}}{\tau}$ \BY{Rule (\ref{rule:hastypeU-ap}) on \pfref{hastypeU1} and \pfref{hastypeU2}}
\end{pfsteps*}
\resetpfcounter

\item[\VExpof{\text{\ref{rule:hastypeU-tlam}}}~\text{through}~\VExpof{\text{\ref{rule:hastypeU-case}}}] These cases follow analagously, i.e. we apply the IH to all candidate expansion expression validation premises, Lemma \ref{lemma:candidate-expansion-type-validation} to all candidate expansion type validation premises, weakening and exchange as needed, and then apply the corresponding typing rule.
\\

\item[\text{(\ref{rule:cvalidE-U-splicede})}] We have
\begin{pfsteps*}
  \item $\ce=\acesplicede{m}{n}$ \BY{assumption}
  \item $\parseUExp{\bsubseq{b}{m}{n}}{\ue}$ \BY{assumption}
  \item $\expandsU{\Delta_S}{\Gamma_S}{\Sigma_S}{\ue}{e}{\tau}$ \BY{assumption} \pflabel{expands}
  \item $\uetsmenv{\Delta_S}{\Sigma_S}$ \BY{assumption} \pflabel{uetsmenv}
  \item $\hastypeU{\Delta_S}{\Gamma_S}{e}{\tau}$ \BY{Theorem \ref{thm:typed-expansion-U} on \pfref{expands} and \pfref{uetsmenv}} \pflabel{hastype}
  \item $\hastypeU{\Dcons{\Delta}{\Delta_S}}{\Gcons{\Gamma}{\Gamma_S}}{e}{\tau}$ \BY{Lemma \ref{lemma:weakening-U} on \pfref{hastype}}
\end{pfsteps*}
\resetpfcounter
\end{byCases}
\end{proof}

The mutual induction used to prove Theorems \ref{thm:typed-expansion-U} and \ref{thm:candidate-expansion-validation-U} can be shown to be well-founded by showing that the following numeric metric on the judgements that we induct over is decreasing:
\begin{align*}
\sizeof{\expandsU{\Delta}{\Gamma}{\Sigma}{\ue}{e}{\tau}} & = \sizeof{\ue}\\
\sizeof{\cvalidE{\Delta}{\Gamma}{\esceneU{\Delta_S}{\Gamma_S}{\Sigma_S}{b}}{\ce}{e}{\tau}} & = \sizeof{b}
\end{align*}
Here, $\sizeof{b}$ is the length of $b$ and $\sizeof{\ue}$ is the sum of the lengths of the literal bodies in $\ue$, i.e. 
\begin{align*}
\sizeof{x} & = 0\\
\sizeof{\aulam{\tau}{x}{\ue}} &= \sizeof{\ue}\\
\sizeof{\auap{\ue_1}{\ue_2}} & = \sizeof{\ue_1} + \sizeof{\ue_2}\\
\sizeof{\autlam{t}{\ue}} & = \sizeof{\ue}\\
\sizeof{\autap{\ue}{\tau}} & = \sizeof{\ue}\\
\sizeof{\aufold{t}{\tau}{\ue}} & = \sizeof{\ue}\\
\sizeof{\auunfold{\ue}} & = \sizeof{\ue}
\end{align*}
\begin{align*}
\sizeof{\autpl{\labelset}{\mapschema{\ue}{i}{\labelset}}} & = \sum_{i \in \labelset} \sizeof{\ue_i}\\
\sizeof{\aupr{\ell}{\ue}} & = \sizeof{\ue}\\
\sizeof{\auin{\labelset}{\ell}{\mapschema{\tau}{i}{\labelset}}{\ue}} & = \sizeof{\ue}\\
\sizeof{\aucase{\labelset}{\ue}{\mapschemab{x}{\ue}{i}{\labelset}}} & = \sizeof{\ue} + \sum_{i \in \labelset} \sizeof{\ue_i}\\
\sizeof{\audefuetsm{\tau}{\ueparse}{a}{\ue}} & = \sizeof{\ue}\\
\sizeof{\autsmap{b}{a}} & = \sizeof{b}
\end{align*}

The only case in the proof of Theorem \ref{thm:typed-expansion-U} that invokes Theorem \ref{thm:candidate-expansion-validation-U} is Case (\ref{rule:expandsU-tsmap}). There, we have that the metric remains stable: \[\sizeof{\cvalidE{\emptyset}{\emptyset}{\esceneU{\Delta}{\Gamma}{\Sigma}{b}}{\ce}{e}{\tau}} = \sizeof{\expandsU{\Delta}{\Gamma}{\Sigma}{\autsmap{b}{a}}{e}{\tau}}=\sizeof{b}\]

The only case in the proof of Theorem \ref{thm:candidate-expansion-validation-U} that invokes Theorem \ref{thm:typed-expansion-U} is Case (\ref{rule:cvalidE-U-splicede}). There, we have that $\parseUExp{\bsubseq{b}{m}{n}}{\ue}$ and Theorem \ref{thm:typed-expansion-U} is invoked on the judgement $\expandsU{\Delta_S}{\Gamma_S}{\Sigma_S}{\ue}{e}{\tau}$. Because the metric is stable when passing from Theorem \ref{thm:typed-expansion-U} to Theorem \ref{thm:candidate-expansion-validation-U}, we must have that it is strictly decreasing in the other direction:
\[\sizeof{\expandsU{\Delta_S}{\Gamma_S}{\Sigma_S}{\ue}{e}{\tau}} < \sizeof{\cvalidE{\Delta}{\Gamma}{\esceneU{\Delta_S}{\Gamma_S}{\Sigma_S}{b}}{\acesplicede{m}{n}}{e}{\tau}}\]
i.e. by the definitions above, 
\[\sizeof{\ue} < \sizeof{b}\]

This is established by appeal to the following two conditions. The first condition simply states that subsequences of $b$ are no longer than $b$.
\begin{condition}[Body Subsequences]\label{condition:body-subsequences} If $\bsubseq{b}{m}{n}=b'$ then $\sizeof{b'} \leq \sizeof{b}$. \end{condition}
The second condition states that an unexpanded expression constructed by parsing a literal body $b$ is strictly smaller, as measured by the metric defined above, than the length of $b$ (because some characters must necessarily be used to invoke a TSM on and delimit each literal body.)
\begin{condition}[Body Parsing]\label{condition:body-parsing} If $\parseUExp{b}{\ue}$ then $\sizeof{\ue} < \sizeof{b}$.\end{condition}

Combining Conditions \ref{condition:body-subsequences} and \ref{condition:body-parsing}, we have that $\sizeof{\ue} < \sizeof{b}$ as needed.

\qed

Finally, the proof of Theorem \ref{thm:candidate-expansion-validation-U} invokes the following lemma about candidate expansion type validation.
\begin{lemma}[Candidate Expansion Type Validation]\label{lemma:candidate-expansion-type-validation}
If $\cvalidT{\Delta}{\tsceneU{\Delta_S}{b}}{\ctau}{\tau}$ then $\istypeU{\Dcons{\Delta}{\Delta_S}}{\tau}$.
\end{lemma}
\begin{proof} By rule induction over Rules (\ref{rules:cvalidT-U}).
\begin{byCases}
\item[\text{(\ref{rule:cvalidT-U-tvar})}] We have 
\begin{pfsteps*}
   \item $\Delta=\Delta', \Dhyp{t}$ \BY{assumption}
   \item $\ctau=t$ \BY{assumption}
   \item $\tau=t$ \BY{assumption}
   \item $\istypeU{\Delta', \Dhyp{t}}{t}$ \BY{Rule (\ref{rule:istypeU-var})} \pflabel{istype}
   \item $\istypeU{\Dcons{\Delta', \Dhyp{t}}{\Delta_S}}{t}$ \BY{Lemma \ref{lemma:weakening-U} over $\Delta_S$ to \pfref{istype}.}
 \end{pfsteps*} 
\resetpfcounter

\item[\text{(\ref{rule:cvalidT-U-parr})}] We have
\begin{pfsteps*}
  \item $\ctau=\aceparr{\ctau_1}{\ctau_2}$ \BY{assumption}
  \item $\tau=\aparr{\tau_1}{\tau_2}$ \BY{assumption}
  \item $\cvalidT{\Delta}{\tsceneU{\Delta_S}{b}}{\ctau_1}{\tau_1}$ \BY{assumption} \pflabel{cvalid-ctau1}
  \item $\cvalidT{\Delta}{\tsceneU{\Delta_S}{b}}{\ctau_2}{\tau_2}$ \BY{assumption} \pflabel{cvalid-ctau2}
  \item $\istypeU{\Dcons{\Delta}{\Delta_S}}{\tau_1}$ \BY{IH on \pfref{cvalid-ctau1}} \pflabel{istype1}
  \item $\istypeU{\Dcons{\Delta}{\Delta_S}}{\tau_2}$ \BY{IH on \pfref{cvalid-ctau2}} \pflabel{istype2}
  \item $\istypeU{\Dcons{\Delta}{\Delta_S}}{\aparr{\tau_1}{\tau_2}}$ \BY{Rule (\ref{rule:istypeU-parr}) on \pfref{istype1} and \pfref{istype2}}
\end{pfsteps*}
\resetpfcounter

\item[\text{(\ref{rule:cvalidT-U-all})}] We have
\begin{pfsteps*}
  \item $\ctau=\aceall{t}{\ctau'}$ \BY{assumption}
  \item $\tau=\aall{t}{\tau'}$ \BY{assumption}
  \item $\cvalidT{\Delta, \Dhyp{t}}{\tsceneU{\Delta_S}{b}}{\ctau'}{\tau'}$ \BY{assumption} \label{cvalidT}
  \item $\istypeU{\Dcons{\Delta, \Dhyp{t}}{\Delta_S}}{\tau'}$ \BY{IH on \pfref{cvalidT}} \pflabel{istypeU1}
  \item $\istypeU{\Dcons{\Delta}{\Delta_S}, \Dhyp{t}}{\tau'}$ \BY{exchange over $\Delta_S$ on \pfref{istypeU1}} \pflabel{istypeU2}
  \item $\istypeU{\Dcons{\Delta}{\Delta_S}}{\aall{t}{\tau'}}$ \BY{Rule (\ref{rule:istypeU-all}) on \pfref{istypeU2}}
\end{pfsteps*}
\resetpfcounter

\item[\VTypof{\text{\ref{rule:istypeU-rec}}}~\text{through}~\VTypof{\text{\ref{rule:istypeU-sum}}}] These cases follow analagously, i.e. we apply the IH to all candidate expansion type validation premises, exchange as needed and then the corresponding type formation rule.
\\
% \item[\text{(\ref{rule:cvalidT-U-rec})}] We have
% \begin{pfsteps*}
%   \item $\ctau=\acerec{t}{\ctau'}$ \BY{assumption}
%   \item $\tau=\arec{t}{\tau'}$ \BY{assumption}
%   \item $\cvalidT{\Delta, \Dhyp{t}}{\tsceneU{\Delta_S}{b}}{\ctau'}{\tau'}$ \BY{assumption} \label{cvalidT}
%   \item $\istypeU{\Dcons{\Delta, \Dhyp{t}}{\Delta_S}}{\tau'}$ \BY{IH on \pfref{cvalidT}} \pflabel{istypeU1}
%   \item $\istypeU{\Dcons{\Delta}{\Delta_S}, \Dhyp{t}}{\tau'}$ \BY{exchange over $\Delta_S$ on \pfref{istypeU1}} \pflabel{istypeU2}
%   \item $\istypeU{\Dcons{\Delta}{\Delta_S}}{\arec{t}{\tau'}}$ \BY{Rule (\ref{rule:istypeU-rec}) on \pfref{istypeU2}}
% \end{pfsteps*}
% \resetpfcounter

% \item[\text{(\ref{rule:cvalidT-U-prod})}] We have
% \begin{pfsteps*}
% \item $\ctau=\aceprod{\labelset}{\mapschema{\ctau}{i}{\labelset}}$ \BY{assumption}  
% \item $\tau=\aprod{\labelset}{\mapschema{\tau}{i}{\labelset}}$ \BY{assumption}
% \item $\{\cvalidT{\Delta}{\tsceneU{\Delta_S}{b}}{\ctau_i}{\tau_i}\}_{i \in \labelset}$ \BY{assumption} \pflabel{cvalidT-ass}
% \item $\{\istypeU{\Dcons{\Delta}{\Delta_S}}{\tau_i}\}_{i \in \labelset}$ \BY{IH on \pfref{cvalidT-ass}$_i$ for each $i \in \labelset$} \pflabel{istype}
% \item $\istypeU{\Dcons{\Delta}{\Delta_S}}{\aprod{\labelset}{\mapschema{\tau}{i}{\labelset}}}$ \BY{Rule (\ref{rule:istypeU-prod}) on \pfref{istype}}
% \end{pfsteps*}
% \resetpfcounter 

% \item[\text{(\ref{rule:cvalidT-U-sum})}] We have
% \begin{pfsteps*}
% \item $\ctau=\acesum{\labelset}{\mapschema{\ctau}{i}{\labelset}}$ \BY{assumption}  
% \item $\tau=\asum{\labelset}{\mapschema{\tau}{i}{\labelset}}$ \BY{assumption}
% \item $\{\cvalidT{\Delta}{\tsceneU{\Delta_S}{b}}{\ctau_i}{\tau_i}\}_{i \in \labelset}$ \BY{assumption} \pflabel{cvalidT-ass}
% \item $\{\istypeU{\Dcons{\Delta}{\Delta_S}}{\tau_i}\}_{i \in \labelset}$ \BY{IH on \pfref{cvalidT-ass}$_i$ for each $i \in \labelset$} \pflabel{istype}
% \item $\istypeU{\Dcons{\Delta}{\Delta_S}}{\asum{\labelset}{\mapschema{\tau}{i}{\labelset}}}$ \BY{Rule (\ref{rule:istypeU-sum}) on \pfref{istype}}
% \end{pfsteps*}
% \resetpfcounter

\item[\text{(\ref{rule:cvalidT-U-splicedt})}] We have
\begin{pfsteps*}
\item $\ctau=\acesplicedt{m}{n}$ \BY{assumption}
\item $\parseTyp{\bsubseq{b}{m}{n}}{\tau}$ \BY{assumption}
\item $\istypeU{\Delta_S}{\tau}$ \BY{assumption}\pflabel{istype}
\item $\istypeU{\Dcons{\Delta}{\Delta_S}}{\tau}$ \BY{Lemma \ref{lemma:weakening-U} over $\Delta$ on \pfref{istype} and exchange}
\end{pfsteps*}
\resetpfcounter
\end{byCases}
\end{proof}

\chapter{Unparameterized Pattern TSMs}\label{sec:pattern-tsms}
In Chapter \ref{chap:tsms}, we considered situations where the programmer needed to \emph{construct} (a.k.a. \emph{introduce}) a value. In this chapter, we consider situations where the programmer needs to \emph{deconstruct} (a.k.a. \emph{eliminate}) a value. In full-scale functional languages like ML and Haskell, values are deconstructed by \emph{pattern matching} over their structure. For example, recall the recursive labeled sum type \lstinline{Rx} defined in Figure \ref{fig:datatype-rx}. We can pattern match over a value, \lstinline{r}, of type \lstinline{Rx} using VerseML's \lstinline{match} construct:
\begin{lstlisting}
fun read_example_rx(r : Rx) : (string * Rx) option => 
  match r with 
    Seq(Str(name), Seq(Str "SSTR: ESTR", ssn)) => Some (name, ssn)
  | _ => None
\end{lstlisting}

Match expressions consist of a \emph{scrutinee}, here \li{r}, and a sequence of \emph{rules} separated by vertical bars, \li{|}, in the concrete syntax. Each rule consists of a \emph{pattern} and an {expression} called the corresponding \emph{branch}, separated by a double arrow, \li{=>}, in the concrete syntax. When the {match} expression is evaluated, the value of the scrutinee is matched against each pattern sequentially. If the value matches, evaluation takes the corresponding branch. Variables in patterns match any value of the appropriate type. In the corresponding branch, the variable stands for that value. Variables can each appear only once in a pattern.  
For example, on Line 3, the pattern \li{Seq(Str(name), Seq(Str "SSTR: ESTR", ssn))} matches values of the form \li{Seq(Str(#$e_1$#), Seq(Str "SSTR: ESTR", #$e_2$#))}, where $e_1$ is a value of type \li{string} and $e_2$ is a value of type \li{Rx}. The variables \li{name} and \li{ssn} stand for the values of $e_1$ and $e_2$, respectively, in \li{Some (name, ssn)}. On Line 4, the pattern \li{_} is the \emph{wildcard pattern} -- it matches any value of the appropriate type and binds no variables.

The behavior of the \li{match} construct when no pattern in the rule sequence matches a value is to raise an exception indicating \emph{match failure}. It is possible to statically determine whether match failure is possible (i.e. whether there exist values of the scrutinee that are not matched by any pattern in the rule sequence). In the example above, our use of the wildcard pattern ensures that match failure cannot occur. A rule sequence that cannot lead to match failure is said to be \emph{exhaustive}. Most compilers warn the programmer when a rule sequence is non-exhaustive.

It is also possible to statically decide when a rule is \emph{redundant} relative to the preceding rules, i.e. when there does not exist a value matched by that rule but not matched by any of the preceding rules. For example, if we add  another rule at the end of the match expression above, it will be redundant because all values match the wildcard pattern. Again, most compilers warn the programmer when a rule is redundant.

Nested pattern matching generalizes the projection and case analysis operators (i.e. the \emph{eliminators}) for products and sums (cf. $\miniVerseUE$ from the previous section) and decreases syntactic cost in situations where eliminators would need to be nested. There remains room for improvement, however, because complex patterns sometimes    individually have high syntactic cost. In Sec. \ref{sec:syntax-examples-regexps}, we considered a hypothetical dialect of ML called ML+Rx that built in derived syntax both for constructing and pattern matching over values of the recursive labeled sum type \li{Rx}. In ML+Rx, we can express the example above at lower syntactic cost as follows:

\begin{lstlisting}
fun read_example_rx(r : Rx) : (string * Rx) option => 
  match r with 
    /SURL@EURLnameSURL: %EURLssn/ => Some (name, ssn)
  | _ => None\end{lstlisting}
\noindent
Dialect formation is not a modular approach, for the reasons discussed in Chapter \ref{chap:intro}, so we seek language constructs that allow us to decrease the syntactic cost of expressing complex patterns to a similar degree.

Expression TSMs -- introduced in Chapter \ref{chap:tsms} -- can decrease the syntactic cost of constructing a value of a specified type. However, expressions are syntactically distinct from patterns, so we cannot simply apply an expression TSM to generate a pattern.\footnote{The fact that certain concrete expression and pattern forms overlap is immaterial to this fundamental distinction. There are many expression forms that the expansion generated by an expression TSM might use that have no corresponding pattern form, e.g.  lambda abstraction.} %For example, the expansion generated by an expression TSM might define or apply a function, but patterns do not contain functions or function applications. 
For this reason, we need to introduce a new (albeit closely related) construct -- the \textbf{pattern TSM}. In this chapter, we consider only \textbf{unparameterized pattern TSMs} (upTSMs), i.e. pattern TSMs that generate patterns that match values of a single specified type, like \li{Rx}. In Chapter \ref{sec:tsms-parameterized}, we will consider both expression and pattern TSMs that specify type and module parameters (peTSMs and ppTSMs). 

\section{Pattern TSMs By Example}\label{sec:ptsms-by-example}
The organization of the remainder of this chapter mirrors that of Chapter \ref{chap:tsms}. We begin in this section with a ``tutorial-style'' introduction to upTSMs in VerseML. In particular, we  discuss an upTSM for patterns matching values of type \li{Rx}. In the next section, we specify a reduced formal system based on $\miniVerseUE$ called $\miniVersePat$ that makes the intuitions developed here mathematically precise.

\subsection{Usage}\label{sec:ptsms-usage}
The VerseML function \li{read_example_rx} defined at the beginning of this chapter can be concretely expressed at lower syntactic cost by applying a upTSM, \li{#\dolla#rx}, as follows:
\begin{lstlisting}
fun read_example_rx(r : Rx) : (string * Rx) option => 
  match r with 
    $rx /SURL@EURLnameSURL: %EURLssn/ => Some (name, ssn)
  | _ => None
\end{lstlisting}
Like expression TSMs, pattern TSMs are applied to \emph{generalized literal forms} (see Figure \ref{fig:literal-forms}). Generalized literal forms are left unparsed when patterns are first parsed. During the subsequent \emph{typed expansion} process, the pattern TSM parses the body of the literal form to generate a \emph{candidate expansion}. The language validates the candidate expansion according to criteria that we will establish in Sec. \ref{sec:ptsms-validation}. If validation succeeds, the language generates the final expansion (or more concisely, simply the expansion) of the pattern. The expansion of the unexpanded pattern \li{#\dolla#rx /SURL@EURLnameSURL: %EURLssn/} from the example above is the following pattern:
\begin{lstlisting}[numbers=none]
Seq(Str(name), Seq(Str "SSTR: ESTR", ssn))
\end{lstlisting}

The checks for exhaustiveness and redundancy can be performed post-expansion in the usual way, so we do not need to consider them further here. 
\subsection{Definition}\label{sec:ptsms-definition}
The definition of the pattern TSM \li{#\dolla#rx} shown being applied in the example above has the following form:
\begin{lstlisting}[numbers=none]
syntax $rx at Rx for patterns {
  static fn(body : Body) : CEPat ParseResult =>
    (* regex pattern parser here *)
}
\end{lstlisting}
This definition first names the pattern TSM. Pattern TSM names, like expression TSM names, must begin with the dollar symbol (\li{#\dolla#}) to distinguish them from labels. Pattern TSM names and expression TSM names are tracked separately, i.e. an expression TSM and a pattern TSM can have the same name without conflict (as is the case here -- the expression TSM described in Sec. \ref{sec:uetsms-definition} is also named \li{#\dolla#rx}). The \emph{sort qualifier} \li{for patterns} indicates that this is a pattern TSM definition, rather than an expression TSM definition (the sort qualifier \li{for expressions} can be written for expression TSMs, though when the sort qualifier is omitted this is the default). Because defining both an expression TSM and a pattern TSM with the same name at the same type is a common idiom, VerseML provides a primitive derived form for combining their definitions:
\begin{lstlisting}[numbers=none]
syntax $rx at Rx for expressions {
  static fn(body : Body) : CEExp ParseResult => 
    (* regex expression parser here *)
} for patterns {
  static fn(body : Body) : CEPat ParseResult => 
    (* regex pattern parser here *)
}
\end{lstlisting}

Pattern TSMs, like expression TSMs, must specify a static \emph{parse function}, delimited by curly braces in the concrete syntax. For a pattern TSM, the parse function must be of type \li{Body -> CEPat ParseResult}. The input type, \li{Body}, gives the parse function access to the body of the provided literal form, and is defined as in Sec. \ref{sec:uetsms-definition} as a synonym for the type \li{string}. The output type, \li{CEPat ParseResult}, is the parameterized type constructor \li{ParseResult}, defined in Figure \ref{fig:indexrange-and-parseresult}, applied to the type \li{CEPat} defined in Figure \ref{fig:CEPat}.  So if parsing succeeds, the pattern TSM returns a value of the form \li{Success #$\ecand$#} where $\ecand$ is a value of type \li{CEPat} that we call the \emph{encoding of the candidate expansion}. If parsing fails, then the pattern TSM returns a value constructed by \li{ParseError} and equipped with an error message and error location. 

The type \li{CEPat} is analagous to the types \li{CEExp} and \li{CETyp} defined in Figure \ref{fig:candidate-exp-verseml}. It encodes the abstract syntax of VerseML patterns (in Figure \ref{fig:CEPat}, some constructors are elided for concision), with the exception of variable patterns (for reasons explained in Sec. \ref{sec:ptsms-hygiene} below), and includes an additional constructor, \li{Spliced}, for referring to spliced subpatterns by their position within the parse stream, discussed next.

\begin{figure}
\begin{lstlisting}[numbers=none]
type CEPat = Wild
           | (* ... *)
           | Spliced of IndexRange
\end{lstlisting}
\caption{Abbreviated definition of \li{CEPat} in the VerseML prelude.}
\label{fig:CEPat}
\end{figure}

\subsection{Splicing}\label{sec:ptsms-splicing}
Patterns that appear directly within the literal body of an unexpanded pattern are called \emph{spliced subpatterns}. For example, the patterns \li{name} and \li{ssn} appear within the unexpanded pattern \li{#\dolla#rx /SURL@EURLnameSURL: %EURLssn/}. 
When the parse function determines that a subsequence of the literal body should be treated as a spliced subpatern (here, by recognizing the characters \li{@} or \li{%} followed by a variable or parenthesized pattern), 
it can refer to it within the candidate expansion that it construct a reference to it for use within the candidate expansion it generates using the \li{Spliced} constructor of the \li{CEPat} type shown in Figure \ref{fig:CEPat}. The \li{Spliced} constructor requires a value of type \li{IndexRange} because spliced subpatterns are referred to indirectly by their position within the literal body. This prevents pattern TSMs from ``forging'' a spliced subpattern (i.e. claiming that some pattern is a spliced subpattern, even though it does not appear in the literal body).

The candidate expansion generated by the pattern TSM \li{#\dolla#rx} for the example above, if written in a hypothetical concrete syntax where references to spliced subpatterns are written \li{spliced<startIdx, endIdx>}, is:
\begin{lstlisting}[numbers=none]
Seq(Str(spliced<1, 4>), Seq(Str "SSTR: ESTR", spliced<8, 10>))
\end{lstlisting}
Here, \li{spliced<1, 4>} refers to the subpattern \li{name} by position, and \li{spliced<8, 10>} refers to the subpattern \li{ssn} by position.

\subsection{Typing}\label{sec:ptsms-validation}
The language validates candidate expansion before a final expansion is generated. One aspect of candidate expansion validation is checking the candidate expansion against the type annotation specified by the pattern TSM, e.g. the type \li{Rx} in the example above.

\subsection{Hygiene}\label{sec:ptsms-hygiene}
In order to check that the candidate expansion is well-typed, the language must parse, type and expand the spliced subpatterns that the candidate expansion refers to (by their position within the literal body, cf. above). To maintain a useful binding discipline, i.e. to allow programmers to reason about variable binding without examining expansions directly, the validation process allows variables (e.g. \lstinline{name} and \lstinline{ssn} above) to occur only in spliced subpatterns (just as variables bound at the use site can only appear in spliced subexpressions when using TSMs). Indeed, there is no constructor for the type \li{CEPat} corresponding to a variable pattern. This protection against ``hidden bindings'' is beneficial because it leaves variable naming entirely up to the client of the pattern TSM. 

\subsection{Final Expansion}\label{sec:ptsms-final-expansion}
If validation succeeds, the semantics generates the \emph{final expansion} of the pattern from the candidate expansion by replacing the references to spliced subpatterns with their final expansions. For example, the final expansion of \li{#\dolla#rx /SURL@EURLnameSURL: %EURLssn/} is:
\begin{lstlisting}[numbers=none]
Seq(Str(name), Seq(Str "SSTR: ESTR", ssn))
\end{lstlisting}

\section{\texorpdfstring{$\miniVersePat$}{miniVerseU}}\label{sec:miniVerseUP}
To make the intuitions developed in the previous section about pattern TSMs precise, we  now introduce a reduced calculus with support for both unparameterized expression TSMs and unparameterized pattern TSMs called $\miniVersePat$.
\subsection{Types, Expanded Expressions, Rules and Patterns}\label{sec:UP-expanded-terms}
The \emph{inner core} of $\miniVersePat$ consists of \emph{types}, $\tau$, \emph{expanded expressions}, $e$, \emph{expanded rules}, $r$, and \emph{expanded patterns}, $p$. Their syntax is specified by the syntax chart in Figure \ref{fig:UP-expanded-terms}. The inner core of $\miniVersePat$ differs from that of $\miniVerseUE$  only in that the case analysis operator has been replaced by the pattern matching operator, so we will omit some definitions that are identical to those in Sec. \ref{fig:U-expanded-terms}. Our formulation of the semantics of pattern matching is adapted from the Harper's formulation in \emph{Practical Foundations for Programming Languages, First Edition} \cite{pfple1}.\footnote{The chapter on pattern matching has, of this writing, been removed from the second edition of \emph{PFPL}, but a copy of the first edition can be found online.}

\begin{figure}
$\begin{array}{lllllll}
\textbf{Sort} & & & \textbf{Uniform Form} & \textbf{Stylized Form} & \textbf{Description}\\
\mathsf{Typ} & \tau & ::= & t & t & \text{variable}\\
&&& \aparr{\tau}{\tau} & \parr{\tau}{\tau} & \text{partial function}\\
&&& \aall{t}{\tau} & \forallt{t}{\tau} & \text{polymorphic}\\
&&& \arec{t}{\tau} & \rect{t}{\tau} & \text{recursive}\\
&&& \aprod{\labelset}{\mapschema{\tau}{i}{\labelset}} & \prodt{\mapschema{\tau}{i}{\labelset}} & \text{labeled product}\\
&&& \asum{\labelset}{\mapschema{\tau}{i}{\labelset}} & \sumt{\mapschema{\tau}{i}{\labelset}} & \text{labeled sum}\\
\mathsf{EExp} & e & ::= & x & x & \text{variable}\\
&&& \aelam{\tau}{x}{e} & \lam{x}{\tau}{e} & \text{abstraction}\\
&&& \aeap{e}{e} & \ap{e}{e} & \text{application}\\
&&& \aetlam{t}{e} & \Lam{t}{e} & \text{type abstraction}\\
&&& \aetap{e}{\tau} & \App{e}{\tau} & \text{type application}\\
&&& \aefold{t}{\tau}{e} & \fold{e} & \text{fold}\\
&&& \aeunfold{e} & \unfold{e} & \text{unfold}\\
&&& \aetpl{\labelset}{\mapschema{e}{i}{\labelset}} & \tpl{\mapschema{e}{i}{\labelset}} & \text{labeled tuple}\\
&&& \aepr{\ell}{e} & \prj{e}{\ell} & \text{projection}\\
&&& \aein{\labelset}{\ell}{\mapschema{\tau}{i}{\labelset}}{e} & \inj{\ell}{e} & \text{injection}\\
&&& \aematchwith{n}{e}{\seqschemaX{r}} & \matchwith{e}{\seqschemaX{r}} & \text{match}\\
\mathsf{ERule} & r & ::= & \aematchrule{n}{p}{\seqschemaX{x}}{e} & \matchrule{p}{e} & \text{rule}\\
\mathsf{EPat} & p & ::= & x & x & \text{variable pattern}\\
&&& \aewildp & \wildp & \text{wildcard pattern}\\
%&&& \aefoldp{p} & \foldp{p} & \text{fold pattern}\\
&&& \aetplp{\labelset}{\mapschema{p}{i}{\labelset}} & \tplp{\mapschema{p}{i}{\labelset}} & \text{labeled tuple pattern}\\
&&& \aeinjp{\ell}{p} & \injp{\ell}{p} & \text{injection pattern}
\end{array}$
\caption[Syntax of types and expanded expressions, rules and patterns in $\miniVersePat$.]{Syntax of types and expanded expressions, rules and patterns in $\miniVersePat$. We adopt the metatheoretic conventions established for our specification of $\miniVerseUE$ in Sec. \ref{sec:miniVerseU} without restating them, and assume that all metavariables not redefined here range over the same metatheoretic objects. We write $\seqschemaX{r}$ for sequences of $n \geq 0$ rule arguments and $\seqschemaX{x}.e$ for expressions binding the sequence of $n \geq 0$ variables $\seqschemaX{x}$. There is no notion of substitution within patterns, i.e. variables in patterns do not stand for terms, but rather serve as references to the bindings in the rule that the pattern appears within. The semantics below makes this clear.}
\label{fig:UP-expanded-terms}
\end{figure}


\subsection{Statics of the Inner Core}
The \emph{statics of the inner core} is specified by judgements of the following form:
\[\begin{array}{ll}
\textbf{Judgement Form} & \textbf{Description}\\
\istypeU{\Delta}{\tau} & \text{$\tau$ is a well-formed type assuming $\Delta$}\\
%\isctxU{\Delta}{\Gamma} & \text{$\Gamma$ is a well-formed typing context assuming $\Delta$}\\
\hastypeU{\Delta}{\Gamma}{e}{\tau} & \text{$e$ has type $\tau$ assuming $\Delta$ and $\Gamma$}\\
\ruleType{\Delta}{\Gamma}{r}{\tau}{\tau'} & \text{$r$ takes values of type $\tau$ to values of type $\tau'$ assuming $\Delta$ and $\Gamma$}\\
\patType{\pctx}{p}{\tau} & \text{$p$ matches values of type $\tau$ and generates hypotheses $\pctx$} 
\end{array}\]

The types of $\miniVersePat$ are exactly those of $\miniVerseUE$, described in Sec. \ref{sec:miniVerseU}, so the \emph{type formation judgement}, $\istypeU{\Delta}{\tau}$, is inductively defined by Rules (\ref{rules:istypeU}). 

The \emph{typing judgement}, $\hastypeU{\Delta}{\Gamma}{e}{\tau}$, assigns types to expressions and is inductively defined by Rules (\ref*{rules:hastypeUP}), which consist of:
\begin{subequations}\label{rules:hastypeUP}
\begin{itemize}
\item Rules (\ref*{rules:hastypeUP}a) through (\ref*{rules:hastypeUP}j) derived identically from Rules (\ref{rule:hastypeU-var}) through (\ref{rule:hastypeU-in}), respectively. Note that we cannot defer directly to the typing rules from Sec. \ref{sec:miniVerseU} because $e$ has been redefined here.
\item The following rule for match expressions: 
\end{itemize}
\setcounter{equation}{10}
\begin{equation}\label{rule:hastypeUP-match}
\inferrule{
  \hastypeU{\Delta}{\Gamma}{e}{\tau}\\
  \{\ruleType{\Delta}{\Gamma}{r_i}{\tau}{\tau'}\}_{1 \leq i < n}
}{\hastypeU{\Delta}{\Gamma}{\aematchwith{n}{e}{\seqschemaX{r}}}{\tau'}}
\end{equation}  
\end{subequations}
The first premise of Rule (\ref{rule:hastypeUP-match}) assigns a type, $\tau$, to the scrutinee, $e$. The second premise then ensures that each rule $r_i$ takes values of type $\tau$ to values of the type assigned to the match expression as a whole, $\tau'$. This is expressed by the \emph{rule typing judgement}, $\ruleType{\Delta}{\Gamma}{r}{\tau}{\tau'}$, which is defined mutually with Rules (\ref{rules:hastypeUP}) by the following rule:
\begin{equation}\label{rule:ruleType}
\inferrule{
  \patType{\pctx}{p}{\tau}\\
  \domof{\pctx} = \seqschemaX{x}\\
  \hastypeU{\Delta}{\Gcons{\Gamma}{\pctx}}{e}{\tau'}
}{\ruleType{\Delta}{\Gamma}{\aematchrule{n}{p}{\seqschemaX{x}}{e}}{\tau}{\tau'}}
\end{equation}
The premises of Rule (\ref{rule:ruleType}) can be understood as follows, in order:
\begin{enumerate}
\item The first premise invokes the \emph{pattern typing judgement}, $\patType{\pctx}{p}{\tau}$, to check that the pattern, $p$, matches values of type $\tau$, and gather the typing hypotheses that the pattern generates in a \emph{pattern typing context}, $\Omega$. Pattern typing contexts, like typing contexts, $\Gamma$, consist of hypotheses $x : \tau$. We specify a distinct metavariable because the {pattern typing judgement} is a \emph{linear hypothetical judgement}, i.e. it does not obey weakening. This is to ensure that the pattern typing context captures exactly those variables bound by a pattern. Algorithmically, one can think of the pattern typing context as an ``output'' of pattern typing.

The pattern typing judgement is inductively defined by the following rules:
\begin{subequations}\label{rules:patType}
\begin{equation}\label{rule:patType-var}
\inferrule{ }{\patType{\Ghyp{x}{\tau}}{x}{\tau}}
\end{equation}
\begin{equation}\label{rule:patType-wild}
\inferrule{ }{\patType{\emptyset}{\aewildp}{\tau}}
\end{equation}
\begin{equation}\label{rule:patType-tpl}
\inferrule{
  \{\patType{\pctx_i}{p_i}{\tau_i}\}_{i \in \labelset}\\
  \cap_{i \in \labelset} \domof{\pctx_i} = \emptyset
}{
  \patType{\Gconsi{i \in \labelset}{\pctx_i}}{\aetplp{\labelset}{\mapschema{p}{i}{\labelset}}}{\aprod{\labelset}{\mapschema{\tau}{i}{\labelset}}}
}
\end{equation}
\begin{equation}\label{rule:patType-inj}
\inferrule{
  \patType{\pctx}{p}{\tau}
}{
  \patType{\pctx}{\aeinjp{\ell}{p}}{\asum{\labelset, \ell}{\mapschema{\tau}{i}{\labelset}; \mapitem{\ell}{\tau}}}
}
\end{equation}
\end{subequations}

Rule (\ref{rule:patType-var}) specifies that a variable pattern, $x$, can match values of any type, $\tau$, and generates the hypothesis that $x$ has the type $\tau$. 

Rule (\ref{rule:patType-wild}) specifies that a wildcard pattern can also match values of any type, $\tau$, but wildcard patterns generate no hypotheses. 

Labeled tuple patterns, $\aetplp{\labelset}{\mapschema{p}{i}{\labelset}}$, specify a subpattern $p_i$ for each label $i \in \labelset$. Rule (\ref{rule:patType-tpl}) specifies that a labeled tuple pattern of this form matches values of the labeled product type $\aprod{\labelset}{\mapschema{\tau}{i}{\labelset}}$. Each subpattern $p_i$ is checked against the corresponding type $\tau_i$, generating hypotheses $\pctx_i$. The rule then ensures that the subpatterns bind distinct variables by checking that the domains of the generated pattern typing contexts $\pctx_i$ are disjoint. The hypotheses generated in the conclusion of the rule are the union of the hypotheses $\pctx_i$.

Injection patterns, $\aeinjp{\ell}{p}$, match values of labeled sum types of the form $\asum{\labelset, \ell}{\mapschema{\tau}{i}{\labelset}; \mapitem{\ell}{\tau}}$, i.e. labeled sum types that specify a case for $\ell$. Rule (\ref{rule:patType-inj}) checks the subpattern $p$ against the corresponding type $\tau$, and passes through the assumptions that $p$ generates.

\item The second premise of Rule (\ref{rule:ruleType}) ensures that the pattern has generated hypotheses for all of the the variables that the branch expression, $e$, binds. This is merely a matter of ``metatheoretic bookkeeping''. In the stylized form for rules, $\matchrule{p}{e}$, the variables bound in $e$ are, implicitly, exactly those mentioned in the pattern.% The bindings for $e$ would be extracted from the pattern implicitly. 
\item The final premise of Rule (\ref{rule:ruleType}) extends the typing context, $\Gamma$, with the hypotheses generated by pattern typing, $\pctx$, and checks the branch expression, $e$, against the branch type, $\tau'$.
\end{enumerate}

The rules above validate the following standard lemmas.

The Weakening Lemma expresses the intuition that extending a typing or type formation contxt with unused variables preserves well-formedness, typing and rule typing.
\begin{lemma}[Weakening]\label{lemma:weakening-UP} All of the following hold: 
\begin{enumerate} 
\item If $\istypeU{\Delta}{\tau}$ then $\istypeU{\Delta, \Dhyp{t}}{\tau}$.
%\item If $\isctxU{\Delta}{\Gamma}$ then $\isctxU{\Delta, \Dhyp{t}}{\Gamma}$.
\item \begin{enumerate}
  \item If $\hastypeU{\Delta}{\Gamma}{e}{\tau}$ then $\hastypeU{\Delta, \Dhyp{t}}{\Gamma}{e}{\tau}$.
  \item If $\ruleType{\Delta}{\Gamma}{r}{\tau}{\tau'}$ then $\ruleType{\Delta, \Dhyp{t}}{\Gamma}{r}{\tau}{\tau'}$.
  \end{enumerate}
\item \begin{enumerate}
  \item If $\hastypeU{\Delta}{\Gamma}{e}{\tau}$ and $\istypeU{\Delta}{\tau''}$ then $\hastypeU{\Delta}{\Gamma, \Ghyp{x}{\tau''}}{e}{\tau}$.
  \item If $\ruleType{\Delta}{\Gamma}{r}{\tau}{\tau'}$ and $\istypeU{\Delta}{\tau''}$ then $\ruleType{\Delta}{\Gamma, \Ghyp{x}{\tau''}}{r}{\tau}{\tau'}$.
  \end{enumerate}
\end{enumerate}
\end{lemma}
\begin{proof-sketch}
\begin{enumerate}
\item By rule induction on Rules (\ref{rules:istypeU}).
%\item By rule induction on Rules (\ref{rules:isctxU}).
\item By mutual rule induction on Rules (\ref{rules:hastypeUP}) and Rule (\ref{rule:ruleType}).
\item By mutual rule induction on Rules (\ref{rules:hastypeUP}) and Rule (\ref{rule:ruleType}).
\end{enumerate}
\end{proof-sketch}

The Substitution Lemma expresses the intuition that substitution of a type for a type variable, or an expression of the appropriate type for an expression variable, preserves well-formedness, typing and rule typing.
\begin{lemma}[Substitution]\label{lemma:substitution-UP} All of the following hold:
\begin{enumerate}
\item If $\istypeU{\Delta, \Dhyp{t}}{\tau}$ and $\istypeU{\Delta}{\tau'}$ then $\istypeU{\Delta}{[\tau'/t]\tau}$.
%\item If $\isctxU{\Delta, \Dhyp{t}}{\Gamma}$ and $\istypeU{\Delta}{\tau'}$ then $\isctxU{\Delta}{[\tau'/t]\Gamma}$.
\item \begin{enumerate}
  \item If $\hastypeU{\Delta, \Dhyp{t}}{\Gamma}{e}{\tau}$ and $\istypeU{\Delta}{\tau'}$ then $\hastypeU{\Delta}{[\tau'/t]\Gamma}{[\tau'/t]e}{[\tau'/t]\tau}$.
  \item If $\ruleType{\Delta, \Dhyp{t}}{\Gamma}{r}{\tau}{\tau'}$ and $\istypeU{\Delta}{\tau''}$ then $\ruleType{\Delta}{[\tau''/t]\Gamma}{[\tau''/t]r}{[\tau''/t]\tau}{[\tau''/t]\tau'}$.
  \end{enumerate}
\item \begin{enumerate}
  \item If $\hastypeU{\Delta}{\Gamma, \Ghyp{x}{\tau'}}{e}{\tau}$ and $\hastypeU{\Delta}{\Gamma}{e'}{\tau'}$ then $\hastypeU{\Delta}{\Gamma}{[e'/x]e}{\tau}$.
  \item If $\ruleType{\Delta}{\Gamma, \Ghyp{x}{\tau''}}{r}{\tau}{\tau'}$ and $\hastypeU{\Delta}{\Gamma}{e}{\tau''}$ then $\ruleType{\Delta}{\Gamma}{[e/x]r}{\tau}{\tau'}$.
  \end{enumerate}
\end{enumerate}\end{lemma}
\begin{proof-sketch}
\begin{enumerate}
\item By rule induction on Rules (\ref{rules:istypeU}).
\item By mutual rule induction on Rules (\ref{rules:hastypeUP}) and Rule (\ref{rule:ruleType}).
\item By mutual rule induction on Rules (\ref{rules:hastypeUP}) and Rule (\ref{rule:ruleType}).
\end{enumerate}
\end{proof-sketch}

The Decomposition Lemma is the converse of the Substitution Lemma.
\begin{lemma}[Decomposition]\label{lemma:decomposition-UP} All of the following hold:
\begin{enumerate}
\item If $\istypeU{\Delta}{[\tau'/t]\tau}$ and $\istypeU{\Delta}{\tau'}$ then $\istypeU{\Delta, \Dhyp{t}}{\tau}$.
%\item If $\isctxU{\Delta}{[\tau'/t]\Gamma}$ and $\istypeU{\Delta}{\tau'}$ then $\isctxU{\Delta, \Dhyp{t}}{\Gamma}$.
\item \begin{enumerate}
  \item If $\hastypeU{\Delta}{[\tau'/t]\Gamma}{[\tau'/t]e}{[\tau'/t]\tau}$ and $\istypeU{\Delta}{\tau'}$ then $\hastypeU{\Delta, \Dhyp{t}}{\Gamma}{e}{\tau}$.
  \item If $\ruleType{\Delta}{[\tau'/t]\Gamma}{[\tau'/t]r}{[\tau'/t]\tau}{[\tau'/t]\tau''}$ and $\istypeU{\Delta}{\tau'}$ then $\ruleType{\Delta, \Dhyp{t}}{\Gamma}{r}{\tau}{\tau''}$.
  \end{enumerate}
\item \begin{enumerate}
  \item If $\hastypeU{\Delta}{\Gamma}{[e'/x]e}{\tau}$ and $\hastypeU{\Delta}{\Gamma}{e'}{\tau'}$ then $\hastypeU{\Delta}{\Gamma, \Ghyp{x}{\tau'}}{e}{\tau}$.
  \item If $\ruleType{\Delta}{\Gamma}{[e'/x]r}{\tau}{\tau''}$ and $\hastypeU{\Delta}{\Gamma}{e'}{\tau'}$ then $\ruleType{\Delta}{\Gamma, \Ghyp{x}{\tau'}}{r}{\tau}{\tau''}$.
  \end{enumerate}
\end{enumerate}\end{lemma}
\begin{proof-sketch}
\begin{enumerate}
\item By rule induction on Rules (\ref{rules:istypeU}) and case analysis over the definition of substitution. In all cases, the derivation of $\istypeU{\Delta}{[\tau'/t]\tau}$ does not depend on the structure of $\tau'$.
%\item Context formation of $[\tau'/t]\Gamma$ does not depend on the structure of $\tau'$.
\item By mutual rule induction on Rules (\ref{rules:hastypeUP}) and Rule (\ref{rule:ruleType}) and case analysis over the definition of substitution. In all cases, the derivation of $\hastypeU{\Delta}{[\tau'/t]\Gamma}{[\tau'/t]e}{[\tau'/t]\tau}$ or $\ruleType{\Delta}{[\tau'/t]\Gamma}{[\tau'/t]r}{[\tau'/t]\tau}{[\tau'/t]\tau''}$ does not depend on the structure of $\tau'$.
\item By mutual rule induction on Rules (\ref{rules:hastypeUP}) and Rule (\ref{rule:ruleType}) and case analysis over the definition of substitution. In all cases, the derivation of $\hastypeU{\Delta}{\Gamma}{[e'/x]e}{\tau}$ or $\ruleType{\Delta}{\Gamma}{[e'/x]r}{\tau}{\tau''}$ does not depend on the structure of $e'$.
\end{enumerate}
\end{proof-sketch}

The Pattern Regularity Lemma expresses the intuition that the hypotheses generated by checking a pattern against a well-formed type involve only well-formed types themselves.
\begin{lemma}[Pattern Regularity]\label{lemma:pattern-regularity-UP} 
If $\patType{\pctx}{p}{\tau}$ and $\istypeU{\Delta}{\tau}$ then $\istypeU{\Delta}{\tau_i}$ for each assumption $x_i : \tau_i$ in $\pctx$.
\end{lemma}
\begin{proof} By rule induction on Rules (\ref{rules:patType}).
\begin{byCases}
\item[\text{(\ref{rule:patType-var})}] We have:
\begin{pfsteps*}
  \item $p=x$ \BY{assumption}
  \item $\pctx=x : \tau$ \BY{assumption}
  \item $\istypeU{\Delta}{\tau}$ \BY{assumption}\pflabel{istypeU}
 \end{pfsteps*}
 \resetpfcounter
\item[\text{(\ref{rule:patType-wild})}] We have $\pctx=\emptyset$ by assumption, so the conclusion trivially holds.
\item[\text{(\ref{rule:patType-tpl})}] We have:
\begin{pfsteps*}
  \item $p=\aetplp{\labelset}{\mapschema{p}{i}{\labelset}}$ \BY{assumption}
  \item $\tau=\aprod{\labelset}{\mapschema{\tau}{i}{\labelset}}$ \BY{assumption}
  \item $\patType{\pctx_i}{p_i}{\tau_i}$ for each $i \in \labelset$ \BY{assumption}\pflabel{patType}
  \item $\pctx=\cup_{i \in \labelset} \pctx_i$ \BY{assumption}
  \item $\istypeU{\Delta}{\aprod{\labelset}{\mapschema{\tau}{i}{\labelset}}}$ \BY{assumption} \pflabel{istypeU}
  \item $\istypeU{\Delta}{\tau_i}$ for each $i \in \labelset$ \BY{Inversion of Rule (\ref{rule:istypeU-prod}) on \pfref{istypeU}}\pflabel{istypeU-each}
  \item $\istypeU{\Delta}{\tau_{ij}}$ for each assumption $x_{ij} : \tau_{ij}$ in $\pctx_i$, for each $i \in \labelset$ \BY{IH on \pfref{patType} and \pfref{istypeU-each} for each $i \in \labelset$} \pflabel{biggy}
  \item $\istypeU{\Delta}{\tau_{ij}}$ for each assumption $x_{ij} : \tau_{ij}$ in $\cup_{i \in \labelset}\pctx_i$ \BY{the definition of pattern context union and \pfref{biggy}}
\end{pfsteps*}
\resetpfcounter
\item[\text{(\ref{rule:patType-inj})}] We have:
\begin{pfsteps*}
  \item $p=\aeinjp{\ell}{p'}$ \BY{assumption}
  \item $\tau=\asum{\labelset, \ell}{\mapschema{\tau}{i}{\labelset}; \mapitem{\ell}{\tau'}}$ \BY{assumption}
  \item $\istypeU{\Delta}{\asum{\labelset, \ell}{\mapschema{\tau}{i}{\labelset}; \mapitem{\ell}{\tau'}}}$ \BY{assumption} \pflabel{istype}
  \item $\patType{\pctx}{p'}{\tau'}$ \BY{assumption} \pflabel{patType}
  \item $\istypeU{\Delta}{\tau'}$ \BY{Inversion of Rule (\ref{rule:istypeU-sum}) on \pfref{istype}} \pflabel{istypeTwo} 
  \item $\istypeU{\Delta}{\tau_i}$ for each assumption $x : \tau_i$ in $\pctx$ \BY{IH on \pfref{patType} and \pfref{istypeTwo}}
\end{pfsteps*}
\end{byCases}
\end{proof}

The Regularity Lemma expresses the intuition that the type assigned to an expression under a well-formed typing context is well-formed. 
\begin{lemma}[Regularity]\label{lemma:regularity-UP} All of the following hold:
\begin{enumerate}
\item If $\hastypeU{\Delta}{\Gamma}{e}{\tau}$ and $\istypeU{\Delta}{\tau_i}$ for each assumption $x_i : \tau_i$ in $\Gamma$ then $\istypeU{\Delta}{\tau}$.
\item If $\ruleType{\Delta}{\Gamma}{r}{\tau}{\tau'}$ and $\istypeU{\Delta}{\tau}$ and $\istypeU{\Delta}{\tau_i}$ for each assumption $x_i : \tau_i$ in $\Gamma$ then $\istypeU{\Delta}{\tau'}$.
\end{enumerate}
\end{lemma}
\begin{proof-sketch} By mutual rule induction on Rules (\ref{rules:hastypeUP}) and Rule (\ref{rule:ruleType}), Lemma \ref{lemma:substitution-UP} and Lemma \ref{lemma:pattern-regularity-UP}.
\end{proof-sketch}
\subsection{Structural Dynamics}\label{sec:dynamics-UP}
The \emph{structural dynamics of }$\miniVersePat$ is organized around judgements of the following form:
\[\begin{array}{ll}
\textbf{Judgement Form} & \textbf{Description}\\
\stepsU{e}{e'} & \text{$e$ transitions to $e'$}\\
\isvalU{e} & \text{$e$ is a value}\\
\matchfail{e} & \text{$e$ raises match failure}
\end{array}\]
We also define auxiliary judgements for \emph{iterated transition}, $\multistepU{e}{e'}$, and \emph{evaluation}, $\evalU{e}{e'}$.

\begin{definition}[Iterated Transition]\label{defn:iterated-transition-UP} Iterated transition, $\multistepU{e}{e'}$, is the reflexive, transitive closure of the transition judgement.\end{definition}

\begin{definition}[Evaluation]\label{defn:evaluation-UP}  $\evalU{e}{e'}$ iff $\multistepU{e}{e'}$ and $\isvalU{e'}$. \end{definition}

As in Sec. \ref{sec:dynamics-U}, our subsequent developments do not make mention of particular rules in the dynamics nor make use of the auxiliary judgements necessary for specifying the dynamics of the match operator (because TSMs operate statically), so we do not reproduce these here. Instead, it suffices to state the following conditions.

The Canonical Forms condition characterizes well-typed values. We assume an \emph{eager} (i.e. \emph{by-value}) formulation of the dynamics. This condition is identical to Condition \ref{condition:canonical-forms-U}.

\begin{condition}[Canonical Forms]\label{condition:canonical-forms-UP} If $\hastypeUC{e}{\tau}$ and $\isvalU{e}$ then:
\begin{enumerate}
\item If $\tau=\aparr{\tau_1}{\tau_2}$ then $e=\aelam{\tau_1}{x}{e'}$ and $\hastypeUCO{\Ghyp{x}{\tau_1}}{e'}{\tau_2}$.
\item If $\tau=\aall{t}{\tau'}$ then $e=\aetlam{t}{e'}$ and $\hastypeUCO{\Dhyp{t}}{e'}{\tau'}$.
\item If $\tau=\arec{t}{\tau'}$ then $e=\aefold{t}{\tau'}{e'}$ and $\hastypeUC{e'}{[\abop{rec}{t.\tau'}/t]\tau'}$ and $\isvalU{e'}$. 
\item If $\tau=\aprod{\labelset}{\mapschema{\tau}{i}{\labelset}}$ then $e=\aetpl{\labelset}{\mapschema{e}{i}{\labelset}}$ and $\hastypeUC{e_i}{\tau_i}$ and $\isvalU{e_i}$ for each $i \in \labelset$.
\item If $\tau=\asum{\labelset}{\mapschema{\tau}{i}{\labelset}}$ then for some label set $L'$ and label $\ell$, we have that $\labelset=\labelset', \ell$ and $\tau=\asum{\labelset', \ell}{\mapschema{\tau}{i}{\labelset'}; \mapitem{\ell}{\tau_\ell}}$ and $e=\aein{\labelset', \ell}{\ell}{\mapschema{\tau}{i}{\labelset'}; \ell \hookrightarrow \tau_\ell}{e'}$ and $\hastypeUC{e'}{\tau_\ell}$ and $\isvalU{e'}$.\end{enumerate}\end{condition}

The Preservation and Progress conditions (together, the {Type Safety} condition) ensure that the statics of the inner core and the structural dynamics are coherent, i.e. evaluation of a well-typed expression preserves typing and does not get stuck.
\begin{condition}[Preservation]\label{condition:preservation-UP} If $\hastypeUC{e}{\tau}$ and $\stepsU{e}{e'}$ then $\hastypeUC{e'}{\tau}$. \end{condition}
\begin{condition}[Progress]\label{condition:progress-UP} If $\hastypeUC{e}{\tau}$ then either $\isvalU{e}$ or $\matchfail{e}$ or there exists an $e'$ such that $\stepsU{e}{e'}$. \end{condition}
%\noindent
%Condition \ref{condition:preservation-UP} is identical to Condition \ref{condition:preservation-U}, while Condition \ref{condition:progress-UP} modifies Condition \ref{condition:progress-U} to allow for match failure. 

We do not specify exhaustiveness and redundancy properties here, because these can be checked post-expansion and are thus not relevant to our subsequent developments (but see \cite{pfple1} for a formal account).


\subsection{Typed Expansion}
Programmers author unexpanded expressions, $\ue$, which can contain unexpanded rules, $\urv$, and unexpanded patterns, $\upv$. The syntax of unexpanded terms is specified by the table in Figure \ref{fig:UP-unexpanded-terms}. 

Unexpanded terms are simultaneously typed and expanded according to the \emph{typed expansion judgements}:
\[\begin{array}{ll}
\textbf{Judgement Form} & \textbf{Description}\\
\expandsUP{\Delta}{\Gamma}{\Sigma}{\Phi}{\ue}{e}{\tau} & \text{$\ue$ expands to $e$ at type $\tau$ assuming $\Delta$ and $\Gamma$ under ueTSM}\\
& \text{environment $\Sigma$ and upTSM environment $\Phi$}\\
\ruleExpands{\Delta}{\Gamma}{\Sigma}{\Phi}{\urv}{r}{\tau}{\tau'} & \text{$\urv$ expands to $r$ taking values of type $\tau$ to values of type $\tau'$}\\
& \text{assuming $\Delta$ and $\Gamma$ under TSM environments $\Sigma$ and $\Phi$}\\
\patExpands{\pctx}{\Phi}{\upv}{p}{\tau} & \text{$\upv$ expands to $p$ at type $\tau$ generating hypotheses $\pctx$ under}\\
& \text{upTSM environment $\Phi$}
\end{array}\]
The \emph{typed expression expansion} judgement, $\expandsUP{\Delta}{\Gamma}{\Sigma}{\Phi}{\ue}{e}{\tau}$, is inductively defined by the following rules:
\begin{figure}
\hspace{-8px}$\arraycolsep=4pt\begin{array}{lllllll}
\textbf{Sort} & & & \textbf{Uniform Form} & \textbf{Stylized Form} & \textbf{Description}\\
\mathsf{UExp} & \ue & ::= & x & x & \text{variable}\\
&&& \aulam{\tau}{x}{\ue} & \lam{x}{\tau}{\ue} & \text{abstraction}\\
&&& \auap{\ue}{\ue} & \ap{\ue}{\ue} & \text{application}\\
&&& \autlam{t}{\ue} & \Lam{t}{\ue} & \text{type abstraction}\\
&&& \autap{\ue}{\tau} & \App{\ue}{\tau} & \text{type application}\\
&&& \aufold{t}{\tau}{\ue} & \fold{\ue} & \text{fold}\\
&&& \auunfold{\ue} & \unfold{\ue} & \text{unfold}\\
&&& \autpl{\labelset}{\mapschema{\ue}{i}{\labelset}} & \tpl{\mapschema{\ue}{i}{\labelset}} & \text{labeled tuple}\\
&&& \aupr{\ell}{\ue} & \prj{\ue}{\ell} & \text{projection}\\
&&& \auin{\labelset}{\ell}{\mapschema{\tau}{i}{\labelset}}{\ue} & \inj{\ell}{\ue} & \text{injection}\\
&&& \aumatchwith{n}{\ue}{\seqschemaX{\urv}} & \matchwith{\ue}{\seqschemaX{\urv}} & \text{match}\\
&&& \audefuetsm{\tau}{\ue}{\tsmv}{\ue} & \texttt{syntax}~a~\texttt{at}~\tau~\texttt{for} & \text{ueTSM definition}\\
&&&                                    & \texttt{expressions}~\{\ue\}~\texttt{in}~\ue\\
&&& \autsmap{b}{\tsmv} & \utsmap{\tsmv}{b} & \text{ueTSM application}\\
&&& \audefuptsm{\tau}{\ue}{\tsmv}{\ue} & \texttt{syntax}~a~\texttt{at}~\tau~\texttt{for} & \text{upTSM definition}\\
&&&                                    & \texttt{patterns}~\{\ue\}~\texttt{in}~\ue\\
\mathsf{URule} & \urv & ::= & \aumatchrule{n}{\upv}{\seqschemaX{x}}{\ue} & \matchrule{\upv}{\ue} & \text{match rule}\\
\mathsf{UPat} & \upv & ::= & x & x & \text{variable pattern}\\
&&& \auwildp & \wildp & \text{wildcard pattern}\\
&&& \autplp{\labelset}{\mapschema{\upv}{i}{\labelset}} & \tplp{\mapschema{\upv}{i}{\labelset}} & \text{labeled tuple pattern}\\
&&& \auinjp{\ell}{\upv} & \injp{\ell}{\upv} & \text{injection pattern}\\
&&& \auapuptsm{b}{a} & \utsmap{\tsmv}{b} & \text{upTSM application}
\end{array}$
\caption[Syntax of unexpanded expressions, rules and patterns in $\miniVersePat$.]{Abstract syntax of unexpanded expressions, rules and patterns in $\miniVersePat$. Metavariable $\tsmv$ ranges over TSM names and $b$ ranges over literal bodies. Literal bodies might contain unparsed terms, so variable renaming and substitution cannot be defined in the usual manner over unexpanded terms (i.e. they should be thought of as partially parsed abstract syntax trees, not abstract binding trees.)}
\label{fig:UP-unexpanded-terms}
\end{figure}
\begin{subequations}\label{rules:expandsUP}
\begin{equation}\label{rule:expandsUP-var}
  \inferrule{ }{\expandsUP{\Delta}{\Gamma, x : \tau}{\Sigma}{\Phi}{x}{x}{\tau}}
\end{equation}
\begin{equation}\label{rule:expandsUP-lam}
  \inferrule{
    \istypeU{\Delta}{\tau}\\
    \expandsUP{\Delta}{\Gamma, x : \tau}{\Sigma}{\Phi}{\ue}{e}{\tau'}
  }{\expandsUPX{\aulam{\tau}{x}{\ue}}{\aelam{\tau}{x}{e}}{\aparr{\tau}{\tau'}}}
\end{equation}
\begin{equation}\label{rule:expandsUP-ap}
  \inferrule{
    \expandsUPX{\ue_1}{e_1}{\aparr{\tau}{\tau'}}\\
    \expandsUPX{\ue_2}{e_2}{\tau}
  }{
    \expandsUPX{\auap{\ue_1}{\ue_2}}{\aeap{e_1}{e_2}}{\tau'}
  }
\end{equation}
\begin{equation}\label{rule:expandsUP-tlam}
  \inferrule{
    \expandsUP{\Delta, \Dhyp{t}}{\Gamma}{\Sigma}{\Phi}{\ue}{e}{\tau}
  }{
    \expandsUPX{\autlam{t}{\ue}}{\aetlam{t}{e}}{\aall{t}{\tau}}
  }
\end{equation}
\begin{equation}\label{rule:expandsUP-tap}
  \inferrule{
    \expandsUPX{\ue}{e}{\aall{t}{\tau}}\\
    \istypeU{\Delta}{\tau'}
  }{
    \expandsUPX{\autap{\ue}{\tau'}}{\aetap{e}{\tau'}}{[\tau'/t]\tau}
  }
\end{equation}
\begin{equation}\label{rule:expandsUP-fold}
  \inferrule{
    \istypeU{\Delta, \Dhyp{t}}{\tau}\\
    \expandsUPX{\ue}{e}{[\arec{t}{\tau}/t]\tau}
  }{
    \expandsUPX{\aufold{t}{\tau}{\ue}}{\aefold{t}{\tau}{e}}{\arec{t}{\tau}}
  }
\end{equation}
\begin{equation}\label{rule:expandsUP-unfold}
  \inferrule{
    \expandsUPX{\ue}{e}{\arec{t}{\tau}}
  }{
    \expandsUPX{\auunfold{\ue}}{\aeunfold{e}}{[\arec{t}{\tau}/t]\tau}
  }
\end{equation}
\begin{equation}\label{rule:expandsUP-tpl}
  \inferrule{
    \{\expandsUPX{\ue_i}{e_i}{\tau_i}\}_{i \in \labelset}
  }{
    \expandsUPX{\autpl{\labelset}{\mapschema{\ue}{i}{\labelset}}}{\aetpl{\labelset}{\mapschema{e}{i}{\labelset}}}{\aprod{\labelset}{\mapschema{\tau}{i}{\labelset}}}
  }
\end{equation}
\begin{equation}\label{rule:expandsUP-pr}
  \inferrule{
    \expandsUPX{\ue}{e}{\aprod{\labelset, \ell}{\mapschema{\tau}{i}{\labelset}; \mapitem{\ell}{\tau}}}
  }{
    \expandsUPX{\aupr{\ell}{\ue}}{\aepr{\ell}{e}}{\tau}
  }
\end{equation}
\begin{equation}\label{rule:expandsUP-in}
  \inferrule{
    \{\istypeU{\Delta}{\tau_i}\}_{i \in \labelset}\\
    \istypeU{\Delta}{\tau}\\
    \expandsUPX{\ue}{e}{\tau}
  }{
    \left\{\shortstack{$\Delta~\Gamma \vdash_{\Sigma;\,\Phi} \auin{\labelset, \ell}{\ell}{\mapschema{\tau}{i}{\labelset}; \mapitem{\ell}{\tau}}{\ue}$\\$\leadsto$\\$\aein{\labelset, \ell}{\ell}{\mapschema{\tau}{i}{\labelset}; \mapitem{\ell}{\tau}}{e} : \asum{\labelset, \ell}{\mapschema{\tau}{i}{\labelset}; \mapitem{\ell}{\tau}}$\vspace{-1.2em}}\right\}
  }
\end{equation}
\begin{equation}\label{rule:expandsUP-match}
\inferrule{
  \expandsUPX{\ue}{e}{\tau}\\
  \{\ruleExpands{\Delta}{\Gamma}{\Sigma}{\Phi}{\urv_i}{r_i}{\tau}{\tau'}\}_{1 \leq i < n}
}{\expandsUPX{\aumatchwith{n}{\ue}{\seqschemaX{\urv}}}{\aematchwith{n}{e}{\seqschemaX{r}}}{\tau'}}
\end{equation}  
\begin{equation}\label{rule:expandsUP-syntax}
\inferrule{
  \istypeU{\Delta}{\tau}\\
  \expandsUP{\emptyset}{\emptyset}{\emptyset}{\emptyset}{\ueparse}{\eparse}{\aparr{\tBody}{\tParseResultExp}}\\\\
  a \notin \domof{\Sigma}\\
  \expandsUP{\Delta}{\Gamma}{\Sigma, \xuetsmbnd{a}{\tau}{\eparse}}{\Phi}{\ue}{e}{\tau'}
}{
  \expandsUPX{\audefuetsm{\tau}{\ueparse}{a}{\ue}}{e}{\tau'}
}
\end{equation}
\begin{equation}\label{rule:expandsUP-tsmap}
\inferrule{
  \encodeBody{b}{\ebody}\\
  \evalU{\ap{\eparse}{\ebody}}{\inj{\lbltxt{Success}}{\ecand}}\\
  \decodeCondE{\ecand}{\ce}\\\\
  \cvalidE{\emptyset}{\emptyset}{\esceneUP{\Delta}{\Gamma}{\Sigma, \xuetsmbnd{a}{\tau}{\eparse}}{\Phi}{b}}{\ce}{e}{\tau}
}{
  \expandsUP{\Delta}{\Gamma}{\Sigma, \xuetsmbnd{a}{\tau}{\eparse}}{\Phi}{\autsmap{b}{a}}{e}{\tau}
}
\end{equation}
\begin{equation}\label{rule:expandsUP-defuptsm}
\inferrule{
  \istypeU{\Delta}{\tau}\\
  \expandsUP{\emptyset}{\emptyset}{\emptyset}{\emptyset}{\ueparse}{\eparse}{\aparr{\tBody}{\tParseResultPat}}\\\\
  a \notin \domof{\Phi}\\
  \expandsUP{\Delta}{\Gamma}{\Sigma}{\Phi, \xuptsmbnd{a}{\tau}{\eparse}}{\ue}{e}{\tau'} 
}{
  \expandsUP{\Delta}{\Gamma}{\Sigma}{\Phi}{\audefuptsm{\tau}{\ueparse}{a}{\ue}}{e}{\tau'}
}
\end{equation}
\end{subequations}
The \emph{typed rule expansion} judgement, $\ruleExpands{\Delta}{\Gamma}{\Sigma}{\Phi}{\urv}{r}{\tau}{\tau'}$, is defined mutually inductively with Rules (\ref{rules:expandsUP}) by the following rule:
\begin{equation}\label{rule:ruleExpands}
\inferrule{
  \patExpands{\pctx}{\Phi}{\upv}{p}{\tau}\\
  \domof{\pctx} = \seqschemaX{x}\\
  \expandsUP{\Delta}{\Gcons{\Gamma}{\pctx}}{\Sigma}{\Phi}{\ue}{e}{\tau'} 
}{
  \ruleExpands{\Delta}{\Gamma}{\Sigma}{\Phi}{\aumatchrule{n}{\upv}{\seqschemaX{x}}{\ue}}{\aematchrule{n}{p}{\seqschemaX{x}}{e}}{\tau}{\tau'}
}
\end{equation}
The \emph{typed pattern expansion} judgement, $\patExpands{\pctx}{\Phi}{\upv}{p}{\tau}$, is inductively defined by the following rules:
\begin{subequations}\label{rules:patExpands}
\begin{equation}\label{rule:patExpands-var}
\inferrule{ }{
  \patExpands{\Ghyp{x}{\tau}}{\Phi}{x}{x}{\tau}
}
\end{equation}
\begin{equation}\label{rule:patExpands-wild}
\inferrule{ }{
  \patExpands{}{\Phi}{\auwildp}{\aewildp}{\tau}
}
\end{equation}
\begin{equation}\label{rule:patExpands-tpl}
\inferrule{
  \{\patExpands{\pctx_i}{\Phi}{\upv_i}{p_i}{\tau_i}\}_{i \in \labelset}\\
  \cap_{i \in \labelset} \domof{\pctx_i} = \emptyset
}{
  \patExpands{\Gconsi{i \in \labelset}{\pctx_i}}{\Phi}{
    \autplp{\labelset}{\mapschema{\upv}{i}{\labelset}}
  }{
    \aetplp{\labelset}{\mapschema{p}{i}{\labelset}}
  }{
    \aprod{\labelset}{\mapschema{\tau}{i}{\labelset}}
  } %{\autplp{\labelset}{\mapschema{\upv}{i}{\labelset}}}{\aetplp{\labelset}{\mapschema}{p}{i}{\labelset}}{...}
}
\end{equation}
\begin{equation}\label{rule:patExpands-in}
\inferrule{
  \patExpands{\pctx}{\Phi}{\upv}{p}{\tau}
}{
  \patExpands{\pctx}{\Phi}{\auinjp{\ell}{\upv}}{\aeinjp{\ell}{p}}{\asum{\labelset, \ell}{\mapschema{\tau}{i}{\labelset}; \mapitem{\ell}{\tau}}}
}
\end{equation}
\begin{equation}\label{rule:patExpands-apuptsm}
\inferrule{
  \encodeBody{b}{\ebody}\\
  \evalU{\ap{\eparse}{\ebody}}{\inj{\lbltxt{Success}}{\ecand}}\\
  \decodeCEPat{\ecand}{\cpv}\\\\
  \cvalidP{\pctx}{\pscene{\Phi, \xuptsmbnd{a}{\tau}{\eparse}}{b}}{\cpv}{p}{\tau}
}{
  \patExpands{\pctx}{\Phi, \xuptsmbnd{a}{\tau}{\eparse}}{\auapuptsm{b}{a}}{p}{\tau}
}
\end{equation}
\end{subequations}
Most of the rules given above correspond to a typing rule in the statics of the inner core. In particular:
\begin{itemize}
\item Rules (\ref{rule:expandsUP-var}) through (\ref{rule:expandsUP-match}) correspond to Rules (\ref{rules:hastypeUP}); 
\item Rule (\ref{rule:ruleExpands}) corresponds to Rule (\ref{rule:ruleType}); and
\item Rules (\ref{rule:patExpands-var}) through (\ref{rule:patExpands-in}) correspond to Rules (\ref{rules:patType}).
\end{itemize}
In each of these rules, the unexpanded and expanded term forms correspond and when a typing judgement appears as the premise in a typing rule, the corresponding typed expansion judgement appears as a premise in the corresponding typed expansion rule. All other premises are carried over directly. These rules do not inspect or extend the ueTSM environment, $\Sigma$, or the upTSM environment, $\Phi$.

Rules (\ref{rule:expandsUP-syntax}) and (\ref{rule:expandsUP-tsmap}) specify the semantics of ueTSM definition and application, which we covered in Sections \ref{sec:U-uetsm-definition} and \ref{sec:U-uetsm-application}, respectively. Rules (\ref{rule:expandsUP-syntax}) and (\ref{rule:expandsUP-tsmap}) differ from Rules (\ref{rule:expandsU-syntax}) and (\ref{rule:expandsU-tsmap}) only in that the upTSM environment, $\Phi$, is threaded through them opaquely. We assume definitions and conditions identical to those given in Secs. \ref{sec:U-uetsm-definition} and \ref{sec:U-uetsm-application} without restating them.

This leaves Rules (\ref{rule:expandsUP-defuptsm}) and (\ref{rule:patExpands-apuptsm}), which specify upTSM definition and application, respectively. These are the topics of the next two subsections.
\subsection{upTSM Definition}
The \emph{upTSM definition form}:
\[\texttt{syntax}~a~\texttt{at}~\tau~\texttt{for~patterns}~\{\ueparse\}~\texttt{in}~\ue\]
allows the programmer to introduce a upTSM named $a$ at type $\tau$ with unexpanded parse function $\ueparse$ into the upTSM environment of $\ue$. The uniform form corresponding to this stylized form is $\audefuptsm{\tau}{\ueparse}{a}{\ue}$. Rule (\ref{rule:expandsUP-defuptsm}), which governs upTSM definitions, is similar to Rule (\ref{rule:expandsUP-syntax}), which governs ueTSM definitions. Its premises can be understood as follows, in order:
\begin{enumerate}
\item The first premise ensures that the type annotation specifies a well-formed type, $\tau$.
\item The second premise types and expands the unexpanded parse function, producing the expanded parse function, $\eparse$. The parse function must be of type
\[\aparr{\tBody}{\tParseResultPat}\]
where $\tParseResultPat$ abbreviates the following labeled sum type:\[\tParseResultPat \triangleq [\mapitem{\lbltxt{Success}}{\tCEPat}, \mapitem{\lbltxt{ParseError}}{\prodt{}}]\]
and $\tBody$ and $\tCEPat$ abbreviate types that we will characterize below.
\item The third premise of Rule (\ref{rule:expandsUP-defuptsm}) checks that there is not already a upTSM named $a$ in the upTSM environment, $\Phi$.

upTSM environments are finite mappings from TSM names, $a$, to \emph{expanded upTSM definitions}, $\xuptsmdef{\tau}{\ueparse}$, where $\tau$ is the upTSM's type annotation and $\eparse$ is the upTSM's expanded parse function. The \emph{upTSM environment formation judgement}, $\uptsmenv{\Delta}{\Phi}$, ensures that the type annotations in $\Phi$ are well-formed assuming $\Delta$ and the parse functions in $\Phi$ are closed and of type $\aparr{\tBody}{\tParseResultPat}$.
\[\begin{array}{ll}
\textbf{Judgement Form} & \textbf{Description}\\
\uptsmenv{\Delta}{\Phi} & \text{upTSM environment $\Phi$ is well-formed assuming $\Delta$}\end{array}\]
This judgement is inductively defined by the following rules:
\begin{subequations}\label{rules:uptsmenv-U}
\begin{equation}\label{rule:uptsmenv-empty}
\inferrule{ }{\uptsmenv{\Delta}{\emptyset}}
\end{equation}
\begin{equation}\label{rule:uptsmenv-ext}
\inferrule{
  \uptsmenv{\Delta}{\Phi}\\
  \istypeU{\Delta}{\tau}\\
  \hastypeU{\emptyset}{\emptyset}{\eparse}{\aparr{\tBody}{\tParseResultPat}}
}{
  \uptsmenv{\Delta}{\Sigma, \xuptsmbnd{a}{\tau}{\eparse}}
}
\end{equation}
\end{subequations}

\item 
The final premise of Rule (\ref{rule:expandsUP-defuptsm}) extends the upTSM environment with the newly determined expanded ueTSM definition and proceeds to assign a type, $\tau'$, and expansion, $e$, to $\ue$.
\end{enumerate}
The conclusion of Rule (\ref{rule:expandsUP-defuptsm}) gives $\tau'$ and $e$ as the type and translation for the expression as a whole.
\subsection{upTSM Application}
The form for applying a upTSM named $a$ to a literal form with literal body $b$ is:
\[
\utsmap{a}{b}
\] 
This stylized form is identical to the stylized form for ueTSM application, differing in that appears within the syntax of unexpanded patterns, $\upv$, rather than unexpanded expressions, $\ue$. %It uses forward slashes as delimiters, though stylized variants of any of the literal forms specified in Figure \ref{fig:literal-forms} would be straightforward to add to the syntax table in Figure \ref{fig:UP-unexpanded-terms} (we omit them for simplicity). 

The uniform form corresponding to this stylized form is $\auapuptsm{b}{a}$, i.e. there is an operator $\texttt{uapuptsm}[b]$ for each literal body $b$ indexed by the TSM name $a$ and taking no arguments.

Rule (\ref{rule:patExpands-apuptsm}), which governs upTSM application, is similar to Rule (\ref{rule:expandsUP-tsmap}), which governs ueTSM application. Its premises can be understood as follows, in order:
\begin{enumerate}
\item The \emph{body encoding judgement} $\encodeBody{b}{\ebody}$ specifies a mapping from the literal body, $b$, to an expanded value, $\ebody$, of type $\tBody$. An inverse mapping is specified by the \emph{body decoding judgement} $\decodeBody{\ebody}{b}$.
\[\begin{array}{ll}
\textbf{Judgement Form} & \textbf{Description}\\
\encodeBody{b}{e} & \text{$b$ encodes to $e$}\\
\decodeBody{e}{b} & \text{$e$ decodes to $b$}
\end{array}\]
As with the corresponding judgements described in Sec. \ref{sec:U-uetsm-application}, we do not pick a particular definition of $\tBody$. Instead, we state the following conditions, which establish an isomorphism between literal bodies and values of type $\tBody$.
\begin{condition}[Body Encoding and Decoding] All of the following hold:
\begin{enumerate}
\item For every literal body $b$, we have that $\encodeBody{b}{\ebody}$ and $\hastypeUC{\ebody}{\tBody}$ and $\isvalU{\ebody}$.
\item If $\hastypeUC{\ebody}{\tBody}$ and $\isvalU{\ebody}$ then $\decodeBody{\ebody}{b}$ for some $b$.
\item If $\encodeBody{b}{\ebody}$ then $\decodeBody{\ebody}{b}$.
\item If $\hastypeUC{\ebody}{\tBody}$ and $\isvalU{\ebody}$ and $\decodeBody{\ebody}{b}$ then $\encodeBody{b}{\ebody}$. 
\item If $\encodeBody{b}{\ebody}$ and $\encodeBody{b}{\ebody'}$ then $\ebody = \ebody'$.
\item If $\hastypeUC{\ebody}{\tBody}$ and $\isvalU{\ebody}$ and $\decodeBody{\ebody}{b}$ and $\decodeBody{\ebody}{b'}$ then $b=b'$.
\end{enumerate}
\end{condition}
\item The second premise applies the expanded parse function $\eparse$ associated with $a$ in the upTSM environment to $\ebody$. If parsing succeeds, i.e. a value of the (stylized) form $\inj{\lbltxt{Success}}{\ecand}$ results from evaluation, then $\ecand$ will be a value of type $\tCEPat$ (assuming a well-formed upTSM environment and transitive application of Assumption \ref{condition:preservation-UP}). We call $\ecand$ the \emph{encoding of the candidate expansion}.

If the parse function produces a value labeled $\lbltxt{ParseError}$, then typed expansion fails. No rule is necessary to handle this case. 

\item The judgement $\decodeCEPat{\ecand}{\cpv}$ decodes the encoding of the candidate expansion, i.e. it maps $\ecand$ onto a \emph{candidate expansion pattern}, $\cpv$. The inverse mapping is specified by the judgement $\encodeCEPat{\cpv}{\ecand}$. %These are analagous to the judgements for candidate expansion expression decoding and encoding defined in Sec. \ref{sec:U-uetsm-application}.
\[\begin{array}{ll}
\textbf{Judgement Form} & \textbf{Description}\\
%\decodeCondE{e}{\ce} & \text{$e$ decodes to $\ce$}\\
\decodeCEPat{p}{\cpv} & \text{$p$ decodes to $\cpv$}\\
%\encodeCondE{\ce}{e} & \text{$\ce$ encodes to $e$}\\
\encodeCEPat{\cpv}{e} & \text{$\cpv$ encodes to $p$}
\end{array}\]

Rather than defining $\tCEPat$ explicitly, and these judgements inductively against this definition, we state the following conditions, which establish an isomorphism between values of type $\tCEPat$ and candidate expansion patterns.
\begin{condition}[Candidate Expansion Pattern Encoding and Decoding] All of the following hold:
\begin{enumerate}
\item If $\hastypeUC{\ecand}{\tCEPat}$ and $\isvalU{\ecand}$ then $\decodeCEPat{\ecand}{\cpv}$ for some $\cpv$.
\item For every $\cpv$, we have $\encodeCEPat{\cpv}{\ecand}$ such that $\hastypeUC{\ecand}{\tCEPat}$ and $\isvalU{\ecand}$.
\item If $\hastypeUC{\ecand}{\tCEPat}$ and $\isvalU{\ecand}$ and $\decodeCEPat{\ecand}{\cpv}$ then $\encodeCEPat{\cpv}{\ecand}$.
\item If $\encodeCEPat{\cpv}{\ecand}$ then $\decodeCEPat{\cpv}{\ecand}$.
\item If $\hastypeUC{\ecand}{\tCEPat}$ and $\isvalU{\ecand}$ and $\decodeCEPat{\ecand}{\cpv}$ and $\decodeCEPat{\ecand}{\cpv'}$ then $\cpv=\cpv'$.
\item If $\encodeCEPat{\cpv}{\ecand}$ and $\encodeCEPat{\cpv}{\ecand'}$ then $\ecand=\ecand'$.
\end{enumerate}
\end{condition}

\item The final premise of Rule (\ref{rule:patExpands-apuptsm}) \emph{validates} the candidate expansion, $\cpv$, and simultaneously generates the \emph{final expansion} $p$ and gathers the hypotheses $\pctx$ generated by the spliced subpatterns that appear within the literal body. This is the topic of the next subsection.
\end{enumerate}

\begin{figure}[p]
\hspace{-5px}$\begin{array}{lllllll}
\textbf{Sort} & & & \textbf{Uniform Form} & \textbf{Stylized Form} & \textbf{Description}\\
\mathsf{CETyp} & \ctau & ::= & t & t & \text{variable}\\
&&& \aceparr{\ctau}{\ctau} & \parr{\ctau}{\ctau} & \text{partial function}\\
&&& \aceall{t}{\ctau} & \forallt{t}{\ctau} & \text{polymorphic}\\
&&& \acerec{t}{\ctau} & \rect{t}{\ctau} & \text{recursive}\\
&&& \aceprod{\labelset}{\mapschema{\ctau}{i}{\labelset}} & \prodt{\mapschema{\ctau}{i}{\labelset}} & \text{labeled product}\\
&&& \acesum{\labelset}{\mapschema{\ctau}{i}{\labelset}} & \sumt{\mapschema{\ctau}{i}{\labelset}} & \text{labeled sum}\\
&&& \acesplicedt{m}{n} & \splicedt{m}{n} & \text{spliced}\\
\mathsf{CEExp} & \ce & ::= & x & x & \text{variable}\\
&&& \acelam{\ctau}{x}{\ce} & \lam{x}{\ctau}{\ce} & \text{abstraction}\\
&&& \aceap{\ce}{\ce} & \ap{\ce}{\ce} & \text{application}\\
&&& \acetlam{t}{\ce} & \Lam{t}{\ce} & \text{type abstraction}\\
&&& \acetap{\ce}{\ctau} & \App{\ce}{\ctau} & \text{type application}\\
&&& \acefold{t}{\ctau}{\ce} & \fold{\ce} & \text{fold}\\
&&& \aceunfold{\ce} & \unfold{\ce} & \text{unfold}\\
&&& \acetpl{\labelset}{\mapschema{\ce}{i}{\labelset}} & \tpl{\mapschema{\ce}{i}{\labelset}} & \text{labeled tuple}\\
&&& \acepr{\ell}{\ce} & \prj{\ce}{\ell} & \text{projection}\\
&&& \acein{\labelset}{\ell}{\mapschema{\ctau}{i}{\labelset}}{\ce} & \inj{\ell}{\ce} & \text{injection}\\
&&& \acematchwith{n}{\ce}{\seqschemaX{\crv}} & \matchwith{\ce}{\seqschemaX{\crv}} & \text{match}\\
&&& \acesplicede{m}{n} & \splicede{m}{n} & \text{spliced}\\
\mathsf{CERule} & \crv & ::= & \acematchrule{n}{p}{\seqschemaX{x}}{\ce} & \matchrule{p}{\ce} & \text{rule}\\
\mathsf{CEPat} & \cpv & ::= & \acewildp & \wildp & \text{wildcard pattern}\\
%&&& \aefoldp{p} & \foldp{p} & \text{fold pattern}\\
&&& \acetplp{\labelset}{\mapschema{\cpv}{i}{\labelset}} & \tplp{\mapschema{\cpv}{i}{\labelset}} & \text{labeled tuple pattern}\\
&&& \aceinjp{\ell}{\cpv} & \injp{\ell}{\cpv} & \text{injection pattern}\\
&&& \acesplicedp{m}{n} & \splicedp{m}{n} & \text{spliced}
\end{array}$
\caption[Syntax of candidate expansion types, expressions, rules and patterns in $\miniVerseUE$]{Abstract syntax of candidate expansion types, expressions, rules and patterns in $\miniVerseUE$.}
\label{fig:UP-candidate-terms}
\end{figure}

\subsection{Candidate Expansion Validation and Final Expansion}\label{sec:ce-validation-UP}
The syntax of candidate expansion types, expressions, rules and patterns is specified in Figure \ref{fig:UP-candidate-terms}.

The \emph{candidate expansion validation judgements} validate candidate expansion terms and simultaneously generate their final expansions.
\[\begin{array}{ll}
\textbf{Judgement Form} & \textbf{Description}\\
\cvalidT{\Delta}{\tscenev}{\ctau}{\tau} & \text{Candidate expansion type $\ctau$ expands to $\tau$ assuming $\Delta$ and }\\
& \text{expression splicing scene $\escenev$.}\\
\cvalidE{\Delta}{\Gamma}{\escenev}{\ce}{e}{\tau} & \text{Candidate expansion expression $\ce$ expands to $e$ at type $\tau$}\\
& \text{assuming $\Delta$ and $\Gamma$ and expression splicing scene $\escenev$.}\\
\cvalidR{\Delta}{\Gamma}{\escenev}{\crv}{r}{\tau}{\tau'} & \text{Candidate expansion rule $\crv$ expands to $r$ taking values of}\\
& \text{type $\tau$ to values of type $\tau'$ assuming $\Delta$ and $\Gamma$ and with}\\
& \text{expression splicing scene $\escenev$.}\\
\cvalidP{\pctx}{\pscenev}{\cpv}{p}{\tau} & \text{Candidate expansion pattern $\cpv$ expands to $p$ at type $\tau$}\\
& \text{with pattern splicing scene $\pscenev$.}
\end{array}\]
\emph{Expression splicing scenes}, $\escenev$, are of the form $\esceneUP{\Delta}{\Gamma}{\Sigma}{\Phi}{b}$. They consist of the type formation context, $\Delta$, the typing context, $\Gamma$, the ueTSM environment, $\Sigma$, the upTSM environment, $\Phi$, and the literal body, $b$, from the ueTSM application site (cf. Rule (\ref{rule:expandsU-tsmap})). \emph{Type splicing scenes}, $\tscenev$, are of the form $\tsceneUP{\Delta}{b}$. We write $\tsfrom{\escenev}$ for the type splicing scene constructed from $\escenev$:
\[\tsfrom{\esceneUP{\Delta}{\Gamma}{\Sigma}{\Phi}{b}} = \tsceneUP{\Delta}{b}\]
\emph{Pattern splicing scenes}, $\pscenev$, are of the form $\pscene{\Phi}{b}$, and consist of the upTSM environment, $\Phi$, and the literal body, $b$, from the upTSM application site (cf. Rule (\ref{rule:patExpands-apuptsm})).

\subsubsection{Candidate Expansion Type Validation}
The \emph{candidate type validation judgement}, $\cvalidT{\Delta}{\tscenev}{\ctau}{\tau}$, is inductively defined by Rules (\ref{rules:cvalidT-U}), given in Sec. \ref{sec:ce-validation-U}.

\subsubsection{Candidate Expansion Expression and Rule Validation}
The \emph{candidate expression validation judgement}, $\cvalidE{\Delta}{\escenev}{\ce}{e}{\tau}$, is inductively defined by the following rules:
\begin{subequations}\label{rules:cvalidE-UP}
\begin{equation}\label{rule:cvalidE-UP-var}
\inferrule{ }{
  \cvalidE{\Delta}{\Gamma, \Ghyp{x}{\tau}}{\escenev}{x}{x}{\tau}
}
\end{equation}
\begin{equation}\label{rule:cvalidE-UP-lam}
\inferrule{
  \cvalidT{\Delta}{\tsfrom{\escenev}}{\ctau}{\tau}\\
  \cvalidE{\Delta}{\Gamma, \Ghyp{x}{\tau}}{\escenev}{\ce}{e}{\tau'}
}{
  \cvalidE{\Delta}{\Gamma}{\escenev}{\acelam{\ctau}{x}{\ce}}{\aelam{\tau}{x}{e}}{\aparr{\tau}{\tau'}}
}
\end{equation}
\begin{equation}\label{rule:cvalidE-UP-ap}
  \inferrule{
    \cvalidE{\Delta}{\Gamma}{\escenev}{\ce_1}{e_1}{\aparr{\tau}{\tau'}}\\
    \cvalidE{\Delta}{\Gamma}{\escenev}{\ce_2}{e_2}{\tau}
  }{
    \cvalidE{\Delta}{\Gamma}{\escenev}{\aceap{\ce_1}{\ce_2}}{\aeap{e_1}{e_2}}{\tau'}
  }
\end{equation}
\begin{equation}\label{rule:cvalidE-UP-tlam}
  \inferrule{
    \cvalidE{\Delta, \Dhyp{t}}{\Gamma}{\escenev}{\ce}{e}{\tau}
  }{
    \cvalidEX{\acetlam{t}{\ce}}{\aetlam{t}{e}}{\aall{t}{\tau}}
  }
\end{equation}
\begin{equation}\label{rule:cvalidE-UP-tap}
  \inferrule{
    \cvalidEX{\ce}{e}{\aall{t}{\tau}}\\
    \cvalidT{\Delta}{\tsfrom{\escenev}}{\ctau'}{\tau'}
  }{
    \cvalidEX{\acetap{\ce}{\ctau'}}{\aetap{e}{\tau'}}{[\tau'/t]\tau}
  }
\end{equation}
\begin{equation}\label{rule:cvalidE-UP-fold}
  \inferrule{
    \cvalidT{\Delta, \Dhyp{t}}{\escenev}{\ctau}{\tau}\\
    \cvalidEX{\ce}{e}{[\arec{t}{\tau}/t]\tau}
  }{
    \cvalidEX{\acefold{t}{\ctau}{\ce}}{\aefold{t}{\tau}{e}}{\arec{t}{\tau}}
  }
\end{equation}
\begin{equation}\label{rule:cvalidE-UP-unfold}
  \inferrule{
    \cvalidEX{\ce}{e}{\arec{t}{\tau}}
  }{
    \cvalidEX{\aceunfold{\ce}}{\aeunfold{e}}{[\arec{t}{\tau}/t]\tau}
  }
\end{equation}
\begin{equation}\label{rule:cvalidE-UP-tpl}
  \inferrule{
    \{\cvalidEX{\ce_i}{e_i}{\tau_i}\}_{i \in \labelset}
  }{
    \cvalidEX{\acetpl{\labelset}{\mapschema{\ce}{i}{\labelset}}}{\aetpl{\labelset}{\mapschema{e}{i}{\labelset}}}{\aprod{\labelset}{\mapschema{\tau}{i}{\labelset}}}
  }
\end{equation}
\begin{equation}\label{rule:cvalidE-UP-pr}
  \inferrule{
    \cvalidEX{\ce}{e}{\aprod{\labelset, \ell}{\mapschema{\tau}{i}{\labelset}; \mapitem{\ell}{\tau}}}
  }{
    \cvalidEX{\acepr{\ell}{\ce}}{\aepr{\ell}{e}}{\tau}
  }
\end{equation}
\begin{equation}\label{rule:cvalidE-UP-in}
  \inferrule{
    \{\cvalidT{\Delta}{\tsfrom{\escenev}}{\ctau_i}{\tau_i}\}_{i \in \labelset}\\
    \cvalidT{\Delta}{\tsfrom{\escenev}}{\ctau}{\tau}\\
    \cvalidEX{\ce}{e}{\tau}
  }{
    \left\{\shortstack{$\Delta~\Gamma \vdash_\Sigma \acein{\labelset, \ell}{\ell}{\mapschema{\ctau}{i}{\labelset}; \mapitem{\ell}{\ctau}}{\ce}$\\$\leadsto$\\$\aein{\labelset, \ell}{\ell}{\mapschema{\tau}{i}{\labelset}; \mapitem{\ell}{\tau}}{e} : \asum{\labelset, \ell}{\mapschema{\tau}{i}{\labelset}; \mapitem{\ell}{\tau}}$\vspace{-1.2em}}\right\}
  }
\end{equation}
\begin{equation}\label{rule:cvalidE-UP-case}
  \inferrule{
    \cvalidEX{\ce}{e}{\asum{\labelset}{\mapschema{\tau}{i}{\labelset}}}\\
    \{\cvalidE{\Delta}{\Gamma, \Ghyp{x_i}{\tau_i}}{\escenev}{\ue_i}{e_i}{\tau}\}_{i \in \labelset}
  }{
    \cvalidEX{\acecase{\labelset}{\ce}{\mapschemab{x}{\ce}{i}{\labelset}}}{\aecase{\labelset}{e}{\mapschemab{x}{e}{i}{\labelset}}}{\tau}
  }
\end{equation}
\begin{equation}\label{rule:cvalidE-UP-splicede}
\inferrule{
  \parseUExp{\bsubseq{b}{m}{n}}{\ue}\\\\
  \Delta \cap \Delta_S = \emptyset\\
  \domof{\Gamma} \cap \domof{\Gamma_S} = \emptyset\\
  \expandsU{\Delta_S}{\Gamma_S}{\Sigma_S}{\ue}{e}{\tau}
}{
  \cvalidE{\Delta}{\Gamma}{\esceneU{\Delta_S}{\Gamma_S}{\Sigma_S}{b}}{\acesplicede{m}{n}}{e}{\tau}
}
\end{equation}
\end{subequations}

\subsubsection{Candidate Expansion Pattern Validation}
...

\subsection{Metatheory}

\chapter{Parameterized TSMs}\label{sec:tsms-parameterized}
\section{Parameterized TSMs By Example}
\subsection{Type Parameters By Example}
\subsection{Module Parameters By Example}
\section{\texorpdfstring{$\miniVerseParam$}{miniVerseForall}}
\subsection{Signatures, Types and Expanded Expressions}
\subsection{Parameter Application and Deferred Substitution}
\subsection{Macro Expansion and Validation}
\subsection{Metatheory}


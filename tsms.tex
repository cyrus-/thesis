% !TEX root = omar-thesis.tex
\chapter{Unparameterized Expression TSMs}\label{chap:tsms}
We now introduce a new primitive -- the \textbf{typed syntax macro} (TSM). TSMs, like term-rewriting macros (Sec. \ref{sec:term-rewriting}), generate expansions. Unlike term-rewriting macros, TSMs are applied to unparsed \emph{generalized literal forms}, which gives them substantially more syntactic flexibility. This chapter focuses on perhaps the simplest manifestation of TSMs: \textbf{unparameterized expression TSMs}, which generate expressions of a single specified type. We will then consider TSMs that generate patterns in Chapter \ref{sec:pattern-tsms} and TSMs defined over parameterized families of types in Chapter \ref{sec:tsms-parameterized}.


%Like the term-rewriting macros just described, TSMs can be parameterized by modules, so they can be used to define syntax valid at any abstract type defined by a module satisfying a specified signature. As we will discuss in the remainder of this section, this addresses all of the problems brought up above, at moderate syntactic cost.

\section{Expression TSMs By Example}\label{sec:tsms-by-example}
%A typed syntax macro is invoked by applying it to a \emph{delimited form}, which can contain  arbitrary syntax in its \emph{body}.  
We begin in this section with a ``tutorial-style'' introduction to unparameterized expression TSMs in VerseML. In particular, we discuss a TSM for constructing values of the recursive labeled sum type \li{Rx} that was defined in Figure \ref{fig:datatype-rx}. We then formally specify unparameterized expression TSMs with a reduced calculus, $\miniVerseU$, in Sec. \ref{sec:tsms-minimal-formalism}. %We conclude in Sec. \ref{sec:uetsms-discussion} 

\subsection{Usage}\label{sec:uetsms-usage}
In the following  concrete VerseML expression, we apply a TSM, identified as \li{#\dolla#rx}, to a \emph{generalized literal form}, \li{/SURLA|T|G|CEURL/}:
\begin{lstlisting}[numbers=none,mathescape=|]
$rx /SURLA|T|G|CEURL/
\end{lstlisting}
Generalized literal forms are left unparsed when concrete expressions are first parsed. It is only during the subsequent \emph{typed expansion} process that the TSM parses the \emph{body} of the provided literal form, i.e. the characters in blue, to generate a \emph{candidate expansion}. The language then \emph{validates} the candidate expansion according to criteria that we will establish in Sec. \ref{sec:uetsms-validation}. If validation succeeds, the language generates the \emph{final expansion} (or more concisely, simply the \emph{expansion}) of the expression. The program will behave as if the expression above had been replaced by its expansion. The expansion of the expression above, written concretely, is:
\begin{lstlisting}[numbers=none]
Or(Str "SSTRAESTR", Or(Str "SSTRTESTR", Or(Str "SSTRGESTR", Str "SSTRCESTR")))
\end{lstlisting}
%The constructors above are those of the type \li{Rx} that was defined in Figure \ref{fig:datatype-rx}.

A number of literal forms, shown in Figure \ref{fig:literal-forms},  are available in VerseML's concrete syntax. Any literal form can be used with any TSM, e.g. we could have equivalently written the example above as \li{#\dolla#rx `SURLA|T|G|CEURL`} (in fact, this would be convenient if we had wanted to express a regex containing forward slashes but not backticks). TSMs have access only to the literal bodies. The concrete syntax of VerseML specifies where the literal bodies begin and end. Because the concrete syntax of the language is never extended directly by library providers, there cannot be syntactic conflicts.

 %The form does not directly determine the expansion. 

\begin{figure}
\begin{lstlisting}
'SURLbody cannot contain an apostropheEURL'
`SURLbody cannot contain a backtickEURL`
[SURLbody cannot contain unmatched square bracketsEURL]
{SURLbody cannot contain an unmatched curly braceEURL}
/SURLbody cannot contain a forward slashEURL/
\SURLbody cannot contain a backslashEURL\
SURL42EURL (* numeric forms *)
SURL42pxEURL (* numeric forms with suffixes *)
\end{lstlisting}
%SURL<tag>body includes enclosing tags</tag>EURL
\caption[Available Generalized Literal Forms]{Generalized literal forms available for use in VerseML's concrete syntax. The characters in blue indicate where the literal bodies are located within each form. In this figure, each line describes how the literal body is constrained by the form shown on that line. The Wyvern language specifies additional forms, including whitespace-delimited forms \cite{TSLs} and multipart forms \cite{sac15}, but for simplicity we leave these out of VerseML.}
\label{fig:literal-forms}
\end{figure}
\subsection{Definition}\label{sec:uetsms-definition}
%The original expression, above, is statically rewritten to this expression.
Let us now take the perspective of the library provider. The definition of the TSM \lstinline{#\dolla#rx} shown being applied above has the following form:
\begin{lstlisting}[numbers=none,mathescape=|]
syntax $rx at Rx {
  static fn(body : Body) : ParseResultExp => 
    (* regex literal parser here *)
}
\end{lstlisting}
This {TSM definition} first identifies the TSM as \lstinline{#\dolla#rx}. 
 TSM identifiers must begin with the dollar symbol (\li{#\dolla#}) to clearly distinguish them from variables (and thereby clearly distinguish macro application from function application). This is inspired by a similar convention enforced by the Rust macro system \cite{Rust/Macros}.

The TSM definition then specifies a \emph{type annotation}, \lstinline{at Rx}, and a \emph{parse function} within curly braces. 
The {parse function} is a \emph{static function} responsible for parsing the literal body when the macro is applied to generate an encoding of the candidate expansion, or indicating an error if one cannot be generated (e.g. when the body is ill-formed according to the syntactic specification that the TSM implements). Static functions are functions that are applied during the typed expansion process. For this reason, they do not have access to surrounding variable bindings (because those variables stand in for dynamic values). For now, let us simply assume that static functions are closed (we discuss introducing a distinct class of static bindings so that static values can be shared between TSM definitions in Sec. \ref{sec:uetsms-static-language}).

The parse function must have type \li{Body -> ParseResultExp}. These types are defined in the VerseML \emph{prelude}, which is a collection of definitions available ambiently. The input type, \lstinline{Body}, gives the parse function access to the {body} of the provided literal form. For our purposes, it suffices to define \li{Body} as an abbreviation for the \li{string} type:
\begin{lstlisting}[numbers=none]
type Body = string
\end{lstlisting} 

The output type, \li{ParseResultExp}, is a labeled sum type that distinguishes between successful parses and parse errors:
\begin{lstlisting}[numbers=none]
type ParseResultExp = Success of CEExp 
                    | ParseError of {msg : string, loc : IndexRange}
\end{lstlisting}

If parsing succeeds, the parse function returns a value of the form \li{Success(#$\ecand$#)}, where $\ecand$ is the \emph{encoding of the candidate expansion}. Encodings of candidate expansions are, for expression TSMs, values of the type \lstinline{CEExp} defined in Figure \ref{fig:candidate-exp-verseml}. 
\begin{figure}
\begin{lstlisting}[numbers=none]
type CETyp = TyVar of var_t 
           | Arrow of CETyp * CETyp 
           | ... 
           | Spliced of IndexRange

type CEExp = Var of var_t 
           | Fn of var_t * CETyp * CEExp
           | App of CEExp * CEExp
           | ... 
           | Spliced of IndexRange
\end{lstlisting}
\caption[Abbreviated definitions of \li{CETyp} and \li{CEExp} in VerseML.]{Abbreviated definitions of the types \li{CETyp} and \li{CEExp} in the VerseML prelude. We assume some suitable type \li{var_t} exists, not shown.}
\label{fig:candidate-exp-verseml}
\end{figure}
The elided constructors in Figure \ref{fig:candidate-exp-verseml} encode the abstract syntax of VerseML expressions and types (as in the SML visible compiler \cite{SML/VisibleCompiler}).
% We will show a complete encoding when we describe our reduced formal system $\miniVerseU$ in Sec. \ref{sec:tsms-minimal-formalism}. 
We discuss the constructors labeled \li{Spliced} in Sec. \ref{sec:splicing-and-hygiene}. To decrease the syntactic cost of working with the types defined in Figure \ref{fig:candidate-exp-verseml}, the prelude provides \emph{quasiquotation syntax} at these types, which is itself implemented using TSMs. We will discuss these TSMs in more detail in Sec. \ref{sec:tsms-for-tsms}. The definitions in Figure \ref{fig:candidate-exp-verseml} are recursive labeled sum types to simplify our exposition, but we could have chosen alternative encodings of terms, e.g. based on abstract binding trees \cite{pfpl}, with only minor modification to the semantics. % It is extended with one additional form used to handled spliced subexpressions, 

If the parse function determines that a candidate expansion cannot be generated, i.e. there is a parse error in the literal body, it returns a value labeled by \li{ParseError}. It must provide an error message and indicate the location of the error within the body of the literal form as a value of type \li{IndexRange}:
\begin{lstlisting}[numbers=none]
type IndexRange = {startIndex : nat, endIndex : nat} (* inclusive *)
\end{lstlisting}
The error message and error location can be used by VerseML compilers when reporting errors to the programmer.

%Notice that the types just described are those that one would expect to find in a typical parser.

%One would find types analagous to those just described in any parser, so for concision, we elide the details of \li{#\dolla#rx}'s parse function.
%The parse function must treat the TSM parameters parametrically, i.e. it does not have access to any values in the supplied module parameter. Only the expansion the parse function generates can refer to module parameters. 
%For example, the following definition is ill-sorted:
%\begin{lstlisting}[numbers=none]
%syntax pattern_bad[Q : PATTERN] at Q.t {
%  static fn (body : Body) : Exp => 
%    if Q.flag then (* ... *) else (* ... *)
%}
%\end{lstlisting}%So the parse function parses the body of the delimited form to generate an encoding of the elaboration.

\subsection{Splicing}\label{sec:splicing-and-hygiene}
To support splicing syntax, like that described in Sec. \ref{sec:syntax-examples-regexps}, the parse function must be able to parse subexpressions out of the supplied literal body. For example, consider the code snippet in Figure \ref{fig:derived-spliced-subexpressions}, expressed instead using the \li{#\dolla#rx} TSM:
\begin{lstlisting}[numbers=none]
val ssn = $rx /SURL\d\d\d-\d\d-\d\d\d\dEURL/
fun example_rx_tsm(name: string) => $rx /SURL@EURLnameSURL: %EURLssn/
\end{lstlisting}
The subexpressions \lstinline{name} and \lstinline{ssn} on the second line appear directly in the body of the literal form, so we call them \emph{spliced subexpressions} (and color them black when typesetting them in this document). When the parse function determines that a subsequence of the literal body should be treated as a spliced subexpression (here, by recognizing the characters \lstinline{@} or \lstinline{%} followed by a variable or parenthesized expression), 
it can refer to it within the candidate expansion it generates using the \li{Spliced} constructor of the \li{CEExp} type shown in Figure \ref{fig:candidate-exp-verseml}. The \li{Spliced} constructor requires a value of type \li{IndexRange} because spliced subexpressions are referred to indirectly by their position within the literal body. This prevents TSMs from ``forging'' a spliced subexpression (i.e. claiming that an expression is a spliced subexpression, even though it does not appear in the body of the literal form). Expressions can also contain types, so one can also mark spliced types in an analagous manner using the \li{Spliced} constructor of the \li{CETyp} type. %In particular, the parse function must provide the index range of spliced subexpressions to the \li{Spliced} constructor of the type \li{MarkedExp}. %Only subexpressions that actually appear in the body of the literal form can be marked as spliced subexpressions.

The candidate expansion generated by \li{#\dolla#rx} for the body of \lstinline{example_rx_tsm}, if written in a hypothetical concrete syntax for candidate expansions where references to spliced subexpressions are written \li{spliced<startIdx, endIndex>}, is:
\begin{lstlisting}[numbers=none]
Seq(Str(spliced<1, 4>), Seq(Str "SSTR: ESTR", spliced<8, 10>))
\end{lstlisting}
Here, \li{spliced<1, 4>} refers to the subexpression \li{name} by position and \li{spliced<8, 10>} refers to the subexpression \li{ssn} by position. 

%For example, had the  would not be a valid expansion, because the  that are not inside spliced subexpressions:
%\begin{lstlisting}[numbers=none]
%Q.Seq(Q.Str(name), Q.Seq(Q.Str ": ", ssn))
%\end{lstlisting}

\subsection{Typing}\label{sec:uetsms-validation}
The language \emph{validates} candidate expansions before a final expansion is generated. One aspect of candidate expansion validation is checking  the candidate expansion against the type annotation specified by the TSM, e.g. the type \li{Rx} in the example above. This maintains a \emph{type discipline}: if a programmer sees a TSM being applied when examining a well-typed program, they need only look up the TSM's type annotation to determine the type of the generated expansion. Determining the type does not require examine the expansion directly.


\subsection{Hygiene}
The spliced subexpressions that the candidate expansion refers to (by their position within the literal body, cf. above) must be parsed, typed and expanded during the candidate expansion validation process (otherwise, the language would not be able to check the type of the candidate expansion). To maintain a useful \emph{binding discipline}, i.e. to allow programmers to reason also about variable binding without examining expansions directly, the validation process maintains two additional properties related to spliced subexpressions: \textbf{context independent expansion} and \textbf{expansion independent splicing}. These are collectively referred to as the \emph{hygiene properties} (because they are conceptually related to the concept of hygiene in term rewriting macro systems, cf. Sec. \ref{sec:term-rewriting}.) 

\paragraph{Context Independent Expansion} Programmers expect to be able to choose variable and symbol names freely, i.e. without needing to satisfy ``hidden assumptions'' made by the TSMs that are applied in scope of a binding. For this reason, context-dependent candidate expansions, i.e. those with free variables or symbols, are deemed invalid (even at application sites where those variables happen to be bound). An example of a TSM that generates context-dependent candidate expansions is shown below:
\begin{lstlisting}[numbers=none]
syntax $bad1 at Rx {
	static fn(body : Body) : ParseResultExp => Success (Var 'SSTRxESTR')
}
\end{lstlisting}
The candidate expansion this TSM generates would be well-typed only when there is an assumption \li{x : Rx} in the application site typing context. This ``hidden assumption'' makes reasoning about binding and renaming especially difficult, so this candidate expansion is deemed invalid (even when \li{#\dolla#bad1} is applied in a context where \li{x} happens to be bound).

Of course, this prohibition does not extend into the spliced subexpressions referred to in a candidate expansion because spliced subexpressions are authored by the TSM client and appear at the application site, and so can justifiably refer to application site bindings. We saw examples of spliced subexpressions that referred to variables bound at the application site in Sec. \ref{sec:splicing-and-hygiene}. Because candidate expansions refer to spliced subexpressions indirectly, checking this property is straightforward -- we only allow access to the application site typing context when typing spliced subexpressions. In the next section, we will formalize this intuition. % The TSM provider can only refer to them opaquely.

In the examples in Sec. \ref{sec:uetsms-usage} and Sec. \ref{sec:splicing-and-hygiene}, the expansion used constructors associated with the \li{Rx} type, e.g. \li{Seq} and \li{Str}. This might appear to violate our prohibition on context-dependent expansions. This is not the case only because in VerseML, constructor labels are not variables or scoped symbols. Syntactically, they must begin with a capital letter (like Haskell's datatype constructors). Different labeled sum types can use common constructor labels without conflict because the type the term is being checked against -- e.g. \li{Rx}, due to the type ascription on \li{#\dolla#rx} -- determines which type of value will be constructed. For dialects of ML where datatype definitions do introduce new variables or scoped symbols, we need parameterized TSMs. We will return to this topic in Chapter \ref{sec:tsms-parameterized}. % Indeed, we used the label \li{Spliced} for two different recursive labeled sum types in Figure \ref{fig:candidate-exp-verseml}.

\paragraph{Expansion Independent Splicing} Spliced subexpressions, as just described, must be given access to application site bindings. The \emph{expansion independent splicing} property ensures that spliced subexpressions have access to \emph{only} those bindings, i.e. a TSM cannot introduce new bindings into spliced subexpressions. For example, consider the following hypothetical candidate expansion (written concretely as above):
\begin{lstlisting}[numbers=none]
fn(x : Rx) => spliced<0, 4>
\end{lstlisting}
The variable \li{x} is not available when typing the indicated spliced subexpression, nor can it shadow any bindings of \li{x} that might appear at the application site.

For TSM providers, the benefit of this property is that they can choose the names of variables used internally within expansions freely, without worrying about whether they might shadow those that a client might have defined at the application site.

TSM clients can, in turn, determine exactly which bindings are available in a spliced subexpression without examining the expansion it appears within. In other words, there can be no ``hidden variables''. 

The trade-off is that this prevents library providers from defining  alternative binding forms. For example, Haskell's derived form for monadic commands (i.e. \li{do}-notation) supports binding the result of executing a command to a variable that is then available in the subsequent commands in a command sequence. In VerseML, this cannot be expressed in the same way. We will show an alternative formulation of Haskell's syntax for monadic commands that uses VerseML's anonymous function syntax to bind variables in Sec. \ref{sec:application-monadic-commands}. We will discuss mechanisms that would allow us to relax this restriction while retaining client control over variable names as future work in Sec. \ref{sec:controlled-binding}.

%These properties suffice to ensure that programmers and tools can freely rename a variable without changing the meaning of the program. The only information that is necessary to perform such a \emph{rename refactoring} is the locations of spliced subexpressions within all the literal forms for which the variable being renamed is in scope; the expansions need not otherwise be examined. It would be straightforward to develop a tool and/or editor plugin to indicate the locations of spliced subexpressions to the user, like we do in this document (by coloring spliced subexpressions black). We discuss tool support as future work in Sec. \ref{sec:interaction-with-tools}.

\subsubsection{Final Expansion}
After checking that the candidate expansion is {valid}, the semantics generates the \emph{final expansion} by replacing the references to spliced subexpressions with their final expansions. For example, the final expansion of the body of \li{example_rx_tsm} is:
\begin{lstlisting}[numbers=none]
Seq(Str(name), Seq(Str "SSTR: ESTR", ssn))
\end{lstlisting}
%Put another way, the  elaboration logic must be valid in any context. 

\subsection{Scoping}
A benefit of specifying TSMs as a language primitive, rather than relying on extralinguistic mechanisms to manipulate the concrete syntax of our language directly, is that TSMs follow standard scoping rules. 

For example, we can define a TSM that is visible only to a single expression like this:
\begin{lstlisting}[numbers=none]
let
  syntax $rx at Rx { (* ... *) }
in 
  (* $rx is in scope here *)
end
(* $rx is no longer in scope *)
\end{lstlisting}

We will consider the question of how TSM definitions can be exported from within modules in Sec. \ref{sec:tsm-packaging}.

\subsection{Comparison to ML+Rx}
Let us compare the VerseML TSM \li{#\dolla#rx} to ML+Rx, the hypothetical syntactic dialect of ML with support for derived forms for regular expressions described in Sec. \ref{sec:syntax-examples-regexps}.

Both ML+Rx and \li{#\dolla#rx} give programmers the ability to use the same standard syntax for constructing regexes, including syntax for splicing in other strings and regexes. In VerseML, however, we incur the additional syntactic cost of explicitly applying the \li{#\dolla#rx} TSM each time we wish to use regex syntax. This cost does not grow with the size of the regex, so it would only be significant in programs that involve a large number of small regexes (which do, of course, exist). In Chapter \ref{chap:tsls} we will consider a design where even this syntactic cost can be eliminated in certain situations.

The benefit of this approach is that we can easily define other TSMs to use alongside the \li{#\dolla#rx} TSM without needing to consider the possibility of syntactic conflict. Furthermore, programmers can rely on the typing discipline and the hygienic binding discipline described above to reason about programs, including those that contain unfamiliar forms. Put pithily, VerseML helps programmers avoid ``conflict and confusion''. 


\section{$\miniVerseU$}\label{sec:tsms-minimal-formalism}

% \begin{figure}[p!]
% $\begin{array}{lllllll}
% \textbf{variables} & \textbf{type variables} & \textbf{labels} & \textbf{label sets} & \textbf{TSM variables} & \textbf{literal bodies} & \textbf{nats}\\
% x & t & \ell & \labelset & \tsmv & b & n\\~\end{array}$\\
% $\begin{array}{ll}
% \textbf{type formation contexts} & \textbf{typing contexts}\\
% \Delta ::= \emptyset ~\vert~ \Delta, t & \Gamma ::= \emptyset ~\vert~ \Gamma, x : \tau\\
% ~
% \end{array}$\\
% ~\\
% $\begin{array}{lcl}
% \gheading{types}\\
% \tau & ::= & t ~\vert~ \parr{\tau}{\tau} ~\vert~ \forallt{t}{\tau} ~\vert~ \rect{t}{\tau} ~\vert~  \prodt{\mapschema{\tau}{i}{\labelset}} ~\vert~ \sumt{\mapschema{\tau}{i}{\labelset}}\\
% ~\\
% \gheading{expanded expressions}\\
% e & ::= & x ~\vert~ \lam{x}{\tau}{e} ~\vert~ \app{e}{e} ~\vert~ \Lam{t}{e} ~\vert~ \App{e}{\tau} ~\vert~ \fold{t}{\tau}{e} ~\vert~ \unfold{e} ~\vert~ \tpl{\mapschema{e}{i}{\labelset}} ~\vert~ \prj{e}{\ell} \\
% & \vert & \inj{\ell}{e} ~\vert~ \caseof{e}{\mapschemab{x}{e}{i}{\labelset}}\\
% ~\\
% \gheading{TSM expressions}\\
% \tsme & ::= & \tsmv ~\vert~ \utsmdef{\tau}{\ue}\\
% ~\\
% \gheading{unexpanded expressions}\\
% \ue & ::= & {x} ~\vert~ \lam{x}{\tau}{\ue} ~\vert~ \ue(\ue) ~\vert~ \Lam{t}{\ue} ~\vert~ \App{\ue}{\tau} ~\vert~ \fold{t}{\tau}{\ue} ~\vert~ \unfold{\ue} ~\vert~ \tpl{\mapschema{\ue}{i}{\labelset}} ~\vert~ \prj{\ue}{\ell} \\
% & \vert & \inj{\ell}{\ue} ~\vert~ \caseof{\ue}{\mapschemab{x}{\ue}{i}{\labelset}}\\
% & \vert & \uletsyntax{\tsmv}{\tsme}{\ue} ~\vert~ \utsmapp{\eta}{b}\\
% ~\\
% \gheading{candidate expansion types}\\
% \mtau & ::= & t ~\vert~ \parr{\mtau}{\mtau} ~\vert~ \forallt{t}{\mtau} ~\vert~ \rect{t}{\mtau} ~\vert~ \prodt{\mapschema{\tau}{i}{\labelset}} ~\vert~ \sumt{\mapschema{\mtau}{i}{\labelset}} \\
% & \vert & \mtspliced{\tau}\\
% ~\\
% \gheading{candidate expansion expressions}\\
% \me & ::= & x ~\vert~ \lam{x}{\mtau}{\me} ~\vert~ \app{\me}{\me} ~\vert~ \Lam{t}{\me} ~|~ \App{\me}{\mtau} ~\vert~ \fold{t}{\mtau}{\me} ~\vert~ \unfold{\me} ~\vert~ \tpl{\mapschema{\me}{i}{\labelset}} ~\vert~ \prj{\me}{\ell} \\
% & \vert & \inj{\ell}{\me} ~\vert~ \caseof{\me}{\mapschemab{x}{\me}{i}{\labelset}}\\
% & \vert & \mspliced{e}
% % \\~
% \end{array}$
% \todo{finish breaking this up into syntax tables}
% \caption[Syntax of $\miniVerseU$]{Syntax of $\miniVerseU$. The forms $\mapschema{V}{i}{\labelset}$ and $\mapschemab{x}{V}{i}{\labelset}$ where $V$ is a metavariable indicate finite mappings from each label $i \in \labelset$ to a term, $V_i$, or binder, $x_i.V_i$, respectively.}
% \label{fig:lambda-tsm-syntax}
% \end{figure}


To make the intuitions developed in the previous section mathematically precise, we will now introduce a reduced calculus with support for unparameterized expression TSMs called $\miniVerseU$. 
%For reference, the syntax of $\miniVerseU$ is specified in Figure \ref{fig:lambda-tsm-syntax}. We will reproduce relevant portions of this specification inline (in tabular form) as we continue. 
%We specify all formal systems in this document within the metatheoretic framework detailed in \emph{PFPL} \cite{pfpl}, and assume familiarity of fundamental background concepts (e.g. abstract binding trees, substitution, implicit identification of terms up to $\alpha$-equivalence, structural induction and rule induction) covered therein. %Familiarity with other accounts of typed lambda calculi should also suffice to understand the formal systems in this document. 



\subsection{Types and Expanded Expressions}

\begin{figure}
$\begin{array}{lllllll}
\textbf{Sort} & & & \textbf{Abstract Form} & \textbf{Stylized Form} & \textbf{Description}\\
\mathsf{Typ} & \tau & ::= & t & t & \text{variable}\\
&&& \aparr{\tau}{\tau} & \parr{\tau}{\tau} & \text{partial function}\\
&&& \aall{t}{\tau} & \forallt{t}{\tau} & \text{polymorphic}\\
&&& \arec{t}{\tau} & \rect{t}{\tau} & \text{recursive}\\
&&& \aprod{\mapschema{\tau}{i}{\labelset}} & \prodt{\mapschema{\tau}{i}{\labelset}} & \text{labeled product}\\
&&& \asum{\mapschema{\tau}{i}{\labelset}} & \sumt{\mapschema{\tau}{i}{\labelset}} & \text{labeled sum}\\
\mathsf{EExp} & e & ::= & x & x & \text{variable}\\
&&& \aelam{\tau}{x}{e} & \lam{x}{\tau}{e} & \text{abstraction}\\
&&& \aeap{e}{e} & \ap{e}{e} & \text{application}\\
&&& \aetlam{t}{e} & \Lam{t}{e} & \text{type abstraction}\\
&&& \aetap{e}{\tau} & \App{e}{\tau} & \text{type application}\\
&&& \aefold{t}{\tau}{e} & \fold{e} & \text{fold}\\
&&& \aeunfold{e} & \unfold{e} & \text{unfold}\\
&&& \aetpl{\mapschema{e}{i}{\labelset}} & \tpl{\mapschema{e}{i}{\labelset}} & \text{labeled tuple}\\
&&& \aepr{\ell}{e} & \prj{e}{\ell} & \text{projection}\\
&&& \aein{\ell}{\mapschema{\tau}{i}{\labelset}}{e} & \inj{\ell}{e} & \text{injection}\\
&&& \aecase{e}{\mapschemab{x}{e}{i}{\labelset}} & \caseof{e}{\mapschemab{x}{e}{i}{\labelset}} & \text{case analysis}
\end{array}$
\caption[Syntax of types and expanded expressions in $\miniVerseU$.]{Syntax of types and expanded expressions in $\miniVerseU$. Metavariable $x$ ranges over variables, $t$ ranges over type variables, $\ell$ ranges over labels and $\labelset$ ranges over sets of labels. The forms $\mapschema{e}{i}{\labelset}$, $\mapschema{\tau}{i}{\labelset}$ and $\mapschemab{x}{e}{i}{\labelset}$ indicate finite mappings from each label $i \in \labelset$ to an expression, type or binder over an expression, respectively. The label set is omitted for concision when writing particular finite mappings, e.g. $\finmap{\mapitem{\ell_1}{e_1}, \mapitem{\ell_2}{e_2}}$. We write $\mapschema{\tau}{i}{\labelset} \otimes \ell \hookrightarrow \tau$ when $\ell \notin \labelset$ for the extension of $\mapschema{\tau}{i}{\labelset}$ that maps $\ell$ to $\tau$. When we use the stylized forms $\fold{e}$ and $\inj{\ell}{e}$, we assume that the reader can infer the type information from context. }
\label{fig:U-expanded-terms}
\end{figure}

At the ``semantic core'' of $\miniVerseU$ are \emph{types}, $\tau$, and \emph{expanded expressions}, $e$. Their syntax is specified by the syntax chart in Figure \ref{fig:U-expanded-terms}. 
Types and expanded expressions form a language with support for partial functions, quantification over types, recursive types, and labeled product and sum types. The reader can consult \emph{PFPL} \cite{pfpl} (or another text on typed programming languages, e.g. \emph{TAPL} \cite{tapl}) for a detailed account of these constructs (or closely related variants thereof). For our purposes, it suffices to recall the following facts and definitions.

\subsubsection{Statics of Expanded Expressions}
The \emph{statics of expanded expressions} is specified by judgements of the following form:

\[\begin{array}{ll}
\textbf{Judgement Form} & \textbf{Description}\\
\istypeU{\Delta}{\tau} & \text{$\tau$ is a well-formed type assuming $\Delta$}\\
\isctxU{\Delta}{\Gamma} & \text{$\Gamma$ is a well-formed typing context assuming $\Delta$}\\
\hastypeU{\Delta}{\Gamma}{e}{\tau} & \text{$e$ has type $\tau$ assuming $\Delta$ and $\Gamma$}
\end{array}\]

\emph{Type formation contexts}, $\Delta$, consist of hypotheses of the form $\Dhyp{t}$ and can be understood as finite sets of type variables. \emph{Typing contexts}, $\Gamma$, consist of hypotheses of the form $\Ghyp{x}{\tau}$ and can be understood as finite mappings from variables to types. Syntactically, we write contexts as comma-separated sequences of {hypotheses} identified up to exchange and contraction. We write empty contexts using the symbol $\emptyset$ (or omit their mention). 

The type formation judgement, $\istypeU{\Delta}{\tau}$, ensures that all free type variables in $\tau$ are tracked by $\Delta$. It is inductively defined by the following rules:
\begin{subequations}\label{rules:istypeU}
\begin{equation}\label{rule:istypeU-var}
\inferrule{ }{\istypeU{\Delta, \Dhyp{t}}{t}}
\end{equation}
\begin{equation}\label{rule:istypeU-parr}
\inferrule{
  \istypeU{\Delta}{\tau_1}\\
  \istypeU{\Delta}{\tau_2}
}{\istypeU{\Delta}{\aparr{\tau_1}{\tau_2}}}
\end{equation}
\begin{equation}\label{rule:istypeU-all}
  \inferrule{
    \istypeU{\Delta, \Dhyp{t}}{\tau}
  }{
    \istypeU{\Delta}{\aall{t}{\tau}}
  }
\end{equation}
\begin{equation}\label{rule:istypeU-rec}
  \inferrule{
    \istypeU{\Delta, \Dhyp{t}}{\tau}
  }{
    \istypeU{\Delta}{\arec{t}{\tau}}
  }
\end{equation}
\begin{equation}\label{rule:istypeU-prod}
  \inferrule{
    \{\istypeU{\Delta}{\tau_i}\}_{i \in \labelset}
  }{
    \istypeU{\Delta}{\aprod{\mapschema{\tau}{i}{\labelset}}}
  }
\end{equation}
\begin{equation}\label{rule:istypeU-sum}
  \inferrule{
    \{\istypeU{\Delta}{\tau_i}\}_{i \in \labelset}
  }{
    \istypeU{\Delta}{\asum{\mapschema{\tau}{i}{\labelset}}}
  }
\end{equation}
\end{subequations}
Premises written $\{\mathcal{J}_i\}_{i \in \labelset}$ mean that for each $i \in \labelset$, the judgement $\mathcal{J}_i$ must be derived. 

The typing context formation judgement, $\isctxU{\Delta}{\Gamma}$, ensures that all types in the typing context are well-formed according to Rules (\ref{rules:istypeU}). It is inductively defined by the following rules:
\begin{subequations}\label{rules:isctxU}
\begin{equation}\label{rule:isctxU-empty}
  \inferrule{ }{
    \isctxU{\Delta}{\emptyset}
  }
\end{equation}
\begin{equation}\label{rule:isctxU-ext}
  \inferrule{
    \isctxU{\Delta}{\Gamma}\\
    \istypeU{\Delta}{\tau}
  }{
    \isctxU{\Delta}{\Gamma, \Ghyp{x}{\tau}}
  }
\end{equation}
\end{subequations}

The typing judgement, $\hastypeU{\Delta}{\Gamma}{e}{\tau}$, assigns types to expressions. It is inductively defined by the following rules:
\begin{subequations}\label{rules:hastypeU}
\begin{equation}\label{rule:hastypeU-var}
  \inferrule{ }{
    \hastypeU{\Delta}{\Gamma, \Ghyp{x}{\tau}}{x}{\tau}
  }
\end{equation}
\begin{equation}\label{rule:hastypeU-lam}
  \inferrule{
    \istypeU{\Delta}{\tau}\\
    \hastypeU{\Delta}{\Gamma, \Ghyp{x}{\tau}}{e}{\tau'}
  }{
    \hastypeU{\Delta}{\Gamma}{\aelam{\tau}{x}{e}}{\aparr{\tau}{\tau'}}
  }
\end{equation}
\begin{equation}\label{rule:hastypeU-ap}
  \inferrule{
    \hastypeU{\Delta}{\Gamma}{e_1}{\aparr{\tau}{\tau'}}\\
    \hastypeU{\Delta}{\Gamma}{e_2}{\tau}
  }{
    \hastypeU{\Delta}{\Gamma}{\aeap{e_1}{e_2}}{\tau'}
  }
\end{equation}
\begin{equation}\label{rule:hastypeU-tlam}
  \inferrule{
    \hastypeU{\Delta, \Dhyp{t}}{\Gamma}{e}{\tau}
  }{
    \hastypeU{\Delta}{\Gamma}{\aetlam{t}{e}}{\aall{t}{\tau}}
  }
\end{equation}
\begin{equation}\label{rule:hastypeU-tap}
  \inferrule{
    \hastypeU{\Delta}{\Gamma}{e}{\aall{t}{\tau}}\\
    \istypeU{\Delta}{\tau'}
  }{
    \hastypeU{\Delta}{\Gamma}{\aetap{e}{\tau'}}{[\tau'/t]\tau}
  }
\end{equation}
\begin{equation}\label{rule:hastypeU-fold}
  \inferrule{\
    \istypeU{\Delta, \Dhyp{t}}{\tau}\\
    \hastypeU{\Delta}{\Gamma}{e}{[\arec{t}{\tau}/t]\tau}
  }{
    \hastypeU{\Delta}{\Gamma}{\aefold{t}{\tau}{e}}{\arec{t}{\tau}}
  }
\end{equation}
\begin{equation}\label{rule:hastypeU-unfold}
  \inferrule{
    \hastypeU{\Delta}{\Gamma}{e}{\arec{t}{\tau}}
  }{
    \hastypeU{\Delta}{\Gamma}{\aeunfold{e}}{[\arec{t}{\tau}/t]\tau}
  }
\end{equation}
\begin{equation}\label{rule:hastypeU-tpl}
  \inferrule{
    \{\hastypeU{\Delta}{\Gamma}{e_i}{\tau_i}\}_{i \in \labelset}
  }{
    \hastypeU{\Delta}{\Gamma}{\aetpl{\mapschema{e}{i}{\labelset}}}{\aprod{\mapschema{\tau}{i}{\labelset}}}
  }
\end{equation}
\begin{equation}\label{rule:hastypeU-pr}
  \inferrule{
    \hastypeU{\Delta}{\Gamma}{e}{\aprod{\mapschema{\tau}{i}{\labelset} \otimes \ell \hookrightarrow \tau}}
  }{
    \hastypeU{\Delta}{\Gamma}{\aepr{\ell}{e}}{\tau}
  }
\end{equation}
\begin{equation}\label{rule:hastypeU-in}
  \inferrule{
    \{\istypeU{\Delta}{\tau_i}\}_{i \in \labelset}\\
    \istypeU{\Delta}{\tau}\\
    \hastypeU{\Delta}{\Gamma}{e}{\tau}
  }{
    \hastypeU{\Delta}{\Gamma}{\aein{\ell}{\mapschema{\tau}{i}{\labelset}\otimes \ell \hookrightarrow \tau}{e}}{\asum{\mapschema{\tau}{i}{\labelset}\otimes\ell \hookrightarrow \tau}}
  }
\end{equation}
\begin{equation}\label{rule:hastypeU-case}
  \inferrule{
    \hastypeU{\Delta}{\Gamma}{e}{\asum{\mapschema{\tau}{i}{\labelset}}}\\
    \{\hastypeU{\Delta}{\Gamma, x_i : \tau_i}{e_i}{\tau}\}_{i \in \labelset}
  }{
    \hastypeU{\Delta}{\Gamma}{\aecase{e}{\mapschemab{x}{e}{i}{\labelset}}}{\tau}
  }
\end{equation}
\end{subequations}

The rules given above validate the following standard lemmas. 

The Weakening Lemma expresses the intuition that extending a context with unused variables preserves well-formedness and typing.
\begin{lemma}[Weakening]\label{lemma:weakening-U} All of the following hold: 
\begin{enumerate} 
\item If $\istypeU{\Delta}{\tau}$ then $\istypeU{\Delta, \Dhyp{t}}{\tau}$.
\item If $\isctxU{\Delta}{\Gamma}$ then $\isctxU{\Delta, \Dhyp{t}}{\Gamma}$.
\item If $\hastypeU{\Delta}{\Gamma}{e}{\tau}$ then $\hastypeU{\Delta, \Dhyp{t}}{\Gamma}{e}{\tau}$.
\item If $\hastypeU{\Delta}{\Gamma}{e}{\tau}$ and $\istypeU{\Delta}{\tau'}$ then $\hastypeU{\Delta}{\Gamma, \Ghyp{x}{\tau'}}{e}{\tau}$.
\end{enumerate}
\end{lemma}
\begin{proof-sketch}
\begin{enumerate}
\item By rule induction on Rules (\ref{rules:istypeU}).
\item By rule induction on Rules (\ref{rules:isctxU}).
\item By rule induction on Rules (\ref{rules:hastypeU}).
\item By rule induction on Rules (\ref{rules:hastypeU}).
\end{enumerate}
\end{proof-sketch}

The Substitution Lemma expresses the intuition that substitution of a type for a type variable, or an expression of the appropriate type for an expression variable, preserves well-formedness and typing. 
\begin{lemma}[Substitution]\label{lemma:substitution-U} All of the following hold:
\begin{enumerate}
\item If $\istypeU{\Delta, \Dhyp{t}}{\tau}$ and $\istypeU{\Delta}{\tau'}$ then $\istypeU{\Delta}{[\tau'/t]\tau}$.
\item If $\isctxU{\Delta, \Dhyp{t}}{\Gamma}$ and $\istypeU{\Delta}{\tau'}$ then $\isctxU{\Delta}{[\tau'/t]\Gamma}$.
\item If $\hastypeU{\Delta, \Dhyp{t}}{\Gamma}{e}{\tau}$ and $\istypeU{\Delta}{\tau'}$ then $\hastypeU{\Delta}{[\tau'/t]\Gamma}{[\tau'/t]e}{[\tau'/t]\tau}$.
\item If $\hastypeU{\Delta}{\Gamma, \Ghyp{x}{\tau'}}{e}{\tau}$ and $\hastypeU{\Delta}{\Gamma}{e'}{\tau'}$ then $\hastypeU{\Delta}{\Gamma}{[e'/x]e}{\tau}$.
\end{enumerate}\end{lemma}
\begin{proof-sketch}
In each case, by rule induction on the derivation of the first assumption.
\end{proof-sketch}
The Decomposition Lemma is the converse of the Substitution Lemma.
\begin{lemma}[Decomposition]\label{lemma:decomposition-U} All of the following hold:
\begin{enumerate}
\item If $\istypeU{\Delta}{[\tau'/t]\tau}$ and $\istypeU{\Delta}{\tau'}$ then $\istypeU{\Delta, \Dhyp{t}}{\tau}$.
\item If $\isctxU{\Delta}{[\tau'/t]\Gamma}$ and $\istypeU{\Delta}{\tau'}$ then $\isctxU{\Delta, \Dhyp{t}}{\Gamma}$.
\item If $\hastypeU{\Delta}{[\tau'/t]\Gamma}{[\tau'/t]e}{[\tau'/t]\tau}$ and $\istypeU{\Delta}{\tau'}$ then $\hastypeU{\Delta, \Dhyp{t}}{\Gamma}{e}{\tau}$.
\item If $\hastypeU{\Delta}{\Gamma}{[e'/x]e}{\tau}$ and $\hastypeU{\Delta}{\Gamma}{e'}{\tau'}$ then $\hastypeU{\Delta}{\Gamma, \Ghyp{x}{\tau'}}{e}{\tau}$.
\end{enumerate}\end{lemma}
\begin{proof-sketch}
\begin{enumerate}
\item Type formation of $[\tau'/t]\tau$ does not depend on the structure of $\tau'$.
\item Context formation of $[\tau'/t]\Gamma$ does not depend on the structure of $\tau'$.
\item The derivation of $\hastypeU{\Delta}{[\tau'/t]\Gamma}{[\tau'/t]e}{[\tau'/t]\tau}$ does not depend on the structure of $\tau'$.
\item Typing of $[e'/x]e$ depends only on the type of $e'$ wherever it occurs, if at all.
\end{enumerate}
\end{proof-sketch}

The Regularity Lemma expresses the intuition that the type assigned to an expression under a well-formed typing context is well-formed. 
\begin{lemma}[Regularity]\label{lemma:regularity-U} If $\isctxU{\Delta}{\Gamma}$ and $\hastypeU{\Delta}{\Gamma}{e}{\tau}$ then $\istypeU{\Delta}{\tau}$.\end{lemma}
\begin{proof-sketch}
By rule induction on Rules (\ref{rules:hastypeU}) and inversion of Rule (\ref{rule:isctxU-ext}). 
\end{proof-sketch}
\subsubsection{Dynamics of Expanded Expressions}
The \emph{dynamics} of expanded expressions is specified as a structural dynamics (a.k.a. structural operational semantics) by judgements of the following form:
\[\begin{array}{ll}
\textbf{Judgement Form} & \textbf{Description}\\
\stepsU{e}{e'} & \text{$e$ transitions to $e'$}\\
\isvalU{e} & \text{$e$ is a value}
\end{array}\]
We also define auxiliary judgements for \emph{iterated transitions}, $\multistepU{e}{e'}$, and \emph{evaluation}, $\evalU{e}{e'}$.

\begin{definition}[Iterated Transition]\label{defn:iterated-transition-U} Iterated transition, $\multistepU{e}{e'}$, is the reflexive, transitive closure of the transition judgement.\end{definition}

\begin{definition}[Evaluation]\label{defn:evaluation-U}  $\evalU{e}{e'}$ iff $\multistepU{e}{e'}$ and $\isvalU{e'}$. \end{definition}

Our subsequent developments do not require making reference to particular rules in the dynamics (because TSMs operate statically), so for concision, we do not reproduce the rules here. Instead, it suffices to state the following conditions.

The Canonical Forms condition characterizes well-typed values. We assume an \emph{eager} (i.e. \emph{by-value}) formulation of the dynamics. 
\begin{condition}[Canonical Forms]\label{condition:canonical-forms-U} If $\hastypeUC{e}{\tau}$ and $\isvalU{e}$ then:
\begin{enumerate}
\item If $\tau=\aparr{\tau_1}{\tau_2}$ then $e=\aelam{\tau_1}{x}{e'}$ and $\hastypeUCO{\Ghyp{x}{\tau_1}}{e'}{\tau_2}$.
\item If $\tau=\aall{t}{\tau'}$ then $e=\aetlam{t}{e'}$ and $\hastypeUCO{\Dhyp{t}}{e'}{\tau'}$.
\item If $\tau=\arec{t}{\tau'}$ then $e=\aefold{t}{\tau'}{e'}$ and $\hastypeUC{e'}{[\abop{rec}{t.\tau'}/t]\tau'}$. 
\item If $\tau=\aprod{\mapschema{\tau}{i}{\labelset}}$ then $e=\aetpl{\mapschema{e}{i}{\labelset}}$ and for each $i \in \labelset$ we have that $\hastypeUC{e_i}{\tau_i}$ and $\isvalU{e_i}$.
\item If $\tau=\asum{\mapschema{\tau}{i}{\labelset}}$ then $e=\aein{\ell}{\mapschema{\tau}{i}{\labelset\setminus \ell} \otimes \ell \hookrightarrow \tau'}{e'}$ and $\ell \in \labelset$ and $\hastypeUC{e'}{\tau'}$ and $\isvalU{e'}$.
\end{enumerate}\end{condition}

We also require that the statics and dynamics be coherent. This is expressed in the standard way by Preservation and Progress conditions (together, these constitute the {Type Safety} Condition). 
\begin{condition}[Preservation]\label{condition:preservation-U} If $\hastypeUC{e}{\tau}$ and $\stepsU{e}{e'}$ then $\hastypeUC{e'}{\tau}$. \end{condition}
\begin{condition}[Progress]\label{condition:progress-U} If $\hastypeUC{e}{\tau}$ then either $\isvalU{e}$ or there exists an $e'$ such that $\stepsU{e}{e'}$. \end{condition}


\subsection{Unexpanded Expressions and Typed Expansion}
Programs evaluate as expanded expressions, but programmers author \emph{unexpanded expressions}, $\ue$. The syntax of unexpanded expressions is specified by the chart in Figure \ref{fig:U-unexpanded-terms}. 

Unexpanded expressions are typed and expanded simultaneously according to the \emph{typed expansion judgement}:
\[\begin{array}{ll}
\textbf{Judgement Form} & \textbf{Description}\\
\expandsUX{\ue}{e}{\tau} & \text{$\ue$ expands to $e$ at type $\tau$ assuming $\Delta$ and $\Gamma$ and macro}\\
& \text{environment $\Sigma$}
\end{array}\]

\begin{figure}
$\begin{array}{lllllll}
\textbf{Sort} & & & \textbf{Abstract Form} & \textbf{Stylized Form} & \textbf{Description}\\
\mathsf{UExp} & \ue & ::= & x & x & \text{variable}\\
&&& \aulam{\tau}{x}{\ue} & \lam{x}{\tau}{\ue} & \text{abstraction}\\
&&& \auap{\ue}{\ue} & \ap{\ue}{\ue} & \text{application}\\
&&& \autlam{t}{\ue} & \Lam{t}{\ue} & \text{type abstraction}\\
&&& \autap{\ue}{\tau} & \App{\ue}{\tau} & \text{type application}\\
&&& \aufold{t}{\tau}{\ue} & \fold{\ue} & \text{fold}\\
&&& \auunfold{\ue} & \unfold{\ue} & \text{unfold}\\
&&& \autpl{\mapschema{\ue}{i}{\labelset}} & \tpl{\mapschema{\ue}{i}{\labelset}} & \text{labeled tuple}\\
&&& \aupr{\ell}{\ue} & \prj{\ue}{\ell} & \text{projection}\\
&&& \auin{\ell}{\mapschema{\tau}{i}{\labelset}}{\ue} & \inj{\ell}{\ue} & \text{injection}\\
&&& \aucase{\ue}{\mapschemab{x}{\ue}{i}{\labelset}} & \caseof{\ue}{\mapschemab{x}{\ue}{i}{\labelset}} & \text{case analysis}\\
&&& \auletsyntax{\tau}{\ue}{\tsmv}{\ue} & \uletsyntax{\tsmv}{\tau}{\ue}{\ue} & \text{macro definition}\\
&&& \autsmap{b}{\tsmv} & \utsmap{\tsmv}{b} & \text{macro application}
\end{array}$
\caption[Syntax of unexpanded expressions in $\miniVerseU$.]{Syntax of unexpanded expressions in $\miniVerseU$. Metavariable $\tsmv$ ranges over macro identifiers and $b$ ranges over literal bodies. Literal bodies might contain unparsed subexpressions, so variable renaming and substitution cannot be defined in the usual manner over unexpanded expressions (i.e. they should be thought of as partially parsed abstract syntax trees, not abstract binding trees.)}
\label{fig:U-unexpanded-terms}
\end{figure}
\noindent
The typed expansion judgement is inductively defined by the following rules:
\begin{subequations}\label{rules:expandsU}
\begin{equation}\label{rule:expandsU-var}
  \inferrule{ }{\expandsU{\Delta}{\Gamma, x : \tau}{\Sigma}{x}{x}{\tau}}
\end{equation}
\begin{equation}\label{rule:expandsU-lam}
  \inferrule{
    \istypeU{\Delta}{\tau}\\
    \expandsU{\Delta}{\Gamma, x : \tau}{\Sigma}{\ue}{e}{\tau'}
  }{\expandsUX{\aulam{\tau}{x}{\ue}}{\aelam{\tau}{x}{e}}{\aparr{\tau}{\tau'}}}
\end{equation}
\begin{equation}\label{rule:expandsU-ap}
  \inferrule{
    \expandsUX{\ue_1}{e_1}{\aparr{\tau}{\tau'}}\\
    \expandsUX{\ue_2}{e_2}{\tau}
  }{
    \expandsUX{\auap{\ue_1}{\ue_2}}{\aeap{e_1}{e_2}}{\tau'}
  }
\end{equation}
\begin{equation}\label{rule:expandsU-tlam}
  \inferrule{
    \expandsU{\Delta, \Dhyp{t}}{\Gamma}{\Sigma}{\ue}{e}{\tau}
  }{
    \expandsUX{\autlam{t}{\ue}}{\aetlam{t}{e}}{\aall{t}{\tau}}
  }
\end{equation}
\begin{equation}\label{rule:expandsU-tap}
  \inferrule{
    \expandsUX{\ue}{e}{\aall{t}{\tau}}\\
    \istypeU{\Delta}{\tau'}
  }{
    \expandsUX{\autap{\ue}{\tau'}}{\aetap{e}{\tau'}}{[\tau'/t]\tau}
  }
\end{equation}
\begin{equation}\label{rule:expandsU-fold}
  \inferrule{
    \istypeU{\Delta, \Dhyp{t}}{\tau}\\
    \expandsUX{\ue}{e}{[\arec{t}{\tau}/t]\tau}
  }{
    \expandsUX{\aufold{t}{\tau}{\ue}}{\aefold{t}{\tau}{e}}{\arec{t}{\tau}}
  }
\end{equation}
\begin{equation}\label{rule:expandsU-unfold}
  \inferrule{
    \expandsUX{\ue}{e}{\arec{t}{\tau}}
  }{
    \expandsUX{\auunfold{\ue}}{\aeunfold{e}}{[\arec{t}{\tau}/t]\tau}
  }
\end{equation}
\begin{equation}\label{rule:expandsU-tpl}
  \inferrule{
    \{\expandsUX{\ue_i}{e_i}{\tau_i}\}_{i \in \labelset}
  }{
    \expandsUX{\autpl{\mapschema{\ue}{i}{\labelset}}}{\aetpl{\mapschema{e}{i}{\labelset}}}{\aprod{\mapschema{\tau}{i}{\labelset}}}
  }
\end{equation}
\begin{equation}\label{rule:expandsU-pr}
  \inferrule{
    \expandsUX{\ue}{e}{\aprod{\mapschema{\tau}{i}{\labelset}\otimes\mapitem{\ell}{\tau}}}
  }{
    \expandsUX{\aupr{\ell}{\ue}}{\aepr{\ell}{e}}{\tau}
  }
\end{equation}
\begin{equation}\label{rule:expandsU-in}
  \inferrule{
    \{\istypeU{\Delta}{\tau_i}\}_{i \in \labelset}\\
    \istypeU{\Delta}{\tau}\\
    \expandsUX{\ue}{e}{\tau}
  }{
    \left\{\shortstack{$\Delta~\Gamma \vdash_\Sigma \auin{\ell}{\mapschema{\tau}{i}{\labelset}\otimes\mapitem{\ell}{\tau}}{\ue}$\\$\leadsto$\\$\aein{\ell}{\mapschema{\tau}{i}{\labelset}\otimes\mapitem{\ell}{\tau}}{e} : \asum{\mapschema{\tau}{i}{\labelset}\otimes\mapitem{\ell}{\tau}}$\vspace{-1.2em}}\right\}
  }
\end{equation}
\begin{equation}\label{rule:expandsU-case}
  \inferrule{
    \expandsUX{\ue}{e}{\asum{\mapschema{\tau}{i}{\labelset}}}\\
    \{\expandsU{\Delta}{\Gamma, \Ghyp{x_i}{\tau_i}}{\Sigma}{\ue_i}{e_i}{\tau}\}_{i \in \labelset}
  }{
    \expandsUX{\aucase{\ue}{\mapschemab{x}{\ue}{i}{\labelset}}}{\aecase{e}{\mapschemab{x}{e}{i}{\labelset}}}{\tau}
  }
\end{equation}
\begin{equation}\label{rule:expandsU-syntax}
\inferrule{
  \istypeU{\Delta}{\tau}\\
  \expandsU{\emptyset}{\emptyset}{\emptyset}{\ueparse}{\eparse}{\aparr{\tBody}{\tParseResultExp}}\\\\
  a \notin \domof{\Sigma}\\
  \expandsU{\Delta}{\Gamma}{\Sigma, \tsmenvbnd{a}{\tau}{\eparse}}{\ue}{e}{\tau'}
}{
  \expandsUX{\auletsyntax{\tau}{\ueparse}{a}{\ue}}{e}{\tau'}
}
\end{equation}
\begin{equation}\label{rule:expandsU-tsmap}
\inferrule{
  \encodeBody{b}{\ebody}\\
  \evalU{\ap{\eparse}{\ebody}}{\inj{\lbltxt{Success}}{\ecand}}\\
  \decodeCondE{\ecand}{\ce}\\\\
  \cvalidE{\emptyset}{\emptyset}{\spctx{\Delta}{\Gamma}{\Sigma, \tsmenvbnd{a}{\tau}{\eparse}}{b}}{\ce}{e}{\tau}
}{
  \expandsU{\Delta}{\Gamma}{\Sigma, \tsmenvbnd{a}{\tau}{\eparse}}{\autsmap{b}{a}}{e}{\tau}
}
\end{equation}
\end{subequations}
Notice that each form of expanded expression (Figure \ref{fig:U-expanded-terms}) corresponds to a form of unexpanded expression (Figure \ref{fig:U-unexpanded-terms}). For each typing rule in Rules (\ref{rules:hastypeU}), there is a corresponding typed expansion rule -- Rules (\ref{rule:expandsU-var}) through (\ref{rule:expandsU-case}) -- where the unexpanded and expanded forms correspond. The premises also correspond -- if a typing judgement appears as a premise of a typing rule, then the corresponding premise in the corresponding typed expansion rule is the corresponding typed expansion judgement. The macro environment is not extended or inspected by these rules (it is only ``threaded through'' them opaquely).

There are two unexpanded expression forms that do not correspond to an expanded expression form: the macro definition form, and the macro application form. The rules governing these two forms interact with the macro environment, and are the topics of the next two subsections, respectively.

\subsection{Macro Definition}
The \emph{macro definition form}: $$\uletsyntax{a}{\tau}{\ueparse}{\ue}$$ allows the programmer to introduce a {macro} identified as $a$ at type $\tau$ with \emph{unexpanded parse function} $\ueparse$ into the macro environment of $\ue$. The abstract form corresponding to this stylized form is $\auletsyntax{\tau}{\ueparse}{a}{\ue}$. 
The premises of Rule (\ref{rule:expandsU-syntax}), which governs this form, can be understood as follows, in order:
\begin{enumerate}
\item The first premise ensures that the type annotation specifies a well-formed type, $\tau$.
\item The second premise types and expands the \emph{unexpanded parse function}, $\ueparse$, to produce the \emph{expanded parse function}, $\eparse$. Notice that this occurs under empty contexts, i.e. parse functions cannot refer to the surrounding variable bindings. This is because parse functions are evaluated when a macro is applied during the typed expansion process (as we will discuss momentarily), not during evaluation of the program they appear within. Parse functions must be of type \[\aparr{\tBody}{\tParseResultExp}\] where 
$\tParseResultExp$ abbreviates the following labeled sum type\footnote{In VerseML, the \li{ParseError} constructor required an error message and an error location, but we omit these in our formalization for simplicity}:
\[
\tParseResultExp \triangleq [\mapitem{\lbltxt{Success}}{\tCEExp}, \mapitem{\lbltxt{ParseError}}{\prodt{}}]
\] and 
 $\tBody$ and $\tCEExp$ abbreviate types that we will characterize below. 
\item Macro environments are finite mappings from macro identifiers, $a$, to \emph{expanded macro definitions}, $\xtsmdef{\tau}{\eparse}$, where $\tau$ is the macro's {type annotation} and $\eparse$ is the macro's {expanded parse function}. The \emph{macro environment formation judgement}, $\macenvOK{\Delta}{\Sigma}$, ensures that the type annotations in $\Sigma$ are well-formed assuming $\Delta$ and the parse functions in $\Sigma$ are  of type $\aparr{\tBody}{\tParseResultExp}$.
\[\begin{array}{ll}
\textbf{Judgement Form} & \textbf{Description}\\
\macenvOK{\Delta}{\Sigma} & \text{Macro environment $\Sigma$ is well-formed assuming $\Delta$.}\end{array}\]
This judgement is inductively defined by the following rules:
\begin{subequations}\label{rules:macenvOK-U}
\begin{equation}\label{rule:macenvOK-empty}
\inferrule{ }{\macenvOK{\Delta}{\emptyset}}
\end{equation}
\begin{equation}\label{rule:macenvOK-ext}
\inferrule{
  \macenvOK{\Delta}{\Sigma}\\\\
  \istypeU{\Delta}{\tau}\\
  \hastypeU{\emptyset}{\emptyset}{\eparse}{\aparr{\tBody}{\tParseResultExp}}
}{
  \macenvOK{\Delta}{\Sigma, \tsmenvbnd{a}{\tau}{\eparse}}
}
\end{equation}
\end{subequations}

The third premise of Rule (\ref{rule:expandsU-syntax}) checks that there is not already a macro identified as $a$ in the macro environment, $\Sigma$. 
\item 
The fourth premise of Rule (\ref{rule:expandsU-syntax}) extends the macro environment with the newly determined expanded macro definition and proceeds to produce a type, $\tau'$, and expansion, $e$, for $\ue$.
\end{enumerate}
The conclusion of Rule (\ref{rule:expandsU-syntax}) specifies $\tau'$ and $e$ as the type and expansion of the expression as a whole, i.e. macro definitions ``disappear'' when an unexpanded expression is expanded (because they specify behavior that is relevant only during typed expansion). 

\subsection{Macro Application}
The last form of unexpanded expression is the form for applying a macro identified as $a$ to a literal form with literal body $b$. 
\[
\utsmap{a}{b}
\] 
The stylized form for macro application, shown above, uses forward slashes as delimiters, though stylized variants of any of the literal forms specified in Figure \ref{fig:literal-forms} would be straightforward to add to the syntax table in Figure \ref{fig:U-unexpanded-terms} (we omit them for concision).

The abstract form corresponding to this stylized form is $\autsmap{b}{a}$, i.e. for each literal body, $b$, there is an operator $\texttt{utsmap}[b]$, indexed by the macro identifier $a$ and taking no arguments. Macro identifiers are symbolic, i.e. one can compare them, but unlike variables, they do not stand in for terms. Instead, they are simply references into the macro environment. 

The premises of Rule (\ref{rule:expandsU-tsmap}), which governs the macro application form, can be understood as follows, in order:
\begin{enumerate}
\item The \emph{body encoding judgement} $\encodeBody{b}{\ebody}$ specifies a mapping from the literal body, $b$, to an expanded value, $\ebody$, of type $\tBody$. An inverse mapping is specified by the \emph{body decoding judgement} $\decodeBody{\ebody}{b}$.
\[\begin{array}{ll}
\textbf{Judgement Form} & \textbf{Description}\\
\encodeBody{b}{e} & \text{$b$ encodes to $e$}\\
\decodeBody{e}{b} & \text{$e$ decodes to $b$}
\end{array}\]
Rather than picking a particular definition of $\tBody$ and inductively defining these judgements against it, we simply state the following sufficient conditions, which establish an isomorphism between literal bodies and values of type $\tBody$.
\begin{condition}[Body Encoding] For every literal body $b$, we have that $\encodeBody{b}{\ebody}$ and $\hastypeUC{\ebody}{\tBody}$ and $\isvalU{\ebody}$. \end{condition}
\begin{condition}[Body Decoding] If $\hastypeUC{\ebody}{\tBody}$ and $\isvalU{\ebody}$ then $\decodeBody{\ebody}{b}$ for some $b$. \end{condition}
\begin{condition}[Body Encoding Inverse] If $\encodeBody{b}{\ebody}$ then $\decodeBody{\ebody}{b}$. \end{condition}
\begin{condition}[Body Decoding Inverse] If $\hastypeUC{\ebody}{\tBody}$ and $\isvalU{\ebody}$ and $\decodeBody{\ebody}{b}$ then $\encodeBody{b}{\ebody}$. \end{condition}
\begin{condition}[Body Encoding Uniqueness] If $\encodeBody{b}{\ebody}$ and $\encodeBody{b}{\ebody'}$ then $\ebody = \ebody'$. \end{condition}
\begin{condition}[Body Decoding Uniqueness] If $\hastypeUC{\ebody}{\tBody}$ and $\isvalU{\ebody}$ and $\decodeBody{\ebody}{b}$ and $\decodeBody{\ebody}{b'}$ then $b=b'$. \end{condition}

\item The second premise applies the expanded parse function $\eparse$ associated with $a$ in the macro environment to $\ebody$. If parsing succeeds, i.e. a value of the (stylized) form $\inj{\lbltxt{Success}}{\ecand}$ results from evaluation, then $\ecand$ will be a value of type $\tCEExp$ (assuming a well-formed macro environment and by the definition of evaluation, transitive application of the Preservation Assumption \ref{condition:preservation-U} and the Canonical Forms Assumption \ref{condition:canonical-forms-U}). We call $\ecand$ the \emph{encoding of the candidate expansion}.

If the parse function produces a value labeled $\lbltxt{ParseError}$, then typed expansion fails. No rule is necessary to handle this case. 

\item The judgement $\decodeCondE{\ecand}{\ce}$ decodes the encoding of the candidate expansion, i.e. it maps $\ecand$ onto a \emph{candidate expansion expression}, $\ce$. The syntax of candidate expansion expressions and \emph{candidate expansion types}, $\ctau$, which can occur within candidate expansion expressions, is specified in Figure \ref{fig:U-candidate-terms}. The inverse mapping is specified by the judgement $\encodeCondE{\ce}{\ecand}$. 
\[\begin{array}{ll}
\textbf{Judgement Form} & \textbf{Description}\\
\decodeCondE{e}{\ce} & \text{$e$ decodes to $\ce$}\\
\encodeCondE{\ce}{e} & \text{$\ce$ encodes to $e$}
\end{array}\]

As with type $\tBody$, rather than defining the type $\tCEExp$ (and other associated types) explicitly, and these judgements inductively against these definitions, we give  the following sufficient conditions. These establish an isomorphism between values of type $\tCEExp$ and candidate expansion expressions.

\begin{condition}[Candidate Expansion Decoding] If $\hastypeUC{\ecand}{\tCEExp}$ and $\isvalU{\ecand}$ then $\decodeCondE{\ecand}{\ce}$ for some $\ce$. \end{condition}
\begin{condition}[Candidate Expansion Encoding] For every $\ce$, we have $\encodeCondE{\ce}{\ecand}$ such that $\hastypeUC{\ecand}{\tCEExp}$ and $\isvalU{\ecand}$. \end{condition}
\begin{condition}[Candidate Expansion Decoding Inverse] If $\hastypeUC{\ecand}{\tCEExp}$ and $\isvalU{\ecand}$ and $\decodeCondE{\ecand}{\ce}$ then $\encodeCondE{\ce}{\ecand}$. \end{condition}
\begin{condition}[Candidate Expansion Encoding Inverse] If $\encodeCondE{\ce}{\ecand}$ then $\decodeCondE{\ce}{\ecand}$. \end{condition}
\begin{condition}[Candidate Expansion Decoding Uniqueness] If $\hastypeUC{\ecand}{\tCEExp}$ and $\isvalU{\ecand}$ and $\decodeCondE{\ecand}{\ce}$ and $\decodeCondE{\ecand}{\ce'}$ then $\ce=\ce'$. \end{condition}
\begin{condition}[Candidate Expansion Encoding Uniqueness] If $\encodeCondE{\ce}{\ecand}$ and $\encodeCondE{\ce}{\ecand'}$ then $\ecand=\ecand'$. \end{condition}

\item The final premise of Rule (\ref{rule:expandsU-tsmap}) \emph{validates} the candidate expansion and simultaneously generates the \emph{final expansion}, $e$. This is the topic of the next subsection.
\end{enumerate}

\subsection{Candidate Expansion Validation}
\begin{figure}
$\begin{array}{lllllll}
\textbf{Sort} & & & \textbf{Abstract Form} & \textbf{Stylized Form} & \textbf{Description}\\
\mathsf{CETyp} & \ctau & ::= & t & t & \text{variable}\\
&&& \aceparr{\ctau}{\ctau} & \parr{\ctau}{\ctau} & \text{partial function}\\
&&& \aceall{t}{\ctau} & \forallt{t}{\ctau} & \text{polymorphic}\\
&&& \acerec{t}{\ctau} & \rect{t}{\ctau} & \text{recursive}\\
&&& \aceprod{\mapschema{\ctau}{i}{\labelset}} & \prodt{\mapschema{\ctau}{i}{\labelset}} & \text{labeled product}\\
&&& \acesum{\mapschema{\ctau}{i}{\labelset}} & \sumt{\mapschema{\ctau}{i}{\labelset}} & \text{labeled sum}\\
&&& \acesplicedt{m}{n} & \splicedt{m}{n} & \text{spliced}\\
\mathsf{CEExp} & \ce & ::= & x & x & \text{variable}\\
&&& \acelam{\ctau}{x}{\ce} & \lam{x}{\ctau}{\ce} & \text{abstraction}\\
&&& \aceap{\ce}{\ce} & \ap{\ce}{\ce} & \text{application}\\
&&& \acetlam{t}{\ce} & \Lam{t}{\ce} & \text{type abstraction}\\
&&& \acetap{\ce}{\ctau} & \App{\ce}{\ctau} & \text{type application}\\
&&& \acefold{t}{\ctau}{\ce} & \fold{\ce} & \text{fold}\\
&&& \aceunfold{\ce} & \unfold{\ce} & \text{unfold}\\
&&& \acetpl{\mapschema{\ce}{i}{\labelset}} & \tpl{\mapschema{\ce}{i}{\labelset}} & \text{labeled tuple}\\
&&& \acepr{\ell}{\ce} & \prj{\ce}{\ell} & \text{projection}\\
&&& \acein{\ell}{\mapschema{\ctau}{i}{\labelset}}{\ce} & \inj{\ell}{\ce} & \text{injection}\\
&&& \acecase{\ce}{\mapschemab{x}{\ce}{i}{\labelset}} & \caseof{\ce}{\mapschemab{x}{\ce}{i}{\labelset}} & \text{case analysis}\\
&&& \acesplicede{m}{n} & \splicede{m}{n} & \text{spliced}
\end{array}$
\caption[Syntax of candidate expansion types and expressions in $\miniVerseU$.]{Syntax of candidate expansion types, $\ctau$ (pronounced ``grave $\tau$''), and expressions, $\ce$ (pronounced ``grave $e$''), in $\miniVerseU$. Metavariables $m$ and $n$ range over natural numbers.}
\label{fig:U-candidate-terms}
\end{figure}

The \emph{candidate expansion validation judgements} validate candidate expansion types, $\ctau$, and candidate expansion expressions, $\ce$, and simultaneously generate their final expansions.
\[\begin{array}{ll}
\textbf{Judgement Form} & \textbf{Description}\\
\cvalidT{\Delta}{\spctxv}{\ctau}{\tau} & \text{Candidate expansion type $\ctau$ expands to $\tau$ under $\Delta$ and }\\
& \text{splicing scene $\spctxv$.}\\
\cvalidE{\Delta}{\Gamma}{\spctxv}{\ce}{e}{\tau} & \text{Candidate expansion expression $\ce$ expands to $e$ at type $\tau$}\\
& \text{under $\Delta$, $\Gamma$ and splicing scene $\spctxv$.}
\end{array}\]
\emph{Splicing scenes}, $\spctxv$, are of the form $\Delta; \Gamma; \Sigma; b$. They consist of the type formation context, $\Delta$, the typing context, $\Gamma$, the macro environment, $\Sigma$, and the literal body, $b$, from the macro application site (cf. Rule (\ref{rule:expandsU-tsmap})).

\subsubsection{Candidate Expansion Type Validation}

The \emph{candidate expansion type validation judgement}, $\cvalidT{\Delta}{\spctxv}{\ctau}{\tau}$, is inductively defined by the following rules:
\begin{subequations}\label{rules:cvalidT-U}
\begin{equation}\label{rule:cvalidT-U-tvar}
\inferrule{ }{
  \cvalidT{\Delta, \Dhyp{t}}{\spctxv}{t}{t}
}
\end{equation}
\begin{equation}\label{rule:cvalidT-U-parr}
  \inferrule{
    \cvalidT{\Delta}{\spctxv}{\ctau_1}{\tau_1}\\
    \cvalidT{\Delta}{\spctxv}{\ctau_2}{\tau_2}
  }{
    \cvalidT{\Delta}{\spctxv}{\aceparr{\ctau_1}{\ctau_2}}{\aparr{\tau_1}{\tau_2}}
  }
\end{equation}
\begin{equation}\label{rule:cvalidT-U-all}
  \inferrule {
    \cvalidT{\Delta, \Dhyp{t}}{\spctxv}{\ctau}{\tau}
  }{
    \cvalidT{\Delta}{\spctxv}{\aceall{t}{\ctau}}{\aall{t}{\tau}}
  }
\end{equation}
\begin{equation}\label{rule:cvalidT-U-rec}
  \inferrule{
    \cvalidT{\Delta, \Dhyp{t}}{\spctxv}{\ctau}{\tau}
  }{
    \cvalidT{\Delta}{\spctxv}{\acerec{t}{\ctau}}{\arec{t}{\tau}}
  }
\end{equation}
\begin{equation}\label{rule:cvalidT-U-prod}
  \inferrule{
    \{\cvalidT{\Delta}{\spctxv}{\ctau_i}{\tau_i}\}_{i \in \labelset}
  }{
    \cvalidT{\Delta}{\spctxv}{\aceprod{\mapschema{\ctau}{i}{\labelset}}}{\aprod{\mapschema{\tau}{i}{\labelset}}}
  }
\end{equation}
\begin{equation}\label{rule:cvalidT-U-sum}
  \inferrule{
    \{\cvalidT{\Delta}{\spctxv}{\ctau_i}{\tau_i}\}_{i \in \labelset}
  }{
    \cvalidT{\Delta}{\spctxv}{\acesum{\mapschema{\ctau}{i}{\labelset}}}{\asum{\mapschema{\tau}{i}{\labelset}}}
  }
\end{equation}
\begin{equation}\label{rule:cvalidT-U-splicedt}
  \inferrule{
    \parseTyp{\bsubseq{b}{m}{n}}{\tau}\\
    \Delta \cap \Delta_S = \emptyset\\
    \istypeU{\Delta_S}{\tau}
  }{
    \cvalidT{\Delta}{\spctx{\Gamma_S}{\Delta_S}{\Sigma_S}{b}}{\acesplicedt{m}{n}}{\tau}
  }
\end{equation}
\end{subequations}
Each form of type, $\tau$, corresponds to a form of candidate expansion type, $\ctau$ (compare Figures \ref{fig:U-expanded-terms} and \ref{fig:U-candidate-terms}). For each type formation rule in Rules (\ref{rules:istypeU}), there is a corresponding candidate expansion type validation rule -- Rules (\ref{rule:cvalidT-U-tvar}) to (\ref{rule:cvalidT-U-sum}) -- where the candidate expansion type and the final expansion correspond. The premises also correspond. 


The only form of candidate expansion type that does not correspond to a form of type is $\acesplicedt{m}{n}$, which is a \emph{reference to a spliced type}, i.e. it indicates that a type should be parsed out from the literal body, $b$, beginning at position $m$ and ending at position $n$. Rule (\ref{rule:cvalidT-U-splicedt}) governs this form. The first premise extracts the indicated subsequence of $b$ (using the metafunction $\bsubseq{b}{m}{n}$) and parses it (using the metafunction $\mathsf{parseTyp}(b)$) to produce the spliced type, $\tau$. We assume sensible definitions of these metafunctions.

 The third premise of Rule (\ref{rule:cvalidT-U-splicedt}) checks that $\tau$ is well-formed under the application site type formation context, $\Delta_S$. The second premise requires that the expansion's type formation context, $\Delta$, be disjoint from the application site type formation context, $\Delta_S$. This can always be achieved by alpha-varying the candidate expansion type that the reference to the spliced type appears within. Such a change in bound variable names cannot ``leak into'' spliced types because the hypotheses in $\Delta$ are not made available to the spliced type $\tau$. This achieves the \emph{expansion independent splicing} property described in Sec. \ref{sec:splicing-and-hygiene} for type variables. Rule (\ref{rule:cvalidT-U-splicedt}) is the only rule where $\Delta_S$ appears, so this also achieves the \emph{context-independent expansion} property described in Sec. \ref{sec:splicing-and-hygiene} for type variables.

\subsubsection{Candidate Expansion Expression Validation}
The \emph{candidate expansion expression validation judgement}, $\cvalidE{\Delta}{\Gamma}{\spctxv}{\ce}{e}{\tau}$, is defined mutually inductively with Rules (\ref{rules:expandsU}), because a typed expansion judgement appears as a premise in Rule (\ref{rule:cvalidE-U-splicede}) below, and a candidate expansion expression validation judgement appears as a premise in Rule (\ref{rule:expandsU-tsmap}) above.
\begin{subequations}\label{rules:cvalidE-U}
\begin{equation}\label{rule:cvalidE-U-var}
\inferrule{ }{
  \cvalidE{\Delta}{\Gamma, \Ghyp{x}{\tau}}{\spctxv}{x}{x}{\tau}
}
\end{equation}
\begin{equation}\label{rule:cvalidE-U-lam}
\inferrule{
  \cvalidT{\Delta}{\spctxv}{\ctau}{\tau}\\
  \cvalidE{\Delta}{\Gamma, \Ghyp{x}{\tau}}{\spctxv}{\ce}{e}{\tau'}
}{
  \cvalidE{\Delta}{\Gamma}{\spctxv}{\acelam{\ctau}{x}{\ce}}{\aelam{\tau}{x}{e}}{\aparr{\tau}{\tau'}}
}
\end{equation}
\begin{equation}\label{rule:cvalidE-U-ap}
  \inferrule{
    \cvalidE{\Delta}{\Gamma}{\spctxv}{\ce_1}{e_1}{\aparr{\tau}{\tau'}}\\
    \cvalidE{\Delta}{\Gamma}{\spctxv}{\ce_2}{e_2}{\tau}
  }{
    \cvalidE{\Delta}{\Gamma}{\spctxv}{\aceap{\ce_1}{\ce_2}}{\aeap{e_1}{e_2}}{\tau'}
  }
\end{equation}
\begin{equation}\label{rule:cvalidE-U-tlam}
  \inferrule{
    \cvalidE{\Delta, \Dhyp{t}}{\Gamma}{\spctxv}{\ce}{e}{\tau}
  }{
    \cvalidEX{\acetlam{t}{\ce}}{\aetlam{t}{e}}{\aall{t}{\tau}}
  }
\end{equation}
\begin{equation}\label{rule:cvalidE-U-tap}
  \inferrule{
    \cvalidEX{\ce}{e}{\aall{t}{\tau}}\\
    \cvalidT{\Delta}{\spctxv}{\ctau'}{\tau'}
  }{
    \cvalidEX{\acetap{\ce}{\ctau'}}{\aetap{e}{\tau'}}{[\tau'/t]\tau}
  }
\end{equation}
\begin{equation}\label{rule:cvalidE-U-fold}
  \inferrule{
    \cvalidT{\Delta, \Dhyp{t}}{\spctxv}{\ctau}{\tau}\\
    \cvalidEX{\ce}{e}{[\arec{t}{\tau}/t]\tau}
  }{
    \cvalidEX{\acefold{t}{\ctau}{\ce}}{\aefold{t}{\tau}{e}}{\arec{t}{\tau}}
  }
\end{equation}
\begin{equation}\label{rule:cvalidE-U-unfold}
  \inferrule{
    \cvalidEX{\ce}{e}{\arec{t}{\tau}}
  }{
    \cvalidEX{\aceunfold{\ce}}{\aeunfold{e}}{[\arec{t}{\tau}/t]\tau}
  }
\end{equation}
\begin{equation}\label{rule:cvalidE-U-tpl}
  \inferrule{
    \{\cvalidEX{\ce_i}{e_i}{\tau_i}\}_{i \in \labelset}
  }{
    \cvalidEX{\acetpl{\mapschema{\ce}{i}{\labelset}}}{\aetpl{\mapschema{e}{i}{\labelset}}}{\aprod{\mapschema{\tau}{i}{\labelset}}}
  }
\end{equation}
\begin{equation}\label{rule:cvalidE-U-pr}
  \inferrule{
    \cvalidEX{\ce}{e}{\aprod{\mapschema{\tau}{i}{\labelset}\otimes\mapitem{\ell}{\tau}}}
  }{
    \cvalidEX{\acepr{\ell}{\ce}}{\aepr{\ell}{e}}{\tau}
  }
\end{equation}
\begin{equation}\label{rule:cvalidE-U-in}
  \inferrule{
    \{\cvalidT{\Delta}{\spctxv}{\ctau_i}{\tau_i}\}_{i \in \labelset}\\
    \cvalidT{\Delta}{\spctxv}{\ctau}{\tau}\\
    \cvalidEX{\ce}{e}{\tau}
  }{
    \left\{\shortstack{$\Delta~\Gamma \vdash_\Sigma \acein{\ell}{\mapschema{\ctau}{i}{\labelset}\otimes\mapitem{\ell}{\ctau}}{\ce}$\\$\leadsto$\\$\aein{\ell}{\mapschema{\tau}{i}{\labelset}\otimes\mapitem{\ell}{\tau}}{e} : \asum{\mapschema{\tau}{i}{\labelset}\otimes\mapitem{\ell}{\tau}}$\vspace{-1.2em}}\right\}
  }
\end{equation}
\begin{equation}\label{rule:cvalidE-U-case}
  \inferrule{
    \cvalidEX{\ce}{e}{\asum{\mapschema{\tau}{i}{\labelset}}}\\
    \{\cvalidE{\Delta}{\Gamma, \Ghyp{x_i}{\tau_i}}{\spctxv}{\ue_i}{e_i}{\tau}\}_{i \in \labelset}
  }{
    \cvalidEX{\acecase{\ce}{\mapschemab{x}{\ce}{i}{\labelset}}}{\aecase{e}{\mapschemab{x}{e}{i}{\labelset}}}{\tau}
  }
\end{equation}
\begin{equation}\label{rule:cvalidE-U-splicede}
\inferrule{
  \parseUExp{\bsubseq{b}{m}{n}}{\ue}\\\\
  \Delta \cap \Delta_S = \emptyset\\
  \domof{\Gamma} \cap \domof{\Gamma_S} = \emptyset\\
  \expandsU{\Delta_S}{\Gamma_S}{\Sigma_S}{\ue}{e}{\tau}
}{
  \cvalidE{\Delta}{\Gamma}{\spctx{\Delta_S}{\Gamma_S}{\Sigma_S}{b}}{\acesplicede{m}{n}}{e}{\tau}
}
\end{equation}
\end{subequations}

Each form of expanded expression, $e$, corresponds to a form of candidate expansion expression, $\ce$ (compare Figure \ref{fig:U-expanded-terms} and Figure \ref{fig:U-candidate-terms}). For each typing rule in Rules \ref{rules:hastypeU}, there is a corresponding candidate expansion expression validation rule -- Rules (\ref{rule:cvalidE-U-var}) to (\ref{rule:cvalidE-U-case}) -- where the candidate expansion expression and expanded expression correspond. The premises also correspond.

The only form of candidate expansion expression that does not correspond to a form of expanded expression is $\acesplicede{m}{n}$, which is a \emph{reference to a spliced unexpanded expression}, i.e. it indicates that an unexpanded expression should be parsed out from the literal body, $b$, beginning at position $m$ and ending at position $n$. Rule (\ref{rule:cvalidE-U-splicede}) governs this form. The first premise extracts the indicated subsequence of $b$ (using the metafunction $\bsubseq{b}{m}{n}$) and parses it (using the metafunction $\mathsf{parseUExp}(b)$) to produce the spliced unexpanded expression, $\ue$. We assume sensible definitions of these metafunctions.

The final premise of Rule (\ref{rule:cvalidE-U-splicede}) types and expands the spliced unexpanded expression $\ue$ under the application site contexts, $\Delta_S$ and $\Gamma_S$, and macro environment, $\Sigma_S$. The second premise requires that the expansion's type formation context, $\Delta$, be disjoint from the application site type formation context, $\Delta_S$. Similarly, the third premise requires that the expansion's typing context, $\Gamma$, be disjoint from the application site typing context, $\Gamma_S$. These requirements can always be satisfied by alpha-varying the candidate expansion expression that the reference to the spliced unexpanded expression appears within. Such a change in bound variable names cannot ``leak into'' spliced unexpanded expressions because the hypotheses in $\Delta$ and $\Gamma$ are not  available to the spliced unexpanded expression $\ue$. This achieves the \emph{expansion independent splicing} property described in Sec. \ref{sec:splicing-and-hygiene} for variables and type variables. Rule (\ref{rule:cvalidE-U-splicede}) is the only rule where $\Delta_S$, $\Gamma_S$ and $\Sigma_S$ appear. This achieves the context-independent expansion property described in Sec. \ref{sec:splicing-and-hygiene} for variables and type variables.

Candidate expansions cannot themselves define or apply TSMs. This simplifies our metatheory. We discuss relaxing this restriction in Sec. \ref{sec:tsms-in-expansions}.


\subsection{Metatheory}
For the judgements we have specified to form a sensible language, we must have that typed expansion of unexpanded expressions be consistent with typing of expanded expressions. Formally, this can be expressed as the following theorem.

\begin{theorem}[Typed Expansion]\label{thm:typed-expansion-U}
If $\expandsU{\Delta}{\Gamma}{\Sigma}{\ue}{e}{\tau}$ and $\macenvOK{\Delta}{\Sigma}$ then $\hastypeU{\Delta}{\Gamma}{e}{\tau}$. 
\end{theorem}
\begin{proof}
By rule induction over Rules (\ref{rules:expandsU}).
\begin{byCases}
\item[\text{(\ref{rule:expandsU-var})}] We have
\begin{pfsteps}
  \item \ue=x \BY{assumption}
  \item e=x \BY{assumption}
  \item \Gamma=\Gamma', \Ghyp{x}{\tau} \BY{assumption}
  \item \hastypeU{\Delta}{\Gamma', \Ghyp{x}{\tau}}{x}{\tau} \BY{Rule (\ref{rule:hastypeU-var})}
\end{pfsteps}
\resetpfcounter

\item[\text{(\ref{rule:expandsU-lam})}] We have 
\begin{pfsteps}
  \item \ue=\aulam{\tau_1}{x}{\ue'} \BY{assumption}
  \item e=\aelam{\tau_1}{x}{e'} \BY{assumption}
  \item \tau=\aparr{\tau_1}{\tau_2} \BY{assumption}
  \item \istypeU{\Delta}{\tau_1} \BY{assumption} \pflabel{istype}
  \item \expandsU{\Delta}{\Gamma, \Ghyp{x}{\tau_1}}{\Sigma}{\ue'}{e'}{\tau_2} \BY{assumption} \pflabel{expandsU}
  \item \macenvOK{\Delta}{\Sigma} \BY{assumption} \pflabel{macenvOK}
  \item \hastypeU{\Delta}{\Gamma, \Ghyp{x}{\tau_1}}{e'}{\tau_2} \BY{IH on \pfref{expandsU} and \pfref{macenvOK}} \pflabel{hastypeU}
  \item \hastypeU{\Delta}{\Gamma}{\aelam{\tau_1}{x}{e'}}{\aparr{\tau_1}{\tau_2}} \BY{Rule (\ref{rule:hastypeU-lam}) on \pfref{istype} and \pfref{hastypeU}}
\end{pfsteps}
\resetpfcounter

\item[\text{(\ref{rule:expandsU-ap})}] We have
\begin{pfsteps}
  \item \ue=\auap{\ue_1}{\ue_2} \BY{assumption}
  \item e=\aeap{e_1}{e_2} \BY{assumption}
  \item \expandsU{\Delta}{\Gamma}{\Sigma}{\ue_1}{e_1}{\aparr{\tau_1}{\tau}} \BY{assumption}\pflabel{expandsU1}
  \item \expandsU{\Delta}{\Gamma}{\Sigma}{\ue_2}{e_2}{\tau_1} \BY{assumption}\pflabel{expandsU2}
  \item \macenvOK{\Delta}{\Sigma} \BY{assumption} \pflabel{macenvOK}
  \item \hastypeU{\Delta}{\Gamma}{e_1}{\aparr{\tau_1}{\tau}} \BY{IH on \pfref{expandsU1} and \pfref{macenvOK}}\pflabel{hastypeU1}
  \item \hastypeU{\Delta}{\Gamma}{e_2}{\tau_1} \BY{IH on \pfref{expandsU2} and \pfref{macenvOK}}\pflabel{hastypeU2}
  \item \hastypeU{\Delta}{\Gamma}{\aeap{e_1}{e_2}}{\tau} \BY{Rule (\ref{rule:hastypeU-ap}) on \pfref{hastypeU1} and \pfref{hastypeU2}}
\end{pfsteps}
\resetpfcounter

\item[\text{(\ref{rule:expandsU-tlam}) through (\ref{rule:expandsU-case})}] These cases follow analagously (i.e. we apply the IH to all typed expansion premises and then apply the corresponding typing rule.)
\\

\item[\text{(\ref{rule:expandsU-syntax})}] We have 
\begin{pfsteps}
  \item \ue=\auletsyntax{\tau'}{\ueparse}{a}{\ue'} \BY{assumption}
  \item \istypeU{\Delta}{\tau'} \BY{assumption}\pflabel{istype}
  \item \expandsU{\emptyset}{\emptyset}{\emptyset}{\ueparse}{\eparse}{\aparr{\tBody}{\tParseResultExp}} \BY{assumption}\pflabel{ueparse}
  \item \expandsU{\Delta}{\Gamma}{\Sigma, \tsmenvbnd{a}{\tau'}{\eparse}}{\ue'}{e}{\tau} \BY{assumption}\pflabel{expandsU}
  \item \macenvOK{\Delta}{\Sigma} \BY{assumption}\pflabel{macenvOK1}
  \item \macenvOK{\emptyset}{\emptyset} \BY{Rule (\ref{rule:macenvOK-empty})}\pflabel{macenvOK2}
  \item \hastypeU{\emptyset}{\emptyset}{\eparse}{\aparr{\tBody}{\tParseResultExp}} \BY{IH on \pfref{ueparse} and \pfref{macenvOK2}}\pflabel{eparse2}
  \item \macenvOK{\Delta}{\Sigma, \tsmenvbnd{a}{\tau'}{\eparse}} \BY{Rule (\ref{rule:macenvOK-ext}) on \pfref{macenvOK1}, \pfref{istype} and \pfref{eparse2}}\pflabel{macenvOK3}
  \item \hastypeU{\Delta}{\Gamma}{e}{\tau} \BY{IH on \pfref{expandsU} and \pfref{macenvOK3}}
\end{pfsteps}
\resetpfcounter 

\item[\text{(\ref{rule:expandsU-tsmap})}] We have 
\begin{pfsteps}
  \item \ue=\autsmap{b}{a} \BY{assumption}
  \item \Sigma=\Sigma', \tsmenvbnd{a}{\tau}{\eparse} \BY{assumption}
  \item \encodeBody{b}{\ebody} \BY{assumption}
  \item \evalU{\eparse(\ebody)}{\inj{\lbltxt{Success}}{\ecand}} \BY{assumption}
  \item \decodeCondE{\ecand}{\ce} \BY{assumption}
  \item \cvalidE{\emptyset}{\emptyset}{\spctx{\Delta}{\Gamma}{\Sigma}{b}}{\ce}{e}{\tau} \BY{assumption}\pflabel{cvalidE}
  \item \macenvOK{\Delta}{\Sigma} \BY{assumption} \pflabel{macenvOK}
  \item \hastypeU{\Delta}{\Gamma}{e}{\tau} \BY{Theorem \ref{thm:candidate-expansion-validation-U} on \pfref{cvalidE} and \pfref{macenvOK}}
\end{pfsteps}
\resetpfcounter
\end{byCases}
\end{proof}
We need to define the following theorem about candidate expansion expression validation mutually with Theorem \ref{thm:typed-expansion-U}. 
\begin{theorem}[Candidate Expansion Expression Validation]\label{thm:candidate-expansion-validation-U}
If $\cvalidE{\Delta}{\Gamma}{\spctx{\Delta_S}{\Gamma_S}{\Sigma_S}{b}}{\ce}{e}{\tau}$ and $\macenvOK{\Delta_S}{\Sigma_S}$ then $\hastypeU{\Dcons{\Delta}{\Delta_S}}{\Gcons{\Gamma}{\Gamma_S}}{e}{\tau}$.
\end{theorem}
\begin{proof} By rule induction over Rules (\ref{rules:cvalidE-U}).
\begin{byCases}
\item[\text{(\ref{rule:cvalidE-U-var})}] We have
\begin{pfsteps*}
  \item $\ce=x$ \BY{assumption}
  \item $e=x$ \BY{assumption}
  \item $\Gamma=\Gamma', \Ghyp{x}{\tau}$ \BY{assumption}
  \item $\hastypeU{\Dcons{\Delta}{\Delta_S}}{\Gamma', \Ghyp{x}{\tau}}{x}{\tau}$ \BY{Rule (\ref{rule:hastypeU-var})} \pflabel{hastypeU}
  \item $\hastypeU{\Dcons{\Delta}{\Delta_S}}{\Gcons{\Gamma', \Ghyp{x}{\tau}}{\Gamma_S}}{x}{\tau}$ \BY{Lemma \ref{lemma:weakening-U} over $\Gamma_S$ to \pfref{hastypeU}}
\end{pfsteps*}
\resetpfcounter

\item[\text{(\ref{rule:cvalidE-U-lam})}] We have
\begin{pfsteps*}
  \item $\ce=\acelam{\ctau_1}{x}{\ce'}$ \BY{assumption}
  \item $e=\aelam{\tau_1}{x}{e'}$ \BY{assumption}
  \item $\tau=\aparr{\tau_1}{\tau_2}$ \BY{assumption}
  \item $\cvalidT{\Delta}{\spctx{\Delta_S}{\Gamma_S}{\Sigma_S}{b}}{\ctau_1}{\tau_1}$ \BY{assumption} \pflabel{cvalidT}
  \item $\cvalidE{\Delta}{\Gamma, \Ghyp{x}{\tau_1}}{\spctx{\Delta_S}{\Gamma_S}{\Sigma_S}{b}}{\ce'}{e'}{\tau_2}$ \BY{assumption} \pflabel{cvalidE}
  \item $\macenvOK{\Delta_S}{\Sigma_S}$ \BY{assumption} \pflabel{macenvOK}
  \item $\istypeU{\Dcons{\Delta}{\Delta_S}}{\tau_1}$ \BY{Lemma \ref{lemma:candidate-expansion-type-validation} on \pfref{cvalidT}} \pflabel{istype}
  \item $\hastypeU{\Dcons{\Delta}{\Delta_S}}{\Gcons{\Gamma, \Ghyp{x}{\tau_1}}{\Gamma_S}}{e'}{\tau_2}$ \BY{IH on \pfref{cvalidE} and \pfref{macenvOK}} \pflabel{hastype1}
  \item $\hastypeU{\Dcons{\Delta}{\Delta_S}}{\Gcons{\Gamma}{\Gamma_S}, \Ghyp{x}{\tau_1}}{e'}{\tau_2}$ \BY{exchange over $\Gamma_S$ on \pfref{hastype1}} \pflabel{hastype2}
  \item $\hastypeU{\Dcons{\Delta}{\Delta_S}}{\Gcons{\Gamma}{\Gamma_S}}{\aelam{\tau_1}{x}{e'}}{\aparr{\tau_1}{\tau_2}}$ \BY{Rule (\ref{rule:hastypeU-lam}) on \pfref{istype} and \pfref{hastype2}}
\end{pfsteps*}
\resetpfcounter

\item[\text{(\ref{rule:cvalidE-U-ap})}] We have
\begin{pfsteps*}
  \item $\ce=\aceap{\ce_1}{\ce_2}$ \BY{assumption}
  \item $e=\aeap{e_1}{e_2}$ \BY{assumption}
  \item $\cvalidE{\Delta}{\Gamma}{\spctx{\Delta_S}{\Gamma_S}{\Sigma_S}{b}}{\ce_1}{e_1}{\aparr{\tau_1}{\tau}}$ \BY{assumption} \pflabel{cvalidE1}
  \item $\cvalidE{\Delta}{\Gamma}{\spctx{\Delta_S}{\Gamma_S}{\Sigma_S}{b}}{\ce_2}{e_2}{\tau_1}$ \BY{assumption} \pflabel{cvalidE2}
  \item $\macenvOK{\Delta_S}{\Sigma_S}$ \BY{assumption} \pflabel{macenvOK}
  \item $\hastypeU{\Dcons{\Delta}{\Delta_S}}{\Gcons{\Gamma}{\Gamma_S}}{e_1}{\aparr{\tau_1}{\tau}}$ \BY{IH on \pfref{cvalidE1} and \pfref{macenvOK}} \pflabel{hastypeU1}
  \item $\hastypeU{\Dcons{\Delta}{\Delta_S}}{\Gcons{\Gamma}{\Gamma_S}}{e_2}{\tau_1}$ \BY{IH on \pfref{cvalidE2} and \pfref{macenvOK}} \pflabel{hastypeU2}
  \item $\hastypeU{\Dcons{\Delta}{\Delta_S}}{\Gcons{\Gamma}{\Gamma_S}}{\aeap{e_1}{e_2}}{\tau}$ \BY{Rule (\ref{rule:hastypeU-ap}) on \pfref{hastypeU1} and \pfref{hastypeU2}}
\end{pfsteps*}
\resetpfcounter

\item[\text{(\ref{rule:cvalidE-U-tlam}) through (\ref{rule:cvalidE-U-case})}] These cases follow analagously (i.e. we apply the IH to all candidate expansion expression validation premises, Lemma \ref{lemma:candidate-expansion-type-validation} to all candidate expansion type validation premises, weakening and exchange as needed, and then apply the corresponding typing rule.)
\\

\item[\text{(\ref{rule:cvalidE-U-splicede})}] We have
\begin{pfsteps*}
  \item $\ce=\acesplicede{m}{n}$ \BY{assumption}
  \item $\parseUExp{\bsubseq{b}{m}{n}}{\ue}$ \BY{assumption}
  \item $\expandsU{\Delta_S}{\Gamma_S}{\Sigma_S}{\ue}{e}{\tau}$ \BY{assumption} \pflabel{expands}
  \item $\macenvOK{\Delta_S}{\Sigma_S}$ \BY{assumption} \pflabel{macenvok}
  \item $\hastypeU{\Delta_S}{\Gamma_S}{e}{\tau}$ \BY{Theorem \ref{thm:typed-expansion-U} on \pfref{expands} and \pfref{macenvok}} \pflabel{hastype}
  \item $\hastypeU{\Dcons{\Delta}{\Delta_S}}{\Gcons{\Gamma}{\Gamma_S}}{e}{\tau}$ \BY{Lemma \ref{lemma:weakening-U} on \pfref{hastype}}
\end{pfsteps*}
\resetpfcounter
\end{byCases}
\end{proof}

The mutual induction used to prove Theorems \ref{thm:typed-expansion-U} and \ref{thm:candidate-expansion-validation-U} can be shown to be well-founded by showing that the following numeric metric on the judgements that we induct over is decreasing:
\begin{align*}
\sizeof{\expandsU{\Delta}{\Gamma}{\Sigma}{\ue}{e}{\tau}} & = \sizeof{\ue}\\
\sizeof{\cvalidE{\Delta}{\Gamma}{\spctx{\Delta_S}{\Gamma_S}{\Sigma_S}{b}}{\ce}{e}{\tau}} & = \sizeof{b}
\end{align*}
Here, $\sizeof{b}$ is the length of $b$ and $\sizeof{\ue}$ is the sum of the lengths of the literal bodies in $\ue$, i.e. 
\begin{align*}
\sizeof{x} & = 0\\
\sizeof{\aulam{\tau}{x}{\ue}} &= \sizeof{\ue}\\
\sizeof{\auap{\ue_1}{\ue_2}} & = \sizeof{\ue_1} + \sizeof{\ue_2}\\
\sizeof{\autlam{t}{\ue}} & = \sizeof{\ue}\\
\sizeof{\autap{\ue}{\tau}} & = \sizeof{\ue}\\
\sizeof{\aufold{t}{\tau}{\ue}} & = \sizeof{\ue}\\
\sizeof{\auunfold{\ue}} & = \sizeof{\ue}\\
\sizeof{\autpl{\mapschema{\ue}{i}{\labelset}}} & = \sum_{i \in \labelset} \sizeof{\ue_i}\\
\sizeof{\aupr{\ell}{\ue}} & = \sizeof{\ue}\\
\sizeof{\auin{\ell}{\mapschema{\tau}{i}{\labelset}}{\ue}} & = \sizeof{\ue}\\
\sizeof{\aucase{\ue}{\mapschemab{x}{\ue}{i}{\labelset}}} & = \sizeof{\ue} + \sum_{i \in \labelset} \sizeof{\ue_i}\\
\sizeof{\auletsyntax{\tau}{\ueparse}{a}{\ue}} & = \sizeof{\ue}\\
\sizeof{\autsmap{b}{a}} & = \sizeof{b}
\end{align*}

The only case in the proof of Theorem \ref{thm:typed-expansion-U} that invokes Theorem \ref{thm:candidate-expansion-validation-U} is Case (\ref{rule:expandsU-tsmap}). There, we have that the metric remains stable: \[\sizeof{\cvalidE{\emptyset}{\emptyset}{\spctx{\Delta}{\Gamma}{\Sigma}{b}}{\ce}{e}{\tau}} = \sizeof{\expandsU{\Delta}{\Gamma}{\Sigma}{\autsmap{b}{a}}{e}{\tau}}=\sizeof{b}\]

The only case in the proof of Theorem \ref{thm:candidate-expansion-validation-U} that invokes Theorem \ref{thm:typed-expansion-U} is Case (\ref{rule:cvalidE-U-splicede}). There, we have that $\parseUExp{\bsubseq{b}{m}{n}}{\ue}$ and Theorem \ref{thm:typed-expansion-U} is invoked on the judgement $\expandsU{\Delta_S}{\Gamma_S}{\Sigma_S}{\ue}{e}{\tau}$. Because the metric is stable when passing from Theorem \ref{thm:typed-expansion-U} to Theorem \ref{thm:candidate-expansion-validation-U}, we must have that it is strictly decreasing in the other direction:
\[\sizeof{\expandsU{\Delta_S}{\Gamma_S}{\Sigma_S}{\ue}{e}{\tau}} < \sizeof{\cvalidE{\Delta}{\Gamma}{\spctx{\Delta_S}{\Gamma_S}{\Sigma_S}{b}}{\acesplicede{m}{n}}{e}{\tau}}\]
i.e. by the definitions above, 
\[\sizeof{\ue} < \sizeof{b}\]

This is established by appeal to the following two conditions. The first condition simply states that subsequences of $b$ are no longer than $b$.
\begin{condition}[Body Subsequences]\label{condition:body-subsequences} If $\bsubseq{b}{m}{n}=b'$ then $\sizeof{b'} \leq \sizeof{b}$. \end{condition}
The second condition states that an unexpanded expression constructed by parsing a literal body $b$ is strictly smaller, as measured by the metric defined above, than the length of $b$ (because some characters must necessarily be used to invoke a TSM on and delimit each literal body.)
\begin{condition}[Body Parsing]\label{condition:body-parsing} If $\parseUExp{b}{\ue}$ then $\sizeof{\ue} < \sizeof{b}$.\end{condition}

Combining Conditions \ref{condition:body-subsequences} and \ref{condition:body-parsing}, we have that $\sizeof{\ue} < \sizeof{b}$ as needed.

\qed

Finally, the proof of Theorem \ref{thm:candidate-expansion-validation-U} invokes the following lemma about candidate expansion type validation.
\begin{lemma}[Candidate Expansion Type Validation]\label{lemma:candidate-expansion-type-validation}
If $\cvalidT{\Delta}{\spctx{\Delta_S}{\Gamma_S}{\Sigma_S}{b}}{\ctau}{\tau}$ then $\istypeU{\Dcons{\Delta}{\Delta_S}}{\tau}$.
\end{lemma}
\begin{proof} By rule induction over Rules (\ref{rules:cvalidT-U}).
\begin{byCases}
\item[\text{(\ref{rule:cvalidT-U-tvar})}] We have 
\begin{pfsteps*}
   \item $\Delta=\Delta', \Dhyp{t}$ \BY{assumption}
   \item $\ctau=t$ \BY{assumption}
   \item $\tau=t$ \BY{assumption}
   \item $\istypeU{\Delta', \Dhyp{t}}{t}$ \BY{Rule (\ref{rule:istypeU-var})} \pflabel{istype}
   \item $\istypeU{\Dcons{\Delta', \Dhyp{t}}{\Delta_S}}{t}$ \BY{Lemma \ref{lemma:weakening-U} over $\Delta_S$ to \pfref{istype}.}
 \end{pfsteps*} 
\resetpfcounter

\item[\text{(\ref{rule:cvalidT-U-parr})}] We have
\begin{pfsteps*}
  \item $\ctau=\aceparr{\ctau_1}{\ctau_2}$ \BY{assumption}
  \item $\tau=\aparr{\tau_1}{\tau_2}$ \BY{assumption}
  \item $\cvalidT{\Delta}{\spctx{\Delta_S}{\Gamma_S}{\Sigma_S}{b}}{\ctau_1}{\tau_1}$ \BY{assumption} \pflabel{cvalid-ctau1}
  \item $\cvalidT{\Delta}{\spctx{\Delta_S}{\Gamma_S}{\Sigma_S}{b}}{\ctau_2}{\tau_2}$ \BY{assumption} \pflabel{cvalid-ctau2}
  \item $\istypeU{\Dcons{\Delta}{\Delta_S}}{\tau_1}$ \BY{IH on \pfref{cvalid-ctau1}} \pflabel{istype1}
  \item $\istypeU{\Dcons{\Delta}{\Delta_S}}{\tau_2}$ \BY{IH on \pfref{cvalid-ctau2}} \pflabel{istype2}
  \item $\istypeU{\Dcons{\Delta}{\Delta_S}}{\aparr{\tau_1}{\tau_2}}$ \BY{Rule (\ref{rule:istypeU-parr}) on \pfref{istype1} and \pfref{istype2}}
\end{pfsteps*}
\resetpfcounter

\item[\text{(\ref{rule:cvalidT-U-all})}] We have
\begin{pfsteps*}
  \item $\ctau=\aceall{t}{\ctau'}$ \BY{assumption}
  \item $\tau=\aall{t}{\tau'}$ \BY{assumption}
  \item $\cvalidT{\Delta, \Dhyp{t}}{\spctx{\Delta_S}{\Gamma_S}{\Sigma_S}{b}}{\ctau'}{\tau'}$ \BY{assumption} \label{cvalidT}
  \item $\istypeU{\Dcons{\Delta, \Dhyp{t}}{\Delta_S}}{\tau'}$ \BY{IH on \pfref{cvalidT}} \pflabel{istypeU1}
  \item $\istypeU{\Dcons{\Delta}{\Delta_S}, \Dhyp{t}}{\tau'}$ \BY{exchange over $\Delta_S$ on \pfref{istypeU1}} \pflabel{istypeU2}
  \item $\istypeU{\Dcons{\Delta}{\Delta_S}}{\aall{t}{\tau'}}$ \BY{Rule (\ref{rule:istypeU-all}) on \pfref{istypeU2}}
\end{pfsteps*}
\resetpfcounter

\item[\text{(\ref{rule:cvalidT-U-rec})}] We have
\begin{pfsteps*}
  \item $\ctau=\acerec{t}{\ctau'}$ \BY{assumption}
  \item $\tau=\arec{t}{\tau'}$ \BY{assumption}
  \item $\cvalidT{\Delta, \Dhyp{t}}{\spctx{\Delta_S}{\Gamma_S}{\Sigma_S}{b}}{\ctau'}{\tau'}$ \BY{assumption} \label{cvalidT}
  \item $\istypeU{\Dcons{\Delta, \Dhyp{t}}{\Delta_S}}{\tau'}$ \BY{IH on \pfref{cvalidT}} \pflabel{istypeU1}
  \item $\istypeU{\Dcons{\Delta}{\Delta_S}, \Dhyp{t}}{\tau'}$ \BY{exchange over $\Delta_S$ on \pfref{istypeU1}} \pflabel{istypeU2}
  \item $\istypeU{\Dcons{\Delta}{\Delta_S}}{\arec{t}{\tau'}}$ \BY{Rule (\ref{rule:istypeU-rec}) on \pfref{istypeU2}}
\end{pfsteps*}
\resetpfcounter

\item[\text{(\ref{rule:cvalidT-U-prod})}] We have
\begin{pfsteps*}
\item $\ctau=\aceprod{\mapschema{\ctau}{i}{\labelset}}$ \BY{assumption}  
\item $\tau=\aprod{\mapschema{\tau}{i}{\labelset}}$ \BY{assumption}
\item $\{\cvalidT{\Delta}{\spctx{\Delta_S}{\Gamma_S}{\Sigma_S}{b}}{\ctau_i}{\tau_i}\}_{i \in \labelset}$ \BY{assumption} \pflabel{cvalidT-ass}
\item $\{\istypeU{\Dcons{\Delta}{\Delta_S}}{\tau_i}\}_{i \in \labelset}$ \BY{IH on \pfref{cvalidT-ass}$_i$ for each $i \in \labelset$} \pflabel{istype}
\item $\istypeU{\Dcons{\Delta}{\Delta_S}}{\aprod{\mapschema{\tau}{i}{\labelset}}}$ \BY{Rule (\ref{rule:istypeU-prod}) on \pfref{istype}}
\end{pfsteps*}
\resetpfcounter 

\item[\text{(\ref{rule:cvalidT-U-sum})}] We have
\begin{pfsteps*}
\item $\ctau=\acesum{\mapschema{\ctau}{i}{\labelset}}$ \BY{assumption}  
\item $\tau=\asum{\mapschema{\tau}{i}{\labelset}}$ \BY{assumption}
\item $\{\cvalidT{\Delta}{\spctx{\Delta_S}{\Gamma_S}{\Sigma_S}{b}}{\ctau_i}{\tau_i}\}_{i \in \labelset}$ \BY{assumption} \pflabel{cvalidT-ass}
\item $\{\istypeU{\Dcons{\Delta}{\Delta_S}}{\tau_i}\}_{i \in \labelset}$ \BY{IH on \pfref{cvalidT-ass}$_i$ for each $i \in \labelset$} \pflabel{istype}
\item $\istypeU{\Dcons{\Delta}{\Delta_S}}{\asum{\mapschema{\tau}{i}{\labelset}}}$ \BY{Rule (\ref{rule:istypeU-sum}) on \pfref{istype}}
\end{pfsteps*}
\resetpfcounter

\item[\text{(\ref{rule:cvalidT-U-splicedt})}] We have
\begin{pfsteps*}
\item $\ctau=\acesplicedt{m}{n}$ \BY{assumption}
\item $\parseTyp{\bsubseq{b}{m}{n}}{\tau}$ \BY{assumption}
\item $\istypeU{\Delta_S}{\tau}$ \BY{assumption}\pflabel{istype}
\item $\istypeU{\Dcons{\Delta}{\Delta_S}}{\tau}$ \BY{Lemma \ref{lemma:weakening-U} over \pfref{istype} and exchange}
\end{pfsteps*}
\end{byCases}
\end{proof}

\chapter{Unparameterized Pattern TSMs}\label{sec:pattern-tsms}
In full-scale functional languages like ML and Haskell, one typically deconstructs a value using \emph{nested pattern matching}. For example, let us return to the definition of the recursive labeled sum type \lstinline{Rx} shown in Figure \ref{fig:datatype-rx}. We can pattern match over a value, \lstinline{r}, of type \lstinline{Rx} using VerseML's \lstinline{match} construct like this:
\begin{lstlisting}
fun read_example_rx(r : Rx) : (string * Rx) option => 
  match r with 
    Seq(Str(name), Seq(Str "SSTR: ESTR", ssn)) => Some (name, ssn)
  | _ => None
\end{lstlisting}

Match expressions consist of a \emph{scrutinee}, here \li{r}, and a sequence of \emph{rules}, separated by a vertical bar, \li{|}, in the concrete syntax. Each rule consists of a \emph{pattern} and an {expression} called the \emph{branch expression}, separated by a double arrow, \li{=>}, in the concrete syntax. When a {match} expression is evaluated, the value of the scrutinee is matched against each pattern sequentially. If the value matches, evaluation branches to the corresponding branch expression. A variable pattern matches any value of the appropriate type. The matching value is available through that variable in the corresponding expression. Each variable bound by a pattern must be distinct. 
 
For example, in the rule on Line 3, the pattern \li{Seq(Str(name), Seq(Str ": ", ssn))} matches values of the form \li{Seq(Str(#$e_1$#), Seq(Str "SSTR: ESTR", #$e_2$#))}, where $e_1$ is a value of type \li{string} and $e_2$ is a value of type \li{Rx}. The values of $e_1$ and $e_2$ are available for use in the expression \li{Some(name, ssn)} through the variables \li{name} and \li{ssn}, respectively. On Line 4, the pattern \li{_} is a \emph{wildcard} pattern -- it matches any value of type \li{Rx} and binds no variables.

If no rule produces a match, an exception indicating match failure is raised. The language statically warns programmers when match failures are possible by checking whether every value of the scrutinee type will be matched by one of rules in the rule sequence. For example, here the use of the wildcard pattern ensures that there can never be a match failure. We say that such a rule sequence is \emph{exhaustive}. The language can also warn programmers when a rule is \emph{redundant} relative to the preceding rules, i.e. when there does not exist a value matched by the rule but not any of the previous rules. For example, had there been another rule at the end of the match expression above, it would be redundant because all values match the wildcard pattern. 

\todo{citations and pointer to more details in ML}

Nested pattern matching is semantically very powerful. However, complex patterns can have high syntactic cost. In Sec. \ref{sec:syntax-examples-regexps}, we considered a hypothetical dialect of ML called ML+Rx that built in primitive syntax for patterns over the type \li{Rx}. In ML+Rx, we can express the example above at lower syntactic cost as follows:

\begin{lstlisting}
fun read_example_rx(r : Rx) : (string * Rx) option => 
  match r with 
    /SURL@EURLnameSURL: %EURLssn/ => Some (name, ssn)
  | _ => None\end{lstlisting}

Dialect formation is not a modular approach, for the reasons discussed in Chapter \ref{chap:intro}, so we seek language constructs that allow us to decrease the syntactic cost of expressing complex patterns.

Expression TSMs, introduced in Chapter \ref{chap:tsms}, decrease the syntactic cost of constructing a value of a specified type. However, expressions are distinct from patterns, so one cannot simply use an expression TSM to generate a pattern (the fact that certain concrete forms overlap is immaterial to this fundamental distinction). %For example, the expansion generated by an expression TSM might define or apply a function, but patterns do not contain functions or function applications. 
For this reason, we need to extend our language with a new (albeit closely related) construct -- the \textbf{pattern TSM}. In this chapter, we will consider only \textbf{unparameterized pattern TSMs}, i.e. pattern TSMs that generate patterns for values of a single type, like \li{Rx}. In Chapter \ref{sec:tsms-parameterized}, we will consider both expression and pattern TSMs defined over parameterized families of types. 

\section{Pattern TSMs By Example}
Pattern TSMs are quite similar to expression TSMs, differing mainly in that the expansions they generate are patterns, rather than expressions. Assuming the abstract syntax of patterns is encoded by the type \lstinline{Pat} (analagous to \lstinline{Exp}), we can define a TPSM at type \lstinline{Rx} as follows:
\begin{lstlisting}[numbers=none]
pattern syntax $rx at Rx {
	static fn (body : Body) : ParseResultPat => 
	  (* regex pattern parser here *)
}
\end{lstlisting}

Using this TPSM, we can rewrite our example as follows:
\begin{lstlisting}[numbers=none]
match r with 
    rx /SURL@EURLnameSURL: %EURLssn/ => display name ssn
  | _ => raise Invalid
\end{lstlisting}
To ensure that the client of the TPSM need not ``guess at'' what variables are bound by the pattern, variables (e.g. \lstinline{name} and \lstinline{ssn} here) can only appear in spliced subpatterns (just as variables bound at the use site can only appear in spliced subexpressions when using TSMs). We leave a formal account of TPSMs (in a reduced calculus that features simple pattern matching) as work that remains to be completed (see Sec. \ref{sec:syntax-timeline}).

ML does not presently support pattern matching over values of an abstract data type. However, there have been proposals for adding support for pattern matching over abstract data types defined by modules having a ``datatype-like'' shape, e.g. those that define a case analysis function like the one specified by \lstinline{RX}, shown in Sec. \ref{sec:examples}. We leave further discussion of such a facility and of parameterized TPSMs also as remaining work (see Sec. \ref{sec:syntax-timeline}). 

\subsection{Usage}
\subsection{Definition}
\subsection{Splicing and Binding}
\subsection{Validation}

\section{$\miniVersePat$}
\subsection{Expanded Expressions and Patterns}
\subsection{Unexpanded Expressions and Patterns}
\subsection{Pattern Macro Definition}
\subsection{Pattern Macro Application}
\subsection{Candidate Expansion Pattern Validation}
\subsection{Metatheory}

\chapter{Parameterized TSMs}\label{sec:tsms-parameterized}
\section{Parameterized TSMs By Example}
\subsection{Value Parameters By Example}
\subsection{Type Parameters By Example}
\subsection{Module Parameters By Example}
\section{$\miniVerseParam$}
\subsection{Signatures, Types and Expanded Expressions}
\subsection{Parameter Application and Deferred Substitution}
\subsection{Macro Expansion and Validation}
\subsection{Metatheory}


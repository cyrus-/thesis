% !TEX root = omar-thesis.tex
\chapter{Unparameterized Expression TSMs}\label{chap:tsms}
We now introduce a new primitive -- the \textbf{typed syntax macro} (TSM). TSMs, like term-rewriting macros (Sec. \ref{sec:term-rewriting}), generate expansions. Unlike term-rewriting macros, TSMs are applied to unparsed \emph{generalized literal forms}, which gives them substantially more syntactic flexibility. This chapter considers perhaps the simplest manifestation of TSMs: \textbf{unparameterized expression TSMs} (ueTSMs), which generate expressions of a single specified type. We will consider unparameterized pattern TSMs (upTSMs) in Chapter \ref{sec:pattern-tsms} and parameterized TSMs (pTSMs) in Chapter \ref{sec:tsms-parameterized}.

%Like the term-rewriting macros just described, TSMs can be parameterized by modules, so they can be used to define syntax valid at any abstract type defined by a module satisfying a specified signature. As we will discuss in the remainder of this section, this addresses all of the problems brought up above, at moderate syntactic cost.

\section{Expression TSMs By Example}\label{sec:tsms-by-example}
%A typed syntax macro is invoked by applying it to a \emph{delimited form}, which can contain  arbitrary syntax in its \emph{body}.  
We begin in this section with a ``tutorial-style'' introduction to ueTSMs in VerseML. In particular, we discuss a ueTSM for constructing values of the recursive labeled sum type \li{Rx} that was defined in Figure \ref{fig:datatype-rx}. We then formally specify ueTSMs with a reduced calculus, $\miniVerseUE$, in Sec. \ref{sec:tsms-minimal-formalism}. %We conclude in Sec. \ref{sec:uetsms-discussion} 

\subsection{Usage}\label{sec:uetsms-usage}
In the following concrete VerseML expression, we apply a TSM named \li{#\dolla#rx} to the \emph{generalized literal form} \li{/SURLA|T|G|CEURL/}:
\begin{lstlisting}[numbers=none,mathescape=|]
$rx /SURLA|T|G|CEURL/
\end{lstlisting}
Generalized literal forms are left unparsed when concrete expressions are first parsed. It is only during the subsequent \emph{typed expansion} process that the TSM parses the \emph{body} of the provided literal form, i.e. the characters between forward slashes in blue here, to generate a \emph{candidate expansion}. The language then \emph{validates} the candidate expansion according to criteria that we will establish in Sec. \ref{sec:uetsms-validation}. If candidate expansion validation succeeds, the language generates the \emph{final expansion} (or more concisely, simply the \emph{expansion}) of the expression. The program will behave as if the expression above has been replaced by its expansion. The expansion of the expression above, written concretely, is:
\begin{lstlisting}[numbers=none]
Or(Str "SSTRAESTR", Or(Str "SSTRTESTR", Or(Str "SSTRGESTR", Str "SSTRCESTR")))
\end{lstlisting}
%The constructors above are those of the type \li{Rx} that was defined in Figure \ref{fig:datatype-rx}.

A number of literal forms, shown in Figure \ref{fig:literal-forms},  are available in VerseML's concrete syntax. Any literal form can be used with any TSM, e.g. we could have equivalently written the example above as \li{#\dolla#rx `SURLA|T|G|CEURL`} (in fact, this would be convenient if we had wanted to express a regex containing forward slashes but not backticks). TSMs have access only to the literal bodies. Because TSMs do not extend the concrete syntax of the language directly, there cannot be syntactic conflicts between TSMs.

 %The form does not directly determine the expansion. 

\begin{figure}
\begin{lstlisting}
'SURLbody cannot contain an apostropheEURL'
`SURLbody cannot contain a backtickEURL`
[SURLbody cannot contain unmatched square bracketsEURL]
{SURLbody cannot contain an unmatched curly braceEURL}
/SURLbody cannot contain a forward slashEURL/
\SURLbody cannot contain a backslashEURL\
\end{lstlisting}
%SURL<tag>body includes enclosing tags</tag>EURL
\caption[Available Generalized Literal Forms]{Generalized literal forms available for use in VerseML's concrete syntax. The characters in blue indicate where the literal bodies are located within each form. In this figure, each line describes how the literal body is constrained by the form shown on that line. The Wyvern language specifies additional forms, including whitespace-delimited forms \cite{TSLs} and multipart forms \cite{sac15}, but for simplicity we leave these out of VerseML.}
\label{fig:literal-forms}
\end{figure}
\subsection{Definition}\label{sec:uetsms-definition}
%The original expression, above, is statically rewritten to this expression.
Let us now take the perspective of the library provider. The definition of the TSM \lstinline{#\dolla#rx} shown being applied above has the following form:
\begin{lstlisting}[numbers=none,mathescape=|]
syntax $rx at Rx {
  static fn(body : Body) : CEExp ParseResult => 
    (* regex literal parser here *)
}
\end{lstlisting}
This {TSM definition} first names the TSM. 
 TSM names must begin with the dollar symbol (\li{#\dolla#}) to clearly distinguish them from variables (and thereby clearly distinguish TSM application from function application). This is inspired by a similar convention enforced by the Rust macro system \cite{Rust/Macros}.

The TSM definition then specifies a \emph{type annotation}, \lstinline{at Rx}, and a \emph{parse function} within curly braces. 
The {parse function} is a \emph{static function} responsible for parsing the literal body when the TSM is applied to generate an encoding of the candidate expansion, or an indication of an error if one cannot be generated (e.g. when the body is ill-formed according to the syntactic specification that the TSM implements). Static functions are functions that are applied during the typed expansion process. For this reason, they do not have access to surrounding variable bindings (because those variables stand in for dynamic values). For now, let us simply assume that static functions are closed (we discuss introducing a distinct class of static bindings so that static values can be shared between TSM definitions in Sec. \ref{sec:uetsms-static-language}).

The parse function must have type \li{Body -> CEExp ParseResult}. These types are defined in the VerseML \emph{prelude}, which is a collection of definitions available ambiently. The input type, \lstinline{Body}, gives the parse function access to the {body} of the provided literal form. For our purposes, it suffices to define \li{Body} as an abbreviation for the \li{string} type:
\begin{lstlisting}[numbers=none]
type Body = string
\end{lstlisting} 

The output type, \li{CEExp ParseResult}, is a labeled sum type that distinguishes between successful parses and parse errors. The parameterized type \li{'a ParseResult} is defined in Figure \ref{fig:indexrange-and-parseresult}.

If parsing succeeds, the parse function returns a value of the form \li{Success(#$\ecand$#)}, where $\ecand$ is the \emph{encoding of the candidate expansion}. Encodings of candidate expansions are, for expression TSMs, values of the type \lstinline{CEExp} defined in Figure \ref{fig:candidate-exp-verseml} (in Chapter \ref{sec:pattern-tsms}, we will introduce pattern TSMs, which generate patterns rather than expressions; this is why \li{ParseResult} is defined as a parameterized type). Expressions can mention types, so we also need to define a type \li{CETyp} in Figure \ref{fig:candidate-exp-verseml}. 
\begin{figure}
\begin{lstlisting}[numbers=none]
type IndexRange = {startIndex : nat, endIndex : nat}

type 'a ParseResult = Success of 'a 
                    | ParseError of {
                        msg : string, loc : IndexRange
                      }
\end{lstlisting}
\caption[Definitions of \li{IndexRange} and \li{ParseResult} in VerseML]{Definitions of \li{IndexRange} and \li{ParseResult} in the VerseML prelude.}
\label{fig:indexrange-and-parseresult}
\end{figure}
\begin{figure}
\begin{lstlisting}[numbers=none]
type CETyp = TyVar of var_t 
           | Arrow of CETyp * CETyp 
           | (* ... *) 
           | Spliced of IndexRange

type CEExp = Var of var_t 
           | Fn of var_t * CETyp * CEExp
           | App of CEExp * CEExp
           | (* ... *) 
           | Spliced of IndexRange
\end{lstlisting}
\caption[Abbreviated definitions of \li{CETyp} and \li{CEExp} in VerseML]{Abbreviated definitions \li{CETyp} and \li{CEExp} in the VerseML prelude. We assume some suitable type \li{var_t} exists, not shown.}
\label{fig:candidate-exp-verseml}
\end{figure}
% We will show a complete encoding when we describe our reduced formal system $\miniVerseUE$ in Sec. \ref{sec:tsms-minimal-formalism}. 
We discuss the constructors labeled \li{Spliced} in Sec. \ref{sec:splicing-and-hygiene}; the remaining constructors (some of which are elided for concision) encode the abstract syntax of VerseML expressions and types. To decrease the syntactic cost of working with the types defined in Figure \ref{fig:candidate-exp-verseml}, the prelude provides \emph{quasiquotation syntax} at these types, which is itself implemented using TSMs. We will discuss these TSMs in more detail in Sec. \ref{sec:tsms-for-tsms}. The definitions in Figure \ref{fig:candidate-exp-verseml} are recursive labeled sum types to simplify our exposition, but we could have chosen alternative encodings of terms, e.g. based on abstract binding trees \cite{pfpl}, with only minor modification to the semantics. % It is extended with one additional form used to handled spliced subexpressions, 

If the parse function determines that a candidate expansion cannot be generated, i.e. there is a parse error in the literal body, it returns a value labeled by \li{ParseError}. It must provide an error message and indicate the location of the error within the body of the literal form as a value of type \li{IndexRange}, also defined in Figure \ref{fig:indexrange-and-parseresult}. This information can be used by VerseML compilers when reporting errors to the programmer.

%Notice that the types just described are those that one would expect to find in a typical parser.

%One would find types analagous to those just described in any parser, so for concision, we elide the details of \li{#\dolla#rx}'s parse function.
%The parse function must treat the TSM parameters parametrically, i.e. it does not have access to any values in the supplied module parameter. Only the expansion the parse function generates can refer to module parameters. 
%For example, the following definition is ill-sorted:
%\begin{lstlisting}[numbers=none]
%syntax pattern_bad[Q : PATTERN] at Q.t {
%  static fn (body : Body) : Exp => 
%    if Q.flag then (* ... *) else (* ... *)
%}
%\end{lstlisting}%So the parse function parses the body of the delimited form to generate an encoding of the elaboration.

\subsection{Splicing}\label{sec:splicing-and-hygiene}
To support splicing syntax, like that described in Sec. \ref{sec:syntax-examples-regexps}, the parse function must be able to parse subexpressions out of the supplied literal body. For example, consider the code snippet in Figure \ref{fig:derived-spliced-subexpressions}, expressed instead using the \li{#\dolla#rx} TSM:
\begin{lstlisting}[numbers=none]
val ssn = $rx /SURL\d\d\d-\d\d-\d\d\d\dEURL/
fun example_rx_tsm(name: string) => $rx /SURL@EURLnameSURL: %EURLssn/
\end{lstlisting}
The subexpressions \lstinline{name} and \lstinline{ssn} on the second line appear directly in the body of the literal form, so we call them \emph{spliced subexpressions} (and color them black when typesetting them in this document). When the parse function determines that a subsequence of the literal body should be treated as a spliced subexpression (here, by recognizing the characters \lstinline{@} or \lstinline{%} followed by a variable or parenthesized expression), 
it can refer to it within the candidate expansion it generates using the \li{Spliced} constructor of the \li{CEExp} type shown in Figure \ref{fig:candidate-exp-verseml}. The \li{Spliced} constructor requires a value of type \li{IndexRange} because spliced subexpressions are referred to indirectly by their position within the literal body. This prevents TSMs from ``forging'' a spliced subexpression (i.e. claiming that an expression is a spliced subexpression, even though it does not appear in the body of the literal form). Expressions can also contain types, so one can also mark spliced types in an analagous manner using the \li{Spliced} constructor of the \li{CETyp} type. %In particular, the parse function must provide the index range of spliced subexpressions to the \li{Spliced} constructor of the type \li{MarkedExp}. %Only subexpressions that actually appear in the body of the literal form can be marked as spliced subexpressions.

The candidate expansion generated by \li{#\dolla#rx} for the body of \lstinline{example_rx_tsm}, if written in a hypothetical concrete syntax for candidate expansions where references to spliced subexpressions are written \li{spliced<startIdx, endIndex>}, is:
\begin{lstlisting}[numbers=none]
Seq(Str(spliced<1, 4>), Seq(Str "SSTR: ESTR", spliced<8, 10>))
\end{lstlisting}
Here, \li{spliced<1, 4>} refers to the subexpression \li{name} by position and \li{spliced<8, 10>} refers to the subexpression \li{ssn} by position. 

%For example, had the  would not be a valid expansion, because the  that are not inside spliced subexpressions:
%\begin{lstlisting}[numbers=none]
%Q.Seq(Q.Str(name), Q.Seq(Q.Str ": ", ssn))
%\end{lstlisting}

\subsection{Typing}\label{sec:uetsms-validation}
The language \emph{validates} candidate expansions before a final expansion is generated. One aspect of candidate expansion validation is checking  the candidate expansion against the type annotation specified by the TSM, e.g. the type \li{Rx} in the example above. This maintains a \emph{type discipline}: if a programmer sees a TSM being applied when examining a well-typed program, they need only look up the TSM's type annotation to determine the type of the generated expansion. Determining the type does not require examine the expansion directly.


\subsection{Hygiene}
The spliced subexpressions that the candidate expansion refers to (by their position within the literal body, cf. above) must be parsed, typed and expanded during the candidate expansion validation process (otherwise, the language would not be able to check the type of the candidate expansion). To maintain a useful \emph{binding discipline}, i.e. to allow programmers to reason also about variable binding without examining expansions directly, the validation process maintains two additional properties related to spliced subexpressions: \textbf{context independent expansion} and \textbf{expansion independent splicing}. These are collectively referred to as the \emph{hygiene properties} (because they are conceptually related to the concept of hygiene in term rewriting macro systems, cf. Sec. \ref{sec:term-rewriting}.) 

\paragraph{Context Independent Expansion} Programmers expect to be able to choose variable and symbol names freely, i.e. without needing to satisfy ``hidden assumptions'' made by the TSMs that are applied in scope of a binding. For this reason, context-dependent candidate expansions, i.e. those with free variables or symbols, are deemed invalid (even at application sites where those variables happen to be bound). An example of a TSM that generates context-dependent candidate expansions is shown below:
\begin{lstlisting}[numbers=none]
syntax $bad1 at Rx {
	static fn(body : Body) : ParseResultExp => Success (Var 'SSTRxESTR')
}
\end{lstlisting}
The candidate expansion this TSM generates would be well-typed only when there is an assumption \li{x : Rx} in the application site typing context. This ``hidden assumption'' makes reasoning about binding and renaming especially difficult, so this candidate expansion is deemed invalid (even when \li{#\dolla#bad1} is applied in a context where \li{x} happens to be bound).

Of course, this prohibition does not extend into the spliced subexpressions referred to in a candidate expansion because spliced subexpressions are authored by the TSM client and appear at the application site, and so can justifiably refer to application site bindings. We saw examples of spliced subexpressions that referred to variables bound at the application site in Sec. \ref{sec:splicing-and-hygiene}. Because candidate expansions refer to spliced subexpressions indirectly, checking this property is straightforward -- we only allow access to the application site typing context when typing spliced subexpressions. In the next section, we will formalize this intuition. % The TSM provider can only refer to them opaquely.

In the examples in Sec. \ref{sec:uetsms-usage} and Sec. \ref{sec:splicing-and-hygiene}, the expansion used constructors associated with the \li{Rx} type, e.g. \li{Seq} and \li{Str}. This might appear to violate our prohibition on context-dependent expansions. This is not the case only because in VerseML, constructor labels are not variables or scoped symbols. Syntactically, they must begin with a capital letter (like Haskell's datatype constructors). Different labeled sum types can use common constructor labels without conflict because the type the term is being checked against -- e.g. \li{Rx}, due to the type ascription on \li{#\dolla#rx} -- determines which type of value will be constructed. For dialects of ML where datatype definitions do introduce new variables or scoped symbols, we need parameterized TSMs. We will return to this topic in Chapter \ref{sec:tsms-parameterized}. % Indeed, we used the label \li{Spliced} for two different recursive labeled sum types in Figure \ref{fig:candidate-exp-verseml}.

\paragraph{Expansion Independent Splicing} Spliced subexpressions, as just described, must be given access to application site bindings. The \emph{expansion independent splicing} property ensures that spliced subexpressions have access to \emph{only} those bindings, i.e. a TSM cannot introduce new bindings into spliced subexpressions. For example, consider the following hypothetical candidate expansion (written concretely as above):
\begin{lstlisting}[numbers=none]
fn(x : Rx) => spliced<0, 4>
\end{lstlisting}
The variable \li{x} is not available when typing the indicated spliced subexpression, nor can it shadow any bindings of \li{x} that might appear at the application site.

For TSM providers, the benefit of this property is that they can choose the names of variables used internally within expansions freely, without worrying about whether they might shadow those that a client might have defined at the application site.

TSM clients can, in turn, determine exactly which bindings are available in a spliced subexpression without examining the expansion it appears within. In other words, there can be no ``hidden variables''. 

The trade-off is that this prevents library providers from defining  alternative binding forms. For example, Haskell's derived form for monadic commands (i.e. \li{do}-notation) supports binding the result of executing a command to a variable that is then available in the subsequent commands in a command sequence. In VerseML, this cannot be expressed in the same way. We will show an alternative formulation of Haskell's syntax for monadic commands that uses VerseML's anonymous function syntax to bind variables in Sec. \ref{sec:application-monadic-commands}. We will discuss mechanisms that would allow us to relax this restriction while retaining client control over variable names as future work in Sec. \ref{sec:controlled-binding}.

%These properties suffice to ensure that programmers and tools can freely rename a variable without changing the meaning of the program. The only information that is necessary to perform such a \emph{rename refactoring} is the locations of spliced subexpressions within all the literal forms for which the variable being renamed is in scope; the expansions need not otherwise be examined. It would be straightforward to develop a tool and/or editor plugin to indicate the locations of spliced subexpressions to the user, like we do in this document (by coloring spliced subexpressions black). We discuss tool support as future work in Sec. \ref{sec:interaction-with-tools}.

\subsection{Final Expansion}
If validation succeeds, the language generates the \emph{final expansion} from the candidate expansion by replacing references to spliced subexpressions with their final expansions. The final expansion of the body of \li{example_rx_tsm} is:
\begin{lstlisting}[numbers=none]
Seq(Str(name), Seq(Str "SSTR: ESTR", ssn))
\end{lstlisting}

\subsection{Scoping}
A benefit of specifying TSMs as a language primitive, rather than relying on extralinguistic mechanisms to manipulate the concrete syntax of our language directly, is that TSMs follow standard scoping rules. 

For example, we can define a TSM that is visible only to a single expression like this:
\begin{lstlisting}[numbers=none]
let x = let 
    syntax $rx at Rx { (* ... *) }
  in (* $rx is in scope here *) end 
in (* $rx is no longer in scope *) end
\end{lstlisting}

We will consider the question of how TSM definitions can be exported from compilation units in Sec. \ref{sec:tsm-packaging}.

\subsection{Comparison to ML+Rx}
Let us compare the VerseML TSM \li{#\dolla#rx} to ML+Rx, the hypothetical syntactic dialect of ML with support for derived forms for regular expressions described in Sec. \ref{sec:syntax-examples-regexps}.

Both ML+Rx and \li{#\dolla#rx} give programmers the ability to use the same standard syntax for constructing regexes, including syntax for splicing in other strings and regexes. In VerseML, however, we incur the additional syntactic cost of explicitly applying the \li{#\dolla#rx} TSM each time we wish to use regex syntax. This cost does not grow with the size of the regex, so it would only be significant in programs that involve a large number of small regexes (which do, of course, exist). In Chapter \ref{chap:tsls} we will consider a design where even this syntactic cost can be eliminated in certain situations.

The benefit of this approach is that we can easily define other TSMs to use alongside the \li{#\dolla#rx} TSM without needing to consider the possibility of syntactic conflict. Furthermore, programmers can rely on the typing discipline and the hygienic binding discipline described above to reason about programs, including those that contain unfamiliar forms. Put pithily, VerseML helps programmers avoid ``conflict and confusion''. 


\section{\texorpdfstring{$\miniVerseUE$}{miniVerseUE}}\label{sec:tsms-minimal-formalism}\label{sec:miniVerseU}

% \begin{figure}[p!]
% $\begin{array}{lllllll}
% \textbf{variables} & \textbf{type variables} & \textbf{labels} & \textbf{label sets} & \textbf{TSM variables} & \textbf{literal bodies} & \textbf{nats}\\
% x & t & \ell & \labelset & \tsmv & b & n\\~\end{array}$\\
% $\begin{array}{ll}
% \textbf{type formation contexts} & \textbf{typing contexts}\\
% \Delta ::= \emptyset ~\vert~ \Delta, t & \Gamma ::= \emptyset ~\vert~ \Gamma, x : \tau\\
% ~
% \end{array}$\\
% ~\\
% $\begin{array}{lcl}
% \gheading{types}\\
% \tau & ::= & t ~\vert~ \parr{\tau}{\tau} ~\vert~ \forallt{t}{\tau} ~\vert~ \rect{t}{\tau} ~\vert~  \prodt{\mapschema{\tau}{i}{\labelset}} ~\vert~ \sumt{\mapschema{\tau}{i}{\labelset}}\\
% ~\\
% \gheading{expanded expressions}\\
% e & ::= & x ~\vert~ \lam{x}{\tau}{e} ~\vert~ \app{e}{e} ~\vert~ \Lam{t}{e} ~\vert~ \App{e}{\tau} ~\vert~ \fold{t}{\tau}{e} ~\vert~ \unfold{e} ~\vert~ \tpl{\mapschema{e}{i}{\labelset}} ~\vert~ \prj{e}{\ell} \\
% & \vert & \inj{\ell}{e} ~\vert~ \caseof{e}{\mapschemab{x}{e}{i}{\labelset}}\\
% ~\\
% \gheading{TSM expressions}\\
% \tsme & ::= & \tsmv ~\vert~ \utsmdef{\tau}{\ue}\\
% ~\\
% \gheading{unexpanded expressions}\\
% \ue & ::= & {x} ~\vert~ \lam{x}{\tau}{\ue} ~\vert~ \ue(\ue) ~\vert~ \Lam{t}{\ue} ~\vert~ \App{\ue}{\tau} ~\vert~ \fold{t}{\tau}{\ue} ~\vert~ \unfold{\ue} ~\vert~ \tpl{\mapschema{\ue}{i}{\labelset}} ~\vert~ \prj{\ue}{\ell} \\
% & \vert & \inj{\ell}{\ue} ~\vert~ \caseof{\ue}{\mapschemab{x}{\ue}{i}{\labelset}}\\
% & \vert & \uesyntax{\tsmv}{\tsme}{\ue} ~\vert~ \utsmapp{\eta}{b}\\
% ~\\
% \gheading{candidate expansion types}\\
% \mtau & ::= & t ~\vert~ \parr{\mtau}{\mtau} ~\vert~ \forallt{t}{\mtau} ~\vert~ \rect{t}{\mtau} ~\vert~ \prodt{\mapschema{\tau}{i}{\labelset}} ~\vert~ \sumt{\mapschema{\mtau}{i}{\labelset}} \\
% & \vert & \mtspliced{\tau}\\
% ~\\
% \gheading{candidate expansion expressions}\\
% \me & ::= & x ~\vert~ \lam{x}{\mtau}{\me} ~\vert~ \app{\me}{\me} ~\vert~ \Lam{t}{\me} ~|~ \App{\me}{\mtau} ~\vert~ \fold{t}{\mtau}{\me} ~\vert~ \unfold{\me} ~\vert~ \tpl{\mapschema{\me}{i}{\labelset}} ~\vert~ \prj{\me}{\ell} \\
% & \vert & \inj{\ell}{\me} ~\vert~ \caseof{\me}{\mapschemab{x}{\me}{i}{\labelset}}\\
% & \vert & \mspliced{e}
% % \\~
% \end{array}$
% \todo{finish breaking this up into syntax tables}
% \caption[Syntax of $\miniVerseUE$]{Syntax of $\miniVerseUE$. The forms $\mapschema{V}{i}{\labelset}$ and $\mapschemab{x}{V}{i}{\labelset}$ where $V$ is a metavariable indicate finite functions from each label $i \in \labelset$ to a term, $V_i$, or binder, $x_i.V_i$, respectively.}
% \label{fig:lambda-tsm-syntax}
% \end{figure}


To make the intuitions developed in the previous section mathematically precise, we will now introduce a reduced formal system called $\miniVerseUE$ with support for ueTSMs. $\miniVerseUE$ consists of an \emph{inner core} and an \emph{outer surface}.
%For reference, the syntax of $\miniVerseUE$ is specified in Figure \ref{fig:lambda-tsm-syntax}. We will reproduce relevant portions of this specification inline (in tabular form) as we continue. 
%We specify all formal systems in this document within the metatheoretic framework detailed in \emph{PFPL} \cite{pfpl}, and assume familiarity of fundamental background concepts (e.g. abstract binding trees, substitution, implicit identification of terms up to $\alpha$-equivalence, structural induction and rule induction) covered therein. %Familiarity with other accounts of typed lambda calculi should also suffice to understand the formal systems in this document. 



\subsection{Syntax of the Inner Core}\label{sec:U-expanded-terms}

\begin{figure}
\hspace{-5px}$\begin{array}{lllllll}
\textbf{Sort} & & & \textbf{Operational Form} & \textbf{Stylized Form} & \textbf{Description}\\
\mathsf{Typ} & \tau & ::= & t & t & \text{variable}\\
&&& \aparr{\tau}{\tau} & \parr{\tau}{\tau} & \text{partial function}\\
&&& \aall{t}{\tau} & \forallt{t}{\tau} & \text{polymorphic}\\
&&& \arec{t}{\tau} & \rect{t}{\tau} & \text{recursive}\\
&&& \aprod{\labelset}{\mapschema{\tau}{i}{\labelset}} & \prodt{\mapschema{\tau}{i}{\labelset}} & \text{labeled product}\\
&&& \asum{\labelset}{\mapschema{\tau}{i}{\labelset}} & \sumt{\mapschema{\tau}{i}{\labelset}} & \text{labeled sum}\\
\mathsf{Exp} & e & ::= & x & x & \text{variable}\\
&&& \aelam{\tau}{x}{e} & \lam{x}{\tau}{e} & \text{abstraction}\\
&&& \aeap{e}{e} & \ap{e}{e} & \text{application}\\
&&& \aetlam{t}{e} & \Lam{t}{e} & \text{type abstraction}\\
&&& \aetap{e}{\tau} & \App{e}{\tau} & \text{type application}\\
&&& \aefold{t}{\tau}{e} & \fold{e} & \text{fold}\\
&&& \aeunfold{e} & \unfold{e} & \text{unfold}\\
&&& \aetpl{\labelset}{\mapschema{e}{i}{\labelset}} & \tpl{\mapschema{e}{i}{\labelset}} & \text{labeled tuple}\\
&&& \aepr{\ell}{e} & \prj{e}{\ell} & \text{projection}\\
&&& \aein{\labelset}{\ell}{\mapschema{\tau}{i}{\labelset}}{e} & \inj{\ell}{e} & \text{injection}\\
&&& \aecase{\labelset}{\tau}{e}{\mapschemab{x}{e}{i}{\labelset}} & \caseof{e}{\mapschemab{x}{e}{i}{\labelset}} & \text{case analysis}
\end{array}$
\caption[Syntax of types and expanded expressions in $\miniVerseUE$]{Abstract syntax of types and expanded expressions, which form the \emph{inner core of }$\miniVerseUE$. Metavariables $x$ range over variables, $t$ over type variables, $\ell$ over labels and $\labelset$ over finite sets of labels. We adopt \emph{PFPL}'s conventions for operational forms, i.e. the names of operators and indexed families of operators are written in $\texttt{typewriter font}$, indexed families of operators specify non-symbolic indices within $[\text{mathematical braces}]$ and symbolic indices within \texttt{[}textual braces\text{]}, and term arguments are grouped arbitrarily (roughly, by ``phase'') using \texttt{\{}textual curly braces\texttt{\}} and \texttt{(}textual rounded braces\texttt{)} \cite{pfpl}. We write $\mapschema{\tau}{i}{\labelset}$ for a sequence of arguments $\tau_i$, one for each $i\in \labelset$, and similarly for arguments of other valences. Operations  parameterized by label sets, e.g. $\aprod{\labelset}{\mapschema{\tau}{i}{\labelset}}$, are identified up to mutual reordering of the label set and the corresponding argument sequence. %When using stylized forms, the label set is omitted when it can be inferred, e.g. the labeled product type $\prodt{\finmap{\mapitem{\ell_1}{e_1}, \mapitem{\ell_2}{e_2}}}$ leaves the label set $\{\ell_1, \ell_2\}$ implicit. 
When we use the stylized forms, we assume that the reader can infer suppressed indices and arguments from the surrounding context. Types and expanded expressions are identified up to $\alpha$-equivalence.}
\label{fig:U-expanded-terms}
\end{figure}

The \emph{inner core of} $\miniVerseUE$ consists of \emph{types}, $\tau$, and \emph{expanded expressions}, $e$. The syntax of the inner core is specified by the syntax chart in Figure \ref{fig:U-expanded-terms}. 
The {inner core} forms a pure language with support for partial functions, quantification over types, recursive types, labeled product types and labeled sum types.  The reader is directed to \emph{PFPL} \cite{pfpl} (or another text on type systems, e.g. \emph{TAPL} \cite{tapl}) for a detailed introductory account of these (or very similar) constructs. We will tersely define the statics of the inner core, and outline the structural dynamics, in the next two subsections, respectively.

\subsection{Statics of the Inner Core}
The \emph{statics of the inner core} is defined by hypothetical judgements of the following form:

\[\begin{array}{ll}
\textbf{Judgement Form} & \textbf{Description}\\
\istypeU{\Delta}{\tau} & \text{$\tau$ is a well-formed type assuming $\Delta$}\\
%\isctxU{\Delta}{\Gamma} & \text{$\Gamma$ is a well-formed typing context assuming $\Delta$}\\
\hastypeU{\Delta}{\Gamma}{e}{\tau} & \text{$e$ is assigned type $\tau$ assuming $\Delta$ and $\Gamma$}
\end{array}\]
\noindent
\emph{Type formation contexts}, $\Delta$, are finite sets of hypotheses of the form $\Dhyp{t}$. Empty finite sets are written $\emptyset$, or omitted entirely within judgements, and non-empty finite sets are written as comma-separated finite sequences identified up to exchange and contraction. We write $\Delta, \Dhyp{t}$, when $\Dhyp{t} \notin \Delta$, for $\Delta$ extended with the hypothesis $\Dhyp{t}$. %Finite sets are written as finite sequences identified up to exchange.% We write $\Dcons{\Delta}{\Delta'}$ for the union of $\Delta$ and $\Delta'$.

The \emph{type formation judgement}, $\istypeU{\Delta}{\tau}$, is inductively defined by the following rules:
\begin{subequations}\label{rules:istypeU}
\begin{equation}\label{rule:istypeU-var}
\inferrule{ }{\istypeU{\Delta, \Dhyp{t}}{t}}
\end{equation}
\begin{equation}\label{rule:istypeU-parr}
\inferrule{
  \istypeU{\Delta}{\tau_1}\\
  \istypeU{\Delta}{\tau_2}
}{\istypeU{\Delta}{\aparr{\tau_1}{\tau_2}}}
\end{equation}
\begin{equation}\label{rule:istypeU-all}
  \inferrule{
    \istypeU{\Delta, \Dhyp{t}}{\tau}
  }{
    \istypeU{\Delta}{\aall{t}{\tau}}
  }
\end{equation}
\begin{equation}\label{rule:istypeU-rec}
  \inferrule{
    \istypeU{\Delta, \Dhyp{t}}{\tau}
  }{
    \istypeU{\Delta}{\arec{t}{\tau}}
  }
\end{equation}
\begin{equation}\label{rule:istypeU-prod}
  \inferrule{
    \{\istypeU{\Delta}{\tau_i}\}_{i \in \labelset}
  }{
    \istypeU{\Delta}{\aprod{\labelset}{\mapschema{\tau}{i}{\labelset}}}
  }
\end{equation}
\begin{equation}\label{rule:istypeU-sum}
  \inferrule{
    \{\istypeU{\Delta}{\tau_i}\}_{i \in \labelset}
  }{
    \istypeU{\Delta}{\asum{\labelset}{\mapschema{\tau}{i}{\labelset}}}
  }
\end{equation}
\end{subequations}
Premises of the form $\{{J}_i\}_{i \in \labelset}$ mean that for each $i \in \labelset$, the judgement ${J}_i$ must hold. 

\emph{Typing contexts}, $\Gamma$, are finite functions that map each variable $x \in \domof{\Gamma}$, to the hypothesis $\Ghyp{x}{\tau}$, for some $\tau$. Empty typing contexts are written $\emptyset$, or omitted entirely within judgements, and non-empty typing contexts are written as finite sequences of hypotheses identified up to exchange (we do not separately write down the finite set $\domof{\Gamma}$ because it can be determined from the listed hypotheses). We write $\Gamma, \Ghyp{x}{\tau}$, when $x \notin \domof{\Gamma}$, for the extension of $\Gamma$ with a mapping from $x$ to $\Ghyp{x}{\tau}$, and $\Gcons{\Gamma}{\Gamma'}$ when $\domof{\Gamma} \cap \domof{\Gamma'} = \emptyset$ for the typing context mapping each $x \in \domof{\Gamma} \cup \domof{\Gamma'}$ to $x : \tau$ if $x : \tau \in \Gamma$ or $x : \tau \in \Gamma'$. We write $\isctxU{\Delta}{\Gamma}$ if every type in $\Gamma$ is well-formed relative to $\Delta$.
\begin{definition}[Typing Context Formation] \label{def:isctxU}
$\isctxU{\Delta}{\Gamma}$ iff for each hypothesis $x : \tau \in \Gamma$, we have $\istypeU{\Delta}{\tau}$.
\end{definition}

The typing judgement, $\hastypeU{\Delta}{\Gamma}{e}{\tau}$, assigns types to expressions. It is inductively defined by the following rules:
\begin{subequations}\label{rules:hastypeU}
\begin{equation}\label{rule:hastypeU-var}
  \inferrule{ }{
    \hastypeU{\Delta}{\Gamma, \Ghyp{x}{\tau}}{x}{\tau}
  }
\end{equation}
\begin{equation}\label{rule:hastypeU-lam}
  \inferrule{
    \istypeU{\Delta}{\tau}\\
    \hastypeU{\Delta}{\Gamma, \Ghyp{x}{\tau}}{e}{\tau'}
  }{
    \hastypeU{\Delta}{\Gamma}{\aelam{\tau}{x}{e}}{\aparr{\tau}{\tau'}}
  }
\end{equation}
\begin{equation}\label{rule:hastypeU-ap}
  \inferrule{
    \hastypeU{\Delta}{\Gamma}{e_1}{\aparr{\tau}{\tau'}}\\
    \hastypeU{\Delta}{\Gamma}{e_2}{\tau}
  }{
    \hastypeU{\Delta}{\Gamma}{\aeap{e_1}{e_2}}{\tau'}
  }
\end{equation}
\begin{equation}\label{rule:hastypeU-tlam}
  \inferrule{
    \hastypeU{\Delta, \Dhyp{t}}{\Gamma}{e}{\tau}
  }{
    \hastypeU{\Delta}{\Gamma}{\aetlam{t}{e}}{\aall{t}{\tau}}
  }
\end{equation}
\begin{equation}\label{rule:hastypeU-tap}
  \inferrule{
    \hastypeU{\Delta}{\Gamma}{e}{\aall{t}{\tau}}\\
    \istypeU{\Delta}{\tau'}
  }{
    \hastypeU{\Delta}{\Gamma}{\aetap{e}{\tau'}}{[\tau'/t]\tau}
  }
\end{equation}
\begin{equation}\label{rule:hastypeU-fold}
  \inferrule{\
    \istypeU{\Delta, \Dhyp{t}}{\tau}\\
    \hastypeU{\Delta}{\Gamma}{e}{[\arec{t}{\tau}/t]\tau}
  }{
    \hastypeU{\Delta}{\Gamma}{\aefold{t}{\tau}{e}}{\arec{t}{\tau}}
  }
\end{equation}
\begin{equation}\label{rule:hastypeU-unfold}
  \inferrule{
    \hastypeU{\Delta}{\Gamma}{e}{\arec{t}{\tau}}
  }{
    \hastypeU{\Delta}{\Gamma}{\aeunfold{e}}{[\arec{t}{\tau}/t]\tau}
  }
\end{equation}
\begin{equation}\label{rule:hastypeU-tpl}
  \inferrule{
    \{\hastypeU{\Delta}{\Gamma}{e_i}{\tau_i}\}_{i \in \labelset}
  }{
    \hastypeU{\Delta}{\Gamma}{\aetpl{\labelset}{\mapschema{e}{i}{\labelset}}}{\aprod{\labelset}{\mapschema{\tau}{i}{\labelset}}}
  }
\end{equation}
\begin{equation}\label{rule:hastypeU-pr}
  \inferrule{
    \hastypeU{\Delta}{\Gamma}{e}{\aprod{\labelset, \ell}{\mapschema{\tau}{i}{\labelset}; \ell \hookrightarrow \tau}}
  }{
    \hastypeU{\Delta}{\Gamma}{\aepr{\ell}{e}}{\tau}
  }
\end{equation}
\begin{equation}\label{rule:hastypeU-in}
  \inferrule{
    \{\istypeU{\Delta}{\tau_i}\}_{i \in \labelset}\\
    \istypeU{\Delta}{\tau}\\
    \hastypeU{\Delta}{\Gamma}{e}{\tau}
  }{
    \hastypeU{\Delta}{\Gamma}{\aein{\labelset, \ell}{\ell}{\mapschema{\tau}{i}{\labelset}; \ell \hookrightarrow \tau}{e}}{\asum{\labelset, \ell}{\mapschema{\tau}{i}{\labelset}; \ell \hookrightarrow \tau}}
  }
\end{equation}
\begin{equation}\label{rule:hastypeU-case}
  \inferrule{
    \hastypeU{\Delta}{\Gamma}{e}{\asum{\labelset}{\mapschema{\tau}{i}{\labelset}}}\\
    \istypeU{\Delta}{\tau}\\
    \{\hastypeU{\Delta}{\Gamma, x_i : \tau_i}{e_i}{\tau}\}_{i \in \labelset}
  }{
    \hastypeU{\Delta}{\Gamma}{\aecase{\labelset}{\tau}{e}{\mapschemab{x}{e}{i}{\labelset}}}{\tau}
  }
\end{equation}
\end{subequations}
Rules (\ref{rules:istypeU}) and (\ref{rules:hastypeU}) are syntax-directed, so we assume an inversion lemma for each rule as needed without stating it separately. The following standard lemmas also hold. 

The Weakening Lemma establishes that extending a context with unnecessary hypotheses preserves well-formedness and typing.
\begin{lemma}[Weakening]\label{lemma:weakening-U} All of the following hold: 
\begin{enumerate} 
\item If $\istypeU{\Delta}{\tau}$ then $\istypeU{\Delta, \Dhyp{t}}{\tau}$.
%\item If $\isctxU{\Delta}{\Gamma}$ then $\isctxU{\Delta, \Dhyp{t}}{\Gamma}$.
\item If $\hastypeU{\Delta}{\Gamma}{e}{\tau}$ then $\hastypeU{\Delta, \Dhyp{t}}{\Gamma}{e}{\tau}$.
\item If $\hastypeU{\Delta}{\Gamma}{e}{\tau}$ and $\istypeU{\Delta}{\tau'}$ then $\hastypeU{\Delta}{\Gamma, \Ghyp{x}{\tau'}}{e}{\tau}$.
\end{enumerate}
\end{lemma}
\begin{proof-sketch} For each part, by rule induction on the assumption. 
%\begin{enumerate} 
%\item By rule induction over Rules (\ref{rules:istypeU}).
%\item By rule induction over Rules (\ref{rules:isctxU}).
%\item By rule induction over Rules (\ref{rules:hastypeU}).
%\item By rule induction over Rules (\ref{rules:hastypeU}).
%\end{enumerate}
\end{proof-sketch}

We assume that renaming of bound variables, $\alpha$-equivalence and substitution are defined as in \emph{PFPL} \cite{pfpl}. The Substitution Lemma establishes that substitution of a well-formed type for a type variable, or an expanded expression of the appropriate type for an expanded expression variable, preserves well-formedness and typing. 
\begin{lemma}[Substitution]\label{lemma:substitution-U} All of the following hold:
\begin{enumerate}
\item If $\istypeU{\Delta, \Dhyp{t}}{\tau}$ and $\istypeU{\Delta}{\tau'}$ then $\istypeU{\Delta}{[\tau'/t]\tau}$.
%\item If $\isctxU{\Delta, \Dhyp{t}}{\Gamma}$ and $\istypeU{\Delta}{\tau'}$ then $\isctxU{\Delta}{[\tau'/t]\Gamma}$.
\item If $\hastypeU{\Delta, \Dhyp{t}}{\Gamma}{e}{\tau}$ and $\istypeU{\Delta}{\tau'}$ then $\hastypeU{\Delta}{[\tau'/t]\Gamma}{[\tau'/t]e}{[\tau'/t]\tau}$.
\item If $\hastypeU{\Delta}{\Gamma, \Ghyp{x}{\tau'}}{e}{\tau}$ and $\hastypeU{\Delta}{\Gamma}{e'}{\tau'}$ then $\hastypeU{\Delta}{\Gamma}{[e'/x]e}{\tau}$.
\end{enumerate}\end{lemma}
\begin{proof-sketch}
For each part, by rule induction on the first assumption.
\end{proof-sketch}

The Decomposition Lemma is the converse of the Substitution Lemma.
\begin{lemma}[Decomposition]\label{lemma:decomposition-U} All of the following hold:
\begin{enumerate}
\item If $\istypeU{\Delta}{[\tau'/t]\tau}$ and $\istypeU{\Delta}{\tau'}$ then $\istypeU{\Delta, \Dhyp{t}}{\tau}$.
%\item If $\isctxU{\Delta}{[\tau'/t]\Gamma}$ and $\istypeU{\Delta}{\tau'}$ then $\isctxU{\Delta, \Dhyp{t}}{\Gamma}$.
\item If $\hastypeU{\Delta}{[\tau'/t]\Gamma}{[\tau'/t]e}{[\tau'/t]\tau}$ and $\istypeU{\Delta}{\tau'}$ then $\hastypeU{\Delta, \Dhyp{t}}{\Gamma}{e}{\tau}$.
\item If $\hastypeU{\Delta}{\Gamma}{[e'/x]e}{\tau}$ and $\hastypeU{\Delta}{\Gamma}{e'}{\tau'}$ then $\hastypeU{\Delta}{\Gamma, \Ghyp{x}{\tau'}}{e}{\tau}$.
\end{enumerate}\end{lemma}
\begin{proof-sketch}
\begin{enumerate}
\item By rule induction over Rules (\ref{rules:istypeU}) and case analysis on the definition of substitution. In all cases, the derivation of $\istypeU{\Delta}{[\tau'/t]\tau}$ does not depend on the form of $\tau'$.
%\item Context formation of $[\tau'/t]\Gamma$ does not depend on the structure of $\tau'$.
\item By rule induction over Rules (\ref{rules:hastypeU}) and case analysis on the definition of substitution. In all cases, the derivation of $\hastypeU{\Delta}{[\tau'/t]\Gamma}{[\tau'/t]e}{[\tau'/t]\tau}$ does not depend on the form of $\tau'$.
\item By rule induction over Rules (\ref{rules:hastypeU}) and case analysis on the definition of substitution. In all cases, the derivation of $\hastypeU{\Delta}{\Gamma}{[e'/x]e}{\tau}$ does not depend on the form of $e'$.
\end{enumerate}
\end{proof-sketch}

The Regularity Lemma establishes that the type assigned to an expanded expression under a well-formed typing context is always well-formed. 
\begin{lemma}[Regularity]\label{lemma:regularity-U} If $\hastypeU{\Delta}{\Gamma}{e}{\tau}$ and $\isctxU{\Delta}{\Gamma}$ then $\istypeU{\Delta}{\tau}$.\end{lemma}
\begin{proof-sketch}
By rule induction over Rules (\ref{rules:hastypeU}) and application of Definition \ref{def:isctxU} and Lemma \ref{lemma:substitution-U}.
\end{proof-sketch}
\subsection{Structural Dynamics}\label{sec:dynamics-U}
The \emph{structural dynamics of }$\miniVerseUE$ is specified as a transition system by judgements of the following form:
\[\begin{array}{ll}
\textbf{Judgement Form} & \textbf{Description}\\
\stepsU{e}{e'} & \text{$e$ transitions to $e'$}\\
\isvalU{e} & \text{$e$ is a value}
\end{array}\]
We also define auxiliary judgements for \emph{iterated transition}, $\multistepU{e}{e'}$, and \emph{evaluation}, $\evalU{e}{e'}$.

\begin{definition}[Iterated Transition]\label{defn:iterated-transition-U} $\multistepU{e}{e'}$ is the reflexive, transitive closure of $\stepsU{e}{e'}$.\end{definition}

\begin{definition}[Evaluation]\label{defn:evaluation-U}  $\evalU{e}{e'}$ iff $\multistepU{e}{e'}$ and $\isvalU{e'}$. \end{definition}

Our subsequent developments do not require making reference to particular rules in the structural dynamics (because TSMs operate statically), so we do not reproduce the rules here. Instead, it suffices to state the following conditions.

The Canonical Forms condition characterizes well-typed values. Satisfying this condition requires an \emph{eager} (i.e. \emph{by-value}) formulation of the dynamics. 
\begin{condition}[Canonical Forms]\label{condition:canonical-forms-U} If $\hastypeUC{e}{\tau}$ and $\isvalU{e}$ then:
\begin{enumerate}
\item If $\tau=\aparr{\tau_1}{\tau_2}$ then $e=\aelam{\tau_1}{x}{e'}$ and $\hastypeUCO{\Ghyp{x}{\tau_1}}{e'}{\tau_2}$.
\item If $\tau=\aall{t}{\tau'}$ then $e=\aetlam{t}{e'}$ and $\hastypeUCO{\Dhyp{t}}{e'}{\tau'}$.
\item If $\tau=\arec{t}{\tau'}$ then $e=\aefold{t}{\tau'}{e'}$ and $\hastypeUC{e'}{[\abop{rec}{t.\tau'}/t]\tau'}$ and $\isvalU{e'}$. 
\item If $\tau=\aprod{\labelset}{\mapschema{\tau}{i}{\labelset}}$ then $e=\aetpl{\labelset}{\mapschema{e}{i}{\labelset}}$ and $\hastypeUC{e_i}{\tau_i}$ and $\isvalU{e_i}$ for each $i \in \labelset$.
\item If $\tau=\asum{\labelset}{\mapschema{\tau}{i}{\labelset}}$ then for some label set $L'$ and label $\ell$ and type $\tau_\ell$, we have that $\labelset=\labelset', \ell$ and $\tau=\asum{\labelset', \ell}{\mapschema{\tau}{i}{\labelset'}; \mapitem{\ell}{\tau_\ell}}$ and $e=\aein{\labelset', \ell}{\ell}{\mapschema{\tau}{i}{\labelset'}; \ell \hookrightarrow \tau_\ell}{e'}$ and $\hastypeUC{e'}{\tau_\ell}$ and $\isvalU{e'}$.
\end{enumerate}\end{condition}

The Preservation condition ensures that evaluation preserves typing.  
\begin{condition}[Preservation]\label{condition:preservation-U} If $\hastypeUC{e}{\tau}$ and $\multistepU{e}{e'}$ then $\hastypeUC{e'}{\tau}$. \end{condition}
The Progress condition ensures that evaluation of a well-typed expanded expression cannot ``get stuck''.
\begin{condition}[Progress]\label{condition:progress-U} If $\hastypeUC{e}{\tau}$ then either $\isvalU{e}$ or there exists an $e'$ such that $\stepsU{e}{e'}$. \end{condition}
 Together, these two conditions constitute the Type Safety condition.

\subsection{Syntax of the Outer Surface}\label{sec:syntax-U}
\begin{figure}
\hspace{-6px}$\arraycolsep=3.5pt\begin{array}{lllllll}
\textbf{Sort} & & & \textbf{Operational Form} & \textbf{Stylized Form} & \textbf{Description}\\
\mathsf{UTyp} & \utau & ::= & \ut & \ut & \text{sigil}\\
&&& \auparr{\utau}{\utau} & \parr{\utau}{\utau} & \text{partial function}\\
&&& \auall{\ut}{\utau} & \forallt{\ut}{\utau} & \text{polymorphic}\\
&&& \aurec{\ut}{\utau} & \rect{\ut}{\utau} & \text{recursive}\\
&&& \auprod{\labelset}{\mapschema{\utau}{i}{\labelset}} & \prodt{\mapschema{\utau}{i}{\labelset}} & \text{labeled product}\\
&&& \ausum{\labelset}{\mapschema{\utau}{i}{\labelset}} & \sumt{\mapschema{\utau}{i}{\labelset}} & \text{labeled sum}\\
\mathsf{UExp} & \ue & ::= & \ux & \ux & \text{sigil}\\
&&& \aulam{\utau}{\ux}{\ue} & \lam{\ux}{\utau}{\ue} & \text{abstraction}\\
&&& \auap{\ue}{\ue} & \ap{\ue}{\ue} & \text{application}\\
&&& \autlam{\ut}{\ue} & \Lam{\ut}{\ue} & \text{type abstraction}\\
&&& \autap{\ue}{\utau} & \App{\ue}{\utau} & \text{type application}\\
&&& \aufold{\ut}{\utau}{\ue} & \fold{\ue} & \text{fold}\\
&&& \auunfold{\ue} & \unfold{\ue} & \text{unfold}\\
&&& \autpl{\labelset}{\mapschema{\ue}{i}{\labelset}} & \tpl{\mapschema{\ue}{i}{\labelset}} & \text{labeled tuple}\\
&&& \aupr{\ell}{\ue} & \prj{\ue}{\ell} & \text{projection}\\
&&& \auin{\labelset}{\ell}{\mapschema{\utau}{i}{\labelset}}{\ue} & \inj{\ell}{\ue} & \text{injection}\\
&&& \aucase{\labelset}{\utau}{\ue}{\mapschemab{\ux}{\ue}{i}{\labelset}} & \caseof{\ue}{\mapschemab{\ux}{\ue}{i}{\labelset}} & \text{case analysis}\\
\LCC  &  & & \lightgray & \lightgray & \lightgray \\
&&& \audefuetsm{\utau}{e}{\tsmv}{\ue} & \uesyntax{\tsmv}{\utau}{e}{\ue} & \text{ueTSM definition}\\ 
&&& \autsmap{b}{\tsmv} & \utsmap{\tsmv}{b} & \text{ueTSM application}\ECC
\end{array}$
\caption[Syntax of unexpanded types and expressions in $\miniVerseUE$]{Abstract syntax of unexpanded types and expressions in $\miniVerseUE$. Metavariable $\ut$ ranges over type sigils, $\ux$ ranges over expression sigils, $\tsmv$  over TSM names and $b$ over sequences of characters, which, when they appear in an unexpanded expression, are called literal bodies. Literal bodies might contain spliced subterms that are only ``surfaced'' during typed expansion, so renaming of bound identifiers and substitution are not defined over unexpanded types and expressions.}
\label{fig:U-unexpanded-terms}
\end{figure}
A $\miniVerseUE$ program ultimately evaluates as an expanded expression. However, the programmer does not write the expanded expression directly. Instead, the programmer writes a textual sequence, $b$, consisting of characters in some suitable alphabet (e.g. in practice, \texttt{ASCII} or \texttt{Unicode}), which is parsed by some partial metafunction $\mathsf{parseUExp}(b)$ to produce an \emph{unex\-panded expression}, $\ue$. Unexpanded expressions can contain \emph{unexpanded types}, $\utau$, so we also need a partial metafunction $\mathsf{parseUTyp}(b)$. The abstract syntax of unexpanded types and expressions, which form  the \emph{outer surface} of $\miniVerseUE$, is defined in Figure \ref{fig:U-unexpanded-terms}. The full definition of the textual syntax of $\miniVerseUE$, which $\mathsf{parseUExp}(b)$ and $\mathsf{parseUTyp}(b)$ implement, is not important for our purposes, so we simply give the following condition, which states that there is some way to textually represent every unexpanded type and expression. %We also assume a metafunction $\mathsf{parseUTyp}(b)$ for parsing unexpanded types, and impose an analagous condition.
\begin{condition}[Textual Representability] Both of the following hold:
\begin{enumerate}
\item For each $\utau$, there exists $b$ such that $\parseUTyp{b}{\utau}$. 
\item For each $\ue$, there exists $b$ such that $\parseUExp{b}{\ue}$.
\end{enumerate}
\end{condition}


Unexpanded types and expressions are given meaning by expansion to types and expanded expressions, respectively, according to the \emph{typed expansion judgements}, which are defined in the next subsection.

Unexpanded types and expressions bind \emph{type sigils}, $\ut$, \emph{expression sigils}, $\ux$, and \emph{TSM names}, $\tsmv$. Sigils are given meaning by expansion to variables during typed expansion. We \textbf{cannot} adopt the usual definitions of $\alpha$-renaming of identifiers, because unexpanded types and expressions are still in a ``partially parsed'' state -- the literal bodies, $b$, within an unexpanded expression might contain spliced subterms that are ``surfaced'' by a TSM only during typed expansion, as we will detail below. %Sigils are given meaning by expansion to variables. %In other words, unexpanded expressions are not abstract binding trees, nor sequences of characters, but a ``transitional'' structure with some characteristics of each of these. 
%For this reason, we will need to handle generating fresh variables explicitly at binding sites in our semantics. %To do so, we distinguish \emph{type sigils}, $\ut$, and \emph{expression sigils}, $\ux$, from type variables, $t$, and expression variables, $x$. Sigils will be given meaning by expansion to variables (which, in turn, are given meaning by substitution, as described above). 

Each inner core form (defined in Figure \ref{fig:U-expanded-terms}) maps onto an outer surface form. We refer to these as the \emph{shared forms}. In particular:
\begin{itemize}
\item Each type variable, $t$, maps onto a unique {type sigil}, written $\sigilof{t}$ (pronounced ``sigil of $t$''). Notice the distinction between $\ut$, which is a metavariable ranging over type sigils, and $\sigilof{t}$, which is a metafunction, written in stylized form, applied to a type variable to produce a type sigil.
\item Each type form maps onto an unexpanded type form $\Uof{\tau}$ as follows: 
  \begin{align*}
  \Uof{t} &= \sigilof{t}\\
  \Uof{\aparr{\tau_1}{\tau_2}} & = \auparr{\Uof{\tau_1}}{\Uof{\tau_2}}\\
  \Uof{\aall{t}{\tau}} & = \auall{\sigilof{t}}{\Uof{\tau}}\\
  \Uof{\arec{t}{\tau}} & = \aurec{\sigilof{t}}{\Uof{\tau}}\\
  \Uof{\aprod{\labelset}{\mapschema{\tau}{i}{\labelset}}} & = \auprod{\labelset}{\mapschemax{\Uofv}{\tau}{i}{\labelset}}\\
  \Uof{\asum{\labelset}{\mapschema{\tau}{i}{\labelset}}} & = \ausum{\labelset}{\mapschemax{\Uofv}{\tau}{i}{\labelset}}
  \end{align*}
\item Each expression variable, $x$, maps onto a unique expression sigil written $\sigilof{x}$. Again, notice the distinction between $\ux$ and $\sigilof{x}$.
\item Each expanded expression form maps onto an unexpanded expression form $\Uof{e}$ as follows:
\begin{align*}
\Uof{x} & = \sigilof{x}\\
\Uof{\aelam{\tau}{x}{e}} & = \aulam{\Uof{\tau}}{\sigilof{x}}{\Uof{e}}\\
\Uof{\aeap{e_1}{e_2}} & = \auap{\Uof{e_1}}{\Uof{e_2}}\\
\Uof{\aetlam{t}{e}} & = \autlam{\sigilof{t}}{\Uof{e}}\\
\Uof{\aetap{e}{\tau}} & = \autap{\Uof{e}}{\Uof{\tau}}\\
\Uof{\aefold{t}{\tau}{e}} & = \aufold{\sigilof{t}}{\Uof\tau}{\Uof e}\\
\Uof{\aeunfold{e}} & = \auunfold{\Uof{e}}\\
\Uof{\aetpl{\labelset}{\mapschema{e}{i}{\labelset}}} & = \autpl{\labelset}{\mapschemax{\Uofv}{e}{i}{\labelset}}\\
\Uof{\aein{\labelset}{\ell}{\mapschema{\tau}{i}{\labelset}}{e}} &= \auin{\labelset}{\ell}{\mapschemax{\Uofv}{\tau}{i}{\labelset}}{\Uof{e}}\\
\Uof{\aecase{\labelset}{\tau}{e}{\mapschemab{x}{e}{i}{\labelset}}} & = \aucase{\labelset}{\Uof\tau}{\Uof{e}}{\mapschemabx{\Uofv}{x}{e}{i}{\labelset}}
\end{align*}
\end{itemize}

There are only two unexpanded expression forms, highlighted in gray in Figure \ref{fig:U-unexpanded-terms}, that do not correspond to expanded expression forms -- the ueTSM definition form and the ueTSM application form. %These are the ``interesting'' forms. % These are the ``interesting'' forms. % Let us define this correspondence by the metafunction $\Uof{e}$:
%\[
%\begin{split}
%\Uof{x} & = x\\
%\Uof{\aelam{\tau}{x}{e}} & = \aulam{\tau}{x}{\Uof{e}}\\
%\Uof{\aeap{e_1}{e_2}} & = \auap{\Uof{e_1}}{\Uof{e_2}}
%\end{split}
%\] and so on for the remaining expanded expression forms.


\subsection{Typed Expansion}\label{sec:typed-expansion-U}
Unexpanded expressions, and the unexpanded types therein, are checked and expanded simultaneously according to the \emph{typed expansion judgements}:
\[\begin{array}{ll}
\textbf{Judgement Form} & \textbf{Description}\\
\expandsTU{\uDelta}{\utau}{\tau} & \text{$\utau$ is well-formed and has expansion $\tau$ assuming $\uDelta$}\\
\expandsUX{\ue}{e}{\tau} & \text{$\ue$ has expansion $e$ and type $\tau$ under ueTSM context $\uSigma$}\\
& \text{assuming $\uDelta$ and $\uGamma$}
\end{array}\]
%\newcommand{\gray}[1]{{\color{gray} #1}}

\subsubsection{Type Expansion}
The \emph{type expansion judgement}, $\expandsTU{\uDelta}{\utau}{\tau}$, is inductively defined by the following rules.
\begin{subequations}\label{rules:expandsTU}
\begin{equation}\label{rule:expandsTU-var}
\inferrule{ }{\expandsTU{\uDelta, \uDhyp{\ut}{t}}{\ut}{t}}
\end{equation}
\begin{equation}\label{rule:expandsTU-parr}
\inferrule{
  \expandsTU{\uDelta}{\utau_1}{\tau_1}\\
  \expandsTU{\uDelta}{\utau_2}{\tau_2}
}{\expandsTU{\uDelta}{\auparr{\utau_1}{\utau_2}}{\aparr{\tau_1}{\tau_2}}}
\end{equation}
\begin{equation}\label{rule:expandsTU-all}
  \inferrule{
    \expandsTU{\uDelta, \uDhyp{\ut}{t}}{\utau}{\tau}
  }{
    \expandsTU{\uDelta}{\auall{\ut}{\utau}}{\aall{t}{\tau}}
  }
\end{equation}
\begin{equation}\label{rule:expandsTU-rec}
  \inferrule{
    \expandsTU{\uDelta, \uDhyp{\ut}{t}}{\utau}{\tau}
  }{
    \expandsTU{\uDelta}{\aurec{\ut}{\utau}}{\arec{t}{\tau}}
  }
\end{equation}
\begin{equation}\label{rule:expandsTU-prod}
  \inferrule{
    \{\expandsTU{\uDelta}{\utau_i}{\tau_i}\}_{i \in \labelset}
  }{
    \expandsTU{\uDelta}{\auprod{\labelset}{\mapschema{\utau}{i}{\labelset}}}{\aprod{\labelset}{\mapschema{\tau}{i}{\labelset}}}
  }
\end{equation}
\begin{equation}\label{rule:expandsTU-sum}
  \inferrule{
    \{\expandsTU{\uDelta}{\utau_i}{\tau_i}\}_{i \in \labelset}
  }{
    \expandsTU{\uDelta}{\ausum{\labelset}{\mapschema{\utau}{i}{\labelset}}}{\asum{\labelset}{\mapschema{\tau}{i}{\labelset}}}
  }
\end{equation}
\end{subequations}
\emph{Unexpanded type formation contexts}, $\uDelta$, are of the form $\uDD{\uD}{\Delta}$, where $\uD$ is a \emph{type sigil expansion context}, and $\Delta$ is a type formation context. A type sigil expansion context, $\uD$, is a finite function that maps each type sigil $\ut \in \domof{\uD}$ to the hypothesis $\vExpands{\ut}{t}$, for some type variable $t$. We write $\ctxUpdate{\uD}{\ut}{t}$ for the type sigil expansion context that maps $\ut$ to $\vExpands{\ut}{t}$ and defers to $\uD$ for all other type sigils (i.e. the previous mapping, if it exists, is updated). 
We define $\uDelta, \uDhyp{\ut}{t}$ when $\uDelta=\uDD{\uD}{\Delta}$ as an abbreviation of  \[\uDD{\ctxUpdate{\uD}{\ut}{t}}{\Delta, \Dhyp{t}}\]%type identifier expansion context is always extended/updated together with 
%We write $\uDeltaOK{\uDelta}$ when $\uDelta=\uDD{\uD}{\Delta}$ and each type variable in $\uD$ also appears in $\Delta$.
%\begin{definition}\label{def:uDeltaOK} $\uDeltaOK{\uDD{\uD}{\Delta}}$ iff for each $\vExpands{\ut}{t} \in \uD$, we have $\Dhyp{t} \in \Delta$.\end{definition}

To understand how type sigil expansion contexts operate, it is instructive to derive an expansion for the unexpanded type $\forallt{\ut}{\forallt{\ut}{\ut}}$, or in operational form, $\auall{\ut}{\auall{\ut}{\ut}}$:
\begin{mathpar}
\inferrule{
  \inferrule{
    \inferrule{ }{
      \expandsTU{\uDD{\vExpands{\ut}{t'}}{{\Dhyp{t}}, {\Dhyp{t'}}}}{\ut}{t'}
    }~\text{(\ref*{rule:expandsTU-var})}
  }{
    \expandsTU{\uDD{\vExpands{\ut}{t}}{\Dhyp{t}}}{\auall{\ut}{\ut}}{\aall{t'}{t'}}
  }~\text{(\ref*{rule:expandsTU-all})}
}{
  \expandsTU{\uDD{\emptyset}{\emptyset}}{\auall{\ut}{\auall{\ut}{\ut}}}{\aall{t}{\aall{t'}{t'}}}
}~\text{(\ref*{rule:expandsTU-all})}
\end{mathpar}
Notice that when a type sigil is bound, a fresh type variable is generated. The type sigil expansion context is extended (when the outermost binding is encountered) or updated (at all inner bindings) and the type formation context is simultaneously extended at each binding (so that typing contexts and ueTSM contexts, discussed below, that contain types that refer to the previous binding remain well-formed). Had we used type variables in the syntax and type formation contexts in the rules above, rather than type sigils and type sigil expansion contexts, derivations for unexpanded types where an inner binding shadows an outer binding would not exist, because by definition we cannot extend a type formation context with a variable it already mentions nor implicitly $\alpha$-vary the unexpanded type to sidestep this problem. 

These rules validate the following lemmas. The Type Expansion Lemma establishes that the expansion of an unexpanded type is a well-formed type.

\begin{lemma}[Type Expansion]\label{lemma:type-expansion-U} If $\expandsTU{\uDD{\uD}{\Delta}}{\utau}{\tau}$ then $\istypeU{\Delta}{\tau}$.\end{lemma}
\begin{proof} By rule induction over Rules (\ref{rules:expandsTU}). In each case, we apply the IH to or over each premise, then apply the corresponding type formation rule in Rules (\ref{rules:istypeU}). \end{proof}

The Type Expressibility Lemma establishes that every well-formed type, $\tau$, can be expressed as a well-formed unexpanded type, $\Uof{\tau}$. This requires defining the metafunction $\Uof{\Delta}$ which maps $\Delta$ onto a an unexpanded type formation context as follows:
\begin{align*}
\Uof{\emptyset} &= \uDD{\emptyset}{\emptyset}\\
\Uof{\Delta, \Dhyp{t}} &= \Uof{\Delta}, \uDhyp{\sigilof{t}}{t}
\end{align*}
\begin{lemma}[Type Expressibility]\label{lemma:type-expressibility} If $\istypeU{\Delta}{\tau}$ then $\expandsTU{\Uof{\Delta}}{\Uof{\tau}}{\tau}$.\end{lemma}
\begin{proof} By rule induction over Rules (\ref{rules:istypeU}) using the definitions of $\Uof{\tau}$ and $\Uof{\Delta}$ above. In each case, we apply the IH to or over each premise, then apply the corresponding type expansion rule in Rules (\ref{rules:expandsTU}).\end{proof}

\subsubsection{Typed Expression Expansion}
\begin{subequations}\label{rules:expandsU}
\emph{Unexpanded typing contexts}, $\uGamma$, are of the form $\uGG{\uG}{\Gamma}$, where $\uG$ is an \emph{expression sigil expansion context}, and $\Gamma$ is a typing context. An expression sigil expansion context, $\uG$, is a finite function that maps each expression sigil $\ux \in \domof{\uG}$ to the hypothesis $\vExpands{\ux}{x}$, for some expression variable, $x$. We write $\ctxUpdate{\uG}{\ux}{x}$ for the expression sigil expansion context that maps $\ux$ to $\vExpands{\ux}{x}$ and defers to $\uG$ for all other expression sigils (i.e. the previous mapping, if it exists, is updated). %We write $\uGammaOK{\uGamma}$ when $\uGamma=\uGG{\uG}{\Gamma}$ and each expression variable in $\uG$ is assigned a type by $\Gamma$.
%\begin{definition} $\uGammaOK{\uGG{\uG}{\Gamma}}$ iff for each $\vExpands{\ux}{x} \in \uG$, we have $\Ghyp{x}{\tau} \in \Gamma$ for some $\tau$.\end{definition}
%\noindent 
We define $\uGamma, \uGhyp{\ux}{x}{\tau}$ when $\uGamma = \uGG{\uG}{\Gamma}$ as an abbreviation of \[\uGG{\uG, \vExpands{\ux}{x}}{\Gamma, \Ghyp{x}{\tau}}\]

The \emph{typed expression expansion judgement}, $\expandsUX{\ue}{e}{\tau}$, is inductively defined by Rules (\ref*{rules:expandsU}) as follows. %These rules validate the following theorem, which establishes that typed expansion produces an expansion of the assigned type. 
%\begin{theorem}[Typed Expression Expansion] If $\expandsU{\uDD{\uD}{\Delta}}{\uGG{\uG}{\Gamma}}{\uSigma}{\ue}{e}{\tau}$ and $\uetsmenv{\Delta}{\uSigma}$ then $\hastypeU{\Delta}{\Gamma}{e}{\tau}$.\end{theorem}
%\begin{proof} This is the first part of Theorem \ref{thm:typed-expansion-U}, defined and proven below.\end{proof}

\paragraph{Shared Forms} Rules (\ref*{rule:expandsU-var}) through (\ref*{rule:expandsU-case}) handle unexpanded expressions of shared form. The first five of these rules are defined below:
%Each of these rules is based on the corresponding typing rule, i.e. Rules (\ref{rule:hastypeU-var}) through (\ref{rule:hastypeU-case}), respectively. For example, the following typed expansion rules are based on the typing rules (\ref{rule:hastypeU-var}), (\ref{rule:hastypeU-lam}) and (\ref{rule:hastypeU-ap}), respectively:% for unexpanded expressions of variable, function and application form, respectively: 
\begin{equation}\label{rule:expandsU-var}
  \inferrule{ }{\expandsU{\uDelta}{\uGamma, \uGhyp{\ux}{x}{\tau}}{\uSigma}{\ux}{x}{\tau}}
\end{equation}
\begin{equation}\label{rule:expandsU-lam}
  \inferrule{
    \expandsTU{\uDelta}{\utau}{\tau}\\
    \expandsU{\uDelta}{\uGamma, \uGhyp{\ux}{x}{\tau}}{\uSigma}{\ue}{e}{\tau'}
  }{\expandsUX{\aulam{\utau}{\ux}{\ue}}{\aelam{\tau}{x}{e}}{\aparr{\tau}{\tau'}}}
\end{equation}
\begin{equation}\label{rule:expandsU-ap}
  \inferrule{
    \expandsUX{\ue_1}{e_1}{\aparr{\tau}{\tau'}}\\
    \expandsUX{\ue_2}{e_2}{\tau}
  }{
    \expandsUX{\auap{\ue_1}{\ue_2}}{\aeap{e_1}{e_2}}{\tau'}
  }
\end{equation}
\begin{equation}\label{rule:expandsU-tlam}
  \inferrule{
    \expandsU{\uDelta, \uDhyp{\ut}{t}}{\uGamma}{\uSigma}{\ue}{e}{\tau}
  }{
    \expandsUX{\autlam{\ut}{\ue}}{\aetlam{t}{e}}{\aall{t}{\tau}}
  }
\end{equation}
\begin{equation}\label{rule:expandsU-tap}
  \inferrule{
    \expandsUX{\ue}{e}{\aall{t}{\tau}}\\
    \expandsTU{\uDelta}{\utau'}{\tau'}
  }{
    \expandsUX{\autap{\utau'}{\ue}}{\aetap{\tau'}{e}}{[\tau'/t]\tau}
  }
\end{equation}
Observe that, in each of these rules, the unexpanded and expanded expression forms in the conclusion correspond, and the premises correspond to those of the typing rule for the expanded expression form, i.e. Rules (\ref{rule:hastypeU-var}) through (\ref{rule:hastypeU-tap}), respectively. In particular, the type expansion premises and type formation premises correspond, and the typed expression expansion and typing premises correspond. The ueTSM context, $\uSigma$, passes opaquely through these rules (we will define ueTSM contexts below). Rules (\ref{rules:expandsTU}) were similarly generated by mechanically transforming Rules (\ref{rules:istypeU}).

We can express this scheme more precisely with the following rule transformation. For each rule in Rules (\ref{rules:istypeU}) and Rules (\ref{rules:hastypeU}),
\begin{mathpar}
\refstepcounter{equation}
% \label{rule:expandsU-tlam}
% \refstepcounter{equation}
% \label{rule:expandsU-tap}
% \refstepcounter{equation}
\label{rule:expandsU-fold}
\refstepcounter{equation}
\label{rule:expandsU-unfold}
\refstepcounter{equation}
\label{rule:expandsU-tpl}
\refstepcounter{equation}
\label{rule:expandsU-pr}
\refstepcounter{equation}
\label{rule:expandsU-in}
\refstepcounter{equation}
\label{rule:expandsU-case}
\inferrule{J_1\\ \cdots \\ J_k}{J}
\end{mathpar}
the corresponding typed expansion rule is 
\begin{mathpar}
\inferrule{
  \Uof{J_1} \\
  \cdots\\
  \Uof{J_k}
}{
  \Uof{J}
}
\end{mathpar}
where
\[\begin{split}
\Uof{\istypeU{\Delta}{\tau}} & = \expandsTU{\Uof{\Delta}}{\Uof{\tau}}{\tau} \\
\Uof{\hastypeU{\Gamma}{\Delta}{e}{\tau}} & = \expandsU{\Uof{\Gamma}}{\Uof{\Delta}}{\uSigma}{\Uof{e}}{e}{\tau}\\
\Uof{\{J_i\}_{i \in \labelset}} & = \{\Uof{J_i}\}_{i \in \labelset}
\end{split}\]
and where:
\begin{itemize}
\item $\Uof{\tau}$ is defined as follows:
  \begin{itemize}
  \item When $\tau$ is of definite form, $\Uof{\tau}$ is defined as in Sec. \ref{sec:syntax-U}.
  \item When $\tau$ is of indefinite form, $\Uof{\tau}$ is a uniquely corresponding metavariable of sort $\mathsf{UTyp}$ also of indefinite form. For example, in Rule (\ref{rule:istypeU-parr}), $\tau_1$ and $\tau_2$ are of indefinite form, i.e. they match arbitrary types. The rule transformation simply ``hats'' them, i.e. $\Uof{\tau_1}=\utau_1$ and $\Uof{\tau_2}=\utau_2$.
  \end{itemize}
\item $\Uof{e}$ is defined as follows
\begin{itemize}
\item When $e$ is of definite form, $\Uof{e}$ is defined as in Sec. \ref{sec:syntax-U}. 
\item When $e$ is of indefinite form, $\Uof{e}$ is a uniquely corresponding metavariable of sort $\mathsf{UExp}$ also of indefinite form. For example, $\Uof{e_1}=\ue_1$ and $\Uof{e_2}=\ue_2$.
\end{itemize}
\item $\Uof{\Delta}$ is defined as follows:
  \begin{itemize} 
  \item When $\Delta$ is of definite form, $\Uof{\Delta}$ is defined as above.
  \item When $\Delta$ is of indefinite form, $\Uof{\Delta}$ is a uniquely corresponding metavariable ranging over unexpanded type formation contexts. For example, $\Uof{\Delta} = \uDelta$.
  \end{itemize}
\item $\Uof{\Gamma}$ is defined as follows:
  \begin{itemize}
  \item When $\Gamma$ is of definite form, $\Uof{\Gamma}$ produces the corresponding unexpanded typing context as follows:
\begin{align*}
\Uof{\emptyset} & = \uGG{\emptyset}{\emptyset}\\
\Uof{\Gamma, \Ghyp{x}{\tau}} & = \Uof{\Gamma}, \uGhyp{\sigilof{x}}{x}{\tau}
\end{align*}
  \item When $\Gamma$ is of indefinite form, $\Uof{\Gamma}$ is a uniquely corresponding metavariable ranging over unexpanded typing contexts. For example, $\Uof{\Gamma} = \uGamma$.
\end{itemize}
\end{itemize}

It is instructive to use this rule transformation to generate Rules (\ref{rules:expandsTU}) and Rules (\ref{rule:expandsU-var}) through (\ref{rule:expandsU-tap}) above. We omit the remaining rules, i.e. Rules (\ref*{rule:expandsU-fold}) through (\ref*{rule:expandsU-case}). By instead defining these rules solely by the rule transformation just described, we avoid having to write down a number of rules that are of limited marginal interest. Moreover, this demonstrates the general technique for generating typed expansion rules for unexpanded types and expressions of shared form, so our exposition is somewhat ``robust'' to changes to the inner core. 

We can now establish the Expressibility Theorem -- that each well-typed expanded expression, $e$, can be expressed as an unexpanded expression, $\ue$, and assigned the same type under the corresponding contexts.

\begin{theorem}[Expressibility] If $\hastypeU{\Delta}{\Gamma}{e}{\tau}$ then $\expandsU{\Uof{\Delta}}{\Uof{\Gamma}}{\uSigma}{\Uof{e}}{e}{\tau}$.\end{theorem}
\begin{proof} By rule induction over Rules (\ref{rules:hastypeU}). The above rule transformation guarantees that this theorem holds by its construction. In particular, in each case, we can apply Lemma \ref{lemma:type-expressibility} to or over each type formation premise, the IH to or over each typing premise, then apply the corresponding rule in Rules (\ref{rules:expandsU}).\end{proof}
%o that when the inner core changes,  typed expansion rules  our exposition somewhat robust to changes to the inner core (though not to changes to the judgement forms in the statics of the inner core).% Even if changes to the judgement forms in the statics of the inner core are needed (e.g. the addition of a symbol context), it is easy to see would correspond to changes in the generic specification above.
% \begin{subequations}\label{rules:expandsU}
% \begin{equation}\label{rule:expandsU-var}
%   \inferrule{ }{\expandsU{\Delta}{\Gamma, x : \tau}{\uSigma}{x}{x}{\tau}}
% \end{equation}
% \begin{equation}\label{rule:expandsU-lam}
%   \inferrule{
%     \istypeU{\Delta}{\tau}\\
%     \expandsU{\Delta}{\Gamma, x : \tau}{\uSigma}{\ue}{e}{\tau'}
%   }{\expandsUX{\aulam{\tau}{x}{\ue}}{\aelam{\tau}{x}{e}}{\aparr{\tau}{\tau'}}}
% \end{equation}
% \begin{equation}\label{rule:expandsU-ap}
%   \inferrule{
%     \expandsUX{\ue_1}{e_1}{\aparr{\tau}{\tau'}}\\
%     \expandsUX{\ue_2}{e_2}{\tau}
%   }{
%     \expandsUX{\auap{\ue_1}{\ue_2}}{\aeap{e_1}{e_2}}{\tau'}
%   }
% \end{equation}
% \begin{equation}\label{rule:expandsU-tlam}
%   \inferrule{
%     \expandsU{\Delta, \Dhyp{t}}{\Gamma}{\uSigma}{\ue}{e}{\tau}
%   }{
%     \expandsUX{\autlam{t}{\ue}}{\aetlam{t}{e}}{\aall{t}{\tau}}
%   }
% \end{equation}
% \begin{equation}\label{rule:expandsU-tap}
%   \inferrule{
%     \expandsUX{\ue}{e}{\aall{t}{\tau}}\\
%     \istypeU{\Delta}{\tau'}
%   }{
%     \expandsUX{\autap{\ue}{\tau'}}{\aetap{e}{\tau'}}{[\tau'/t]\tau}
%   }
% \end{equation}
% \begin{equation}\label{rule:expandsU-fold}
%   \inferrule{
%     \istypeU{\Delta, \Dhyp{t}}{\tau}\\
%     \expandsUX{\ue}{e}{[\arec{t}{\tau}/t]\tau}
%   }{
%     \expandsUX{\aufold{t}{\tau}{\ue}}{\aefold{t}{\tau}{e}}{\arec{t}{\tau}}
%   }
% \end{equation}
% \begin{equation}\label{rule:expandsU-unfold}
%   \inferrule{
%     \expandsUX{\ue}{e}{\arec{t}{\tau}}
%   }{
%     \expandsUX{\auunfold{\ue}}{\aeunfold{e}}{[\arec{t}{\tau}/t]\tau}
%   }
% \end{equation}
% \begin{equation}\label{rule:expandsU-tpl}
%   \inferrule{
%     \{\expandsUX{\ue_i}{e_i}{\tau_i}\}_{i \in \labelset}
%   }{
%     \expandsUX{\autpl{\labelset}{\mapschema{\ue}{i}{\labelset}}}{\aetpl{\labelset}{\mapschema{e}{i}{\labelset}}}{\aprod{\labelset}{\mapschema{\tau}{i}{\labelset}}}
%   }
% \end{equation}
% \begin{equation}\label{rule:expandsU-pr}
%   \inferrule{
%     \expandsUX{\ue}{e}{\aprod{\labelset, \ell}{\mapschema{\tau}{i}{\labelset}; \mapitem{\ell}{\tau}}}
%   }{
%     \expandsUX{\aupr{\ell}{\ue}}{\aepr{\ell}{e}}{\tau}
%   }
% \end{equation}
% \begin{equation}\label{rule:expandsU-in}
%   \inferrule{
%     \{\istypeU{\Delta}{\tau_i}\}_{i \in \labelset}\\
%     \istypeU{\Delta}{\tau}\\
%     \expandsUX{\ue}{e}{\tau}
%   }{
%     \left\{\shortstack{$\Delta~\Gamma \vdash_\uSigma \auin{\labelset, \ell}{\ell}{\mapschema{\tau}{i}{\labelset}; \mapitem{\ell}{\tau}}{\ue}$\\$\leadsto$\\$\aein{\labelset, \ell}{\ell}{\mapschema{\tau}{i}{\labelset}; \mapitem{\ell}{\tau}}{e} : \asum{\labelset, \ell}{\mapschema{\tau}{i}{\labelset}; \mapitem{\ell}{\tau}}$\vspace{-1.2em}}\right\}
%   }
% \end{equation}
% \begin{equation}\label{rule:expandsU-case}
%   \inferrule{
%     \expandsUX{\ue}{e}{\asum{\labelset}{\mapschema{\tau}{i}{\labelset}}}\\
%     \{\expandsU{\Delta}{\Gamma, \Ghyp{x_i}{\tau_i}}{\uSigma}{\ue_i}{e_i}{\tau}\}_{i \in \labelset}
%   }{
%     \expandsUX{\aucase{\labelset}{\ue}{\mapschemab{x}{\ue}{i}{\labelset}}}{\aecase{\labelset}{e}{\mapschemab{x}{e}{i}{\labelset}}}{\tau}
%   }
% \end{equation}
\end{subequations}
\paragraph{ueTSM Definition and Application} The two remaining typed expansion rules, Rules (\ref{rule:expandsU-syntax}) and (\ref{rule:expandsU-tsmap}), govern the ueTSM definition and application forms, and are defined in the next two subsections, respectively. 

% \begin{equation}\label{rule:expandsU-syntax}
% \inferrule{
%   \istypeU{\Delta}{\tau}\\
%   \expandsU{\emptyset}{\emptyset}{\emptyset}{\ueparse}{\eparse}{\aparr{\tBody}{\tParseResultExp}}\\\\
%   a \notin \domof{\uSigma}\\
%   \expandsU{\Delta}{\Gamma}{\uSigma, \xuetsmbnd{\tsmv}{\tau}{\eparse}}{\ue}{e}{\tau'}
% }{
%   \expandsUX{\audefuetsm{\tau}{\ueparse}{\tsmv}{\ue}}{e}{\tau'}
% }
% \end{equation}
% \begin{equation}\label{rule:expandsU-tsmap}
% \inferrule{
%   \encodeBody{b}{\ebody}\\
%   \evalU{\ap{\eparse}{\ebody}}{\inj{\lbltxt{Success}}{\ecand}}\\
%   \decodeCondE{\ecand}{\ce}\\\\
%   \cvalidE{\emptyset}{\emptyset}{\esceneU{\Delta}{\Gamma}{\uSigma, \xuetsmbnd{\tsmv}{\tau}{\eparse}}{b}}{\ce}{e}{\tau}
% }{
%   \expandsU{\Delta}{\Gamma}{\uSigma, \xuetsmbnd{\tsmv}{\tau}{\eparse}}{\autsmap{b}{\tsmv}}{e}{\tau}
% }
% \end{equation}
%\end{subequations}

%Notice that each form of expanded expression (Figure \ref{fig:U-expanded-terms}) corresponds to a form of unexpanded expression (Figure \ref{fig:U-unexpanded-terms}). For each typing rule in Rules (\ref{rules:hastypeU}), there is a corresponding typed expansion rule -- Rules (\ref{rule:expandsU-var}) through (\ref{rule:expandsU-case}) -- where the unexpanded and expanded forms correspond. The premises also correspond -- if a typing judgement appears as a premise of a typing rule, then the corresponding premise in the corresponding typed expansion rule is the corresponding typed expansion judgement. The ueTSM context is not extended or inspected by these rules (it is only ``threaded through'' them opaquely).

%There are two unexpanded expression forms that do not correspond to an expanded expression form: the ueTSM definition form, and the ueTSM application form. The rules governing these two forms interact with the ueTSM context, and are the topics of the next two subsections, respectively.

\subsection{ueTSM Definitions}\label{sec:U-uetsm-definition}
The stylized ueTSM definition form is \[\uesyntax{\tsmv}{\utau}{\eparse}{\ue}\] 
%The operational form corresponding to this stylized form is \[\audefuetsm{\utau}{\eparse}{\tsmv}{\ue}\]
An unexpanded expression of this form defines a {ueTSM} named $\tsmv$ with \emph{unexpanded type annotation} $\utau$ and \emph{parse function} $\eparse$ for use within $\ue$. 

The parse function is an expanded expression because parse functions are applied statically (i.e. during typed expansion of $\ue$), as we will discuss when describing ueTSM application below, and evaluation is defined only for closed expanded expressions. This construction simplifies our exposition, though it is not entirely practical because it provides no way for TSM providers to share values between parse functions, nor any way to use TSMs when defining other TSMs. We discuss enriching the language to eliminate these limitations in Sec. \ref{sec:uetsms-static-language}, but it is pedagogically simpler to leave the necessary machinery out of our calculus for now.%$\miniVerseUE$.

Rule (\ref*{rule:expandsU-syntax}) defines typed expansion of ueTSM definitions (we use stylized forms for clarity):
\begin{subequations}[resume]
% \begin{equation}\label{rule:expandsU-syntax}
% \inferrule{
%   \istypeU{\Delta}{\tau}\\
%   \expandsU{\emptyset}{\emptyset}{\emptyset}{\ueparse}{\eparse}{\aparr{\tBody}{\tParseResultExp}}\\\\
%   \expandsU{\Delta}{\Gamma}{\uSigma, \xuetsmbnd{\tsmv}{\tau}{\eparse}}{\ue}{e}{\tau'}
% }{
%   \expandsUX{\audefuetsm{\tau}{\ueparse}{\tsmv}{\ue}}{e}{\tau'}
% }
% \end{equation}
\begin{equation}\label{rule:expandsU-syntax}
\inferrule{
  \expandsTU{\uDelta}{\utau}{\tau}\\
  \hastypeU{\emptyset}{\emptyset}{\eparse}{\parr{\tBody}{\tParseResultExp}}\\\\
  \expandsU{\uDelta}{\uGamma}{\uSigma, \uShyp{\tsmv}{a}{\tau}{\eparse}}{\ue}{e}{\tau'}
}{
  \expandsUX{\uesyntax{\tsmv}{\utau}{\eparse}{\ue}}{e}{\tau'}
}
\end{equation}
\end{subequations}
The premises of this rule can be understood as follows, in order:
\begin{enumerate}
\item The first premise ensures that the unexpanded type annotation is well-formed and expands it to produce the \emph{type annotation}, $\tau$.

\item The second premise checks that the parse function, $\eparse$, is closed and of type \[\parr{\tBody}{\tParseResultExp}\] %to generate the \emph{expanded parse function}, $\eparse$. 
 %Notice that this occurs under empty contexts, i.e. parse functions cannot refer to the surrounding bindings. 
%The parse function must be of type $\aparr{\tBody}{\tParseResultExp}$ where the type abbreviations $\tBody$ and $\tParseResultExp$ are defined as follows.

The type abbreviated $\tBody$ classifies encodings of literal bodies, $b$. The mapping from literal bodies to values of type $\tBody$ is defined by the \emph{body encoding judgement} $\encodeBody{b}{\ebody}$. An inverse mapping is defined   by the \emph{body decoding judgement} $\decodeBody{\ebody}{b}$.
\[\begin{array}{ll}
\textbf{Judgement Form} & \textbf{Description}\\
\encodeBody{b}{e} & \text{$b$ has encoding $e$}\\
\decodeBody{e}{b} & \text{$e$ has decoding $b$}
\end{array}\]
Rather than defining $\tBody$ explicitly, and these judgements inductively against that definition (which would be tedious and uninteresting), it suffices to define the following condition, which establishes an isomorphism between literal bodies and values of type $\tBody$ mediated by the judgements above.
\begin{condition}[Body Isomorphism] All of the following must hold:
\begin{enumerate}
\item For every literal body $b$, we have that $\encodeBody{b}{\ebody}$ for some $\ebody$ such that $\hastypeUC{\ebody}{\tBody}$ and $\isvalU{\ebody}$.
\item If $\hastypeUC{\ebody}{\tBody}$ and $\isvalU{\ebody}$ then $\decodeBody{\ebody}{b}$ for some $b$.
\item If $\encodeBody{b}{\ebody}$ then $\decodeBody{\ebody}{b}$.
\item If $\hastypeUC{\ebody}{\tBody}$ and $\isvalU{\ebody}$ and $\decodeBody{\ebody}{b}$ then $\encodeBody{b}{\ebody}$. 
\item If $\encodeBody{b}{\ebody}$ and $\encodeBody{b}{\ebody'}$ then $\ebody = \ebody'$.
\item If $\hastypeUC{\ebody}{\tBody}$ and $\isvalU{\ebody}$ and $\decodeBody{\ebody}{b}$ and $\decodeBody{\ebody}{b'}$ then $b=b'$.
\end{enumerate}
\end{condition}

$\tParseResultExp$ abbreviates a labeled sum type that distinguishes successful parses from parse errors\footnote{In VerseML, the \li{ParseError} constructor of \li{ParseResult} required an error message and an error location, but we omit these in our formalization for simplicity}:
\[\tParseResultExp \triangleq [\mapitem{\lbltxt{Success}}{\tCEExp}, \mapitem{\lbltxt{ParseError}}{\prodt{}}]\] 

The type abbreviated $\tCEExp$ classifies encodings of \emph{candidate expansion expressions} (or \emph{ce-expression}), $\ce$ (pronounced ``grave $e$''). The syntax of ce-expressions will be described in Sec. \ref{sec:ce-syntax-U}. The mapping from ce-expressions to values of type $\tCEExp$ is defined by the \emph{ce-expression encoding judgement}, $\encodeCondE{\ce}{e}$. An inverse mapping is defined by the \emph{ce-expression decoding judgement}, $\decodeCondE{e}{\ce}$.

\[\begin{array}{ll}
\textbf{Judgement Form} & \textbf{Description}\\
\encodeCondE{\ce}{e} & \text{$\ce$ has encoding $e$}\\
\decodeCondE{e}{\ce} & \text{$e$ has decoding $\ce$}
\end{array}\]

Again, rather than picking a particular definition of $\tCEExp$ and defining the judgements above inductively against it, we only state the following condition, which establishes an isomorphism between values of type $\tCEExp$ and ce-expressions.

\begin{condition}[Candidate Expansion Expression Isomorphism] All of the following hold:
\begin{enumerate}
\item For every $\ce$, we have $\encodeCondE{\ce}{\ecand}$ for some $\ecand$ such that $\hastypeUC{\ecand}{\tCEExp}$ and $\isvalU{\ecand}$.
\item If $\hastypeUC{\ecand}{\tCEExp}$ and $\isvalU{\ecand}$ then $\decodeCondE{\ecand}{\ce}$ for some $\ce$.
\item If $\encodeCondE{\ce}{\ecand}$ then $\decodeCondE{\ecand}{\ce}$.
\item If $\hastypeUC{\ecand}{\tCEExp}$ and $\isvalU{\ecand}$ and $\decodeCondE{\ecand}{\ce}$ then $\encodeCondE{\ce}{\ecand}$.
\item If $\encodeCondE{\ce}{\ecand}$ and $\encodeCondE{\ce}{\ecand'}$ then $\ecand=\ecand'$.
\item If $\hastypeUC{\ecand}{\tCEExp}$ and $\isvalU{\ecand}$ and $\decodeCondE{\ecand}{\ce}$ and $\decodeCondE{\ecand}{\ce'}$ then $\ce=\ce'$.
\end{enumerate}
\end{condition}


\item The final premise of Rule (\ref{rule:expandsU-syntax}) extends the ueTSM context, $\uSigma$, with the newly determined {ueTSM definition}, and proceeds to assign a type, $\tau'$, and expansion, $e$, to $\ue$. The conclusion of Rule (\ref{rule:expandsU-syntax}) assigns this type and expansion to the ueTSM definition as a whole.% i.e. TSMs define behavior that is relevant during typed expansion, but not during evaluation. 



\emph{ueTSM contexts}, $\uSigma$, are of the form $\uAS{\uA}{\Sigma}$, where $\uA$ is a \emph{TSM naming context} and $\Sigma$ is a \emph{ueTSM definition context}. 

A \emph{TSM naming context}, $\uA$, is a finite function mapping each TSM name $\tsmv \in \domof{\uA}$ to the \emph{TSM name-symbol mapping}, $\vExpands{\tsmv}{a}$, for some \emph{symbol}, $a$. We write $\ctxUpdate{\uA}{\tsmv}{a}$ for the ueTSM naming context that maps $\tsmv$ to $\vExpands{\tsmv}{a}$, and defers to $\uA$ for all other TSM names (i.e. the previous mapping, if it exists, is updated).

A \emph{ueTSM definition context}, $\Sigma$, is a finite function mapping each symbol $a \in \domof{\Sigma}$ to an \emph{expanded ueTSM definition}, $\xuetsmbnd{a}{\tau}{\eparse}$, where $\tau$ is the ueTSM's type annotation, and $\eparse$ is its parse function. We write $\Sigma, \xuetsmbnd{a}{\tau}{\eparse}$ when $a \notin \domof{\Sigma}$ for the extension of $\Sigma$ that maps $a$ to $\xuetsmbnd{a}{\tau}{\eparse}$. We write $\uetsmenv{\Delta}{\Sigma}$  when all the type annotations in $\Sigma$ are well-formed assuming $\Delta$, and the parse functions in $\Sigma$ are closed and of type $\parr{\tBody}{\tParseResultExp}$.

\begin{definition}[ueTSM Definition Context Formation]\label{def:ueTSM-def-ctx-formation} $\uetsmenv{\Delta}{\Sigma}$ iff for each $\xuetsmbnd{\tsmv}{\tau}{\eparse} \in \Sigma$, we have $\istypeU{\Delta}{\tau}$ and $\hastypeU{\emptyset}{\emptyset}{\eparse}{\parr{\tBody}{\tParseResultExp}}$.\end{definition}

We define $\uSigma, \uShyp{\tsmv}{a}{\tau}{\eparse}$, when $\uSigma=\uAS{\uA}{\Sigma}$, as an abbreviation of \[\uAS{\ctxUpdate{\uA}{\tsmv}{a}}{\Sigma, \xuetsmbnd{a}{\tau}{\eparse}}\]

\end{enumerate}

% \[\begin{array}{ll}
% \textbf{Judgement Form} & \textbf{Description}\\
% \uetsmenv{\Delta}{\uSigma} & \text{$\uSigma$ is well-formed assuming $\Delta$}\end{array}\]
% This judgement is inductively defined by the following rules:
% \begin{subequations}[intermezzo]\label{rules:uetsmenv-U}
% \begin{equation}\label{rule:uetsmenv-empty}
% \inferrule{ }{\uetsmenv{\Delta}{\emptyset}}
% \end{equation}
% \begin{equation}\label{rule:uetsmenv-ext}
% \inferrule{
%   \uetsmenv{\Delta}{\uSigma}\\
%   \istypeU{\Delta}{\tau}\\
%   \hastypeU{\emptyset}{\emptyset}{\eparse}{\aparr{\tBody}{\tParseResultExp}}
% }{
%   \uetsmenv{\Delta}{\uSigma, \xuetsmbnd{\tsmv}{\tau}{\eparse}}
% }
% \end{equation}
% \end{subequations}

\subsection{ueTSM Application}\label{sec:U-uetsm-application}
The stylized unexpanded expression form for applying a ueTSM named $\tsmv$ to a literal form with literal body $b$ is:
\[
\utsmap{\tsmv}{b}
\] 
This stylized form uses forward slashes to delimit the literal body, but stylized variants of any of the literal forms specified in Figure \ref{fig:literal-forms} could also be added to Figure \ref{fig:U-unexpanded-terms}. % (we omit them for simplicity).
The corresponding operational form is $\autsmap{b}{\tsmv}$. %i.e. for each literal body $b$, the operator $\texttt{uapuetsm}[b]$ is indexed by the TSM name $\tsmv$ and takes no arguments. %\footnote{This is in following the conventions in \emph{PFPL} \cite{pfpl}, where operators parameters allow for the use of metatheoretic objects that are not syntax trees or binding trees, e.g. $\mathsf{str}[s]$ and $\mathsf{num}[n]$.} This operator is indexed by the TSM name $\tsmv$ and takes no arguments. 

The typed expansion rule governing ueTSM application is below:
\begin{subequations}[resume]
% \begin{equation}\label{rule:expandsU-tsmap}
% \inferrule{
%   \encodeBody{b}{\ebody}\\
%   \evalU{\ap{\eparse}{\ebody}}{\inj{\lbltxt{Success}}{\ecand}}\\
%   \decodeCondE{\ecand}{\ce}\\\\
%   \cvalidE{\emptyset}{\emptyset}{\esceneU{\Delta}{\Gamma}{\uSigma, \xuetsmbnd{\tsmv}{\tau}{\eparse}}{b}}{\ce}{e}{\tau}
% }{
%   \expandsU{\Delta}{\Gamma}{\uSigma, \xuetsmbnd{\tsmv}{\tau}{\eparse}}{\autsmap{b}{\tsmv}}{e}{\tau}
% }
% \end{equation}
\begin{equation}\label{rule:expandsU-tsmap}
\inferrule{
  \encodeBody{b}{\ebody}\\
  \evalU{\ap{\eparse}{\ebody}}{\inj{\lbltxt{Success}}{\ecand}}\\
  \decodeCondE{\ecand}{\ce}\\\\
  \cvalidE{\emptyset}{\emptyset}{\esceneU{\uDelta}{\uGamma}{\uSigma, \uShyp{\tsmv}{a}{\tau}{\eparse}}{b}}{\ce}{e}{\tau}
}{
  \expandsU{\uDelta}{\uGamma}{\uSigma, \uShyp{\tsmv}{a}{\tau}{\eparse}}{\utsmap{\tsmv}{b}}{e}{\tau}
}
\end{equation}
\end{subequations}
The premises of Rule (\ref{rule:expandsU-tsmap}) can be understood as follows, in order:
\begin{enumerate}
\item The first premise determines the encoding of the literal body, $\ebody$ (see above).
\item The second premise applies the parse function $\eparse$, which appears in the ueTSM context associated with $\tsmv$, to $\ebody$. If parsing succeeds, i.e. a value of the (stylized) form $\inj{\lbltxt{Success}}{\ecand}$ results from evaluation, then $\ecand$ will be a value of type $\tCEExp$ (assuming a well-formed ueTSM context, by transitive application of the Preservation assumption). We call $\ecand$ the \emph{encoding of the candidate expansion}.

If the parse function produces a value labeled $\lbltxt{ParseError}$, then typed expansion fails. No rule is necessary to handle this case. 

\item The third premise decodes the encoding of the candidate expansion to produce the \emph{candidate expansion}, $\ce$ (see above).




\item The final premise of Rule (\ref{rule:expandsU-tsmap}) \emph{validates} the candidate expansion and simultaneously generates the \emph{final expansion}, $e$. This is the topic of Sec. \ref{sec:ce-validation-U}.
\end{enumerate}
\subsection{Syntax of Candidate Expansions}\label{sec:ce-syntax-U}

\begin{figure}
\hspace{-5px}$\arraycolsep=3.5pt\begin{array}{lllllll}
\textbf{Sort} & & & \textbf{Operational Form} & \textbf{Stylized Form} & \textbf{Description}\\
\mathsf{CETyp} & \ctau & ::= & t & t & \text{variable}\\
&&& \aceparr{\ctau}{\ctau} & \parr{\ctau}{\ctau} & \text{partial function}\\
&&& \aceall{t}{\ctau} & \forallt{t}{\ctau} & \text{polymorphic}\\
&&& \acerec{t}{\ctau} & \rect{t}{\ctau} & \text{recursive}\\
&&& \aceprod{\labelset}{\mapschema{\ctau}{i}{\labelset}} & \prodt{\mapschema{\ctau}{i}{\labelset}} & \text{labeled product}\\
&&& \acesum{\labelset}{\mapschema{\ctau}{i}{\labelset}} & \sumt{\mapschema{\ctau}{i}{\labelset}} & \text{labeled sum}\\
\LCC &&& \lightgray & \lightgray & \lightgray\\
&&& \acesplicedt{m}{n} & \splicedt{m}{n} & \text{spliced}\\\ECC
\mathsf{CEExp} & \ce & ::= & x & x & \text{variable}\\
&&& \acelam{\ctau}{x}{\ce} & \lam{x}{\ctau}{\ce} & \text{abstraction}\\
&&& \aceap{\ce}{\ce} & \ap{\ce}{\ce} & \text{application}\\
&&& \acetlam{t}{\ce} & \Lam{t}{\ce} & \text{type abstraction}\\
&&& \acetap{\ce}{\ctau} & \App{\ce}{\ctau} & \text{type application}\\
&&& \acefold{t}{\ctau}{\ce} & \fold{\ce} & \text{fold}\\
&&& \aceunfold{\ce} & \unfold{\ce} & \text{unfold}\\
&&& \acetpl{\labelset}{\mapschema{\ce}{i}{\labelset}} & \tpl{\mapschema{\ce}{i}{\labelset}} & \text{labeled tuple}\\
&&& \acepr{\ell}{\ce} & \prj{\ce}{\ell} & \text{projection}\\
&&& \acein{\labelset}{\ell}{\mapschema{\ctau}{i}{\labelset}}{\ce} & \inj{\ell}{\ce} & \text{injection}\\
&&& \acecase{\labelset}{\tau}{\ce}{\mapschemab{x}{\ce}{i}{\labelset}} & \caseof{\ce}{\mapschemab{x}{\ce}{i}{\labelset}} & \text{case analysis}\\
\LCC &&& \lightgray & \lightgray & \lightgray\\
&&& \acesplicede{m}{n} & \splicede{m}{n} & \text{spliced}\ECC
\end{array}$
\caption[Syntax of candidate expansion types and expressions in $\miniVerseUE$]{Abstract syntax of candidate expansion types and expressions in $\miniVerseUE$. Metavariables $m$ and $n$ range over natural numbers. Candidate expansion types and expressions are identified up to $\alpha$-equivalence.}
\label{fig:U-candidate-terms}
\end{figure}

Figure \ref{fig:U-candidate-terms} defines the syntax of candidate expansion types (or \emph{ce-types}), $\ctau$, and candidate expansion expressions  (or \emph{ce-expressions}), $\ce$. Candidate expansion types and expressions are identified up to $\alpha$-equivalence in the usual manner.

Each inner core form maps onto a candidate expansion form. We refer to these as the \emph{shared forms}. In particular:

\begin{itemize}
  \item Each type form maps onto a ce-type form according to the metafunction $\Cof{\tau}$, defined as follows:
  \begin{align*}
  \Cof{t} & = t\\
  \Cof{\aparr{\tau_1}{\tau_2}} & = \aceparr{\Cof{\tau_1}}{\Cof{\tau_2}}\\
  \Cof{\aall{t}{\tau}} & = \aceall{t}{\Cof{\tau}}\\
  \Cof{\arec{t}{\tau}} & = \acerec{t}{\Cof{\tau}}\\
  \Cof{\aprod{\labelset}{\mapschema{\tau}{i}{\labelset}}} & = \aceprod{\labelset}{\mapschemax{\Cofv}{\ctau}{i}{\labelset}}\\
  \Cof{\asum{\labelset}{\mapschema{\tau}{i}{\labelset}}} & = \acesum{\labelset}{\mapschemax{\Cofv}{\ctau}{i}{\labelset}}
  \end{align*}
  \item Each expanded expression form maps onto a ce-expression form according to the metafunction $\Cof{e}$, defined as follows:
  \begin{align*}
  \Cof{x} & = x\\
  \Cof{\aelam{\tau}{x}{e}} & = \acelam{\Cof{\tau}}{x}{\Cof{e}}\\
  \Cof{\aeap{e_1}{e_2}} & = \aceap{\Cof{e_1}}{\Cof{e_2}}\\
  \Cof{\aetlam{t}{e}} & = \acetlam{t}{\Cof{e}}\\
  \Cof{\aetap{e}{\tau}} & = \acetap{\Cof{e}}{\Cof{\tau}}\\
  \Cof{\aefold{t}{\tau}{e}} & = \acefold{t}{\Cof\tau}{\Cof e}\\
  \Cof{\aeunfold{e}} & = \aceunfold{\Cof{e}}\\
  \Cof{\aetpl{\labelset}{\mapschema{e}{i}{\labelset}}} & = \acetpl{\labelset}{\mapschemax{\Cofv}{e}{i}{\labelset}}\\
  \Cof{\aein{\labelset}{\ell}{\mapschema{\tau}{i}{\labelset}}{e}} &= \acein{\labelset}{\ell}{\mapschemax{\Cofv}{\tau}{i}{\labelset}}{\Cof{e}}\\
  \Cof{\aecase{\labelset}{\tau}{e}{\mapschemab{x}{e}{i}{\labelset}}} & = \acecase{\labelset}{\Cof\tau}{\Cof{e}}{\mapschemacx{\Cofv}{x}{e}{i}{\labelset}}
\end{align*}

\end{itemize}

There are two other candidate expansion forms, highlighted in gray in Figure \ref{fig:U-candidate-terms}: a ce-type form for \emph{references to spliced unexpanded types}, $\acesplicedt{m}{n}$, and a ce-expression form for \emph{references to spliced unexpanded expressions}, $\acesplicede{m}{n}$. %TSM utilize these to splice types and unexpanded expressions out of literal bodies.

\subsection{Candidate Expansion Validation}\label{sec:ce-validation-U}



The \emph{candidate expansion validation judgements} validate ce-types and ce-expressions and simultaneously generate their final expansions.% are types and expanded expressions, respectively.
\[\begin{array}{ll}
\textbf{Judgement Form} & \textbf{Description}\\
\cvalidT{\Delta}{\tscenev}{\ctau}{\tau} & \text{Candidate expansion type $\ctau$ is well-formed and has expansion $\tau$}\\
& \text{assuming $\Delta$ and type splicing scene $\tscenev$.}\\
\cvalidE{\Delta}{\Gamma}{\escenev}{\ce}{e}{\tau} & \text{Candidate expansion expression $\ce$ has expansion $e$ and type $\tau$}\\
& \text{assuming $\Delta$ and $\Gamma$ and expression splicing scene $\escenev$.}
\end{array}\]
\emph{Expression splicing scenes}, $\escenev$, are of the form $\esceneU{\uDelta}{\uGamma}{\uSigma}{b}$, and \emph{type splicing scenes}, $\tscenev$, are of the form $\tsceneU{\Delta}{b}$. We write $\tsfrom{\escenev}$ for the type splicing scene constructed by dropping the unexpanded typing context and ueTSM context from $\escenev$:
\[\tsfrom{\esceneU{\uDelta}{\uGamma}{\uSigma}{b}} = \tsceneU{\uDelta}{b}\]

The purpose of splicing scenes is to ``remember'', during the candidate expansion validation process, the unexpanded type formation context, $\uDelta$, unexpanded typing context, $\uGamma$, ueTSM context, $\uSigma$, and the literal body, $b$, from the ueTSM application site (cf. Rule (\ref{rule:expandsU-tsmap})), because these are necessary to validate references to spliced unexpanded types and expressions that appear within a candidate expansion.

\subsubsection{Candidate Expansion Type Validation}
The \emph{candidate expansion type validation judgement}, $\cvalidT{\Delta}{\tscenev}{\ctau}{\tau}$, is inductively defined by Rules (\ref*{rules:cvalidT-U}) as follows.

\paragraph{Shared Forms} Rules (\ref*{rule:cvalidT-U-tvar}) through (\ref*{rule:cvalidT-U-sum}), which validate ce-types of shared form, are defined below.
%Each of these rules is defined based on the corresponding type formation rule, i.e. Rules (\ref{rule:istypeU-var}) through (\ref{rule:istypeU-sum}), respectively. For example, the following candidate expansion type validation rules are based on type formation rules (\ref{rule:istypeU-var}), (\ref{rule:istypeU-parr}) and (\ref{rule:istypeU-all}), respectively: 
\begin{subequations}\label{rules:cvalidT-U}
\begin{equation}\label{rule:cvalidT-U-tvar}
\inferrule{ }{
  \cvalidT{\Delta, \Dhyp{t}}{\tscenev}{t}{t}
}
\end{equation}
\begin{equation}\label{rule:cvalidT-U-parr}
  \inferrule{
    \cvalidT{\Delta}{\tscenev}{\ctau_1}{\tau_1}\\
    \cvalidT{\Delta}{\tscenev}{\ctau_2}{\tau_2}
  }{
    \cvalidT{\Delta}{\tscenev}{\aceparr{\ctau_1}{\ctau_2}}{\aparr{\tau_1}{\tau_2}}
  }
\end{equation}
\begin{equation}\label{rule:cvalidT-U-all}
  \inferrule {
    \cvalidT{\Delta, \Dhyp{t}}{\tscenev}{\ctau}{\tau}
  }{
    \cvalidT{\Delta}{\tscenev}{\aceall{t}{\ctau}}{\aall{t}{\tau}}
  }
\end{equation}
\begin{equation}\label{rule:cvalidT-U-rec}
  \inferrule{
    \cvalidT{\Delta, \Dhyp{t}}{\tscenev}{\ctau}{\tau}
  }{
    \cvalidT{\Delta}{\tscenev}{\acerec{t}{\ctau}}{\arec{t}{\tau}}
  }
\end{equation}
\begin{equation}\label{rule:cvalidT-U-prod}
  \inferrule{
    \{\cvalidT{\Delta}{\tscenev}{\ctau_i}{\tau_i}\}_{i \in \labelset}
  }{
    \cvalidT{\Delta}{\tscenev}{\aceprod{\labelset}{\mapschema{\ctau}{i}{\labelset}}}{\aprod{\labelset}{\mapschema{\tau}{i}{\labelset}}}
  }
\end{equation}
\begin{equation}\label{rule:cvalidT-U-sum}
  \inferrule{
    \{\cvalidT{\Delta}{\tscenev}{\ctau_i}{\tau_i}\}_{i \in \labelset}
  }{
    \cvalidT{\Delta}{\tscenev}{\acesum{\labelset}{\mapschema{\ctau}{i}{\labelset}}}{\asum{\labelset}{\mapschema{\tau}{i}{\labelset}}}
  }
\end{equation}

Observe that, in each of these rules, the ce-type form and the type form in the conclusion correspond, and the premises correspond to those of the corresponding type formation rule, i.e. Rules (\ref{rules:istypeU}). The type splicing scene, $\tscenev$, passes opaquely through these rules. 
The following lemma establishes that each type can be expressed as a well-formed ce-type, under the same type formation context and any type splicing scene.
\begin{lemma}[Candidate Expansion Type Expressibility]\label{lemma:ce-type-expressibility-U} If $\istypeU{\Delta}{\tau}$ then $\cvalidT{\Delta}{\tscenev}{\Cof{\tau}}{\tau}$. \end{lemma}
\begin{proof}
By rule induction over Rules (\ref{rules:istypeU}). In each case, we apply the IH on or over each premise, then apply the corresponding ce-type validation rule in Rules (\ref{rules:cvalidT-U}).
\end{proof}
% We can express this scheme more precisely with the following rule transformation. For each rule in Rules (\ref{rules:istypeU}), 
% \begin{mathpar}
% % \refstepcounter{equation}
% % \label{rule:cvalidT-U-rec}
% % \refstepcounter{equation}
% % \label{rule:cvalidT-U-prod}
% % \refstepcounter{equation}
% % \label{rule:cvalidT-U-sum}
% % \inferrule{J_1\\\cdots\\J_k}{J}
% \end{mathpar}
% the corresponding candidate expansion type validation rule is
% \begin{mathpar}
% \inferrule{
%   \VTypof{J_1}\\
%   \cdots\\
%   \VTypof{J_k}
% }{
%   \VTypof{J}
% }
% \end{mathpar}
% where 
% \[\begin{split}
% \VTypof{\istypeU{\Delta}{\tau}} & = \cvalidT{\Delta}{\tscenev}{\VTypof{\tau}}{\tau}\\
% \VTypof{\{J_i\}_{i \in \labelset}} & = \{\VTypof{J_i}\}_{i \in \labelset}
% \end{split}\]
% and where $\VTypof{\tau}$, when $\tau$ is a metapattern of sort $\mathsf{Typ}$, is a metapattern of sort $\mathsf{CETyp}$ defined as follows:
% \begin{itemize}
% \item When $\tau$ is of definite form, $\VTypof{\tau}$ is defined as follows:
% \begin{align*}
% \VTypof{t} & = t\\
% \VTypof{\aparr{\tau_1}{\tau_2}} & = \aceparr{\VTypof{\tau_1}}{\VTypof{\tau_2}}\\
% \VTypof{\aall{t}{\tau}} & = \aceall{t}{\VTypof{\tau}}\\
% \VTypof{\arec{t}{\tau}} & = \acerec{t}{\VTypof{\tau}}\\
% \VTypof{\aprod{\labelset}{\mapschema{\tau}{i}{\labelset}}} & = \aceprod{\labelset}{\mapschemax{\VTypofv}{\tau}{i}{\labelset}}\\
% \VTypof{\asum{\labelset}{\mapschema{\tau}{i}{\labelset}}} & = \acesum{\labelset}{\mapschemax{\VTypofv}{\tau}{i}{\labelset}}
% \end{align*}
% \item When $\tau$ is of indefinite form, $\VTypof{\tau}$ is a uniquely corresponding metapattern also of indefinite form. For example, $\VTypof{\tau_1}=\ctau_1$ and $\VTypof{\tau_2}=\ctau_2$.
% \end{itemize}

% It is instructive to use this rule transformation to generate Rules (\ref{rule:cvalidT-U-tvar}) through (\ref{rule:cvalidT-U-all}) above. We omit the remaining rules, i.e. Rules (\ref*{rule:cvalidT-U-rec}) through (\ref*{rule:cvalidT-U-sum}). 

Notice that in Rule (\ref{rule:cvalidT-U-tvar}), only type variables tracked by the candidate expansion type formation context, $\Delta$, are validated. Type variables in the application site unexpanded type formation context, which appears within the type splicing scene, $\tscenev$, are not validated. Indeed, $\tscenev$ is not inspected by any of the rules above. This achieves \emph{context-independent expansion} as described in Sec. \ref{sec:splicing-and-hygiene} for type variables -- ueTSMs cannot impose ``hidden constraints'' on the application site unexpanded type formation context, because the type variables bound at the application site are simply not directly available to ce-types.

\paragraph{References to Spliced Types} The only ce-type form that does not correspond to a type form is $\acesplicedt{m}{n}$, which is a \emph{reference to a spliced unexpanded type}, i.e. it indicates that an unexpanded type should be parsed out from the literal body, $b$, which appears in the type splicing scene, beginning at position $m$ and ending at position $n$. 

Rule (\ref*{rule:cvalidT-U-splicedt}) governs this form:
\begin{equation}\label{rule:cvalidT-U-splicedt}
  \inferrule{
    \parseUTyp{\bsubseq{b}{m}{n}}{\utau}\\
    \expandsTU{\uDD{\uD}{\Delta_\text{app}}}{\utau}{\tau}\\
    \Delta \cap \Delta_\text{app} = \emptyset
  }{
    \cvalidT{\Delta}{\tsceneU{\uDD{\uD}{\Delta_\text{app}}}{b}}{\acesplicedt{m}{n}}{\tau}
  }
\end{equation}
The first premise of this rule extracts the indicated subsequence of $b$ using the partial metafunction $\bsubseq{b}{m}{n}$ and parses it using the partial metafunction $\mathsf{parseUTyp}(b)$, described in Sec. \ref{sec:syntax-U}, to produce the spliced unexpanded type itself, $\utau$.

The second premise of Rule (\ref{rule:cvalidT-U-splicedt}) performs type expansion of $\utau$ under the application site unexpanded type formation context, $\uDD{\uD}{\Delta_\text{app}}$, which appears in the type splicing scene. The hypotheses in the candidate expansion type formation context, $\Delta$, are not made available to $\tau$. %This enforces \emph{expansion independent splicing} as described in Sec. \ref{sec:splicing-and-hygiene} for type variables that appear in candidate expansion types. 

The third premise of Rule (\ref{rule:cvalidT-U-splicedt}) imposes the constraint that the candidate expansion's type formation context, $\Delta$, be disjoint from the application site type formation context, $\Delta_\text{app}$. This premise can always be discharged by $\alpha$-varying the candidate expansion that the reference to the spliced type appears within. 

This achieves \emph{expansion-independent splicing} as described in Sec. \ref{sec:splicing-and-hygiene} for type variables -- the TSM provider can choose type variable names freely within a candidate expansion, because the language prevents them from shadowing type variables at the application site (by $\alpha$-varying the candidate expansion as needed).%Such a change in bound variable names is possible again because variables bound by the ueTSM provider in a candidate expansion cannot ``leak into'' spliced terms because the hypotheses in $\Delta$ are not made available to the spliced type, $\tau$. 

Rules (\ref{rules:cvalidT-U}) validate the following lemma, which establishes that the final expansion of a valid ce-type is a well-formed type under the combined type formation context.
\begin{lemma}[Candidate Expansion Type Validation]\label{lemma:candidate-expansion-type-validation}
If $\cvalidT{\Delta}{\tsceneU{\uDD{\uD}{\Delta_\text{app}}}{b}}{\ctau}{\tau}$ then $\istypeU{\Dcons{\Delta}{\Delta_\text{app}}}{\tau}$.
\end{lemma}
\begin{proof} By rule induction over Rules (\ref{rules:cvalidT-U}).
\begin{byCases}
\item[\text{(\ref{rule:cvalidT-U-tvar})}] We have 
\begin{pfsteps*}
   \item $\Delta=\Delta', \Dhyp{t}$ \BY{assumption}
   \item $\ctau=t$ \BY{assumption}
   \item $\tau=t$ \BY{assumption}
   \item $\istypeU{\Delta', \Dhyp{t}}{t}$ \BY{Rule (\ref{rule:istypeU-var})} \pflabel{istype}
   \item $\istypeU{\Dcons{\Delta', \Dhyp{t}}{\Delta_\text{app}}}{t}$ \BY{Lemma \ref{lemma:weakening-U} over $\Delta_\text{app}$ to \pfref{istype}}
 \end{pfsteps*} 
\resetpfcounter

\item[\text{(\ref{rule:cvalidT-U-parr})}] We have
\begin{pfsteps*}
  \item $\ctau=\aceparr{\ctau_1}{\ctau_2}$ \BY{assumption}
  \item $\tau=\aparr{\tau_1}{\tau_2}$ \BY{assumption}
  \item $\cvalidT{\Delta}{\tsceneU{\uDD{\uD}{\Delta_\text{app}}}{b}}{\ctau_1}{\tau_1}$ \BY{assumption} \pflabel{cvalid-ctau1}
  \item $\cvalidT{\Delta}{\tsceneU{\uDD{\uD}{\Delta_\text{app}}}{b}}{\ctau_2}{\tau_2}$ \BY{assumption} \pflabel{cvalid-ctau2}
  \item $\istypeU{\Dcons{\Delta}{\Delta_\text{app}}}{\tau_1}$ \BY{IH on \pfref{cvalid-ctau1}} \pflabel{istype1}
  \item $\istypeU{\Dcons{\Delta}{\Delta_\text{app}}}{\tau_2}$ \BY{IH on \pfref{cvalid-ctau2}} \pflabel{istype2}
  \item $\istypeU{\Dcons{\Delta}{\Delta_\text{app}}}{\aparr{\tau_1}{\tau_2}}$ \BY{Rule (\ref{rule:istypeU-parr}) on \pfref{istype1} and \pfref{istype2}}
\end{pfsteps*}
\resetpfcounter

\item[\text{(\ref{rule:cvalidT-U-all})}] We have
\begin{pfsteps*}
  \item $\ctau=\aceall{t}{\ctau'}$ \BY{assumption}
  \item $\tau=\aall{t}{\tau'}$ \BY{assumption}
  \item $\cvalidT{\Delta, \Dhyp{t}}{\tsceneU{\uDD{\uD}{\Delta_\text{app}}}{b}}{\ctau'}{\tau'}$ \BY{assumption} \label{cvalidT}
  \item $\istypeU{\Dcons{\Delta, \Dhyp{t}}{\Delta_\text{app}}}{\tau'}$ \BY{IH on \pfref{cvalidT}} \pflabel{istypeU1}
  \item $\istypeU{\Dcons{\Delta}{\Delta_\text{app}}, \Dhyp{t}}{\tau'}$ \BY{exchange over $\Delta_\text{app}$ on \pfref{istypeU1}} \pflabel{istypeU2}
  \item $\istypeU{\Dcons{\Delta}{\Delta_\text{app}}}{\aall{t}{\tau'}}$ \BY{Rule (\ref{rule:istypeU-all}) on \pfref{istypeU2}}
\end{pfsteps*}
\resetpfcounter

\item[{\text{(\ref{rule:cvalidT-U-rec})}}~\text{through}~{\text{(\ref{rule:cvalidT-U-sum})}}] These cases follow analagously, i.e. we apply the IH to or over all ce-type validation premises, apply exchange as needed, and then apply the corresponding type formation rule.
\\
% \item[\text{(\ref{rule:cvalidT-U-rec})}] We have
% \begin{pfsteps*}
%   \item $\ctau=\acerec{t}{\ctau'}$ \BY{assumption}
%   \item $\tau=\arec{t}{\tau'}$ \BY{assumption}
%   \item $\cvalidT{\Delta, \Dhyp{t}}{\tsceneU{\Delta_\text{app}}{b}}{\ctau'}{\tau'}$ \BY{assumption} \label{cvalidT}
%   \item $\istypeU{\Dcons{\Delta, \Dhyp{t}}{\Delta_\text{app}}}{\tau'}$ \BY{IH on \pfref{cvalidT}} \pflabel{istypeU1}
%   \item $\istypeU{\Dcons{\Delta}{\Delta_\text{app}}, \Dhyp{t}}{\tau'}$ \BY{exchange over $\Delta_\text{app}$ on \pfref{istypeU1}} \pflabel{istypeU2}
%   \item $\istypeU{\Dcons{\Delta}{\Delta_\text{app}}}{\arec{t}{\tau'}}$ \BY{Rule (\ref{rule:istypeU-rec}) on \pfref{istypeU2}}
% \end{pfsteps*}
% \resetpfcounter

% \item[\text{(\ref{rule:cvalidT-U-prod})}] We have
% \begin{pfsteps*}
% \item $\ctau=\aceprod{\labelset}{\mapschema{\ctau}{i}{\labelset}}$ \BY{assumption}  
% \item $\tau=\aprod{\labelset}{\mapschema{\tau}{i}{\labelset}}$ \BY{assumption}
% \item $\{\cvalidT{\Delta}{\tsceneU{\Delta_\text{app}}{b}}{\ctau_i}{\tau_i}\}_{i \in \labelset}$ \BY{assumption} \pflabel{cvalidT-ass}
% \item $\{\istypeU{\Dcons{\Delta}{\Delta_\text{app}}}{\tau_i}\}_{i \in \labelset}$ \BY{IH on \pfref{cvalidT-ass}$_i$ for each $i \in \labelset$} \pflabel{istype}
% \item $\istypeU{\Dcons{\Delta}{\Delta_\text{app}}}{\aprod{\labelset}{\mapschema{\tau}{i}{\labelset}}}$ \BY{Rule (\ref{rule:istypeU-prod}) on \pfref{istype}}
% \end{pfsteps*}
% \resetpfcounter 

% \item[\text{(\ref{rule:cvalidT-U-sum})}] We have
% \begin{pfsteps*}
% \item $\ctau=\acesum{\labelset}{\mapschema{\ctau}{i}{\labelset}}$ \BY{assumption}  
% \item $\tau=\asum{\labelset}{\mapschema{\tau}{i}{\labelset}}$ \BY{assumption}
% \item $\{\cvalidT{\Delta}{\tsceneU{\Delta_\text{app}}{b}}{\ctau_i}{\tau_i}\}_{i \in \labelset}$ \BY{assumption} \pflabel{cvalidT-ass}
% \item $\{\istypeU{\Dcons{\Delta}{\Delta_\text{app}}}{\tau_i}\}_{i \in \labelset}$ \BY{IH on \pfref{cvalidT-ass}$_i$ for each $i \in \labelset$} \pflabel{istype}
% \item $\istypeU{\Dcons{\Delta}{\Delta_\text{app}}}{\asum{\labelset}{\mapschema{\tau}{i}{\labelset}}}$ \BY{Rule (\ref{rule:istypeU-sum}) on \pfref{istype}}
% \end{pfsteps*}
% \resetpfcounter

\item[\text{(\ref{rule:cvalidT-U-splicedt})}] We have
\begin{pfsteps*}
\item $\ctau=\acesplicedt{m}{n}$ \BY{assumption}
\item $\parseUTyp{\bsubseq{b}{m}{n}}{\utau}$ \BY{assumption}
\item $\expandsTU{\uDD{\uD}{\Delta_\text{app}}}{\utau}{\tau}$ \BY{assumption} \label{expandsTU}
\item $\istypeU{\Delta_\text{app}}{\tau}$ \BY{Lemma \ref{lemma:type-expansion-U} on \pfref{expandsTU}}\pflabel{istype}
\item $\istypeU{\Dcons{\Delta}{\Delta_\text{app}}}{\tau}$ \BY{Lemma \ref{lemma:weakening-U} over $\Delta$ on \pfref{istype} and exchange over $\Delta$}
\end{pfsteps*}
\resetpfcounter
\end{byCases}
\end{proof}
\end{subequations}
% \begin{subequations}\label{rules:cvalidT-U}
% \begin{equation}\label{rule:cvalidT-U-tvar}
% \inferrule{ }{
%   \cvalidT{\Delta, \Dhyp{t}}{\tscenev}{t}{t}
% }
% \end{equation}
% \begin{equation}\label{rule:cvalidT-U-parr}
%   \inferrule{
%     \cvalidT{\Delta}{\tscenev}{\ctau_1}{\tau_1}\\
%     \cvalidT{\Delta}{\tscenev}{\ctau_2}{\tau_2}
%   }{
%     \cvalidT{\Delta}{\tscenev}{\aceparr{\ctau_1}{\ctau_2}}{\aparr{\tau_1}{\tau_2}}
%   }
% \end{equation}
% \begin{equation}\label{rule:cvalidT-U-all}
%   \inferrule {
%     \cvalidT{\Delta, \Dhyp{t}}{\tscenev}{\ctau}{\tau}
%   }{
%     \cvalidT{\Delta}{\tscenev}{\aceall{t}{\ctau}}{\aall{t}{\tau}}
%   }
% \end{equation}
% \begin{equation}\label{rule:cvalidT-U-rec}
%   \inferrule{
%     \cvalidT{\Delta, \Dhyp{t}}{\tscenev}{\ctau}{\tau}
%   }{
%     \cvalidT{\Delta}{\tscenev}{\acerec{t}{\ctau}}{\arec{t}{\tau}}
%   }
% \end{equation}
% \begin{equation}\label{rule:cvalidT-U-prod}
%   \inferrule{
%     \{\cvalidT{\Delta}{\tscenev}{\ctau_i}{\tau_i}\}_{i \in \labelset}
%   }{
%     \cvalidT{\Delta}{\tscenev}{\aceprod{\labelset}{\mapschema{\ctau}{i}{\labelset}}}{\aprod{\labelset}{\mapschema{\tau}{i}{\labelset}}}
%   }
% \end{equation}
% \begin{equation}\label{rule:cvalidT-U-sum}
%   \inferrule{
%     \{\cvalidT{\Delta}{\tscenev}{\ctau_i}{\tau_i}\}_{i \in \labelset}
%   }{
%     \cvalidT{\Delta}{\tscenev}{\acesum{\labelset}{\mapschema{\ctau}{i}{\labelset}}}{\asum{\labelset}{\mapschema{\tau}{i}{\labelset}}}
%   }
% \end{equation}
% \end{subequations}
%Each form of type, $\tau$, corresponds to a form of candidate expansion type, $\ctau$ (compare Figures \ref{fig:U-expanded-terms} and \ref{fig:U-candidate-terms}). For each type formation rule in Rules (\ref{rules:istypeU}), there is a corresponding candidate expansion type validation rule -- Rules (\ref{rule:cvalidT-U-tvar}) to (\ref{rule:cvalidT-U-sum}) -- where the candidate expansion type and the final expansion correspond. The premises also correspond. 



\subsubsection{Candidate Expansion Expression Validation}
The \emph{candidate expansion expression validation judgement}, $\cvalidE{\Delta}{\Gamma}{\escenev}{\ce}{e}{\tau}$, is defined mutually inductively with the typed expansion judgement by Rules (\ref*{rules:cvalidE-U}) as follows.% This is necessary because a typed expansion judgement appears as a premise in Rule (\ref{rule:cvalidE-U-splicede}) below, and a candidate expansion expression validation judgement appears as a premise in Rule (\ref{rule:expandsU-tsmap}) above.

\paragraph{Shared Forms} Rules (\ref*{rule:cvalidE-U-var}) through (\ref*{rule:cvalidE-U-case}) validate ce-expressions of shared form. The first three of these rules are defined below:
%For each expanded expression form defined in Figure \ref{fig:U-expanded-terms}, Figure \ref{fig:U-candidate-terms} defines a corresponding candidate expansion expression form. The validation rules for candidate expansion expressions of these forms are each based on the corresponding typing rule in Rules (\ref{rules:hastypeU}). For example, the validation rules for candidate expansion expressions of variable, function and function application form  are based on Rules (\ref{rule:hastypeU-var}) through (\ref{rule:hastypeU-ap}), respectively:
\begin{subequations}\label{rules:cvalidE-U}
\begin{equation}\label{rule:cvalidE-U-var}
\inferrule{ }{
  \cvalidE{\Delta}{\Gamma, \Ghyp{x}{\tau}}{\escenev}{x}{x}{\tau}
}
\end{equation}
\begin{equation}\label{rule:cvalidE-U-lam}
\inferrule{
  \cvalidT{\Delta}{\tsfrom{\escenev}}{\ctau}{\tau}\\
  \cvalidE{\Delta}{\Gamma, \Ghyp{x}{\tau}}{\escenev}{\ce}{e}{\tau'}
}{
  \cvalidE{\Delta}{\Gamma}{\escenev}{\acelam{\ctau}{x}{\ce}}{\aelam{\tau}{x}{e}}{\aparr{\tau}{\tau'}}
}
\end{equation}
\begin{equation}\label{rule:cvalidE-U-ap}
  \inferrule{
    \cvalidE{\Delta}{\Gamma}{\escenev}{\ce_1}{e_1}{\aparr{\tau}{\tau'}}\\
    \cvalidE{\Delta}{\Gamma}{\escenev}{\ce_2}{e_2}{\tau}
  }{
    \cvalidE{\Delta}{\Gamma}{\escenev}{\aceap{\ce_1}{\ce_2}}{\aeap{e_1}{e_2}}{\tau'}
  }
\end{equation}
Observe that, in each of these rules, the ce-expression form and the expanded expression form in the conclusion correspond, and the premises correspond to those of the corresponding typing rule, i.e. Rules (\ref{rule:hastypeU-var}) through (\ref{rule:hastypeU-ap}), respectively. The expression splicing scene, $\escenev$, passes opaquely through these rules.

We can express this scheme more precisely with the following rule transformation. For each rule in Rules (\ref{rules:hastypeU}),
\begin{mathpar}\refstepcounter{equation}
\label{rule:cvalidE-U-tlam}
\refstepcounter{equation}
\label{rule:cvalidE-U-tap}
\refstepcounter{equation}
\label{rule:cvalidE-U-fold}
\refstepcounter{equation}
\label{rule:cvalidE-U-unfold}
\refstepcounter{equation}
\label{rule:cvalidE-U-tpl}
\refstepcounter{equation}
\label{rule:cvalidE-U-pr}
\refstepcounter{equation}
\label{rule:cvalidE-U-in}
\refstepcounter{equation}
\label{rule:cvalidE-U-case}
  \inferrule{
    J_1\\
    \cdots\\
    J_k
  }{
    J
  }
\end{mathpar}
the corresponding candidate expansion expression validation rule is 
\begin{mathpar}
  \inferrule{
    \Cof{J_1}\\
    \cdots\\
    \Cof{J_k}
  }{
    \Cof{J}
  }
\end{mathpar}
where 
\[\begin{split}
  \Cof{\istypeU{\Delta}{\tau}} & = \cvalidT{\Delta}{\tsfrom{\escenev}}{\Cof{\tau}}{\tau}\\
  \Cof{\hastypeU{\Delta}{\Gamma}{e}{\tau}} & = \cvalidE{\Delta}{\Gamma}{\escenev}{\Cof{e}}{e}{\tau}\\
  \Cof{\{J_i\}_{i \in \labelset}} & = \{\Cof{J_i}\}_{i \in \labelset}
\end{split}\]
and where:
\begin{itemize}
\item $\Cof{\tau}$ is defined as follows:
  \begin{itemize}
  \item When $\tau$ is of definite form, $\Cof{\tau}$ is defined as in Sec. \ref{sec:ce-syntax-U}.
  \item When $\tau$ is of indefinite form, $\Cof{\tau}$ is a uniquely corresponding metavariable of sort $\mathsf{CETyp}$ also of indefinite form. For example, $\Cof{\tau_1}=\ctau_1$ and $\Cof{\tau_2}=\ctau_2$.
  \end{itemize}
\item $\Cof{e}$ is defined as follows
  \begin{itemize}
  \item When $e$ is of definite form, $\Cof{e}$ is defined as in Sec. \ref{sec:ce-syntax-U}. 
  \item When $e$ is of indefinite form, $\Cof{e}$ is a uniquely corresponding metavariable of sort $\mathsf{CEExp}$ also of indefinite form. For example, $\Cof{e_1}=\ce_1$ and $\Cof{e_2}=\ce_2$.
  \end{itemize}
\end{itemize}

It is instructive to use this rule transformation to generate Rules (\ref{rule:cvalidE-U-var}) through (\ref{rule:cvalidE-U-ap}) above. We omit the remaining rules for shared forms, i.e. Rules (\ref*{rule:cvalidE-U-tlam}) through (\ref*{rule:cvalidE-U-case}).

The following lemma establishes that each well-typed expanded expression, $e$, can be expressed as a valid ce-expression, $\Cof{e}$, that is assigned the same type under any expression splicing scene.
\begin{lemma}[Candidate Expansion Expression Expressibility]\label{lemma:ce-expressions-expressibility-U} If $\hastypeU{\Delta}{\Gamma}{e}{\tau}$ then $\cvalidE{\Delta}{\Gamma}{\escenev}{\Cof{e}}{e}{\tau}$.\end{lemma}
\begin{proof} By rule induction over Rules (\ref{rules:hastypeU}). The rule transformation above guarantees that this lemma holds by construction. In particular, in each case, we apply Lemma \ref{lemma:ce-type-expressibility-U} to or over each type formation premise, the IH to or over each typing premise, then apply the corresponding ce-expression validation rule in Rules (\ref{rule:cvalidE-U-var}) through (\ref{rule:cvalidE-U-case}).
\end{proof}

Notice that in Rule (\ref{rule:cvalidE-U-var}), only variables tracked by the candidate expansion typing context, $\Gamma$, are validated. Variables  in the application site unexpanded typing context, which appears within the expression splicing scene $\escenev$, are not validated. Indeed, $\escenev$ is not inspected by any of the rules above. This achieves \emph{context-independent expansion} as described in Sec. \ref{sec:splicing-and-hygiene} -- ueTSMs cannot impose ``hidden constraints'' on the application site unexpanded typing context, because the variable bindings at the application site are not directly available to candidate expansions.

\paragraph{References to Spliced Unexpanded Expressions} The only ce-expression form that does not correspond to an expanded expression form is $\acesplicede{m}{n}$, which is a \emph{reference to a spliced unexpanded expression}, i.e. it indicates that an unexpanded expression should be parsed out from the literal body beginning at position $m$ and ending at position $n$. Rule (\ref*{rule:cvalidE-U-splicede}) governs this form:
\begin{equation}\label{rule:cvalidE-U-splicede}
\inferrule{
  \parseUExp{\bsubseq{b}{m}{n}}{\ue}\\
  \expandsU{\uDD{\uD}{\Delta_\text{app}}}{\uGG{\uG}{\Gamma_\text{app}}}{\uSigma}{\ue}{e}{\tau}\\\\
  \Delta \cap \Delta_\text{app} = \emptyset\\
  \domof{\Gamma} \cap \domof{\Gamma_\text{app}} = \emptyset
}{
  \cvalidE{\Delta}{\Gamma}{\esceneU{\uDD{\uD}{\Delta_\text{app}}}{\uGG{\uG}{\Gamma_\text{app}}}{\uSigma}{b}}{\acesplicede{m}{n}}{e}{\tau}
}
\end{equation}
% \begin{equation}\label{rule:cvalidE-U-splicede}
% \inferrule{
%   \parseUExp{\bsubseq{b}{m}{n}}{\ue}\\\\
%   \expandsU{\Delta_\text{app}}{\Gamma_\text{app}}{\uSigma}{\ue}{e}{\tau}\\
%   \Delta \cap \Delta_\text{app} = \emptyset\\
%   \domof{\Gamma} \cap \domof{\Gamma_\text{app}} = \emptyset
% }{
%   \cvalidE{\Delta}{\Gamma}{\esceneU{\Delta_\text{app}}{\Gamma_\text{app}}{\uSigma}{b}}{\splicede{m}{n}}{e}{\tau}
% }
% \end{equation}
The first premise of this rule extracts the indicated subsequence of $b$ using the partial metafunction $\bsubseq{b}{m}{n}$ and parses it using the partial metafunction $\mathsf{parseUExp}(b)$, described in Sec. \ref{sec:syntax-U}, to produce the referenced spliced unexpanded expression, $\ue$.

The second premise of Rule (\ref{rule:cvalidE-U-splicede}) types and expands the spliced unexpanded expression $\ue$ assuming the application site contexts that appear in the expression splicing scene. The hypotheses in the candidate expansion type formation context, $\Delta$, and typing context, $\Gamma$, are not made available to $\ue$. 

The third premise of Rule (\ref{rule:cvalidE-U-splicede}) imposes the constraint that the candidate expansion's type formation context, $\Delta$, be disjoint from the application site type formation context, $\Delta_\text{app}$. Similarly, the fourth premise requires that the candidate expansion's typing context, $\Gamma$, be disjoint from the application site typing context, $\Gamma_\text{app}$. These two premises can always be discharged by $\alpha$-varying the ce-expression that the reference to the spliced unexpanded expression appears within. 

This achieves \emph{expansion-independent splicing} as described in Sec. \ref{sec:splicing-and-hygiene} -- the TSM provider can choose variable names freely within a candidate expansion, because the language prevents them from shadowing those at the application site (by $\alpha$-varying the candidate expansion as needed).
\end{subequations}
% \begin{subequations}\label{rules:cvalidE-U}
% \begin{equation}\label{rule:cvalidE-U-var}
% \inferrule{ }{
%   \cvalidE{\Delta}{\Gamma, \Ghyp{x}{\tau}}{\escenev}{x}{x}{\tau}
% }
% \end{equation}
% \begin{equation}\label{rule:cvalidE-U-lam}
% \inferrule{
%   \cvalidT{\Delta}{\tsfrom{\escenev}}{\ctau}{\tau}\\
%   \cvalidE{\Delta}{\Gamma, \Ghyp{x}{\tau}}{\escenev}{\ce}{e}{\tau'}
% }{
%   \cvalidE{\Delta}{\Gamma}{\escenev}{\acelam{\ctau}{x}{\ce}}{\aelam{\tau}{x}{e}}{\aparr{\tau}{\tau'}}
% }
% \end{equation}
% \begin{equation}\label{rule:cvalidE-U-ap}
%   \inferrule{
%     \cvalidE{\Delta}{\Gamma}{\escenev}{\ce_1}{e_1}{\aparr{\tau}{\tau'}}\\
%     \cvalidE{\Delta}{\Gamma}{\escenev}{\ce_2}{e_2}{\tau}
%   }{
%     \cvalidE{\Delta}{\Gamma}{\escenev}{\aceap{\ce_1}{\ce_2}}{\aeap{e_1}{e_2}}{\tau'}
%   }
% \end{equation}
% \begin{equation}\label{rule:cvalidE-U-tlam}
%   \inferrule{
%     \cvalidE{\Delta, \Dhyp{t}}{\Gamma}{\escenev}{\ce}{e}{\tau}
%   }{
%     \cvalidEX{\acetlam{t}{\ce}}{\aetlam{t}{e}}{\aall{t}{\tau}}
%   }
% \end{equation}
% \begin{equation}\label{rule:cvalidE-U-tap}
%   \inferrule{
%     \cvalidEX{\ce}{e}{\aall{t}{\tau}}\\
%     \cvalidT{\Delta}{\tsfrom{\escenev}}{\ctau'}{\tau'}
%   }{
%     \cvalidEX{\acetap{\ce}{\ctau'}}{\aetap{e}{\tau'}}{[\tau'/t]\tau}
%   }
% \end{equation}
% \begin{equation}\label{rule:cvalidE-U-fold}
%   \inferrule{
%     \cvalidT{\Delta, \Dhyp{t}}{\escenev}{\ctau}{\tau}\\
%     \cvalidEX{\ce}{e}{[\arec{t}{\tau}/t]\tau}
%   }{
%     \cvalidEX{\acefold{t}{\ctau}{\ce}}{\aefold{t}{\tau}{e}}{\arec{t}{\tau}}
%   }
% \end{equation}
% \begin{equation}\label{rule:cvalidE-U-unfold}
%   \inferrule{
%     \cvalidEX{\ce}{e}{\arec{t}{\tau}}
%   }{
%     \cvalidEX{\aceunfold{\ce}}{\aeunfold{e}}{[\arec{t}{\tau}/t]\tau}
%   }
% \end{equation}
% \begin{equation}\label{rule:cvalidE-U-tpl}
%   \inferrule{
%     \{\cvalidEX{\ce_i}{e_i}{\tau_i}\}_{i \in \labelset}
%   }{
%     \cvalidEX{\acetpl{\labelset}{\mapschema{\ce}{i}{\labelset}}}{\aetpl{\labelset}{\mapschema{e}{i}{\labelset}}}{\aprod{\labelset}{\mapschema{\tau}{i}{\labelset}}}
%   }
% \end{equation}
% \begin{equation}\label{rule:cvalidE-U-pr}
%   \inferrule{
%     \cvalidEX{\ce}{e}{\aprod{\labelset, \ell}{\mapschema{\tau}{i}{\labelset}; \mapitem{\ell}{\tau}}}
%   }{
%     \cvalidEX{\acepr{\ell}{\ce}}{\aepr{\ell}{e}}{\tau}
%   }
% \end{equation}
% \begin{equation}\label{rule:cvalidE-U-in}
%   \inferrule{
%     \{\cvalidT{\Delta}{\tsfrom{\escenev}}{\ctau_i}{\tau_i}\}_{i \in \labelset}\\
%     \cvalidT{\Delta}{\tsfrom{\escenev}}{\ctau}{\tau}\\
%     \cvalidEX{\ce}{e}{\tau}
%   }{
%     \left\{\shortstack{$\Delta~\Gamma \vdash_\uSigma \acein{\labelset, \ell}{\ell}{\mapschema{\ctau}{i}{\labelset}; \mapitem{\ell}{\ctau}}{\ce}$\\$\leadsto$\\$\aein{\labelset, \ell}{\ell}{\mapschema{\tau}{i}{\labelset}; \mapitem{\ell}{\tau}}{e} : \asum{\labelset, \ell}{\mapschema{\tau}{i}{\labelset}; \mapitem{\ell}{\tau}}$\vspace{-1.2em}}\right\}
%   }
% \end{equation}
% \begin{equation}\label{rule:cvalidE-U-case}
%   \inferrule{
%     \cvalidEX{\ce}{e}{\asum{\labelset}{\mapschema{\tau}{i}{\labelset}}}\\
%     \{\cvalidE{\Delta}{\Gamma, \Ghyp{x_i}{\tau_i}}{\escenev}{\ue_i}{e_i}{\tau}\}_{i \in \labelset}
%   }{
%     \cvalidEX{\acecase{\labelset}{\ce}{\mapschemab{x}{\ce}{i}{\labelset}}}{\aecase{\labelset}{e}{\mapschemab{x}{e}{i}{\labelset}}}{\tau}
%   }
% \end{equation}
% \begin{equation}\label{rule:cvalidE-U-splicede}
% \inferrule{
%   \parseUExp{\bsubseq{b}{m}{n}}{\ue}\\\\
%   \Delta \cap \Delta_\text{app} = \emptyset\\
%   \domof{\Gamma} \cap \domof{\Gamma_\text{app}} = \emptyset\\
%   \expandsU{\Delta_\text{app}}{\Gamma_\text{app}}{\uSigma}{\ue}{e}{\tau}
% }{
%   \cvalidE{\Delta}{\Gamma}{\esceneU{\Delta_\text{app}}{\Gamma_\text{app}}{\uSigma}{b}}{\acesplicede{m}{n}}{e}{\tau}
% }
% \end{equation}
% \end{subequations}

% Each form of expanded expression, $e$, corresponds to a form of candidate expansion expression, $\ce$ (compare Figure \ref{fig:U-expanded-terms} and Figure \ref{fig:U-candidate-terms}). For each typing rule in Rules \ref{rules:hastypeU}, there is a corresponding candidate expansion expression validation rule -- Rules (\ref{rule:cvalidE-U-var}) to (\ref{rule:cvalidE-U-case}) -- where the candidate expansion expression and expanded expression correspond. The premises also correspond.


%Candidate expansions cannot themselves define or apply TSMs. This simplifies our metatheory, though it can be inconvenient at times for TSM providers. We discuss adding the ability to use TSMs within candidate expansions in Sec. \ref{sec:tsms-in-expansions}.


\subsection{Metatheory}
For the judgements we have defined to form a sensible language, we must have that typed expansion and candidate expansion expression validation be consistent with typing. Formally, this can be expressed as follows.

\begin{theorem}[Typed Expansion]\label{thm:typed-expansion-U} Both of the following hold:
\begin{enumerate}
\item If $\expandsU{\uDD{\uD}{\Delta}}{\uGG{\uG}{\Gamma}}{\uAS{\uA}{\Sigma}}{\ue}{e}{\tau}$ and $\uetsmenv{\Delta}{\Sigma}$ then $\hastypeU{\Delta}{\Gamma}{e}{\tau}$.
\item If $\cvalidE{\Delta}{\Gamma}{\esceneU{\uDD{\uD}{\Delta_\text{app}}}{\uGG{\uG}{\Gamma_\text{app}}}{\uAS{\uA}{\Sigma}}{b}}{\ce}{e}{\tau}$ and $\uetsmenv{\Delta_\text{app}}{\Sigma}$ and $\Delta \cap \Delta_\text{app} = \emptyset$ and $\domof{\Gamma} \cap \domof{\Gamma_\text{app}} = \emptyset$ then $\hastypeU{\Dcons{\Delta}{\Delta_\text{app}}}{\Gcons{\Gamma}{\Gamma_\text{app}}}{e}{\tau}$.
\end{enumerate}
\end{theorem}
\begin{proof}
By mutual rule induction over Rules (\ref{rules:expandsU}) and Rules (\ref{rules:cvalidE-U}). 

The proof of part 1 proceeds by inducting over the typed expansion assumption. In the following cases, let $\uDelta=\uDD{\uD}{\Delta}$ and $\uGamma=\uGG{\uG}{\Gamma}$ and $\uSigma=\uAS{\uA}{\Sigma}$.
\begin{byCases}
\item[\text{(\ref{rule:expandsU-var})}] We have
\begin{pfsteps}
  \item \ue=\ux \BY{assumption}
  \item e=x \BY{assumption}
  \item \Gamma=\Gamma', \Ghyp{x}{\tau} \BY{assumption}
  \item \hastypeU{\Delta}{\Gamma', \Ghyp{x}{\tau}}{x}{\tau} \BY{Rule (\ref{rule:hastypeU-var})}
\end{pfsteps}
\resetpfcounter

\item[\text{(\ref{rule:expandsU-lam})}] We have 
\begin{pfsteps}
  \item \ue=\aulam{\utau_1}{\ux}{\ue'} \BY{assumption}
  \item e=\aelam{\tau_1}{x}{e'} \BY{assumption}
  \item \tau=\aparr{\tau_1}{\tau_2} \BY{assumption}
  \item \expandsTU{\uDelta}{\utau_1}{\tau_1} \BY{assumption} \pflabel{istype}
  \item \expandsU{\uDelta}{\uGamma, \uGhyp{\ux}{x}{\tau_1}}{\uSigma}{\ue'}{e'}{\tau_2} \BY{assumption} \pflabel{expandsU}
  \item \uetsmenv{\Delta}{\Sigma} \BY{assumption} \pflabel{uetsmenv}
  \item \istypeU{\Delta}{\tau_1} \BY{Lemma \ref{lemma:type-expansion-U} on \pfref{istype}} \pflabel{istype2}
  \item \hastypeU{\Delta}{\Gamma, \Ghyp{x}{\tau_1}}{e'}{\tau_2} \BY{IH on \pfref{expandsU} and \pfref{uetsmenv}} \pflabel{hastypeU}
  \item \hastypeU{\Delta}{\Gamma}{\aelam{\tau_1}{x}{e'}}{\aparr{\tau_1}{\tau_2}} \BY{Rule (\ref{rule:hastypeU-lam}) on \pfref{istype2} and \pfref{hastypeU}}
\end{pfsteps}
\resetpfcounter

\item[\text{(\ref{rule:expandsU-ap})}] We have
\begin{pfsteps}
  \item \ue=\auap{\ue_1}{\ue_2} \BY{assumption}
  \item e=\aeap{e_1}{e_2} \BY{assumption}
  \item \expandsU{\uDelta}{\uGamma}{\uSigma}{\ue_1}{e_1}{\aparr{\tau_1}{\tau}} \BY{assumption}\pflabel{expandsU1}
  \item \expandsU{\uDelta}{\uGamma}{\uSigma}{\ue_2}{e_2}{\tau_1} \BY{assumption}\pflabel{expandsU2}
  \item \uetsmenv{\Delta}{\Sigma} \BY{assumption} \pflabel{uetsmenv}
  \item \hastypeU{\Delta}{\Gamma}{e_1}{\aparr{\tau_1}{\tau}} \BY{IH on \pfref{expandsU1} and \pfref{uetsmenv}}\pflabel{hastypeU1}
  \item \hastypeU{\Delta}{\Gamma}{e_2}{\tau_1} \BY{IH on \pfref{expandsU2} and \pfref{uetsmenv}}\pflabel{hastypeU2}
  \item \hastypeU{\Delta}{\Gamma}{\aeap{e_1}{e_2}}{\tau} \BY{Rule (\ref{rule:hastypeU-ap}) on \pfref{hastypeU1} and \pfref{hastypeU2}}
\end{pfsteps}
\resetpfcounter

\item[\text{(\ref{rule:expandsU-tlam})}~\text{through}~\text{(\ref{rule:expandsU-case})}] These cases follow analagously, i.e. we apply Lemma \ref{lemma:type-expansion-U} to or over the type expansion premises and the IH to or over the typed expression expansion premises and then apply the corresponding typing rule.
\\

\item[\text{(\ref{rule:expandsU-syntax})}] We have 
\begin{pfsteps}
  \item \ue=\audefuetsm{\utau'}{\eparse}{\tsmv}{\ue'} \BY{assumption}
  \item \expandsTU{\uDelta}{\utau'}{\tau'} \BY{assumption} \pflabel{expandsTU}
  \item \hastypeU{\emptyset}{\emptyset}{\eparse}{\aparr{\tBody}{\tParseResultExp}} \BY{assumption}\pflabel{eparse}
  \item \expandsU{\uDelta}{\uGamma}{\uSigma, \uShyp{\tsmv}{a}{\tau'}{\eparse}}{\ue'}{e}{\tau} \BY{assumption}\pflabel{expandsU}
  \item \uetsmenv{\Delta}{\Sigma} \BY{assumption}\pflabel{uetsmenv1}
  \item \istypeU{\Delta}{\tau'} \BY{Lemma \ref{lemma:type-expansion-U} to \pfref{expandsTU}} \pflabel{istype}
  \item \uetsmenv{\Delta}{\Sigma, \xuetsmbnd{\tsmv}{\tau'}{\eparse}} \BY{Definition \ref{def:ueTSM-def-ctx-formation} on \pfref{uetsmenv1}, \pfref{istype} and \pfref{eparse}}\pflabel{uetsmenv3}
  \item \hastypeU{\Delta}{\Gamma}{e}{\tau} \BY{IH on \pfref{expandsU} and \pfref{uetsmenv3}}
\end{pfsteps}
\resetpfcounter 

\item[\text{(\ref{rule:expandsU-tsmap})}] We have 
\begin{pfsteps}
  \item \ue=\autsmap{b}{\tsmv} \BY{assumption}
  \item \uA = \uA', \vExpands{\tsmv}{a} \BY{assumption}
  \item \Sigma=\Sigma', \xuetsmbnd{a}{\tau}{\eparse} \BY{assumption}
  \item \encodeBody{b}{\ebody} \BY{assumption}
  \item \evalU{\eparse(\ebody)}{\inj{\lbltxt{Success}}{\ecand}} \BY{assumption}
  \item \decodeCondE{\ecand}{\ce} \BY{assumption}
  \item \cvalidE{\emptyset}{\emptyset}{\esceneU{\uDelta}{\uGamma}{\uSigma}{b}}{\ce}{e}{\tau} \BY{assumption}\pflabel{cvalidE}
  \item \uetsmenv{\Delta}{\Sigma} \BY{assumption} \pflabel{uetsmenv}
  \item \emptyset \cap \Delta = \emptyset \BY{finite set intersection identity} \pflabel{delta-cap}
  \item \domof{\emptyset} \cap \domof{\Gamma} = \emptyset \BY{finite set intersection identity} \pflabel{gamma-cap}
  \item \hastypeU{\emptyset \cup \Delta}{\emptyset \cup \Gamma}{e}{\tau} \BY{IH, part 2 on \pfref{cvalidE}, \pfref{uetsmenv}, \pfref{delta-cap}, and \pfref{gamma-cap}} \pflabel{penultimate}
  \item \hastypeU{\Delta}{\Gamma}{e}{\tau} \BY{definition of finite set union over \pfref{penultimate}}
\end{pfsteps}
\resetpfcounter
\end{byCases}

The second part of the theorem proceeds by induction over the candidate expansion expression validation assumption as follows. In the following cases, let $\uDelta_\text{app}=\uDD{\uD}{\Delta_\text{app}}$ and $\uGamma_\text{app}=\uGG{\uG}{\Gamma_\text{app}}$ and $\uSigma = \uAS{\uA}{\Sigma}$.
\begin{byCases}
\item[\text{(\ref{rule:cvalidE-U-var})}] We have
\begin{pfsteps*}
  \item $\ce=x$ \BY{assumption}
  \item $e=x$ \BY{assumption}
  \item $\Gamma=\Gamma', \Ghyp{x}{\tau}$ \BY{assumption}
  \item $\hastypeU{\Dcons{\Delta}{\Delta_\text{app}}}{\Gamma', \Ghyp{x}{\tau}}{x}{\tau}$ \BY{Rule (\ref{rule:hastypeU-var})} \pflabel{hastypeU}
  \item $\hastypeU{\Dcons{\Delta}{\Delta_\text{app}}}{\Gcons{\Gamma', \Ghyp{x}{\tau}}{\Gamma_\text{app}}}{x}{\tau}$ \BY{Lemma \ref{lemma:weakening-U} over $\Gamma_\text{app}$ to \pfref{hastypeU}}
\end{pfsteps*}
\resetpfcounter

\item[\text{(\ref{rule:cvalidE-U-lam})}] We have
\begin{pfsteps*}
  \item $\ce=\acelam{\ctau_1}{x}{\ce'}$ \BY{assumption}
  \item $e=\aelam{\tau_1}{x}{e'}$ \BY{assumption}
  \item $\tau=\aparr{\tau_1}{\tau_2}$ \BY{assumption}
  \item $\cvalidT{\Delta}{\tsceneU{\uDelta_\text{app}}{b}}{\ctau_1}{\tau_1}$ \BY{assumption} \pflabel{cvalidT}
  \item $\cvalidE{\Delta}{\Gamma, \Ghyp{x}{\tau_1}}{\esceneU{\uDelta_\text{app}}{\uGamma_\text{app}}{\uSigma}{b}}{\ce'}{e'}{\tau_2}$ \BY{assumption} \pflabel{cvalidE}
  \item $\uetsmenv{\Delta_\text{app}}{\Sigma}$ \BY{assumption} \pflabel{uetsmenv}
  \item $\Delta \cap \Delta_\text{app}=\emptyset$ \BY{assumption} \pflabel{delta-disjoint}
  \item $\domof{\Gamma} \cap \domof{\Gamma_\text{app}}=\emptyset$ \BY{assumption} \pflabel{gamma-disjoint}
  \item $x \notin \domof{\Gamma_\text{app}}$ \BY{identification convention} \pflabel{x-fresh}
  \item $\domof{\Gamma, x : \tau_1} \cap \domof{\Gamma_\text{app}}=\emptyset$ \BY{\pfref{gamma-disjoint} and \pfref{x-fresh}} \pflabel{gamma-disjoint2}
  \item $\istypeU{\Dcons{\Delta}{\Delta_\text{app}}}{\tau_1}$ \BY{Lemma \ref{lemma:candidate-expansion-type-validation} on \pfref{cvalidT}} \pflabel{istype}
  \item $\hastypeU{\Dcons{\Delta}{\Delta_\text{app}}}{\Gcons{\Gamma, \Ghyp{x}{\tau_1}}{\Gamma_\text{app}}}{e'}{\tau_2}$ \BY{IH on \pfref{cvalidE}, \pfref{uetsmenv}, \pfref{delta-disjoint} and \pfref{gamma-disjoint2}} \pflabel{hastype1}
  \item $\hastypeU{\Dcons{\Delta}{\Delta_\text{app}}}{\Gcons{\Gamma}{\Gamma_\text{app}}, \Ghyp{x}{\tau_1}}{e'}{\tau_2}$ \BY{exchange over $\Gamma_\text{app}$ on \pfref{hastype1}} \pflabel{hastype2}
  \item $\hastypeU{\Dcons{\Delta}{\Delta_\text{app}}}{\Gcons{\Gamma}{\Gamma_\text{app}}}{\aelam{\tau_1}{x}{e'}}{\aparr{\tau_1}{\tau_2}}$ \BY{Rule (\ref{rule:hastypeU-lam}) on \pfref{istype} and \pfref{hastype2}}
\end{pfsteps*}
\resetpfcounter

\item[\text{(\ref{rule:cvalidE-U-ap})}] We have
\begin{pfsteps*}
  \item $\ce=\aceap{\ce_1}{\ce_2}$ \BY{assumption}
  \item $e=\aeap{e_1}{e_2}$ \BY{assumption}
  \item $\cvalidE{\Delta}{\Gamma}{\esceneU{\uDelta_\text{app}}{\uGamma_\text{app}}{\uSigma}{b}}{\ce_1}{e_1}{\aparr{\tau_2}{\tau}}$ \BY{assumption} \pflabel{cvalidE1}
  \item $\cvalidE{\Delta}{\Gamma}{\esceneU{\uDelta_\text{app}}{\uGamma_\text{app}}{\uSigma}{b}}{\ce_2}{e_2}{\tau_2}$ \BY{assumption} \pflabel{cvalidE2}
  \item $\uetsmenv{\Delta_\text{app}}{\Sigma}$ \BY{assumption} \pflabel{uetsmenv}
  \item $\Delta \cap \Delta_\text{app}=\emptyset$ \BY{assumption} \pflabel{delta-disjoint}
  \item $\domof{\Gamma} \cap \domof{\Gamma_\text{app}}=\emptyset$ \BY{assumption} \pflabel{gamma-disjoint}
  \item $\hastypeU{\Dcons{\Delta}{\Delta_\text{app}}}{\Gcons{\Gamma}{\Gamma_\text{app}}}{e_1}{\aparr{\tau_2}{\tau}}$ \BY{IH on \pfref{cvalidE1}, \pfref{uetsmenv}, \pfref{delta-disjoint} and \pfref{gamma-disjoint}} \pflabel{hastypeU1}
  \item $\hastypeU{\Dcons{\Delta}{\Delta_\text{app}}}{\Gcons{\Gamma}{\Gamma_\text{app}}}{e_2}{\tau_2}$ \BY{IH on \pfref{cvalidE2}, \pfref{uetsmenv}, \pfref{delta-disjoint} and \pfref{gamma-disjoint}} \pflabel{hastypeU2}
  \item $\hastypeU{\Dcons{\Delta}{\Delta_\text{app}}}{\Gcons{\Gamma}{\Gamma_\text{app}}}{\aeap{e_1}{e_2}}{\tau}$ \BY{Rule (\ref{rule:hastypeU-ap}) on \pfref{hastypeU1} and \pfref{hastypeU2}}
\end{pfsteps*}
\resetpfcounter

\item[\text{(\ref{rule:cvalidE-U-tlam})}] We have
\begin{pfsteps}
  \item \ce=\acetlam{t}{\ce'} \BY{assumption}
  \item e = \aetlam{t}{e'} \BY{assumption}
  \item \tau = \aall{t}{\tau'}\BY{assumption}
  \item \cvalidE{\Delta, \Dhyp{t}}{\Gamma}{\esceneU{\uDelta_\text{app}}{\uGamma_\text{app}}{\uSigma}{b}}{\ce'}{e'}{\tau'} \BY{assumption} \pflabel{cvalidE}
  \item \uetsmenv{\Delta_\text{app}}{\Sigma} \BY{assumption} \pflabel{uetsmenv}
  \item \Delta \cap \Delta_\text{app}=\emptyset \BY{assumption} \pflabel{delta-disjoint}
  \item \domof{\Gamma} \cap \domof{\Gamma_\text{app}}=\emptyset \BY{assumption} \pflabel{gamma-disjoint}
  \item \Dhyp{t} \notin \Delta_\text{app} \BY{identification convention}\pflabel{t-fresh}
  \item \Delta, \Dhyp{t} \cap \Delta_\text{app} = \emptyset \BY{\pfref{delta-disjoint} and \pfref{t-fresh}}\pflabel{delta-disjoint2}
  \item \hastypeU{\Dcons{\Delta, \Dhyp{t}}{\Delta_\text{app}}}{\Gcons{\Gamma}{\Gamma_\text{app}}}{e'}{\tau'} \BY{IH on \pfref{cvalidE}, \pfref{uetsmenv}, \pfref{delta-disjoint2} and \pfref{gamma-disjoint}}\pflabel{hastype1}
  \item \hastypeU{\Dcons{\Delta}{\Delta_\text{app}, \Dhyp{t}}}{\Gcons{\Gamma}{\Gamma_\text{app}}}{e'}{\tau'} \BY{exchange over $\Delta_\text{app}$ on \pfref{hastype1}}\pflabel{hastype2}
  \item \hastypeU{\Dcons{\Delta}{\Delta_\text{app}}}{\Gcons{\Gamma}{\Gamma_\text{app}}}{\aetlam{t}{e'}}{\aall{t}{\tau'}} \BY{Rule (\ref{rule:hastypeU-tlam}) on \pfref{hastype2}}
\end{pfsteps}
\resetpfcounter

\item[{\text{(\ref{rule:cvalidE-U-tap})}}~\text{through}~{\text{(\ref{rule:cvalidE-U-case})}}] These cases follow analagously, i.e. we apply the IH to all ce-expression validation judgements, Lemma \ref{lemma:candidate-expansion-type-validation} to all ce-type validation judgements, the identification convention to ensure that extended contexts remain disjoint, weakening and exchange as needed, and the corresponding typing rule.
\\

\item[\text{(\ref{rule:cvalidE-U-splicede})}] We have
\begin{pfsteps*}
  \item $\ce=\acesplicede{m}{n}$ \BY{assumption}
  \item $\parseUExp{\bsubseq{b}{m}{n}}{\ue}$ \BY{assumption}
  \item $\expandsU{\uDelta_\text{app}}{\uGamma_\text{app}}{\uSigma}{\ue}{e}{\tau}$ \BY{assumption} \pflabel{expands}
  \item $\uetsmenv{\Delta_\text{app}}{\Sigma}$ \BY{assumption} \pflabel{uetsmenv}
  \item $\Delta \cap \Delta_\text{app}=\emptyset$ \BY{assumption} \pflabel{delta-disjoint}
  \item $\domof{\Gamma} \cap \domof{\Gamma_\text{app}}=\emptyset$ \BY{assumption} \pflabel{gamma-disjoint}
  \item $\hastypeU{\Delta_\text{app}}{\Gamma_\text{app}}{e}{\tau}$ \BY{IH, part 1 on \pfref{expands} and \pfref{uetsmenv}} \pflabel{hastype}
  \item $\hastypeU{\Dcons{\Delta}{\Delta_\text{app}}}{\Gcons{\Gamma}{\Gamma_\text{app}}}{e}{\tau}$ \BY{Lemma \ref{lemma:weakening-U} over $\Delta$ and $\Gamma$ and exchange on \pfref{hastype}}
\end{pfsteps*}
\resetpfcounter
\end{byCases}

The mutual induction can be shown to be well-founded by showing that the following numeric metric on the judgements that we induct over is decreasing:
\begin{align*}
\sizeof{\expandsU{\uDelta}{\uGamma}{\uSigma}{\ue}{e}{\tau}} & = \sizeof{\ue}\\
\sizeof{\cvalidE{\Delta}{\Gamma}{\esceneU{\uDelta_\text{app}}{\uGamma_\text{app}}{\uSigma}{b}}{\ce}{e}{\tau}} & = \sizeof{b}
\end{align*}
where $\sizeof{b}$ is the length of $b$ and $\sizeof{\ue}$ is the sum of the lengths of the literal bodies in $\ue$,
\begin{align*}
\sizeof{x} & = 0\\
\sizeof{\aulam{\tau}{x}{\ue}} &= \sizeof{\ue}\\
\sizeof{\auap{\ue_1}{\ue_2}} & = \sizeof{\ue_1} + \sizeof{\ue_2}\\
\sizeof{\autlam{t}{\ue}} & = \sizeof{\ue}\\
\sizeof{\autap{\ue}{\tau}} & = \sizeof{\ue}\\
\sizeof{\aufold{t}{\tau}{\ue}} & = \sizeof{\ue}\\
\sizeof{\auunfold{\ue}} & = \sizeof{\ue}\\
%\end{align*}
%\begin{align*}
\sizeof{\autpl{\labelset}{\mapschema{\ue}{i}{\labelset}}} & = \sum_{i \in \labelset} \sizeof{\ue_i}\\
\sizeof{\aupr{\ell}{\ue}} & = \sizeof{\ue}\\
\sizeof{\auin{\labelset}{\ell}{\mapschema{\tau}{i}{\labelset}}{\ue}} & = \sizeof{\ue}\\
\sizeof{\aucase{\labelset}{\tau}{\ue}{\mapschemab{x}{\ue}{i}{\labelset}}} & = \sizeof{\ue} + \sum_{i \in \labelset} \sizeof{\ue_i}\\
\sizeof{\audefuetsm{\tau}{\ueparse}{\tsmv}{\ue}} & = \sizeof{\ue}\\
\sizeof{\autsmap{b}{\tsmv}} & = \sizeof{b}
\end{align*}

The only case in the proof of part 1 that invokes part 2 is Case (\ref{rule:expandsU-tsmap}). There, we have that the metric remains stable: \begin{align*}
 & \sizeof{\expandsU{\uDelta}{\uGamma}{\uSigma, \uShyp{\tsmv}{a}{\tau}{\eparse}}{\autsmap{b}{\tsmv}}{e}{\tau}}\\
=& \sizeof{\cvalidE{\emptyset}{\emptyset}{\esceneU{\uDelta}{\uGamma}{\uSigma, \uShyp{\tsmv}{a}{\tau}{\eparse}}{b}}{\ce}{e}{\tau}}\\
=&\sizeof{b}\end{align*}

The only case in the proof of part 2 that invokes part 1 is Case (\ref{rule:cvalidE-U-splicede}). There, we have that $\parseUExp{\bsubseq{b}{m}{n}}{\ue}$ and the IH is applied to the judgement $\expandsU{\uDelta_\text{app}}{\uGamma_\text{app}}{\uSigma}{\ue}{e}{\tau}$ where $\uDelta_\text{app}=\uDD{\uD}{\Delta_\text{app}}$ and $\uGamma_\text{app}=\uGG{\uG}{\Gamma_\text{app}}$ and $\uSigma=\uAS{\uA}{\Sigma}$. Because the metric is stable when passing from part 1 to part 2, we must have that it is strictly decreasing in the other direction:
\[\sizeof{\expandsU{\uDelta_\text{app}}{\uGamma_\text{app}}{\uSigma}{\ue}{e}{\tau}} < \sizeof{\cvalidE{\Delta}{\Gamma}{\esceneU{\uDelta_\text{app}}{\uGamma_\text{app}}{\uSigma}{b}}{\acesplicede{m}{n}}{e}{\tau}}\]
i.e. by the definitions above, 
\[\sizeof{\ue} < \sizeof{b}\]

This is established by appeal to the following two conditions. The first condition simply states that subsequences of $b$ are no longer than $b$.
\begin{condition}[Body Subsequences]\label{condition:body-subsequences} If $\bsubseq{b}{m}{n}=b'$ then $\sizeof{b'} \leq \sizeof{b}$. \end{condition}
The second condition states that an unexpanded expression constructed by parsing a textual sequence $b$ is strictly smaller, as measured by the metric defined above, than the length of $b$, because some characters must necessarily be used to invoke a TSM on and delimit each literal body.
\begin{condition}[Body Parsing]\label{condition:body-parsing} If $\parseUExp{b}{\ue}$ then $\sizeof{\ue} < \sizeof{b}$.\end{condition}

Combining Conditions \ref{condition:body-subsequences} and \ref{condition:body-parsing}, we have that $\sizeof{\ue} < \sizeof{b}$ as needed.

\end{proof}
% We need to define the following theorem about candidate expansion expression validation mutually with Theorem \ref{thm:typed-expansion-U}. 
% \begin{theorem}[Candidate Expansion Expression Validation]\label{thm:candidate-expansion-validation-U}
% If $\cvalidE{\Delta}{\Gamma}{\esceneU{\Delta_\text{app}}{\Gamma_\text{app}}{\uSigma}{b}}{\ce}{e}{\tau}$ and $\uetsmenv{\Delta_\text{app}}{\uSigma}$ then $\hastypeU{\Dcons{\Delta}{\Delta_\text{app}}}{\Gcons{\Gamma}{\Gamma_\text{app}}}{e}{\tau}$.
% \end{theorem}
% \begin{proof} By rule induction over Rules (\ref{rules:cvalidE-U}).
% \begin{byCases}
% \item[\text{(\ref{rule:cvalidE-U-var})}] We have
% \begin{pfsteps*}
%   \item $\ce=x$ \BY{assumption}
%   \item $e=x$ \BY{assumption}
%   \item $\Gamma=\Gamma', \Ghyp{x}{\tau}$ \BY{assumption}
%   \item $\hastypeU{\Dcons{\Delta}{\Delta_\text{app}}}{\Gamma', \Ghyp{x}{\tau}}{x}{\tau}$ \BY{Rule (\ref{rule:hastypeU-var})} \pflabel{hastypeU}
%   \item $\hastypeU{\Dcons{\Delta}{\Delta_\text{app}}}{\Gcons{\Gamma', \Ghyp{x}{\tau}}{\Gamma_\text{app}}}{x}{\tau}$ \BY{Lemma \ref{lemma:weakening-U} over $\Gamma_\text{app}$ to \pfref{hastypeU}}
% \end{pfsteps*}
% \resetpfcounter

% \item[\text{(\ref{rule:cvalidE-U-lam})}] We have
% \begin{pfsteps*}
%   \item $\ce=\acelam{\ctau_1}{x}{\ce'}$ \BY{assumption}
%   \item $e=\aelam{\tau_1}{x}{e'}$ \BY{assumption}
%   \item $\tau=\aparr{\tau_1}{\tau_2}$ \BY{assumption}
%   \item $\cvalidT{\Delta}{\esceneU{\Delta_\text{app}}{\Gamma_\text{app}}{\uSigma}{b}}{\ctau_1}{\tau_1}$ \BY{assumption} \pflabel{cvalidT}
%   \item $\cvalidE{\Delta}{\Gamma, \Ghyp{x}{\tau_1}}{\esceneU{\Delta_\text{app}}{\Gamma_\text{app}}{\uSigma}{b}}{\ce'}{e'}{\tau_2}$ \BY{assumption} \pflabel{cvalidE}
%   \item $\uetsmenv{\Delta_\text{app}}{\uSigma}$ \BY{assumption} \pflabel{uetsmenv}
%   \item $\istypeU{\Dcons{\Delta}{\Delta_\text{app}}}{\tau_1}$ \BY{Lemma \ref{lemma:candidate-expansion-type-validation} on \pfref{cvalidT}} \pflabel{istype}
%   \item $\hastypeU{\Dcons{\Delta}{\Delta_\text{app}}}{\Gcons{\Gamma, \Ghyp{x}{\tau_1}}{\Gamma_\text{app}}}{e'}{\tau_2}$ \BY{IH on \pfref{cvalidE} and \pfref{uetsmenv}} \pflabel{hastype1}
%   \item $\hastypeU{\Dcons{\Delta}{\Delta_\text{app}}}{\Gcons{\Gamma}{\Gamma_\text{app}}, \Ghyp{x}{\tau_1}}{e'}{\tau_2}$ \BY{exchange over $\Gamma_\text{app}$ on \pfref{hastype1}} \pflabel{hastype2}
%   \item $\hastypeU{\Dcons{\Delta}{\Delta_\text{app}}}{\Gcons{\Gamma}{\Gamma_\text{app}}}{\aelam{\tau_1}{x}{e'}}{\aparr{\tau_1}{\tau_2}}$ \BY{Rule (\ref{rule:hastypeU-lam}) on \pfref{istype} and \pfref{hastype2}}
% \end{pfsteps*}
% \resetpfcounter

% \item[\text{(\ref{rule:cvalidE-U-ap})}] We have
% \begin{pfsteps*}
%   \item $\ce=\aceap{\ce_1}{\ce_2}$ \BY{assumption}
%   \item $e=\aeap{e_1}{e_2}$ \BY{assumption}
%   \item $\cvalidE{\Delta}{\Gamma}{\esceneU{\Delta_\text{app}}{\Gamma_\text{app}}{\uSigma}{b}}{\ce_1}{e_1}{\aparr{\tau_1}{\tau}}$ \BY{assumption} \pflabel{cvalidE1}
%   \item $\cvalidE{\Delta}{\Gamma}{\esceneU{\Delta_\text{app}}{\Gamma_\text{app}}{\uSigma}{b}}{\ce_2}{e_2}{\tau_1}$ \BY{assumption} \pflabel{cvalidE2}
%   \item $\uetsmenv{\Delta_\text{app}}{\uSigma}$ \BY{assumption} \pflabel{uetsmenv}
%   \item $\hastypeU{\Dcons{\Delta}{\Delta_\text{app}}}{\Gcons{\Gamma}{\Gamma_\text{app}}}{e_1}{\aparr{\tau_1}{\tau}}$ \BY{IH on \pfref{cvalidE1} and \pfref{uetsmenv}} \pflabel{hastypeU1}
%   \item $\hastypeU{\Dcons{\Delta}{\Delta_\text{app}}}{\Gcons{\Gamma}{\Gamma_\text{app}}}{e_2}{\tau_1}$ \BY{IH on \pfref{cvalidE2} and \pfref{uetsmenv}} \pflabel{hastypeU2}
%   \item $\hastypeU{\Dcons{\Delta}{\Delta_\text{app}}}{\Gcons{\Gamma}{\Gamma_\text{app}}}{\aeap{e_1}{e_2}}{\tau}$ \BY{Rule (\ref{rule:hastypeU-ap}) on \pfref{hastypeU1} and \pfref{hastypeU2}}
% \end{pfsteps*}
% \resetpfcounter

% \item[\VExpof{\text{\ref{rule:hastypeU-tlam}}}~\text{through}~\VExpof{\text{\ref{rule:hastypeU-case}}}] These cases follow analagously, i.e. we apply the IH to all candidate expansion expression validation premises, Lemma \ref{lemma:candidate-expansion-type-validation} to all candidate expansion type validation premises, weakening and exchange as needed, and then apply the corresponding typing rule.
% \\

% \item[\text{(\ref{rule:cvalidE-U-splicede})}] We have
% \begin{pfsteps*}
%   \item $\ce=\acesplicede{m}{n}$ \BY{assumption}
%   \item $\parseUExp{\bsubseq{b}{m}{n}}{\ue}$ \BY{assumption}
%   \item $\expandsU{\Delta_\text{app}}{\Gamma_\text{app}}{\uSigma}{\ue}{e}{\tau}$ \BY{assumption} \pflabel{expands}
%   \item $\uetsmenv{\Delta_\text{app}}{\uSigma}$ \BY{assumption} \pflabel{uetsmenv}
%   \item $\hastypeU{\Delta_\text{app}}{\Gamma_\text{app}}{e}{\tau}$ \BY{Theorem \ref{thm:typed-expansion-U} on \pfref{expands} and \pfref{uetsmenv}} \pflabel{hastype}
%   \item $\hastypeU{\Dcons{\Delta}{\Delta_\text{app}}}{\Gcons{\Gamma}{\Gamma_\text{app}}}{e}{\tau}$ \BY{Lemma \ref{lemma:weakening-U} on \pfref{hastype}}
% \end{pfsteps*}
% \resetpfcounter
% \end{byCases}
% \end{proof}


%\qed

\chapter{Unparameterized Pattern TSMs}\label{sec:pattern-tsms}
In Chapter \ref{chap:tsms}, we considered situations where the programmer needed to \emph{construct} (a.k.a. \emph{introduce}) a value. In this chapter, we consider situations where the programmer needs to \emph{deconstruct} (a.k.a. \emph{eliminate}) a value. In full-scale functional languages like ML and Haskell, values are deconstructed by \emph{pattern matching} over their structure. For example, recall the recursive labeled sum type \lstinline{Rx} defined in Figure \ref{fig:datatype-rx}. We can pattern match over a value, \lstinline{r}, of type \lstinline{Rx} using VerseML's \lstinline{match} construct:
\begin{lstlisting}
fun read_example_rx(r : Rx) : (string * Rx) option => 
  match r with 
    Seq(Str(name), Seq(Str "SSTR: ESTR", ssn)) => Some (name, ssn)
  | _ => None
\end{lstlisting}

Match expressions consist of a \emph{scrutinee}, here \li{r}, and a sequence of \emph{rules} separated by vertical bars, \li{|}, in the concrete syntax. Each rule consists of a \emph{pattern} and an {expression} called the corresponding \emph{branch}, separated by a double arrow, \li{=>}, in the concrete syntax. When the {match} expression is evaluated, the value of the scrutinee is matched against each pattern sequentially. If the value matches, evaluation takes the corresponding branch. Variables in patterns match any value of the appropriate type. In the corresponding branch, the variable stands for that value. Variables can each appear only once in a pattern.  
For example, on Line 3, the pattern \li{Seq(Str(name), Seq(Str "SSTR: ESTR", ssn))} matches values of the form \li{Seq(Str(#$e_1$#), Seq(Str "SSTR: ESTR", #$e_2$#))}, where $e_1$ is a value of type \li{string} and $e_2$ is a value of type \li{Rx}. The variables \li{name} and \li{ssn} stand for the values of $e_1$ and $e_2$, respectively, in \li{Some (name, ssn)}. On Line 4, the pattern \li{_} is the \emph{wildcard pattern} -- it matches any value of the appropriate type and binds no variables.

The behavior of the \li{match} construct when no pattern in the rule sequence matches a value is to raise an exception indicating \emph{match failure}. It is possible to statically determine whether match failure is possible (i.e. whether there exist values of the scrutinee that are not matched by any pattern in the rule sequence). In the example above, our use of the wildcard pattern ensures that match failure cannot occur. A rule sequence that cannot lead to match failure is said to be \emph{exhaustive}. Most compilers warn the programmer when a rule sequence is non-exhaustive.

It is also possible to statically decide when a rule is \emph{redundant} relative to the preceding rules, i.e. when there does not exist a value matched by that rule but not matched by any of the preceding rules. For example, if we add  another rule at the end of the match expression above, it will be redundant because all values match the wildcard pattern. Again, most compilers warn the programmer when a rule is redundant.

Nested pattern matching generalizes the projection and case analysis operators (i.e. the \emph{eliminators}) for products and sums (cf. $\miniVerseUE$ from the previous section) and decreases syntactic cost in situations where eliminators would need to be nested. There remains room for improvement, however, because complex patterns sometimes    individually have high syntactic cost. In Sec. \ref{sec:syntax-examples-regexps}, we considered a hypothetical dialect of ML called ML+Rx that built in derived syntax both for constructing and pattern matching over values of the recursive labeled sum type \li{Rx}. In ML+Rx, we can express the example above at lower syntactic cost as follows:

\begin{lstlisting}
fun read_example_rx(r : Rx) : (string * Rx) option => 
  match r with 
    /SURL@EURLnameSURL: %EURLssn/ => Some (name, ssn)
  | _ => None\end{lstlisting}
\noindent
Dialect formation is not a modular approach, for the reasons discussed in Chapter \ref{chap:intro}, so we seek language constructs that allow us to decrease the syntactic cost of expressing complex patterns to a similar degree.

Expression TSMs -- introduced in Chapter \ref{chap:tsms} -- can decrease the syntactic cost of constructing a value of a specified type. However, expressions are syntactically distinct from patterns, so we cannot simply apply an expression TSM to generate a pattern.\footnote{The fact that certain concrete expression and pattern forms overlap is immaterial to this fundamental distinction. There are many expression forms that the expansion generated by an expression TSM might use that have no corresponding pattern form, e.g.  lambda abstraction.} %For example, the expansion generated by an expression TSM might define or apply a function, but patterns do not contain functions or function applications. 
For this reason, we need to introduce a new (albeit closely related) construct -- the \textbf{pattern TSM}. In this chapter, we consider only \textbf{unparameterized pattern TSMs} (upTSMs), i.e. pattern TSMs that generate patterns that match values of a single specified type, like \li{Rx}. In Chapter \ref{sec:tsms-parameterized}, we will consider both expression and pattern TSMs that specify type and module parameters (peTSMs and ppTSMs). 

\section{Pattern TSMs By Example}\label{sec:ptsms-by-example}
The organization of the remainder of this chapter mirrors that of Chapter \ref{chap:tsms}. We begin in this section with a ``tutorial-style'' introduction to upTSMs in VerseML. In particular, we  discuss an upTSM for patterns matching values of type \li{Rx}. In the next section, we specify a reduced formal system based on $\miniVerseUE$ called $\miniVersePat$ that makes the intuitions developed here mathematically precise.

\subsection{Usage}\label{sec:ptsms-usage}
The VerseML function \li{read_example_rx} defined at the beginning of this chapter can be concretely expressed at lower syntactic cost by applying a upTSM, \li{#\dolla#rx}, as follows:
\begin{lstlisting}
fun read_example_rx(r : Rx) : (string * Rx) option => 
  match r with 
    $rx /SURL@EURLnameSURL: %EURLssn/ => Some (name, ssn)
  | _ => None
\end{lstlisting}
Like expression TSMs, pattern TSMs are applied to \emph{generalized literal forms} (see Figure \ref{fig:literal-forms}). Generalized literal forms are left unparsed when patterns are first parsed. During the subsequent \emph{typed expansion} process, the pattern TSM parses the body of the literal form to generate a \emph{candidate expansion}. The language validates the candidate expansion according to criteria that we will establish in Sec. \ref{sec:ptsms-validation}. If validation succeeds, the language generates the final expansion (or more concisely, simply the expansion) of the pattern. The expansion of the unexpanded pattern \li{#\dolla#rx /SURL@EURLnameSURL: %EURLssn/} from the example above is the following pattern:
\begin{lstlisting}[numbers=none]
Seq(Str(name), Seq(Str "SSTR: ESTR", ssn))
\end{lstlisting}

The checks for exhaustiveness and redundancy can be performed post-expansion in the usual way, so we do not need to consider them further here. 
\subsection{Definition}\label{sec:ptsms-definition}
The definition of the pattern TSM \li{#\dolla#rx} shown being applied in the example above has the following form:
\begin{lstlisting}[numbers=none]
syntax $rx at Rx for patterns {
  static fn(body : Body) : CEPat ParseResult =>
    (* regex pattern parser here *)
}
\end{lstlisting}
This definition first names the pattern TSM. Pattern TSM names, like expression TSM names, must begin with the dollar symbol (\li{#\dolla#}) to distinguish them from labels. Pattern TSM names and expression TSM names are tracked separately, i.e. an expression TSM and a pattern TSM can have the same name without conflict (as is the case here -- the expression TSM described in Sec. \ref{sec:uetsms-definition} is also named \li{#\dolla#rx}). The \emph{sort qualifier} \li{for patterns} indicates that this is a pattern TSM definition, rather than an expression TSM definition (the sort qualifier \li{for expressions} can be written for expression TSMs, though when the sort qualifier is omitted this is the default). Because defining both an expression TSM and a pattern TSM with the same name at the same type is a common idiom, VerseML provides a primitive derived form for combining their definitions:
\begin{lstlisting}[numbers=none]
syntax $rx at Rx for expressions {
  static fn(body : Body) : CEExp ParseResult => 
    (* regex expression parser here *)
} for patterns {
  static fn(body : Body) : CEPat ParseResult => 
    (* regex pattern parser here *)
}
\end{lstlisting}

Pattern TSMs, like expression TSMs, must specify a static \emph{parse function}, delimited by curly braces in the concrete syntax. For a pattern TSM, the parse function must be of type \li{Body -> CEPat ParseResult}. The input type, \li{Body}, gives the parse function access to the body of the provided literal form, and is defined as in Sec. \ref{sec:uetsms-definition} as a synonym for the type \li{string}. The output type, \li{CEPat ParseResult}, is the parameterized type constructor \li{ParseResult}, defined in Figure \ref{fig:indexrange-and-parseresult}, applied to the type \li{CEPat} defined in Figure \ref{fig:CEPat}.  So if parsing succeeds, the pattern TSM returns a value of the form \li{Success #$\ecand$#} where $\ecand$ is a value of type \li{CEPat} that we call the \emph{encoding of the candidate expansion}. If parsing fails, then the pattern TSM returns a value constructed by \li{ParseError} and equipped with an error message and error location. 

The type \li{CEPat} is analagous to the types \li{CEExp} and \li{CETyp} defined in Figure \ref{fig:candidate-exp-verseml}. It encodes the abstract syntax of VerseML patterns (in Figure \ref{fig:CEPat}, some constructors are elided for concision), with the exception of variable patterns (for reasons explained in Sec. \ref{sec:ptsms-hygiene} below), and includes an additional constructor, \li{Spliced}, for referring to spliced subpatterns by their position within the parse stream, discussed next.

\begin{figure}
\begin{lstlisting}[numbers=none]
type CEPat = Wild
           | (* ... *)
           | Spliced of IndexRange
\end{lstlisting}
\caption[Abbreviated definition of \li{CEPat} in VerseML]{Abbreviated definition of \li{CEPat} in the VerseML prelude.}
\label{fig:CEPat}
\end{figure}

\subsection{Splicing}\label{sec:ptsms-splicing}
Patterns that appear directly within the literal body of an unexpanded pattern are called \emph{spliced subpatterns}. For example, the patterns \li{name} and \li{ssn} appear within the unexpanded pattern \li{#\dolla#rx /SURL@EURLnameSURL: %EURLssn/}. 
When the parse function determines that a subsequence of the literal body should be treated as a spliced subpatern (here, by recognizing the characters \li{@} or \li{%} followed by a variable or parenthesized pattern), 
it can refer to it within the candidate expansion that it construct a reference to it for use within the candidate expansion it generates using the \li{Spliced} constructor of the \li{CEPat} type shown in Figure \ref{fig:CEPat}. The \li{Spliced} constructor requires a value of type \li{IndexRange} because spliced subpatterns are referred to indirectly by their position within the literal body. This prevents pattern TSMs from ``forging'' a spliced subpattern (i.e. claiming that some pattern is a spliced subpattern, even though it does not appear in the literal body).

The candidate expansion generated by the pattern TSM \li{#\dolla#rx} for the example above, if written in a hypothetical concrete syntax where references to spliced subpatterns are written \li{spliced<startIdx, endIdx>}, is:
\begin{lstlisting}[numbers=none]
Seq(Str(spliced<1, 4>), Seq(Str "SSTR: ESTR", spliced<8, 10>))
\end{lstlisting}
Here, \li{spliced<1, 4>} refers to the subpattern \li{name} by position, and \li{spliced<8, 10>} refers to the subpattern \li{ssn} by position.

\subsection{Typing}\label{sec:ptsms-validation}
The language validates candidate expansion before a final expansion is generated. One aspect of candidate expansion validation is checking the candidate expansion against the type annotation specified by the pattern TSM, e.g. the type \li{Rx} in the example above.

\subsection{Hygiene}\label{sec:ptsms-hygiene}
In order to check that the candidate expansion is well-typed, the language must parse, type and expand the spliced subpatterns that the candidate expansion refers to (by their position within the literal body, cf. above). To maintain a useful binding discipline, i.e. to allow programmers to reason about variable binding without examining expansions directly, the validation process allows variables (e.g. \lstinline{name} and \lstinline{ssn} above) to occur only in spliced subpatterns (just as variables bound at the use site can only appear in spliced subexpressions when using TSMs). Indeed, there is no constructor for the type \li{CEPat} corresponding to a variable pattern. This protection against ``hidden bindings'' is beneficial because it leaves variable naming entirely up to the client of the pattern TSM. A pattern TSM cannot inadvertently shadow a binding at the application site.

\subsection{Final Expansion}\label{sec:ptsms-final-expansion}
If validation succeeds, the semantics generates the \emph{final expansion} of the pattern from the candidate expansion by replacing the references to spliced subpatterns with their final expansions. For example, the final expansion of \li{#\dolla#rx /SURL@EURLnameSURL: %EURLssn/} is:
\begin{lstlisting}[numbers=none]
Seq(Str(name), Seq(Str "SSTR: ESTR", ssn))
\end{lstlisting}

\section{\texorpdfstring{$\miniVersePat$}{miniVerseU}}\label{sec:miniVerseUP}
To make the intuitions developed in the previous section about pattern TSMs precise, we  now introduce $\miniVersePat$, a small language with support for both unparameterized expression TSMs and unparameterized pattern TSMs.
\subsection{Syntax of the Inner Core}\label{sec:UP-expanded-terms}
The \emph{inner core} of $\miniVersePat$ consists of \emph{types}, $\tau$, \emph{expanded expressions}, $e$, \emph{expanded rules}, $r$, and \emph{expanded patterns}, $p$. Their syntax is specified by the syntax chart in Figure \ref{fig:UP-expanded-terms}. The inner core of $\miniVersePat$ differs from that of $\miniVerseUE$  only in that the case analysis operator has been replaced by the pattern matching operator\footnote{We do not also remove the projection operator because it has lower syntactic cost than pattern matching when only a single field from a labeled tuple is needed.}, so we will gloss some definitions that are identical to those in Sec. \ref{sec:miniVerseU}. The new constructs are highlighted in gray. Our formulation of the semantics of pattern matching is adapted from Harper's formulation in \emph{Practical Foundations for Programming Languages, First Edition} \cite{pfple1}.\footnote{The chapter on pattern matching has, of this writing, been removed from the draft second edition of \emph{PFPL}, but a copy of the first edition can be found online.}

\begin{figure}
$\begin{array}{lllllll}
\textbf{Sort} & & & \textbf{Operational Form} & \textbf{Stylized Form} & \textbf{Description}\\
\mathsf{Typ} & \tau & ::= & t & t & \text{variable}\\
&&& \aparr{\tau}{\tau} & \parr{\tau}{\tau} & \text{partial function}\\
&&& \aall{t}{\tau} & \forallt{t}{\tau} & \text{polymorphic}\\
&&& \arec{t}{\tau} & \rect{t}{\tau} & \text{recursive}\\
&&& \aprod{\labelset}{\mapschema{\tau}{i}{\labelset}} & \prodt{\mapschema{\tau}{i}{\labelset}} & \text{labeled product}\\
&&& \asum{\labelset}{\mapschema{\tau}{i}{\labelset}} & \sumt{\mapschema{\tau}{i}{\labelset}} & \text{labeled sum}\\
\mathsf{Exp} & e & ::= & x & x & \text{variable}\\
&&& \aelam{\tau}{x}{e} & \lam{x}{\tau}{e} & \text{abstraction}\\
&&& \aeap{e}{e} & \ap{e}{e} & \text{application}\\
&&& \aetlam{t}{e} & \Lam{t}{e} & \text{type abstraction}\\
&&& \aetap{e}{\tau} & \App{e}{\tau} & \text{type application}\\
&&& \aefold{t}{\tau}{e} & \fold{e} & \text{fold}\\
&&& \aeunfold{e} & \unfold{e} & \text{unfold}\\
&&& \aetpl{\labelset}{\mapschema{e}{i}{\labelset}} & \tpl{\mapschema{e}{i}{\labelset}} & \text{labeled tuple}\\
&&& \aepr{\ell}{e} & \prj{e}{\ell} & \text{projection}\\
&&& \aein{\labelset}{\ell}{\mapschema{\tau}{i}{\labelset}}{e} & \inj{\ell}{e} & \text{injection}\\
\LCC \lightgray & \lightgray & \lightgray & \lightgray & \lightgray & \lightgray \\
&&& \aematchwith{n}{\tau}{e}{\seqschemaX{r}} & \matchwith{e}{\seqschemaX{r}} & \text{match}\\
\mathsf{ERule} & r & ::= & \aematchrule{n}{p}{\seqschemaX{x}}{e} & \matchrule{p}{e} & \text{rule}\\
\mathsf{EPat} & p & ::= & x & x & \text{variable pattern}\\
&&& \aewildp & \wildp & \text{wildcard pattern}\\
%&&& \aefoldp{p} & \foldp{p} & \text{fold pattern}\\
&&& \aetplp{\labelset}{\mapschema{p}{i}{\labelset}} & \tplp{\mapschema{p}{i}{\labelset}} & \text{labeled tuple pattern}\\
&&& \aeinjp{\ell}{p} & \injp{\ell}{p} & \text{injection pattern}\ECC
\end{array}$
\caption[Syntax of types and expanded expressions, rules and patterns in $\miniVersePat$]{Syntax of types and expanded expressions, rules and patterns (collectively, expanded terms) in $\miniVersePat$. We adopt the metatheoretic conventions established for our specification of $\miniVerseUE$ in Sec. \ref{sec:miniVerseU} without restating them, unless otherwise specified. We write $\seqschemaX{r}$ for sequences of $n \geq 0$ rule arguments and $\seqschemaX{x}.e$ for expressions binding the sequence of $n \geq 0$ variables $\seqschemaX{x}$. Variable patterns are not pattern variables, i.e. they do not stand for terms, but rather serve as references to the bindings in the rule that the pattern appears within. The semantics below will clarify this. Types and expanded terms are identified up to $\alpha$-equivalence.}
\label{fig:UP-expanded-terms}
\end{figure}


\subsection{Statics of the Inner Core}
The \emph{statics of the inner core} is specified by judgements of the following form:
\[\begin{array}{ll}
\textbf{Judgement Form} & \textbf{Description}\\
\istypeU{\Delta}{\tau} & \text{$\tau$ is a well-formed type assuming $\Delta$}\\
%\isctxU{\Delta}{\Gamma} & \text{$\Gamma$ is a well-formed typing context assuming $\Delta$}\\
\hastypeU{\Delta}{\Gamma}{e}{\tau} & \text{$e$ has type $\tau$ assuming $\Delta$ and $\Gamma$}\\
\ruleType{\Delta}{\Gamma}{r}{\tau}{\tau'} & \text{$r$ takes values of type $\tau$ to values of type $\tau'$ assuming $\Delta$ and $\Gamma$}\\
\patType{\pctx}{p}{\tau} & \text{$p$ matches values of type $\tau$ and generates hypotheses $\pctx$} 
\end{array}\]

The types of $\miniVersePat$ are exactly those of $\miniVerseUE$, described in Sec. \ref{sec:miniVerseU}, so the \emph{type formation judgement}, $\istypeU{\Delta}{\tau}$, is inductively defined by Rules (\ref{rules:istypeU}). 

The \emph{typing judgement}, $\hastypeU{\Delta}{\Gamma}{e}{\tau}$, assigns types to expressions and is inductively defined by Rules (\ref*{rules:hastypeUP}), which consist of:
\begin{subequations}\label{rules:hastypeUP}
\refstepcounter{equation}%
\begin{itemize}
\label{rule:hastypeUP-var}
\refstepcounter{equation}\label{rule:hastypeUP-lam}
\refstepcounter{equation}\label{rule:hastypeUP-ap}
\refstepcounter{equation}\label{rule:hastypeUP-tlam}
\refstepcounter{equation}\label{rule:hastypeUP-tap}
\refstepcounter{equation}\label{rule:hastypeUP-fold}
\refstepcounter{equation}\label{rule:hastypeUP-unfold}
\refstepcounter{equation}\label{rule:hastypeUP-tpl}
\refstepcounter{equation}\label{rule:hastypeUP-pr}
\refstepcounter{equation}\label{rule:hastypeUP-in}
\item Rules defined identically to Rules (\ref{rule:hastypeU-var}) through (\ref{rule:hastypeU-in}). We will refer to these rules as Rules (\ref*{rule:hastypeUP-var}) through (\ref*{rule:hastypeUP-in}). %Note that we cannot defer directly to the typing rules from Sec. \ref{sec:miniVerseU} because $e$ has been redefined here.
\item The following rule for match expressions: 
\end{itemize}
\begin{equation}\label{rule:hastypeUP-match}
\inferrule{
  \hastypeU{\Delta}{\Gamma}{e}{\tau}\\
  \istypeU{\Delta}{\tau'}\\
  \{\ruleType{\Delta}{\Gamma}{r_i}{\tau}{\tau'}\}_{1 \leq i \leq n}\\
}{\hastypeU{\Delta}{\Gamma}{\aematchwith{n}{\tau'}{e}{\seqschemaX{r}}}{\tau'}}
\end{equation}  
\end{subequations}
The first premise of Rule (\ref{rule:hastypeUP-match}) assigns a type, $\tau$, to the scrutinee, $e$. The second premise checks that the type of the expression as a whole, $\tau'$, is well-formed.\footnote{The second premise of Rule (\ref{rule:hastypeUP-match}), and the type argument in the match form, are necessary to maintain regularity, defined below, but only because when $n=0$, the type $\tau'$ is arbitrary. In all other cases, $\tau'$ can be determined by assigning types to the  branch expressions.} The third premise then ensures that each rule $r_i$, for $1 \leq i \leq n$, takes values of type $\tau$ to values of the type of the match expression as a whole, $\tau'$. This is expressed by the \emph{rule typing judgement}, $\ruleType{\Delta}{\Gamma}{r}{\tau}{\tau'}$, which is defined mutually with Rules (\ref{rules:hastypeUP}) by the following rule:
\begin{equation}\label{rule:ruleType}
\inferrule{
  \patType{\pctx}{p}{\tau}\\
  \domof{\pctx} = \seqschemaX{x}\\
  \hastypeU{\Delta}{\Gcons{\Gamma}{\pctx}}{e}{\tau'}
}{\ruleType{\Delta}{\Gamma}{\aematchrule{n}{p}{\seqschemaX{x}}{e}}{\tau}{\tau'}}
\end{equation}
The premises of Rule (\ref{rule:ruleType}) can be understood as follows, in order:
\begin{enumerate}
\item The first premise invokes the \emph{pattern typing judgement}, $\patType{\pctx}{p}{\tau}$, to check that the pattern, $p$, matches values of type $\tau$, and to gather the typing hypotheses that the pattern generates in a \emph{pattern typing context}, $\Omega$. Pattern typing contexts, like typing contexts, $\Gamma$, are finite functions from variables to  hypotheses of the form $x : \tau$. Algorithmically, however, one should consider the pattern typing context the ``output'' of the pattern typing judgement. %We use the letter $\pctx$ rather than $\Gamma$ only to emphasize that 

The pattern typing judgement is inductively defined by the following rules:
\begin{subequations}\label{rules:patType}
\begin{equation}\label{rule:patType-var}
\inferrule{ }{\patType{\Ghyp{x}{\tau}}{x}{\tau}}
\end{equation}
\begin{equation}\label{rule:patType-wild}
\inferrule{ }{\patType{\emptyset}{\aewildp}{\tau}}
\end{equation}
\begin{equation}\label{rule:patType-tpl}
\inferrule{
  \{\patType{\pctx_i}{p_i}{\tau_i}\}_{i \in \labelset}\\
  \{\{\domof{\pctx_i} \cap \domof{\pctx_j} = \emptyset\}_{j \in \labelset \setminus i}\}_{i \in \labelset}
}{
  \patType{\Gconsi{i \in \labelset}{\pctx_i}}{\aetplp{\labelset}{\mapschema{p}{i}{\labelset}}}{\aprod{\labelset}{\mapschema{\tau}{i}{\labelset}}}
}
\end{equation}
\begin{equation}\label{rule:patType-inj}
\inferrule{
  \patType{\pctx}{p}{\tau}
}{
  \patType{\pctx}{\aeinjp{\ell}{p}}{\asum{\labelset, \ell}{\mapschema{\tau}{i}{\labelset}; \mapitem{\ell}{\tau}}}
}
\end{equation}
\end{subequations}

Rule (\ref{rule:patType-var}) specifies that a variable pattern, $x$, can match values of any type, $\tau$, and generates the hypothesis that $x$ has the type $\tau$. 

Rule (\ref{rule:patType-wild}) specifies that a wildcard pattern can also match values of any type, $\tau$, but wildcard patterns generate no hypotheses. 

Labeled tuple patterns, $\aetplp{\labelset}{\mapschema{p}{i}{\labelset}}$, specify a subpattern $p_i$ for each label $i \in \labelset$. Rule (\ref{rule:patType-tpl}) specifies that a labeled tuple pattern of this form matches values of the labeled product type $\aprod{\labelset}{\mapschema{\tau}{i}{\labelset}}$. The first premise checks each subpattern $p_i$ against the corresponding type $\tau_i$, generating hypotheses $\pctx_i$. The second premise ensures that no variables are multiply bound by checking that the domains of the generated pattern typing contexts $\pctx_i$ are mutually disjoint. The hypotheses generated in the conclusion of the rule are the union of the hypotheses generated by the subpatterns. 

Injection patterns, $\aeinjp{\ell}{p}$, match values of labeled sum types of the form $\asum{\labelset, \ell}{\mapschema{\tau}{i}{\labelset}; \mapitem{\ell}{\tau}}$, i.e. labeled sum types that define a case for the label $\ell$. Rule (\ref{rule:patType-inj}) checks the subpattern $p$ against the corresponding type $\tau$, and passes through the assumptions that $p$ generates.

\item The second premise of Rule (\ref{rule:ruleType}) ensures that pattern typing of $p$ has generated hypotheses for all of the variables that the branch expression, $e$, binds. This is merely a matter of ``metatheoretic bookkeeping''. In the stylized form for rules, $\matchrule{p}{e}$, the variables bound in $e$ are, implicitly, exactly those mentioned in p.% The bindings for $e$ would be extracted from the pattern implicitly. 
\item The final premise of Rule (\ref{rule:ruleType}) extends the typing context, $\Gamma$, with the hypotheses generated by pattern typing, $\pctx$, and checks the branch expression, $e$, against the branch type, $\tau'$.
\end{enumerate}

The rules above are syntax-directed, so we assume an inversion lemma for each rule as needed without stating it separately or proving it explicitly. The following standard lemmas also hold.

The Weakening Lemma establishes that extending the context with unnecessary hypotheses preserves well-formedness and typing.
\begin{lemma}[Weakening]\label{lemma:weakening-UP} All of the following hold: 
\begin{enumerate} 
\item If $\istypeU{\Delta}{\tau}$ then $\istypeU{\Delta, \Dhyp{t}}{\tau}$.
%\item If $\isctxU{\Delta}{\Gamma}$ then $\isctxU{\Delta, \Dhyp{t}}{\Gamma}$.
\item \begin{enumerate}
  \item If $\hastypeU{\Delta}{\Gamma}{e}{\tau}$ then $\hastypeU{\Delta, \Dhyp{t}}{\Gamma}{e}{\tau}$.
  \item If $\ruleType{\Delta}{\Gamma}{r}{\tau}{\tau'}$ then $\ruleType{\Delta, \Dhyp{t}}{\Gamma}{r}{\tau}{\tau'}$.
  \end{enumerate}
\item \begin{enumerate}
  \item If $\hastypeU{\Delta}{\Gamma}{e}{\tau}$ and $\istypeU{\Delta}{\tau''}$ then $\hastypeU{\Delta}{\Gamma, \Ghyp{x}{\tau''}}{e}{\tau}$.
  \item If $\ruleType{\Delta}{\Gamma}{r}{\tau}{\tau'}$ and $\istypeU{\Delta}{\tau''}$ then $\ruleType{\Delta}{\Gamma, \Ghyp{x}{\tau''}}{r}{\tau}{\tau'}$.
  \end{enumerate}
\end{enumerate}
\end{lemma}
\begin{proof-sketch}
\begin{enumerate}
\item By rule induction over Rules (\ref{rules:istypeU}).
%\item By rule induction over Rules (\ref{rules:isctxU}).
\item By mutual rule induction over Rules (\ref{rules:hastypeUP}) and Rule (\ref{rule:ruleType}).
\item By mutual rule induction over Rules (\ref{rules:hastypeUP}) and Rule (\ref{rule:ruleType}).
\end{enumerate}
\end{proof-sketch}

The {pattern typing judgement} is a \emph{linear hypothetical judgement}, i.e. it does \emph{not} obey weakening of the pattern typing context. This is to ensure that the pattern typing context captures exactly those hypotheses generated by a pattern, and no others.

We assume that renaming of bound variables, $\alpha$-equivalence and substitution are defined as in \emph{PFPL} \cite{pfpl}, with the additional stipulation that the variables that are bound by the branch expression in an expanded rule are renamed together with those in the corresponding expanded pattern. The Substitution Lemma establishes that substitution of a well-formed type for a type variable, or an expanded expression of the appropriate type for an expanded expression variable, preserves well-formedness and typing.
\begin{lemma}[Substitution]\label{lemma:substitution-UP} All of the following hold:
\begin{enumerate}
\item If $\istypeU{\Delta, \Dhyp{t}}{\tau}$ and $\istypeU{\Delta}{\tau'}$ then $\istypeU{\Delta}{[\tau'/t]\tau}$.
%\item If $\isctxU{\Delta, \Dhyp{t}}{\Gamma}$ and $\istypeU{\Delta}{\tau'}$ then $\isctxU{\Delta}{[\tau'/t]\Gamma}$.
\item \begin{enumerate}
  \item If $\hastypeU{\Delta, \Dhyp{t}}{\Gamma}{e}{\tau}$ and $\istypeU{\Delta}{\tau'}$ then $\hastypeU{\Delta}{[\tau'/t]\Gamma}{[\tau'/t]e}{[\tau'/t]\tau}$.
  \item If $\ruleType{\Delta, \Dhyp{t}}{\Gamma}{r}{\tau}{\tau''}$ and $\istypeU{\Delta}{\tau'}$ then $\ruleType{\Delta}{[\tau'/t]\Gamma}{[\tau'/t]r}{[\tau'/t]\tau}{[\tau'/t]\tau''}$.
  \end{enumerate}
\item \begin{enumerate}
  \item If $\hastypeU{\Delta}{\Gamma, \Ghyp{x}{\tau'}}{e}{\tau}$ and $\hastypeU{\Delta}{\Gamma}{e'}{\tau'}$ then $\hastypeU{\Delta}{\Gamma}{[e'/x]e}{\tau}$.
  \item If $\ruleType{\Delta}{\Gamma, \Ghyp{x}{\tau'}}{r}{\tau}{\tau''}$ and $\hastypeU{\Delta}{\Gamma}{e'}{\tau''}$ then $\ruleType{\Delta}{\Gamma}{[e'/x]r}{\tau}{\tau''}$.
  \end{enumerate}
\end{enumerate}\end{lemma}
\begin{proof-sketch}
\begin{enumerate}
\item By rule induction over Rules (\ref{rules:istypeU}).
\item By mutual rule induction over Rules (\ref{rules:hastypeUP}) and Rule (\ref{rule:ruleType}).
\item By mutual rule induction over Rules (\ref{rules:hastypeUP}) and Rule (\ref{rule:ruleType}).
\end{enumerate}
\end{proof-sketch}

The Decomposition Lemma is the converse of the Substitution Lemma.
\begin{lemma}[Decomposition]\label{lemma:decomposition-UP} All of the following hold:
\begin{enumerate}
\item If $\istypeU{\Delta}{[\tau'/t]\tau}$ and $\istypeU{\Delta}{\tau'}$ then $\istypeU{\Delta, \Dhyp{t}}{\tau}$.
%\item If $\isctxU{\Delta}{[\tau'/t]\Gamma}$ and $\istypeU{\Delta}{\tau'}$ then $\isctxU{\Delta, \Dhyp{t}}{\Gamma}$.
\item \begin{enumerate}
  \item If $\hastypeU{\Delta}{[\tau'/t]\Gamma}{[\tau'/t]e}{[\tau'/t]\tau}$ and $\istypeU{\Delta}{\tau'}$ then $\hastypeU{\Delta, \Dhyp{t}}{\Gamma}{e}{\tau}$.
  \item If $\ruleType{\Delta}{[\tau'/t]\Gamma}{[\tau'/t]r}{[\tau'/t]\tau}{[\tau'/t]\tau''}$ and $\istypeU{\Delta}{\tau'}$ then $\ruleType{\Delta, \Dhyp{t}}{\Gamma}{r}{\tau}{\tau''}$.
  \end{enumerate}
\item \begin{enumerate}
  \item If $\hastypeU{\Delta}{\Gamma}{[e'/x]e}{\tau}$ and $\hastypeU{\Delta}{\Gamma}{e'}{\tau'}$ then $\hastypeU{\Delta}{\Gamma, \Ghyp{x}{\tau'}}{e}{\tau}$.
  \item If $\ruleType{\Delta}{\Gamma}{[e'/x]r}{\tau}{\tau''}$ and $\hastypeU{\Delta}{\Gamma}{e'}{\tau'}$ then $\ruleType{\Delta}{\Gamma, \Ghyp{x}{\tau'}}{r}{\tau}{\tau''}$.
  \end{enumerate}
\end{enumerate}\end{lemma}
\begin{proof-sketch}
\begin{enumerate}
\item By rule induction over Rules (\ref{rules:istypeU}) and case analysis over the definition of substitution. In all cases, the derivation of $\istypeU{\Delta}{[\tau'/t]\tau}$ does not depend on the form of $\tau'$.
%\item Context formation of $[\tau'/t]\Gamma$ does not depend on the structure of $\tau'$.
\item By mutual rule induction over Rules (\ref{rules:hastypeUP}) and Rule (\ref{rule:ruleType}) and case analysis over the definition of substitution. In all cases, the derivation of $\hastypeU{\Delta}{[\tau'/t]\Gamma}{[\tau'/t]e}{[\tau'/t]\tau}$ or $\ruleType{\Delta}{[\tau'/t]\Gamma}{[\tau'/t]r}{[\tau'/t]\tau}{[\tau'/t]\tau''}$ does not depend on the form of $\tau'$.
\item By mutual rule induction over Rules (\ref{rules:hastypeUP}) and Rule (\ref{rule:ruleType}) and case analysis over the definition of substitution. In all cases, the derivation of $\hastypeU{\Delta}{\Gamma}{[e'/x]e}{\tau}$ or $\ruleType{\Delta}{\Gamma}{[e'/x]r}{\tau}{\tau''}$ does not depend on the form of $e'$.
\end{enumerate}
\end{proof-sketch}

The Pattern Regularity Lemma establishes that the hypotheses generated by checking a pattern against a well-formed type involve only well-formed types.
\begin{lemma}[Pattern Regularity]\label{lemma:pattern-regularity-UP} 
If $\patType{\pctx}{p}{\tau}$ and $\istypeU{\Delta}{\tau}$ then $\istypeU{\Delta}{\tau_i}$ for each assumption $x_i : \tau_i$ in $\pctx$.
\end{lemma}
\begin{proof} By rule induction over Rules (\ref{rules:patType}).
\begin{byCases}
\item[\text{(\ref{rule:patType-var})}] We have:
\begin{pfsteps*}
  \item $p=x$ \BY{assumption}
  \item $\pctx=x : \tau$ \BY{assumption}
  \item $\istypeU{\Delta}{\tau}$ \BY{assumption}\pflabel{istypeU}
 \end{pfsteps*}
 \resetpfcounter
\item[\text{(\ref{rule:patType-wild})}] We have $\pctx=\emptyset$ by assumption, so the conclusion trivially holds.
\item[\text{(\ref{rule:patType-tpl})}] We have:
\begin{pfsteps*}
  \item $p=\aetplp{\labelset}{\mapschema{p}{i}{\labelset}}$ \BY{assumption}
  \item $\tau=\aprod{\labelset}{\mapschema{\tau}{i}{\labelset}}$ \BY{assumption}
  \item $\patType{\pctx_i}{p_i}{\tau_i}$ for each $i \in \labelset$ \BY{assumption}\pflabel{patType}
  \item $\pctx=\cup_{i \in \labelset} \pctx_i$ \BY{assumption}
  \item $\istypeU{\Delta}{\aprod{\labelset}{\mapschema{\tau}{i}{\labelset}}}$ \BY{assumption} \pflabel{istypeU}
  \item $\istypeU{\Delta}{\tau_i}$ for each $i \in \labelset$ \BY{Inversion of Rule (\ref{rule:istypeU-prod}) on \pfref{istypeU}}\pflabel{istypeU-each}
  \item $\istypeU{\Delta}{\tau_{ij}}$ for each $x_{ij} : \tau_{ij}$ in $\pctx_i$, for each $i \in \labelset$ \BY{IH on \pfref{patType} and \pfref{istypeU-each} for each $i \in \labelset$} \pflabel{biggy}
  \item $\istypeU{\Delta}{\tau_{ij}}$ for each $x_{ij} : \tau_{ij}$ in $\cup_{i \in \labelset}\pctx_i$ \BY{the definition of pattern context union and \pfref{biggy}}
\end{pfsteps*}
\resetpfcounter
\item[\text{(\ref{rule:patType-inj})}] We have:
\begin{pfsteps*}
  \item $p=\aeinjp{\ell}{p'}$ \BY{assumption}
  \item $\tau=\asum{\labelset, \ell}{\mapschema{\tau}{i}{\labelset}; \mapitem{\ell}{\tau'}}$ \BY{assumption}
  \item $\istypeU{\Delta}{\asum{\labelset, \ell}{\mapschema{\tau}{i}{\labelset}; \mapitem{\ell}{\tau'}}}$ \BY{assumption} \pflabel{istype}
  \item $\patType{\pctx}{p'}{\tau'}$ \BY{assumption} \pflabel{patType}
  \item $\istypeU{\Delta}{\tau'}$ \BY{Inversion of Rule (\ref{rule:istypeU-sum}) on \pfref{istype}} \pflabel{istypeTwo} 
  \item $\istypeU{\Delta}{\tau_i}$ for each assumption $x : \tau_i$ in $\pctx$ \BY{IH on \pfref{patType} and \pfref{istypeTwo}}
\end{pfsteps*}
\end{byCases}
\end{proof}

Finally, the Regularity Lemma establishes that the type assigned to an expression under a well-formed typing context is well-formed. 
\begin{lemma}[Regularity]\label{lemma:regularity-UP} All of the following hold:
\begin{enumerate}
\item If $\hastypeU{\Delta}{\Gamma}{e}{\tau}$ and $\istypeU{\Delta}{\tau_i}$ for each assumption $x_i : \tau_i$ in $\Gamma$ then $\istypeU{\Delta}{\tau}$.
\item If $\ruleType{\Delta}{\Gamma}{r}{\tau}{\tau'}$ and $\istypeU{\Delta}{\tau}$ and $\istypeU{\Delta}{\tau_i}$ for each assumption $x_i : \tau_i$ in $\Gamma$ then $\istypeU{\Delta}{\tau'}$.
\end{enumerate}
\end{lemma}
\begin{proof-sketch} By mutual rule induction over Rules (\ref{rules:hastypeUP}) and Rule (\ref{rule:ruleType}), and Lemma \ref{lemma:substitution-UP} and Lemma \ref{lemma:pattern-regularity-UP}.
\end{proof-sketch}
\subsection{Structural Dynamics}\label{sec:dynamics-UP}
The \emph{structural dynamics of }$\miniVersePat$ is specified as a transition system, and is organized around judgements of the following form:
\[\begin{array}{ll}
\textbf{Judgement Form} & \textbf{Description}\\
\stepsU{e}{e'} & \text{$e$ transitions to $e'$}\\
\isvalU{e} & \text{$e$ is a value}\\
\matchfail{e} & \text{$e$ raises match failure}
\end{array}\]
We also define auxiliary judgements for \emph{iterated transition}, $\multistepU{e}{e'}$, and \emph{evaluation}, $\evalU{e}{e'}$.

\begin{definition}[Iterated Transition]\label{defn:iterated-transition-UP} Iterated transition, $\multistepU{e}{e'}$, is the reflexive, transitive closure of the transition judgement, $\stepsU{e}{e'}$.\end{definition}

\begin{definition}[Evaluation]\label{defn:evaluation-UP}  $\evalU{e}{e'}$ iff $\multistepU{e}{e'}$ and $\isvalU{e'}$. \end{definition}

As in Sec. \ref{sec:dynamics-U}, our subsequent developments do not make mention of particular rules in the dynamics, nor do they make mention of judgements that are used only for defining the dynamics of the match operator, so we do not provide these details here. Instead, it suffices to state the following conditions.

The Canonical Forms condition characterizes well-typed values. Satisfying this condition requires an \emph{eager} (i.e. \emph{by-value}) formulation of the dynamics. This condition is identical to Condition \ref{condition:canonical-forms-U}.

\begin{condition}[Canonical Forms]\label{condition:canonical-forms-UP} If $\hastypeUC{e}{\tau}$ and $\isvalU{e}$ then:
\begin{enumerate}
\item If $\tau=\aparr{\tau_1}{\tau_2}$ then $e=\aelam{\tau_1}{x}{e'}$ and $\hastypeUCO{\Ghyp{x}{\tau_1}}{e'}{\tau_2}$.
\item If $\tau=\aall{t}{\tau'}$ then $e=\aetlam{t}{e'}$ and $\hastypeUCO{\Dhyp{t}}{e'}{\tau'}$.
\item If $\tau=\arec{t}{\tau'}$ then $e=\aefold{t}{\tau'}{e'}$ and $\hastypeUC{e'}{[\abop{rec}{t.\tau'}/t]\tau'}$ and $\isvalU{e'}$. 
\item If $\tau=\aprod{\labelset}{\mapschema{\tau}{i}{\labelset}}$ then $e=\aetpl{\labelset}{\mapschema{e}{i}{\labelset}}$ and $\hastypeUC{e_i}{\tau_i}$ and $\isvalU{e_i}$ for each $i \in \labelset$.
\item If $\tau=\asum{\labelset}{\mapschema{\tau}{i}{\labelset}}$ then for some label set $L'$ and label $\ell$, we have that $\labelset=\labelset', \ell$ and $\tau=\asum{\labelset', \ell}{\mapschema{\tau}{i}{\labelset'}; \mapitem{\ell}{\tau_\ell}}$ and $e=\aein{\labelset', \ell}{\ell}{\mapschema{\tau}{i}{\labelset'}; \ell \hookrightarrow \tau_\ell}{e'}$ and $\hastypeUC{e'}{\tau_\ell}$ and $\isvalU{e'}$.\end{enumerate}\end{condition}

The Preservation condition ensures that evaluation preserves typing.
\begin{condition}[Preservation]\label{condition:preservation-UP} If $\hastypeUC{e}{\tau}$ and $\stepsU{e}{e'}$ then $\hastypeUC{e'}{\tau}$. \end{condition}
The Progress condition ensures that evaluation of a well-typed expression preserves typing and cannot ``get stuck''.
\begin{condition}[Progress]\label{condition:progress-UP} If $\hastypeUC{e}{\tau}$ then either $\isvalU{e}$ or $\matchfail{e}$ or there exists an $e'$ such that $\stepsU{e}{e'}$. \end{condition}
 
The Preservation and Progress conditions together establish type safety.
%\noindent
%Condition \ref{condition:preservation-UP} is identical to Condition \ref{condition:preservation-U}, while Condition \ref{condition:progress-UP} modifies Condition \ref{condition:progress-U} to allow for match failure. 

We do not define exhaustiveness and redundancy properties here, because these can be checked post-expansion and so are also not relevant to our subsequent developments (but see \cite{pfple1} for a formal account).

\begin{figure}
\hspace{-8px}$\arraycolsep=4pt\begin{array}{lllllll}
\textbf{Sort} & & & \textbf{Operational Form} & \textbf{Stylized Form} & \textbf{Description}\\
\mathsf{UExp} & \ue & ::= & x & x & \text{variable}\\
&&& \aulam{\tau}{x}{\ue} & \lam{x}{\tau}{\ue} & \text{abstraction}\\
&&& \auap{\ue}{\ue} & \ap{\ue}{\ue} & \text{application}\\
&&& \autlam{t}{\ue} & \Lam{t}{\ue} & \text{type abstraction}\\
&&& \autap{\ue}{\tau} & \App{\ue}{\tau} & \text{type application}\\
&&& \aufold{t}{\tau}{\ue} & \fold{\ue} & \text{fold}\\
&&& \auunfold{\ue} & \unfold{\ue} & \text{unfold}\\
&&& \autpl{\labelset}{\mapschema{\ue}{i}{\labelset}} & \tpl{\mapschema{\ue}{i}{\labelset}} & \text{labeled tuple}\\
&&& \aupr{\ell}{\ue} & \prj{\ue}{\ell} & \text{projection}\\
&&& \auin{\labelset}{\ell}{\mapschema{\tau}{i}{\labelset}}{\ue} & \inj{\ell}{\ue} & \text{injection}\\
&&& \aumatchwith{n}{\tau}{\ue}{\seqschemaX{\urv}} & \matchwith{\ue}{\seqschemaX{\urv}} & \text{match}\\
\LCC &&& \gray & \gray & \gray \\
&&& \audefuetsm{\tau}{e}{\tsmv}{\ue} & \texttt{syntax}~a~\texttt{at}~\tau~\texttt{for} & \text{ueTSM definition}\\
&&&                                    & \texttt{expressions}~\{e\}~\texttt{in}~\ue\\
&&& \autsmap{b}{\tsmv} & \utsmap{\tsmv}{b} & \text{ueTSM application}\\\ECC
\LCC &&& \lightgray & \lightgray & \lightgray\\
&&& \audefuptsm{\tau}{e}{\tsmv}{\ue} & \texttt{syntax}~a~\texttt{at}~\tau~\texttt{for} & \text{upTSM definition}\\
&&&                                    & \texttt{patterns}~\{e\}~\texttt{in}~\ue\\\ECC
\mathsf{URule} & \urv & ::= & \aumatchrule{n}{\upv}{\seqschemaX{x}}{\ue} & \matchrule{\upv}{\ue} & \text{match rule}\\
\mathsf{UPat} & \upv & ::= & x & x & \text{variable pattern}\\
&&& \auwildp & \wildp & \text{wildcard pattern}\\
&&& \autplp{\labelset}{\mapschema{\upv}{i}{\labelset}} & \tplp{\mapschema{\upv}{i}{\labelset}} & \text{labeled tuple pattern}\\
&&& \auinjp{\ell}{\upv} & \injp{\ell}{\upv} & \text{injection pattern}\\
\LCC &&& \lightgray & \lightgray & \lightgray\\
&&& \auapuptsm{b}{\tsmv} & \utsmap{\tsmv}{b} & \text{upTSM application}\ECC
\end{array}$
\caption[Syntax of unexpanded expressions, rules and patterns in $\miniVersePat$]{Abstract syntax of unexpanded expressions, rules and patterns in $\miniVersePat$. Metavariable $\tsmv$ ranges over TSM names and $b$ ranges over literal bodies. Literal bodies might contain unparsed terms, so variable renaming and substitution cannot be defined in the usual manner over unexpanded terms.}
\label{fig:UP-unexpanded-terms}
\end{figure}

\subsection{Syntax of the Outer Surface}
Programs ultimately evaluate as expanded expressions, but programmers do not write expanded terms directly. Instead, the \emph{outer surface of} $\miniVersePat$ consists of types, defined above, and \emph{unexpanded expressions}, $\ue$, \emph{unexpanded rules}, $\urv$, and \emph{unexpanded patterns}, $\upv$ (collectively, \emph{unexpanded terms}). The syntax of unexpanded terms is specified in Figure \ref{fig:UP-unexpanded-terms}. 
Notice that each expanded term form corresponds to an unexpanded term form. We refer to these as the \emph{shared forms}. In addition, four unexpanded term forms do not correspond to expanded term forms: the ueTSM definition form, the ueTSM application form, the upTSM definition form and the upTSM application form. The forms related to ueTSMs are highlighted in dark gray, and the forms related to upTSMs are highlighted in light gray.

\subsection{Typed Expansion}
Unexpanded terms are typed and expanded according to the \emph{typed expansion judgements}:
\[\begin{array}{ll}
\textbf{Judgement Form} & \textbf{Description}\\
\expandsUP{\Delta}{\Gamma}{\uSigma}{\Phi}{\ue}{e}{\tau} & \text{$\ue$ has expansion $e$ and type $\tau$ under ueTSM context $\uSigma$}\\
& \text{and upTSM context $\Phi$ assuming $\Delta$ and $\Gamma$}\\
\ruleExpands{\Delta}{\Gamma}{\uSigma}{\Phi}{\urv}{r}{\tau}{\tau'} & \text{$\urv$ has expansion $r$ and takes values of type $\tau$ to values of}\\
& \text{type $\tau'$ under TSM environments $\uSigma$ and $\Phi$ assuming $\Delta$ and $\Gamma$}\\
\patExpands{\pctx}{\Phi}{\upv}{p}{\tau} & \text{$\upv$ has expansion $p$ and type $\tau$ and generates hypotheses $\pctx$ }\\
& \text{under upTSM context $\Phi$}
\end{array}\]

The \emph{typed expression expansion} judgement, $\expandsUP{\Delta}{\Gamma}{\uSigma}{\Phi}{\ue}{e}{\tau}$, and the \emph{typed rule expansion judgement}, $\ruleExpands{\Delta}{\Gamma}{\uSigma}{\Phi}{\urv}{r}{\tau}{\tau'}$ are defined mutually inductively by Rules (\ref*{rules:expandsUP}) and Rule (\ref*{rule:ruleExpands}), respectively, and the \emph{typed pattern expansion judgement}, $\patExpands{\pctx}{\Phi}{\upv}{p}{\tau}$, is inductively defined by Rules (\ref*{rules:patExpands}) as follows.

\paragraph{Shared Forms} Rules (\ref*{rule:expandsUP-var}) through (\ref*{rule:expandsUP-match}) define typed expansion of  unexpanded expressions of shared form. The first three of these rules are shown below:
\begin{subequations}\label{rules:expandsUP}
\begin{equation}\label{rule:expandsUP-var}
  \inferrule{ }{\expandsUP{\Delta}{\Gamma, x : \tau}{\uSigma}{\Phi}{x}{x}{\tau}}
\end{equation}
\begin{equation}\label{rule:expandsUP-lam}
  \inferrule{
    \istypeU{\Delta}{\tau}\\
    \expandsUP{\Delta}{\Gamma, x : \tau}{\uSigma}{\Phi}{\ue}{e}{\tau'}
  }{\expandsUPX{\aulam{\tau}{x}{\ue}}{\aelam{\tau}{x}{e}}{\aparr{\tau}{\tau'}}}
\end{equation}
\begin{equation}\label{rule:expandsUP-ap}
  \inferrule{
    \expandsUPX{\ue_1}{e_1}{\aparr{\tau}{\tau'}}\\
    \expandsUPX{\ue_2}{e_2}{\tau}
  }{
    \expandsUPX{\auap{\ue_1}{\ue_2}}{\aeap{e_1}{e_2}}{\tau'}
  }
\end{equation}
% \begin{equation}\label{rule:expandsUP-tlam}
%   \inferrule{
%     \expandsUP{\Delta, \Dhyp{t}}{\Gamma}{\uSigma}{\Phi}{\ue}{e}{\tau}
%   }{
%     \expandsUPX{\autlam{t}{\ue}}{\aetlam{t}{e}}{\aall{t}{\tau}}
%   }
% \end{equation}
% \begin{equation}\label{rule:expandsUP-tap}
%   \inferrule{
%     \expandsUPX{\ue}{e}{\aall{t}{\tau}}\\
%     \istypeU{\Delta}{\tau'}
%   }{
%     \expandsUPX{\autap{\ue}{\tau'}}{\aetap{e}{\tau'}}{[\tau'/t]\tau}
%   }
% \end{equation}
% \begin{equation}\label{rule:expandsUP-fold}
%   \inferrule{
%     \istypeU{\Delta, \Dhyp{t}}{\tau}\\
%     \expandsUPX{\ue}{e}{[\arec{t}{\tau}/t]\tau}
%   }{
%     \expandsUPX{\aufold{t}{\tau}{\ue}}{\aefold{t}{\tau}{e}}{\arec{t}{\tau}}
%   }
% \end{equation}
% \begin{equation}\label{rule:expandsUP-unfold}
%   \inferrule{
%     \expandsUPX{\ue}{e}{\arec{t}{\tau}}
%   }{
%     \expandsUPX{\auunfold{\ue}}{\aeunfold{e}}{[\arec{t}{\tau}/t]\tau}
%   }
% \end{equation}
% \begin{equation}\label{rule:expandsUP-tpl}
%   \inferrule{
%     \{\expandsUPX{\ue_i}{e_i}{\tau_i}\}_{i \in \labelset}
%   }{
%     \expandsUPX{\autpl{\labelset}{\mapschema{\ue}{i}{\labelset}}}{\aetpl{\labelset}{\mapschema{e}{i}{\labelset}}}{\aprod{\labelset}{\mapschema{\tau}{i}{\labelset}}}
%   }
% \end{equation}
% \begin{equation}\label{rule:expandsUP-pr}
%   \inferrule{
%     \expandsUPX{\ue}{e}{\aprod{\labelset, \ell}{\mapschema{\tau}{i}{\labelset}; \mapitem{\ell}{\tau}}}
%   }{
%     \expandsUPX{\aupr{\ell}{\ue}}{\aepr{\ell}{e}}{\tau}
%   }
% \end{equation}
% \begin{equation}\label{rule:expandsUP-in}
%   \inferrule{
%     \{\istypeU{\Delta}{\tau_i}\}_{i \in \labelset}\\
%     \istypeU{\Delta}{\tau}\\
%     \expandsUPX{\ue}{e}{\tau}
%   }{
%     \left\{\shortstack{$\Delta~\Gamma \vdash_{\uSigma;\,\Phi} \auin{\labelset, \ell}{\ell}{\mapschema{\tau}{i}{\labelset}; \mapitem{\ell}{\tau}}{\ue}$\\$\leadsto$\\$\aein{\labelset, \ell}{\ell}{\mapschema{\tau}{i}{\labelset}; \mapitem{\ell}{\tau}}{e} : \asum{\labelset, \ell}{\mapschema{\tau}{i}{\labelset}; \mapitem{\ell}{\tau}}$\vspace{-1.2em}}\right\}
%   }
% \end{equation}
Observe that, in each of these rules, the unexpanded and expanded expression forms in the conclusion correspond, and the premises correspond to those of the typing rule for the expanded expression form, i.e. Rules (\ref{rule:hastypeUP-var}), (\ref{rule:hastypeUP-lam}) and (\ref{rule:hastypeUP-ap}), respectively. In particular, the type formation premises correspond directly, and the typed expansion premises correspond to the typing premises. The ueTSM context, $\uSigma$, and upTSM context, $\Phi$, pass opaquely through these rules.
\refstepcounter{equation}
\label{rule:expandsUP-tlam}
\refstepcounter{equation}
\label{rule:expandsUP-tap}
\refstepcounter{equation}
\label{rule:expandsUP-fold}
\refstepcounter{equation}
\label{rule:expandsUP-unfold}
\refstepcounter{equation}
\label{rule:expandsUP-tpl}
\refstepcounter{equation}
\label{rule:expandsUP-pr}
\refstepcounter{equation}
\label{rule:expandsUP-in}

Rule (\ref*{rule:expandsUP-match}), below, defines typed expansion of unexpanded match expressions and corresponds analagously to Rule (\ref{rule:hastypeUP-match}).% The typed rule expansion premise corresponds to the rule typing premise of Rule (\ref{rule:ruleType}).
\begin{equation}\label{rule:expandsUP-match}
\inferrule{
  \expandsUPX{\ue}{e}{\tau}\\
  \istypeU{\Delta}{\tau'}\\
  \{\ruleExpands{\Delta}{\Gamma}{\uSigma}{\Phi}{\urv_i}{r_i}{\tau}{\tau'}\}_{1 \leq i \leq n}
}{\expandsUPX{\aumatchwith{n}{\tau'}{\ue}{\seqschemaX{\urv}}}{\aematchwith{n}{\tau'}{e}{\seqschemaX{r}}}{\tau'}}
\end{equation}  


We can express this scheme more precisely with the following rule transformation. For each rule in Rules (\ref{rules:hastypeUP}),
\begin{mathpar}
%\refstepcounter{equation}
%\label{rule:expandsU-case}
\inferrule{J_1\\ \cdots \\ J_k}{J}
\end{mathpar}
the corresponding typed expansion rule is 
\begin{mathpar}
\inferrule{
  \Uof{J_1} \\
  \cdots\\
  \Uof{J_k}
}{
  \Uof{J}
}
\end{mathpar}
where
\[\begin{split}
\Uof{\istypeU{\Delta}{\tau}} & = \istypeU{\Delta}{\tau} \\
\Uof{\hastypeU{\Gamma}{\Delta}{e}{\tau}} & = \expandsUP{\Gamma}{\Delta}{\uSigma}{\Phi}{\Uof{e}}{e}{\tau}\\
\Uof{\ruleType{\Gamma}{\Delta}{r}{\tau}{\tau'}} & = \ruleExpands{\Gamma}{\Delta}{\uSigma}{\Phi}{\Uof{r}}{r}{\tau}{\tau'}\\
\Uof{\{J_i\}_{i \in \labelset}} & = \{\Uof{J_i}\}_{i \in \labelset}
\end{split}\]
and where $\Uof{e}$, when $e$ is a metapattern of sort $\mathsf{Exp}$, is a metapattern of sort $\mathsf{UExp}$ defined as follows:
\begin{itemize}
\item When $e$ is of definite form, $\Uof{e}$ is defined as follows:
\begin{align*}
\Uof{x} & = x\\
\Uof{\aelam{\tau}{x}{e}} & = \aulam{\tau}{x}{\Uof{e}}\\
\Uof{\aeap{e_1}{e_2}} & = \auap{\Uof{e_1}}{\Uof{e_2}}\\
\Uof{\aetlam{t}{e}} & = \autlam{t}{\Uof{e}}\\
\Uof{\aetap{e}{\tau}} & = \autap{\Uof{e}}{\tau}\\
\Uof{\aefold{t}{\tau}{e}} & = \aufold{t}{\tau}{e}\\
\Uof{\aeunfold{e}} & = \auunfold{\Uof{e}}\\
\Uof{\aetpl{\labelset}{\mapschema{e}{i}{\labelset}}} & = \autpl{\labelset}{\mapschemax{\Uofv}{e}{i}{\labelset}}\\
\Uof{\aein{\labelset}{\ell}{\mapschema{\tau}{i}{\labelset}}{e}} &= \auin{\labelset}{\ell}{\mapschema{\tau}{i}{\labelset}}{\Uof{e}}\\
\Uof{\aematchwith{n}{\tau}{e}{\seqschemaX{r}}} &= \aumatchwith{n}{\tau}{\Uof{e}}{\seqschemaXx{\Uofv}{r}}
\end{align*}
\item When $e$ is of indefinite form, $\Uof{e}$ is a uniquely corresponding metapattern of indefinite form. %For example, in Rule (\ref{rule:hastypeUP-ap}), $e_1$ and $e_2$ are of indefinite form, i.e. they match arbitrary expanded expressions. The rule transformation simply ``hats'' them, i.e. $\Uof{e_1}=\ue_1$ and $\Uof{e_2}=\ue_2$.
\end{itemize}
and where $\Uof{r}$, when $r$ is a metapattern of sort $\mathsf{ERule}$ of indefinite form, is a uniquely corresponding  metapattern of sort $\mathsf{URule}$ of indefinite form. 

It is instructive to use this rule transformation to generate Rules (\ref{rule:expandsUP-var}) through (\ref{rule:expandsUP-ap}) and Rule (\ref{rule:expandsUP-match}) above. We omit the remaining rules generated by this transformation, i.e. Rules (\ref*{rule:expandsUP-tlam}) through (\ref*{rule:expandsUP-in}). 

The typed rule expansion judgement is defined by Rule (\ref*{rule:ruleExpands}), below.
\end{subequations}
\begin{equation}\label{rule:ruleExpands}
\inferrule{
  \patExpands{\pctx}{\Phi}{\upv}{p}{\tau}\\
  \domof{\pctx} = \seqschemaX{x}\\
  \expandsUP{\Delta}{\Gcons{\Gamma}{\pctx}}{\uSigma}{\Phi}{\ue}{e}{\tau'} 
}{
  \ruleExpands{\Delta}{\Gamma}{\uSigma}{\Phi}{\aumatchrule{n}{\upv}{\seqschemaX{x}}{\ue}}{\aematchrule{n}{p}{\seqschemaX{x}}{e}}{\tau}{\tau'}
}
\end{equation}
As in the typed expression expansion judgements, the unexpanded and expanded forms in the conclusion of the rule above correspond. The premises correspond to those of the rule typing rule, i.e. Rule (\ref{rule:ruleType}). In particular, the typed pattern expansion premise above corresponds to the pattern typing premise of Rule (\ref{rule:ruleType}), the second premise corresponds directly, and the typed expression expansion premise in the rule above corresponds to the typing premise of Rule (\ref{rule:ruleType}). 

The typed pattern expansion judgement for patterns of shared form is defined by the following rules.
\begin{subequations}[intermezzo]\label{rules:patExpands}
\begin{equation}\label{rule:patExpands-var}
\inferrule{ }{
  \patExpands{\Ghyp{x}{\tau}}{\Phi}{x}{x}{\tau}
}
\end{equation}
\begin{equation}\label{rule:patExpands-wild}
\inferrule{ }{
  \patExpands{\emptyset}{\Phi}{\auwildp}{\aewildp}{\tau}
}
\end{equation}
\begin{equation}\label{rule:patExpands-tpl}
\inferrule{
  \{\patExpands{\pctx_i}{\Phi}{\upv_i}{p_i}{\tau_i}\}_{i \in \labelset}\\
  \{\{\domof{\Omega_i} \cap \domof{\Omega_j} = \emptyset\}_{j \in \labelset \setminus i}\}_{i \in \labelset}
}{
  % \patExpands{\Gconsi{i \in \labelset}{\pctx_i}}{\Phi}{
  %   \autplp{\labelset}{\mapschema{\upv}{i}{\labelset}}
  % }{
  %   \aetplp{\labelset}{\mapschema{p}{i}{\labelset}}
  % }{
  %   \aprod{\labelset}{\mapschema{\tau}{i}{\labelset}}
  % } %{\autplp{\labelset}{\mapschema{\upv}{i}{\labelset}}}{\aetplp{\labelset}{\mapschema}{p}{i}{\labelset}}{...}
  \left(\shortstack{$\Gconsi{i \in \labelset}{\pctx_i} \vdash_\Phi \autplp{\labelset}{\mapschema{\upv}{i}{\labelset}}$\\$\leadsto$\\$\aetplp{\labelset}{\mapschema{p}{i}{\labelset}} : \aprod{\labelset}{\mapschema{\tau}{i}{\labelset}}$\vspace{-1.2em}}\right)
}
\end{equation}
\begin{equation}\label{rule:patExpands-in}
\inferrule{
  \patExpands{\pctx}{\Phi}{\upv}{p}{\tau}
}{
  \patExpands{\pctx}{\Phi}{\auinjp{\ell}{\upv}}{\aeinjp{\ell}{p}}{\asum{\labelset, \ell}{\mapschema{\tau}{i}{\labelset}; \mapitem{\ell}{\tau}}}
}
\end{equation}
\end{subequations}
Again, the unexpanded and expanded pattern forms in the conclusion correspond and the premises correspond to those of the corresponding pattern typing rule, i.e. Rules (\ref{rule:patType-var}) through (\ref{rule:patType-inj}), respectively. The upTSM context, $\Phi$, passes through these rules opaquely.
%By instead defining these rules by the rule transformation just described, we avoid having to list a number of rules that are individually uninteresting. Moreover, this approach makes our exposition somewhat robust to changes to the inner core (though not to changes to the judgement forms in the statics of the inner core).

\paragraph{ueTSM Definition and Application} Rules (\ref*{rule:expandsUP-syntax}) and (\ref*{rule:expandsUP-tsmap}) define typed expansion of ueTSM definitions and ueTSM application, respectively.  
\begin{subequations}[resume]
\begin{equation}\label{rule:expandsUP-syntax}
\inferrule{
  \istypeU{\Delta}{\tau}\\
  \hastypeU{\emptyset}{\emptyset}{\eparse}{\aparr{\tBody}{\tParseResultExp}}\\\\
  \expandsUP{\Delta}{\Gamma}{\uSigma, \xuetsmbnd{\tsmv}{\tau}{\eparse}}{\Phi}{\ue}{e}{\tau'}
}{
  \expandsUPX{\uesyntax{\tsmv}{\tau}{\eparse}{\ue}}{e}{\tau'}
}
\end{equation}
\begin{equation}\label{rule:expandsUP-tsmap}
\inferrule{
  \encodeBody{b}{\ebody}\\
  \evalU{\ap{\eparse}{\ebody}}{\inj{\lbltxt{Success}}{\ecand}}\\
  \decodeCondE{\ecand}{\ce}\\\\
  \cvalidE{\emptyset}{\emptyset}{\esceneUP{\Delta}{\Gamma}{\uSigma, \xuetsmbnd{\tsmv}{\tau}{\eparse}}{\Phi}{b}}{\ce}{e}{\tau}
}{
  \expandsUP{\Delta}{\Gamma}{\uSigma, \xuetsmbnd{\tsmv}{\tau}{\eparse}}{\Phi}{\utsmap{\tsmv}{b}}{e}{\tau}
}
\end{equation}
\end{subequations}
These rules are nearly identical to Rules (\ref{rule:expandsU-syntax}) and (\ref{rule:expandsU-tsmap}), respectively, differing only in that the upTSM context, $\Phi$, passes through them opaquely. The premises of these rules, and the following auxiliary definitions and conditions, can be understood as described in Sec. \ref{sec:U-uetsm-definition} and \ref{sec:U-uetsm-application}, respectively. 

The type $\tParseResultExp$ is defined as follows:
\[\tParseResultExp \triangleq [\mapitem{\lbltxt{Success}}{\tCEExp}, \mapitem{\lbltxt{ParseError}}{\prodt{}}]\]

The \emph{body encoding judgement} $\encodeBody{b}{\ebody}$ specifies a mapping from the literal body, $b$, to an expanded value, $\ebody$, of type $\tBody$. An inverse mapping is specified by the \emph{body decoding judgement} $\decodeBody{\ebody}{b}$.
\[\begin{array}{ll}
\textbf{Judgement Form} & \textbf{Description}\\
\encodeBody{b}{e} & \text{$b$ has encoding $e$}\\
\decodeBody{e}{b} & \text{$e$ has decoding $b$}
\end{array}\]
The following condition establishes an isomorphism between literal bodies and values of type $\tBody$.
\begin{condition}[Body Isomorphism] All of the following hold:
\begin{enumerate}
\item For every literal body $b$, we have that $\encodeBody{b}{\ebody}$ and $\hastypeUC{\ebody}{\tBody}$ and $\isvalU{\ebody}$.
\item If $\hastypeUC{\ebody}{\tBody}$ and $\isvalU{\ebody}$ then $\decodeBody{\ebody}{b}$ for some $b$.
\item If $\encodeBody{b}{\ebody}$ then $\decodeBody{\ebody}{b}$.
\item If $\hastypeUC{\ebody}{\tBody}$ and $\isvalU{\ebody}$ and $\decodeBody{\ebody}{b}$ then $\encodeBody{b}{\ebody}$. 
\item If $\encodeBody{b}{\ebody}$ and $\encodeBody{b}{\ebody'}$ then $\ebody = \ebody'$.
\item If $\hastypeUC{\ebody}{\tBody}$ and $\isvalU{\ebody}$ and $\decodeBody{\ebody}{b}$ and $\decodeBody{\ebody}{b'}$ then $b=b'$.
\end{enumerate}
\end{condition}

The \emph{candidate expansion expression decoding judgement}, $\decodeCondE{\ecand}{\ce}$, decodes $\ecand$ to produce a \emph{candidate expansion expression}, $\ce$  (pronounced ``grave $e$''). 
The inverse mapping is specified by the judgement $\encodeCondE{\ce}{\ecand}$. 
%The \emph{candidate expansion decoding judgement}, $\decodeCondE{e}{\ce}$, 
\[\begin{array}{ll}
\textbf{Judgement Form} & \textbf{Description}\\
\encodeCondE{\ce}{e} & \text{$\ce$ has encoding $e$}\\
\decodeCondE{e}{\ce} & \text{$e$ has decoding $\ce$}
\end{array}\]

The syntax of candidate expansion terms is defined in Figure \ref{fig:UP-candidate-terms}, and described in Sec. \ref{sec:ce-syntax-UP} below. The follow condition establishes an isomorphism between values of type $\tCEExp$ and candidate expansion expressions.
\begin{condition}[Candidate Expansion Expression Isomorphism] All of the following hold:
\begin{enumerate}
\item If $\hastypeUC{\ecand}{\tCEExp}$ and $\isvalU{\ecand}$ then $\decodeCondE{\ecand}{\ce}$ for some $\ce$.
\item For every $\ce$, we have $\encodeCondE{\ce}{\ecand}$ such that $\hastypeUC{\ecand}{\tCEExp}$ and $\isvalU{\ecand}$.
\item If $\hastypeUC{\ecand}{\tCEExp}$ and $\isvalU{\ecand}$ and $\decodeCondE{\ecand}{\ce}$ then $\encodeCondE{\ce}{\ecand}$.
\item If $\encodeCondE{\ce}{\ecand}$ then $\decodeCondE{\ecand}{\ce}$.
\item If $\hastypeUC{\ecand}{\tCEExp}$ and $\isvalU{\ecand}$ and $\decodeCondE{\ecand}{\ce}$ and $\decodeCondE{\ecand}{\ce'}$ then $\ce=\ce'$.
\item If $\encodeCondE{\ce}{\ecand}$ and $\encodeCondE{\ce}{\ecand'}$ then $\ecand=\ecand'$.
\end{enumerate}
\end{condition}


ueTSM contexts, $\uSigma$, are finite functions from TSM names, $\tsmv$, to {ueTSM definitions}, $\xuetsmdef{\tau}{\eparse}$, where $\tau$ is the ueTSM's {type annotation} and $\eparse$ is its {parse function}. The \emph{ueTSM context formation judgement}, $\uetsmenv{\Delta}{\uSigma}$, ensures that the type annotations in $\uSigma$ are well-formed assuming $\Delta$, and that the parse functions in $\uSigma$ are  of type $\aparr{\tBody}{\tParseResultExp}$.
\[\begin{array}{ll}
\textbf{Judgement Form} & \textbf{Description}\\
\uetsmenv{\Delta}{\uSigma} & \text{$\uSigma$ is well-formed assuming $\Delta$}\end{array}\]
This judgement is inductively defined by the following rules:
\begin{subequations}[intermezzo]\label{rules:uetsmenv-UP}
\begin{equation}\label{rule:uetsmenv-empty-UP}
\inferrule{ }{\uetsmenv{\Delta}{\emptyset}}
\end{equation}
\begin{equation}\label{rule:uetsmenv-ext-UP}
\inferrule{
  \uetsmenv{\Delta}{\uSigma}\\
  \istypeU{\Delta}{\tau}\\
  \hastypeU{\emptyset}{\emptyset}{\eparse}{\aparr{\tBody}{\tParseResultExp}}
}{
  \uetsmenv{\Delta}{\uSigma, \xuetsmbnd{\tsmv}{\tau}{\eparse}}
}
\end{equation}
\end{subequations}



\paragraph{upTSM Definition and Application}
Rules (\ref{rule:expandsUP-defuptsm}) and (\ref{rule:patExpands-apuptsm}) define upTSM definition and application, and are defined in the next two subsections, respectively.





\subsection{upTSM Definition}
The \emph{upTSM definition form}: 
\[\usyntaxup{\tsmv}{\tau}{\eparse}{\ue}\]
allows the programmer to introduce a upTSM named $\tsmv$ at type $\tau$ with parse function $\eparse$ into the upTSM context of $\ue$. The operational form corresponding to this stylized form is $\audefuptsm{\tau}{\eparse}{\tsmv}{\ue}$. Rule (\ref{rule:expandsUP-defuptsm}), defines typed expansion of upTSM definitions (for clarity, we use the stylized form in the conclusion of the rule):
\begin{subequations}[resume]
\begin{equation}\label{rule:expandsUP-defuptsm}
\inferrule{
  \istypeU{\Delta}{\tau}\\
  \hastypeU{\emptyset}{\emptyset}{\eparse}{\aparr{\tBody}{\tParseResultPat}}\\\\
  \expandsUP{\Delta}{\Gamma}{\uSigma}{\Phi, \xuptsmbnd{\tsmv}{\tau}{\eparse}}{\ue}{e}{\tau'} 
}{
  \expandsUP{\Delta}{\Gamma}{\uSigma}{\Phi}{\usyntaxup{\tsmv}{\tau}{\eparse}{\ue}}{e}{\tau'}
}
\end{equation}
\end{subequations}
This rule is similar to Rule (\ref{rule:expandsUP-syntax}), which governs ueTSM definitions. Its premises can be understood as follows, in order:
\begin{enumerate}
\item The first premise ensures that the type annotation is well-formed.
\item The second premise checks that the parse function, $\eparse$, is of type \[\aparr{\tBody}{\tParseResultPat}\] %to generate the \emph{expanded parse function}, $\eparse$. 
Parse functions are applied statically (i.e. during typed expansion), as we will discuss when describing ueTSM application below, and evaluation is defined only for closed expanded expressions, so the parse function must be closed. %Notice that this occurs under empty contexts, i.e. parse functions cannot refer to the surrounding bindings. 
%The parse function must be of type $\aparr{\tBody}{\tParseResultExp}$ where the type abbreviations $\tBody$ and $\tParseResultExp$ are defined as follows.

The type abbreviated $\tBody$ is characterized above. 

$\tParseResultPat$ abbreviates a labeled sum type that distinguishes successful parses from parse errors:%\footnote{In VerseML, the \li{ParseError} constructor of \li{ParseResult} required an error message and an error location, but we omit these in our formalization for simplicity}:
\[\tParseResultPat \triangleq [\mapitem{\lbltxt{Success}}{\tCEPat}, \mapitem{\lbltxt{ParseError}}{\prodt{}}]\] 

The type abbreviated $\tCEPat$ classifies encodings of \emph{candidate expansion patterns} (or \emph{ce-patterns}), $\cpv$ (pronounced ``grave $p$''). The syntax of ce-patterns will be described in Sec. \ref{sec:ce-syntax-UP}. The mapping from ce-patterns to values of type $\tCEPat$ is defined by the \emph{ce-pattern encoding judgement}, $\encodeCEPat{\cpv}{e}$. The inverse mapping is defined by the \emph{ce-pattern decoding judgement}, $\decodeCEPat{e}{\cpv}$.

\[\begin{array}{ll}
\textbf{Judgement Form} & \textbf{Description}\\
\encodeCEPat{\cpv}{e} & \text{$\cpv$ has encoding $e$}\\
\decodeCEPat{e}{\cpv} & \text{$e$ has decoding $\cpv$}
\end{array}\]

Again, rather than picking a particular definition of $\tCEPat$ and defining the judgements above inductively against it, we only state the following condition, which establishes an isomorphism between values of type $\tCEPat$ and ce-patterns.

\begin{condition}[Candidate Expansion Pattern Isomorphism] All of the following must hold:
\begin{enumerate}
\item If $\hastypeUC{\ecand}{\tCEPat}$ and $\isvalU{\ecand}$ then $\decodeCEPat{\ecand}{\cpv}$ for some $\cpv$.
\item For every $\cpv$, we have $\encodeCEPat{\cpv}{\ecand}$ such that $\hastypeUC{\ecand}{\tCEPat}$ and $\isvalU{\ecand}$.
\item If $\hastypeUC{\ecand}{\tCEPat}$ and $\isvalU{\ecand}$ and $\decodeCEPat{\ecand}{\cpv}$ then $\encodeCEPat{\cpv}{\ecand}$.
\item If $\encodeCEPat{\cpv}{\ecand}$ then $\decodeCEPat{\ecand}{\cpv}$.
\item If $\hastypeUC{\ecand}{\tCEPat}$ and $\isvalU{\ecand}$ and $\decodeCEPat{\ecand}{\cpv}$ and $\decodeCEPat{\ecand}{\cpv'}$ then $\cpv=\cpv'$.
\item If $\encodeCEPat{\cpv}{\ecand}$ and $\encodeCEPat{\cpv}{\ecand'}$ then $\ecand=\ecand'$.
\end{enumerate}
\end{condition}


\item The final premise of Rule (\ref{rule:expandsUP-defuptsm}) extends the upTSM context with the newly determined {upTSM definition}, and proceeds to assign a type, $\tau'$, and expansion, $e$, to $\ue$. The conclusion of Rule (\ref{rule:expandsUP-defuptsm}) assigns this type and expansion to the ueTSM definition as a whole.% i.e. TSMs define behavior that is relevant during typed expansion, but not during evaluation. 
\end{enumerate}
upTSM contexts, $\Phi$, are finite functions from TSM names, $\tsmv$, to \emph{upTSM definitions}, $\xuptsmdef{\tau}{\eparse}$, where $\tau$ is the upTSM's type annotation and $\eparse$ is the upTSM's parse function. The \emph{upTSM context formation judgement}, $\uptsmenv{\Delta}{\Phi}$, ensures that the type annotations in $\Phi$ are well-formed assuming $\Delta$ and the parse functions in $\Phi$ are of type $\aparr{\tBody}{\tParseResultPat}$.
\[\begin{array}{ll}
\textbf{Judgement Form} & \textbf{Description}\\
\uptsmenv{\Delta}{\Phi} & \text{upTSM context $\Phi$ is well-formed assuming $\Delta$}\end{array}\]
This judgement is inductively defined by the following rules:
\begin{subequations}\label{rules:uptsmenv-U}
\begin{equation}\label{rule:uptsmenv-empty}
\inferrule{ }{\uptsmenv{\Delta}{\emptyset}}
\end{equation}
\begin{equation}\label{rule:uptsmenv-ext}
\inferrule{
  \uptsmenv{\Delta}{\Phi}\\
  \istypeU{\Delta}{\tau}\\
  \hastypeU{\emptyset}{\emptyset}{\eparse}{\aparr{\tBody}{\tParseResultPat}}
}{
  \uptsmenv{\Delta}{\uSigma, \xuptsmbnd{\tsmv}{\tau}{\eparse}}
}
\end{equation}
\end{subequations}

\subsection{upTSM Application}\label{sec:uptsm-application}
The stylized unexpanded pattern form for applying a upTSM named $\tsmv$ to a literal form with literal body $b$ is:
\[
\utsmap{\tsmv}{b}
\] 
This stylized form is identical to the stylized form for ueTSM application, differing in that appears within the syntax of unexpanded patterns, $\upv$, rather than unexpanded expressions, $\ue$. %It uses forward slashes as delimiters, though stylized variants of any of the literal forms specified in Figure \ref{fig:literal-forms} would be straightforward to add to the syntax table in Figure \ref{fig:UP-unexpanded-terms} (we omit them for simplicity). 
The corresponding operational form is $\auapuptsm{b}{\tsmv}$.%, i.e. there is an operator $\texttt{uapuptsm}[b]$ for each literal body $b$ indexed by the TSM name $\tsmv$ and taking no arguments.

Rule (\ref{rule:patExpands-apuptsm}), below, governs upTSM application. 
\addtocounter{equation}{-3}
\begin{subequations}
\addtocounter{equation}{4}
\begin{equation}\label{rule:patExpands-apuptsm}
\inferrule{
  \encodeBody{b}{\ebody}\\
  \evalU{\ap{\eparse}{\ebody}}{\inj{\lbltxt{Success}}{\ecand}}\\
  \decodeCEPat{\ecand}{\cpv}\\\\
  \cvalidP{\pctx}{\pscene{\Phi, \xuptsmbnd{\tsmv}{\tau}{\eparse}}{b}}{\cpv}{p}{\tau}
}{
  \patExpands{\pctx}{\Phi, \xuptsmbnd{\tsmv}{\tau}{\eparse}}{\auapuptsm{b}{\tsmv}}{p}{\tau}
}
\end{equation}
\end{subequations}
\addtocounter{equation}{1}

\noindent
This rule is similar to Rule (\ref{rule:expandsUP-tsmap}), which governs ueTSM application. Its premises can be understood as follows, in order:
\begin{enumerate}
\item The first premise determines the encoding of the literal body, $\ebody$ (see above).
\item The second premise applies the parse function $\eparse$ to $\ebody$. If parsing succeeds, i.e. a value of the (stylized) form $\inj{\lbltxt{Success}}{\ecand}$ results from evaluation, then $\ecand$ will be a value of type $\tCEPat$ (assuming a well-formed upTSM context, by transitive application of Assumption \ref{condition:preservation-UP}). We call $\ecand$ the \emph{encoding of the candidate expansion}.
\item The third premise decodes the encoding of the candidate expansion to produce \emph{candidate expansion}, $\cpv$ (see above).
\item The final premise of Rule (\ref{rule:patExpands-apuptsm}) \emph{validates} the candidate expansion and simultaneously generates its final expansion, $p$, and its pattern typing context, $\pctx$. This is the topic of Sec. \ref{sec:ce-validation-UP}.
\end{enumerate}

\begin{figure}[p]
\hspace{-5px}$\begin{array}{lllllll}
\textbf{Sort} & & & \textbf{Operational Form} & \textbf{Stylized Form} & \textbf{Description}\\
\mathsf{CETyp} & \ctau & ::= & t & t & \text{variable}\\
&&& \aceparr{\ctau}{\ctau} & \parr{\ctau}{\ctau} & \text{partial function}\\
&&& \aceall{t}{\ctau} & \forallt{t}{\ctau} & \text{polymorphic}\\
&&& \acerec{t}{\ctau} & \rect{t}{\ctau} & \text{recursive}\\
&&& \aceprod{\labelset}{\mapschema{\ctau}{i}{\labelset}} & \prodt{\mapschema{\ctau}{i}{\labelset}} & \text{labeled product}\\
&&& \acesum{\labelset}{\mapschema{\ctau}{i}{\labelset}} & \sumt{\mapschema{\ctau}{i}{\labelset}} & \text{labeled sum}\\
\LCC &&& \gray & \gray & \gray\\
&&& \acesplicedt{m}{n} & \splicedt{m}{n} & \text{spliced}\\\ECC
\mathsf{CEExp} & \ce & ::= & x & x & \text{variable}\\
&&& \acelam{\ctau}{x}{\ce} & \lam{x}{\ctau}{\ce} & \text{abstraction}\\
&&& \aceap{\ce}{\ce} & \ap{\ce}{\ce} & \text{application}\\
&&& \acetlam{t}{\ce} & \Lam{t}{\ce} & \text{type abstraction}\\
&&& \acetap{\ce}{\ctau} & \App{\ce}{\ctau} & \text{type application}\\
&&& \acefold{t}{\ctau}{\ce} & \fold{\ce} & \text{fold}\\
&&& \aceunfold{\ce} & \unfold{\ce} & \text{unfold}\\
&&& \acetpl{\labelset}{\mapschema{\ce}{i}{\labelset}} & \tpl{\mapschema{\ce}{i}{\labelset}} & \text{labeled tuple}\\
&&& \acepr{\ell}{\ce} & \prj{\ce}{\ell} & \text{projection}\\
&&& \acein{\labelset}{\ell}{\mapschema{\ctau}{i}{\labelset}}{\ce} & \inj{\ell}{\ce} & \text{injection}\\
&&& \acematchwith{n}{\tau}{\ce}{\seqschemaX{\crv}} & \matchwith{\ce}{\seqschemaX{\crv}} & \text{match}\\
\LCC &&& \gray & \gray & \gray\\
&&& \acesplicede{m}{n} & \splicede{m}{n} & \text{spliced}\\\ECC
\mathsf{CERule} & \crv & ::= & \acematchrule{n}{p}{\seqschemaX{x}}{\ce} & \matchrule{p}{\ce} & \text{rule}\\
\mathsf{CEPat} & \cpv & ::= & \acewildp & \wildp & \text{wildcard pattern}\\
%&&& \aefoldp{p} & \foldp{p} & \text{fold pattern}\\
&&& \acetplp{\labelset}{\mapschema{\cpv}{i}{\labelset}} & \tplp{\mapschema{\cpv}{i}{\labelset}} & \text{labeled tuple pattern}\\
&&& \aceinjp{\ell}{\cpv} & \injp{\ell}{\cpv} & \text{injection pattern}\\
\LCC &&& \lightgray & \lightgray & \lightgray\\
&&& \acesplicedp{m}{n} & \splicedp{m}{n} & \text{spliced}\ECC
\end{array}$
\caption[Syntax of candidate expansion types and candidate expansion terms in $\miniVersePat$]{Abstract syntax of candidate expansion types and candidate expansion expressions, rules and patterns (collectively, candidate expansion terms) in $\miniVerseUE$. Candidate expansion types and terms are identified up to $\alpha$-equivalence.}
\label{fig:UP-candidate-terms}
\end{figure}

\subsection{Syntax of Candidate Expansions}\label{sec:ce-syntax-UP}
Figure \ref{fig:UP-candidate-terms} defines the syntax of candidate expansion types (or \emph{ce-types}), $\ctau$, candidate expansion expressions (or \emph{ce-expressions}), $\ce$, candidate expansion rules (or \emph{ce-rules}), $\crv$, and candidate expansion patterns (or \emph{ce-patterns}), $\cpv$. The syntax of ce-types is identical to that given in Figure \ref{fig:U-candidate-terms}, which was described in Sec. \ref{sec:ce-syntax-U}. 

Observe that for each expanded term form, except for the form for variable patterns, there is a corresponding ce-term form. We refer to these as the \emph{shared forms}. There are two other ce-term forms: a ce-expression form for \emph{references to spliced unexpanded expressions}, $\acesplicede{m}{n}$, highlighted in dark gray, and a ce-pattern form for \emph{references to spliced unexpanded patterns}, $\acesplicedp{m}{n}$, highlighted in light gray.

\subsection{Candidate Expansion Validation}\label{sec:ce-validation-UP}
The \emph{candidate expansion validation judgements} validate ce-types and ce-terms and simultaneously generate their final expansions.
\[\begin{array}{ll}
\textbf{Judgement Form} & \textbf{Description}\\
\cvalidT{\Delta}{\tscenev}{\ctau}{\tau} & \text{$\ctau$ is well-formed and has expansion $\tau$ assuming $\Delta$ and type}\\
& \text{splicing scene $\tscenev$}\\
\cvalidE{\Delta}{\Gamma}{\escenev}{\ce}{e}{\tau} & \text{$\ce$ has expansion $e$ and type $\tau$ assuming $\Delta$ and $\Gamma$ and expression}\\
& \text{splicing scene $\escenev$}\\
\cvalidR{\Delta}{\Gamma}{\escenev}{\crv}{r}{\tau}{\tau'} & \text{$\crv$ has expansion $r$ and takes values of type $\tau$ to values of type $\tau'$}\\
& \text{assuming $\Delta$ and $\Gamma$ and expression splicing scene $\escenev$}\\
\cvalidP{\pctx}{\pscenev}{\cpv}{p}{\tau} & \text{$\cpv$ expands to $p$ and matches values of type $\tau$ generating}\\
& \text{assumptions $\pctx$ assuming pattern splicing scene $\pscenev$}
\end{array}\]
\emph{Expression splicing scenes}, $\escenev$, are of the form $\esceneUP{\Delta}{\Gamma}{\uSigma}{\Phi}{b}$, \emph{type splicing scenes}, $\tscenev$, are of the form $\tsceneUP{\Delta}{b}$, and \emph{pattern splicing scenes}, $\pscenev$, are of the form $\pscene{\Phi}{b}$. Their purpose is to ``remember'', during candidate expansion validation, the contexts, TSM environments and literal bodies from the TSM application site (cf. Rules (\ref{rule:expandsUP-tsmap}) and (\ref{rule:patExpands-apuptsm})), because these are necessary to validate references to spliced types and terms. We write $\tsfrom{\escenev}$ for the type splicing scene constructed by dropping the typing context and TSM environments from $\escenev$:
\[\tsfrom{\esceneUP{\Delta}{\Gamma}{\uSigma}{\Phi}{b}} = \tsceneUP{\Delta}{b}\]

\subsubsection{Candidate Expansion Type Validation}
The \emph{candidate type validation judgement}, $\cvalidT{\Delta}{\tscenev}{\ctau}{\tau}$, is inductively defined by Rules (\ref{rules:cvalidT-U}), which were defined in Sec. \ref{sec:ce-validation-U}.

\subsubsection{Candidate Expansion Expression and Rule Validation}
\begin{subequations}\label{rules:cvalidE-UP}
The \emph{ce-expression validation judgement}, $\cvalidE{\Delta}{\Gamma}{\escenev}{\ce}{e}{\tau}$, and the \emph{ce-rule validation judgement}, $\cvalidR{\Delta}{\Gamma}{\escenev}{\crv}{r}{\tau}{\tau'}$, are defined mutually inductively with Rules (\ref{rules:expandsUP}) and Rule (\ref{rule:ruleExpands}) by Rules (\ref*{rules:cvalidE-UP}) and Rule (\ref*{rule:cvalidR-UP}), respectively, as follows.

Rules (\ref*{rules:cvalidE-UP}) define ce-expression validation and consist of the following rules:
\begin{itemize}
  \item \refstepcounter{equation}\label{rule:cvalidE-UP-var}
\refstepcounter{equation}\label{rule:cvalidE-UP-lam}
\refstepcounter{equation}\label{rule:cvalidE-UP-ap}
\refstepcounter{equation}\label{rule:cvalidE-UP-tlam}
\refstepcounter{equation}\label{rule:cvalidE-UP-tap}
\refstepcounter{equation}\label{rule:cvalidE-UP-fold}
\refstepcounter{equation}\label{rule:cvalidE-UP-unfold}
\refstepcounter{equation}\label{rule:cvalidE-UP-tpl}
\refstepcounter{equation}\label{rule:cvalidE-UP-prj}
\refstepcounter{equation}\label{rule:cvalidE-UP-in}
Rules defined identically to Rules (\ref{rule:cvalidE-U-var}) through (\ref{rule:cvalidE-U-in}). We will refer to these as Rules (\ref*{rule:cvalidE-UP-var}) through (\ref*{rule:cvalidE-UP-in}).
  \item The following rule for match ce-expressions:
  \begin{equation}\label{rule:cvalidE-UP-match}
\inferrule{
  \cvalidE{\Delta}{\Gamma}{\escenev}{\ce}{e}{\tau}\\
  \cvalidT{\Delta}{\tsfrom{\escenev}}{\ctau'}{\tau'}\\\\
  \{\cvalidR{\Delta}{\Gamma}{\escenev}{\crv_i}{r_i}{\tau}{\tau'}\}_{1 \leq i \leq n}
}{\cvalidE{\Delta}{\Gamma}{\escenev}{\acematchwith{n}{\ctau'}{\ce}{\seqschemaX{\crv}}}{\aematchwith{n}{\tau'}{e}{\seqschemaX{r}}}{\tau'}}
\end{equation}
\item The following rule for references to spliced unexpanded expressions, which can be understood as described in Sec. \ref{sec:ce-validation-U}.
\begin{equation}\label{rule:cvalidE-UP-splicede}
\inferrule{
  \parseUExp{\bsubseq{b}{m}{n}}{\ue}\\\\
  \expandsUP{\Delta_\text{app}}{\Gamma_\text{app}}{\uSigma}{\Phi_S}{\ue}{e}{\tau}\\
    \Delta \cap \Delta_\text{app} = \emptyset\\
  \domof{\Gamma} \cap \domof{\Gamma_\text{app}} = \emptyset\\
}{
  \cvalidE{\Delta}{\Gamma}{\esceneUP{\Delta_\text{app}}{\Gamma_\text{app}}{\uSigma}{\Phi_S}{b}}{\acesplicede{m}{n}}{e}{\tau}
}
\end{equation}
\end{itemize}

% \begin{equation}\label{rule:cvalidE-UP-var}
% \inferrule{ }{
%   \cvalidE{\Delta}{\Gamma, \Ghyp{x}{\tau}}{\escenev}{x}{x}{\tau}
% }
% \end{equation}
% \begin{equation}\label{rule:cvalidE-UP-lam}
% \inferrule{
%   \cvalidT{\Delta}{\tsfrom{\escenev}}{\ctau}{\tau}\\
%   \cvalidE{\Delta}{\Gamma, \Ghyp{x}{\tau}}{\escenev}{\ce}{e}{\tau'}
% }{
%   \cvalidE{\Delta}{\Gamma}{\escenev}{\acelam{\ctau}{x}{\ce}}{\aelam{\tau}{x}{e}}{\aparr{\tau}{\tau'}}
% }
% \end{equation}
% \begin{equation}\label{rule:cvalidE-UP-ap}
%   \inferrule{
%     \cvalidE{\Delta}{\Gamma}{\escenev}{\ce_1}{e_1}{\aparr{\tau}{\tau'}}\\
%     \cvalidE{\Delta}{\Gamma}{\escenev}{\ce_2}{e_2}{\tau}
%   }{
%     \cvalidE{\Delta}{\Gamma}{\escenev}{\aceap{\ce_1}{\ce_2}}{\aeap{e_1}{e_2}}{\tau'}
%   }
% \end{equation}
% \begin{equation}\label{rule:cvalidE-UP-tlam}
%   \inferrule{
%     \cvalidE{\Delta, \Dhyp{t}}{\Gamma}{\escenev}{\ce}{e}{\tau}
%   }{
%     \cvalidEX{\acetlam{t}{\ce}}{\aetlam{t}{e}}{\aall{t}{\tau}}
%   }
% \end{equation}
% \begin{equation}\label{rule:cvalidE-UP-tap}
%   \inferrule{
%     \cvalidEX{\ce}{e}{\aall{t}{\tau}}\\
%     \cvalidT{\Delta}{\tsfrom{\escenev}}{\ctau'}{\tau'}
%   }{
%     \cvalidEX{\acetap{\ce}{\ctau'}}{\aetap{e}{\tau'}}{[\tau'/t]\tau}
%   }
% \end{equation}
% \begin{equation}\label{rule:cvalidE-UP-fold}
%   \inferrule{
%     \cvalidT{\Delta, \Dhyp{t}}{\escenev}{\ctau}{\tau}\\
%     \cvalidEX{\ce}{e}{[\arec{t}{\tau}/t]\tau}
%   }{
%     \cvalidEX{\acefold{t}{\ctau}{\ce}}{\aefold{t}{\tau}{e}}{\arec{t}{\tau}}
%   }
% \end{equation}
% \begin{equation}\label{rule:cvalidE-UP-unfold}
%   \inferrule{
%     \cvalidEX{\ce}{e}{\arec{t}{\tau}}
%   }{
%     \cvalidEX{\aceunfold{\ce}}{\aeunfold{e}}{[\arec{t}{\tau}/t]\tau}
%   }
% \end{equation}
% \begin{equation}\label{rule:cvalidE-UP-tpl}
%   \inferrule{
%     \{\cvalidEX{\ce_i}{e_i}{\tau_i}\}_{i \in \labelset}
%   }{
%     \cvalidEX{\acetpl{\labelset}{\mapschema{\ce}{i}{\labelset}}}{\aetpl{\labelset}{\mapschema{e}{i}{\labelset}}}{\aprod{\labelset}{\mapschema{\tau}{i}{\labelset}}}
%   }
% \end{equation}
% \begin{equation}\label{rule:cvalidE-UP-pr}
%   \inferrule{
%     \cvalidEX{\ce}{e}{\aprod{\labelset, \ell}{\mapschema{\tau}{i}{\labelset}; \mapitem{\ell}{\tau}}}
%   }{
%     \cvalidEX{\acepr{\ell}{\ce}}{\aepr{\ell}{e}}{\tau}
%   }
% \end{equation}
% \begin{equation}\label{rule:cvalidE-UP-in}
%   \inferrule{
%     \{\cvalidT{\Delta}{\tsfrom{\escenev}}{\ctau_i}{\tau_i}\}_{i \in \labelset}\\
%     \cvalidT{\Delta}{\tsfrom{\escenev}}{\ctau}{\tau}\\
%     \cvalidEX{\ce}{e}{\tau}
%   }{
%     \left\{\shortstack{$\Delta~\Gamma \vdash_\uSigma \acein{\labelset, \ell}{\ell}{\mapschema{\ctau}{i}{\labelset}; \mapitem{\ell}{\ctau}}{\ce}$\\$\leadsto$\\$\aein{\labelset, \ell}{\ell}{\mapschema{\tau}{i}{\labelset}; \mapitem{\ell}{\tau}}{e} : \asum{\labelset, \ell}{\mapschema{\tau}{i}{\labelset}; \mapitem{\ell}{\tau}}$\vspace{-1.2em}}\right\}
%   }
% \end{equation}
% \begin{equation}\label{rule:cvalidE-UP-case}
%   \inferrule{
%     \cvalidEX{\ce}{e}{\asum{\labelset}{\mapschema{\tau}{i}{\labelset}}}\\
%     \{\cvalidE{\Delta}{\Gamma, \Ghyp{x_i}{\tau_i}}{\escenev}{\ue_i}{e_i}{\tau}\}_{i \in \labelset}
%   }{
%     \cvalidEX{\acecase{\labelset}{\tau}{\ce}{\mapschemab{x}{\ce}{i}{\labelset}}}{\aecase{\labelset}{\tau}{e}{\mapschemab{x}{e}{i}{\labelset}}}{\tau}
%   }
% \end{equation}
\end{subequations}
%The \emph{ce-rule validation judgement}, $\cvalidR{\Delta}{\Gamma}{\escenev}{\crv}{r}{\tau}{\tau'}$, is defined mutually inductively with Rules (\ref{rules:cvalidE-UP}) by 
Rule (\ref*{rule:cvalidR-UP}) defines ce-rule validation and is defined as follows:
\begin{equation}\label{rule:cvalidR-UP}
\inferrule{
  \patType{\pctx}{p}{\tau}\\
  \domof{\pctx} = \seqschemaX{x}\\
  \cvalidE{\Delta}{\Gcons{\Gamma}{\pctx}}{\escenev}{\ce}{e}{\tau'}
}{
  \cvalidR{\Delta}{\Gamma}{\escenev}{\acematchrule{n}{p}{\seqschemaX{x}}{\ce}}{\aematchrule{n}{p}{\seqschemaX{x}}{e}}{\tau}{\tau'}
}
\end{equation}
Notice that expanded patterns, $p$, not ce-patterns, $\cpv$, appear in ce-rules. This is because ce-expressions are generated only by ueTSMs. It would not be sensible for a ueTSM to extract a spliced subpattern from a literal body.

\subsubsection{Candidate Expansion Pattern Validation}
upTSMs generate candidate expansions of ce-pattern form, as described in Sec. \ref{sec:uptsm-application}. The \emph{ce-pattern validation judgement}, $\cvalidP{\pctx}{\pscenev}{\cpv}{p}{\tau}$, which appears as the final premise of Rule (\ref{rule:expandsUP-tsmap}), validates ce-patterns by checking that the pattern matches values of type $\tau$, and simultaneously generates the final expansion, $p$, and the hypotheses $\pctx$. Hypotheses can be generated only by spliced subpatterns, so there is no ce-pattern form corresponding to variable patterns (this is also why $\pctx$ does not appear to the left of the turnstile in the judgement form). The pattern splicing scene, $\pscenev$, is used to ``remember'' the upTSM context and literal body from the upTSM application site.

The ce-pattern validation judgement is defined mutually inductively with Rules (\ref{rules:patExpands}) by the following rules.

\begin{subequations}\label{rules:cvalidP-UP}
\begin{equation}\label{rule:cvalidP-UP-wild}
\inferrule{ }{
  \cvalidP{\emptyset}{\pscenev}{\acewildp}{\aewildp}{\tau}
}
\end{equation}
\begin{equation}\label{rule:cvalidP-UP-tpl}
\inferrule{
  \{\cvalidP{\pctx_i}{\pscenev}{\cpv_i}{p_i}{\tau_i}\}_{i \in \labelset}\\
  \{\{\domof{\Omega_i} \cap \domof{\Omega_j} = \emptyset\}_{j \in \labelset \setminus i}\}_{i \in \labelset}
}{
\left(\shortstack{$\vdash^{\Gconsi{i \in \labelset}{\pctx_i}; \pscenev} \acetplp{\labelset}{\mapschema{\cpv}{i}{\labelset}}$\\$\leadsto$\\$\aetplp{\labelset}{\mapschema{p}{i}{\labelset}} : \aprod{\labelset}{\mapschema{\tau}{i}{\labelset}}$\vspace{-1.2em}}\right)
}
\end{equation}
\begin{equation}\label{rule:cvalidP-UP-in}
\inferrule{
  \cvalidP{\pctx}{\pscenev}{\cpv}{p}{\tau}
}{
  \cvalidP{\pctx}{\pscenev}{\aceinjp{\ell}{\cpv}}{\aeinjp{\ell}{p}}{\asum{\labelset, \ell}{\mapschema{\tau}{i}{\labelset}; \mapitem{\ell}{\tau}}}
}
\end{equation}
\begin{equation}\label{rule:cvalidP-UP-spliced}
\inferrule{
  \parseUPat{\bsubseq{b}{m}{n}}{\upv}\\
  \patExpands{\pctx}{\Phi}{\upv}{p}{\tau}
}{
  \cvalidP{\pctx}{\pscene{\Phi}{b}}{\acesplicedp{m}{n}}{p}{\tau}
}
\end{equation}
\end{subequations}

Rules (\ref{rule:cvalidP-UP-wild}) through (\ref{rule:cvalidP-UP-in}) handle ce-patterns of shared form, and correspond to Rules (\ref{rule:patType-wild}) through (\ref{rule:patType-inj}). Rule (\ref{rule:cvalidP-UP-spliced}) handles references to spliced unexpanded patterns. The first premise parses the indicated subsequence of the literal body, $b$, to produce the referenced unexpanded pattern, $\upv$, and the second premise types and expands $\upv$ under the upTSM context $\Phi$ from the upTSM application site, producing the hypotheses $\pctx$. These are the hypotheses generated in the conclusion of the rule.

Notice that none of these rules explicitly add any hypotheses to the pattern typing context, so upTSMs cannot introduce any hypotheses other than those that come from such spliced subpatterns. This achieves the ``no hidden assumptions'' hygiene property described in Sec. \ref{sec:ptsms-hygiene}.

\subsection{Metatheory}
\begin{theorem}[Typed Pattern Expansion] Both of the following hold:
\begin{enumerate}
  \item If $\patExpands{\pctx}{\Phi}{\upv}{p}{\tau}$ and $\uptsmenv{\Delta}{\Phi}$ then $\patType{\pctx}{p}{\tau}$.
  \item If $\cvalidP{\pctx}{\pscene{\Phi}{b}}{\cpv}{p}{\tau}$ and $\uptsmenv{\Delta}{\Phi}$ then $\patType{\pctx}{p}{\tau}$.
\end{enumerate}
\end{theorem}
\begin{proof}
  By mutual rule induction on Rules (\ref{rules:patExpands}) and (\ref{rules:cvalidP-UP}).
\end{proof}

\begin{theorem}[Typed Expansion] All of the following hold:
\begin{enumerate}
  \item \begin{enumerate}
    \item If $\expandsUP{\Delta}{\Gamma}{\uSigma}{\Phi}{\ue}{e}{\tau}$ and $\uetsmenv{\Delta}{\uSigma}$ and $\uptsmenv{\Delta}{\Phi}$ then $\hastypeU{\Delta}{\Gamma}{e}{\tau}$.
    \item If $\ruleExpands{\Delta}{\Gamma}{\uSigma}{\Phi}{\urv}{r}{\tau}{\tau'}$ and $\uetsmenv{\Delta}{\uSigma}$ and $\uptsmenv{\Delta}{\Phi}$ then $\ruleType{\Delta}{\Gamma}{r}{\tau}{\tau'}$.
  \end{enumerate}
  \item \begin{enumerate}
    \item If $\cvalidE{\Delta}{\Gamma}{\esceneUP{\Delta_\text{app}}{\Gamma_\text{app}}{\uSigma}{\Phi_S}{b}}{\ce}{e}{\tau}$ and $\uetsmenv{\Delta_\text{app}}{\uSigma}$ and $\uptsmenv{\Delta_\text{app}}{\Phi_S}$ then $\hastypeU{\Dcons{\Delta}{\Delta_\text{app}}}{\Gcons{\Gamma}{\Gamma_\text{app}}}{e}{\tau}$. 
    \item If $\cvalidR{\Delta}{\Gamma}{\esceneUP{\Delta_\text{app}}{\Gamma_\text{app}}{\uSigma}{\Phi_S}{b}}{\crv}{r}{\tau}{\tau'}$ and $\uetsmenv{\Delta_\text{app}}{\uSigma}$ and $\uptsmenv{\Delta_\text{app}}{\Phi_S}$ then $\ruleType{\Dcons{\Delta}{\Delta_\text{app}}}{\Gcons{\Gamma}{\Gamma_\text{app}}}{r}{\tau}{\tau'}$.
  \end{enumerate}
\end{enumerate}
\end{theorem}
\begin{proof}
  By mutual rule induction on Rules (\ref{rules:expandsUP}), Rule (\ref{rule:ruleExpands}), Rules (\ref{rules:cvalidE-UP}) and Rule (\ref{rule:cvalidR-UP}).
\end{proof}

\chapter{Parameterized TSMs}\label{sec:tsms-parameterized}
\section{Parameterized TSMs By Example}
\subsection{Type Parameters By Example}
\subsection{Module Parameters By Example}
\section{\texorpdfstring{$\miniVerseParam$}{miniVerseForall}}
\subsection{Signatures, Types and Expanded Expressions}
\subsection{Parameter Application and Deferred Substitution}
\subsection{Macro Expansion and Validation}
\subsection{Metatheory}


% !TEX root = omar-thesis.tex
\chapter{Unparameterized TSM Implicits}\label{chap:tsls}
Using TSMs, a library provider can control the expansion of generalized literal forms, and thereby control the syntactic cost of common idioms. However, library clients must explicitly prefix each such form with a TSM name. In situations where the client is repeatedly using a TSM throughout a codebase, this can be inelegant. To further lower the syntactic cost of using TSMs, so that it compares to the syntactic cost of using derived forms built primitively into a language, VerseML allows clients to designate, for any type, one expression TSM and one pattern TSM as that type's \emph{designated TSMs} within a delimited scope. When VerseML's \emph{local type inference} system encounters a generalized literal form not prefixed by a TSM name (an \emph{unadorned literal form}), it implicitly applies the TSM designated at the type that the expression or pattern is being checked against.

\section{TSM Implicits By Example}\label{sec:tsm-implicits-by-example}
We begin in this section by introducing TSM implicits by example in VerseML. In Sec. \ref{sec:b-miniverse}, we formalize unparameterized TSM implicits with a reduced calculus, $\miniVerseUB$. We will also return to the topic of TSM implicits after introducing parameterized TSMs in Chapter \ref{chap:ptsms}.

\subsection{Designation}
In the example in Figure \ref{fig:implicits-example}, Lines 1 through 3 designate the expression TSM named \li{#\dolla#rx}, defined in Section \ref{sec:uetsms-definition}, and the pattern TSM named \li{#\dolla#rx}, defined in Sec. \ref{sec:ptsms-definition}, both at type \li{Rx}. These designations influence typed expansion of Lines 5 through 9.  %The scope of this declaration could be further restricted using the clauses shown in comments below. 
\begin{figure}
\begin{lstlisting}
implicit syntax 
  $rx at Rx for expressions
  $rx at Rx for patterns
in
  fun is_ssn(s : string) => rx_match /SURL\d\d\d-\d\d-\d\d\d\dEURL/ s
  fun name_from_example_rx(r : Rx) : string option => 
    match r with 
      /SURL@EURLnameSURL: %EURL_/ => Some name
    | _ => None
end
\end{lstlisting}
\caption{An example of TSM implicits in VerseML}
\label{fig:implicits-example}
\end{figure}


Expression and pattern TSMs need not be designated together, nor have the same name if they are. However, this is a common idiom, so for convenience, VerseML also provides a derived designation form that combines the two designations in Figure \ref{fig:implicits-example}:
\begin{lstlisting}[numbers=none]
implicit syntax $rx at Rx in (* ... *) end 
\end{lstlisting}

The type annotation on a designation is technically redundant -- the definition of the designated TSM determines the designated type. It is included in our examples for readability, but can be omitted if desired.

\subsection{Usage}

On Line 5 of Figure \ref{fig:implicits-example}, we apply a function \li{rx_match} (not shown), which has type \li{Rx -> string -> MatchResult}, to an expression of unadorned literal form. During typed expansion, the expression TSM \li{#\dolla#rx} is applied implicitly to this form to determine the expression's expansion,  because \li{#\dolla#rx} is the designated TSM at the argument type \li{Rx}. %f we had instead applied it explicitly, Line 2 would be written as follows:
% \begin{lstlisting}[numbers=none]
% fun is_ssn(s : string) => rx_match ($rx /SURL\d\d\d-\d\d-\d\d\d\dEURL/) s
% \end{lstlisting}

Similarly, a pattern of unadorned literal form appears on Line 8. Because it appears in a syntactic position where it must match values of type \li{Rx}, the pattern TSM \li{#\dolla#rx} is implicitly applied to determine its expansion.

\subsection{Analytic and Synthetic Positions}
During typed expansion of a subexpression, $e'$, of an expresssion, $e$, we say that $e'$ appears in \emph{analytic position} if the type that $e'$ must necessarily have can be determined based on the surrounding context, without examining $e'$. For example, an expression appearing as a function argument is in analytic position because the function's type determines the argument's type. Similarly, an expression may appear in analytic position due to a \emph{type ascription}, either directly on the expression, or ``further up'' in the expression:
\begin{lstlisting}[numbers=none]
val ssn = /SURL\d\d\d-\d\d-\d\d\d\dEURL/ : Rx
val ssn : Rx = /SURL\d\d\d-\d\d-\d\d\d\dEURL/
fun ssn() : Rx => /SURL\d\d\d-\d\d-\d\d\d\dEURL/
\end{lstlisting}

If the type that $e'$ must be assigned cannot be determined from context -- i.e. $e'$ must be examined to synthesize its type -- we instead say that the expression appears in a \emph{synthetic position}. For example, a top-level expression, or an expression appearing in a binding or function definition without a type ascription, appears in synthetic position.

Expressions of unadorned literal form can only appear in analytic position, because their type must be known to be able to determine the designated TSM that will control their expansion. For example, typed expansion of the following expression will fail because subexpressions of unadorned literal form appear in synthetic position:\newpage
\begin{lstlisting}[numbers=none]
let 
  val ssn = /SURL\d\d\d-\d\d-\d\d\d\dEURL/ (* INVALID *)
  fun ssn() => /SURL\d\d\d-\d\d-\d\d\d\dEURL/ (* INVALID *)
in 
  (* ... *) 
end
\end{lstlisting}

Patterns can always be of unadorned literal form in VerseML, because the scrutinee of a match expression is always in synthetic position, and so the type of value that each pattern appearing within the match expression must match is always known without examining the pattern itself. 
\section{\texorpdfstring{$\miniVerseUB$}{Bidirectional miniVerseU}}\label{sec:b-miniverse}
To formalize TSM implicits, we will now develop a reduced calculus called $\miniVerseUB$, or ``Bidirectional $\miniVersePat$'' (so named because it  explicitly distinguishes type analysis from type synthesis during typed expansion, as explained below).

\subsection{Inner Core}
The inner core of $\miniVerseUB$ is the same as the inner core of $\miniVersePat$, as described in Sections \ref{sec:inner-core-syntax-UP} through \ref{sec:dynamics-UP}. It consists of types, $\tau$, expanded expressions, $e$, expanded rules, $r$, and expanded patterns, $p$.

\subsection{Syntax of the Outer Surface}

\begin{figure}
\hspace{-8px}$\arraycolsep=4pt\begin{array}{lllllll}
\textbf{Sort} & & & \textbf{Operational Form} & \textbf{Stylized Form} & \textbf{Description}\\
\mathsf{UTyp} & \utau & ::= & \ut & \ut & \text{sigil}\\
&&& \auparr{\utau}{\utau} & \parr{\utau}{\utau} & \text{partial function}\\
&&& \auall{\ut}{\utau} & \forallt{\ut}{\utau} & \text{polymorphic}\\
&&& \aurec{\ut}{\utau} & \rect{\ut}{\utau} & \text{recursive}\\
&&& \auprod{\labelset}{\mapschema{\utau}{i}{\labelset}} & \prodt{\mapschema{\utau}{i}{\labelset}} & \text{labeled product}\\
&&& \ausum{\labelset}{\mapschema{\utau}{i}{\labelset}} & \sumt{\mapschema{\utau}{i}{\labelset}} & \text{labeled sum}\\
\mathsf{UExp} & \ue & ::= & \ux & \ux & \text{sigil}\\
&&& \auasc{\utau}{\ue} & \asc{\ue}{\utau} & \text{ascription}\\
&&& \auletsyn{\ux}{\ue}{\ue} & \letsyn{\ux}{\ue}{\ue} & \text{value binding}\\
&&& \auanalam{\ux}{\ue} & \analam{\ux}{\ue} & \text{abstraction (unannotated)}\\
&&& \aulam{\utau}{\ux}{\ue} & \lam{\ux}{\utau}{\ue} & \text{abstraction (annotated)}\\
&&& \auap{\ue}{\ue} & \ap{\ue}{\ue} & \text{application}\\
&&& \autlam{\ut}{\ue} & \Lam{\ut}{\ue} & \text{type abstraction}\\
&&& \autap{\ue}{\utau} & \App{\ue}{\utau} & \text{type application}\\
&&& \auanafold{\ue} & \fold{\ue} & \text{fold}\\
&&& \auunfold{\ue} & \unfold{\ue} & \text{unfold}\\
&&& \autpl{\labelset}{\mapschema{\ue}{i}{\labelset}} & \tpl{\mapschema{\ue}{i}{\labelset}} & \text{labeled tuple}\\
&&& \aupr{\ell}{\ue} & \prj{\ue}{\ell} & \text{projection}\\
&&& \auanain{\ell}{\ue} & \inj{\ell}{\ue} & \text{injection}\\
&&& \aumatchwithb{n}{\ue}{\seqschemaX{\urv}} & \matchwith{\ue}{\seqschemaX{\urv}} & \text{match}\\
&&& \audefuetsm{\utau}{e}{\tsmv}{\ue} & \texttt{syntax}~\tsmv~\texttt{at}~\utau~\texttt{for} & \text{ueTSM definition}\\
&&&                                    & \texttt{expressions}~\{e\}~\texttt{in}~\ue\\
\LCC &&& \lightgray & \lightgray & \lightgray \\
&&& \auimplicite{\tsmv}{\ue} & \texttt{implicit\,syntax}~\tsmv~\texttt{for} & \text{ueTSM designation}\\
&&&                          & \texttt{expressions\,in}~\ue\\ \ECC
&&& \autsmap{b}{\tsmv} & \utsmap{\tsmv}{b} & \text{ueTSM application}\\%\ECC
\LCC &&& \lightgray & \lightgray & \lightgray \\
&&& \auelit{b} & {\lit{b}}  & \text{ueTSM unadorned literal}\\\ECC
&&& \audefuptsm{\utau}{e}{\tsmv}{\ue} & \texttt{syntax}~\tsmv~\texttt{at}~\utau~\texttt{for} & \text{upTSM definition}\\
&&&                                    & \texttt{patterns}~\{e\}~\texttt{in}~\ue\\
\LCC &&& \lightgray & \lightgray & \lightgray \\
&&& \auimplicitp{\tsmv}{\ue} & \texttt{implicit\,syntax}~\tsmv~\texttt{for} & \text{upTSM designation}\\
&&&                          & \texttt{patterns\,in}~\ue\\ \ECC
\mathsf{URule} & \urv & ::= & \aumatchrule{\upv}{\ue} & \matchrule{\upv}{\ue} & \text{match rule}\\
\mathsf{UPat} & \upv & ::= & \ux & \ux & \text{sigil pattern}\\
&&& \auwildp & \wildp & \text{wildcard pattern}\\
&&& \aufoldp{\upv} & \foldp{\upv} & \text{fold pattern}\\
&&& \autplp{\labelset}{\mapschema{\upv}{i}{\labelset}} & \tplp{\mapschema{\upv}{i}{\labelset}} & \text{labeled tuple pattern}\\
&&& \auinjp{\ell}{\upv} & \injp{\ell}{\upv} & \text{injection pattern}\\
&&& \auapuptsm{b}{\tsmv} & \utsmap{\tsmv}{b} & \text{upTSM application}\\
\LCC &&& \lightgray & \lightgray & \lightgray\\
&&& \auplit{b} & \lit{b} & \text{upTSM unadorned literal}\ECC
\end{array}$
\caption[Syntax of unexpanded types, expressions, rules and patterns in $\miniVerseUB$]{Abstract syntax of unexpanded types, expressions, rules and patterns in $\miniVerseUB$.}
\label{fig:B-unexpanded-terms}
\end{figure}
A $\miniVerseUB$ program ultimately evaluates as an expanded expression. However, the programmer does not write the expanded expression directly. Instead, the programmer writes a textual sequence, $b$, consisting of characters in some suitable alphabet (e.g. in practice, \texttt{ASCII} or \texttt{Unicode}), which is parsed by some partial metafunction $\mathsf{parseUExp}(b)$ to produce an \emph{unex\-panded expression}, $\ue$. Unexpanded expressions can contain \emph{unexpanded types}, $\utau$, \emph{unexpanded rules}, $\urv$, and \emph{unexpanded patterns}, $\upv$, so we also need partial metafunctions $\mathsf{parseUTyp}(b)$, $\mathsf{parseURule}(b)$ and $\mathsf{parseUPat}(b)$. The abstract syntax of unexpanded types, expressions, rules and patterns, which form  the \emph{outer surface} of $\miniVerseUB$, is defined in Figure \ref{fig:UP-unexpanded-terms}. The full definition of the textual syntax of $\miniVerseUB$ is not important for our purposes, so we simply give the following condition, which states that there is some way to textually represent every unexpanded type, expression, rule and pattern. %We also assume a metafunction $\mathsf{parseUTyp}(b)$ for parsing unexpanded types, and impose an analagous condition.
\begin{condition}[Textual Representability] All of the following must hold:
\begin{enumerate}
\item For each $\utau$, there exists $b$ such that $\parseUTyp{b}{\utau}$. 
\item For each $\ue$, there exists $b$ such that $\parseUExp{b}{\ue}$.
\item For each $\urv$, there exists $b$ such that $\parseURule{b}{\urv}$.
\item For each $\upv$, there exists $b$ such that $\parseUPat{b}{\upv}$.
\end{enumerate}
\end{condition}

As in $\miniVersePat$, unexpanded types and expressions bind \emph{type sigils}, $\ut$, \emph{expression sigils}, $\ux$, and \emph{TSM names}, $\tsmv$. Sigils are given meaning by expansion to variables during typed expansion. We \textbf{cannot} adopt the usual definition of $\alpha$-renaming of identifiers, because unexpanded types and expressions are still in a ``partially parsed'' state -- the literal bodies, $b$, within an unexpanded expression might contain spliced subterms that are ``surfaced'' by a TSM only during typed expansion, as we will detail below.

Each inner core form (defined in Figure \ref{fig:UP-expanded-terms}) maps onto an outer surface form. In particular:
\begin{itemize}
\item Each type variable, $t$, maps onto a unique {type sigil}, written $\sigilof{t}$. %(pronounced ``sigil of $t$''). %Notice the distinction between $\ut$, which is a metavariable ranging over type sigils, and $\sigilof{t}$, which is a metafunction, written in stylized form, applied to a type variable to produce a type sigil.
\item Each type form, $\tau$, maps onto an unexpanded type form, $\Uof{\tau}$, according to the definition of $\Uof{\tau}$ in Sec. \ref{sec:syntax-U}.
\item Each expression variable, $x$, maps onto a unique expression sigil, written $\sigilof{x}$. %Again, notice the distinction between $\ux$ and $\sigilof{x}$.
\item Each expanded expression form, $e$, maps onto an unexpanded expression form $\Uof{e}$ as follows:
\begin{align*}
\Uof{x} & = \sigilof{x}\\
\Uof{\aelam{\tau}{x}{e}} & = \aulam{\Uof{\tau}}{\sigilof{x}}{\Uof{e}}\\
\Uof{\aeap{e_1}{e_2}} & = \auap{\Uof{e_1}}{\Uof{e_2}}\\
\Uof{\aetlam{t}{e}} & = \autlam{\sigilof{t}}{\Uof{e}}\\
\Uof{\aetap{e}{\tau}} & = \autap{\Uof{e}}{\Uof{\tau}}\\
\Uof{\aefold{t}{\tau}{e}} & = \auasc{\aurec{\sigilof{t}}{\Uof{\tau}}}{\auanafold{\Uof e}}\\
\Uof{\aeunfold{e}} & = \auunfold{\Uof{e}}\\
\Uof{\aetpl{\labelset}{\mapschema{e}{i}{\labelset}}} & = \autpl{\labelset}{\mapschemax{\Uofv}{e}{i}{\labelset}}\\
\Uof{\aein{\labelset}{\ell}{\mapschema{\tau}{i}{\labelset}}{e}} &= \auasc{\ausum{\labelset}{\mapschemax{\Uofv}{\tau}{i}{\labelset}}}{\auanain{\ell}{\Uof{e}}}\\
\Uof{\aematchwith{n}{\tau}{e}{\seqschemaX{r}}} &= \auasc{{\Uof{\tau}}}{\aumatchwithb{n}{\Uof{e}}{\seqschemaXx{\Uofv}{r}}}\\
\end{align*}
Notice that some type arguments that appear in $e$ appear within a type ascription in $\Uof{e}$. 
\item The expanded rule form maps onto the unexpanded rule form as follows:
\begin{align*}
\Uof{\aematchrule{p}{e}} & = \aumatchrule{\Uof{p}}{\Uof{e}}
\end{align*}
\item Each expanded pattern form, $p$, maps onto the unexpanded pattern form $\Uof{p}$ as follows:
\begin{align*}
\Uof{x} & = \sigilof{x}\\
\Uof{\aewildp} &= \auwildp\\
\Uof{\aefoldp{p}} &= \aufoldp{\Uof{p}}\\
\Uof{\aetplp{\labelset}{\mapschema{p}{i}{\labelset}}} & = \autplp{\labelset}{\mapschemax{\Uofv}{p}{i}{\labelset}}\\
\Uof{\aeinjp{\ell}{p}} & = \auinjp{\ell}{\Uof{p}}
\end{align*}
\end{itemize}

%Eight unexpanded forms relate to TSMs: the unexpanded expression forms for ueTSM definition, ueTSM designation, ueTSM application, ueTSM unadorned literals, upTSM definition and upTSM designation, and the unexpanded pattern forms for upTSM application and upTSM undorned literals. 
The forms related to TSM implicits are highlighted in gray in Figure \ref{fig:B-unexpanded-terms}.

\subsection{Bidirectionally Typed Expansion}
Unexpanded terms are checked and expanded simultaneously according to the \emph{bidirectionally typed expansion judgements}:
\[\begin{array}{ll}
\textbf{Judgement Form} & \textbf{Description}\\
\expandsTU{\uDelta}{\utau}{\tau} & \text{$\utau$ is well-formed and has expansion $\tau$ assuming $\uDelta$}\\
\esyn{\uDelta}{\uGamma}{\uPsi}{\uPhi}{\ue}{e}{\tau} & \text{$\ue$ has expansion $e$ and synthesizes type $\tau$ under $\uPsi$ and $\uPhi$}\\
& \text{assuming $\uDelta$ and $\uGamma$}\\
\eana{\uDelta}{\uGamma}{\uPsi}{\uPhi}{\ue}{e}{\tau} & \text{$\ue$ has expansion $e$ when analyzed against type $\tau$ under}\\
& \text{$\uPsi$ and $\uPhi$ assuming $\uDelta$ and $\uGamma$}\\
\rsyn{\uDelta}{\uGamma}{\uPsi}{\uPhi}{\urv}{r}{\tau}{\tau'} & \text{$\urv$ has expansion $r$ and takes values of type $\tau$ to values of}\\
& \text{synthesized type $\tau'$ under $\uPsi$ and $\uPhi$ assuming $\uDelta$ and $\uGamma$}\\
\rana{\uDelta}{\uGamma}{\uPsi}{\uPhi}{\urv}{r}{\tau}{\tau'} & \text{$\urv$ has expansion $r$ and takes values of type $\tau$ to values of}\\
& \text{type $\tau'$ when $\tau's$ is provided for analysis under $\uPsi$}\\
& \text{and $\uPhi$ assuming $\uDelta$ and $\uGamma$}\\
\patExpands{\upctx}{\uPhi}{\upv}{p}{\tau} & \text{$\upv$ has expansion $p$ and type $\tau$ and generates hypotheses $\upctx$ }\\
& \text{under upTSM context $\uPhi$ assuming $\Delta$}
\end{array}\]

\subsubsection{Type Expansion}
\emph{Unexpanded type formation contexts}, $\uDelta$, were defined in Sec. \ref{thm:typed-expansion-U}. The \emph{type expansion judgement}, $\expandsTU{\uDelta}{\utau}{\tau}$, is inductively defined by Rules (\ref{rules:expandsTU}).

\subsubsection{Typed Expression Expansion}
In order to clearly define the semantics of TSM implicits, we must make a judgemental distinction between type synthesis and type analysis. In the latter, the type is presumed known, while in the former, it must be synthesized by examining the term that is the subject of the judgement. Expressions of unadorned literal form can only be analyzed against a known type.%in analytic position.

The \emph{typed expression expansion judgements}, $\esynX{\ue}{e}{\tau}$, for type synthesis, and $\eanaX{\ue}{e}{\tau}$, for type analysis, are defined mutually inductively by Rules (\ref{rules:esyn}) and Rules (\ref{rules:eana}), respectively, as follows. 

\paragraph{Type Synthesis} \emph{Unexpanded typing contexts}, $\uGamma$, were defined in Sec. \ref{sec:typed-expansion-U}. Sigils that appear in $\uGamma$ have the expansion and synthesize the type that $\uGamma$ assigns to them.
\begin{subequations}\label{rules:esyn}
\begin{equation}\label{rule:esyn-var}
  \inferrule{ }{ 
    \esyn{\uDelta}{\uGamma, \uGhyp{\ux}{x}{\tau}}{\uPsi}{\uPhi}{\ux}{x}{\tau}
  }
\end{equation}

A \emph{type ascription} can be placed on an unexpanded expression to specify the type that it should be analyzed against. The ascribed type is synthesized if type analysis succeeds.
\begin{equation}\label{rule:esyn-asc}
  \inferrule{
    \expandsTU{\uDelta}{\utau}{\tau}\\
    \eanaX{\ue}{e}{\tau}
  }{
    \esynX{\auasc{\utau}{\ue}}{e}{\tau}
  }
\end{equation}

We define let-binding of a value in synthetic position primitively in $\miniVerseUB$. The following rule governs such bindings in synthetic position.
\begin{equation}\label{rule:esyn-let}
  \inferrule{
    \esynX{\ue}{e}{\tau}\\
    \esyn{\uDelta}{\uGamma, \uGhyp{\ux}{x}{\tau}}{\uPsi}{\uPhi}{\ue'}{e'}{\tau'}
  }{
    \esynX{\auletsyn{\ux}{\ue}{\ue'}}{\aeap{\aelam{\tau}{x}{e'}}{e}}{\tau'}
  }
\end{equation}

Functions with an argument type annotation can appear in synthetic position.
\begin{equation}\label{rule:esyn-lam}
  \inferrule{
    \expandsTU{\uDelta}{\utau_1}{\tau_1}\\
    \esyn{\uDelta}{\uGamma, \uGhyp{\ux}{x}{\tau_1}}{\uPsi}{\uPhi}{\ue}{e}{\tau_2}
  }{
    \esynX{\aulam{\utau_1}{\ux}{\ue}}{\aelam{\tau_1}{x}{e}}{\aparr{\tau_1}{\tau_2}}
  }
\end{equation}

Function applications can appear in synthetic position. The argument is analyzed against the argument type synthesized by the function.
\begin{equation}\label{rule:esyn-ap}
  \inferrule{
    \esynX{\ue_1}{e_1}{\aparr{\tau_2}{\tau}}\\
    \eanaX{\ue_2}{e_2}{\tau_2}
  }{
    \esynX{\auap{\ue_1}{\ue_2}}{\aeap{e_1}{e_2}}{\tau}
  }
\end{equation}

Type lambdas and type applications can appear in synthetic position.
\begin{equation}\label{rule:esyn-tlam}
  \inferrule{
    \esyn{\uDelta, \uDhyp{\ut}{t}}{\uGamma}{\uPsi}{\uPhi}{\ue}{e}{\tau}
  }{
    \esynX{\autlam{\ut}{\ue}}{\aetlam{t}{e}}{\aall{t}{\tau}}
  }
\end{equation}
\begin{equation}\label{rule:esyn-tap}
  \inferrule{
    \esynX{\ue}{e}{\aall{t}{\tau}}\\
    \expandsTU{\uDelta}{\utau'}{\tau'}
  }{
    \esynX{\autap{\ue}{\utau'}}{\aetap{e}{\tau'}}{[\tau'/t]\tau}
  }
\end{equation}

Unfoldings can appear in synthetic position.
\begin{equation}\label{rule:esyn-unfold}
  \inferrule{
    \esynX{\ue}{e}{\arec{t}{\tau}}
  }{
    \esynX{\auunfold{\ue}}{\aeunfold{e}}{[\arec{t}{\tau}/t]\tau}
  }
\end{equation}

Labeled tuples can appear in synthetic position. Each of the field values are then in synthetic position. 
\begin{equation}\label{rule:esyn-tpl}
  \inferrule{
    \{\esynX{\ue_i}{e_i}{\tau_i}\}_{i \in \labelset}
  }{
    \esynX{\autpl{\labelset}{\mapschema{\ue}{i}{\labelset}}}{\aetpl{\labelset}{\mapschema{e}{i}{\labelset}}}{\aprod{\labelset}{\mapschema{\tau}{i}{\labelset}}}
  }
\end{equation}

Fields can be projected out of a labeled tuple in synthetic position.
\begin{equation}\label{rule:esyn-pr}
  \inferrule{
    \esynX{\ue}{e}{\aprod{\labelset, \ell}{\mapschema{\tau}{i}{\labelset}; \mapitem{\ell}{\tau}}}
  }{
    \esynX{\aupr{\ell}{\ue}}{\aepr{\ell}{e}}{\tau}
  }
\end{equation}

Match expressions can appear in synthetic position.
\begin{equation}\label{rule:esyn-match}
  \inferrule{
    \esynX{\ue}{e}{\tau}\\
    \{\rsynX{\urv_i}{r_i}{\tau}{\tau'}\}_{1 \leq i \leq n}
  }{
    \esynX{\aumatchwithb{n}{\ue}{\seqschemaX{\urv}}}{\aematchwith{n}{\tau'}{e}{\seqschemaX{r}}}{\tau'}
  }
\end{equation}
\end{subequations}

\begin{subequations}[resume]
ueTSMs can be defined and applied in synthetic position.
\begin{equation}\label{rule:esyn-defuetsm}
\inferrule{
  \expandsTU{\uDelta}{\utau}{\tau}\\
  \hastypeU{\emptyset}{\emptyset}{\eparse}{\aparr{\tBody}{\tParseResultExp}}\\\\
  \esyn{\uDelta}{\uGamma}{\uPsi, \uShyp{\tsmv}{a}{\tau}{\eparse}}{\uPhi}{\ue}{e}{\tau'}
}{
  \esynX{\usyntaxueP{\tsmv}{\utau}{\eparse}{\ue}}{e}{\tau'}
}
\end{equation}
\begin{equation}\label{rule:esyn-apuetsm}
\inferrule{
  \encodeBody{b}{\ebody}\\
  \evalU{\ap{\eparse}{\ebody}}{\inj{\lbltxt{Success}}{\ecand}}\\
  \decodeCondE{\ecand}{\ce}\\\\
  \cana{\emptyset}{\emptyset}{\esceneUP{\uDelta}{\uGamma}{\uPsi, \uShyp{\tsmv}{a}{\tau}{\eparse}}{\uPhi}{b}}{\ce}{e}{\tau}
}{
  \esyn{\uDelta}{\uGamma}{\uPsi, \uShyp{\tsmv}{a}{\tau}{\eparse}}{\uPhi}{\utsmap{\tsmv}{b}}{e}{\tau}
}
\end{equation}
These rules are nearly identical to Rules (\ref{rule:expandsUP-syntax}) and (\ref{rule:expandsUP-tsmap}), differing only in that the typed expansion premises have been replaced by corresponding synthetic typed expansion premises. The premises of these rules can be understood as described in Sections \ref{sec:U-uetsm-definition} and \ref{sec:U-uetsm-application}. The body encoding judgement and candidate expansion expression decoding judgements were characterized in Sec. \ref{sec:typed-expansion-UP}. We discuss candidate expansion validation in Sec. \ref{sec:ce-validation-B} below.

To support ueTSM implicits, ueTSM contexts, $\uPsi$, are redefined to take the form $\uASI{\uA}{\Psi}{\uI}$. TSM naming contexts, $\uA$, and ueTSM definition contexts, $\Psi$, were defined in Sec. \ref{sec:typed-expansion-UP}. We write $\uPsi, \uShyp{\tsmv}{a}{\tau}{\eparse}$ when $\uPsi=\uASI{\uA}{\Psi}{\uI}$ as shorthand for \[\uASI{\ctxUpdate{\uA}{\tsmv}{a}}{\Psi, \xuetsmbnd{a}{\tau}{\eparse}}{\uI}\]

\emph{TSM designation contexts}, $\uI$, are finite functions that map each type $\tau \in \domof{\uI}$ to the \emph{TSM designation} $\designate{\tau}{a}$, for some symbol $a$. We write $\uI \uplus \designate{\tau}{a}$ for the TSM designation context that maps $\tau$ to $\designate{\tau}{a}$ and defers to $\uI$ for all other types (i.e. the previous designation, if any, is updated). 

The TSM designation context in the ueTSM context is updated by expressions of ueTSM designation form. Such expressions can appear in synthetic position, where they are governed by the following rule:% We write $\uIOK{\Delta}{\uI}$ when each type in $\uI$ is well-formed assuming $\Delta$.
%\begin{definition}[TSM Designation Context Well-Formedness] $\uIOK{\Delta}{{\uI}$ iff for each $\designate{\tau}{a}$ we have $\istypeU{\Delta}{\tau}$.\end{definition}
\begin{equation}\label{rule:esyn-implicite}
  \inferrule{
    \esyn{\uDelta}{\uGamma}{\uASI{\uA \uplus \vExpands{\tsmv}{a}}{\Psi, \xuetsmbnd{a}{\tau}{\eparse}}{\uI \uplus \designate{\tau}{a}}}{\uPhi}{\ue}{e}{\tau'}
  }{
    \esyn{\uDelta}{\uGamma}{\uASI{\uA \uplus \vExpands{\tsmv}{a}}{\Psi, \xuetsmbnd{a}{\tau}{\eparse}}{\uI}}{\uPhi}{\implicite{\tsmv}{\ue}}{e}{\tau'}
  }
\end{equation}

Like ueTSMs, upTSMs can be defined in synthetic position.
\begin{equation}\label{rule:esyn-defuptsm}
\inferrule{
  \expandsTU{\uDelta}{\utau}{\tau}\\
  \hastypeU{\emptyset}{\emptyset}{\eparse}{\aparr{\tBody}{\tParseResultPat}}\\\\
  \esyn{\uDelta}{\uGamma}{\uPsi}{\uPhi, \uPhyp{\tsmv}{a}{\tau}{\eparse}}{\ue}{e}{\tau'}
}{
  \esynX{\usyntaxup{\tsmv}{\utau}{\eparse}{\ue}}{e}{\tau'}
}
\end{equation}
This rule is nearly identical to Rule (\ref{rule:expandsUP-defuptsm}), differing only in that the typed expansion premise has been replaced by the corresponding synthetic typed expansion premise. The premises can be understood as described in Section \ref{sec:uptsm-definition}.

To support upTSM implicits, upTSM contexts, $\uPhi$, are redefined to take the form $\uASI{\uA}{\Phi}{\uI}$. upTSM definition contexts, $\Phi$, were defined in Sec. \ref{sec:uptsm-definition}. We write $\uPhi, \uPhyp{\tsmv}{a}{\tau}{\eparse}$ when $\uPhi=\uASI{\uA}{\Phi}{\uI}$ as shorthand for \[\uASI{\ctxUpdate{\uA}{\tsmv}{a}}{\Phi, \xuptsmbnd{a}{\tau}{\eparse}}{\uI}\]

The TSM designation context in the upTSM context is updated by expressions of upTSM designation form. Such expressions can appear in synthetic position, where they are governed by the following rule:% We write $\uIOK{\Delta}{\uI}$ when each type in $\uI$ is well-formed assuming $\Delta$.
%\begin{definition}[TSM Designation Context Well-Formedness] $\uIOK{\Delta}{{\uI}$ iff for each $\designate{\tau}{a}$ we have $\istypeU{\Delta}{\tau}$.\end{definition}
\begin{equation}\label{rule:esyn-implicitp}
  \inferrule{
    \esyn{\uDelta}{\uGamma}{\uPsi}{\uASI{\uA\uplus\vExpands{\tsmv}{a}}{\Phi, \xuptsmbnd{a}{\tau}{\eparse}}{\uI \uplus \designate{\tau}{a}}}{\ue}{e}{\tau'}
  }{
    \esyn{\uDelta}{\uGamma}{\uPsi}{\uASI{\uA\uplus\vExpands{\tsmv}{a}}{\Phi, \xuetsmbnd{a}{\tau}{\eparse}}{\uI}}{\implicitp{\tsmv}{\ue}}{e}{\tau'}
  }
\end{equation}
\end{subequations}

\paragraph{Type Analysis}
\begin{subequations}\label{rules:eana}
Type analysis subsumes type synthesis, in that when a type can be synthesized for an unexpanded expression, that unexpanded expression can also be analyzed against that type, producing the same expansion. This is expressed by the following \emph{subsumption rule} for unexpanded expressions.
\begin{equation}\label{rule:eana-subsume}
  \inferrule{
    \esynX{\ue}{e}{\tau}
  }{
    \eanaX{\ue}{e}{\tau}
  }
\end{equation}

Additional rules are needed for certain forms in order to propagate types for analysis into subexpressions, and for forms that can appear only in analytic position.

Rule (\ref{rule:esyn-let}) governed value bindings in synthetic position. The following rule governs value bindings in analytic position.
\begin{equation}\label{rule:eana-let}
  \inferrule{
    \esynX{\ue}{e}{\tau}\\
    \eana{\uDelta}{\uGamma, \uGhyp{\ux}{x}{\tau}}{\uPsi}{\uPhi}{\ue'}{e'}{\tau'}
  }{
    \eanaX{\auletsyn{\ux}{\ue}{\ue'}}{\aeap{\aelam{\tau}{x}{e'}}{e}}{\tau'}
  }
\end{equation}

An unannotated function can appear only in analytic position. The argument type is determined from the type that the unannotated function is being analyzed against. 
\begin{equation}\label{rule:eana-analam}
  \inferrule{
    \eana{\uDelta}{\uGamma, \uGhyp{\ux}{x}{\tau_1}}{\uPsi}{\uPhi}{\ue}{e}{\tau_2}
  }{
    \eanaX{\auanalam{\ux}{\ue}}{\aelam{\tau_1}{x}{e}}{\aparr{\tau_1}{\tau_2}}
  }
\end{equation}

Rule (\ref{rule:esyn-tlam}) governed type lambdas in synthetic position. The following rule governs type lambdas in analytic position.
\begin{equation}\label{rule:eana-tlam}
  \inferrule{
    \eana{\uDelta, \uDhyp{\ut}{t}}{\uGamma}{\uPsi}{\uPhi}{\ue}{e}{\tau}
  }{
    \eanaX{\autlam{\ut}{\ue}}{\aetlam{t}{e}}{\aall{t}{\tau}}
  }
\end{equation}

Values of recursive types can be introduced only in analytic position.
\begin{equation}\label{rule:eana-fold}
  \inferrule{
    \eanaX{\ue}{e}{[\arec{t}{\tau}/t]\tau}
  }{
    \eanaX{\auanafold{\ue}}{\aefold{t}{\tau}{e}}{\arec{t}{\tau}}
  }
\end{equation}

Rule (\ref{rule:esyn-tpl}) governed labeled tuples in synthetic position. The following rule governs labeled tuples in analytic position.
\begin{equation}\label{rule:eana-tpl}
  \inferrule{
    \{\eanaX{\ue_i}{e_i}{\tau_i}\}_{i \in \labelset}
  }{
    \eanaX{\autpl{\labelset}{\mapschema{\ue}{i}{\labelset}}}{\aetpl{\labelset}{\mapschema{e}{i}{\labelset}}}{\aprod{\labelset}{\mapschema{\tau}{i}{\labelset}}}
  }
\end{equation}

Values of labeled sum type can appear only in analytic position.
\begin{equation}\label{rule:eana-in}
  \inferrule{
    \eanaX{\ue}{e}{\tau}
  }{
    \left(\shortstack{$\uDelta~\uGamma \vdash_{\uPsi; \uPhi} \auanain{\ell}{\ue}$\\$\leadsto$\\$\auin{\labelset, \ell}{\ell}{\mapschema{\tau}{i}{\labelset}; \mapitem{\ell}{\tau}}{e} \Leftarrow \asum{\labelset, \ell}{\mapschema{\tau}{i}{\labelset}; \mapitem{\ell}{\tau}}$\vspace{-1.2em}}\right)
    %\eanaX{\auanain{\ell}{\ue}}{\aein{\labelset, \ell}{\ell}{\mapschema{\tau}{i}{\labelset}; \mapitem{\ell}{\tau}}{e}}{\asum{\labelset, \ell}{\mapschema{\tau}{i}{\labelset}; \mapitem{\ell}{\tau}}}
  }
\end{equation}

Rule (\ref{rule:esyn-match}) governed match expressions in synthetic position. The following rule governs match expressions in analytic position.
\begin{equation}\label{rule:eana-match}
  \inferrule{
    \esynX{\ue}{e}{\tau}\\
    \{\ranaX{\urv_i}{r_i}{\tau}{\tau'}\}_{1 \leq i \leq n}
  }{
    \eanaX{\aumatchwithb{n}{\ue}{\seqschemaX{\urv}}}{\aematchwith{n}{\tau'}{e}{\seqschemaX{r}}}{\tau'}
  }
\end{equation}
\end{subequations}


\begin{subequations}[resume]
Rule (\ref{rule:esyn-defuetsm}) governed ueTSM definitions in synthetic position. The following rule governs ueTSM definitions in analytic position.
\begin{equation}\label{rule:eana-defuetsm}
\inferrule{
  \expandsTU{\uDelta}{\utau}{\tau}\\
  \hastypeU{\emptyset}{\emptyset}{\eparse}{\aparr{\tBody}{\tParseResultExp}}\\\\
  \eana{\uDelta}{\uGamma}{\uPsi, \uShyp{\tsmv}{a}{\tau}{\eparse}}{\uPhi}{\ue}{e}{\tau'}
}{
  \eanaX{\usyntaxueP{\tsmv}{\utau}{\eparse}{\ue}}{e}{\tau'}
}
\end{equation}

Rule (\ref{rule:esyn-implicite}) governed ueTSM designations in synthetic position. The following rule governs ueTSM designations in analytic position.
\begin{equation}\label{rule:eana-implicite}
  \inferrule{
    \eana{\uDelta}{\uGamma}{\uASI{\uA \uplus \vExpands{\tsmv}{a}}{\Psi, \xuetsmbnd{a}{\tau}{\eparse}}{\uI \uplus \designate{\tau}{a}}}{\uPhi}{\ue}{e}{\tau'}
  }{
    \eana{\uDelta}{\uGamma}{\uASI{\uA \uplus \vExpands{\tsmv}{a}}{\Psi, \xuetsmbnd{a}{\tau}{\eparse}}{\uI}}{\uPhi}{\implicite{\tsmv}{\ue}}{e}{\tau'}
  }
\end{equation}

An expression of unadorned literal form can appear only in analytic position. The following rule extracts the TSM designated at the type that the expression is being analyzed against from the TSM designation context in the ueTSM context and applies it implicitly, i.e. the premises correspond to those of Rule (\ref{rule:esyn-apuetsm}).
\begin{equation}\label{rule:eana-lit}
  \inferrule{
    \encodeBody{b}{\ebody}\\
    \evalU{\ap{\eparse}{\ebody}}{\inj{\lbltxt{Success}}{\ecand}}\\
    \decodeCondE{\ecand}{\ce}\\\\
    \cana{\emptyset}{\emptyset}{\esceneUP{\uDelta}{\uGamma}{\uASI{\uA}{\Psi, \xuetsmbnd{a}{\tau}{\eparse}}{\uI \uplus \designate{\tau}{a}}}{\uPhi}{b}}{\ce}{e}{\tau}
  }{
    \eana{\uDelta}{\uGamma}{\uASI{\uA}{\Psi, \xuetsmbnd{a}{\tau}{\eparse}}{\uI \uplus \designate{\tau}{a}}}{\uPhi}{\auelit{b}}{e}{\tau}
  }
\end{equation}

Rule (\ref{rule:esyn-defuptsm}) governed upTSM definitions in synthetic position. The following rule governs upTSM definitions in analytic position.
\begin{equation}\label{rule:eana-defuptsm}
\inferrule{
  \expandsTU{\uDelta}{\utau}{\tau}\\
  \hastypeU{\emptyset}{\emptyset}{\eparse}{\aparr{\tBody}{\tParseResultPat}}\\\\
  \eana{\uDelta}{\uGamma}{\uPsi}{\uPhi, \uPhyp{\tsmv}{a}{\tau}{\eparse}}{\ue}{e}{\tau'}
}{
  \eanaX{\usyntaxup{\tsmv}{\utau}{\eparse}{\ue}}{e}{\tau'}
}
\end{equation}

Rule (\ref{rule:esyn-implicitp}) governed upTSM designations in synthetic position. The following rule governs upTSM designations in analytic position.
\begin{equation}\label{rule:eana-implicitp}
  \inferrule{
    \eana{\uDelta}{\uGamma}{\uPsi}{\uASI{\uA\uplus\vExpands{\tsmv}{a}}{\Phi, \xuptsmbnd{a}{\tau}{\eparse}}{\uI \uplus \designate{\tau}{a}}}{\ue}{e}{\tau'}
  }{
    \eana{\uDelta}{\uGamma}{\uPsi}{\uASI{\uA\uplus\vExpands{\tsmv}{a}}{\Phi, \xuetsmbnd{a}{\tau}{\eparse}}{\uI}}{\implicitp{\tsmv}{\ue}}{e}{\tau'}
  }
\end{equation}

\end{subequations}

\subsubsection{Typed Rule Expansion}

The synthetic typed rule expansion judgement is invoked iteratively by Rule (\ref{rule:esyn-match}) to synthesize a type, $\tau'$, from the branch expressions in the rule sequence. This judgement is defined mutually inductively with Rules (\ref{rules:esyn}) and Rules (\ref{rules:eana}) by the following rule. 
\begin{equation}\label{rule:rsyn}
  \inferrule{
    \patExpands{\uGG{\uG'}{\Gamma'}}{\uPhi}{\upv}{p}{\tau}\\
    \esyn{\uDD{\uD}{\Delta}}{\uGG{\uG \uplus \uG'}{\Gamma \cup \Gamma'}}{\uPsi}{\uPhi}{\ue}{e}{\tau'}
  }{
    \rsyn{\uDD{\uD}{\Delta}}{\uGG{\uG}{\Gamma}}{\uPsi}{\uPhi}{\aumatchrule{\upv}{\ue}}{\aematchrule{p}{e}}{\tau}{\tau'}
  }
\end{equation}
The analytic typed rule expansion judgement is invoked iteratively by Rule (\ref{rule:eana-match}). This judgement is defined mutually inductively with Rules (\ref{rules:esyn}), Rules (\ref{rules:eana}), and Rule (\ref{rule:rsyn}) by the following rule, which is the analytic analag of Rule (\ref{rule:rsyn}).
\begin{equation}\label{rule:rana}
  \inferrule{
    \patExpands{\uGG{\uG'}{\Gamma'}}{\uPhi}{\upv}{p}{\tau}\\
    \eana{\uDD{\uD}{\Delta}}{\uGG{\uG \uplus \uG'}{\Gamma \cup \Gamma'}}{\uPsi}{\uPhi}{\ue}{e}{\tau'}
  }{
    \rana{\uDD{\uD}{\Delta}}{\uGG{\uG}{\Gamma}}{\uPsi}{\uPhi}{\aumatchrule{\upv}{\ue}}{\aematchrule{p}{e}}{\tau}{\tau'}
  }
\end{equation}

The premises of these rules can be understood as described in Sec. \ref{sec:typed-expansion-UP}.% We will define typed pattern expansion below.


\subsubsection{Typed Pattern Expansion}
The typed pattern expansion judgement is inductively defined by Rules (\ref{rules:patExpands-B}) as follows. %As in $\miniVersePat$, \emph{unexpanded pattern typing contexts}, $\upctx$, are defined identically to unexpanded typing contexts (i.e. we only use a distinct metavariable to emphasize their distinct roles in the judgements above). 

The following rules are written identically to the typed pattern expansion rules for shared pattern forms in $\miniVersePat$, i.e. Rules (\ref{rule:patExpands-var}) through (\ref{rule:patExpands-in}).
\begin{subequations}\label{rules:patExpands-B}
\begin{equation}\label{rule:patExpands-B-var}
\inferrule{ }{
  \patExpands{\uGG{\vExpands{\ux}{x}}{\Ghyp{x}{\tau}}}{\uPhi}{\ux}{x}{\tau}
}
\end{equation}
\begin{equation}\label{rule:patExpands-B-wild}
\inferrule{ }{
  \patExpands{\uGG{\emptyset}{\emptyset}}{\uPhi}{\auwildp}{\aewildp}{\tau}
}
\end{equation}
\begin{equation}\label{rule:patExpands-B-fold}
\inferrule{ 
  \patExpands{\upctx}{\uPhi}{\upv}{p}{[\arec{t}{\tau}/t]\tau}
}{
  \patExpands{\upctx}{\uPhi}{\aufoldp{\upv}}{\aefoldp{p}}{\arec{t}{\tau}}
}
\end{equation}
\begin{equation}\label{rule:patExpands-B-tpl}
\inferrule{
  \{\patExpands{{\upctx_i}}{\uPhi}{\upv_i}{p_i}{\tau_i}\}_{i \in \labelset}\\
}{
  % \patExpands{\Gconsi{i \in \labelset}{\pctx_i}}{\Phi}{
  %   \autplp{\labelset}{\mapschema{\upv}{i}{\labelset}}
  % }{
  %   \aetplp{\labelset}{\mapschema{p}{i}{\labelset}}
  % }{
  %   \aprod{\labelset}{\mapschema{\tau}{i}{\labelset}}
  % } %{\autplp{\labelset}{\mapschema{\upv}{i}{\labelset}}}{\aetplp{\labelset}{\mapschema}{p}{i}{\labelset}}{...}
  \left(\shortstack{$\Delta \vdash_{\uPhi} \autplp{\labelset}{\mapschema{\upv}{i}{\labelset}}$\\$\leadsto$\\$\aetplp{\labelset}{\mapschema{p}{i}{\labelset}} : \aprod{\labelset}{\mapschema{\tau}{i}{\labelset}} \dashV \Gconsi{i \in \labelset}{\upctx_i}$\vspace{-1.2em}}\right)
}
\end{equation}
\begin{equation}\label{rule:patExpands-B-in}
\inferrule{
  \patExpands{\upctx}{\uPhi}{\upv}{p}{\tau}
}{
  \patExpands{\upctx}{\uPhi}{\auinjp{\ell}{\upv}}{\aeinjp{\ell}{p}}{\asum{\labelset, \ell}{\mapschema{\tau}{i}{\labelset}; \mapitem{\ell}{\tau}}}
}
\end{equation}

The following rule governs upTSM application. It is written identically to Rule (\ref{rule:patExpands-apuptsm}).
\begin{equation}\label{rule:patExpands-B-apuptsm}
\inferrule{
  \encodeBody{b}{\ebody}\\
  \evalU{\ap{\eparse}{\ebody}}{\inj{\lbltxt{Success}}{\ecand}}\\
  \decodeCEPat{\ecand}{\cpv}\\\\
  \cvalidP{\upctx}{\pscene{\Delta}{\uPhi, \uPhyp{\tsmv}{a}{\tau}{\eparse}}{b}}{\cpv}{p}{\tau}
}{
  \patExpands{\upctx}{\uPhi, \uPhyp{\tsmv}{a}{\tau}{\eparse}}{\utsmap{\tsmv}{b}}{p}{\tau}
}
\end{equation}

Unexpanded patterns of unadorned literal form are governed by the following rule, which extracts the designated upTSM from the upTSM context and applies it implicitly, i.e. the premises correspond to those of Rule (\ref{rule:patExpands-B-apuptsm}).
\begin{equation}\label{rule:patExpands-B-lit}
\inferrule{
  \encodeBody{b}{\ebody}\\
  \evalU{\ap{\eparse}{\ebody}}{\inj{\lbltxt{Success}}{\ecand}}\\
  \decodeCEPat{\ecand}{\cpv}\\\\
  \cvalidP{\upctx}{\pscene{\Delta}{\uASI{\uA}{\Phi, \xuptsmbnd{a}{\tau}{\eparse}}{\uI, \designate{\tau}{a}}}{b}}{\cpv}{p}{\tau}
}{
  \patExpands{\upctx}{\uASI{\uA}{\Phi, \xuptsmbnd{a}{\tau}{\eparse}}{\uI, \designate{\tau}{a}}}{\lit{b}}{p}{\tau}
}
\end{equation}

\end{subequations}


\subsubsection{Outer Surface Expressibility}
The following lemma establishes that each well-typed expanded pattern can be expressed as an unexpanded pattern matching values of the same type and generating the same hypotheses and corresponding sigil updates. The metafunction $\Uof{\pctx}$ was defined in \ref{sec:typed-expansion-UP}.
\begin{lemma}[Pattern Expressibility]\label{lemma:pattern-expressibility-B} If $\patType{\pctx}{p}{\tau}$ then $\patExpands{\Uof{\pctx}}{\uPhi}{\Uof{p}}{p}{\tau}$.\end{lemma}
\begin{proof} By rule induction over Rules (\ref{rules:patType}), using the definitions of $\Uof{\pctx}$ and $\Uof{p}$. In each case, we can apply the IH to or over each premise, then apply the corresponding rule in Rules (\ref{rules:patExpands-B}).\end{proof}

We can now establish the Expressibility Theorem -- that each well-typed expanded expression, $e$, can be expressed as an unexpanded expression, $\ue$, which synthesizes the same type under the corresponding contexts.

\begin{theorem}[Expressibility] Both of the following hold:
\begin{enumerate}
\item If $\hastypeU{\Delta}{\Gamma}{e}{\tau}$ then $\esyn{\Uof{\Delta}}{\Uof{\Gamma}}{\uPsi}{\uPhi}{\Uof{e}}{e}{\tau}$.
\item If $\ruleType{\Delta}{\Gamma}{r}{\tau}{\tau'}$ then $\rsyn{\Uof{\Delta}}{\Uof{\Gamma}}{\uPsi}{\uPhi}{\Uof{r}}{r}{\tau}{\tau'}$.
\end{enumerate}
\end{theorem}
\begin{proof} By mutual rule induction over Rules (\ref{rules:hastypeUP}) and Rule (\ref{rule:ruleType}) using the definitions of $\Uof{\Delta}$, $\Uof{\Gamma}$, $\Uof{e}$ and $\Uof{r}$. In each case, we apply the IH, part 1 to or over each typing premise, the IH, part 2 over each rule typing premise, Lemma \ref{lemma:type-expressibility} to or over each type formation premise, Lemma \ref{lemma:pattern-expressibility-B} to each pattern typing premise, then derive the conclusion by applying Rules (\ref{rules:esyn}) and Rule (\ref{rule:rsyn}).  
\end{proof} 


\begin{figure}[p]
\hspace{-5px}$\arraycolsep=4pt\begin{array}{lllllll}
\textbf{Sort} & & & \textbf{Operational Form} & \textbf{Stylized Form} & \textbf{Description}\\
\mathsf{CETyp} & \ctau & ::= & t & t & \text{variable}\\
&&& \aceparr{\ctau}{\ctau} & \parr{\ctau}{\ctau} & \text{partial function}\\
&&& \aceall{t}{\ctau} & \forallt{t}{\ctau} & \text{polymorphic}\\
&&& \acerec{t}{\ctau} & \rect{t}{\ctau} & \text{recursive}\\
&&& \aceprod{\labelset}{\mapschema{\ctau}{i}{\labelset}} & \prodt{\mapschema{\ctau}{i}{\labelset}} & \text{labeled product}\\
&&& \acesum{\labelset}{\mapschema{\ctau}{i}{\labelset}} & \sumt{\mapschema{\ctau}{i}{\labelset}} & \text{labeled sum}\\
%\LCC &&& \gray & \gray & \gray\\
&&& \acesplicedt{m}{n} & \splicedt{m}{n} & \text{spliced}\\%\ECC
\mathsf{CEExp} & \ce & ::= & x & x & \text{variable}\\
&&& \aceasc{\ctau}{\ce} & \asc{\ce}{\ctau} & \text{ascription}\\
&&& \aceletsyn{x}{\ce}{\ce} & \letsyn{x}{\ce}{\ce} & \text{value binding}\\
&&& \aceanalam{x}{\ce} & \analam{x}{\ce} & \text{abstraction (unannotated)}\\
&&& \acelam{\ctau}{x}{\ce} & \lam{x}{\ctau}{\ce} & \text{abstraction (annotated)}\\
&&& \aceap{\ce}{\ce} & \ap{\ce}{\ce} & \text{application}\\
&&& \acetlam{t}{\ce} & \Lam{t}{\ce} & \text{type abstraction}\\
&&& \acetap{\ce}{\ctau} & \App{\ce}{\ctau} & \text{type application}\\
&&& \aceanafold{\ce} & \fold{\ce} & \text{fold}\\
&&& \aceunfold{\ce} & \unfold{\ce} & \text{unfold}\\
&&& \acetpl{\labelset}{\mapschema{\ce}{i}{\labelset}} & \tpl{\mapschema{\ce}{i}{\labelset}} & \text{labeled tuple}\\
&&& \acepr{\ell}{\ce} & \prj{\ce}{\ell} & \text{projection}\\
&&& \aceanain{\ell}{\ce} & \inj{\ell}{\ce} & \text{injection}\\
&&& \acematchwithb{n}{\ce}{\seqschemaX{\urv}} & \matchwith{\ce}{\seqschemaX{\crv}} & \text{match}\\%\LCC &&& \gray & \gray & \gray\\
&&& \acesplicede{m}{n} & \splicede{m}{n} & \text{spliced}\\%\ECC
\mathsf{CERule} & \crv & ::= & \acematchrule{p}{\ce} & \matchrule{p}{\ce} & \text{rule}\\
\mathsf{CEPat} & \cpv & ::= & \acewildp & \wildp & \text{wildcard pattern}\\
&&& \acefoldp{p} & \foldp{p} & \text{fold pattern}\\
&&& \acetplp{\labelset}{\mapschema{\cpv}{i}{\labelset}} & \tplp{\mapschema{\cpv}{i}{\labelset}} & \text{labeled tuple pattern}\\
&&& \aceinjp{\ell}{\cpv} & \injp{\ell}{\cpv} & \text{injection pattern}\\
&&& \acesplicedp{m}{n} & \splicedp{m}{n} & \text{spliced}
\end{array}$
\caption[Syntax of candidate expansion terms in $\miniVerseUB$]{Abstract syntax of candidate expansion types, expressions, rules and patterns in $\miniVerseUB$. Candidate expansion terms are identified up to $\alpha$-equivalence.}
\label{fig:B-candidate-terms}
\end{figure}

\subsection{Syntax of Candidate Expansions}\label{sec:ce-syntax-B}
Figure \ref{fig:B-candidate-terms} defines the syntax of candidate expansion types (or \emph{ce-types}), $\ctau$, candidate expansion expressions (or \emph{ce-expressions}), $\ce$, candidate expansion rules (or \emph{ce-rules}), $\crv$, and candidate expansion patterns (or \emph{ce-patterns}), $\cpv$. %The syntax of ce-types is identical to that given in Figure \ref{fig:U-candidate-terms}, which was described in Sec. \ref{sec:ce-syntax-U}. 
Candidate expansion terms are identified up to $\alpha$-equivalence in the usual manner.

Each inner core form, except for the variable pattern form, maps onto a candidate expansion form. In particular:

\begin{itemize}
  \item Each type form maps onto a ce-type form according to the metafunction $\Cof{\tau}$, defined in Sec. \ref{sec:ce-syntax-U}.
  \item Each expanded expression form maps onto a ce-expression form according to the metafunction $\Cof{e}$, defined as follows:
  \begin{align*}
\Cof{x} & = {x}\\
\Cof{\aelam{\tau}{x}{e}} & = \acelam{\Cof{\tau}}{{x}}{\Cof{e}}\\
\Cof{\aeap{e_1}{e_2}} & = \aceap{\Cof{e_1}}{\Cof{e_2}}\\
\Cof{\aetlam{t}{e}} & = \acetlam{{t}}{\Cof{e}}\\
\Cof{\aetap{e}{\tau}} & = \acetap{\Cof{e}}{\Cof{\tau}}\\
\Cof{\aefold{t}{\tau}{e}} & = \aceasc{\acerec{{t}}{\Cof{\tau}}}{\aceanafold{\Cof e}}\\
\Cof{\aeunfold{e}} & = \aceunfold{\Cof{e}}\\
\Cof{\aetpl{\labelset}{\mapschema{e}{i}{\labelset}}} & = \acetpl{\labelset}{\mapschemax{\Cofv}{e}{i}{\labelset}}\\
\Cof{\aein{\labelset}{\ell}{\mapschema{\tau}{i}{\labelset}}{e}} &= \aceasc{\acesum{\labelset}{\mapschemax{\Cofv}{\tau}{i}{\labelset}}}{\aceanain{\ell}{\Cof{e}}}\\
\Cof{\aematchwith{n}{\tau}{e}{\seqschemaX{r}}} &= \aceasc{{\Cof{\tau}}}{\acematchwithb{n}{\Cof{e}}{\seqschemaXx{\Cofv}{r}}}\\
\end{align*}
  \item The expanded rule form maps onto the ce-rule form according to the metafunction $\Cof{r}$, defined as follows:
  \begin{align*}
  \Cof{\aematchrule{p}{e}} & = \acematchrule{p}{\Cof{e}}
  \end{align*}
  \item Each expanded pattern form, except for the variable pattern form, maps onto a ce-pattern form according to the metafunction $\Cof{p}$, defined in Sec. \ref{sec:ce-syntax-UP}.
  % \begin{align*}
  % \Cof{\aewildp} & = \acewildp\\
  % \Cof{\aefoldp{p}} & = \acefoldp{\Cof{p}}\\
  % \Cof{\aetplp{\labelset}{\mapschema{p}{i}{\labelset}}} & = \acetplp{\labelset}{\mapschemax{\Cofv}{p}{i}{\labelset}}\\
  % \Cof{\aeinjp{\ell}{p}} & = \aceinjp{\ell}{\Cof{p}}
  % \end{align*}
\end{itemize}

There are three other candidate expansion forms: a ce-type form for \emph{references to spliced unexpanded types}, $\acesplicedt{m}{n}$, a ce-expression form for \emph{references to spliced unexpanded expressions}, $\acesplicede{m}{n}$, and a ce-pattern form for \emph{references to spliced unexpanded patterns}, $\acesplicedp{m}{n}$. %TSM utilize these to splice types and unexpanded expressions out of literal bodies.

\subsection{Bidirectional Candidate Expansion Validation}\label{sec:ce-validation-B}
The \emph{bidirectional candidate expansion validation judgements} validate ce-terms and simultaneously generate their final expansions.
\[\begin{array}{ll}
\textbf{Judgement Form} & \textbf{Description}\\
\cvalidT{\Delta}{\tscenev}{\ctau}{\tau} & \text{$\ctau$ is well-formed and has expansion $\tau$ assuming $\Delta$ and type}\\
& \text{splicing scene $\tscenev$}\\
\csynX{\ce}{e}{\tau} & \text{$\ce$ has expansion $e$ and synthesizes type $\tau$ assuming $\Delta$ and $\Gamma$}\\
& \text{and expression splicing scene $\escenev$}\\
\canaX{\ce}{e}{\tau} & \text{$\ce$ has expansion $e$ when analyzed against type $\tau$ assuming $\Delta$}\\
& \text{and $\Gamma$ and expression splicing scene $\escenev$}\\
%\cvalidE{\Delta}{\Gamma}{\escenev}{\ce}{e}{\tau} & \text{$\ce$ has expansion $e$ and type $\tau$ assuming $\Delta$ and $\Gamma$ and expression}\\
%& \text{splicing scene $\escenev$}\\
\crsyn{\Delta}{\Gamma}{\escenev}{\crv}{r}{\tau}{\tau'} & \text{$\crv$ has expansion $r$ and takes values of type $\tau$ to values of}\\
& \text{synthesized type $\tau'$ assuming $\Delta$ and $\Gamma$ and $\escenev$}\\
\crana{\Delta}{\Gamma}{\escenev}{\crv}{r}{\tau}{\tau'} & \text{$\crv$ has expansion $r$ and takes values of type $\tau$ to values of}\\
& \text{type $\tau'$ when $\tau'$ is provided for analysis assuming $\Delta$ and $\Gamma$}\\
& \text{and $\escenev$}\\
\cvalidP{\upctx}{\pscenev}{\cpv}{p}{\tau} & \text{$\cpv$ expands to $p$ and matches values of type $\tau$ generating}\\
& \text{assumptions $\upctx$ assuming pattern splicing scene $\pscenev$}
\end{array}\]
\emph{Expression splicing scenes}, $\escenev$, are of the form $\esceneUP{\uDelta}{\uGamma}{\uPsi}{\uPhi}{b}$, \emph{type splicing scenes}, $\tscenev$, are of the form $\tsceneUP{\uDelta}{b}$, and \emph{pattern splicing scenes}, $\pscenev$, are of the form $\pscene{\Delta}{\uPhi}{b}$. Their purpose is to ``remember'', during candidate expansion validation, the contexts, TSM environments and literal bodies from the TSM application site (cf. Rules (\ref{rule:expandsUP-tsmap}) and (\ref{rule:patExpands-apuptsm})), because these are necessary to validate references to spliced terms. We write $\tsfrom{\escenev}$ for the type splicing scene constructed by dropping the unexpanded typing context and TSM environments from $\escenev$:
\[\tsfrom{\esceneUP{\uDelta}{\uGamma}{\uPsi}{\uPhi}{b}} = \tsceneUP{\uDelta}{b}\]

\subsubsection{Candidate Expansion Type Validation}
The \emph{ce-type validation judgement}, $\cvalidT{\Delta}{\tscenev}{\ctau}{\tau}$, is inductively defined by Rules (\ref{rules:cvalidT-U}), which were defined in Sec. \ref{sec:ce-validation-U}.

\subsubsection{Bidirectional Candidate Expansion Expression Validation}
Like the bidirectionally typed expression expansion judgements, the bidirectional ce-expression validation judgements distinguish type synthesis from type analysis. The \emph{synthetic ce-expression validation judgement}, $\csynX{\ce}{e}{\tau}$, and the \emph{analytic ce-expression validation judgement}, $\canaX{\ce}{e}{\tau}$, are defined mutually inductively with Rules (\ref{rules:esyn}) and Rules (\ref{rules:eana}) by Rules (\ref{rules:csyn}) and Rules (\ref{rules:cana}), respectively, as follows.

\paragraph{Type Synthesis} \begin{subequations}\label{rules:csyn}
Synthetic ce-expression validation is governed by the following rules.
\begin{equation}\label{rule:csyn-var}
  \inferrule{ }{ 
    \csyn{\Delta}{\Gamma, \Ghyp{x}{\tau}}{\escenev}{x}{x}{\tau}
  }
\end{equation}
\begin{equation}\label{rule:csyn-asc}
  \inferrule{
    \cvalidT{\Delta}{\tsfrom{\escenev}}{\ctau}{\tau}\\
    \canaX{\ce}{e}{\tau}
  }{
    \csynX{\aceasc{\ctau}{\ce}}{e}{\tau}
  }
\end{equation}
\begin{equation}\label{rule:csyn-let}
  \inferrule{
    \csynX{\ce}{e}{\tau}\\
    \csyn{\Delta}{\Gamma, \Ghyp{x}{\tau}}{\escenev}{\ce'}{e'}{\tau'}
  }{
    \csynX{\aceletsyn{x}{\ce}{\ce'}}{\aeap{\aelam{\tau}{x}{e'}}{e}}{\tau'}
  }
\end{equation}
\begin{equation}\label{rule:csyn-lam}
  \inferrule{
    \cvalidT{\Delta}{\tsfrom{\escenev}}{\ctau_1}{\tau_1}\\
    \csyn{\Delta}{\Gamma, \Ghyp{x}{\tau_1}}{\escenev}{\ce}{e}{\tau_2}
  }{
    \csynX{\acelam{\ctau_1}{x}{\ce}}{\aelam{\tau_1}{x}{e}}{\aparr{\tau_1}{\tau_2}}
  }
\end{equation}
\begin{equation}\label{rule:csyn-ap}
  \inferrule{
    \csynX{\ce_1}{e_1}{\aparr{\tau_2}{\tau}}\\
    \canaX{\ce_2}{e_2}{\tau_2}
  }{
    \csynX{\aceap{\ce_1}{\ce_2}}{\aeap{e_1}{e_2}}{\tau}
  }
\end{equation}
\begin{equation}\label{rule:csyn-tlam}
  \inferrule{
    \csyn{\Delta, \Dhyp{t}}{\Gamma}{\escenev}{\ce}{e}{\tau}
  }{
    \csynX{\acetlam{t}{\ce}}{\aetlam{t}{e}}{\aall{t}{\tau}}
  }
\end{equation}
\begin{equation}\label{rule:csyn-tap}
  \inferrule{
    \csynX{\ce}{e}{\aall{t}{\tau}}\\
    \cvalidT{\Delta}{\tsfrom{\escenev}}{\ctau'}{\tau'}
  }{
    \csynX{\acetap{\ce}{\ctau'}}{\aetap{e}{\tau'}}{[\tau'/t]\tau}
  }
\end{equation}
\begin{equation}\label{rule:csyn-unfold}
  \inferrule{
    \csynX{\ce}{e}{\arec{t}{\tau}}
  }{
    \csynX{\aceunfold{\ce}}{\aeunfold{e}}{[\arec{t}{\tau}/t]\tau}
  }
\end{equation}
\begin{equation}\label{rule:csyn-tpl}
  \inferrule{
    \{\csynX{\ce_i}{e_i}{\tau_i}\}_{i \in \labelset}
  }{
    \csynX{\acetpl{\labelset}{\mapschema{\ce}{i}{\labelset}}}{\aetpl{\labelset}{\mapschema{e}{i}{\labelset}}}{\aprod{\labelset}{\mapschema{\tau}{i}{\labelset}}}
  }
\end{equation}
\begin{equation}\label{rule:csyn-pr}
  \inferrule{
    \csynX{\ce}{e}{\aprod{\labelset, \ell}{\mapschema{\tau}{i}{\labelset}; \mapitem{\ell}{\tau}}}
  }{
    \esynX{\acepr{\ell}{\ce}}{\aepr{\ell}{e}}{\tau}
  }
\end{equation}
\begin{equation}\label{rule:csyn-match}
  \inferrule{
    \csynX{\ce}{e}{\tau}\\
    \{\crsynX{\crv_i}{r_i}{\tau}{\tau'}\}_{1 \leq i \leq n}
  }{
    \csynX{\acematchwithb{n}{\ce}{\seqschemaX{\crv}}}{\aematchwith{n}{\tau'}{e}{\seqschemaX{r}}}{\tau'}
  }
\end{equation}
\begin{equation}\label{rule:csyn-splicede}
\inferrule{
  \parseUExp{\bsubseq{b}{m}{n}}{\ue}\\
  \esyn{\uDD{\uD}{\Delta_\text{app}}}{\uGG{\uG}{\Gamma_\text{app}}}{\uPsi}{\uPhi}{\ue}{e}{\tau}\\\\
    \Delta \cap \Delta_\text{app} = \emptyset\\
  \domof{\Gamma} \cap \domof{\Gamma_\text{app}} = \emptyset\\
}{
  \csyn{\Delta}{\Gamma}{\esceneUP{\uDD{\uD}{\Delta_\text{app}}}{\uGG{\uG}{\Gamma_\text{app}}}{\uPsi}{\uPhi}{b}}{\acesplicede{m}{n}}{e}{\tau}
}
\end{equation}
\end{subequations}

Rules (\ref{rule:csyn-var}) through (\ref{rule:csyn-match}) are analagous to Rules (\ref{rule:esyn-var}) through (\ref{rule:esyn-match}). Rule (\ref{rule:csyn-splicede}) governs references to spliced unexpanded expressions in synthetic position, and can be understood as described in Sec. \ref{sec:ce-validation-U}.


\paragraph{Type Analysis} \begin{subequations}\label{rules:cana}
Analytic ce-expression validation is governed by the following rules.
\begin{equation}\label{rule:cana-subsume}
  \inferrule{
    \csynX{\ce}{e}{\tau}
  }{
    \canaX{\ce}{e}{\tau}
  }
\end{equation}
\begin{equation}\label{rule:cana-let}
  \inferrule{
    \csynX{\ce}{e}{\tau}\\
    \cana{\Delta}{\Gamma, \Ghyp{x}{\tau}}{\escenev}{\ce'}{e'}{\tau'}
  }{
    \canaX{\aceletsyn{x}{\ce}{\ce'}}{\aeap{\aelam{\tau}{x}{e'}}{e}}{\tau'}
  }
\end{equation}
\begin{equation}\label{rule:cana-analam}
  \inferrule{
    \cana{\Delta}{\Gamma, \Ghyp{x}{\tau_1}}{\escenev}{\ce}{e}{\tau_2}
  }{
    \canaX{\aceanalam{x}{\ue}}{\aelam{\tau_1}{x}{e}}{\aparr{\tau_1}{\tau_2}}
  }
\end{equation}
\begin{equation}\label{rule:cana-tlam}
  \inferrule{
    \cana{\Delta, \Dhyp{t}}{\Gamma}{\escenev}{\ce}{e}{\tau}
  }{
    \canaX{\acetlam{t}{\ce}}{\aetlam{t}{e}}{\aall{t}{\tau}}
  }
\end{equation}
\begin{equation}\label{rule:cana-fold}
  \inferrule{
    \canaX{\ce}{e}{[\arec{t}{\tau}/t]\tau}
  }{
    \canaX{\aceanafold{\ce}}{\aefold{t}{\tau}{e}}{\arec{t}{\tau}}
  }
\end{equation}
\begin{equation}\label{rule:cana-tpl}
  \inferrule{
    \{\canaX{\ce_i}{e_i}{\tau_i}\}_{i \in \labelset}
  }{
    \canaX{\acetpl{\labelset}{\mapschema{\ce}{i}{\labelset}}}{\aetpl{\labelset}{\mapschema{e}{i}{\labelset}}}{\aprod{\labelset}{\mapschema{\tau}{i}{\labelset}}}
  }
\end{equation}
\begin{equation}\label{rule:cana-in}
  \inferrule{
    \canaX{\ce}{e}{\tau}
  }{
    \left(\shortstack{$\Delta~\Gamma \vdash^{\escenev} \aceanain{\ell}{\ce}$\\$\leadsto$\\$\auin{\labelset, \ell}{\ell}{\mapschema{\tau}{i}{\labelset}; \mapitem{\ell}{\tau}}{e} \Leftarrow \asum{\labelset, \ell}{\mapschema{\tau}{i}{\labelset}; \mapitem{\ell}{\tau}}$\vspace{-1.2em}}\right)
    %\eanaX{\auanain{\ell}{\ue}}{\aein{\labelset, \ell}{\ell}{\mapschema{\tau}{i}{\labelset}; \mapitem{\ell}{\tau}}{e}}{\asum{\labelset, \ell}{\mapschema{\tau}{i}{\labelset}; \mapitem{\ell}{\tau}}}
  }
\end{equation}
\begin{equation}\label{rule:cana-match}
  \inferrule{
    \csynX{\ce}{e}{\tau}\\
    \{\cranaX{\crv_i}{r_i}{\tau}{\tau'}\}_{1 \leq i \leq n}
  }{
    \canaX{\acematchwithb{n}{\ce}{\seqschemaX{\crv}}}{\aematchwith{n}{\tau'}{e}{\seqschemaX{r}}}{\tau'}
  }
\end{equation}
\begin{equation}\label{rule:cana-splicede}
\inferrule{
  \parseUExp{\bsubseq{b}{m}{n}}{\ue}\\
  \eana{\uDD{\uD}{\Delta_\text{app}}}{\uGG{\uG}{\Gamma_\text{app}}}{\uPsi}{\uPhi}{\ue}{e}{\tau}\\\\
    \Delta \cap \Delta_\text{app} = \emptyset\\
  \domof{\Gamma} \cap \domof{\Gamma_\text{app}} = \emptyset\\
}{
  \cana{\Delta}{\Gamma}{\esceneUP{\uDD{\uD}{\Delta_\text{app}}}{\uGG{\uG}{\Gamma_\text{app}}}{\uPsi}{\uPhi}{b}}{\acesplicede{m}{n}}{e}{\tau}
}
\end{equation}
\end{subequations}

Rules (\ref{rule:cana-subsume}) through (\ref{rule:cana-match}) are analagous to Rules (\ref{rule:eana-subsume}) through (\ref{rule:eana-match}). Rule (\ref{rule:cana-splicede}) governs references to spliced unexpanded expressions in analytic position. 

\subsubsection{Bidirectional Candidate Expansion Rule Validation}
The \emph{synthetic ce-rule validation judgement} is defined mutually inductively with Rules (\ref{rules:esyn}) by the following rule.
\begin{equation}\label{rule:crsyn}
\inferrule{
  \patType{\pctx}{p}{\tau}\\
  \csyn{\Delta}{\Gcons{\Gamma}{\pctx}}{\escenev}{\ce}{e}{\tau'}
}{
  \crsyn{\Delta}{\Gamma}{\escenev}{\acematchrule{p}{\ce}}{\aematchrule{p}{e}}{\tau}{\tau'}
}
\end{equation}

The \emph{analytic ce-rule validation judgement} is defined mutually inductively with Rules (\ref{rules:eana}) by the following rule.
\begin{equation}\label{rule:crana}
\inferrule{
  \patType{\pctx}{p}{\tau}\\
  \cana{\Delta}{\Gcons{\Gamma}{\pctx}}{\escenev}{\ce}{e}{\tau'}
}{
  \crana{\Delta}{\Gamma}{\escenev}{\acematchrule{p}{\ce}}{\aematchrule{p}{e}}{\tau}{\tau'}
}
\end{equation}

\subsubsection{Candidate Expansion Pattern Validation}
The \emph{ce-pattern validation judgement} is inductively defined by the following rules, which are written identically to Rules (\ref{rules:cvalidP-UP}).
\begin{subequations}\label{rules:cvalidP-B}
\begin{equation}\label{rule:cvalidP-B-wild}
\inferrule{ }{
  \cvalidP{\uGG{\emptyset}{\emptyset}}{\pscenev}{\acewildp}{\aewildp}{\tau}
}
\end{equation}
\begin{equation}\label{rule:cvalidP-B-fold}
\inferrule{
  \cvalidP{\upctx}{\pscenev}{\cpv}{p}{[\arec{t}{\tau}/t]\tau}
}{
  \cvalidP{\upctx}{\pscenev}{\acefoldp{\cpv}}{\aefoldp{p}}{\arec{t}{\tau}}
}
\end{equation}
\begin{equation}\label{rule:cvalidP-B-tpl}
\inferrule{
  \{\cvalidP{\upctx_i}{\pscenev}{\cpv_i}{p_i}{\tau_i}\}_{i \in \labelset}
}{
\left(\shortstack{$\vdash^{\pscenev} \acetplp{\labelset}{\mapschema{\cpv}{i}{\labelset}}$\\$\leadsto$\\$\aetplp{\labelset}{\mapschema{p}{i}{\labelset}} : \aprod{\labelset}{\mapschema{\tau}{i}{\labelset}}~\dashVx^{\,\Gconsi{i \in \labelset}{\upctx_i}}$\vspace{-1.2em}}\right)
}
\end{equation}
\begin{equation}\label{rule:cvalidP-B-in}
\inferrule{
  \cvalidP{\upctx}{\pscenev}{\cpv}{p}{\tau}
}{
  \cvalidP{\upctx}{\pscenev}{\aceinjp{\ell}{\cpv}}{\aeinjp{\ell}{p}}{\asum{\labelset, \ell}{\mapschema{\tau}{i}{\labelset}; \mapitem{\ell}{\tau}}}
}
\end{equation}
\begin{equation}\label{rule:cvalidP-B-spliced}
\inferrule{
  \parseUPat{\bsubseq{b}{m}{n}}{\upv}\\
  \patExpands{\upctx}{\uPhi}{\upv}{p}{\tau}
}{
  \cvalidP{\upctx}{\pscene{\Delta}{\uPhi}{b}}{\acesplicedp{m}{n}}{p}{\tau}
}
\end{equation}
\end{subequations}

\subsubsection{Candidate Expansion Expressibility}
The following lemma establishes that each well-typed expanded expression, $e$, can be expressed as a valid ce-expression, $\Cof{e}$, that synthesizes the same type under the same contexts and any expression splicing scene.
\begin{theorem}[Candidate Expansion Expression Expressibility]\label{lemma:ce-expressions-expressibility-B} Both of the following hold:
\begin{enumerate}
\item If $\hastypeU{\Delta}{\Gamma}{e}{\tau}$ then $\csyn{\Delta}{\Gamma}{\escenev}{\Cof{e}}{e}{\tau}$.
\item If $\ruleType{\Delta}{\Gamma}{r}{\tau}{\tau'}$ then $\crsyn{\Delta}{\Gamma}{\escenev}{\Cof{r}}{r}{\tau}{\tau'}$.
\end{enumerate}
\end{theorem}
\begin{proof} By mutual rule induction over Rules (\ref{rules:hastypeUP}) and Rule (\ref{rule:ruleType}). In each case, we apply the IH, part 1 to or over each typing premise, the IH, part 2 over each rule typing premise, Lemma \ref{lemma:ce-type-expressibility-U} to or over each type formation premise and then derive the conclusion by applying Rules (\ref{rules:csyn}) and Rule (\ref{rule:crsyn}) as needed.
\end{proof}

The following lemma establishes that every well-typed expanded pattern that generates no hypotheses can be expressed as a ce-pattern.
\begin{lemma}[Candidate Expansion Pattern Expressibility]\label{lemma:ce-pattern-expressibility-B} If $\patType{\emptyset}{p}{\tau}$ then $\cvalidP{\uGG{\emptyset}{\emptyset}}{\pscene{\Delta}{\uPhi}{b}}{\Cof{p}}{p}{\tau}$.\end{lemma}
\begin{proof} The proof is nearly identical to the proof of Lemma \ref{lemma:ce-pattern-expressibility-U}, differing only in that each mention of a rule in Rules (\ref{rules:cvalidP-UP}) is replaced by a mention of the corresponding rule in Rules (\ref{rules:cvalidP-B}).
\end{proof}

\subsection{Metatheory}
The following theorem establishes that typed pattern expansion produces an expanded pattern that matches values of the specified type and generates the same hypotheses. It must be stated mutually with the corresponding theorem about candidate expansion patterns, because the judgements are mutually defined.
\begin{theorem}[Typed Pattern Expansion]\label{thm:typed-pattern-expansion-B} Both of the following hold:
\begin{enumerate}
  \item If $\patExpands{\uGG{\uG}{\pctx}}{\uASI{\uA}{\Phi}{\uI}}{\upv}{p}{\tau}$ then $\patType{\pctx}{p}{\tau}$.
  \item If $\cvalidP{\uGG{\uG}{\pctx}}{\pscene{\Delta}{\uASI{\uA}{\Phi}{\uI}}{b}}{\cpv}{p}{\tau}$ then $\patType{\pctx}{p}{\tau}$.
\end{enumerate}
\end{theorem}
\begin{proof} My mutual rule induction over Rules (\ref{rules:patExpands-B}) and Rules (\ref{rules:cvalidP-B}).
\begin{enumerate}
\item We induct on the premise. In the following, let $\upctx=\uGG{\uG}{\pctx}$ and $\uPhi=\uASI{\uA}{\Phi}{\uI}$.
  \begin{byCases}
    \item[\text{(\ref{rule:patExpands-B-var}) through (\ref{rule:patExpands-B-apuptsm})}] In each of these cases, the proof is written  identically to the proof of the corresponding case in the proof of Theorem \ref{thm:typed-pattern-expansion}.
    \item[\text{(\ref{rule:patExpands-B-lit})}] We have:
      \begin{pfsteps*}
         \item $\upv=\auplit{b}$ \BY{assumption}
         \item $\Phi=\Phi', \xuptsmbnd{a}{\tau}{\eparse}$ \BY{assumption}
         \item $\uI=\uI', \designate{\tau}{a}$ \BY{assumption}
         \item $\encodeBody{b}{\ebody}$ \BY{assumption}
         \item $\evalU{\eparse(\ebody)}{\inj{\lbltxt{Success}}{\ecand}}$ \BY{assumption}
         \item $\decodeCEPat{\ecand}{\cpv}$ \BY{assumption}
         \item $\cvalidP{\uGG{\uG}{\pctx}}{\pscene{\Delta}{\uASI{\uA}{\Phi', \xuptsmbnd{a}{\tau}{\eparse}}{\uI', \designate{\tau}{a}}}{b}}{\cpv}{p}{\tau}$ \BY{assumption} \pflabel{cvalidP}
%        \item $\uptsmenv{\Delta}{\Phi', \xuptsmbnd{a}{\tau}{\eparse}}$ \BY{assumption} \pflabel{env}
         \item $\patType{\pctx}{p}{\tau}$ \BY{IH, part 2 on \pfref{cvalidP}}
       \end{pfsteps*} 
       \resetpfcounter
  \end{byCases}
\item We induct on the premise. In the following, let $\upctx=\uGG{\uG}{\pctx}$ and $\uPhi=\uASI{\uA}{\Phi}{\uI}$.
  \begin{byCases}
    \item[\text{(\ref{rule:cvalidP-B-wild}) through (\ref{rule:cvalidP-B-spliced})}] In each case, the proof is written identically to the proof of the corresponding case in the proof of Theorem \ref{thm:typed-pattern-expansion}.
  \end{byCases}
\end{enumerate}
The mutual induction can be shown to be well-founded by showing that the following numeric metric on the judgements that we induct on is decreasing:
\begin{align*}
\sizeof{\patExpands{\upctx}{\uPhi}{\upv}{p}{\tau}} & = \sizeof{\upv}\\
\sizeof{{\cvalidP{\upctx}{\pscene{\Delta}{\uPhi}{b}}{\cpv}{p}{\tau}}} & = \sizeof{b}
\end{align*}
where $\sizeof{b}$ is the length of $b$ and $\sizeof{\upv}$ is the sum of the lengths of the literal bodies in $\upv$,
\begin{align*}
\sizeof{\ux} & = 0\\
\sizeof{\aufoldp{\upv}} & = \sizeof{\upv}\\
\sizeof{\autplp{\labelset}{\mapschema{\upv}{i}{\labelset}}} & = \sum_{i \in \labelset} \sizeof{\upv_i}\\
\sizeof{\auinjp{\ell}{\upv}} & = \sizeof{\upv}\\
\sizeof{\auapuptsm{b}{\tsmv}} & = \sizeof{b}\\
\sizeof{\auplit{b}} & = \sizeof{b}
\end{align*}

The only case in the proof of part 1 that invokes part 2 are Case (\ref{rule:patExpands-B-apuptsm}) and (\ref{rule:patExpands-B-lit}). There, we have that the metric remains stable: \begin{align*}
 & \sizeof{\patExpands{\upctx}{\uPhi, \uShyp{\tsmv}{a}{\tau}{\eparse}}{\auapuptsm{b}{\tsmv}}{p}{\tau}}\\
=& \sizeof{\patExpands{\upctx}{\uASI{\uA}{\Phi', \xuptsmbnd{a}{\tau}{\eparse}}{\uI', \designate{\tau}{a}}}{\auplit{b}}{p}{\tau}}\\
=& \sizeof{{\cvalidP{\upctx}{\pscene{\Delta}{\uPhi, \uShyp{\tsmv}{a}{\tau}{\eparse}}{b}}{\cpv}{p}{\tau}}}\\
=&\sizeof{b}\end{align*}

The only case in the proof of part 2 that invokes part 1 is Case (\ref{rule:cvalidP-B-spliced}). There, we have that $\parseUPat{\bsubseq{b}{m}{n}}{\upv}$ and the IH is applied to the judgement $\patExpands{\upctx}{\uPhi}{\upv}{p}{\tau}$. Because the metric is stable when passing from part 1 to part 2, we must have that it is strictly decreasing in the other direction:
\[\sizeof{\patExpands{\upctx}{\uPhi}{\upv}{p}{\tau}} < \sizeof{{\cvalidP{\upctx}{\pscene{\Delta}{\uPhi}{b}}{\acesplicedp{m}{n}}{p}{\tau}}}\]
i.e. by the definitions above, 
\[\sizeof{\upv} < \sizeof{b}\]

This is established by appeal to Condition \ref{condition:body-subsequences}, which states that subsequences of $b$ are no longer than $b$, and the following condition, which states that an unexpanded pattern constructed by parsing a textual sequence $b$ is strictly smaller, as measured by the metric defined above, than the length of $b$, because some characters must necessarily be used to delimit each literal body.
\begin{condition}[Pattern Parsing Monotonicity]\label{condition:pattern-parsing-B} If $\parseUPat{b}{\upv}$ then $\sizeof{\upv} < \sizeof{b}$.\end{condition}

Combining Conditions \ref{condition:body-subsequences} and \ref{condition:pattern-parsing-B}, we have that $\sizeof{\ue} < \sizeof{b}$ as needed.
\end{proof}

Finally, the following theorem establishes that bidirectionally typed expression and rule expansion produces expanded expressions and rules of the appropriate type under the appropriate contexts. These statements must be stated mutually with the corresponding statements about birectional ce-expression and ce-rule validation because the judgements are mutually defined. 

\begin{theorem}[Typed Expansion] Letting $\uPsi=\uASI{\uA}{\Psi}{\uI}$, if $\uetsmenv{\Delta}{\Psi}$ then all of the following hold:
\begin{enumerate}
  \item \begin{enumerate}
    \item \begin{enumerate}
      \item If $\esyn{\uDD{\uD}{\Delta}}{\uGG{\uG}{\Gamma}}{\uPsi}{\uPhi}{\ue}{e}{\tau}$ then $\hastypeU{\Delta}{\Gamma}{e}{\tau}$.
      \item If $\rsyn{\uDD{\uD}{\Delta}}{\uGG{\uG}{\Gamma}}{\uPsi}{\uPhi}{\urv}{r}{\tau}{\tau'}$  then $\ruleType{\Delta}{\Gamma}{r}{\tau}{\tau'}$.
    \end{enumerate}
    \item \begin{enumerate}
      \item If $\eana{\uDD{\uD}{\Delta}}{\uGG{\uG}{\Gamma}}{\uPsi}{\uPhi}{\ue}{e}{\tau}$ and $\istypeU{\Delta}{\tau}$ then $\hastypeU{\Delta}{\Gamma}{e}{\tau}$.
      \item If $\rana{\uDD{\uD}{\Delta}}{\uGG{\uG}{\Gamma}}{\uPsi}{\uPhi}{\urv}{r}{\tau}{\tau'}$ and $\istypeU{\Delta}{\tau'}$ then $\ruleType{\Delta}{\Gamma}{r}{\tau}{\tau'}$.
    \end{enumerate}
  \end{enumerate}
  \item \begin{enumerate}
    \item \begin{enumerate}
      \item If $\csyn{\Delta}{\Gamma}{\esceneUP{\uDD{\uD}{\Delta_\text{app}}}{\uGG{\uG}{\Gamma_\text{app}}}{\uPsi}{\uPhi}{b}}{\ce}{e}{\tau}$ and $\Delta \cap \Delta_\text{app}=\emptyset$ and $\domof{\Gamma} \cap \domof{\Gamma_\text{app}}=\emptyset$ then $\hastypeU{\Dcons{\Delta}{\Delta_\text{app}}}{\Gcons{\Gamma}{\Gamma_\text{app}}}{e}{\tau}$. 
      \item If $\crsyn{\Delta}{\Gamma}{\esceneUP{\uDD{\uD}{\Delta_\text{app}}}{\uGG{\uG}{\Gamma_\text{app}}}{\uPsi}{\uPhi}{b}}{\crv}{r}{\tau}{\tau'}$ and $\Delta \cap \Delta_\text{app}=\emptyset$ and $\domof{\Gamma} \cap \domof{\Gamma_\text{app}}=\emptyset$ then $\ruleType{\Dcons{\Delta}{\Delta_\text{app}}}{\Gcons{\Gamma}{\Gamma_\text{app}}}{r}{\tau}{\tau'}$.
    \end{enumerate}
    \item \begin{enumerate}
      \item If $\cana{\Delta}{\Gamma}{\esceneUP{\uDD{\uD}{\Delta_\text{app}}}{\uGG{\uG}{\Gamma_\text{app}}}{\uPsi}{\uPhi}{b}}{\ce}{e}{\tau}$ and $\Delta \cap \Delta_\text{app}=\emptyset$ and $\domof{\Gamma} \cap \domof{\Gamma_\text{app}}=\emptyset$ and $\istypeU{\Dcons{\Delta}{\Delta_\text{app}}}{\tau}$ then $\hastypeU{\Dcons{\Delta}{\Delta_\text{app}}}{\Gcons{\Gamma}{\Gamma_\text{app}}}{e}{\tau}$. 
      \item If $\crana{\Delta}{\Gamma}{\esceneUP{\uDD{\uD}{\Delta_\text{app}}}{\uGG{\uG}{\Gamma_\text{app}}}{\uPsi}{\uPhi}{b}}{\crv}{r}{\tau}{\tau'}$ and $\Delta \cap \Delta_\text{app}=\emptyset$ and $\domof{\Gamma} \cap \domof{\Gamma_\text{app}}=\emptyset$ and $\istypeU{\Dcons{\Delta}{\Delta_\text{app}}}{\tau'}$ then $\ruleType{\Dcons{\Delta}{\Delta_\text{app}}}{\Gcons{\Gamma}{\Gamma_\text{app}}}{r}{\tau}{\tau'}$.
    \end{enumerate}
  \end{enumerate}
\end{enumerate}
\end{theorem}
\begin{proof} By mutual rule induction over Rules (\ref{rules:esyn}), Rules (\ref{rules:eana}), Rule (\ref{rule:rsyn}), Rule (\ref{rule:rana}), Rules (\ref{rules:csyn}), Rules (\ref{rules:cana}), Rule (\ref{rule:crsyn}) and Rule (\ref{rule:crana}). In the following, we refer to the induction hypothesis applied to the assumption $\uetsmenv{\Delta}{\Psi}$ as simply the ``IH''. When we apply the induction hypothesis to a different argument, we refer to it as the ``Outer IH''.

\begin{enumerate}
  \item In the following, let $\uDelta=\uDD{\uD}{\Delta}$ and $\uGamma=\uGG{\uG}{\Gamma}$. We have:
  \begin{enumerate}
    \item \begin{enumerate}
      \item We induct on the assumption.
        \begin{byCases}
          \item[\text{(\ref{rule:esyn-var})}] We have:
            \begin{pfsteps*}
              \item $e=x$ \BY{assumption}
              \item $\Gamma=\Gamma', \Ghyp{x}{\tau}$ \BY{assumption}
              \item $\hastypeU{\Delta}{\Gamma', \Ghyp{x}{\tau}}{x}{\tau}$ \BY{Rule (\ref{rule:hastypeUP-var})}
            \end{pfsteps*}
            \resetpfcounter
          \item[\text{(\ref{rule:esyn-asc})}] We have:
            \begin{pfsteps*}
               \item $\ue=\auasc{\utau}{\ue'}$ \BY{assumption}
               \item $\expandsTU{\uDelta}{\utau}{\tau}$ \BY{assumption}\pflabel{expandsTU}
               \item $\eanaX{\ue'}{e}{\tau}$ \BY{assumption}\pflabel{eanaX}
               \item $\istypeU{\Delta}{\tau}$ \BY{Lemma \ref{lemma:type-expansion-U} on \pfref{expandsTU}}\pflabel{istype}
               \item $\hastypeU{\Delta}{\Gamma}{e}{\tau}$ \BY{IH, part 1(b)(i) to \pfref{eanaX} and \pfref{istype}}
             \end{pfsteps*}
             \resetpfcounter
          \item[\text{(\ref{rule:esyn-let}) through (\ref{rule:esyn-match})}] In each of these cases, we apply:
            \begin{itemize}
              \item Lemma \ref{lemma:type-expansion-U} to or over all type expansion premises.
              \item The IH, part 1(a)(i) to or over all synthetic typed expression expansion premises.
              \item The IH, part 1(a)(ii) to or over all synthetic rule expansion premises.
              \item The IH, part 1(b)(i) to or over all analytic typed expression expansion premises.
            \end{itemize}
            We then derive the conclusion by applying Rules (\ref{rules:hastypeUP}) and Rule (\ref{rule:ruleType}) as needed.
          \item[\text{(\ref{rule:esyn-defuetsm})}] We have:
            \begin{pfsteps*}
              \item $\ue=\audefuetsm{\utau'}{\eparse}{\tsmv}{\ue'}$ \BY{assumption}
              \item $\expandsTU{\uDelta}{\utau'}{\tau'}$ \BY{assumption} \pflabel{expandsTU}
              \item $\hastypeU{\emptyset}{\emptyset}{\eparse}{\aparr{\tBody}{\tParseResultExp}}$ \BY{assumption}\pflabel{eparse}
              \item $\esyn{\uDelta}{\uGamma}{\uASI{\ctxUpdate{\uA}{\tsmv}{a}}{\Psi, \xuetsmbnd{a}{\tau'}{\eparse}}{\uI}}{\uPhi}{\ue'}{e}{\tau}$ \BY{assumption}\pflabel{expandsU}
              \item $\uetsmenv{\Delta}{\Psi}$ \BY{assumption}\pflabel{uetsmenv1}
              \item $\istypeU{\Delta}{\tau'}$ \BY{Lemma \ref{lemma:type-expansion-U} to \pfref{expandsTU}} \pflabel{istype}
              \item $\uetsmenv{\Delta}{\Psi, \xuetsmbnd{\tsmv}{\tau'}{\eparse}}$ \BY{Definition \ref{def:ueTSM-def-ctx-formation-UP} on \pfref{uetsmenv1}, \pfref{istype} and \pfref{eparse}}\pflabel{uetsmenv3}
              \item $\hastypeU{\Delta}{\Gamma}{e}{\tau}$ \BY{Outer IH, part 1(a)(i) on \pfref{uetsmenv3} and \pfref{expandsU}}
            \end{pfsteps*}
            \resetpfcounter
          \item[\text{(\ref{rule:esyn-apuetsm})}] We have:
            \begin{pfsteps*}
              \item $\ue=\autsmap{b}{\tsmv}$ \BY{assumption}
              \item $\uPsi = \uASI{\ctxUpdate{\uA'}{\tsmv}{a}}{\Psi', \xuetsmbnd{a}{\tau}{\eparse}}{\uI}$ \BY{assumption}
              \item $\encodeBody{b}{\ebody}$ \BY{assumption}
              \item $\evalU{\eparse(\ebody)}{\inj{\lbltxt{Success}}{\ecand}}$ \BY{assumption}
              \item $\decodeCondE{\ecand}{\ce}$ \BY{assumption}
              \item $\cana{\emptyset}{\emptyset}{\esceneUP{\uDelta}{\uGamma}{\uPsi}{\uPhi}{b}}{\ce}{e}{\tau}$ \BY{assumption}\pflabel{cvalidE}
              \item $\uetsmenv{\Delta}{\Psi}$ \BY{assumption} \pflabel{uetsmenv}
              \item $\istypeU{\Delta}{\tau}$ \BY{Definition \ref{def:ueTSM-def-ctx-formation-UP} on \pfref{uetsmenv}} \pflabel{istype}
              \item $\emptyset \cap \Delta = \emptyset$ \BY{finite set intersection identity} \pflabel{delta-cap}
              \item ${\emptyset} \cap \domof{\Gamma} = \emptyset$ \BY{finite set intersection identity} \pflabel{gamma-cap}
              \item $\hastypeU{\emptyset \cup \Delta}{\emptyset \cup \Gamma}{e}{\tau}$ \BY{IH, part 2(a)(i) on \pfref{cvalidE}, \pfref{delta-cap}, \pfref{gamma-cap} and \pfref{istype}} \pflabel{penultimate}
              \item $\hastypeU{\Delta}{\Gamma}{e}{\tau}$ \BY{definition of finite set and finite function union over \pfref{penultimate}}               
             \end{pfsteps*} 
             \resetpfcounter
          \item[\text{(\ref{rule:esyn-implicite})}] We have:
            \begin{pfsteps*}
              \item $\ue=\auimplicite{\tsmv}{\ue}$ \BY{assumption}
              \item $\uPsi=\uASI{\uA' \uplus \vExpands{\tsmv}{a}}{\Psi', \xuetsmbnd{a}{\tau'}{\eparse}}{\uI}$ \BY{assumption}
              \item $\esyn{\uDelta}{\uGamma}{\uASI{\uA' \uplus \vExpands{\tsmv}{a}}{\Psi', \xuetsmbnd{a}{\tau'}{\eparse}}{\uI \uplus \designate{\tau}{a}}}{\uPhi}{\ue}{e}{\tau}$ \BY{assumption} \pflabel{esyn}
              \item $\hastypeU{\Delta}{\Gamma}{e}{\tau}$ \BY{IH, part 1(a)(i) on \pfref{esyn}}
            \end{pfsteps*}
            \resetpfcounter
          \item[\text{(\ref{rule:esyn-defuptsm})}] We have:
            \begin{pfsteps*}
              \item $\ue=\audefuptsm{\utau'}{\eparse}{\tsmv}{\ue'}$ \BY{assumption}
              \item $\expandsTU{\uDelta}{\utau'}{\tau'}$ \BY{assumption} \pflabel{expandsTU}
            %  \item \hastypeU{\emptyset}{\emptyset}{\eparse}{\aparr{\tBody}{\tParseResultExp}} \BY{assumption}\pflabel{eparse}
              \item $\esyn{\uDelta}{\uGamma}{\uPsi}{\uPhi, \uPhyp{\tsmv}{a}{\tau'}{\eparse}}{\ue'}{e}{\tau}$ \BY{assumption}\pflabel{expandsU}
            %  \item \uetsmenv{\Delta}{\Psi} \BY{assumption}\pflabel{uetsmenv1}
            %  \item \istypeU{\Delta}{\tau'} \BY{Lemma \ref{lemma:type-expansion-U} to \pfref{expandsTU}} \pflabel{istype}
            %  \item \uetsmenv{\Delta}{\Psi, \xuetsmbnd{\tsmv}{\tau'}{\eparse}} \BY{Definition \ref{def:ueTSM-def-ctx-formation} on \pfref{uetsmenv1}, \pfref{istype} and \pfref{eparse}}\pflabel{uetsmenv3}
              \item $\hastypeU{\Delta}{\Gamma}{e}{\tau}$ \BY{IH, part 1(a)(i) on \pfref{expandsU}}
            \end{pfsteps*}
            \resetpfcounter
          \item[\text{(\ref{rule:esyn-implicitp})}] We have:
            \begin{pfsteps*}
              \item $\ue=\auimplicitp{\tsmv}{\ue}$ \BY{assumption}
              \item $\uPhi=\uASI{\uA \uplus \vExpands{\tsmv}{a}}{\Phi, \xuptsmbnd{a}{\tau'}{\eparse}}{\uI}$ \BY{assumption}
              \item $\esyn{\uDelta}{\uGamma}{\uPsi}{\uASI{\uA \uplus \vExpands{\tsmv}{a}}{\Phi, \xuptsmbnd{a}{\tau'}{\eparse}}{\uI \uplus \designate{\tau}{a}}}{\ue}{e}{\tau}$ \BY{assumption} \pflabel{esyn}
              \item $\hastypeU{\Delta}{\Gamma}{e}{\tau}$ \BY{IH, part 1(a)(i) on \pfref{esyn}}
            \end{pfsteps*}
            \resetpfcounter
        \end{byCases}
      \item We induct on the assumption. There is one case.
        \begin{byCases}
          \item[\text{(\ref{rule:rsyn})}] We have:
            \begin{pfsteps*}
              \item $\urv=\aumatchrule{\upv}{\ue}$ \BY{assumption}
              \item $r=\aematchrule{p}{e}$ \BY{assumption}
              \item $\patExpands{\uGG{\uA'}{\pctx}}{\uPhi}{\upv}{p}{\tau}$ \BY{assumption} \pflabel{patExpands}
              \item $\esyn{\uDelta}{\uGG{{\uA}\uplus{\uA'}}{\Gcons{\Gamma}{\pctx}}}{\uPsi}{\uPhi}{\ue}{e}{\tau'}$ \BY{assumption} \pflabel{expandsUP}
              \item $\patType{\pctx}{p}{\tau}$ \BY{Theorem \ref{thm:typed-pattern-expansion-B}, part 1 on \pfref{patExpands}}\pflabel{patType}
              \item $\hastypeU{\Delta}{\Gcons{\Gamma}{\pctx}}{e}{\tau'}$ \BY{IH, part 1(a)(i) on \pfref{expandsUP}} \pflabel{hasType}
              \item $\ruleType{\Delta}{\Gamma}{\aematchrule{p}{e}}{\tau}{\tau'}$ \BY{Rule (\ref{rule:ruleType}) on \pfref{patType} and \pfref{hasType}}
            \end{pfsteps*}
            \resetpfcounter
        \end{byCases}
    \end{enumerate}
    \item \begin{enumerate}
      \item We induct on the assumption.
        \begin{byCases}
          \item[\text{(\ref{rule:eana-subsume})}] We have:
            \begin{pfsteps*}
              \item $\esynX{\ue}{e}{\tau}$ \BY{assumption} \pflabel{esyn}
              \item $\hastypeU{\Delta}{\Gamma}{e}{\tau}$ \BY{IH, part 1(a)(i) on \pfref{esyn}}
            \end{pfsteps*}
          \item[\text{(\ref{rule:eana-let}) through (\ref{rule:eana-match})}] In each of these cases, we apply:
            \begin{itemize}
              \item Lemma \ref{lemma:type-expansion-U} to or over all type expansion premises.
              \item The IH, part 1(a)(i) to or over all synthetic typed expression expansion premises.
              \item The IH, part 1(a)(ii) to or over all synthetic rule expansion premises.
              \item The IH, part 1(b)(i) to or over all analytic typed expression expansion premises.
            \end{itemize}
            We then derive the conclusion by applying Rules (\ref{rules:hastypeUP}) and Rule (\ref{rule:ruleType}) as needed. 
          \item[\text{(\ref{rule:eana-defuetsm})}] We have:
            \begin{pfsteps*}
              \item $\ue=\audefuetsm{\utau'}{\eparse}{\tsmv}{\ue'}$ \BY{assumption}
              \item $\expandsTU{\uDelta}{\utau'}{\tau'}$ \BY{assumption} \pflabel{expandsTU}
              \item $,$ \BY{assumption}\pflabel{eparse}
              \item $\eana{\uDelta}{\uGamma}{\uPsi, \uShyp{\tsmv}{a}{\tau'}{\eparse}}{\uPhi}{\ue'}{e}{\tau}$ \BY{assumption}\pflabel{expandsU}
              \item $\uetsmenv{\Delta}{\Psi}$ \BY{assumption}\pflabel{uetsmenv1}
              \item $\istypeU{\Delta}{\tau'}$ \BY{Lemma \ref{lemma:type-expansion-U} to \pfref{expandsTU}} \pflabel{istype}
              \item $\uetsmenv{\Delta}{\Psi, \xuetsmbnd{\tsmv}{\tau'}{\eparse}}$ \BY{Definition \ref{def:ueTSM-def-ctx-formation-UP} on \pfref{uetsmenv1}, \pfref{istype} and \pfref{eparse}}\pflabel{uetsmenv3}
            %  \item \uetsmenv{\Delta}{\Psi} \BY{assumption}\pflabel{uetsmenv1}
            %  \item \istypeU{\Delta}{\tau'} \BY{Lemma \ref{lemma:type-expansion-U} to \pfref{expandsTU}} \pflabel{istype}
            %  \item \uetsmenv{\Delta}{\Psi, \xuetsmbnd{\tsmv}{\tau'}{\eparse}} \BY{Definition \ref{def:ueTSM-def-ctx-formation} on \pfref{uetsmenv1}, \pfref{istype} and \pfref{eparse}}\pflabel{uetsmenv3}
              \item $\hastypeU{\Delta}{\Gamma}{e}{\tau}$ \BY{IH, part 1(b)(i) on \pfref{expandsU}}
            \end{pfsteps*}
            \resetpfcounter
          \item[\text{(\ref{rule:eana-implicite})}] We have:
            \begin{pfsteps*}
              \item $\ue=\autsmap{b}{\tsmv}$ \BY{assumption}
              \item $\uPsi = \uPsi', \uShyp{\tsmv}{a}{\tau}{\eparse}$ \BY{assumption}
              \item $\encodeBody{b}{\ebody}$ \BY{assumption}
              \item $\evalU{\eparse(\ebody)}{\inj{\lbltxt{Success}}{\ecand}}$ \BY{assumption}
              \item $\decodeCondE{\ecand}{\ce}$ \BY{assumption}
              \item $\cana{\emptyset}{\emptyset}{\esceneUP{\uDelta}{\uGamma}{\uPsi}{\uPhi}{b}}{\ce}{e}{\tau}$ \BY{assumption}\pflabel{cvalidE}
            %  \item \uetsmenv{\Delta}{\Psi} \BY{assumption} \pflabel{uetsmenv}
              \item $\emptyset \cap \Delta = \emptyset$ \BY{finite set intersection identity} \pflabel{delta-cap}
              \item ${\emptyset} \cap \domof{\Gamma} = \emptyset$ \BY{finite set intersection identity} \pflabel{gamma-cap}
              \item $\hastypeU{\emptyset \cup \Delta}{\emptyset \cup \Gamma}{e}{\tau}$ \BY{IH, part 2(b)(i) on \pfref{cvalidE}, \pfref{delta-cap}, and \pfref{gamma-cap}} \pflabel{penultimate}
              \item $\hastypeU{\Delta}{\Gamma}{e}{\tau}$ \BY{definition of finite set union over \pfref{penultimate}}               
             \end{pfsteps*} 
             \resetpfcounter
          \item[\text{(\ref{rule:eana-lit})}] We have:
            \begin{pfsteps*}
              \item $\ue=\auelit{b}$ \BY{assumption}
              \item $\uPsi=\uASI{\uA}{\Psi, \xuetsmbnd{a}{\tau}{\eparse}}{\uI \uplus \designate{\tau}{a}}$ \BY{assumption}
              \item $\encodeBody{b}{\ebody}$ \BY{assumption}
              \item $\evalU{\ap{\eparse}{\ebody}}{\inj{\lbltxt{Success}}{\ecand}}$ \BY{assumption}
              \item $\decodeCondE{\ecand}{\ce}$ \BY{assumption}
              \item $\cana{\emptyset}{\emptyset}{\esceneUP{\uDelta}{\uGamma}{\uASI{\uA}{\Psi, \xuetsmbnd{a}{\tau}{\eparse}}{\uI \uplus \designate{\tau}{a}}}{\uPhi}{b}}{\ce}{e}{\tau}$ \BY{assumption} \pflabel{cvalidE}
              \item $\emptyset \cap \Delta = \emptyset$ \BY{finite set intersection identity} \pflabel{delta-cap}
              \item ${\emptyset} \cap \domof{\Gamma} = \emptyset$ \BY{finite set intersection identity} \pflabel{gamma-cap}
              \item $\hastypeU{\emptyset \cup \Delta}{\emptyset \cup \Gamma}{e}{\tau}$ \BY{IH, part 2(a)(i) on \pfref{cvalidE}, \pfref{delta-cap}, and \pfref{gamma-cap}} \pflabel{penultimate}
              \item $\hastypeU{\Delta}{\Gamma}{e}{\tau}$ \BY{definition of finite set union over \pfref{penultimate}}
            \end{pfsteps*}
            \resetpfcounter
          \item[\text{(\ref{rule:eana-defuptsm})}] We have:
            \begin{pfsteps*}
              \item $\ue=\audefuptsm{\utau'}{\eparse}{\tsmv}{\ue'}$ \BY{assumption}
              \item $\expandsTU{\uDelta}{\utau'}{\tau'}$ \BY{assumption} \pflabel{expandsTU}
            %  \item \hastypeU{\emptyset}{\emptyset}{\eparse}{\aparr{\tBody}{\tParseResultExp}} \BY{assumption}\pflabel{eparse}
              \item $\eana{\uDelta}{\uGamma}{\uPsi}{\uPhi, \uPhyp{\tsmv}{a}{\tau'}{\eparse}}{\ue'}{e}{\tau}$ \BY{assumption}\pflabel{expandsU}
            %  \item \uetsmenv{\Delta}{\Psi} \BY{assumption}\pflabel{uetsmenv1}
            %  \item \istypeU{\Delta}{\tau'} \BY{Lemma \ref{lemma:type-expansion-U} to \pfref{expandsTU}} \pflabel{istype}
            %  \item \uetsmenv{\Delta}{\Psi, \xuetsmbnd{\tsmv}{\tau'}{\eparse}} \BY{Definition \ref{def:ueTSM-def-ctx-formation} on \pfref{uetsmenv1}, \pfref{istype} and \pfref{eparse}}\pflabel{uetsmenv3}
              \item $\hastypeU{\Delta}{\Gamma}{e}{\tau}$ \BY{IH, part 1(b)(i) on \pfref{expandsU}}
            \end{pfsteps*}
            \resetpfcounter
          \item[\text{(\ref{rule:eana-implicitp})}] We have:
            \begin{pfsteps*}
              \item $\ue=\auimplicitp{\tsmv}{\ue}$ \BY{assumption}
              \item $\uPhi=\uASI{\uA \uplus \vExpands{\tsmv}{a}}{\Phi, \xuptsmbnd{a}{\tau'}{\eparse}}{\uI}$ \BY{assumption}
              \item $\eana{\uDelta}{\uGamma}{\uPsi}{\uASI{\uA \uplus \vExpands{\tsmv}{a}}{\Phi, \xuptsmbnd{a}{\tau'}{\eparse}}{\uI \uplus \designate{\tau}{a}}}{\ue}{e}{\tau}$ \BY{assumption} \pflabel{esyn}
              \item $\hastypeU{\Delta}{\Gamma}{e}{\tau}$ \BY{IH, part 1(b)(i) on \pfref{esyn}}
            \end{pfsteps*}
            \resetpfcounter
        \end{byCases}
      \item We induct on the assumption. There is one case.
        \begin{byCases}
          \item[\text{(\ref{rule:rana})}] We have:
            \begin{pfsteps*}
              \item $\urv=\aumatchrule{\upv}{\ue}$ \BY{assumption}
              \item $r=\aematchrule{p}{e}$ \BY{assumption}
              \item $\patExpands{\uGG{\uA'}{\pctx}}{\uPhi}{\upv}{p}{\tau}$ \BY{assumption} \pflabel{patExpands}
              \item $\eana{\uDelta}{\uGG{{\uA}\uplus{\uA'}}{\Gcons{\Gamma}{\pctx}}}{\uPsi}{\uPhi}{\ue}{e}{\tau'}$ \BY{assumption} \pflabel{expandsUP}
              \item $\patType{\pctx}{p}{\tau}$ \BY{Theorem \ref{thm:typed-pattern-expansion-B}, part 1 on \pfref{patExpands}}\pflabel{patType}
              \item $\hastypeU{\Delta}{\Gcons{\Gamma}{\pctx}}{e}{\tau'}$ \BY{IH, part 1(b)(i) on \pfref{expandsUP}} \pflabel{hasType}
              \item $\ruleType{\Delta}{\Gamma}{\aematchrule{p}{e}}{\tau}{\tau'}$ \BY{Rule (\ref{rule:ruleType}) on \pfref{patType} and \pfref{hasType}}
            \end{pfsteps*}
            \resetpfcounter
        \end{byCases}
    \end{enumerate}
  \end{enumerate}
  \item In the following, let $\uDelta=\uDD{\uD}{\Delta_\text{app}}$ and $\uGamma=\uGG{\uG}{\Gamma_\text{app}}$ and $\escenev=\esceneUP{\uDelta}{\uGamma}{\uPsi}{\uPhi}{b}$.
  \begin{enumerate}
    \item \begin{enumerate}
      \item We induct on the assumption.
        \begin{byCases}
          \item[\text{(\ref{rule:csyn-var})}] We have:
            \begin{pfsteps*}
              \item $e=x$ \BY{assumption}
              \item $\Gamma=\Gamma', \Ghyp{x}{\tau}$ \BY{assumption}
              \item $\hastypeU{\Delta}{\Gamma', \Ghyp{x}{\tau}}{x}{\tau}$ \BY{Rule (\ref{rule:hastypeUP-var})}
            \end{pfsteps*}
            \resetpfcounter 
          \item[\text{(\ref{rule:csyn-asc})}] We have:
            \begin{pfsteps*}
               \item $\ce=\aceasc{\ctau}{\ce'}$ \BY{assumption}
               \item $\Delta \cap \Delta_\text{app}=\emptyset$ \BY{assumption} \pflabel{delta-disjoint}
               \item $\domof{\Gamma} \cap \domof{\Gamma_\text{app}}=\emptyset$ \BY{assumption} \pflabel{gamma-disjoint}
               \item $\cvalidT{\Delta}{\tsfrom{\escenev}}{\ctau}{\tau}$ \BY{assumption}\pflabel{expandsTU}
               \item $\canaX{\ce'}{e}{\tau}$ \BY{assumption}\pflabel{eanaX}
               \item $\istypeU{\Delta \cup \Delta_\text{app}}{\tau}$ \BY{Lemma \ref{lemma:candidate-expansion-type-validation} on \pfref{expandsTU}}\pflabel{istype}
               \item $\hastypeU{\Delta}{\Gamma}{e}{\tau}$ \BY{IH, part 2(b)(i) to \pfref{eanaX}, \pfref{delta-disjoint}, \pfref{gamma-disjoint} and  \pfref{istype}}
             \end{pfsteps*}
             \resetpfcounter
          \item[\text{(\ref{rule:csyn-let}) through (\ref{rule:csyn-match})}] In each of these cases, we apply:
            \begin{itemize}
              \item Lemma \ref{lemma:candidate-expansion-type-validation} to or over all ce-type validation premises.
              \item The IH, part 2(a)(i) to or over all synthetic ce-expression validation premises.
              \item The IH, part 2(a)(ii) to or over all synthetic ce-rule validation premises.
              \item The IH, part 2(b)(i) to or over all analytic ce-expression validation premises.
            \end{itemize}
            We then derive the conclusion by applying Rules (\ref{rules:hastypeUP}), Rule (\ref{rule:ruleType}), Lemma \ref{lemma:weakening-UP},  the identification convention and exchange as needed.
          \item[\text{(\ref{rule:csyn-splicede})}] We have:
            \begin{pfsteps*}
              \item $\ce=\acesplicede{m}{n}$ \BY{assumption}
              \item $\parseUExp{\bsubseq{b}{m}{n}}{\ue}$ \BY{assumption}
              \item $\esyn{\uDelta}{\uGamma}{\uPsi}{\uPhi}{\ue}{e}{\tau}$ \BY{assumption} \pflabel{expands}
            %  \item $\uetsmenv{\Delta_\text{app}}{\Psi}$ \BY{assumption} \pflabel{uetsmenv}
              \item $\Delta \cap \Delta_\text{app}=\emptyset$ \BY{assumption} \pflabel{delta-disjoint}
              \item $\domof{\Gamma} \cap \domof{\Gamma_\text{app}}=\emptyset$ \BY{assumption} \pflabel{gamma-disjoint}
              \item $\hastypeU{\Delta_\text{app}}{\Gamma_\text{app}}{e}{\tau}$ \BY{IH, part 1(a)(i) on \pfref{expands}} \pflabel{hastype}
              \item $\hastypeU{\Dcons{\Delta}{\Delta_\text{app}}}{\Gcons{\Gamma}{\Gamma_\text{app}}}{e}{\tau}$ \BY{Lemma \ref{lemma:weakening-UP} over $\Delta$ and $\Gamma$ and exchange on \pfref{hastype}}
            \end{pfsteps*}
            \resetpfcounter
        \end{byCases}
      \item We induct on the assumption. There is one case.
        \begin{byCases}
          \item[\text{(\ref{rule:crsyn})}] We have:
            \begin{pfsteps*}
              \item $\crv=\acematchrule{p}{\ce}$ \BY{assumption}
              \item $r=\aematchrule{p}{e}$ \BY{assumption}
              \item $\patType{\pctx}{p}{\tau}$ \BY{assumption} \pflabel{patType}
              \item $\csyn{\Delta}{\Gcons{\Gamma}{\pctx}}{\esceneUP{\uDelta}{\uGamma}{\uPsi}{\uPhi}{b}}{\ce}{e}{\tau'}$ \BY{assumption} \pflabel{cvalidE}
              \item $\Delta \cap \Delta_\text{app} = \emptyset$ \BY{assumption}\pflabel{delta-disjoint}
              \item $\domof{\Gamma} \cap \domof{\pctx} = \emptyset$ \BY{identification convention}\pflabel{gamma-disjoint1}
              \item $\domof{\Gamma_\text{app}} \cap \domof{\pctx} = \emptyset$ \BY{identification convention}\pflabel{gamma-disjoint2}
              \item $\domof{\Gamma} \cap \domof{\Gamma_\text{app}} = \emptyset$ \BY{assumption}\pflabel{gamma-disjoint3}
              \item $\domof{\Gcons{\Gamma}{\pctx}} \cap \domof{\Gamma_\text{app}} = \emptyset$ \BY{standard finite set definitions and identities on \pfref{gamma-disjoint1}, \pfref{gamma-disjoint2} and \pfref{gamma-disjoint3}}\pflabel{gamma-disjoint4}
              \item $\hastypeU{\Dcons{\Delta}{\Delta_\text{app}}}{\Gcons{\Gcons{\Gamma}{\pctx}}{\Gamma_\text{app}}}{e}{\tau'}$ \BY{IH, part 2(a)(i) on \pfref{cvalidE}, \pfref{delta-disjoint} and \pfref{gamma-disjoint4}}\pflabel{hastype}
              \item $\hastypeU{\Dcons{\Delta}{\Delta_\text{app}}}{\Gcons{\Gcons{\Gamma}{\Gamma_\text{app}}}{\pctx}}{e}{\tau'}$ \BY{exchange of $\pctx$ and $\Gamma_\text{app}$ on \pfref{hastype}}\pflabel{hastype2}
              \item $\ruleType{\Dcons{\Delta}{\Delta_\text{app}}}{\Gcons{\Gamma}{\Gamma_\text{app}}}{\aematchrule{p}{e}}{\tau}{\tau'}$ \BY{Rule (\ref{rule:ruleType}) on \pfref{patType} and \pfref{hastype2}}
            \end{pfsteps*}
            \resetpfcounter
        \end{byCases}
    \end{enumerate}
    \item  \begin{enumerate}
      \item We induct on the assumption.
        \begin{byCases}
          \item[\text{(\ref{rule:cana-subsume})}] We have:
            \begin{pfsteps*}
              \item $\csynX{\ce}{e}{\tau}$ \BY{assumption} \pflabel{esyn}
              \item $\hastypeU{\Delta}{\Gamma}{e}{\tau}$ \BY{IH, part 2(a)(i) on \pfref{esyn}}
            \end{pfsteps*}
          \item[\text{(\ref{rule:cana-let}) through (\ref{rule:eana-match})}] In each of these cases, we apply:
            \begin{itemize}
              \item Lemma \ref{lemma:candidate-expansion-type-validation} to or over all ce-type validation premises.
              \item The IH, part 2(a)(i) to or over all synthetic ce-expression validation premises.
              \item The IH, part 2(a)(ii) to or over all synthetic ce-rule validation premises.
              \item The IH, part 2(b)(i) to or over all analytic ce-expression validation premises.
            \end{itemize}
            We then derive the conclusion by applying Rules (\ref{rules:hastypeUP}), Rule (\ref{rule:ruleType}), Lemma \ref{lemma:weakening-UP},  the identification convention and exchange as needed.
          \item[\text{(\ref{rule:cana-splicede})}] We have:
            \begin{pfsteps*}
              \item $\ce=\acesplicede{m}{n}$ \BY{assumption}
              \item $\parseUExp{\bsubseq{b}{m}{n}}{\ue}$ \BY{assumption}
              \item $\eana{\uDelta}{\uGamma}{\uPsi}{\uPhi}{\ue}{e}{\tau}$ \BY{assumption} \pflabel{expands}
              \item $\istypeU{\Delta \cup \Delta_\text{app}}{\tau}$ \BY{assumption} \pflabel{istype}
            %  \item $\uetsmenv{\Delta_\text{app}}{\Psi}$ \BY{assumption} \pflabel{uetsmenv}
              \item $\Delta \cap \Delta_\text{app}=\emptyset$ \BY{assumption} \pflabel{delta-disjoint}
              \item $\domof{\Gamma} \cap \domof{\Gamma_\text{app}}=\emptyset$ \BY{assumption} \pflabel{gamma-disjoint}
              \item $\hastypeU{\Delta_\text{app}}{\Gamma_\text{app}}{e}{\tau}$ \BY{IH, part 1(b)(i) on \pfref{expands}, \pfref{delta-disjoint}, \pfref{gamma-disjoint} and \pfref{istype}} \pflabel{hastype}
              \item $\hastypeU{\Dcons{\Delta}{\Delta_\text{app}}}{\Gcons{\Gamma}{\Gamma_\text{app}}}{e}{\tau}$ \BY{Lemma \ref{lemma:weakening-UP} over $\Delta$ and $\Gamma$ and exchange on \pfref{hastype}}
            \end{pfsteps*}
            \resetpfcounter
        \end{byCases}
      \item We induct on the assumption. There is one case.
        \begin{byCases}
          \item[\text{(\ref{rule:crana})}] We have:    
            \begin{pfsteps*}
                \item $\crv=\acematchrule{p}{\ce}$ \BY{assumption}
                \item $r=\aematchrule{p}{e}$ \BY{assumption}
                \item $\patType{\pctx}{p}{\tau}$ \BY{assumption} \pflabel{patType}
                \item $\cana{\Delta}{\Gcons{\Gamma}{\pctx}}{\esceneUP{\uDelta}{\uGamma}{\uPsi}{\uPhi}{b}}{\ce}{e}{\tau'}$ \BY{assumption} \pflabel{cvalidE}
                \item $\istypeU{\Delta \cup \Delta_\text{app}}{\tau'}$ \BY{assumption} \pflabel{istype}
                \item $\domof{\Gamma} \cap \domof{\Gamma_\text{app}} = \emptyset$ \BY{assumption}\pflabel{gamma-disjoint3}
                \item $\Delta \cap \Delta_\text{app} = \emptyset$ \BY{assumption}\pflabel{delta-disjoint}
                \item $\domof{\Gamma} \cap \domof{\pctx} = \emptyset$ \BY{identification convention}\pflabel{gamma-disjoint1}
                \item $\domof{\Gamma_\text{app}} \cap \domof{\pctx} = \emptyset$ \BY{identification convention}\pflabel{gamma-disjoint2}
                \item $\domof{\Gcons{\Gamma}{\pctx}} \cap \domof{\Gamma_\text{app}} = \emptyset$ \BY{standard finite set definitions and identities on \pfref{gamma-disjoint1}, \pfref{gamma-disjoint2} and \pfref{gamma-disjoint3}}\pflabel{gamma-disjoint4}
                \item $\hastypeU{\Dcons{\Delta}{\Delta_\text{app}}}{\Gcons{\Gcons{\Gamma}{\pctx}}{\Gamma_\text{app}}}{e}{\tau'}$ \BY{IH, part 2(b)(i) on \pfref{cvalidE}, \pfref{delta-disjoint}, \pfref{gamma-disjoint4} and \pfref{istype}}\pflabel{hastype}
                \item $\hastypeU{\Dcons{\Delta}{\Delta_\text{app}}}{\Gcons{\Gcons{\Gamma}{\Gamma_\text{app}}}{\pctx}}{e}{\tau'}$ \BY{exchange of $\pctx$ and $\Gamma_\text{app}$ on \pfref{hastype}}\pflabel{hastype2}
                \item $\ruleType{\Dcons{\Delta}{\Delta_\text{app}}}{\Gcons{\Gamma}{\Gamma_\text{app}}}{\aematchrule{p}{e}}{\tau}{\tau'}$ \BY{Rule (\ref{rule:ruleType}) on \pfref{patType} and \pfref{hastype2}}
              \end{pfsteps*}
              \resetpfcounter

        \end{byCases}
    \end{enumerate}
  \end{enumerate}
\end{enumerate}

We must now show that the induction is well-founded. All applications of the IH are on subterms except the following.  

\begin{itemize}
\item The only cases in the proof of part 1 that invoke the IH, part 2 are Case (\ref{rule:esyn-apuetsm}) in the proof of part 1(a)(i) and Case (\ref{rule:eana-lit}) in the proof of part 1(b)(i). The only cases in the proof of part 2 that invoke the IH, part 1 are Case (\ref{rule:csyn-splicede}) in the proof of part 2(a)(i) and Case (\ref{rule:cana-splicede}) in the proof of part 2(b)(i). We can show that the following metric on the judgements that we induct on is stable in one direction and strictly decreasing in the other direction:
\begin{align*}
\sizeof{\esyn{\uDelta}{\uGamma}{\uPsi}{\uPhi}{\ue}{e}{\tau}} & = \sizeof{\ue}\\
\sizeof{\eana{\uDelta}{\uGamma}{\uPsi}{\uPhi}{\ue}{e}{\tau}} & = \sizeof{\ue}\\
\sizeof{\csyn{\Delta}{\Gamma}{\esceneUP{\uDelta}{\uGamma}{\uPsi}{\uPhi}{b}}{\ce}{e}{\tau}} & = \sizeof{b}\\
\sizeof{\cana{\Delta}{\Gamma}{\esceneUP{\uDelta}{\uGamma}{\uPsi}{\uPhi}{b}}{\ce}{e}{\tau}} & = \sizeof{b}
\end{align*}
where $\sizeof{b}$ is the length of $b$ and $\sizeof{\ue}$ is the sum of the lengths of the ueTSM literal bodies in $\ue$,
\begin{align*}
\sizeof{\ux} & = 0\\
\sizeof{\auasc{\utau}{\ue}} & = \sizeof{\ue}\\
\sizeof{\auletsyn{\ux}{\ue}{\ue'}} & = \sizeof{\ue} + \sizeof{\ue'}\\
\sizeof{\auanalam{\ux}{\ue}} & = \sizeof{\ue}\\
\sizeof{\aulam{\utau}{\ux}{\ue}} &= \sizeof{\ue}\\
\sizeof{\auap{\ue_1}{\ue_2}} & = \sizeof{\ue_1} + \sizeof{\ue_2}\\
\sizeof{\autlam{\ut}{\ue}} & = \sizeof{\ue}\\
\sizeof{\autap{\ue}{\utau}} & = \sizeof{\ue}\\
\sizeof{\auanafold{\ue}} & = \sizeof{\ue}\\
\sizeof{\auunfold{\ue}} & = \sizeof{\ue}\\
%\end{align*}
%\begin{align*}
\sizeof{\autpl{\labelset}{\mapschema{\ue}{i}{\labelset}}} & = \sum_{i \in \labelset} \sizeof{\ue_i}\\
\sizeof{\aupr{\ell}{\ue}} & = \sizeof{\ue}\\
\sizeof{\auanain{\ell}{\ue}} & = \sizeof{\ue}\\
%\sizeof{\aucase{\labelset}{\utau}{\ue}{\mapschemab{\ux}{\ue}{i}{\labelset}}} & = \sizeof{\ue} + \sum_{i \in \labelset} \sizeof{\ue_i}\\
\sizeof{\aumatchwithb{n}{\ue}{\seqschemaX{\urv}}} & = \sizeof{\ue} + \sum_{1 \leq i \leq n} \sizeof{r_i}\\
\sizeof{\audefuetsm{\utau}{\eparse}{\tsmv}{\ue}} & = \sizeof{\ue}\\
\sizeof{\auimplicite{\tsmv}{\ue}} & = \sizeof{\ue}\\
\sizeof{\autsmap{b}{\tsmv}} & = \sizeof{b}\\
\sizeof{\auelit{b}} & = \sizeof{b}\\
\sizeof{\audefuptsm{\utau}{\eparse}{\tsmv}{\ue}} & = \sizeof{\ue}\\
\sizeof{\auimplicitp{\tsmv}{\ue}} & = \sizeof{\ue}
\end{align*}
and $\sizeof{r}$ is defined as follows:
\begin{align*}
\sizeof{\aumatchrule{\upv}{\ue}} & = \sizeof{\ue}
\end{align*}

Going from part 1 to part 2, the metric remains stable:
\begin{align*}
 & \sizeof{\esyn{\uDelta}{\uGamma}{\uPsi}{\uPhi}{\autsmap{b}{\tsmv}}{e}{\tau}}\\
=& \sizeof{\eana{\uDelta}{\uGamma}{\uPsi}{\uPhi}{\auelit{b}}{e}{\tau}}\\
=& \sizeof{\cana{\emptyset}{\emptyset}{\esceneUP{\uDelta}{\uGamma}{\uPsi}{\uPhi}{b}}{\ce}{e}{\tau}}\\
=&\sizeof{b}\end{align*}

Going from part 2 to part 1, in each case we have that $\parseUExp{\bsubseq{b}{m}{n}}{\ue}$ and the IH is applied to the judgements $\esyn{\uDelta}{\uGamma}{\uPsi}{\uPhi}{\ue}{e}{\tau}$ and $\eana{\uDelta}{\uGamma}{\uPsi}{\uPhi}{\ue}{e}{\tau}$, respectively. Because the metric is stable when passing from part 1 to part 2, we must have that it is strictly decreasing in the other direction:
\[\sizeof{\esyn{\uDelta}{\uGamma}{\uPsi}{\uPhi}{\ue}{e}{\tau}} < \sizeof{\csyn{\Delta}{\Gamma}{\esceneUP{\uDelta}{\uGamma}{\uPsi}{\uPhi}{b}}{\acesplicede{m}{n}}{e}{\tau}}\]
and
\[\sizeof{\eana{\uDelta}{\uGamma}{\uPsi}{\uPhi}{\ue}{e}{\tau}} < \sizeof{\cana{\Delta}{\Gamma}{\esceneUP{\uDelta}{\uGamma}{\uPsi}{\uPhi}{b}}{\acesplicede{m}{n}}{e}{\tau}}\]
i.e. by the definitions above, 
\[\sizeof{\ue} < \sizeof{b}\]

This is established by appeal to Condition \ref{condition:body-subsequences}, which states that subsequences of $b$ are no longer than $b$, and the following condition, which states that an unexpanded expression constructed by parsing a textual sequence $b$ is strictly smaller, as measured by the metric defined above, than the length of $b$, because some characters must necessarily be used to delimit each literal body.
\begin{condition}[Expression Parsing Monotonicity]\label{condition:body-parsing-B} If $\parseUExp{b}{\ue}$ then $\sizeof{\ue} < \sizeof{b}$.\end{condition}

Combining Conditions \ref{condition:body-subsequences} and \ref{condition:body-parsing-B}, we have that $\sizeof{\ue} < \sizeof{b}$ as needed.
\item In Case (\ref{rule:eana-subsume}) of the proof of part 1(b)(i), we apply the IH, part 1(a)(i), with $\ue=\ue$. This is well-founded because all applications of the IH, part 1(b)(i) elsewhere in the proof are on strictly smaller terms.
\item Similarly, in Case (\ref{rule:cana-subsume}) of the proof of part 2(b)(i), we apply the IH, part 2(a)(i), with $\ce=\ce$. This is well-founded because all applications of the IH, part 2(b)(i) elsewhere in the proof are on strictly smaller terms.
\end{itemize}
\end{proof} 
\section{Related Work}
 \todo{cite/comment on past work on bidirectional typechecking}
 \todo{TSLs}
 \todo{ichikawa modularity paper}
 \todo{other things from related work section of ECOOP paper}
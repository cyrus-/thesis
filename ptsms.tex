% !TEX root = omar-thesis.tex

\chapter{Parametric TSMs (pTSMs)}\label{chap:ptsms}
% \begin{quote}\textit{The recent development of programming languages suggests that the simul\-taneous achievement of simplicity 
% and generality in language design is a serious unsolved 
% problem.}\begin{flushright}--- John Reynolds (1970) \cite{Reynolds70}\end{flushright}
% \end{quote}
% ~\\
In the two previous chapters, we introduced simple TSMs (sTSMs). Simple TSMs are defined at a single type, like \li{rx}, and the expansions that they generate have access to bindings at the application site only through spliced expressions that the client provides. In this chapter, we introduce \emph{parametric TSMs} (pTSMs). pTSMs can be defined over a type- and module-parameterized family of types, and the expansions that they generate can refer to the supplied type and module parameters. 

This chapter is organized like the preceding chapters. We begin in Sec. \ref{sec:parameterized-tsms-by-example} by introducing parametric TSMs by example in VerseML. In particular, we discuss type parameters in Sec. \ref{sec:type-parameters} and module parameters in Sec. \ref{sec:module-parameters}. We then develop a reduced calculus of parametric TSMs, $\miniVerseParam$, in Sec. \ref{sec:miniVerseP}.
\section{Parametric TSMs By Example}\label{sec:parameterized-tsms-by-example}

\subsection{Type Parameters}\label{sec:type-parameters}
Recall from Sec. \ref{sec:lists} the definition of the type-parameterized family of list types:
\begin{lstlisting}[numbers=none]
type list('a) = rec(self => Nil + Cons of 'a * self)
\end{lstlisting}

% \emph{Kinds} classify constructor expressions, much like types classify expressions. Types are constructor expressions of kind \li{T}, and type constructors are constructor expressions of arrow kind. Here, \li{list} takes a single type parameter, so it has arrow kind \li{T -> T}.

ML dialects commonly define derived syntactic forms for constructing and pattern matching over values of list type. VerseML, in contrast, does not build in derived list forms. Instead, Figure \ref{fig:petsm-list} defines a \emph{parametric expression TSM} (peTSM) and a \emph{parametric pattern TSM} (ppTSM), both named \li{#\dolla#list}.
\begin{figure}[h]
\begin{lstlisting}
syntax $list('a) at list('a) for expressions by 
  static fn(b : body) : parse_result(proto_expr) => (* ... *)
and for patterns by 
  static fn(b : body) : parse_result(proto_pat) => (* ... *) 
end
\end{lstlisting}
\caption{The type-parameterized \texttt{\$list} TSMs.}
\label{fig:petsm-list}
\end{figure}

Line 1 specifies a single type parameter, \li{'a}. This type parameter appears in the type annotation, which establishes that:
\begin{itemize}
\item the peTSM \li{#\dolla#list}, when applied to a type \li{T} and a generalized literal form, can only generate expansions of type \li{list(T)}; and 
\item the ppTSM \li{#\dolla#list}, when applied to a type \li{T} and a generalized literal form, can only generate expansions that match values of type \li{list(T)}.
\end{itemize}
For example, we can apply \li{#\dolla#list} to \li{int} and generalized literal forms delimited by square brackets as follows:
\begin{lstlisting}[numbers=none]
val y = $list int [3SURL, EURL4SURL, EURL5]
val x = $list int [1SURL, EURL2SURL :: EURLy]
\end{lstlisting}
The parse functions (whose definitions are elided above for concision) break the literal body up into  spliced terms -- here, by recognizing commas and, optionally, a trailing spliced term prefixed by two colons (\li{SURL::EURL}), which becomes the tail of the list. The final expansion of the example above is equivalent to the following when the list value constructors are in scope:
\begin{lstlisting}[numbers=none]
val y = Cons(3, Cons(4, Cons(5, Nil)))
val x = Cons(1, Cons(2, y))
\end{lstlisting}
Once again, due to the prohibition on context-dependent expansions, the expansion itself must use the explicit \li{fold} and \li{inj} operators rather than the list value constructors \li{Cons} and \li{Nil}.

\subsection{Module Parameters}\label{sec:module-parameters}
Let us now address the inconvenience of needing to use explicit \li{fold} and \li{inj}  operators by  defining a module-parameterized TSM.

Recall that in Figure \ref{fig:LIST}, we defined a signature \li{LIST} that exported the definition of \li{list} and specified the list value constructors (and some other values.) The definition of \li{#\dolla#list'} shown in Figure \ref{fig:ptsm-listprime} takes modules matching this signature as an additional parameter.

\begin{figure}[h]
\begin{lstlisting}[numbers=none]
syntax $list' (L : LIST) 'a at 'a L.list for expressions by 
  static fn(b : body) : parse_result(proto_expr) => (* ... *)
and for patterns by 
  static fn(b : body) : parse_result(proto_pat) => (* ... *)
end
\end{lstlisting}
\caption{The type- and module-parameterized \texttt{\$list'} TSMs.}
\label{fig:ptsm-listprime}
\end{figure}
% differing only in that any type parameter that the peTSM specifies can appear free in the generated expansion.

We can apply \li{#\dolla#list'} to the module \li{List} and the type \li{int} as follows:
\begin{lstlisting}[numbers=none]
val y = $list' List int [3SURL, EURL4SURL, EURL5]
val x = $list' List int [1SURL, EURL2SURL :: EURLy]
\end{lstlisting}
The expansion is:
\begin{lstlisting}[numbers=none]
val y = List.Cons(3, List.Cons(4, List.Cons(5, List.Nil)))
val x = List.Cons(1, List.Cons(2, y))
\end{lstlisting}
There is no need to use explicit \li{fold} and \li{inj} operators in this expansion, because the expansion projects the constuctors out of the provided module parameter. The TSM itself did not assume that the module would be named \li{List} (indeed, internally, it refers to it as \li{L}.)

This makes matters simpler for the TSM provider, but there is a syntactic cost associated with supplying a module parameter at each TSM application site. To reduce this cost, VerseML supports partial parameter application in TSM abbreviations. For example, we can define \li{#\dolla#list} by partially applying \li{#\dolla#list'} as follows:
\begin{lstlisting}[numbers=none]
let syntax $list = $list' List
\end{lstlisting}
(This abbreviates both the expression and pattern TSMs -- sort qualifiers can be added to restrict the abbreviation if desired.)


% Similarly, in lieu of derived list pattern forms, we define the following \emph{parameterized pattern TSM} (ppTSM):
% \begin{lstlisting}[numbers=none]
% syntax $list('a) at list('a) for patterns {
%   static fn(body : Body) : ParseResult(CEPat) => (* ... *)
% }
% \end{lstlisting}
% Again, Line 1 names the ppTSM \li{#\dolla#list} and specifies a single type parameter, \li{'a}. This type parameter appears in the type annotation, which specifies that \li{#\dolla#list}, when apply to a type \li{'a} and a generalized literal form, will only generate patterns that match values of type \li{list('a)}. 

% For example, we can apply the ppTSM \li{#\dolla#list} and the \li{#\dolla#list} to define the polymorphic map function as follows.
% \begin{lstlisting}[numbers=none]
% fun map (f : 'a -> 'b) (x : list('a)) => match x { 
%   $list('a) [] => $list('b) []
% | $list('a) [hdSURL :: EURLtl] => $list('b) [f hdSURL :: EURLmap f tl]
% }
% \end{lstlisting}
% The expansion of this function definition, written textually, is:
% \begin{lstlisting}[numbers=none]
% fun map (f : 'a -> 'b) (x : list('a)) : 'b list => match x { 
%   Nil => Nil
% | Cons(hd, tl) => Cons(f hd, map f tl)
% }
% \end{lstlisting}
% This is somewhat unsatisfying, however, because the expansion is more concise than the unexpanded definition of \li{map}. To further reduce syntactic cost, we can designate \li{#\dolla#list} as the implicit TSM for both expressions and patterns at all types \li{'a list} around our definition of \li{map} as follows.
% \begin{lstlisting}[numbers=none]
% implicit syntax $list('a) in
%   fun map (f : 'a -> 'b) (x : 'a list) : 'b list => match x {
%     [] => []
%   | [hdSURL :: EURLtl] => [f hdSURL :: EURLmap f tl]
%   }
% end
% \end{lstlisting}
% By designating an implicit TSM, we no longer need to explicitly apply \li{#\dolla#list} within expressions in analytic position or patterns.

% When designating an implicit TSM, we assume that free type variables in the type annotation, e.g. here \li{'a}, range over all types. We can make this more explicit by specifying a type parameter explicitly as follows:
% \begin{lstlisting}[numbers=none]
% implicit syntax('a) $list('a) at list('a) in
% 	(* ... *)
% end
% \end{lstlisting}
% All type parameters must appear in the type annotation.
% \subsection{More Examples}
Module parameters also allow us to define TSMs that operate uniformly over module-parameterized families of abstract types. For example, the module-parameterized TSM \texttt{\$r} defined in Figure \ref{fig:param-tsm-r} supports the POSIX regex syntax for any type \li{R.t} where \li{R : RX}. 

\begin{figure}[h]
\begin{lstlisting}
syntax $r(R : RX) at R.t by 
  static fn(b : body) : parse_result(proto_expr) => (* ... *)
end
\end{lstlisting}
\vspace{-5px}
\caption{The module-parameterized TSM \texttt{\$r}.}
\label{fig:param-tsm-r}
\end{figure}

To be clear: parameters are available to the generated expansion, but they are not available to the parse function that generates the expansion. For example, the following TSM definition is not well-typed because it refers to \li{M} from within the parse function:
\begin{lstlisting}[numbers=none]
syntax $badM(M : A) at T by 
  static fn(b : body) => let x = M.x in (* ... *)
end
\end{lstlisting}
(In the next chapter, we will define a mechanism that allows parse functions to access common constructs.)
% \subsubsection{Queues}
% Consider the following signature for working with persistent queues:
% \begin{lstlisting}[numbers=none]
% signature QUEUE = sig
%   type queue('a)
%   val empty  : queue('a)
%   val insert : 'a * queue('a) -> queue('a)
%   val remove : queue('a) -> option('a * queue('a))
% end 
% \end{lstlisting}
% Structures that match this signature must define a type constructor \li{queue} of kind \li{T -> T} and three values -- \li{empty} introduces the empty queue, \li{insert} inserts a value onto the back of a queue, and \li{remove} removes the element at the front of the queue and returns it and the remaining queue, or \li{None} if the queue is empty.%one for inserting an item into a queue, and one for removing a value from a queue.

% There are many possible structures that implement this signature. For example, we can define a structure \li{ListQueue} that represents queues internally as lists, where the head of the list is the back of the queue. With this representation, \li{insert} is a constant time operation, but \li{remove} is a linear time operation. Alternatively, we might define a structure \li{TwoListQueue} that represents queues internally as a pair of lists, maintaining the invariant that one is the reverse of the other, so that both \li{insert} and \li{remove} are constant time operations (see \cite{harper1997programming} for the details of this and other possibilities). 

% Regardless of the implementation that the client chooses, we would like for the client to be able to introduce queues more naturally and at lower syntactic cost than is possible by directly applying the functions specified by the signature above. In VerseML, we can give clients of structures matching the signature \li{QUEUE} this ability by defining the following parameterized expression TSM:
% \begin{lstlisting}[numbers=none]
% syntax $queue(Q : QUEUE)('a) at Q.queue('a) {
%   static fn(body : Body) : ParseResult(CEExp) => (* ... *)
% }
% \end{lstlisting}
% This peTSM specifies one module parameter, \li{Q}, which must match the signature \li{QUEUE}, and one type parameter, \li{'a} (implicitly of kind \li{T}). These appear in the type annotation, which specifies that expansions that arise from applying \li{#\dolla#queue} to a module \li{Q : QUEUE} and a type \li{'a} will be of type \li{Q.queue('a)}. For example:
% \begin{lstlisting}
% val q = $queue TwoListQueue int [SURL> EURL1SURL, EURL2SURL, EURL3]
% val q' = $queue TwoListQueue int [qSURL > EURL4SURL, EURL5]
% \end{lstlisting}
% On Line 1, the initial angle bracket (\li{SURL>EURL}) indicates that the items are inserted in left-to-right order. The items in the queue are given as spliced subexpressions separated by commas. Line 2 inserts two additional items onto the back of the queue \li{q}. The expansion of this example, written textually, is:
% \begin{lstlisting}
% val q : TwoListQueue.queue(int) = 
%   TwoListQueue.insert(1, 
%     TwoListQueue.insert(2, 
%       TwoListQueue.insert(3, 
%         TwoListQueue.empty)))
% val q' : TwoListQueue.queue(int) = 
%   TwoListQueue.insert(4, TwoListQueue.insert(5, q))
% \end{lstlisting}
% Notice that the expansion can refer to the module parameter \li{TwoListQueue}.

% We can further reduce syntactic cost by defining a synonym for the partial application of \li{#\dolla#queue} to the module parameter \li{TwoListQueue}:
% \begin{lstlisting}[numbers=none]
% syntax $tlq = $queue TwoListQueue
% val q = $tlq int [SURL> EURL1SURL, EURL2SURL, EURL3]
% \end{lstlisting}
% We can further define a synonym for the partial application of \li{#\dolla#tlq} to a type parameter:
% \begin{lstlisting}[numbers=none]
% syntax $tlqi = $tlq int (* = $queue TwoListQueue int *)
% val q' = $tlqi [qSURL > EURL4SURL, EURL5]
% \end{lstlisting}
% \subsection{Module Parameters}
% VerseML also provides a module language based on the Standard ML module language \cite{MacQueen:1984:MSM:800055.802036}. The module language consists of \emph{module expressions} classified by \emph{signatures}. %Signatures specify type components, which may be opaque or transparent, value components, and module components.h

% %In Sec. \ref{sec:motivating-examples}, we gave several examples of signatures and discussed how one might introduce derived forms that  across a module-parameterized family of types.



% Another way to reduce syntactic cost is by designating \li{#\dolla#queue Q 'a} the implicit TSM at all types of the form \li{Q.queue('a)} where \li{Q : QUEUE}. This is written as follows:
% \begin{lstlisting}[numbers=none]
% implicit syntax (Q : QUEUE) ('a) => $queue Q 'a in
%   val q : TwoListQueue.queue(int) = [SURL> EURL1SURL, EURL2SURL, EURL3]
%   val q' : TwoListQueue.queue(int) = [qSURL > EURL4SURL, EURL5]
% end
% \end{lstlisting}
% This designation is particularly useful for clients who need to construct a queue as an argument to a function. For example, consider a function 
% \begin{lstlisting}[numbers=none]
% enqueue_jobs : Q.queue(Job) -> Ticket
% \end{lstlisting}
% for some module \li{Q : QUEUE} and types \li{Job} and \li{Ticket}. We can enqueue a sequence of jobs \li{j1} through \li{j4} under the TSM designation above as follows:
% \begin{lstlisting}[numbers=none]
% enqueue_jobs [SURL> EURLj1SURL, EURLj2SURL, EURLj3SURL, EURLj4]
% \end{lstlisting}

\section{\texorpdfstring{$\miniVerseParam$}{miniVerseP}}\label{sec:miniVerseP}
We will now define a reduced dialect of VerseML called $\miniVerseParam$ that supports parametric expression and pattern TSMs (peTSMs and ppTSMs.) This language, like $\miniVersePat$, consists of an unexpanded language (UL) defined by typed expansion to an expanded language (XL).

\subsection{Syntax of the Expanded Language (XL)}\label{sec:P-expanded-terms}

\begin{figure}[p] 
\[\begin{array}{lllllll}
\textbf{Sort} & & & \textbf{Operational Form} 
%& \textbf{Stylized Form} 
& \textbf{Description}\\
\mathsf{Sig} & \sigma & ::= & \asignature{\kappa}{u}{\tau} 
%& \signature{u}{\kappa}{\tau} 
& \text{signature}\\
\mathsf{Mod} & M & ::= & X 
%& X 
& \text{module variable}\\
&&& \astruct{c}{e} 
%& \struct{c}{e} 
& \text{structure}\\
&&& \aseal{\sigma}{M} 
%& \seal{M}{\sigma} 
& \text{seal}\\
&&& \amlet{\sigma}{M}{X}{M} %& \mlet{X}{M}{M}{\sigma} 
& \text{definition}
\end{array}\]
\caption[Syntax of signatures and module expressions in $\miniVerseParam$]{Syntax of signatures and module expressions in $\miniVerseParam$.}
\label{fig:P-modules-signatures}
\end{figure}


\begin{figure}[p] 
\[\begin{array}{lrlllll}
\textbf{Sort} & & & \textbf{Operational Form} 
%& \textbf{Stylized Form} 
& \textbf{Description}\\
\mathsf{Kind} & \kappa & ::= & \akdarr{\kappa}{u}{\kappa} 
%& \kdarr{u}{\kappa}{\kappa} 
& \text{dependent function}\\
&&& \akunit 
%& \kunit 
& \text{nullary product}\\
&&& \akdbprod{\kappa}{u}{\kappa} 
%& \kdbprod{u}{\kappa}{\kappa} 
& \text{dependent product}\\
%&&& \akdprodstd & \kdprodstd & \text{labeled dependent product}\\
&&& \akty 
%& \kty
& \text{types}\\
&&& \aksing{\tau} 
%& \ksing{\tau} 
& \text{singleton}\\
\mathsf{Con} & c, \tau & ::= & u 
%& u 
& \text{constructor variable}\\
&&& t 
%& t 
& \text{type variable}
\\
&&& \acabs{u}{c} 
%& \cabs{u}{c} 
& \text{abstraction}\\
&&& \acapp{c}{c} 
%& \capp{c}{c} 
& \text{application}\\
&&& \actriv 
%& \ctriv 
& \text{trivial}\\
&&& \acpair{c}{c}
% & \cpair{c}{c} 
& \text{pair}\\
&&& \acprl{c} 
%& \cprl{c} 
& \text{left projection}\\
&&& \acprr{c} 
%& \cprr{c} 
& \text{right projection}\\
%&&& \adtplX & \dtplX & \text{labeled dependent tuple}\\
%&&& \adprj{\ell}{c} & \prj{c}{\ell} & \text{projection}\\
&&& \aparr{\tau}{\tau} 
%& \parr{\tau}{\tau} 
& \text{partial function}\\
&&& \aallu{\kappa}{u}{\tau} 
%& \forallu{u}{\kappa}{\tau} 
& \text{polymorphic}\\
&&& \arec{t}{\tau} 
%& \rect{t}{\tau} 
& \text{recursive}\\
&&& \aprod{\labelset}{\mapschema{\tau}{i}{\labelset}} 
%& \prodt{\mapschema{\tau}{i}{\labelset}} 
& \text{labeled product}\\
&&& \asum{\labelset}{\mapschema{\tau}{i}{\labelset}} 
%& \sumt{\mapschema{\tau}{i}{\labelset}} 
& \text{labeled sum}\\
&&& \amcon{M} 
%& \mcon{M} 
& \text{constructor component}
\end{array}\]
\caption[Syntax of kinds and constructors in $\miniVerseParam$]{Syntax of kinds and constructors in $\miniVerseParam$. By convention, we choose the metavariable $\tau$ for constructors that, in well-formed terms, must necessarily be of kind $\kty$, and the metavariable $c$ otherwise. Similarly, we use constructor variables $t$ to stand for constructors of kind $\kty$, and constructor variables $u$ otherwise.}
\label{fig:P-kinds-constructors}
\end{figure}

\begin{figure}
\[\begin{array}{lllllll}
\textbf{Sort} & & & \textbf{Operational Form} 
%& \textbf{Stylized Form} 
& \textbf{Description}\\
\mathsf{Exp} & e & ::= & x 
%& x 
& \text{variable}\\
&&& \aelam{\tau}{x}{e} 
%& \lam{x}{\tau}{e} 
& \text{abstraction}\\
&&& \aeap{e}{e} 
%& \ap{e}{e} 
& \text{application}\\
&&& \aeclam{\kappa}{u}{e} %& \clam{u}{\kappa}{e} 
& \text{constructor abstraction}\\
&&& \aecap{e}{c} %& \cAp{e}{c} 
& \text{constructor application}\\
&&& \aefold{t}{\tau}{e} %& \fold{e} 
& \text{fold}\\
&&& \aeunfold{e} %& \unfold{e} 
& \text{unfold}\\
&&& \aetpl{\labelset}{\mapschema{e}{i}{\labelset}} 
%& \tpl{\mapschema{e}{i}{\labelset}} 
& \text{labeled tuple}\\
&&& \aepr{\ell}{e} 
%& \prj{e}{\ell} 
& \text{projection}\\
&&& \aein{\labelset}{\ell}{\mapschema{\tau}{i}{\labelset}}{e} 
%& \inj{\ell}{e} 
& \text{injection}\\
&&& \aematchwith{n}{\tau}{e}{\seqschemaX{r}} 
%& \matchwith{e}{\seqschemaX{r}} 
& \text{match}\\
&&& \amval{M} 
%& \mval{M} 
& \text{value component}\\
\mathsf{Rule} & r & ::= & \aematchrule{p}{e} 
%& \matchrule{p}{e} 
& \text{rule}\\
\mathsf{Pat} & p & ::= & x 
%& x 
& \text{variable pattern}\\
&&& \aewildp 
%& \wildp 
& \text{wildcard pattern}\\
&&& \aefoldp{p} 
%& \foldp{p} 
& \text{fold pattern}\\
&&& \aetplp{\labelset}{\mapschema{p}{i}{\labelset}} 
%& \tplp{\mapschema{p}{i}{\labelset}} 
& \text{labeled tuple pattern}\\
&&& \aeinjp{\ell}{p} 
%& \injp{\ell}{p} 
& \text{injection pattern}
\end{array}\]
\caption[Syntax of expanded expressions, rules and patterns in $\miniVerseParam$]{Syntax of expanded expressions, rules and patterns in $\miniVerseParam$.}
\label{fig:P-expanded-terms}
\end{figure}


Figure \ref{fig:P-modules-signatures} defines the syntax of the \emph{expanded module language}. Figure \ref{fig:P-kinds-constructors} defines the syntax of the \emph{expanded type construction language}. Figure \ref{fig:P-expanded-terms} defines the syntax of the \emph{expanded expression language}.


\subsection{Statics of the Expanded Language}
The module and type construction languages are based closely on those defined by Harper in \emph{PFPL} \cite{pfpl}. These languages, in turn, are based on the languages developed by Lee et al. \cite{conf/popl/LeeCH07}, and also by Dreyer \cite{dreyer2005understanding}. All of these incorporate Stone and Harper's \emph{dependent singleton kinds} formalism to track type identity \cite{stone2006extensional}. The expression language is similar to that of $\miniVersePat$, defined in Chapter \ref{chap:uptsms}.

The \emph{statics of the expanded language} is defined by a collection of judgements that we organize into three groups. 

The first group of judgements, which we refer to as the \emph{statics of the expanded module language}, define the statics of expanded signatures and module expressions.

\vspace{5px}
$\begin{array}{ll}
\textbf{Judgement Form} & \textbf{Description}\\
\issigX{\sigma} & \text{$\sigma$ is a signature }\\
\sigequalX{\sigma}{\sigma'} & \text{$\sigma$ and $\sigma'$ are definitionally equal signatures}\\
\sigsubX{\sigma}{\sigma'} & \text{$\sigma$ is a sub-signature of $\sigma'$}\\
\hassigX{M}{\sigma} & \text{$M$ matches $\sigma$}\\
\ismvalX{M} & \text{$M$ is, or stands for, a module value}
\end{array}$
\vspace{5px}

The second group of judgements, which we refer to as the \emph{statics of the expanded type construction language}, define the statics of expanded kinds and constructors.

\vspace{5px}
$\begin{array}{ll}
\textbf{Judgement Form} & \textbf{Description}\\
\iskindX{\kappa} & \text{$\kappa$ is a kind}\\
\kequalX{\kappa}{\kappa'} & \text{$\kappa$ and $\kappa'$ are definitional equal kinds}\\
\ksubX{\kappa}{\kappa'} & \text{$\kappa$ is a subkind of $\kappa'$}\\
\haskindX{c}{\kappa} & \text{$c$ has kind $\kappa$}\\
\cequalX{c}{c'}{\kappa} & \text{$c$ and $c'$ are equivalent as constructors of kind $\kappa$}
\end{array}$
\vspace{5px}

The third group of judgements, which we refer to as the \emph{statics of the expanded expression language}, define the statics of types, expanded expressions, rules and patterns.

\vspace{5px}
$\begin{array}{ll}
\textbf{Judgement Form} & \textbf{Description}\\
% \istypeP{\Omega}{\tau} & \text{$\tau$ is a well-formed type}\\
% \tequalPX{\tau}{\tau'} & \text{$\tau$ and $\tau'$ are definitionally equal types}\\
\issubtypePX{\tau}{\tau'} & \text{$\tau$ is a subtype of $\tau'$}\\
\hastypeP{\Omega}{e}{\tau} & \text{$e$ is assigned type $\tau$}\\
\ruleTypeP{\Omega}{r}{\tau}{\tau'} & \text{$r$ takes values of type $\tau$ to values of type $\tau'$}\\
\patTypeP{\Omega'}{p}{\tau} & \text{$p$ matches values of type $\tau$ and generates hypotheses $\Omega'$} 
\end{array}$
\vspace{5px}


A \emph{unified context}, $\Omega$, is a finite function. 
We write
\begin{itemize}
\item $\Omega, X : \sigma$ when $X \notin \domof{\Omega}$ and $\issigX{\sigma}$ for the extension of $\Omega$ with a mapping from $X$ to the hypothesis $X : \sigma$.
\item $\Omega, x : \tau$ when $x \notin \domof{\Omega}$ and $\haskindX{\tau}{\akty}$ for the extension of $\Omega$ with a mapping from $x$ to the hypothesis $x : \tau$
\item $\Omega, u :: \kappa$ when $u \notin \domof{\Omega}$ and $\iskindX{\kappa}$ for the extension of $\Omega$ with a mapping from $u$ to the hypothesis $u :: \kappa$
\end{itemize}
A well-formed unified context is one that can be constructed by some sequence of such extensions, starting from the empty context, $\emptyset$. We identify unified contexts up to exchange and contraction in the usual manner.

The complete set of rules is given in Appendix \ref{appendix:P-statics}. A comprehensive introductory account of these constructs is beyond the scope of this work (see \cite{pfpl}.) Instead, let us summarize the key features of the expanded language by example. 

Modules arise by a \emph{phase-splitting} approach -- the constructor components of the module are ``tupled'' into a single constructor component and the value components of the module are ``tupled'' into a single value component \cite{harper1989higher}. Signatures, $\sigma$, are also split in this way -- a single \emph{kind} classifies the constructor component and a single type classifies the value component of the classified module. The type in the signature can refer to the constructor component of the module through a mediating constructor variable, $u$. The key rule is reproduced below:
\begin{equation*}\tag{\ref{rule:hassig-struct}}
\inferrule{
  \haskindX{c}{\kappa}\\
  \hastypeP{\Omega}{e}{[c/u]\tau}
}{
  \hassigX{\astruct{c}{e}}{\asignature{\kappa}{u}{\tau}}
}
\end{equation*}

For example, consider the VerseML signature on the left, and the corresponding $\miniVerseParam$ signature on the right of Figure \ref{fig:corresponding-signatures}.

\begin{figure}
\begin{minipage}{0.35\textwidth}
\begin{lstlisting}
sig
  type t
  type t' = t * t
  val x : t
  val y : t -> t'
end
\end{lstlisting}
\end{minipage}
\begin{minipage}{0.5\textwidth}\vspace{3px}
{\footnotesize\[
\begin{array}{l}
\asignature{\akdbprod{\\
\quad\quad \akty}{t}{\\
\quad\quad \aksing{
  \aprod{\lbltxt{1}; \lbltxt{2}}{
    \mapitem{\lbltxt{1}}{t}; \mapitem{\lbltxt{2}}{t}
  }
}}\\}{u}{
 \aprod{\lbltxt{x}; \lbltxt{y}}{
  \mapitem{\lbltxt{x}}{\acprl{u}}; \\ 
\quad\quad \mapitem{\lbltxt{y}}{\aparr{
    \acprl{u}
  }{
    \acprr{u}
  }}
}\\
}
\end{array}
\]}
\end{minipage}
\caption{A VerseML signature and the corresponding $\miniVerseParam$ signature.}
\label{fig:corresponding-signatures}
\end{figure}

The kind in the signature on the right (Lines 1-3) is a \emph{dependent product kind} and the type (Lines 4-5) is a product type. Let us consider these in turn.

On Lines 2-3 (left), we specified an abstract type component \li{t}, and then a translucent type component \li{t'} equal to \li{t * t}. Abstract type components have kind $\akty$, as seen on Line 2 (right). The constructor variable $t$ stands for this abstract type component in the second component of the dependent product kind. The second component is not held abstract, i.e. it is translucent, so it is classified by a corresponding \emph{singleton kind}, rather than by the kind $\akty$, as shown on Line 3 (right). A singleton kind $\aksing{\tau}$ classifies only those types definitionally equal to $\tau$. A subkinding relation is necessary to ensure that constructors of singleton kind can appear where a constructor of kind $\akty$ is needed -- the key rule is reproduced below:
\begin{equation*}\tag{\ref{rule:ksub-sing}}
\inferrule{
  \haskindX{\tau}{\akty}
}{
  \ksubX{\aksing{\tau}}{\akty}
}
\end{equation*}

Lines 4-5 (right) define a product type that classifies the value component of matching modules. The constructor variable \li{u} stands for the constructor component of the matching module. The left- and right-projection operations $\acprl{c}$ and $\acprr{c}$ on the right correspond to \li{t} and \li{t'} on the left. (In practice, we would use labeled dependent product kinds, but for simplicity, we stick to binary dependent product kinds here.)

Consider another example: the VerseML \li{LIST} signature from Figure \ref{fig:LIST}, partially reproduced below:
\begin{lstlisting}
sig 
  type list('a) = rec(self => Nil + Cons of 'a * self)
  val Nil : list('a)
  val Cons : 'a * list('a) -> list('a)
  (* ... *)
end
\end{lstlisting}
This VerseML signature corresponds to the following $\miniVerseParam$ signature:
\[
\arraycolsep=1px\begin{array}{ll}

\sigma_\texttt{LIST} & \defeq \asignature{\kappa_\texttt{LIST}}{list}{\tau_\texttt{LIST}}\\
\kappa_\texttt{LIST} & \defeq \akdarr{\akty}{\alpha}{\aksing{
  \arec{self}{
    \asum{L_\texttt{list}}{\\
    & \quad\quad 
      \mapitem{\lbltxt{Nil}}{\aprod{}{}}; \\
    & \quad\quad 
      \mapitem{\lbltxt{Cons}}{
        \aprod{\lbltxt{1}; \lbltxt{2}}{
          \mapitem{\lbltxt{1}}{\alpha}; 
          \mapitem{\lbltxt{2}}{self}
        }
      }
    }
  } 
}}\\
L_\texttt{list} & \defeq \lbltxt{Nil}, \lbltxt{Cons}\\
\tau_\texttt{LIST} & \defeq \aprod{L_\texttt{list}}{\\&
  \quad\quad \mapitem{\lbltxt{Nil}}{
    \aallu{\akty}{\alpha}{\acapp{list}{\alpha}}
  }; \\&
  \quad\quad \mapitem{\lbltxt{Cons}}{
    \aallu{\akty}{\alpha}{
      \aparr{\\&\quad\quad\quad
        \aprod{\lbltxt{1}; \lbltxt{2}}{
          \mapitem{\lbltxt{1}}{\alpha}; 
          \mapitem{\lbltxt{2}}{\acapp{list}{\alpha}}
        }
      }{\\&\quad\quad\quad
        \acapp{list}{\alpha}
      }
    }
  }
}
\end{array}
\]

Here, there is only a single constructor component, so no tupling of the constructor component is necessary. This single constructor component is a type function, so it has dependent function kind: the argument kind is $\akty$ and the return kind is a singleton kind, because the type function is not abstract. (Had we held the list type function abstract, its kind would instead be $\akdarr{\akty}{\_}{\akty}$.
%A well-formed unified inner context is one where there are no cycles in the dependency graph between the hypotheses (constructed in the obvious manner) and for each hypothesis, the constructor, kinds or signature involved is well-formed relative to the unified inner context.

At the top level, a program consists of a module expression, $M$. The module let binding form allows the programmer to bind a module to a module variable, $X$:
\begin{equation*}\tag{\ref{rule:hassig-let}}
\inferrule{
  \hassigX{M}{\sigma}\\
  \issigX{\sigma'}\\
  \hassig{\Omega, X : \sigma}{M'}{\sigma'}  
}{
  \hassigX{\amlet{\sigma'}{M}{X}{M'}}{\sigma'}
}
\end{equation*}

To be able to refer to the constructor component of a module, $M$, within a constructor appearing elsewhere, we need the constructor projection form, $\amcon{M}$. The kinding rule for this form is reproduced below:
\begin{equation*}\tag{\ref{rule:haskind-stat}}
\inferrule{
  \ismvalX{M}\\
  \hassigX{M}{\asignature{\kappa}{u}{\tau}}
}{
  \haskindX{\amcon{M}}{\kappa}
}
\end{equation*}
Similarly, the value projection form, $\amval{M}$, projects out the value component of $M$ within an expression. The typing rule for this form is reproduced below:
\begin{equation*}\tag{\ref{rule:hastypeP-dyn}}
\inferrule{
  \ismvalX{M}\\
  \hassigX{M}{\asignature{\kappa}{u}{\tau}}
}{
  \hastypeP{\Omega}{\amval{M}}{[\amcon{M}/u]\tau}
}
\end{equation*}
The first premise of both of these rules requires that $M$ be, or stand for, a \emph{module value}, according to the following rules:
\begin{equation*}\tag{\ref{rule:ismval-struct}}
\inferrule{ }{
  \ismvalX{\astruct{c}{e}}
}
\end{equation*}
\begin{equation*}\tag{\ref{rule:ismval-var}}
\inferrule{ }{
  \ismval{\Omega, X : \sigma}{X}
}
\end{equation*}
The reason for this restriction has to do with the \emph{sealing} operation:
\begin{equation*}\tag{\ref{rule:hassig-seal}}
\inferrule{
  \issigX{\sigma}\\
  \hassigX{M}{\sigma}
}{
  \hassigX{\aseal{\sigma}{M}}{\sigma}
}
\end{equation*}
Sealing is intended to support \emph{representation independence}, i.e. to ensure that the abstract constructor components of a sealed module are not equivalent to those of any other sealed module. In other words, sealing is \emph{generative}. The module value restriction above achieves this behavior by simple syntactic means -- a sealed module is not a module value, so it has to first be bound to a module variable. Different bindings, even if of the same underlying module, therefore induce distinct constructor projections.

The judgements above obey standard lemmas, including Weakening, Substitution and Decomposition (see Appendix \ref{appendix:P-statics}.)

We omit certain features of the ML module system in  $\miniVerseParam$, such as its support for hierarchical modules and functors. Our formulation also does not support ``width'' subtyping and subkinding for simplicity. These are straightforward extensions of $\miniVerseParam$, but because their inclusion would not change the semantics of parametric TSMs, we did not include them (see \cite{pfpl} for a discussion of these features.)

\subsection{Structural Dynamics}
The structural dynamics of modules is defined as a transition system, and is organized around judgements of the following form:

\vspace{10px}
$\begin{array}{ll}
\textbf{Judgement Form} & \textbf{Description}\\
\stepsU{M}{M'} & \text{$M$ transitions to $M'$}\\
\isvalP{M} & \text{$M$ is a module value}\\
\matchfail{M} & \text{$M$ raises match failure}
\end{array}$
\vspace{10px}

The structural dynamics of expressions is also defined as a transition system, and is organized around judgements of the following form:

\vspace{10px}
$\begin{array}{ll}
\textbf{Judgement Form} & \textbf{Description}\\
\stepsU{e}{e'} & \text{$e$ transitions to $e'$}\\
\isvalP{e} & \text{$e$ is a value}\\
\matchfail{e} & \text{$e$ raises match failure}
\end{array}$
\vspace{10px}

We also define auxiliary judgements for \emph{iterated transition}, $\multistepU{e}{e'}$, and \emph{evaluation}, $\evalU{e}{e'}$ of expressions.

\begingroup
\def\thetheorem{\ref{defn:iterated-transition-P}}
\begin{definition}[Iterated Transition] Iterated transition, $\multistepU{e}{e'}$, is the reflexive, transitive closure of the transition judgement, $\stepsU{e}{e'}$.\end{definition}
% \addtocounter{theorem}{-1}
\endgroup

\begingroup
\def\thetheorem{\ref{defn:evaluation-P}}
\begin{definition}[Evaluation] $\evalU{e}{e'}$ iff $\multistepU{e}{e'}$ and $\isvalU{e'}$. \end{definition}
% \addtocounter{theorem}{-1}
\endgroup

As in previous chapters, our subsequent developments do not make mention of particular rules in the dynamics, nor do they make mention of other judgements, not listed above, that are used only for defining the dynamics of the match operator, so we do not produce these details here. Instead, it suffices to state the following conditions.

The Preservation condition ensures that evaluation preserves typing.
\begingroup
\def\thetheorem{\ref{condition:preservation-P}}
\begin{condition}[Preservation] ~
\begin{enumerate}
\item If $\hassig{}{M}{\sigma}$ and $\stepsU{M}{M'}$ then $\hassig{}{M}{\sigma}$.
\item If $\hastypeUC{e}{\tau}$ and $\stepsU{e}{e'}$ then $\hastypeUC{e'}{\tau}$.
\end{enumerate}
\end{condition}
\endgroup

The Progress condition ensures that evaluation of a well-typed expanded expression cannot ``get stuck''. We must consider the possibility of match failure in this condition.
\begingroup
\def\thetheorem{\ref{condition:progress-P}}
\begin{condition}[Progress] ~
\begin{enumerate}
\item If $\hassig{}{M}{\sigma}$ then either $\isvalU{M}$ or $\matchfail{M}$ or there exists an $M'$ such that $\stepsU{M}{M'}$.
\item If $\hastypeUC{e}{\tau}$ then either $\isvalU{e}$ or $\matchfail{e}$ or there exists an $e'$ such that $\stepsU{e}{e'}$.
\end{enumerate}
\end{condition}
% \addtocounter{theorem}{-1}
\endgroup

\subsection{Syntax of the Unexpanded Language}
The syntax of the unexpanded language is defined in Figure \ref{fig:P-unexpanded-modules-signatures} through Figure \ref{fig:P-macro-expressions-types}.

Each expanded form, except for constructors of the form $\amcon{M}$ and expressions of the form $\amval{M}$ where $M$ is not a module variable, has a corresponding unexpanded form. We refer to these as the \emph{common forms}. The correspondence is defined in Appendix \ref{appendix:P-shared-forms}. (Projection out of a module expression of a form other than the module variable form is supported in the XL only because this is needed to give the language  a conventional structural dynamics. Programmers refer to modules exclusively through module variables in the code that they write.) In addition to the common forms, there are several forms related to pTSMs, highlighted in yellow in these figures.

There is also a context-free textual syntax for the UL. For our purposes, we need only posit the existence of partial metafunctions that satisfy the following condition. 
\begingroup
\def\thetheorem{\ref{condition:textual-representability-P}}
\begin{condition}[Textual Representability] All of the following must hold:
\begin{enumerate}
% \item For each $\usigma$, there exists $b$ such that $\parseUSig{b}{\usigma}$.
% \item For each $\uM$, there exists $b$ such that $\parseUMod{b}{\uM}$.
\item For each $\ukappa$, there exists $b$ such that $\parseUKind{b}{\ukappa}$.
\item For each $\uc$, there exists $b$ such that $\parseUCon{b}{\uc}$.
\item For each $\ue$, there exists $b$ such that $\parseUExp{b}{\ue}$.
\item For each $\upv$, there exists $b$ such that $\parseUPat{b}{\upv}$.
\end{enumerate}
\end{condition}
\endgroup

\begin{figure}[p] \vspace{-15px}
$\arraycolsep=4pt\begin{array}{lllllll}
\textbf{Sort} & & 
%& \textbf{Operational Form} 
& \textbf{Stylized Form} & \textbf{Description}\\
\mathsf{USig} & \usigma & ::= 
%& \ausignature{\ukappa}{\uu}{\utau} 
& \signature{\uu}{\ukappa}{\utau} & \text{signature}\\
\mathsf{UMod} & \uM & ::= 
%& \uX 
& \uX & \text{module identifier}\\
&&
%& \austruct{\uc}{\ue} 
& \struct{\uc}{\ue} & \text{structure}\\
&&
%& \auseal{\usigma}{\uM} 
& \seal{\uM}{\usigma} & \text{seal}\\
&&
%& \aumlet{\usigma}{\uM}{\uX}{\uM} 
& \mlet{\uX}{\uM}{\uM}{\usigma} & \text{definition}\\
\LCC &&
%& \lightgray 
& \color{Yellow} & \color{Yellow}\\
&&
%& \aumdefpetsm{\urho}{e}{\tsmv}{\uM} 
& \defpetsm{\tsmv}{\urho}{e}{\uM} & \text{peTSM definition}\\
%&&&                                    & \texttt{expressions}~\{e\}~\texttt{in}~\uM\\
&&
%& \aumletpetsm{\uepsilon}{\tsmv}{\uM} 
& \uletpetsm{\tsmv}{\uepsilon}{\uM} & \text{peTSM binding}\\
% &&&                                  & \texttt{expressions}~\texttt{in}~\uM\\
% &&& ... & ... & \text{peTSM designation}\\
&&
%& \audefpptsm{\urho}{e}{\tsmv}{\uM} 
& \defpptsm{\tsmv}{\urho}{e}{\uM} & \text{ppTSM definition}\\
% &&&                                    & \texttt{patterns}~\{e\}~\texttt{in}~\uM\\
&&
%& \auletpptsm{\uepsilon}{\tsmv}{\uM} 
& \uletpptsm{\tsmv}{\uepsilon}{\uM} & \text{ppTSM binding}\ECC%
% &&& & \texttt{patterns}~\texttt{in}~\uM\\
% &&& ... & ... & \text{ppTSM designation}\ECC
\end{array}$\vspace{-5px}
\caption[Syntax of unexpanded module expressions and signatures in $\miniVerseParam$]{Syntax of unexpanded module expressions and signatures in $\miniVerseParam$.}\vspace{-5px}
\label{fig:P-unexpanded-modules-signatures}
\end{figure}
\begin{figure}[p] \vspace{-10px}
\[\begin{array}{lrlllll}
\textbf{Sort} & & 
%& \textbf{Operational Form} 
& \textbf{Stylized Form} & \textbf{Description}\\
\mathsf{UKind} & \ukappa & ::= 
%& \aukdarr{\ukappa}{\uu}{\ukappa} 
& \kdarr{\uu}{\ukappa}{\ukappa} & \text{dependent function}\\
&&
%& \aukunit 
& \kunit & \text{nullary product}\\
&&
%& \aukdbprod{\ukappa}{\uu}{\ukappa} 
& \kdbprod{\uu}{\ukappa}{\ukappa} & \text{dependent product}\\
%&&& \akdprodstd & \kdprodstd & \text{labeled dependent product}\\
&&
%& \aukty 
& \kty & \text{types}\\
&&
%& \auksing{\utau} 
& \ksing{\utau} & \text{singleton}\\
\mathsf{UCon} & \uc, \utau & ::= 
%& \uu 
& \uu & \text{constructor identifier}\\
&&
%& \ut 
& \ut & \\
&&
%& \aucasc{\ukappa}{\uc} 
& \casc{\uc}{\ukappa} & \text{ascription}\\
&&
%& \aucabs{\uu}{\uc} 
& \cabs{\uu}{\uc} & \text{abstraction}\\
&&
%& \aucapp{c}{c} 
& \capp{c}{c} & \text{application}\\
&&
%& \auctriv 
& \ctriv & \text{trivial}\\
&&
%& \aucpair{\uc}{\uc} 
& \cpair{\uc}{\uc} & \text{pair}\\
&&
%& \aucprl{\uc} 
& \cprl{\uc} & \text{left projection}\\
&&
%& \aucprr{\uc} 
& \cprr{\uc} & \text{right projection}\\
%&&& \adtplX & \dtplX & \text{labeled dependent tuple}\\
%&&& \adprj{\ell}{c} & \prj{c}{\ell} & \text{projection}\\
&&
%& \auparr{\utau}{\utau} 
& \parr{\utau}{\utau} & \text{partial function}\\
&&
%& \auallu{\ukappa}{\uu}{\utau} 
& \forallu{\uu}{\ukappa}{\utau} & \text{polymorphic}\\
&&
%& \aurec{\ut}{\utau} 
& \rect{\ut}{\utau} & \text{recursive}\\
&&
%& \auprod{\labelset}{\mapschema{\utau}{i}{\labelset}} 
& \prodt{\mapschema{\utau}{i}{\labelset}} & \text{labeled product}\\
&&
%& \ausum{\labelset}{\mapschema{\utau}{i}{\labelset}} 
& \sumt{\mapschema{\utau}{i}{\labelset}} & \text{labeled sum}\\
&&
%& \aumcon{\uX} 
& \mcon{\uX} & \text{constructor component}
\end{array}\]\vspace{-5px}
\caption[Syntax of unexpanded kinds and constructors in $\miniVerseParam$]{Syntax of unexpanded kinds and constructors in $\miniVerseParam$.}\vspace{-10px}
\label{fig:P-unexpanded-kinds-constructors}
\end{figure}

% \clearpage
\begin{figure}[p]
\[\begin{array}{lllllll}
\textbf{Sort} & & 
%& \textbf{Operational Form} 
& \textbf{Stylized Form} & \textbf{Description}\\
\mathsf{UExp} & \ue & ::= 
%& \ux 
& \ux & \text{identifier}\\
% &&
%& \auasc{\utau}{\ue} 
% & \asc{\ue}{\utau} & \text{ascription}\\
% &&
%& \auletsyn{\ux}{\ue}{\ue} 
% & \letsyn{\ux}{\ue}{\ue} & \text{value binding}\\
% &&
%& \auanalam{\ux}{\ue} 
% & \analam{\ux}{\ue} & \text{abstraction (unannotated)}\\
&&
%& \aulam{\utau}{\ux}{\ue} 
& \lam{\ux}{\utau}{\ue} & \text{abstraction}\\
&&
%& \auap{\ue}{\ue} 
& \ap{\ue}{\ue} & \text{application}\\
&&
%& \auclam{\ukappa}{\uu}{\ue} 
& \clam{\uu}{\ukappa}{\ue} & \text{constructor abstraction}\\
&&
%& \aucap{\ue}{\uc} 
& \cAp{\ue}{\uc} & \text{constructor application}\\
&&
%& \auanafold{\ue} 
& \fold{\ue} & \text{fold}\\
&&
%& \auunfold{\ue} 
& \unfold{\ue} & \text{unfold}\\
&&
%& \autpl{\labelset}{\mapschema{\ue}{i}{\labelset}} 
& \tpl{\mapschema{\ue}{i}{\labelset}} & \text{labeled tuple}\\
&&
%& \aupr{\ell}{\ue} 
& \prj{\ue}{\ell} & \text{projection}\\
&&
%& \auanain{\ell}{\ue} 
& \inj{\ell}{\ue} & \text{injection}\\
&&
%& \aumatchwithb{n}{\ue}{\seqschemaX{\urv}} 
& \matchwith{\ue}{\seqschemaX{\urv}} & \text{match}\\
&&
%& \aumval{\uX} 
& \mval{\uX} & \text{value component}\\
\LCC &&
% %& \color{Yellow} 
& \color{Yellow} & \color{Yellow} \\
% &&& \audefpetsm{\urho}{e}{\tsmv}{\ue} & \texttt{syntax}~\tsmv~\texttt{at}~\urho~\texttt{for} & \text{peTSM definition}\\
% &&&                                    & \texttt{expressions}~\{e\}~\texttt{in}~\ue\\
% &&& \auletpetsm{\uepsilon}{\tsmv}{\ue} & \texttt{let}~\texttt{syntax}~\tsmv=\uepsilon~\texttt{for} & \text{peTSM binding}\\
% &&&                                  & \texttt{expressions}~\texttt{in}~\ue\\
% &&& ... & ... & \text{peTSM designation}\\
&&
%& \auappetsm{b}{\uepsilon} 
& \utsmap{\uepsilon}{b} & \text{peTSM application}\ECC\\%\ECC
% &&& \auelit{b} & {\lit{b}}  & \text{peTSM unadorned literal}\\
% &&& \audefpptsm{\urho}{e}{\tsmv}{\ue} & \texttt{syntax}~\tsmv~\texttt{at}~\urho~\texttt{for} & \text{ppTSM definition}\\
% &&&                                    & \texttt{patterns}~\{e\}~\texttt{in}~\ue\\
% &&& \auletpptsm{\uepsilon}{\tsmv}{\ue} & \texttt{let}~\texttt{syntax}~\tsmv=\uepsilon~\texttt{for} & \text{ppTSM binding}\\
% &&& & \texttt{patterns}~\texttt{in}~\ue\\
% &&& ... & ... & \text{ppTSM designation}\\\ECC
\mathsf{URule} & \urv & ::= 
%& \aumatchrule{\upv}{\ue} 
& \matchrule{\upv}{\ue} & \text{match rule}\\
\mathsf{UPat} & \upv & ::= 
%& \ux 
& \ux & \text{identifier pattern}\\
&&
%& \auwildp 
& \wildp & \text{wildcard pattern}\\
&&
%& \aufoldp{\upv} 
& \foldp{\upv} & \text{fold pattern}\\
&&
%& \autplp{\labelset}{\mapschema{\upv}{i}{\labelset}} 
& \tplp{\mapschema{\upv}{i}{\labelset}} & \text{labeled tuple pattern}\\
&&
% & \auinjp{\ell}{\upv} 
& \injp{\ell}{\upv} 
& \text{injection pattern}\\
\LCC &&
%& \lightgray 
& \color{Yellow} & \color{Yellow}\\
&&
%& \auappptsm{b}{\uepsilon} 
& \utsmap{\uepsilon}{b} & \text{ppTSM application}\ECC
% &&& \auplit{b} & \lit{b} & \text{ppTSM unadorned literal}\ECC
\end{array}\]
\caption[Syntax of unexpanded expressions, rules and patterns in $\miniVerseParam$]{Syntax of unexpanded expressions, rules and patterns in $\miniVerseParam$.}
\label{fig:P-unexpanded-terms}
\end{figure}

% \clearpage


\subsection{Typed Expansion}

\begin{figure}[p]
\[\begin{array}{lllllll}
\textbf{Sort} & & 
%& \textbf{Operational Form} 
& \textbf{Stylized Form} 
& \textbf{Description}\\
\LCC \color{Yellow}&\color{Yellow}& \color{Yellow}
%& \lightgray 
& \color{Yellow} & \color{Yellow}\\
\mathsf{UMType} & \urho & ::= 
%& \autype{\utau} 
& \utau & \text{type annotation}\\
&&
%& \aualltypes{\ut}{\urho} 
& \alltypes{\ut}{\urho} & \text{type parameterization}\\
&&
%& \auallmods{\usigma}{\uX}{\urho} 
& \allmods{\uX}{\usigma}{\urho} & \text{module parameterization}\\
\mathsf{UMExp} & \uepsilon & ::= 
%& \abindref{\tsmv} 
& \tsmv & \text{TSM binding reference}\\
&&
%& \auabstype{\ut}{\uepsilon} 
& \abstype{\ut}{\uepsilon} & \text{type abstraction}\\
&&
%& \auabsmod{\usigma}{\uX}{\uepsilon} 
& \absmod{\uX}{\usigma}{\uepsilon} & \text{module abstraction}\\
&&
%& \auaptype{\utau}{\uepsilon} 
& \aptype{\uepsilon}{\utau} & \text{type application}\\
&&
%& \auapmod{\uM}{\uepsilon} 
& \apmod{\uepsilon}{\uX} & \text{module application}\ECC
\end{array}
\]
\caption{Syntax of unexpanded TSM types and expressions.}
\label{fig:P-macro-expressions-types-u}
\end{figure}
\begin{figure}[t]
\[\begin{array}{lllllll}
\textbf{Sort} & & & \textbf{Operational Form} 
%& \textbf{Stylized Form} 
& \textbf{Description}\\
\LCC \color{Yellow}&\color{Yellow}& \color{Yellow}
%& \lightgray 
& \color{Yellow} & \color{Yellow}\\
\mathsf{MType} & \rho & ::= & \aetype{\tau} 
%& \tau 
& \text{type annotation}\\
&&& \aealltypes{t}{\rho} 
%& \alltypes{t}{\rho} 
& \text{type parameterization}\\
&&& \aeallmods{\sigma}{X}{\rho} 
%& \allmods{X}{\sigma}{\rho} 
& \text{module parameterization}\\
\mathsf{MExp} & \epsilon & ::= & \adefref{a} 
%& a 
& \text{TSM definition reference}\\
&&& \aeabstype{t}{\epsilon} 
%& \abstype{t}{\epsilon} 
& \text{type abstraction}\\
&&& \aeabsmod{\sigma}{X}{\epsilon} 
%& \absmod{X}{\sigma}{\epsilon} 
& \text{module abstraction}\\
&&& \aeaptype{\tau}{\epsilon} 
%& \aptype{\epsilon}{\tau} 
& \text{type application}\\
&&& \aeapmod{M}{\epsilon} 
%& \aptype{\epsilon}{M} 
& \text{module application}\ECC
\end{array}\]
\caption[Syntax of TSM types and expressions in $\miniVerseParam$]{Syntax of TSM types and expressions.}
\label{fig:P-macro-expressions-types}
\end{figure}
Typed expansion is defined by three groups of judgements.

The first group of judgements defines signature and module expansion.

\vspace{10px}
$\begin{array}{ll}
\textbf{Judgement Form} & \textbf{Description}\\
\sigExpandsPX{\usigma}{\sigma} & \text{$\usigma$ has well-formed expansion $\sigma$}\\
\mExpandsPX{\uM}{M}{\sigma} & \text{$\uM$ has expansion $M$ matching $\sigma$}
\end{array}$
\vspace{10px}

The second group of judgements defines kind and constructor expansion.

\vspace{10px}
$\begin{array}{ll}
\textbf{Judgement Form} & \textbf{Description}\\
\kExpandsX{\ukappa}{\kappa} & \text{$\ukappa$ has well-formed expansion $\kappa$}\\
\cExpandsX{\uc}{c}{\kappa} & \text{$\uc$ has expansion $c$ of kind $\kappa$}
\end{array}$
\vspace{10px}

The third group of judgements defines expression, rule and pattern expansion.

\vspace{10px}
$\begin{array}{ll}
\textbf{Judgement Form} & \textbf{Description}\\
% \tExpandsPX{\utau}{\tau} & \text{$\utau$ has well-formed expansion $\tau$}\\
\expandsPX{\ue}{e}{\tau} & \text{$\ue$ has expansion $e$ of type $\tau$}\\
% \eanaPX{\ue}{e}{\tau} & \text{$\ue$ has expansion $e$ when analyzed against type $\tau$}\\
\rExpandsSP{\uOmega}{\uPsi}{\uPhi}{\urv}{r}{\tau}{\tau'} & \text{$\urv$ has expansion $r$ taking values of type $\tau$ to values of type $\tau'$}\\
% & \text{synthesized type $\tau'$}\\
% \ranaPX{\urv}{r}{\tau}{\tau'} & \text{$\urv$ has expansion $r$ and takes values of type $\tau$ to values of}\\
% & \text{type $\tau'$ when $\tau's$ is provided for analysis}\\
\patExpandsP{\uOmega}{\uPhi}{\upv}{p}{\tau} & \text{$\upv$ has expansion $p$ matching at $\tau$  generating hypotheses $\uOmega$}
\end{array}$
\vspace{10px}
 
An \emph{unexpanded unified context}, $\uOmega$, takes the form $\uOmegaEx{\uD}{\uG}{\uMctx}{\Omega}$, where $\uMctx$ is a \emph{module identifier expansion context}, $\uD$ is a \emph{constructor identifier expansion context}, $\uG$ is an \emph{expression identifier expansion context} and $\Omega$ is a unified context. Identifier expansion contexts are defined in Appendix \ref{appendix:u-unified-ctxs} and operate as described in Sec. \ref{sec:miniVerseU}, mapping identifiers to variables.

The judgements above are defined by the rules given in Appendix \ref{appendix:typed-expansion-P}. Most of these rules simply serve to ``mirror'' corresponding typing rules, as was described in Sec. \ref{sec:miniVerseU}. The interesting rules, governing the forms highlighted in yellow above, will be reproduced as we discuss them below.

\subsection{TSM Definitions}
For simplicity in $\miniVerseParam$, TSMs are defined only at the level of module expressions. (Adding support for TSM definitions scoped to a single expression would be a straightforward exercise, so we omit the details.)

\subsubsection{peTSM Definitions}
The rule governing peTSM definitions is reproduced below:
\begin{equation*}\tag{\ref{rule:mExpandsP-syntaxpe}}
\inferrule{
  \tsmtyExpands{\uOmega}{\urho}{\rho}\\
  \hastypeP{\emptyset}{\eparse}{\aparr{\tBody}{\tParseResultPCEExp}}\\\\
  \evalU{\eparse}{\eparse'}\\
  \mExpandsP{\uOmega}{\uAS{\uA \uplus \mapitem{\tsmv}{\adefref{a}}}{\Psi, \petsmdefn{a}{\rho}{\eparse'}}}{\uPhi}{\uM}{M}{\sigma}
}{
  \mExpandsP{\uOmega}{\uAS{\uA}{\Psi}}{\uPhi}{\defpetsm{\tsmv}{\urho}{\eparse}{\uM}}{M}{\sigma}
}
\end{equation*}

peTSM definitions differ from ueTSM definitions in that the unexpanded type annotation is an \emph{unexpanded TSM type}, $\urho$, rather than an unexpanded type, $\utau$. This unexpanded TSM type defines the parameterization of the TSM. The first premise of the rule above expands the unexpanded TSM type to produce a \emph{TSM type}, $\rho$. The straightforward TSM type expansion rules are reproduced below.
\begin{equation*}\tag{\ref{rule:tsmtyExpands-type}}
\inferrule{
  \cExpandsX{\utau}{\tau}{\akty}
}{
  \tsmtyExpands{\uOmega}{{\utau}}{\aetype{\tau}}
}
\end{equation*}
\begin{equation*}\tag{\ref{rule:tsmtyExpands-alltypes}}
\inferrule{
  \tsmtyExpands{\uOmega, \uKhyp{\ut}{t}{\akty}}{\urho}{\rho}
}{
  \tsmtyExpands{\uOmega}{\alltypes{\ut}{\urho}}{\aealltypes{t}{\rho}}
}
\end{equation*}
\begin{equation*}\tag{\ref{rule:tsmtyExpands-allmods}}
\inferrule{
  \sigExpandsPX{\usigma}{\sigma}\\
  \tsmtyExpands{\uOmega, \uMhyp{\uX}{X}{\sigma}}{\urho}{\rho}
}{
  \tsmtyExpands{\uOmega}{\allmods{\uX}{\usigma}{\urho}}{\aeallmods{\sigma}{X}{\rho}}
}
\end{equation*}
(Strictly speaking, quantification over types is redundant here -- we could simply use a module with a single type component in such situations. We include this rule as a guide for languages that support type functions without a full ML-style module system.)

The second premise of Rule (\ref{rule:mExpandsP-syntaxpe}) checks that the parse function is of the appropriate type. The types $\tBody$ and $\tParseResultPCEExp$ are characterized in Appendix \ref{appendix:typed-expansion-P}. The type $\tPProtoExpr$ classifies \emph{encodings of parameterized proto-expressions}, which we will return to when we discuss TSM application below.

The third premise of Rule (\ref{rule:mExpandsP-syntaxpe}) evaluates the parse function to a value.

The final premise of Rule (\ref{rule:mExpandsP-syntaxpe}) extends the \emph{peTSM context}, $\uPsi$, which consists of a \emph{TSM identifier expansion context}, $\uA$, and a \emph{peTSM definition context}, $\Psi$. A peTSM definition context maps TSM names, $a$, to an expanded peTSM definition, $\petsmdefn{a}{\rho}{\eparse}$, where $\rho$ is the TSM type annotation and $\eparse$ is its parse function. A TSM identifier context maps TSM identifiers, $\tsmv$, to \emph{TSM expressions}, $\epsilon$. In this case, the TSM expression is simply a reference to the newly introduced TSM definition, $\adefref{a}$. We will discuss the other TSM expression forms when we discuss TSM abbreviations below.

\subsubsection{ppTSM Definitions}
The rule governing ppTSM definitions is similar, and is reproduced below:
\begin{equation*}\tag{\ref{rule:mExpandsP-syntaxpp}}
\inferrule{ 
  \tsmtyExpands{\uOmega}{\urho}{\rho}\\
  \hastypeP{\emptyset}{\eparse}{\aparr{\tBody}{\tParseResultCEPat }}\\\\
  \evalU{\eparse}{\eparse'}\\
  \mExpandsP{\uOmega}{\uPsi}{\uAS{\uA \uplus \mapitem{\tsmv}{\adefref{a}}}{\Phi, \pptsmdefn{a}{\rho}{\eparse'}}}{\uM}{M}{\sigma}
}{
  \mExpandsP{\uOmega}{\uPsi}{\uAS{\uA}{\Phi}}{\defpptsm{\tsmv}{\urho}{\eparse}{\uM}}{M}{\sigma}
}
\end{equation*}
This rule differs from Rule (\ref{rule:mExpandsP-syntaxpe}) only in the type of the parse function and in the fact that the \emph{ppTSM context}, $\uPhi$, rather than the peTSM context, is extended.

\subsection{TSM Abbreviations}
It is also possible to abbreviate a TSM expression by binding it to a TSM identifier at the level of modules.

\subsubsection{peTSM Abbreviations}
The rule governing peTSM abbreviations is reproduced below:
\begin{equation}\tag{\ref{rule:mExpandsP-letpetsm}}
\inferrule{
  \tsmexpExpandsExp{\uOmega}{\uAS{\uA}{\Psi}}{\uepsilon}{\epsilon}{\rho}\\
  \mExpandsP{\uOmega}{\uAS{\uA\uplus\mapitem{\tsmv}{\epsilon}}{\Psi}}{\uPhi}{\uM}{M}{\sigma}
}{
  \mExpandsP{\uOmega}{\uAS{\uA}{\Psi}}{\uPhi}{\uletpetsm{\tsmv}{\uepsilon}{\uM}}{M}{\sigma}
}
\end{equation}
Here, $\uepsilon$ is an \emph{unexpanded TSM expression}. The first premise of the rule above expands it, producing a TSM expression $\epsilon$ at TSM type $\rho$. The second premise updates the peTSM identifier expansion context with this TSM expression.

The rules below govern peTSM expression expansion. The first rule handles the base case, when the unexpanded TSM expression is a TSM identifier, $\tsmv$, by looking it up in $\uA$ and determining its TSM type according to the TSM expression typing judgement, $\hastsmtypeExp{\Omega}{\Psi}{\epsilon}{\rho}$ (which mirrors the rules below, and is defined in Appendix \ref{appendix:typed-expansion-P}.)
\begin{equation*}\tag{\ref{rule:tsmexpExpandsExp-bindref}}
\inferrule{
  \hastsmtypeExp{\Omega}{\Psi}{\epsilon}{\rho}  
}{
  \tsmexpExpandsExp{\uOmegaEx{\uD}{\uG}{\uMctx}{\Omega}}{\uAS{\uA, \mapitem{\tsmv}{\epsilon}}{\Psi}}{{\tsmv}}{\epsilon}{\rho}
}
\end{equation*}

The following two rules allow a TSM expression to itself abstract over a type or module. (This is necessary to support abbreviated application of parameters other than the first.)
\begin{equation*}\tag{\ref{rule:tsmexpExpandsExp-abstype}}
\inferrule{
  \tsmexpExpandsExp{\uOmega, \uKhyp{\ut}{t}{\akty}}{\uPsi}{\uepsilon}{\epsilon}{\rho}
}{
  \tsmexpExpandsExp{\uOmega}{\uPsi}{\abstype{\ut}{\uepsilon}}{\aeabstype{t}{\epsilon}}{\aealltypes{t}{\rho}}
}
\end{equation*}
\begin{equation*}\tag{\ref{rule:tsmexpExpandsExp-absmod}}
\inferrule{
  \sigExpandsPX{\usigma}{\sigma}\\
  \tsmexpExpandsExp{\uOmega, \uMhyp{\uX}{X}{\sigma}}{\uPsi}{\uepsilon}{\epsilon}{\rho}
}{
  \tsmexpExpandsExp{\uOmega}{\uPsi}{\absmod{\uX}{\usigma}{\uepsilon}}{\aeabsmod{\sigma}{X}{\epsilon}}{\aeallmods{\sigma}{X}{\rho}}
}
\end{equation*}

The final two rules define the semantics of parameter application.
\begin{equation*}\tag{\ref{rule:tsmexpExpandsExp-aptype}}
\inferrule{
  \tsmexpExpandsExp{\uOmega}{\uPsi}{\uepsilon}{\epsilon}{\aealltypes{t}{\rho}}\\
  \cExpandsX{\utau}{\tau}{\akty}
}{
  \tsmexpExpandsExp{\uOmega}{\uPsi}{\aptype{\uepsilon}{\utau}}{\aeaptype{\tau}{\epsilon}}{[\tau/t]\rho} 
}
\end{equation*}
\begin{equation*}\tag{\ref{rule:tsmexpExpandsExp-apmod}}
\inferrule{
  \tsmexpExpandsExp{\uOmega}{\uPsi}{\uepsilon}{\epsilon}{\aeallmods{\sigma}{X'}{\rho}}\\
  \mExpandsPX{\uX}{X}{\sigma}
}{
  \tsmexpExpandsExp{\uOmega}{\uPsi}{\apmod{\uepsilon}{\uX}}{\aeapmod{X}{\epsilon}}{[X/X']\rho}
}
\end{equation*}

\subsubsection{ppTSM Abbreviations}
The rule governing ppTSM abbreviations is analagous:
\begin{equation*}\tag{\ref{rule:mExpandsP-letpptsm}}
\inferrule{
  \tsmexpExpandsPat{\uOmega}{\uAS{\uA}{\Phi}}{\uepsilon}{\epsilon}{\rho}\\
  \mExpandsP{\uOmega}{\uPsi}{\uAS{\uA\uplus\mapitem{\tsmv}{\epsilon}}{\Phi}}{\uM}{M}{\sigma}
}{
  \mExpandsP{\uOmega}{\uPsi}{\uAS{\uA}{\Phi}}{\uletpptsm{\tsmv}{\uepsilon}{\uM}}{M}{\sigma}
}
\end{equation*}
The ppTSM expression expansion judgement appearing as the first premise is defined analagously to the peTSM expression expansion judgement defined above, differing only in that the rule for TSM identifiers consults the ppTSM context rather than the peTSM context. The rules are reproduced in Appendix \ref{appendix:typed-expansion-P}.

\subsection{TSM Application}

\subsubsection{peTSM Application}
The rule for applying an unexpanded peTSM expression $\uepsilon$ to a generalized literal form with body $b$ is reproduced below:
\begin{equation*}\tag{\ref{rule:expandsP-apuetsm}}
\inferrule{
  \uOmega = \uOmegaEx{\uD}{\uG}{\uMctx}{\Omega_\text{app}}\\
  \uPsi=\uAS{\uA}{\Psi}\\\\
  \tsmexpExpandsExp{\uOmega}{\uPsi}{\uepsilon}{\epsilon}{\aetype{\tau_\text{final}}}\\
  \tsmexpEvalsExp{\Omega_\text{app}}{\Psi}{\epsilon}{\epsilon_\text{normal}}\\\\
  \tsmdefof{\epsilon_\text{normal}}=a\\
  \Psi = \Psi', \petsmdefn{a}{\rho}{\eparse}\\\\
  \encodeBody{b}{\ebody}\\
  \evalU{\ap{\eparse}{\ebody}}{{\lbltxt{SuccessE}}\cdot{e_\text{pproto}}}\\
  \decodePCEExp{e_\text{pproto}}{\pce}\\\\
  \prepce{\Omega_\text{app}}{\Psi}{\pce}{\ce}{\epsilon_\text{normal}}{\aetype{\tau_\text{proto}}}{\omega}{\Omega_\text{params}}\\\\
  \segOK{\segof{\ce}}{b}\\
  \cvalidEP{\Omega_\text{params}}{\esceneP{\uOmega}{\uPsi}{\uPhi}{b}}{\ce}{e}{\tau_\text{proto}}
}{
  \expandsP{\uOmega}{\uPsi}{\uPhi}{\utsmap{\uepsilon}{b}}{[\omega]e}{[\omega]\tau_\text{proto}}
}
\end{equation*}

The first two premises simply deconstruct $\uOmega$ and $\uPsi$. Next, we expand $\uepsilon$ according to the rules already described above. The resulting TSM expression, $\epsilon$, must be defined at a type (i.e. all parameters must be specified.)

The fourth premise performs \emph{peTSM expression normalization}. Normalization is defined in terms of a simple structural dynamics for peTSM expressions as follows:
\begin{equation*}\tag{\ref{rule:tsmexpEvalsExp}}
\inferrule{
  \tsmexpMultistepsExp{\Omega}{\Psi}{\epsilon}{\epsilon'}\\
  \tsmexpNormalExp{\Omega}{\Psi}{\epsilon'}
}{
  \tsmexpEvalsExp{\Omega}{\Psi}{\epsilon}{\epsilon'}
}
\end{equation*}
where the multistep judgement, $\tsmexpMultistepsExp{\Omega}{\Psi}{\epsilon}{\epsilon'}$, is defined as the reflexive, transitive closure of the stepping judgement defined by the following rules:
\begin{equation*}\tag{\ref{rule:tsmexpStepsExp-aptype-1}}
\inferrule{
  \tsmexpStepsExp{\Omega}{\Psi}{\epsilon}{\epsilon'}
}{
  \tsmexpStepsExp{\Omega}{\Psi}{\aeaptype{\tau}{\epsilon}}{\aeaptype{\tau}{\epsilon'}}
}
\end{equation*}
\begin{equation*}\tag{\ref{rule:tsmexpStepsExp-aptype-2}}
\inferrule{ }{
  \tsmexpStepsExp{\Omega}{\Psi}{\aeaptype{\tau}{\aeabstype{t}{\epsilon}}}{[\tau/t]\epsilon}
}
\end{equation*}
\begin{equation*}\tag{\ref{rule:tsmexpStepsExp-apmod-1}}
\inferrule{
  \tsmexpStepsExp{\Omega}{\Psi}{\epsilon}{\epsilon'}
}{
  \tsmexpStepsExp{\Omega}{\Psi}{\aeapmod{X}{\epsilon}}{\aeapmod{X}{\epsilon'}}
}
\end{equation*}
\begin{equation*}\tag{\ref{rule:tsmexpStepsExp-apmod-2}}
\inferrule{ }{
  \tsmexpStepsExp{\Omega}{\Psi}{\aeapmod{X}{\aeabsmod{\sigma}{X'}{\epsilon}}}{[X/X']\epsilon}
}
\end{equation*}
and the peTSM expression normal forms are defined as follows:
\begin{equation*}\tag{\ref{rule:tsmexpNormalExp-defref}}
\inferrule{ }{
  \tsmexpNormalExp{\Omega}{\Psi, \petsmdefn{a}{\rho}{\eparse}}{\adefref{a}}
}
\end{equation*}
\begin{equation*}\tag{\ref{rule:tsmexpNormalExp-abstype}}
\inferrule{ }{
  \tsmexpNormalExp{\Omega}{\Psi}{\aeabstype{t}{\epsilon}}
}
\end{equation*}
\begin{equation*}\tag{\ref{rule:tsmexpNormalExp-absmod}}
\inferrule{ }{
  \tsmexpNormalExp{\Omega}{\Psi}{\aeabsmod{\sigma}{X}{\epsilon}}
}
\end{equation*}
\begin{equation*}\tag{\ref{rule:tsmexpNormalExp-aptype}}
\inferrule{
  \epsilon \neq \aeabstype{t}{\epsilon'}\\
  \tsmexpNormalExp{\Omega}{\Psi}{\epsilon}
}{
  \tsmexpNormalExp{\Omega}{\Psi}{\aeaptype{\tau}{\epsilon}}
}
\end{equation*}
\begin{equation*}\tag{\ref{rule:tsmexpNormalExp-apmod}}
\inferrule{
  \epsilon \neq \aeabsmod{\sigma}{X'}{\epsilon'}\\
  \tsmexpNormalExp{\Omega}{\Psi}{\epsilon}
}{
  \tsmexpNormalExp{\Omega}{\Psi}{\aeapmod{X}{\epsilon}}
}
\end{equation*}
Normalization leaves only those parameters specified directly by the TSM definition at the root of the TSM expression (eliminating those specified within $\epsilon$ due, for example, to a parameterized TSM abbreviation.) 

The TSM definition at the root of the normalized TSM expression is extracted by the third row of premises in Rule (\ref{rule:expandsP-apuetsm}), which first appeal to the following metafunction to produce the TSM definition's name, and then look up this name within $\Psi$.
\begin{align}
\tsmdefof{\adefref{a}} & = a \tag{\ref{eqn:tsmdefof-adefref}}\\
\tsmdefof{\aeabstype{t}{\epsilon}} & = \tsmdefof{\epsilon} \tag{\ref{eqn:tsmdefof-abstype}}\\
\tsmdefof{\aeabsmod{\sigma}{X}{\epsilon}} & = \tsmdefof{\epsilon} \tag{\ref{eqn:tsmdefof-absmod}}\\
\tsmdefof{\aeaptype{\tau}{\epsilon}} & = \tsmdefof{\epsilon} \tag{\ref{eqn:tsmdefof-aptype}}\\
\tsmdefof{\aeapmod{X}{\epsilon}} & = \tsmdefof{\epsilon} \tag{\ref{eqn:tsmdefof-apmod}}
\end{align}

The fourth row of premises in Rule (\ref{rule:expandsP-apuetsm}) encode the body as a value of the type $\tBody$, then apply the parse function and decode the result, producing a \emph{parameterized proto-expression}, $\pce$. The relevant judgements are characterized in Appendix \ref{appendix:typed-expansion-P}. 

Parameterized proto-expressions, $\pce$, are ABTs that serve to introduce the parameter bindings into a proto-expression, $\ce$. The operational and stylized syntax of parameterized proto-expression is given in Figure \ref{fig:P-pceexp}. (VerseML inserts these binders automatically, but it is technically simpler to assume that they appear explicitly in this way.)

\begin{figure}[h]
\[\begin{array}{lllllll}
\textbf{Sort} & & & \textbf{Operational Form} & \textbf{Stylized Form} & \textbf{Description}\\
\LCC \color{Yellow}&\color{Yellow}&\color{Yellow}& \color{Yellow} & \color{Yellow} & \color{Yellow}\\
\mathsf{PPrExpr} & \pce & ::= & \apceexp{\ce} & \pceexp{\ce} & \text{proto-expression}\\
&&& \apcebindtype{t}{\pce} & \pcebindtype{t}{\pce} & \text{type binding}\\
&&& \apcebindmod{X}{\pce} & \pcebindmod{X}{\pce} & \text{module binding}\ECC
\end{array}\]
\caption[Syntax of parameterized proto-expressions in $\miniVerseParam$]{Syntax of parameterized proto-expressions.}
\label{fig:P-pceexp}
\end{figure}

The judgement on the fifth row of Rule (\ref{rule:expandsP-apuetsm}) then \emph{deparameterizes} $\pce$ to produce the underlying proto-expression, $\ce$, and deparameterized type, $\tau_\text{proto}$, together with a \emph{substitution} for each of the bindings, $\omega$, based on the parameters that appear in the normalized TSM expression, $\epsilon_\text{normal}$.  The substitution is well-formed with respect to a \emph{parameter context}, $\Omega_\text{params}$. The rules governing proto-expression deparameterization are reproduced below:
\begin{equation*}\tag{\ref{rule:prepce-ceexp}}
\inferrule{ }{
  \prepce{\Omega_\text{app}}{\Psi, \petsmdefn{a}{\rho}{\eparse}}{\apceexp{\ce}}{\ce}{\adefref{a}}{\rho}{\emptyset}{\emptyset}
}
\end{equation*}
\begin{equation*}\tag{\ref{rule:prepce-alltypes}}
\inferrule{
  \prepce{\Omega_\text{app}}{\Psi}{\pce}{\ce}{\epsilon}{\aealltypes{t}{\rho}}{\omega}{\Omega}\\
  t \notin \domof{\Omega_\text{app}}
}{
  \prepce{\Omega_\text{app}}{\Psi}{\apcebindtype{t}{\pce}}{\ce}{\aeaptype{\tau}{\epsilon}}{\rho}{\omega, \tau/t}{\Omega, t :: \akty}
}
\end{equation*}
\begin{equation*}\tag{\ref{rule:prepce-allmods}}
\inferrule{
  \prepce{\Omega_\text{app}}{\Psi}{\pce}{\ce}{\epsilon}{\aeallmods{\sigma}{X}{\rho}}{\omega}{\Omega}\\
  X \notin \domof{\Omega_\text{app}}
}{
  \prepce{\Omega_\text{app}}{\Psi}{\apcebindmod{X}{\pce}}{\ce}{\aeapmod{X'}{\epsilon}}{\rho}{\omega, X'/X}{\Omega, X : \sigma}
}
\end{equation*}

The final row of premises in Rule (\ref{rule:expandsP-apuetsm}) performs proto-expansion validation. This involves first checking that the segmentation of $\ce$ is valid, and then checking that the proto-expansion is well-typed under the parameter context, $\Omega_\text{param}$. This step, which operates essentially as it did in $\miniVersePat$, is detailed below. The uconclusion of the rule applies the parameter substitution, $\omega$, to the resulting proto-expression and its type. 

\subsubsection{ppTSM Application}



The rule governing ppTSM application is similar:
\begin{equation*}\tag{\ref{rule:patExpandsP-apuptsm}}
\inferrule{
  \uOmega=\uOmegaEx{\uD}{\uG}{\uMctx}{\Omega_\text{app}}\\
  \uPhi=\uAS{\uA}{\Phi}\\\\
  \tsmexpExpandsPat{\uOmega}{\uPhi}{\uepsilon}{\epsilon}{\aetype{\tau_\text{final}}}\\
  \tsmexpEvalsPat{\Omega_\text{app}}{\Phi}{\epsilon}{\epsilon_\text{normal}}\\\\
  \tsmdefof{\epsilon_\text{normal}}=a\\
  \Phi = \Phi', \pptsmdefn{a}{\rho}{\eparse}\\\\
  \encodeBody{b}{\ebody}\\
  \evalU{\ap{\eparse}{\ebody}}{{\lbltxt{SuccessP}}\cdot{\ecand}}\\
  \decodeCEPat{\ecand}{\cpv}\\\\
  \prepcp{\Omega_\text{app}}{\Phi}{\epsilon_\text{normal}}{\aetype{\tau_\text{proto}}}{\omega}{\Omega_\text{params}}\\\\
      \segOK{\segof{\cpv}}{b}\\
  \cvalidPP{\uOmega'}{\psceneP{\uOmega}{\uPhi}{b}}{\cpv}{p}{\tau_\text{proto}}
}{
  \patExpandsP{\uOmega'}{\uPhi}{\utsmap{\uepsilon}{b}}{p}{[\omega]\tau_\text{proto}}
}
\end{equation*}

Notice on the third row that the encoding of the proto-expansion, $\ecand$, decodes to a proto-pattern, $\cpv$ -- we do not need the notion of a ``parameterized proto-pattern'' because patterns do not make reference to bindings in the surrounding context. Similarly, the ``deparameterization'' judgement above need not inspect $\cpv$ -- it serves only to generate a substitution, $\omega$, which is applied to the type, $\tau_\text{proto}$, in the conclusion. The necessary auxiliary definitions are given in Appendix \ref{appendix:typed-expansion-P}.

\subsection{Syntax of Proto-Expansions}\label{sec:ce-syntax-P}

\begin{figure}[p] 
\[\begin{array}{lrlllll}
\textbf{Sort} & & & \textbf{Operational Form} & \textbf{Stylized Form} & \textbf{Description}\\
\mathsf{PrKind} & \cekappa & ::= & \acekdarr{\cekappa}{u}{\cekappa} & \kdarr{u}{\cekappa}{\cekappa} & \text{dependent function}\\
&&& \acekunit & \kunit & \text{nullary product}\\
&&& \acekdbprod{\cekappa}{u}{\cekappa} & \kdbprod{u}{\cekappa}{\cekappa} & \text{dependent product}\\
%&&& \akdprodstd & \kdprodstd & \text{labeled dependent product}\\
&&& \acekty & \kty & \text{types}\\
&&& \aceksing{\ctau} & \ksing{\ctau} & \text{singleton}\\
\LCC &&& \color{Yellow} & \color{Yellow} & \color{Yellow}\\
&&& \acesplicedk{m}{n} & \splicedk{m}{n} & \text{spliced kind}\ECC\\
\mathsf{PrCon} & \cec, \ctau & ::= & u & u & \text{constructor variable}\\
&&& t & t & \text{type variable}\\
% &&& \acecasc{\cekappa}{\cec} & \casc{\cec}{\cekappa} & \text{ascription}\\
&&& \acecabs{u}{\cec} & \cabs{u}{\cec} & \text{abstraction}\\
&&& \acecapp{\cec}{\cec} & \capp{\cec}{\cec} & \text{application}\\
&&& \acectriv & \ctriv & \text{trivial}\\
&&& \acecpair{\cec}{\cec} & \cpair{\cec}{\cec} & \text{pair}\\
&&& \acecprl{\cec} & \cprl{\cec} & \text{left projection}\\
&&& \acecprr{\cec} & \cprr{\cec} & \text{right projection}\\
%&&& \adtplX & \dtplX & \text{labeled dependent tuple}\\
%&&& \adprj{\ell}{c} & \prj{c}{\ell} & \text{projection}\\
&&& \aceparr{\ctau}{\ctau} & \parr{\ctau}{\ctau} & \text{partial function}\\
&&& \aceallu{\cekappa}{u}{\ctau} & \forallu{u}{\cekappa}{\ctau} & \text{polymorphic}\\
&&& \acerec{t}{\ctau} & \rect{t}{\ctau} & \text{recursive}\\
&&& \aceprod{\labelset}{\mapschema{\ctau}{i}{\labelset}} & \prodt{\mapschema{\ctau}{i}{\labelset}} & \text{labeled product}\\
&&& \acesum{\labelset}{\mapschema{\ctau}{i}{\labelset}} & \sumt{\mapschema{\ctau}{i}{\labelset}} & \text{labeled sum}\\
&&& \acemcon{X} & \mcon{X} & \text{constructor component}\\
\LCC &&& \color{Yellow} & \color{Yellow} & \color{Yellow}\\
&&& \acesplicedc{m}{n} & \splicedc{m}{n} & \text{spliced constructor}\ECC
\end{array}\]
\caption[Syntax of proto-kinds and proto-constructors in $\miniVerseParam$]{Syntax of proto-kinds and proto-constructors in $\miniVerseParam$.}
\label{fig:P-ce-kinds-constructors}
\end{figure}

\begin{figure}[p]
\[\arraycolsep=4pt\begin{array}{lllllll}
\textbf{Sort} & & & \textbf{Operational Form} & \textbf{Stylized Form} & \textbf{Description}\\
\mathsf{PrExp} & \ce & ::= & x & x & \text{variable}\\
% &&& \aceasc{\ctau}{\ce} & \asc{\ce}{\ctau} & \text{ascription}\\
% &&& \aceletsyn{x}{\ce}{\ce} & \letsyn{x}{\ce}{\ce} & \text{value binding}\\
% &&& \aceanalam{x}{\ce} & \analam{x}{\ce} & \text{abstraction (unannotated)}\\
&&& \acelam{\ctau}{x}{\ce} & \lam{x}{\ctau}{\ce} & \text{abstraction}\\
&&& \aceap{\ce}{\ce} & \ap{\ce}{\ce} & \text{application}\\
&&& \aceclam{\cekappa}{u}{\ce} & \clam{u}{\cekappa}{\ce} & \text{constructor abstraction}\\
&&& \acecap{\ce}{\cec} & \cAp{\ce}{\cec} & \text{constructor application}\\
&&& \acefold{t}{\ctau}{\ce} & \fold{\ce} & \text{fold}\\
&&& \aceunfold{\ce} & \unfold{\ce} & \text{unfold}\\
&&& \acetpl{\labelset}{\mapschema{\ce}{i}{\labelset}} & \tpl{\mapschema{\ce}{i}{\labelset}} & \text{labeled tuple}\\
&&& \acepr{\ell}{\ce} & \prj{\ce}{\ell} & \text{projection}\\
&&& \aceanain{\ell}{\ce} & \inj{\ell}{\ce} & \text{injection}\\
&&& \acematchwith{n}{\ctau}{\ce}{\seqschemaX{\urv}} & \matchwith{\ce}{\seqschemaX{\crv}} & \text{match}\\
&&& \acemval{X} & \mval{X} & \text{value component}\\
\LCC &&& \color{Yellow} & \color{Yellow} & \color{Yellow}\\
&&& \acesplicede{m}{n} & \splicede{m}{n} & \text{spliced expression}\ECC\\
\mathsf{PrRule} & \crv & ::= & \acematchrule{p}{\ce} & \matchrule{p}{\ce} & \text{rule}\\
\mathsf{PrPat} & \cpv & ::= & \acewildp & \wildp & \text{wildcard pattern}\\
&&& \acefoldp{p} & \foldp{p} & \text{fold pattern}\\
&&& \acetplp{\labelset}{\mapschema{\cpv}{i}{\labelset}} & \tplp{\mapschema{\cpv}{i}{\labelset}} & \text{labeled tuple pattern}\\
&&& \aceinjp{\ell}{\cpv} & \injp{\ell}{\cpv} & \text{injection pattern}\\
&&& \acemval{X} & \mval{X} & \text{value component}\\
\LCC &&& \color{Yellow} & \color{Yellow} & \color{Yellow}\\
&&& \acesplicedp{m}{n} & \splicedp{m}{n} & \text{spliced pattern} \ECC
\end{array}\]
\caption[Syntax of proto-expressions, proto-rules and proto-patterns in $\miniVerseParam$]{Syntax of proto-expressions, proto-rules and proto-patterns in $\miniVerseParam$.}
\label{fig:P-candidate-terms}
\end{figure}
Figure \ref{fig:P-ce-kinds-constructors} defines the syntax of proto-kinds, $\cekappa$ and proto-constructors, $\cec$. Figure \ref{fig:P-candidate-terms} defines the syntax of proto-expressions, $\ce$, proto-rules, $\crv$, and proto-patterns, $\cpv$. All of these are ABTs. %The syntax of ce-types is identical to that given in Figure \ref{fig:U-candidate-terms}, which was described in Sec. \ref{sec:ce-syntax-U}. 

The mapping from expanded forms to proto-expansion forms is given in Appendix \ref{appendix:P-proto-expansion-validation}. The only ``interesting'' forms are the forms for references to spliced unexpanded terms, highlighted in yellow in Figure \ref{fig:P-ce-kinds-constructors} and Figure \ref{fig:P-candidate-terms}.

\subsection{Proto-Expansion Validation}
Proto-expansion validation operates essentially as described in Sec. \ref{sec:ce-validation-U}. It is governed by two groups of judgements. The first group of judgements defines proto-kind and proto-constructor validation.

\vspace{10px}\noindent
$\begin{array}{ll}
\textbf{Judgement Form} & \textbf{Description}\\
\cvalidKX{\cekappa}{\kappa} & \text{$\cekappa$ has well-formed expansion $\kappa$}\\
\cvalidCX{\cec}{c}{\kappa} & \text{$\cec$ has expansion $c$ of kind $\kappa$}\\
\end{array}$
\vspace{10px}

The second group of judgements defines proto-expression, proto-rule and proto-pattern validation.

\vspace{10px}\noindent
$\arraycolsep=4pt\begin{array}{ll}
\textbf{Judgement Form} & \textbf{Description}\\
% \cvalidTP{\Omega}{\cscenev}{\ctau}{\tau} & \text{$\ctau$ has expansion $\tau$}\\
\cvalidEPX{\ce}{e}{\tau} & \text{$\ce$ has expansion $e$ of type $\tau$}\\
\cvalidRP{\Omega}{\escenev}{\crv}{r}{\tau}{\tau'} & \text{$\crv$ has expansion $r$ taking values of type $\tau$ to values of type $\tau'$}\\
\cvalidPPE{\uOmega}{\pscenev}{\cpv}{p}{\tau} & \text{$\cpv$ has expansion $p$ matching against $\tau$ generating hypotheses $\uOmega$}
\end{array}$
\vspace{10px}

\emph{Expression splicing scenes}, $\escenev$, are of the form $\esceneP{\uOmega}{\uPsi}{\uPhi}{b}$, \emph{constructor splicing scenes}, $\cscenev$, are of the form $\tsceneUP{\uOmega}{b}$, and \emph{pattern splicing scenes}, $\pscenev$, are of the form $\psceneP{\uOmega}{\uPhi}{b}$. Their purpose is to ``remember'', during proto-expansion validation, the contexts and literal bodies from the TSM application site (cf. Rules (\ref{rule:expandsP-apuetsm}) and (\ref{rule:patExpandsP-apuptsm}) above), because these are necessary to validate references to spliced terms. We write $\csfrom{\escenev}$ for the constructor splicing scene constructed by dropping the TSM contexts from $\escenev$:
\[\csfrom{\esceneP{\uOmega}{\uPsi}{\uPhi}{b}} = \tsceneUP{\uOmega}{b}\]

The rules governing these judgements are reproduced in Appendix \ref{appendix:P-proto-expansion-validation}.

\subsection{Metatheory}
A more detailed account of the metatheory is given in Appendix \ref{appendix:metatheory-P}. We will summarize the key theorems below.

\subsubsection{TSM Expression Evaluation}
The following theorems establish a notion of TSM type safety based on preservation and progress for TSM expression evaluation.

\begingroup
\def\thetheorem{\ref{thm:peTSM-preservation}}
\begin{theorem}[peTSM Preservation]
% \label{thm:peTSM-preservation}
If $\hastsmtypeExp{\Omega}{\Psi}{\epsilon}{\rho}$ and $\tsmexpStepsExp{\Omega}{\Psi}{\epsilon}{\epsilon'}$ then $\hastsmtypeExp{\Omega}{\Psi}{\epsilon'}{\rho}$.
\end{theorem}
\endgroup

\begingroup
\def\thetheorem{\ref{thm:ppTSM-preservation}}
\begin{theorem}[ppTSM Preservation]
% \label{thm:ppTSM-preservation}
If $\hastsmtypePat{\Omega}{\Phi}{\epsilon}{\rho}$ and $\tsmexpStepsPat{\Omega}{\Phi}{\epsilon}{\epsilon'}$ then $\hastsmtypePat{\Omega}{\Phi}{\epsilon'}{\rho}$.
\end{theorem}
\endgroup

\begingroup
\def\thetheorem{\ref{thm:peTSM-progress}}
\begin{theorem}[peTSM Progress]
% \label{thm:peTSM-progress}
If $\hastsmtypeExp{\Omega}{\Psi}{\epsilon}{\rho}$ then either $\tsmexpStepsExp{\Omega}{\Psi}{\epsilon}{\epsilon'}$ for some $\epsilon'$ or $\tsmexpNormalExp{\Omega}{\Psi}{\epsilon}$.
\end{theorem}
\endgroup

\begingroup
\def\thetheorem{\ref{thm:ppTSM-progress}}
\begin{theorem}[ppTSM Progress]
% \label{thm:ppTSM-progress}
If $\hastsmtypePat{\Omega}{\Phi}{\epsilon}{\rho}$ then either $\tsmexpStepsPat{\Omega}{\Phi}{\epsilon}{\epsilon'}$ for some $\epsilon'$ or $\tsmexpNormalPat{\Omega}{\Phi}{\epsilon}$.
\end{theorem}
\endgroup

\subsubsection{Typed Expansion}
There are also a number of theorems that establish that typed expansion generates a well-typed expansion.

The top-level theorem is the typed expansion theorem for modules. 

\begingroup
\def\thetheorem{\ref{thm:module-expansion-P}}
\begin{theorem}[Module Expansion]
% \label{thm:module-expansion-P}
If $\mExpandsP{\uOmegaEx{\uD}{\uG}{\uMctx}{\Omega}}{\uPsi}{\uPhi}{\uM}{M}{\sigma}$ then $\hassig{\Omega}{M}{\sigma}$.
\end{theorem}
\endgroup

(The proof of this theorem requires proving the corresponding theorems about the other typed expansion judgements, see Appendix \ref{appendix:metatheory-P}.)

\subsubsection{Reasoning Principles}
We can also make explicit the reasoning principles available to programmers.

peTSM application is guaranteed to produce a segmentation of the literal body, maintain context independence (by allowing access only to the parameters) and respect the type annotation on the peTSM definition.

\begingroup
\def\thetheorem{\ref{thm:petsm-reasoning-principles}}
\begin{theorem}[peTSM Segmentation, Context Independence and Typing]
If $\expandsP{\uOmega}{\uPsi}{\uPhi}{\utsmap{\uepsilon}{b}}{e}{\tau}$ then:
\begin{enumerate}
  \item $\uOmega = \uOmegaEx{\uD}{\uG}{\uMctx}{\Omega_\text{app}}$
  \item $\uPsi=\uAS{\uA}{\Psi}$
  \item $\tsmexpExpandsExp{\uOmega}{\uPsi}{\uepsilon}{\epsilon}{\aetype{\tau_\text{final}}}$
  \item $\tsmexpEvalsExp{\Omega_\text{app}}{\Psi}{\epsilon}{\epsilon_\text{normal}}$
  \item $\tsmdefof{\epsilon_\text{normal}}=a$
  \item $\Psi = \Psi', \petsmdefn{a}{\rho}{\eparse}$
  \item $\encodeBody{b}{\ebody}$
  \item $\evalU{\ap{\eparse}{\ebody}}{{\lbltxt{SuccessE}}\cdot{e_\text{pproto}}}$
  \item $\decodePCEExp{e_\text{pproto}}{\pce}$
  \item $\prepce{\Omega_\text{app}}{\Psi}{\pce}{\ce}{\epsilon_\text{normal}}{\aetype{\tau_\text{proto}}}{\omega}{\Omega_\text{params}}$
  \item (\textbf{Segmentation}) $\segOK{\segof{\ce}}{b}$
  \item $\cvalidEP{\Omega_\text{params}}{\esceneP{\uOmega}{\uPsi}{\uPhi}{b}}{\ce}{e'}{\tau_\text{proto}}$
  \item $e = [\omega]e'$
  \item $\tau = [\omega]\tau_\text{proto}$
  \item (\textbf{Context Independence}) $\domof{\Omega_\text{app}} \cap \domof{\Omega_\text{params}} = \emptyset$
  \item (\textbf{Typing}) $\tau_\text{final} = [\omega]\tau_\text{proto}$
\end{enumerate}
\end{theorem}
\endgroup

Similarly, ppTSM application is guaranteed to produce a segmentation of the literal body and respect the type annotation on the ppTSM definition.
\begingroup
\def\thetheorem{\ref{thm:pptsm-reasoning-principles}}
\begin{theorem}[ppTSM Segmentation and Typing]
If $\patExpandsP{\uOmega'}{\uPhi}{\utsmap{\uepsilon}{b}}{p}{\tau}$ then:
\begin{enumerate}
  \item $\uOmega=\uOmegaEx{\uD}{\uG}{\uMctx}{\Omega_\text{app}}$
  \item $\uPhi=\uAS{\uA}{\Phi}$
  \item $\tsmexpExpandsPat{\uOmega}{\uPhi}{\uepsilon}{\epsilon}{\aetype{\tau_\text{final}}}$
  \item $\tsmexpEvalsPat{\Omega_\text{app}}{\Phi}{\epsilon}{\epsilon_\text{normal}}$
  \item $\tsmdefof{\epsilon_\text{normal}}=a$
  \item $\Phi = \Phi', \pptsmdefn{a}{\rho}{\eparse}$
  \item $\encodeBody{b}{\ebody}$
  \item $\evalU{\ap{\eparse}{\ebody}}{{\lbltxt{SuccessP}}\cdot{\ecand}}$
  \item $\decodeCEPat{\ecand}{\cpv}$
  \item $\prepcp{\Omega_\text{app}}{\Phi}{\epsilon_\text{normal}}{\aetype{\tau_\text{proto}}}{\omega}{\Omega_\text{params}}$
  \item (\textbf{Segmentation}) $\segOK{\segof{\cpv}}{b}$
  \item $\cvalidPP{\uOmega'}{\psceneP{\uOmega}{\uPhi}{b}}{\cpv}{p}{\tau_\text{proto}}$
  \item $\tau = [\omega]\tau_\text{proto}$
  \item (\textbf{Typing}) $\tau_\text{final} = [\omega]\tau_\text{proto}$
\end{enumerate}
\end{theorem}
\endgroup

Spliced terms have access only to the bindings at the application site.
\begingroup
\def\thetheorem{\ref{thm:petsm-shadowing-prohibition}}
\begin{theorem}[peTSM Shadowing Prohibition]
~
\begin{enumerate}
\item If $\cvalidK{\Omega}{\tsceneP{\uOmegaEx{\uD}{\uG}{\uMctx}{\Omega_\text{app}}}{b}}{\acesplicedk{m}{n}}{\kappa}$ then:
  \begin{enumerate}
    \item $\parseUKind{\bsubseq{b}{m}{n}}{\ukappa}$
    \item $\kExpands{\uOmega}{\ukappa}{\kappa}$
    \item $\domof{\Omega} \cap \domof{\Omega_\text{app}} = \emptyset$
  \end{enumerate}
\item If $\cvalidC{\Omega}{\tsceneP{\uOmegaEx{\uD}{\uG}{\uMctx}{\Omega_\text{app}}}{b}}{\acesplicedc{m}{n}}{c}{\kappa}$ then:
  \begin{enumerate}
    \item $\parseUCon{\bsubseq{b}{m}{n}}{\uc}$
    \item $\cExpands{\uOmega}{\uc}{c}{\kappa}$
    \item $\domof{\Omega} \cap \domof{\Omega_\text{app}} = \emptyset$
  \end{enumerate}
\item If $\cvalidEP{\Omega}{\esceneP{\uOmegaEx{\uD}{\uG}{\uMctx}{\Omega_\text{app}}}{\uPsi}{\uPhi}{b}}{\acesplicede{m}{n}}{e}{\tau}$ then:
  \begin{enumerate}
    \item $\parseUExp{\bsubseq{b}{m}{n}}{\ue}$
    \item $\expandsP{\uOmega}{\uPsi}{\uPhi}{\ue}{e}{\tau}$
    \item $\domof{\Omega} \cap \domof{\Omega_\text{app}} = \emptyset$
  \end{enumerate}
\end{enumerate}
\end{theorem}
\endgroup

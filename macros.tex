% !TEX root = omar-thesis.tex
\newcommand{\dolla}{\texttt{\$}}  % used so I don't screw up syntax highlighting when using $ in an identifier inline

% \newcommand{\gheading}[1]{\multicolumn{3}{l}{\textbf{#1}}}

\newcommand{\elided}{{\color{gray}\cdots}}

% Calculi Names
\newcommand{\miniVerseU}{\mathsf{miniVerse}_\textbf{U}}
\newcommand{\miniVersePat}{\mathsf{miniVerse}_\textbf{UP}}
\newcommand{\miniVerseParam}{\mathsf{miniVerse}_\mathbf{\forall}}
\newcommand{\miniVerseTSL}{\mathsf{miniVerse}_\textbf{TSL}}

% General abstract syntax
\newcommand{\aboppz}[2]{\texttt{#1}\texttt{[}#2\texttt{]}}
\newcommand{\abop}[2]{\texttt{#1}\texttt{(}#2\texttt{)}}
\newcommand{\abopi}[3]{\texttt{#1}[#2]\texttt{(}#3\texttt{)}}
\newcommand{\abopic}[4]{\texttt{#1}[#2]\texttt{\{}#3\texttt{\}(}#4\texttt{)}}
\newcommand{\abopp}[3]{\texttt{#1}\texttt{[}#2\texttt{](}#3\texttt{)}}
\newcommand{\abopc}[3]{\texttt{#1}\texttt{\{}#2\texttt{\}(}#3\texttt{)}}
\newcommand{\abopbc}[4]{\texttt{#1}\texttt{[}#2\texttt{]\{}#3\texttt{\}(}#4\texttt{)}}
\newcommand{\abopcc}[4]{\texttt{#1}\texttt{\{}#2\texttt{\}\{}#3\texttt{\}(}#4\texttt{)}}

% Types / candidate expansion types
\newcommand{\parr}[2]{#1 \rightharpoonup #2}
\newcommand{\aparr}[2]{\abop{parr}{#1; #2}}
\newcommand{\aceparr}[2]{\abop{ceparr}{#1; #2}}

\newcommand{\forallt}[2]{\forall #1.#2}
\newcommand{\aall}[2]{\abop{all}{#1.#2}}
\newcommand{\aceall}[2]{\abop{ceall}{#1.#2}}

\newcommand{\rect}[2]{\mu #1.#2}
\newcommand{\arec}[2]{\abop{rec}{#1.#2}}
\newcommand{\acerec}[2]{\abop{cerec}{#1.#2}}

\newcommand{\prodt}[1]{\langle #1 \rangle}
\newcommand{\aprod}[1]{\abop{prod}{#1}}
\newcommand{\aceprod}[1]{\abop{ceprod}{#1}}

\newcommand{\sumt}[1]{[#1]}
\newcommand{\asum}[1]{\abop{sum}{#1}}
\newcommand{\acesum}[1]{\abop{cesum}{#1}}

% Labels and maps
\newcommand{\labelset}{L}
\newcommand{\mapschema}[3]{\{#2 \hookrightarrow #1_{#2}\}_{#2 \in #3}}
\newcommand{\mapschemab}[4]{\{#3 \hookrightarrow #1_{#3}.#2_{#3}\}_{#3 \in #4}}
\newcommand{\finmap}[1]{#1}
\newcommand{\mapitem}[2]{#1 \hookrightarrow #2}
\newcommand{\lbltxt}[1]{\mathtt{#1}}

% sequences
\newcommand{\seqschema}[4]{\{#1_{#2}\}_{#3 \leq #2 < #4}}
\newcommand{\seqschemaX}[1]{\seqschema{#1}{i}{1}{n}}

% Expanded/Unexpanded/Candidate expressions
\newcommand{\lam}[3]{\lambda #1{:}#2.#3}
\newcommand{\aelam}[3]{\abopc{elam}{#1}{#2.#3}}
\newcommand{\aulam}[3]{\abopc{ulam}{#1}{#2.#3}}
\newcommand{\acelam}[3]{\abopc{celam}{#1}{#2.#3}}

\newcommand{\ap}[2]{#1(#2)}
\newcommand{\aeap}[2]{\abop{eap}{#1; #2}}
\newcommand{\auap}[2]{\abop{uap}{#1; #2}}
\newcommand{\aceap}[2]{\abop{ceap}{#1; #2}}

\newcommand{\Lam}[2]{\Lambda #1.#2}
\newcommand{\aetlam}[2]{\abop{etlam}{#1.#2}}
\newcommand{\autlam}[2]{\abop{utlam}{#1.#2}}
\newcommand{\acetlam}[2]{\abop{cetlam}{#1.#2}}

\newcommand{\App}[2]{#1\texttt{[}#2\texttt{]}}
\newcommand{\aetap}[2]{\abopc{etap}{#2}{#1}}
\newcommand{\autap}[2]{\abopc{utap}{#2}{#1}}
\newcommand{\acetap}[2]{\abopc{cetap}{#2}{#1}}

\newcommand{\fold}[1]{\texttt{fold}(#1)}
\newcommand{\aefold}[3]{\abopc{efold}{#1.#2}{#3}}
\newcommand{\aufold}[3]{\abopc{ufold}{#1.#2}{#3}}
\newcommand{\acefold}[3]{\abopc{cefold}{#1.#2}{#3}}

\newcommand{\unfold}[1]{\texttt{unfold}(#1)}
\newcommand{\aeunfold}[1]{\abop{eunfold}{#1}}
\newcommand{\auunfold}[1]{\abop{uunfold}{#1}}
\newcommand{\aceunfold}[1]{\abop{ceunfold}{#1}}

\newcommand{\tpl}[1]{\langle #1\rangle}
\newcommand{\aetpl}[1]{\abop{etpl}{#1}}
\newcommand{\autpl}[1]{\abop{utpl}{#1}}
\newcommand{\acetpl}[1]{\abop{cetpl}{#1}}

\newcommand{\prj}[2]{#1 \cdot #2}
\newcommand{\aepr}[2]{\abopp{epr}{#1}{#2}}
\newcommand{\aupr}[2]{\abopp{upr}{#1}{#2}}
\newcommand{\acepr}[2]{\abopp{cepr}{#1}{#2}}

\newcommand{\inj}[2]{#1 \cdot #2}
\newcommand{\aein}[3]{\abopbc{ein}{#1}{#2}{#3}}
\newcommand{\auin}[3]{\abopbc{uin}{#1}{#2}{#3}}
\newcommand{\acein}[3]{\abopbc{cein}{#1}{#2}{#3}}

\newcommand{\caseof}[2]{\texttt{case}~#1~#2}
\newcommand{\aecase}[2]{\abop{ecase}{#1; #2}}
\newcommand{\aucase}[2]{\abop{ucase}{#1; #2}}
\newcommand{\acecase}[2]{\abop{cecase}{#1; #2}}

% Expanded expressions
\newcommand{\etxt}[1]{e_\text{#1}}

% Statics of miniVerseU expanded expressions
\newcommand{\istypeU}[2]{#1 \vdash #2~\mathsf{type}}
\newcommand{\isctxU}[2]{#1 \vdash #2~\mathsf{ctx}}
\newcommand{\hastypeU}[4]{#1~#2 \vdash #3 : #4}
\newcommand{\hastypeUC}[2]{\vdash #1 : #2}
\newcommand{\hastypeUCO}[3]{#1 \vdash #2 : #3}

\newcommand{\Dhyp}[1]{#1~\mathsf{type}}
\newcommand{\Dcons}[2]{#1 \cup #2}
\newcommand{\Ghyp}[2]{#1 : #2}
\newcommand{\Gcons}[2]{#1 \cup #2}

% Dynamics of miniVerseU
\newcommand{\isvalU}[1]{#1~\mathsf{val}}
\newcommand{\stepsU}[2]{#1 \mapsto #2}
\newcommand{\multistepU}[2]{#1 \mapsto^{*} #2}
\newcommand{\evalU}[2]{#1 \Downarrow #2}

% Unexpanded expressions
\newcommand{\ue}{\hat{e}}
\newcommand{\uletsyntax}[4]{\texttt{syntax}~#1~\texttt{at}~#2~\{#3\}~\texttt{in}~#4}
\newcommand{\auletsyntax}[4]{\abopcc{usyntax}{#2}{#1}{#3.#4}}
\newcommand{\utsmap}[2]{#1~\texttt{/}#2\texttt{/}}
\newcommand{\autsmap}[2]{\texttt{utsmap}[#1]\texttt{[}#2\texttt{]}}
\newcommand{\uet}[1]{\ue_\text{#1}}
\newcommand{\ueparse}{\uet{parse}}

% TSM expressions
\newcommand{\tsmv}{a}
\newcommand{\utsmdef}[2]{\texttt{syntax}~@~#1~\texttt{\{}#2\texttt{\}}}
\newcommand{\istsmU}[2]{#1 \vdash #2~\mathsf{tsm}}

\newcommand{\tBody}{\mathtt{Body}}
\newcommand{\tParseResultExp}{\mathtt{ParseResultExp}} \newcommand{\tCEExp}{\mathtt{CEExp}} % Typed expansion
\newcommand{\expandsU}[6]{#1~#2 \vdash_{#3} #4 \leadsto #5 : #6} % there's a multiline one in the document done manually
\newcommand{\expandsUX}[3]{\expandsU{\Delta}{\Gamma}{\Sigma}{#1}{#2}{#3}}
\newcommand{\expandsUC}[3]{\vdash #1 \leadsto #2 : #3}
\newcommand{\domof}[1]{\text{dom}(#1)}
\newcommand{\xtsmdef}[2]{\abop{syntax}{#1;\,#2}}
\newcommand{\tsmenvbnd}[3]{#1 \hookrightarrow \xtsmdef{#2}{#3}}
\newcommand{\macenvOK}[2]{#1 \vdash #2~\mathsf{menv}}
\newcommand{\encodeBody}[2]{#1 \downarrow #2}
\newcommand{\decodeBody}[2]{#1 \uparrow #2}
\newcommand{\ebody}{\etxt{body}}
\newcommand{\eparse}{\etxt{parse}}
\newcommand{\ecand}{\etxt{cand}}
\newcommand{\decodeCondE}[2]{#1 \uparrow_\mathsf{CEExp} #2}
\newcommand{\encodeCondE}[2]{#1 \downarrow_\mathsf{CEExp} #2}

% Candidate Expansions
\newcommand{\ce}{\grave{e}}
\newcommand{\ctau}{\grave{\tau}}

\newcommand{\splicedt}[2]{\texttt{spliced}\langle#1, #2\rangle}
\newcommand{\acesplicedt}[2]{\texttt{cesplicedt}[#1; #2]}
\newcommand{\splicede}[2]{\texttt{spliced}\langle#1, #2\rangle}
\newcommand{\acesplicede}[2]{\texttt{cesplicede}[#1; #2]}

\newcommand{\mtau}{\dot{\tau}}
\newcommand{\mtspliced}[1]{\texttt{spliced}(#1)}

% Candidate expansion validation
\newcommand{\cvalidT}[4]{#1\vdash^{#2} #3 \leadsto #4~\mathsf{type}}
\newcommand{\cvalidE}[6]{#1~#2\vdash^{#3} #4 \leadsto #5 : #6}
\newcommand{\cvalidEX}[3]{\cvalidE{\Delta}{\Gamma}{\spctxv}{#1}{#2}{#3}}
\newcommand{\spctxv}{\mathcal{S}}
\newcommand{\spctx}[4]{#1; #2; #3; #4}
\newcommand{\parseTyp}[2]{\mathsf{parseTyp}(#1)=#2}
\newcommand{\parseUExp}[2]{\mathsf{parseUExp}(#1)=#2}
\newcommand{\bsubseq}[3]{\mathsf{subseq}(#1; #2; #3)}

\newcommand{\sizeof}[1]{\Vert #1 \Vert}

% Pattern matching
\newcommand{\matchwith}[2]{\texttt{match}~#1~#2}
\newcommand{\aematchwith}[2]{\abop{ematch}{#1; #2}}
\newcommand{\aumatchwith}[2]{\abop{umatch}{#1; #2}}

\newcommand{\matchrule}[2]{#1 \Rightarrow #2}
\newcommand{\aematchrule}[4]{\abopic{rule}{#1}{#2}{#3.#4}}

\newcommand{\eR}{R}
\newcommand{\uR}{\hat{R}}
\newcommand{\cR}{\grave{R}}

\newcommand{\erv}{r}
\newcommand{\urv}{\hat{r}}
\newcommand{\crv}{\grave{r}}

\newcommand{\epv}{p}
\newcommand{\upv}{\hat{p}}
\newcommand{\cpv}{\grave{p}}

\newcommand{\wildp}{\_}
\newcommand{\aewildp}{\texttt{ewildp}}

\newcommand{\foldp}[1]{\abop{fold}{#1}}
\newcommand{\aefoldp}[1]{\abop{efoldp}{#1}}

\newcommand{\tplp}[1]{\langle #1 \rangle}
\newcommand{\aetplp}[1]{\abop{etplp}{#1}}

\newcommand{\injp}[2]{#1 \cdot #2}
\newcommand{\aeinjp}[2]{\abopp{einp}{#1}{#2}}

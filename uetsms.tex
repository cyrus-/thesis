% !TEX root = omar-thesis.tex
\chapter{Unparameterized Expression TSMs}\label{chap:uetsms}
We now introduce a new primitive -- the \textbf{typed syntax macro} (TSM). TSMs, like term-rewriting macros (Sec. \ref{sec:term-rewriting}), generate expansions. Unlike term-rewriting macros, TSMs are applied to unparsed \emph{generalized literal forms}, which gives them substantially more syntactic flexibility. This chapter considers perhaps the simplest manifestation of TSMs: \textbf{unparameterized expression TSMs} (ueTSMs), which generate expressions of a single specified type. We will consider unparameterized pattern TSMs (upTSMs) in Chapter \ref{chap:uptsms} and parameterized TSMs (pTSMs) in Chapter \ref{chap:ptsms}.

%Like the term-rewriting macros just described, TSMs can be parameterized by modules, so they can be used to define syntax valid at any abstract type defined by a module satisfying a specified signature. As we will discuss in the remainder of this section, this addresses all of the problems brought up above, at moderate syntactic cost.

\section{Expression TSMs By Example}\label{sec:tsms-by-example}
%A typed syntax macro is invoked by applying it to a \emph{delimited form}, which can contain  arbitrary syntax in its \emph{body}.  
We begin in this section with a ``tutorial-style'' introduction to ueTSMs in VerseML. In particular, we discuss a ueTSM for constructing values of the recursive labeled sum type \li{Rx} that was defined in Figure \ref{fig:datatype-rx}. We then formally specify ueTSMs with a reduced calculus, $\miniVerseUE$, in Sec. \ref{sec:tsms-minimal-formalism}. %We conclude in Sec. \ref{sec:uetsms-discussion} 

\subsection{Usage}\label{sec:uetsms-usage}
In the following concrete VerseML expression, we apply a TSM named \li{#\dolla#rx} to the \emph{generalized literal form} \li{/SURLA|T|G|CEURL/}:
\begin{lstlisting}[numbers=none,mathescape=|]
$rx /SURLA|T|G|CEURL/
\end{lstlisting}
Generalized literal forms are left unparsed when concrete expressions are first parsed. It is only during the subsequent \emph{typed expansion} process that the TSM parses the \emph{body} of the provided literal form, i.e. the characters between forward slashes in blue here, to generate a \emph{candidate expansion}. The language then \emph{validates} the candidate expansion according to criteria that we will establish in Sec. \ref{sec:uetsms-validation}. If candidate expansion validation succeeds, the language generates the \emph{final expansion} (or more concisely, simply the \emph{expansion}) of the expression. The program will behave as if the expression above has been replaced by its expansion. The expansion of the expression above, written concretely, is:
\begin{lstlisting}[numbers=none]
Or(Str "SSTRAESTR", Or(Str "SSTRTESTR", Or(Str "SSTRGESTR", Str "SSTRCESTR")))
\end{lstlisting}
%The constructors above are those of the type \li{Rx} that was defined in Figure \ref{fig:datatype-rx}.

A number of literal forms, shown in Figure \ref{fig:literal-forms},  are available in VerseML's concrete syntax. Any literal form can be used with any TSM, e.g. we could have equivalently written the example above as \li{#\dolla#rx `SURLA|T|G|CEURL`} (in fact, this would be convenient if we had wanted to express a regex containing forward slashes but not backticks). TSMs have access only to the literal bodies. Because TSMs do not extend the concrete syntax of the language directly, there cannot be syntactic conflicts between TSMs.

 %The form does not directly determine the expansion. 

\begin{figure}
\begin{lstlisting}
'SURLbody cannot contain an apostropheEURL'
`SURLbody cannot contain a backtickEURL`
[SURLbody cannot contain unmatched square bracketsEURL]
{SURLbody cannot contain an unmatched curly braceEURL}
/SURLbody cannot contain a forward slashEURL/
\SURLbody cannot contain a backslashEURL\
\end{lstlisting}
%SURL<tag>body includes enclosing tags</tag>EURL
\caption[Available Generalized Literal Forms]{Generalized literal forms available for use in VerseML's concrete syntax. The characters in blue indicate where the literal bodies are located within each form. In this figure, each line describes how the literal body is constrained by the form shown on that line. The Wyvern language specifies additional forms, including whitespace-delimited forms \cite{TSLs} and multipart forms \cite{sac15}, but for simplicity we leave these out of VerseML.}
\label{fig:literal-forms}
\end{figure}
\subsection{Definition}\label{sec:uetsms-definition}
%The original expression, above, is statically rewritten to this expression.
Let us now take the perspective of the library provider. The definition of the TSM \lstinline{#\dolla#rx} shown being applied above has the following form:
\begin{lstlisting}[numbers=none,mathescape=|]
syntax $rx at Rx {
  static fn(body : Body) : CEExp ParseResult => 
    (* regex literal parser here *)
}
\end{lstlisting}
This {TSM definition} first names the TSM. 
 TSM names must begin with the dollar symbol (\li{#\dolla#}) to clearly distinguish them from variables (and thereby clearly distinguish TSM application from function application). This is inspired by a similar convention enforced by the Rust macro system \cite{Rust/Macros}.

The TSM definition then specifies a \emph{type annotation}, \lstinline{at Rx}, and a \emph{parse function} within curly braces. 
The {parse function} is a \emph{static function} responsible for parsing the literal body when the TSM is applied to generate an encoding of the candidate expansion, or an indication of an error if one cannot be generated (e.g. when the body is ill-formed according to the syntactic specification that the TSM implements). Static functions are functions that are applied during the typed expansion process. For this reason, they do not have access to surrounding variable bindings (because those variables stand in for dynamic values). For now, let us simply assume that static functions are closed (we discuss introducing a distinct class of static bindings so that static values can be shared between TSM definitions in Sec. \ref{sec:uetsms-static-language}).

The parse function must have type \li{Body -> CEExp ParseResult}. These types are defined in the VerseML \emph{prelude}, which is a collection of definitions available ambiently. The input type, \lstinline{Body}, gives the parse function access to the {body} of the provided literal form. For our purposes, it suffices to define \li{Body} as an abbreviation for the \li{string} type:
\begin{lstlisting}[numbers=none]
type Body = string
\end{lstlisting} 

The output type, \li{CEExp ParseResult}, is a labeled sum type that distinguishes between successful parses and parse errors. The parameterized type \li{'a ParseResult} is defined in Figure \ref{fig:indexrange-and-parseresult}.

If parsing succeeds, the parse function returns a value of the form \li{Success(#$\ecand$#)}, where $\ecand$ is the \emph{encoding of the candidate expansion}. Encodings of candidate expansions are, for expression TSMs, values of the type \lstinline{CEExp} defined in Figure \ref{fig:candidate-exp-verseml} (in Chapter \ref{chap:ptsms}, we will introduce pattern TSMs, which generate patterns rather than expressions; this is why \li{ParseResult} is defined as a parameterized type). Expressions can mention types, so we also need to define a type \li{CETyp} in Figure \ref{fig:candidate-exp-verseml}. 
\begin{figure}
\begin{lstlisting}[numbers=none]
type IndexRange = {startIndex : nat, endIndex : nat}

type 'a ParseResult = Success of 'a 
                    | ParseError of {
                        msg : string, loc : IndexRange
                      }
\end{lstlisting}
\caption[Definitions of \li{IndexRange} and \li{ParseResult} in VerseML]{Definitions of \li{IndexRange} and \li{ParseResult} in the VerseML prelude.}
\label{fig:indexrange-and-parseresult}
\end{figure}
\begin{figure}
\begin{lstlisting}[numbers=none]
type CETyp = TyVar of var_t 
           | Arrow of CETyp * CETyp 
           | (* ... *) 
           | Spliced of IndexRange

type CEExp = Var of var_t 
           | Fn of var_t * CETyp * CEExp
           | App of CEExp * CEExp
           | (* ... *) 
           | Spliced of IndexRange
\end{lstlisting}
\caption[Abbreviated definitions of \li{CETyp} and \li{CEExp} in VerseML]{Abbreviated definitions \li{CETyp} and \li{CEExp} in the VerseML prelude. We assume some suitable type \li{var_t} exists, not shown.}
\label{fig:candidate-exp-verseml}
\end{figure}
% We will show a complete encoding when we describe our reduced formal system $\miniVerseUE$ in Sec. \ref{sec:tsms-minimal-formalism}. 
We discuss the constructors labeled \li{Spliced} in Sec. \ref{sec:splicing-and-hygiene}; the remaining constructors (some of which are elided for concision) encode the abstract syntax of VerseML expressions and types. To decrease the syntactic cost of working with the types defined in Figure \ref{fig:candidate-exp-verseml}, the prelude provides \emph{quasiquotation syntax} at these types, which is itself implemented using TSMs. We will discuss these TSMs in more detail in Sec. \ref{sec:tsms-for-tsms}. The definitions in Figure \ref{fig:candidate-exp-verseml} are recursive labeled sum types to simplify our exposition, but we could have chosen alternative encodings of terms, e.g. based on abstract binding trees \cite{pfpl}, with only minor modification to the semantics. % It is extended with one additional form used to handled spliced subexpressions, 

If the parse function determines that a candidate expansion cannot be generated, i.e. there is a parse error in the literal body, it returns a value labeled by \li{ParseError}. It must provide an error message and indicate the location of the error within the body of the literal form as a value of type \li{IndexRange}, also defined in Figure \ref{fig:indexrange-and-parseresult}. This information can be used by VerseML compilers when reporting errors to the programmer.

%Notice that the types just described are those that one would expect to find in a typical parser.

%One would find types analagous to those just described in any parser, so for concision, we elide the details of \li{#\dolla#rx}'s parse function.
%The parse function must treat the TSM parameters parametrically, i.e. it does not have access to any values in the supplied module parameter. Only the expansion the parse function generates can refer to module parameters. 
%For example, the following definition is ill-sorted:
%\begin{lstlisting}[numbers=none]
%syntax pattern_bad[Q : PATTERN] at Q.t {
%  static fn (body : Body) : Exp => 
%    if Q.flag then (* ... *) else (* ... *)
%}
%\end{lstlisting}%So the parse function parses the body of the delimited form to generate an encoding of the elaboration.

\subsection{Splicing}\label{sec:splicing-and-hygiene}
To support splicing syntax, like that described in Sec. \ref{sec:syntax-examples-regexps}, the parse function must be able to parse subexpressions out of the supplied literal body. For example, consider the code snippet in Figure \ref{fig:derived-spliced-subexpressions}, expressed instead using the \li{#\dolla#rx} TSM:
\begin{lstlisting}[numbers=none]
val ssn = $rx /SURL\d\d\d-\d\d-\d\d\d\dEURL/
fun example_rx_tsm(name: string) => $rx /SURL@EURLnameSURL: %EURLssn/
\end{lstlisting}
The subexpressions \lstinline{name} and \lstinline{ssn} on the second line appear directly in the body of the literal form, so we call them \emph{spliced subexpressions} (and color them black when typesetting them in this document). When the parse function determines that a subsequence of the literal body should be treated as a spliced subexpression (here, by recognizing the characters \lstinline{@} or \lstinline{%} followed by a variable or parenthesized expression), 
it can refer to it within the candidate expansion it generates using the \li{Spliced} constructor of the \li{CEExp} type shown in Figure \ref{fig:candidate-exp-verseml}. The \li{Spliced} constructor requires a value of type \li{IndexRange} because spliced subexpressions are referred to indirectly by their position within the literal body. This prevents TSMs from ``forging'' a spliced subexpression (i.e. claiming that an expression is a spliced subexpression, even though it does not appear in the body of the literal form). Expressions can also contain types, so one can also mark spliced types in an analagous manner using the \li{Spliced} constructor of the \li{CETyp} type. %In particular, the parse function must provide the index range of spliced subexpressions to the \li{Spliced} constructor of the type \li{MarkedExp}. %Only subexpressions that actually appear in the body of the literal form can be marked as spliced subexpressions.

The candidate expansion generated by \li{#\dolla#rx} for the body of \lstinline{example_rx_tsm}, if written in a hypothetical concrete syntax for candidate expansions where references to spliced subexpressions are written \li{spliced<startIdx, endIndex>}, is:
\begin{lstlisting}[numbers=none]
Seq(Str(spliced<1, 4>), Seq(Str "SSTR: ESTR", spliced<8, 10>))
\end{lstlisting}
Here, \li{spliced<1, 4>} refers to the subexpression \li{name} by position and \li{spliced<8, 10>} refers to the subexpression \li{ssn} by position. 

%For example, had the  would not be a valid expansion, because the  that are not inside spliced subexpressions:
%\begin{lstlisting}[numbers=none]
%Q.Seq(Q.Str(name), Q.Seq(Q.Str ": ", ssn))
%\end{lstlisting}

\subsection{Typing}\label{sec:uetsms-validation}
The language \emph{validates} candidate expansions before a final expansion is generated. One aspect of candidate expansion validation is checking  the candidate expansion against the type annotation specified by the TSM, e.g. the type \li{Rx} in the example above. This maintains a \emph{type discipline}: if a programmer sees a TSM being applied when examining a well-typed program, they need only look up the TSM's type annotation to determine the type of the generated expansion. Determining the type does not require examine the expansion directly.


\subsection{Hygiene}
The spliced subexpressions that the candidate expansion refers to (by their position within the literal body, cf. above) must be parsed, typed and expanded during the candidate expansion validation process (otherwise, the language would not be able to check the type of the candidate expansion). To maintain a useful \emph{binding discipline}, i.e. to allow programmers to reason also about variable binding without examining expansions directly, the validation process maintains two additional properties related to spliced subexpressions: \textbf{context independent expansion} and \textbf{expansion independent splicing}. These are collectively referred to as the \emph{hygiene properties} (because they are conceptually related to the concept of hygiene in term rewriting macro systems, cf. Sec. \ref{sec:term-rewriting}.) 

\paragraph{Context Independent Expansion} Programmers expect to be able to choose variable and symbol names freely, i.e. without needing to satisfy ``hidden assumptions'' made by the TSMs that are applied in scope of a binding. For this reason, context-dependent candidate expansions, i.e. those with free variables or symbols, are deemed invalid (even at application sites where those variables happen to be bound). An example of a TSM that generates context-dependent candidate expansions is shown below:
\begin{lstlisting}[numbers=none]
syntax $bad1 at Rx {
	static fn(body : Body) : ParseResultExp => Success (Var 'SSTRxESTR')
}
\end{lstlisting}
The candidate expansion this TSM generates would be well-typed only when there is an assumption \li{x : Rx} in the application site typing context. This ``hidden assumption'' makes reasoning about binding and renaming especially difficult, so this candidate expansion is deemed invalid (even when \li{#\dolla#bad1} is applied in a context where \li{x} happens to be bound).

Of course, this prohibition does not extend into the spliced subexpressions referred to in a candidate expansion because spliced subexpressions are authored by the TSM client and appear at the application site, and so can justifiably refer to application site bindings. We saw examples of spliced subexpressions that referred to variables bound at the application site in Sec. \ref{sec:splicing-and-hygiene}. Because candidate expansions refer to spliced subexpressions indirectly, checking this property is straightforward -- we only allow access to the application site typing context when typing spliced subexpressions. In the next section, we will formalize this intuition. % The TSM provider can only refer to them opaquely.

In the examples in Sec. \ref{sec:uetsms-usage} and Sec. \ref{sec:splicing-and-hygiene}, the expansion used constructors associated with the \li{Rx} type, e.g. \li{Seq} and \li{Str}. This might appear to violate our prohibition on context-dependent expansions. This is not the case only because in VerseML, constructor labels are not variables or scoped symbols. Syntactically, they must begin with a capital letter (like Haskell's datatype constructors). Different labeled sum types can use common constructor labels without conflict because the type the term is being checked against -- e.g. \li{Rx}, due to the type ascription on \li{#\dolla#rx} -- determines which type of value will be constructed. For dialects of ML where datatype definitions do introduce new variables or scoped symbols, we need parameterized TSMs. We will return to this topic in Chapter \ref{chap:ptsms}. % Indeed, we used the label \li{Spliced} for two different recursive labeled sum types in Figure \ref{fig:candidate-exp-verseml}.

\paragraph{Expansion Independent Splicing} Spliced subexpressions, as just described, must be given access to application site bindings. The \emph{expansion independent splicing} property ensures that spliced subexpressions have access to \emph{only} those bindings, i.e. a TSM cannot introduce new bindings into spliced subexpressions. For example, consider the following hypothetical candidate expansion (written concretely as above):
\begin{lstlisting}[numbers=none]
fn(x : Rx) => spliced<0, 4>
\end{lstlisting}
The variable \li{x} is not available when typing the indicated spliced subexpression, nor can it shadow any bindings of \li{x} that might appear at the application site.

For TSM providers, the benefit of this property is that they can choose the names of variables used internally within expansions freely, without worrying about whether they might shadow those that a client might have defined at the application site.

TSM clients can, in turn, determine exactly which bindings are available in a spliced subexpression without examining the expansion it appears within. In other words, there can be no ``hidden variables''. 

The trade-off is that this prevents library providers from defining  alternative binding forms. For example, Haskell's derived form for monadic commands (i.e. \li{do}-notation) supports binding the result of executing a command to a variable that is then available in the subsequent commands in a command sequence. In VerseML, this cannot be expressed in the same way. We will show an alternative formulation of Haskell's syntax for monadic commands that uses VerseML's anonymous function syntax to bind variables in Sec. \ref{sec:application-monadic-commands}. We will discuss mechanisms that would allow us to relax this restriction while retaining client control over variable names as future work in Sec. \ref{sec:controlled-binding}.

%These properties suffice to ensure that programmers and tools can freely rename a variable without changing the meaning of the program. The only information that is necessary to perform such a \emph{rename refactoring} is the locations of spliced subexpressions within all the literal forms for which the variable being renamed is in scope; the expansions need not otherwise be examined. It would be straightforward to develop a tool and/or editor plugin to indicate the locations of spliced subexpressions to the user, like we do in this document (by coloring spliced subexpressions black). We discuss tool support as future work in Sec. \ref{sec:interaction-with-tools}.

\subsection{Final Expansion}
If validation succeeds, the language generates the \emph{final expansion} from the candidate expansion by replacing references to spliced subexpressions with their final expansions. The final expansion of the body of \li{example_rx_tsm} is:
\begin{lstlisting}[numbers=none]
Seq(Str(name), Seq(Str "SSTR: ESTR", ssn))
\end{lstlisting}

\subsection{Scoping}
A benefit of specifying TSMs as a language primitive, rather than relying on extralinguistic mechanisms to manipulate the concrete syntax of our language directly, is that TSMs follow standard scoping rules.

For example, we can define a TSM that is visible only to a single expression like this:
\begin{lstlisting}[numbers=none]
let x = (syntax $rx at Rx { (* ... *) } in 
           (* $rx is in scope here *) 
         end)
in (* $rx is no longer in scope *) end
\end{lstlisting}


We will consider the question of how TSM definitions can be exported from compilation units in Sec. \ref{sec:tsm-packaging}.

\subsection{Comparison to ML+Rx}
Let us compare the VerseML TSM \li{#\dolla#rx} to ML+Rx, the hypothetical syntactic dialect of ML with support for derived forms for regular expressions described in Sec. \ref{sec:syntax-examples-regexps}.

Both ML+Rx and \li{#\dolla#rx} give programmers the ability to use the same standard syntax for constructing regexes, including syntax for splicing in other strings and regexes. In VerseML, however, we incur the additional syntactic cost of explicitly applying the \li{#\dolla#rx} TSM each time we wish to use regex syntax. This cost does not grow with the size of the regex, so it would only be significant in programs that involve a large number of small regexes (which do, of course, exist). In Chapter \ref{chap:tsls} we will consider a design where even this syntactic cost can be eliminated in certain situations.

The benefit of this approach is that we can easily define other TSMs to use alongside the \li{#\dolla#rx} TSM without needing to consider the possibility of syntactic conflict. Furthermore, programmers can rely on the typing discipline and the hygienic binding discipline described above to reason about programs, including those that contain unfamiliar forms. Put pithily, VerseML helps programmers avoid ``conflict and confusion''. 


\section{\texorpdfstring{$\miniVerseUE$}{miniVerseUE}}\label{sec:tsms-minimal-formalism}\label{sec:miniVerseU}

% \begin{figure}[p!]
% $\begin{array}{lllllll}
% \textbf{variables} & \textbf{type variables} & \textbf{labels} & \textbf{label sets} & \textbf{TSM variables} & \textbf{literal bodies} & \textbf{nats}\\
% x & t & \ell & \labelset & \tsmv & b & n\\~\end{array}$\\
% $\begin{array}{ll}
% \textbf{type formation contexts} & \textbf{typing contexts}\\
% \Delta ::= \emptyset ~\vert~ \Delta, t & \Gamma ::= \emptyset ~\vert~ \Gamma, x : \tau\\
% ~
% \end{array}$\\
% ~\\
% $\begin{array}{lcl}
% \gheading{types}\\
% \tau & ::= & t ~\vert~ \parr{\tau}{\tau} ~\vert~ \forallt{t}{\tau} ~\vert~ \rect{t}{\tau} ~\vert~  \prodt{\mapschema{\tau}{i}{\labelset}} ~\vert~ \sumt{\mapschema{\tau}{i}{\labelset}}\\
% ~\\
% \gheading{expanded expressions}\\
% e & ::= & x ~\vert~ \lam{x}{\tau}{e} ~\vert~ \app{e}{e} ~\vert~ \Lam{t}{e} ~\vert~ \App{e}{\tau} ~\vert~ \fold{t}{\tau}{e} ~\vert~ \unfold{e} ~\vert~ \tpl{\mapschema{e}{i}{\labelset}} ~\vert~ \prj{e}{\ell} \\
% & \vert & \inj{\ell}{e} ~\vert~ \caseof{e}{\mapschemab{x}{e}{i}{\labelset}}\\
% ~\\
% \gheading{TSM expressions}\\
% \tsme & ::= & \tsmv ~\vert~ \utsmdef{\tau}{\ue}\\
% ~\\
% \gheading{unexpanded expressions}\\
% \ue & ::= & {x} ~\vert~ \lam{x}{\tau}{\ue} ~\vert~ \ue(\ue) ~\vert~ \Lam{t}{\ue} ~\vert~ \App{\ue}{\tau} ~\vert~ \fold{t}{\tau}{\ue} ~\vert~ \unfold{\ue} ~\vert~ \tpl{\mapschema{\ue}{i}{\labelset}} ~\vert~ \prj{\ue}{\ell} \\
% & \vert & \inj{\ell}{\ue} ~\vert~ \caseof{\ue}{\mapschemab{x}{\ue}{i}{\labelset}}\\
% & \vert & \uesyntax{\tsmv}{\tsme}{\ue} ~\vert~ \utsmapp{\eta}{b}\\
% ~\\
% \gheading{candidate expansion types}\\
% \mtau & ::= & t ~\vert~ \parr{\mtau}{\mtau} ~\vert~ \forallt{t}{\mtau} ~\vert~ \rect{t}{\mtau} ~\vert~ \prodt{\mapschema{\tau}{i}{\labelset}} ~\vert~ \sumt{\mapschema{\mtau}{i}{\labelset}} \\
% & \vert & \mtspliced{\tau}\\
% ~\\
% \gheading{candidate expansion expressions}\\
% \me & ::= & x ~\vert~ \lam{x}{\mtau}{\me} ~\vert~ \app{\me}{\me} ~\vert~ \Lam{t}{\me} ~|~ \App{\me}{\mtau} ~\vert~ \fold{t}{\mtau}{\me} ~\vert~ \unfold{\me} ~\vert~ \tpl{\mapschema{\me}{i}{\labelset}} ~\vert~ \prj{\me}{\ell} \\
% & \vert & \inj{\ell}{\me} ~\vert~ \caseof{\me}{\mapschemab{x}{\me}{i}{\labelset}}\\
% & \vert & \mspliced{e}
% % \\~
% \end{array}$
% \todo{finish breaking this up into syntax tables}
% \caption[Syntax of $\miniVerseUE$]{Syntax of $\miniVerseUE$. The forms $\mapschema{V}{i}{\labelset}$ and $\mapschemab{x}{V}{i}{\labelset}$ where $V$ is a metavariable indicate finite functions from each label $i \in \labelset$ to a term, $V_i$, or binder, $x_i.V_i$, respectively.}
% \label{fig:lambda-tsm-syntax}
% \end{figure}


To make the intuitions developed in the previous section mathematically precise, we will now introduce a reduced language called $\miniVerseUE$ with support for ueTSMs. $\miniVerseUE$ consists of an \emph{inner core} and an \emph{outer surface}.
%For reference, the syntax of $\miniVerseUE$ is specified in Figure \ref{fig:lambda-tsm-syntax}. We will reproduce relevant portions of this specification inline (in tabular form) as we continue. 
%We specify all formal systems in this document within the metatheoretic framework detailed in \emph{PFPL} \cite{pfpl}, and assume familiarity of fundamental background concepts (e.g. abstract binding trees, substitution, implicit identification of terms up to $\alpha$-equivalence, structural induction and rule induction) covered therein. %Familiarity with other accounts of typed lambda calculi should also suffice to understand the formal systems in this document. 



\subsection{Syntax of the Inner Core}\label{sec:U-expanded-terms}

\begin{figure}
\hspace{-5px}$\begin{array}{lllllll}
\textbf{Sort} & & & \textbf{Operational Form} & \textbf{Stylized Form} & \textbf{Description}\\
\mathsf{Typ} & \tau & ::= & t & t & \text{variable}\\
&&& \aparr{\tau}{\tau} & \parr{\tau}{\tau} & \text{partial function}\\
&&& \aall{t}{\tau} & \forallt{t}{\tau} & \text{polymorphic}\\
&&& \arec{t}{\tau} & \rect{t}{\tau} & \text{recursive}\\
&&& \aprod{\labelset}{\mapschema{\tau}{i}{\labelset}} & \prodt{\mapschema{\tau}{i}{\labelset}} & \text{labeled product}\\
&&& \asum{\labelset}{\mapschema{\tau}{i}{\labelset}} & \sumt{\mapschema{\tau}{i}{\labelset}} & \text{labeled sum}\\
\mathsf{Exp} & e & ::= & x & x & \text{variable}\\
&&& \aelam{\tau}{x}{e} & \lam{x}{\tau}{e} & \text{abstraction}\\
&&& \aeap{e}{e} & \ap{e}{e} & \text{application}\\
&&& \aetlam{t}{e} & \Lam{t}{e} & \text{type abstraction}\\
&&& \aetap{e}{\tau} & \App{e}{\tau} & \text{type application}\\
&&& \aefold{t}{\tau}{e} & \fold{e} & \text{fold}\\
&&& \aeunfold{e} & \unfold{e} & \text{unfold}\\
&&& \aetpl{\labelset}{\mapschema{e}{i}{\labelset}} & \tpl{\mapschema{e}{i}{\labelset}} & \text{labeled tuple}\\
&&& \aepr{\ell}{e} & \prj{e}{\ell} & \text{projection}\\
&&& \aein{\labelset}{\ell}{\mapschema{\tau}{i}{\labelset}}{e} & \inj{\ell}{e} & \text{injection}\\
&&& \aecase{\labelset}{\tau}{e}{\mapschemab{x}{e}{i}{\labelset}} & \caseof{e}{\mapschemab{x}{e}{i}{\labelset}} & \text{case analysis}
\end{array}$
\caption[Syntax of types and expanded expressions in $\miniVerseUE$]{Abstract syntax of types and expanded expressions, which form the \emph{inner core of }$\miniVerseUE$. Metavariables $x$ range over variables, $t$ over type variables, $\ell$ over labels and $\labelset$ over finite sets of labels. We adopt \emph{PFPL}'s conventions for operational forms, i.e. the names of operators and indexed families of operators are written in $\texttt{typewriter font}$, indexed families of operators specify non-symbolic indices within $[\text{mathematical braces}]$ and symbolic indices within \texttt{[}textual braces\text{]}, and term arguments are grouped arbitrarily (roughly, by ``phase'') using \texttt{\{}textual curly braces\texttt{\}} and \texttt{(}textual rounded braces\texttt{)} \cite{pfpl}. We write $\mapschema{\tau}{i}{\labelset}$ for a sequence of arguments $\tau_i$, one for each $i\in \labelset$, and similarly for arguments of other valences. Operations  parameterized by label sets, e.g. $\aprod{\labelset}{\mapschema{\tau}{i}{\labelset}}$, are identified up to mutual reordering of the label set and the corresponding argument sequence. %When using stylized forms, the label set is omitted when it can be inferred, e.g. the labeled product type $\prodt{\finmap{\mapitem{\ell_1}{e_1}, \mapitem{\ell_2}{e_2}}}$ leaves the label set $\{\ell_1, \ell_2\}$ implicit. 
When we use the stylized forms, we assume that the reader can infer suppressed indices and arguments from the surrounding context. Types and expanded expressions are identified up to $\alpha$-equivalence.}
\label{fig:U-expanded-terms}
\end{figure}

The \emph{inner core of} $\miniVerseUE$ consists of \emph{types}, $\tau$, and \emph{expanded expressions}, $e$. The syntax of the inner core is specified by the syntax chart in Figure \ref{fig:U-expanded-terms}. 
The {inner core} forms a pure language with support for partial functions, quantification over types, recursive types, labeled product types and labeled sum types.  The reader is directed to \emph{PFPL} \cite{pfpl} (or another text on type systems, e.g. \emph{TAPL} \cite{tapl}) for a detailed introductory account of these (or very similar) constructs. We will tersely define the statics of the inner core, and outline the structural dynamics, in the next two subsections, respectively.

\subsection{Statics of the Inner Core}
The \emph{statics of the inner core} is defined by hypothetical judgements of the following form:

\[\begin{array}{ll}
\textbf{Judgement Form} & \textbf{Description}\\
\istypeU{\Delta}{\tau} & \text{$\tau$ is a well-formed type assuming $\Delta$}\\
%\isctxU{\Delta}{\Gamma} & \text{$\Gamma$ is a well-formed typing context assuming $\Delta$}\\
\hastypeU{\Delta}{\Gamma}{e}{\tau} & \text{$e$ is assigned type $\tau$ assuming $\Delta$ and $\Gamma$}
\end{array}\]
\noindent
\emph{Type formation contexts}, $\Delta$, are finite sets of hypotheses of the form $\Dhyp{t}$. Empty finite sets are written $\emptyset$, or omitted entirely within judgements, and non-empty finite sets are written as comma-separated finite sequences identified up to exchange and contraction. We write $\Delta, \Dhyp{t}$, when $\Dhyp{t} \notin \Delta$, for $\Delta$ extended with the hypothesis $\Dhyp{t}$. %Finite sets are written as finite sequences identified up to exchange.% We write $\Dcons{\Delta}{\Delta'}$ for the union of $\Delta$ and $\Delta'$.

The \emph{type formation judgement}, $\istypeU{\Delta}{\tau}$, is inductively defined by the following rules:
\begin{subequations}\label{rules:istypeU}
\begin{equation}\label{rule:istypeU-var}
\inferrule{ }{\istypeU{\Delta, \Dhyp{t}}{t}}
\end{equation}
\begin{equation}\label{rule:istypeU-parr}
\inferrule{
  \istypeU{\Delta}{\tau_1}\\
  \istypeU{\Delta}{\tau_2}
}{\istypeU{\Delta}{\aparr{\tau_1}{\tau_2}}}
\end{equation}
\begin{equation}\label{rule:istypeU-all}
  \inferrule{
    \istypeU{\Delta, \Dhyp{t}}{\tau}
  }{
    \istypeU{\Delta}{\aall{t}{\tau}}
  }
\end{equation}
\begin{equation}\label{rule:istypeU-rec}
  \inferrule{
    \istypeU{\Delta, \Dhyp{t}}{\tau}
  }{
    \istypeU{\Delta}{\arec{t}{\tau}}
  }
\end{equation}
\begin{equation}\label{rule:istypeU-prod}
  \inferrule{
    \{\istypeU{\Delta}{\tau_i}\}_{i \in \labelset}
  }{
    \istypeU{\Delta}{\aprod{\labelset}{\mapschema{\tau}{i}{\labelset}}}
  }
\end{equation}
\begin{equation}\label{rule:istypeU-sum}
  \inferrule{
    \{\istypeU{\Delta}{\tau_i}\}_{i \in \labelset}
  }{
    \istypeU{\Delta}{\asum{\labelset}{\mapschema{\tau}{i}{\labelset}}}
  }
\end{equation}
\end{subequations}
Premises of the form $\{{J}_i\}_{i \in \labelset}$ mean that for each $i \in \labelset$, the judgement ${J}_i$ must hold. 

\emph{Typing contexts}, $\Gamma$, are finite functions that map each variable $x \in \domof{\Gamma}$, to the hypothesis $\Ghyp{x}{\tau}$, for some $\tau$. Empty typing contexts are written $\emptyset$, or omitted entirely within judgements, and non-empty typing contexts are written as finite sequences of hypotheses identified up to exchange (we do not separately write down the finite set $\domof{\Gamma}$ because it can be determined from the listed hypotheses). We write $\Gamma, \Ghyp{x}{\tau}$, when $x \notin \domof{\Gamma}$, for the extension of $\Gamma$ with a mapping from $x$ to $\Ghyp{x}{\tau}$, and $\Gcons{\Gamma}{\Gamma'}$ when $\domof{\Gamma} \cap \domof{\Gamma'} = \emptyset$ for the typing context mapping each $x \in \domof{\Gamma} \cup \domof{\Gamma'}$ to $x : \tau$ if $x : \tau \in \Gamma$ or $x : \tau \in \Gamma'$. We write $\isctxU{\Delta}{\Gamma}$ if every type in $\Gamma$ is well-formed relative to $\Delta$.
\begin{definition}[Typing Context Formation] \label{def:isctxU}
$\isctxU{\Delta}{\Gamma}$ iff for each hypothesis $x : \tau \in \Gamma$, we have $\istypeU{\Delta}{\tau}$.
\end{definition}

The typing judgement, $\hastypeU{\Delta}{\Gamma}{e}{\tau}$, assigns types to expressions. It is inductively defined by the following rules:
\begin{subequations}\label{rules:hastypeU}
\begin{equation}\label{rule:hastypeU-var}
  \inferrule{ }{
    \hastypeU{\Delta}{\Gamma, \Ghyp{x}{\tau}}{x}{\tau}
  }
\end{equation}
\begin{equation}\label{rule:hastypeU-lam}
  \inferrule{
    \istypeU{\Delta}{\tau}\\
    \hastypeU{\Delta}{\Gamma, \Ghyp{x}{\tau}}{e}{\tau'}
  }{
    \hastypeU{\Delta}{\Gamma}{\aelam{\tau}{x}{e}}{\aparr{\tau}{\tau'}}
  }
\end{equation}
\begin{equation}\label{rule:hastypeU-ap}
  \inferrule{
    \hastypeU{\Delta}{\Gamma}{e_1}{\aparr{\tau}{\tau'}}\\
    \hastypeU{\Delta}{\Gamma}{e_2}{\tau}
  }{
    \hastypeU{\Delta}{\Gamma}{\aeap{e_1}{e_2}}{\tau'}
  }
\end{equation}
\begin{equation}\label{rule:hastypeU-tlam}
  \inferrule{
    \hastypeU{\Delta, \Dhyp{t}}{\Gamma}{e}{\tau}
  }{
    \hastypeU{\Delta}{\Gamma}{\aetlam{t}{e}}{\aall{t}{\tau}}
  }
\end{equation}
\begin{equation}\label{rule:hastypeU-tap}
  \inferrule{
    \hastypeU{\Delta}{\Gamma}{e}{\aall{t}{\tau}}\\
    \istypeU{\Delta}{\tau'}
  }{
    \hastypeU{\Delta}{\Gamma}{\aetap{e}{\tau'}}{[\tau'/t]\tau}
  }
\end{equation}
\begin{equation}\label{rule:hastypeU-fold}
  \inferrule{\
    \istypeU{\Delta, \Dhyp{t}}{\tau}\\
    \hastypeU{\Delta}{\Gamma}{e}{[\arec{t}{\tau}/t]\tau}
  }{
    \hastypeU{\Delta}{\Gamma}{\aefold{t}{\tau}{e}}{\arec{t}{\tau}}
  }
\end{equation}
\begin{equation}\label{rule:hastypeU-unfold}
  \inferrule{
    \hastypeU{\Delta}{\Gamma}{e}{\arec{t}{\tau}}
  }{
    \hastypeU{\Delta}{\Gamma}{\aeunfold{e}}{[\arec{t}{\tau}/t]\tau}
  }
\end{equation}
\begin{equation}\label{rule:hastypeU-tpl}
  \inferrule{
    \{\hastypeU{\Delta}{\Gamma}{e_i}{\tau_i}\}_{i \in \labelset}
  }{
    \hastypeU{\Delta}{\Gamma}{\aetpl{\labelset}{\mapschema{e}{i}{\labelset}}}{\aprod{\labelset}{\mapschema{\tau}{i}{\labelset}}}
  }
\end{equation}
\begin{equation}\label{rule:hastypeU-pr}
  \inferrule{
    \hastypeU{\Delta}{\Gamma}{e}{\aprod{\labelset, \ell}{\mapschema{\tau}{i}{\labelset}; \ell \hookrightarrow \tau}}
  }{
    \hastypeU{\Delta}{\Gamma}{\aepr{\ell}{e}}{\tau}
  }
\end{equation}
\begin{equation}\label{rule:hastypeU-in}
  \inferrule{
    \{\istypeU{\Delta}{\tau_i}\}_{i \in \labelset}\\
    \istypeU{\Delta}{\tau}\\
    \hastypeU{\Delta}{\Gamma}{e}{\tau}
  }{
    \hastypeU{\Delta}{\Gamma}{\aein{\labelset, \ell}{\ell}{\mapschema{\tau}{i}{\labelset}; \ell \hookrightarrow \tau}{e}}{\asum{\labelset, \ell}{\mapschema{\tau}{i}{\labelset}; \ell \hookrightarrow \tau}}
  }
\end{equation}
\begin{equation}\label{rule:hastypeU-case}
  \inferrule{
    \hastypeU{\Delta}{\Gamma}{e}{\asum{\labelset}{\mapschema{\tau}{i}{\labelset}}}\\
    \istypeU{\Delta}{\tau}\\
    \{\hastypeU{\Delta}{\Gamma, x_i : \tau_i}{e_i}{\tau}\}_{i \in \labelset}
  }{
    \hastypeU{\Delta}{\Gamma}{\aecase{\labelset}{\tau}{e}{\mapschemab{x}{e}{i}{\labelset}}}{\tau}
  }
\end{equation}
\end{subequations}
Rules (\ref{rules:istypeU}) and (\ref{rules:hastypeU}) are syntax-directed, so we assume an inversion lemma for each rule as needed without stating it separately. The following standard lemmas also hold. 

The Weakening Lemma establishes that extending a context with unnecessary hypotheses preserves well-formedness and typing.
\begin{lemma}[Weakening]\label{lemma:weakening-U} All of the following hold: 
\begin{enumerate} 
\item If $\istypeU{\Delta}{\tau}$ then $\istypeU{\Delta, \Dhyp{t}}{\tau}$.
%\item If $\isctxU{\Delta}{\Gamma}$ then $\isctxU{\Delta, \Dhyp{t}}{\Gamma}$.
\item If $\hastypeU{\Delta}{\Gamma}{e}{\tau}$ then $\hastypeU{\Delta, \Dhyp{t}}{\Gamma}{e}{\tau}$.
\item If $\hastypeU{\Delta}{\Gamma}{e}{\tau}$ and $\istypeU{\Delta}{\tau'}$ then $\hastypeU{\Delta}{\Gamma, \Ghyp{x}{\tau'}}{e}{\tau}$.
\end{enumerate}
\end{lemma}
\begin{proof-sketch} For each part, by rule induction on the assumption. 
%\begin{enumerate} 
%\item By rule induction over Rules (\ref{rules:istypeU}).
%\item By rule induction over Rules (\ref{rules:isctxU}).
%\item By rule induction over Rules (\ref{rules:hastypeU}).
%\item By rule induction over Rules (\ref{rules:hastypeU}).
%\end{enumerate}
\end{proof-sketch}

We assume that renaming of bound variables, $\alpha$-equivalence and substitution are defined as in \emph{PFPL} \cite{pfpl}. The Substitution Lemma establishes that substitution of a well-formed type for a type variable, or an expanded expression of the appropriate type for an expanded expression variable, preserves well-formedness and typing. 
\begin{lemma}[Substitution]\label{lemma:substitution-U} All of the following hold:
\begin{enumerate}
\item If $\istypeU{\Delta, \Dhyp{t}}{\tau}$ and $\istypeU{\Delta}{\tau'}$ then $\istypeU{\Delta}{[\tau'/t]\tau}$.
%\item If $\isctxU{\Delta, \Dhyp{t}}{\Gamma}$ and $\istypeU{\Delta}{\tau'}$ then $\isctxU{\Delta}{[\tau'/t]\Gamma}$.
\item If $\hastypeU{\Delta, \Dhyp{t}}{\Gamma}{e}{\tau}$ and $\istypeU{\Delta}{\tau'}$ then $\hastypeU{\Delta}{[\tau'/t]\Gamma}{[\tau'/t]e}{[\tau'/t]\tau}$.
\item If $\hastypeU{\Delta}{\Gamma, \Ghyp{x}{\tau'}}{e}{\tau}$ and $\hastypeU{\Delta}{\Gamma}{e'}{\tau'}$ then $\hastypeU{\Delta}{\Gamma}{[e'/x]e}{\tau}$.
\end{enumerate}\end{lemma}
\begin{proof-sketch}
For each part, by rule induction on the first assumption.
\end{proof-sketch}

The Decomposition Lemma is the converse of the Substitution Lemma.
\begin{lemma}[Decomposition]\label{lemma:decomposition-U} All of the following hold:
\begin{enumerate}
\item If $\istypeU{\Delta}{[\tau'/t]\tau}$ and $\istypeU{\Delta}{\tau'}$ then $\istypeU{\Delta, \Dhyp{t}}{\tau}$.
%\item If $\isctxU{\Delta}{[\tau'/t]\Gamma}$ and $\istypeU{\Delta}{\tau'}$ then $\isctxU{\Delta, \Dhyp{t}}{\Gamma}$.
\item If $\hastypeU{\Delta}{[\tau'/t]\Gamma}{[\tau'/t]e}{[\tau'/t]\tau}$ and $\istypeU{\Delta}{\tau'}$ then $\hastypeU{\Delta, \Dhyp{t}}{\Gamma}{e}{\tau}$.
\item If $\hastypeU{\Delta}{\Gamma}{[e'/x]e}{\tau}$ and $\hastypeU{\Delta}{\Gamma}{e'}{\tau'}$ then $\hastypeU{\Delta}{\Gamma, \Ghyp{x}{\tau'}}{e}{\tau}$.
\end{enumerate}\end{lemma}
\begin{proof-sketch}
\begin{enumerate}
\item By rule induction over Rules (\ref{rules:istypeU}) and case analysis on the definition of substitution. In all cases, the derivation of $\istypeU{\Delta}{[\tau'/t]\tau}$ does not depend on the form of $\tau'$.
%\item Context formation of $[\tau'/t]\Gamma$ does not depend on the structure of $\tau'$.
\item By rule induction over Rules (\ref{rules:hastypeU}) and case analysis on the definition of substitution. In all cases, the derivation of $\hastypeU{\Delta}{[\tau'/t]\Gamma}{[\tau'/t]e}{[\tau'/t]\tau}$ does not depend on the form of $\tau'$.
\item By rule induction over Rules (\ref{rules:hastypeU}) and case analysis on the definition of substitution. In all cases, the derivation of $\hastypeU{\Delta}{\Gamma}{[e'/x]e}{\tau}$ does not depend on the form of $e'$.
\end{enumerate}
\end{proof-sketch}

The Regularity Lemma establishes that the type assigned to an expanded expression under a well-formed typing context is always well-formed. 
\begin{lemma}[Regularity]\label{lemma:regularity-U} If $\hastypeU{\Delta}{\Gamma}{e}{\tau}$ and $\isctxU{\Delta}{\Gamma}$ then $\istypeU{\Delta}{\tau}$.\end{lemma}
\begin{proof-sketch}
By rule induction over Rules (\ref{rules:hastypeU}) and application of Definition \ref{def:isctxU} and Lemma \ref{lemma:substitution-U}.
\end{proof-sketch}
\subsection{Structural Dynamics}\label{sec:dynamics-U}
The \emph{structural dynamics of }$\miniVerseUE$ is specified as a transition system by judgements of the following form:
\[\begin{array}{ll}
\textbf{Judgement Form} & \textbf{Description}\\
\stepsU{e}{e'} & \text{$e$ transitions to $e'$}\\
\isvalU{e} & \text{$e$ is a value}
\end{array}\]
We also define auxiliary judgements for \emph{iterated transition}, $\multistepU{e}{e'}$, and \emph{evaluation}, $\evalU{e}{e'}$.

\begin{definition}[Iterated Transition]\label{defn:iterated-transition-U} $\multistepU{e}{e'}$ is the reflexive, transitive closure of $\stepsU{e}{e'}$.\end{definition}

\begin{definition}[Evaluation]\label{defn:evaluation-U}  $\evalU{e}{e'}$ iff $\multistepU{e}{e'}$ and $\isvalU{e'}$. \end{definition}

Our subsequent developments do not require making reference to particular rules in the structural dynamics (because TSMs operate statically), so we do not reproduce the rules here. Instead, it suffices to state the following conditions.

The Canonical Forms condition characterizes well-typed values. Satisfying this condition requires an \emph{eager} (i.e. \emph{by-value}) formulation of the dynamics. 
\begin{condition}[Canonical Forms]\label{condition:canonical-forms-U} If $\hastypeUC{e}{\tau}$ and $\isvalU{e}$ then:
\begin{enumerate}
\item If $\tau=\aparr{\tau_1}{\tau_2}$ then $e=\aelam{\tau_1}{x}{e'}$ and $\hastypeUCO{\Ghyp{x}{\tau_1}}{e'}{\tau_2}$.
\item If $\tau=\aall{t}{\tau'}$ then $e=\aetlam{t}{e'}$ and $\hastypeUCO{\Dhyp{t}}{e'}{\tau'}$.
\item If $\tau=\arec{t}{\tau'}$ then $e=\aefold{t}{\tau'}{e'}$ and $\hastypeUC{e'}{[\abop{rec}{t.\tau'}/t]\tau'}$ and $\isvalU{e'}$. 
\item If $\tau=\aprod{\labelset}{\mapschema{\tau}{i}{\labelset}}$ then $e=\aetpl{\labelset}{\mapschema{e}{i}{\labelset}}$ and $\hastypeUC{e_i}{\tau_i}$ and $\isvalU{e_i}$ for each $i \in \labelset$.
\item If $\tau=\asum{\labelset}{\mapschema{\tau}{i}{\labelset}}$ then for some label set $L'$ and label $\ell$ and type $\tau_\ell$, we have that $\labelset=\labelset', \ell$ and $\tau=\asum{\labelset', \ell}{\mapschema{\tau}{i}{\labelset'}; \mapitem{\ell}{\tau_\ell}}$ and $e=\aein{\labelset', \ell}{\ell}{\mapschema{\tau}{i}{\labelset'}; \ell \hookrightarrow \tau_\ell}{e'}$ and $\hastypeUC{e'}{\tau_\ell}$ and $\isvalU{e'}$.
\end{enumerate}\end{condition}

The Preservation condition ensures that evaluation preserves typing.  
\begin{condition}[Preservation]\label{condition:preservation-U} If $\hastypeUC{e}{\tau}$ and $\multistepU{e}{e'}$ then $\hastypeUC{e'}{\tau}$. \end{condition}
The Progress condition ensures that evaluation of a well-typed expanded expression cannot ``get stuck''.
\begin{condition}[Progress]\label{condition:progress-U} If $\hastypeUC{e}{\tau}$ then either $\isvalU{e}$ or there exists an $e'$ such that $\stepsU{e}{e'}$. \end{condition}
 Together, these two conditions constitute the Type Safety Condition.

\subsection{Syntax of the Outer Surface}\label{sec:syntax-U}
\begin{figure}
\hspace{-6px}$\arraycolsep=3.5pt\begin{array}{lllllll}
\textbf{Sort} & & & \textbf{Operational Form} & \textbf{Stylized Form} & \textbf{Description}\\
\mathsf{UTyp} & \utau & ::= & \ut & \ut & \text{sigil}\\
&&& \auparr{\utau}{\utau} & \parr{\utau}{\utau} & \text{partial function}\\
&&& \auall{\ut}{\utau} & \forallt{\ut}{\utau} & \text{polymorphic}\\
&&& \aurec{\ut}{\utau} & \rect{\ut}{\utau} & \text{recursive}\\
&&& \auprod{\labelset}{\mapschema{\utau}{i}{\labelset}} & \prodt{\mapschema{\utau}{i}{\labelset}} & \text{labeled product}\\
&&& \ausum{\labelset}{\mapschema{\utau}{i}{\labelset}} & \sumt{\mapschema{\utau}{i}{\labelset}} & \text{labeled sum}\\
\mathsf{UExp} & \ue & ::= & \ux & \ux & \text{sigil}\\
&&& \aulam{\utau}{\ux}{\ue} & \lam{\ux}{\utau}{\ue} & \text{abstraction}\\
&&& \auap{\ue}{\ue} & \ap{\ue}{\ue} & \text{application}\\
&&& \autlam{\ut}{\ue} & \Lam{\ut}{\ue} & \text{type abstraction}\\
&&& \autap{\ue}{\utau} & \App{\ue}{\utau} & \text{type application}\\
&&& \aufold{\ut}{\utau}{\ue} & \fold{\ue} & \text{fold}\\
&&& \auunfold{\ue} & \unfold{\ue} & \text{unfold}\\
&&& \autpl{\labelset}{\mapschema{\ue}{i}{\labelset}} & \tpl{\mapschema{\ue}{i}{\labelset}} & \text{labeled tuple}\\
&&& \aupr{\ell}{\ue} & \prj{\ue}{\ell} & \text{projection}\\
&&& \auin{\labelset}{\ell}{\mapschema{\utau}{i}{\labelset}}{\ue} & \inj{\ell}{\ue} & \text{injection}\\
&&& \aucase{\labelset}{\utau}{\ue}{\mapschemab{\ux}{\ue}{i}{\labelset}} & \caseof{\ue}{\mapschemab{\ux}{\ue}{i}{\labelset}} & \text{case analysis}\\
\LCC  &  & & \lightgray & \lightgray & \lightgray \\
&&& \audefuetsm{\utau}{e}{\tsmv}{\ue} & \uesyntax{\tsmv}{\utau}{e}{\ue} & \text{ueTSM definition}\\ 
&&& \autsmap{b}{\tsmv} & \utsmap{\tsmv}{b} & \text{ueTSM application}\ECC
\end{array}$
\caption[Syntax of unexpanded types and expressions in $\miniVerseUE$]{Abstract syntax of unexpanded types and expressions in $\miniVerseUE$. Metavariable $\ut$ ranges over type sigils, $\ux$ ranges over expression sigils, $\tsmv$  over TSM names and $b$ over textual sequences, which, when they appear in an unexpanded expression, are called literal bodies. Literal bodies might contain spliced subterms that are only ``surfaced'' during typed expansion, so renaming of bound identifiers and substitution are not defined over unexpanded types and expressions.}
\label{fig:U-unexpanded-terms}
\end{figure}
A $\miniVerseUE$ program ultimately evaluates as an expanded expression. However, the programmer does not write the expanded expression directly. Instead, the programmer writes a textual sequence, $b$, consisting of characters in some suitable alphabet (e.g. in practice, \texttt{ASCII} or \texttt{Unicode}), which is parsed by some partial metafunction $\mathsf{parseUExp}(b)$ to produce an \emph{unex\-panded expression}, $\ue$. Unexpanded expressions can contain \emph{unexpanded types}, $\utau$, so we also need a partial metafunction $\mathsf{parseUTyp}(b)$. The abstract syntax of unexpanded types and expressions, which form  the \emph{outer surface} of $\miniVerseUE$, is defined in Figure \ref{fig:U-unexpanded-terms}. The full definition of the textual syntax of $\miniVerseUE$, which $\mathsf{parseUExp}(b)$ and $\mathsf{parseUTyp}(b)$ implement, is not important for our purposes, so we simply give the following condition, which states that there is some way to textually represent every unexpanded type and expression. %We also assume a metafunction $\mathsf{parseUTyp}(b)$ for parsing unexpanded types, and impose an analagous condition.
\begin{condition}[Textual Representability] Both of the following must hold:
\begin{enumerate}
\item For each $\utau$, there exists $b$ such that $\parseUTyp{b}{\utau}$. 
\item For each $\ue$, there exists $b$ such that $\parseUExp{b}{\ue}$.
\end{enumerate}
\end{condition}


Unexpanded types and expressions are given meaning by expansion to types and expanded expressions, respectively, according to the \emph{typed expansion judgements}, which are defined in the next subsection.

Unexpanded types and expressions bind \emph{type sigils}, $\ut$, \emph{expression sigils}, $\ux$, and \emph{TSM names}, $\tsmv$. Sigils are given meaning by expansion to variables during typed expansion. We \textbf{cannot} adopt the usual definitions of $\alpha$-renaming of identifiers, because unexpanded types and expressions are still in a ``partially parsed'' state -- the literal bodies, $b$, within an unexpanded expression might contain spliced subterms that are ``surfaced'' by a TSM only during typed expansion, as we will detail below. %Sigils are given meaning by expansion to variables. %In other words, unexpanded expressions are not abstract binding trees, nor sequences of characters, but a ``transitional'' structure with some characteristics of each of these. 
%For this reason, we will need to handle generating fresh variables explicitly at binding sites in our semantics. %To do so, we distinguish \emph{type sigils}, $\ut$, and \emph{expression sigils}, $\ux$, from type variables, $t$, and expression variables, $x$. Sigils will be given meaning by expansion to variables (which, in turn, are given meaning by substitution, as described above). 

Each inner core form (defined in Figure \ref{fig:U-expanded-terms}) maps onto an outer surface form. We refer to these as the \emph{shared forms}. In particular:
\begin{itemize}
\item Each type variable, $t$, maps onto a unique {type sigil}, written $\sigilof{t}$ (pronounced ``sigil of $t$''). Notice the distinction between $\ut$, which is a metavariable ranging over type sigils, and $\sigilof{t}$, which is a metafunction, written in stylized form, applied to a type variable to produce a type sigil.
\item Each type form, $\tau$, maps onto an unexpanded type form, $\Uof{\tau}$, as follows: 
  \begin{align*}
  \Uof{t} &= \sigilof{t}\\
  \Uof{\aparr{\tau_1}{\tau_2}} & = \auparr{\Uof{\tau_1}}{\Uof{\tau_2}}\\
  \Uof{\aall{t}{\tau}} & = \auall{\sigilof{t}}{\Uof{\tau}}\\
  \Uof{\arec{t}{\tau}} & = \aurec{\sigilof{t}}{\Uof{\tau}}\\
  \Uof{\aprod{\labelset}{\mapschema{\tau}{i}{\labelset}}} & = \auprod{\labelset}{\mapschemax{\Uofv}{\tau}{i}{\labelset}}\\
  \Uof{\asum{\labelset}{\mapschema{\tau}{i}{\labelset}}} & = \ausum{\labelset}{\mapschemax{\Uofv}{\tau}{i}{\labelset}}
  \end{align*}
\item Each expression variable, $x$, maps onto a unique expression sigil written $\sigilof{x}$. Again, notice the distinction between $\ux$ and $\sigilof{x}$.
\item Each expanded expression form, $e$, maps onto an unexpanded expression form, $\Uof{e}$, as follows:
\begin{align*}
\Uof{x} & = \sigilof{x}\\
\Uof{\aelam{\tau}{x}{e}} & = \aulam{\Uof{\tau}}{\sigilof{x}}{\Uof{e}}\\
\Uof{\aeap{e_1}{e_2}} & = \auap{\Uof{e_1}}{\Uof{e_2}}\\
\Uof{\aetlam{t}{e}} & = \autlam{\sigilof{t}}{\Uof{e}}\\
\Uof{\aetap{e}{\tau}} & = \autap{\Uof{e}}{\Uof{\tau}}\\
\Uof{\aefold{t}{\tau}{e}} & = \aufold{\sigilof{t}}{\Uof\tau}{\Uof e}\\
\Uof{\aeunfold{e}} & = \auunfold{\Uof{e}}\\
\Uof{\aetpl{\labelset}{\mapschema{e}{i}{\labelset}}} & = \autpl{\labelset}{\mapschemax{\Uofv}{e}{i}{\labelset}}\\
\Uof{\aein{\labelset}{\ell}{\mapschema{\tau}{i}{\labelset}}{e}} &= \auin{\labelset}{\ell}{\mapschemax{\Uofv}{\tau}{i}{\labelset}}{\Uof{e}}\\
\Uof{\aecase{\labelset}{\tau}{e}{\mapschemab{x}{e}{i}{\labelset}}} & = \aucase{\labelset}{\Uof\tau}{\Uof{e}}{\mapschemabx{\Uofv}{x}{e}{i}{\labelset}}
\end{align*}
\end{itemize}

There are only two unexpanded expression forms, highlighted in gray in Figure \ref{fig:U-unexpanded-terms}, that do not correspond to expanded expression forms -- the ueTSM definition form and the ueTSM application form. %These are the ``interesting'' forms. % These are the ``interesting'' forms. % Let us define this correspondence by the metafunction $\Uof{e}$:
%\[
%\begin{split}
%\Uof{x} & = x\\
%\Uof{\aelam{\tau}{x}{e}} & = \aulam{\tau}{x}{\Uof{e}}\\
%\Uof{\aeap{e_1}{e_2}} & = \auap{\Uof{e_1}}{\Uof{e_2}}
%\end{split}
%\] and so on for the remaining expanded expression forms.


\subsection{Typed Expansion}\label{sec:typed-expansion-U}
Unexpanded expressions, and the unexpanded types therein, are checked and expanded simultaneously according to the \emph{typed expansion judgements}:
\[\begin{array}{ll}
\textbf{Judgement Form} & \textbf{Description}\\
\expandsTU{\uDelta}{\utau}{\tau} & \text{$\utau$ is well-formed and has expansion $\tau$ assuming $\uDelta$}\\
\expandsUX{\ue}{e}{\tau} & \text{$\ue$ has expansion $e$ and type $\tau$ under ueTSM context $\uSigma$}\\
& \text{assuming $\uDelta$ and $\uGamma$}
\end{array}\]
%\newcommand{\gray}[1]{{\color{gray} #1}}

\subsubsection{Type Expansion}
The \emph{type expansion judgement}, $\expandsTU{\uDelta}{\utau}{\tau}$, is inductively defined by the following rules.
\begin{subequations}\label{rules:expandsTU}
\begin{equation}\label{rule:expandsTU-var}
\inferrule{ }{\expandsTU{\uDelta, \uDhyp{\ut}{t}}{\ut}{t}}
\end{equation}
\begin{equation}\label{rule:expandsTU-parr}
\inferrule{
  \expandsTU{\uDelta}{\utau_1}{\tau_1}\\
  \expandsTU{\uDelta}{\utau_2}{\tau_2}
}{\expandsTU{\uDelta}{\auparr{\utau_1}{\utau_2}}{\aparr{\tau_1}{\tau_2}}}
\end{equation}
\begin{equation}\label{rule:expandsTU-all}
  \inferrule{
    \expandsTU{\uDelta, \uDhyp{\ut}{t}}{\utau}{\tau}
  }{
    \expandsTU{\uDelta}{\auall{\ut}{\utau}}{\aall{t}{\tau}}
  }
\end{equation}
\begin{equation}\label{rule:expandsTU-rec}
  \inferrule{
    \expandsTU{\uDelta, \uDhyp{\ut}{t}}{\utau}{\tau}
  }{
    \expandsTU{\uDelta}{\aurec{\ut}{\utau}}{\arec{t}{\tau}}
  }
\end{equation}
\begin{equation}\label{rule:expandsTU-prod}
  \inferrule{
    \{\expandsTU{\uDelta}{\utau_i}{\tau_i}\}_{i \in \labelset}
  }{
    \expandsTU{\uDelta}{\auprod{\labelset}{\mapschema{\utau}{i}{\labelset}}}{\aprod{\labelset}{\mapschema{\tau}{i}{\labelset}}}
  }
\end{equation}
\begin{equation}\label{rule:expandsTU-sum}
  \inferrule{
    \{\expandsTU{\uDelta}{\utau_i}{\tau_i}\}_{i \in \labelset}
  }{
    \expandsTU{\uDelta}{\ausum{\labelset}{\mapschema{\utau}{i}{\labelset}}}{\asum{\labelset}{\mapschema{\tau}{i}{\labelset}}}
  }
\end{equation}
\end{subequations}
\emph{Unexpanded type formation contexts}, $\uDelta$, are of the form $\uDD{\uD}{\Delta}$, where $\uD$ is a \emph{type sigil expansion context}, and $\Delta$ is a type formation context. A type sigil expansion context, $\uD$, is a finite function that maps each type sigil $\ut \in \domof{\uD}$ to the hypothesis $\vExpands{\ut}{t}$, for some type variable $t$. We write $\ctxUpdate{\uD}{\ut}{t}$ for the type sigil expansion context that maps $\ut$ to $\vExpands{\ut}{t}$ and defers to $\uD$ for all other type sigils (i.e. the previous mapping, if it exists, is updated). 
We define $\uDelta, \uDhyp{\ut}{t}$ when $\uDelta=\uDD{\uD}{\Delta}$ as an abbreviation of  \[\uDD{\ctxUpdate{\uD}{\ut}{t}}{\Delta, \Dhyp{t}}\]%type identifier expansion context is always extended/updated together with 
%We write $\uDeltaOK{\uDelta}$ when $\uDelta=\uDD{\uD}{\Delta}$ and each type variable in $\uD$ also appears in $\Delta$.
%\begin{definition}\label{def:uDeltaOK} $\uDeltaOK{\uDD{\uD}{\Delta}}$ iff for each $\vExpands{\ut}{t} \in \uD$, we have $\Dhyp{t} \in \Delta$.\end{definition}

To understand how type sigil expansion contexts operate, it is instructive to derive an expansion for the unexpanded type $\forallt{\ut}{\forallt{\ut}{\ut}}$, or in operational form, $\auall{\ut}{\auall{\ut}{\ut}}$:
\begin{mathpar}
\inferrule{
  \inferrule{
    \inferrule{ }{
      \expandsTU{\uDD{\vExpands{\ut}{t'}}{{\Dhyp{t}}, {\Dhyp{t'}}}}{\ut}{t'}
    }~\text{(\ref*{rule:expandsTU-var})}
  }{
    \expandsTU{\uDD{\vExpands{\ut}{t}}{\Dhyp{t}}}{\auall{\ut}{\ut}}{\aall{t'}{t'}}
  }~\text{(\ref*{rule:expandsTU-all})}
}{
  \expandsTU{\uDD{\emptyset}{\emptyset}}{\auall{\ut}{\auall{\ut}{\ut}}}{\aall{t}{\aall{t'}{t'}}}
}~\text{(\ref*{rule:expandsTU-all})}
\end{mathpar}
Notice that when a type sigil is bound, a fresh type variable is generated. The type sigil expansion context is extended (when the outermost binding is encountered) or updated (at all inner bindings) and the type formation context is simultaneously extended at each binding (so that typing contexts and ueTSM contexts, discussed below, that contain types that refer to the previous binding remain well-formed). Had we used type variables in the syntax and type formation contexts in the rules above, rather than type sigils and type sigil expansion contexts, derivations for unexpanded types where an inner binding shadows an outer binding would not exist, because by definition we cannot extend a type formation context with a variable it already mentions nor implicitly $\alpha$-vary the unexpanded type to sidestep this problem. 

These rules validate the following lemmas. The Type Expansion Lemma establishes that the expansion of an unexpanded type is a well-formed type.

\begin{lemma}[Type Expansion]\label{lemma:type-expansion-U} If $\expandsTU{\uDD{\uD}{\Delta}}{\utau}{\tau}$ then $\istypeU{\Delta}{\tau}$.\end{lemma}
\begin{proof} By rule induction over Rules (\ref{rules:expandsTU}). In each case, we apply the IH to or over each premise, then apply the corresponding type formation rule in Rules (\ref{rules:istypeU}). \end{proof}

The Type Expressibility Lemma establishes that every well-formed type, $\tau$, can be expressed as a well-formed unexpanded type, $\Uof{\tau}$. This requires defining the metafunction $\Uof{\Delta}$ which maps $\Delta$ onto a an unexpanded type formation context as follows:
\begin{align*}
\Uof{\emptyset} &= \uDD{\emptyset}{\emptyset}\\
\Uof{\Delta, \Dhyp{t}} &= \Uof{\Delta}, \uDhyp{\sigilof{t}}{t}
\end{align*}
\begin{lemma}[Type Expressibility]\label{lemma:type-expressibility} If $\istypeU{\Delta}{\tau}$ then $\expandsTU{\Uof{\Delta}}{\Uof{\tau}}{\tau}$.\end{lemma}
\begin{proof} By rule induction over Rules (\ref{rules:istypeU}) using the definitions of $\Uof{\tau}$ and $\Uof{\Delta}$ above. In each case, we apply the IH to or over each premise, then apply the corresponding type expansion rule in Rules (\ref{rules:expandsTU}).\end{proof}

\subsubsection{Typed Expression Expansion}
\begin{subequations}\label{rules:expandsU}
\emph{Unexpanded typing contexts}, $\uGamma$, are of the form $\uGG{\uG}{\Gamma}$, where $\uG$ is an \emph{expression sigil expansion context}, and $\Gamma$ is a typing context. An expression sigil expansion context, $\uG$, is a finite function that maps each expression sigil $\ux \in \domof{\uG}$ to the hypothesis $\vExpands{\ux}{x}$, for some expression variable, $x$. We write $\ctxUpdate{\uG}{\ux}{x}$ for the expression sigil expansion context that maps $\ux$ to $\vExpands{\ux}{x}$ and defers to $\uG$ for all other expression sigils (i.e. the previous mapping, if it exists, is updated). %We write $\uGammaOK{\uGamma}$ when $\uGamma=\uGG{\uG}{\Gamma}$ and each expression variable in $\uG$ is assigned a type by $\Gamma$.
%\begin{definition} $\uGammaOK{\uGG{\uG}{\Gamma}}$ iff for each $\vExpands{\ux}{x} \in \uG$, we have $\Ghyp{x}{\tau} \in \Gamma$ for some $\tau$.\end{definition}
%\noindent 
We define $\uGamma, \uGhyp{\ux}{x}{\tau}$ when $\uGamma = \uGG{\uG}{\Gamma}$ as an abbreviation of \[\uGG{\uG, \vExpands{\ux}{x}}{\Gamma, \Ghyp{x}{\tau}}\]

The \emph{typed expression expansion judgement}, $\expandsUX{\ue}{e}{\tau}$, is inductively defined by Rules (\ref*{rules:expandsU}) as follows. %These rules validate the following theorem, which establishes that typed expansion produces an expansion of the assigned type. 
%\begin{theorem}[Typed Expression Expansion] If $\expandsU{\uDD{\uD}{\Delta}}{\uGG{\uG}{\Gamma}}{\uSigma}{\ue}{e}{\tau}$ and $\uetsmenv{\Delta}{\uSigma}$ then $\hastypeU{\Delta}{\Gamma}{e}{\tau}$.\end{theorem}
%\begin{proof} This is the first part of Theorem \ref{thm:typed-expansion-U}, defined and proven below.\end{proof}

\paragraph{Shared Forms} Rules (\ref*{rule:expandsU-var}) through (\ref*{rule:expandsU-case}) handle unexpanded expressions of shared form. The first five of these rules are defined below:
%Each of these rules is based on the corresponding typing rule, i.e. Rules (\ref{rule:hastypeU-var}) through (\ref{rule:hastypeU-case}), respectively. For example, the following typed expansion rules are based on the typing rules (\ref{rule:hastypeU-var}), (\ref{rule:hastypeU-lam}) and (\ref{rule:hastypeU-ap}), respectively:% for unexpanded expressions of variable, function and application form, respectively: 
\begin{equation}\label{rule:expandsU-var}
  \inferrule{ }{\expandsU{\uDelta}{\uGamma, \uGhyp{\ux}{x}{\tau}}{\uSigma}{\ux}{x}{\tau}}
\end{equation}
\begin{equation}\label{rule:expandsU-lam}
  \inferrule{
    \expandsTU{\uDelta}{\utau}{\tau}\\
    \expandsU{\uDelta}{\uGamma, \uGhyp{\ux}{x}{\tau}}{\uSigma}{\ue}{e}{\tau'}
  }{\expandsUX{\aulam{\utau}{\ux}{\ue}}{\aelam{\tau}{x}{e}}{\aparr{\tau}{\tau'}}}
\end{equation}
\begin{equation}\label{rule:expandsU-ap}
  \inferrule{
    \expandsUX{\ue_1}{e_1}{\aparr{\tau}{\tau'}}\\
    \expandsUX{\ue_2}{e_2}{\tau}
  }{
    \expandsUX{\auap{\ue_1}{\ue_2}}{\aeap{e_1}{e_2}}{\tau'}
  }
\end{equation}
\begin{equation}\label{rule:expandsU-tlam}
  \inferrule{
    \expandsU{\uDelta, \uDhyp{\ut}{t}}{\uGamma}{\uSigma}{\ue}{e}{\tau}
  }{
    \expandsUX{\autlam{\ut}{\ue}}{\aetlam{t}{e}}{\aall{t}{\tau}}
  }
\end{equation}
\begin{equation}\label{rule:expandsU-tap}
  \inferrule{
    \expandsUX{\ue}{e}{\aall{t}{\tau}}\\
    \expandsTU{\uDelta}{\utau'}{\tau'}
  }{
    \expandsUX{\autap{\ue}{\utau'}}{\aetap{e}{\tau'}}{[\tau'/t]\tau}
  }
\end{equation}
Observe that, in each of these rules, the unexpanded and expanded expression forms in the conclusion correspond, and the premises correspond to those of the typing rule for the expanded expression form, i.e. Rules (\ref{rule:hastypeU-var}) through (\ref{rule:hastypeU-tap}), respectively. In particular, each type expansion premise in each rule above corresponds to a  type formation premise in the corresponding typing rule, and each typed expression expansion premise in each rule above corresponds to a typing premise in the corresponding typing rule. The type assigned in the conclusion of each rule above is identical to the type assigned in the conclusion of the corresponding typing rule. The ueTSM context, $\uSigma$, passes opaquely through these rules (we will define ueTSM contexts below). Rules (\ref{rules:expandsTU}) were similarly generated by mechanically transforming Rules (\ref{rules:istypeU}).

We can express this scheme more precisely with the following rule transformation. For each rule in Rules (\ref{rules:istypeU}) and Rules (\ref{rules:hastypeU}),
\begin{mathpar}
\refstepcounter{equation}
% \label{rule:expandsU-tlam}
% \refstepcounter{equation}
% \label{rule:expandsU-tap}
% \refstepcounter{equation}
\label{rule:expandsU-fold}
\refstepcounter{equation}
\label{rule:expandsU-unfold}
\refstepcounter{equation}
\label{rule:expandsU-tpl}
\refstepcounter{equation}
\label{rule:expandsU-pr}
\refstepcounter{equation}
\label{rule:expandsU-in}
\refstepcounter{equation}
\label{rule:expandsU-case}
\inferrule{J_1\\ \cdots \\ J_k}{J}
\end{mathpar}
the corresponding typed expansion rule is 
\begin{mathpar}
\inferrule{
  \Uof{J_1} \\
  \cdots\\
  \Uof{J_k}
}{
  \Uof{J}
}
\end{mathpar}
where
\[\begin{split}
\Uof{\istypeU{\Delta}{\tau}} & = \expandsTU{\Uof{\Delta}}{\Uof{\tau}}{\tau} \\
\Uof{\hastypeU{\Gamma}{\Delta}{e}{\tau}} & = \expandsU{\Uof{\Gamma}}{\Uof{\Delta}}{\uSigma}{\Uof{e}}{e}{\tau}\\
\Uof{\{J_i\}_{i \in \labelset}} & = \{\Uof{J_i}\}_{i \in \labelset}
\end{split}\]
and where:
\begin{itemize}
\item $\Uof{\tau}$ is defined as follows:
  \begin{itemize}
  \item When $\tau$ is of definite form, $\Uof{\tau}$ is defined as in Sec. \ref{sec:syntax-U}.
  \item When $\tau$ is of indefinite form, $\Uof{\tau}$ is a uniquely corresponding metavariable of sort $\mathsf{UTyp}$ also of indefinite form. For example, in Rule (\ref{rule:istypeU-parr}), $\tau_1$ and $\tau_2$ are of indefinite form, i.e. they match arbitrary types. The rule transformation simply ``hats'' them, i.e. $\Uof{\tau_1}=\utau_1$ and $\Uof{\tau_2}=\utau_2$.
  \end{itemize}
\item $\Uof{e}$ is defined as follows
\begin{itemize}
\item When $e$ is of definite form, $\Uof{e}$ is defined as in Sec. \ref{sec:syntax-U}. 
\item When $e$ is of indefinite form, $\Uof{e}$ is a uniquely corresponding metavariable of sort $\mathsf{UExp}$ also of indefinite form. For example, $\Uof{e_1}=\ue_1$ and $\Uof{e_2}=\ue_2$.
\end{itemize}
\item $\Uof{\Delta}$ is defined as follows:
  \begin{itemize} 
  \item When $\Delta$ is of definite form, $\Uof{\Delta}$ is defined as above.
  \item When $\Delta$ is of indefinite form, $\Uof{\Delta}$ is a uniquely corresponding metavariable ranging over unexpanded type formation contexts. For example, $\Uof{\Delta} = \uDelta$.
  \end{itemize}
\item $\Uof{\Gamma}$ is defined as follows:
  \begin{itemize}
  \item When $\Gamma$ is of definite form, $\Uof{\Gamma}$ produces the corresponding unexpanded typing context as follows:
\begin{align*}
\Uof{\emptyset} & = \uGG{\emptyset}{\emptyset}\\
\Uof{\Gamma, \Ghyp{x}{\tau}} & = \Uof{\Gamma}, \uGhyp{\sigilof{x}}{x}{\tau}
\end{align*}
  \item When $\Gamma$ is of indefinite form, $\Uof{\Gamma}$ is a uniquely corresponding metavariable ranging over unexpanded typing contexts. For example, $\Uof{\Gamma} = \uGamma$.
\end{itemize}
\end{itemize}

It is instructive to use this rule transformation to generate Rules (\ref{rules:expandsTU}) and Rules (\ref{rule:expandsU-var}) through (\ref{rule:expandsU-tap}) above. We omit the remaining rules, i.e. Rules (\ref*{rule:expandsU-fold}) through (\ref*{rule:expandsU-case}). By instead defining these rules solely by the rule transformation just described, we avoid having to write down a number of rules that are of limited marginal interest. Moreover, this demonstrates the general technique for generating typed expansion rules for unexpanded types and expressions of shared form, so our exposition is somewhat ``robust'' to changes to the inner core. 

We can now establish the Expressibility Theorem -- that each well-typed expanded expression, $e$, can be expressed as an unexpanded expression, $\ue$, and assigned the same type under the corresponding contexts.

\begin{theorem}[Expressibility] If $\hastypeU{\Delta}{\Gamma}{e}{\tau}$ then $\expandsU{\Uof{\Delta}}{\Uof{\Gamma}}{\uSigma}{\Uof{e}}{e}{\tau}$.\end{theorem}
\begin{proof} By rule induction over Rules (\ref{rules:hastypeU}). The above rule transformation guarantees that this theorem holds by its construction. In particular, in each case, we can apply Lemma \ref{lemma:type-expressibility} to or over each type formation premise, the IH to or over each typing premise, then apply the corresponding rule in Rules (\ref{rules:expandsU}).\end{proof}
%o that when the inner core changes,  typed expansion rules  our exposition somewhat robust to changes to the inner core (though not to changes to the judgement forms in the statics of the inner core).% Even if changes to the judgement forms in the statics of the inner core are needed (e.g. the addition of a symbol context), it is easy to see would correspond to changes in the generic specification above.
% \begin{subequations}\label{rules:expandsU}
% \begin{equation}\label{rule:expandsU-var}
%   \inferrule{ }{\expandsU{\Delta}{\Gamma, x : \tau}{\uSigma}{x}{x}{\tau}}
% \end{equation}
% \begin{equation}\label{rule:expandsU-lam}
%   \inferrule{
%     \istypeU{\Delta}{\tau}\\
%     \expandsU{\Delta}{\Gamma, x : \tau}{\uSigma}{\ue}{e}{\tau'}
%   }{\expandsUX{\aulam{\tau}{x}{\ue}}{\aelam{\tau}{x}{e}}{\aparr{\tau}{\tau'}}}
% \end{equation}
% \begin{equation}\label{rule:expandsU-ap}
%   \inferrule{
%     \expandsUX{\ue_1}{e_1}{\aparr{\tau}{\tau'}}\\
%     \expandsUX{\ue_2}{e_2}{\tau}
%   }{
%     \expandsUX{\auap{\ue_1}{\ue_2}}{\aeap{e_1}{e_2}}{\tau'}
%   }
% \end{equation}
% \begin{equation}\label{rule:expandsU-tlam}
%   \inferrule{
%     \expandsU{\Delta, \Dhyp{t}}{\Gamma}{\uSigma}{\ue}{e}{\tau}
%   }{
%     \expandsUX{\autlam{t}{\ue}}{\aetlam{t}{e}}{\aall{t}{\tau}}
%   }
% \end{equation}
% \begin{equation}\label{rule:expandsU-tap}
%   \inferrule{
%     \expandsUX{\ue}{e}{\aall{t}{\tau}}\\
%     \istypeU{\Delta}{\tau'}
%   }{
%     \expandsUX{\autap{\ue}{\tau'}}{\aetap{e}{\tau'}}{[\tau'/t]\tau}
%   }
% \end{equation}
% \begin{equation}\label{rule:expandsU-fold}
%   \inferrule{
%     \istypeU{\Delta, \Dhyp{t}}{\tau}\\
%     \expandsUX{\ue}{e}{[\arec{t}{\tau}/t]\tau}
%   }{
%     \expandsUX{\aufold{t}{\tau}{\ue}}{\aefold{t}{\tau}{e}}{\arec{t}{\tau}}
%   }
% \end{equation}
% \begin{equation}\label{rule:expandsU-unfold}
%   \inferrule{
%     \expandsUX{\ue}{e}{\arec{t}{\tau}}
%   }{
%     \expandsUX{\auunfold{\ue}}{\aeunfold{e}}{[\arec{t}{\tau}/t]\tau}
%   }
% \end{equation}
% \begin{equation}\label{rule:expandsU-tpl}
%   \inferrule{
%     \{\expandsUX{\ue_i}{e_i}{\tau_i}\}_{i \in \labelset}
%   }{
%     \expandsUX{\autpl{\labelset}{\mapschema{\ue}{i}{\labelset}}}{\aetpl{\labelset}{\mapschema{e}{i}{\labelset}}}{\aprod{\labelset}{\mapschema{\tau}{i}{\labelset}}}
%   }
% \end{equation}
% \begin{equation}\label{rule:expandsU-pr}
%   \inferrule{
%     \expandsUX{\ue}{e}{\aprod{\labelset, \ell}{\mapschema{\tau}{i}{\labelset}; \mapitem{\ell}{\tau}}}
%   }{
%     \expandsUX{\aupr{\ell}{\ue}}{\aepr{\ell}{e}}{\tau}
%   }
% \end{equation}
% \begin{equation}\label{rule:expandsU-in}
%   \inferrule{
%     \{\istypeU{\Delta}{\tau_i}\}_{i \in \labelset}\\
%     \istypeU{\Delta}{\tau}\\
%     \expandsUX{\ue}{e}{\tau}
%   }{
%     \left\{\shortstack{$\Delta~\Gamma \vdash_\uSigma \auin{\labelset, \ell}{\ell}{\mapschema{\tau}{i}{\labelset}; \mapitem{\ell}{\tau}}{\ue}$\\$\leadsto$\\$\aein{\labelset, \ell}{\ell}{\mapschema{\tau}{i}{\labelset}; \mapitem{\ell}{\tau}}{e} : \asum{\labelset, \ell}{\mapschema{\tau}{i}{\labelset}; \mapitem{\ell}{\tau}}$\vspace{-1.2em}}\right\}
%   }
% \end{equation}
% \begin{equation}\label{rule:expandsU-case}
%   \inferrule{
%     \expandsUX{\ue}{e}{\asum{\labelset}{\mapschema{\tau}{i}{\labelset}}}\\
%     \{\expandsU{\Delta}{\Gamma, \Ghyp{x_i}{\tau_i}}{\uSigma}{\ue_i}{e_i}{\tau}\}_{i \in \labelset}
%   }{
%     \expandsUX{\aucase{\labelset}{\ue}{\mapschemab{x}{\ue}{i}{\labelset}}}{\aecase{\labelset}{e}{\mapschemab{x}{e}{i}{\labelset}}}{\tau}
%   }
% \end{equation}
\end{subequations}
\paragraph{ueTSM Definition and Application} The two remaining typed expansion rules, Rules (\ref{rule:expandsU-syntax}) and (\ref{rule:expandsU-tsmap}), govern the ueTSM definition and application forms, and are defined in the next two subsections, respectively. 

% \begin{equation}\label{rule:expandsU-syntax}
% \inferrule{
%   \istypeU{\Delta}{\tau}\\
%   \expandsU{\emptyset}{\emptyset}{\emptyset}{\ueparse}{\eparse}{\aparr{\tBody}{\tParseResultExp}}\\\\
%   a \notin \domof{\uSigma}\\
%   \expandsU{\Delta}{\Gamma}{\uSigma, \xuetsmbnd{\tsmv}{\tau}{\eparse}}{\ue}{e}{\tau'}
% }{
%   \expandsUX{\audefuetsm{\tau}{\ueparse}{\tsmv}{\ue}}{e}{\tau'}
% }
% \end{equation}
% \begin{equation}\label{rule:expandsU-tsmap}
% \inferrule{
%   \encodeBody{b}{\ebody}\\
%   \evalU{\ap{\eparse}{\ebody}}{\inj{\lbltxt{Success}}{\ecand}}\\
%   \decodeCondE{\ecand}{\ce}\\\\
%   \cvalidE{\emptyset}{\emptyset}{\esceneU{\Delta}{\Gamma}{\uSigma, \xuetsmbnd{\tsmv}{\tau}{\eparse}}{b}}{\ce}{e}{\tau}
% }{
%   \expandsU{\Delta}{\Gamma}{\uSigma, \xuetsmbnd{\tsmv}{\tau}{\eparse}}{\autsmap{b}{\tsmv}}{e}{\tau}
% }
% \end{equation}
%\end{subequations}

%Notice that each form of expanded expression (Figure \ref{fig:U-expanded-terms}) corresponds to a form of unexpanded expression (Figure \ref{fig:U-unexpanded-terms}). For each typing rule in Rules (\ref{rules:hastypeU}), there is a corresponding typed expansion rule -- Rules (\ref{rule:expandsU-var}) through (\ref{rule:expandsU-case}) -- where the unexpanded and expanded forms correspond. The premises also correspond -- if a typing judgement appears as a premise of a typing rule, then the corresponding premise in the corresponding typed expansion rule is the corresponding typed expansion judgement. The ueTSM context is not extended or inspected by these rules (it is only ``threaded through'' them opaquely).

%There are two unexpanded expression forms that do not correspond to an expanded expression form: the ueTSM definition form, and the ueTSM application form. The rules governing these two forms interact with the ueTSM context, and are the topics of the next two subsections, respectively.

\subsection{ueTSM Definitions}\label{sec:U-uetsm-definition}
The stylized ueTSM definition form is \[\uesyntax{\tsmv}{\utau}{\eparse}{\ue}\] 
%The operational form corresponding to this stylized form is \[\audefuetsm{\utau}{\eparse}{\tsmv}{\ue}\]
An unexpanded expression of this form defines a {ueTSM} named $\tsmv$ with \emph{unexpanded type annotation} $\utau$ and \emph{parse function} $\eparse$ for use within $\ue$. 

The parse function is an expanded expression because parse functions are applied statically (i.e. during typed expansion of $\ue$), as we will discuss when describing ueTSM application below, and evaluation is defined only for closed expanded expressions. This construction simplifies our exposition, though it is not entirely practical because it provides no way for TSM providers to share values between parse functions, nor any way to use TSMs when defining other TSMs. We discuss enriching the language to eliminate these limitations in Sec. \ref{sec:uetsms-static-language}, but it is pedagogically simpler to leave the necessary machinery out of our calculus for now.%$\miniVerseUE$.

Rule (\ref*{rule:expandsU-syntax}) defines typed expansion of ueTSM definitions (we use stylized forms for clarity):
\begin{subequations}[resume]
% \begin{equation}\label{rule:expandsU-syntax}
% \inferrule{
%   \istypeU{\Delta}{\tau}\\
%   \expandsU{\emptyset}{\emptyset}{\emptyset}{\ueparse}{\eparse}{\aparr{\tBody}{\tParseResultExp}}\\\\
%   \expandsU{\Delta}{\Gamma}{\uSigma, \xuetsmbnd{\tsmv}{\tau}{\eparse}}{\ue}{e}{\tau'}
% }{
%   \expandsUX{\audefuetsm{\tau}{\ueparse}{\tsmv}{\ue}}{e}{\tau'}
% }
% \end{equation}
\begin{equation}\label{rule:expandsU-syntax}
\inferrule{
  \expandsTU{\uDelta}{\utau}{\tau}\\
  \hastypeU{\emptyset}{\emptyset}{\eparse}{\parr{\tBody}{\tParseResultExp}}\\\\
  \expandsU{\uDelta}{\uGamma}{\uSigma, \uShyp{\tsmv}{a}{\tau}{\eparse}}{\ue}{e}{\tau'}
}{
  \expandsUX{\uesyntax{\tsmv}{\utau}{\eparse}{\ue}}{e}{\tau'}
}
\end{equation}
\end{subequations}
The premises of this rule can be understood as follows, in order:
\begin{enumerate}
\item The first premise ensures that the unexpanded type annotation is well-formed and expands it to produce the \emph{type annotation}, $\tau$.

\item The second premise checks that the parse function, $\eparse$, is closed and of type \[\parr{\tBody}{\tParseResultExp}\] %to generate the \emph{expanded parse function}, $\eparse$. 
 %Notice that this occurs under empty contexts, i.e. parse functions cannot refer to the surrounding bindings. 
%The parse function must be of type $\aparr{\tBody}{\tParseResultExp}$ where the type abbreviations $\tBody$ and $\tParseResultExp$ are defined as follows.

The type abbreviated $\tBody$ classifies encodings of literal bodies, $b$. The mapping from literal bodies to values of type $\tBody$ is defined by the \emph{body encoding judgement} $\encodeBody{b}{\ebody}$. An inverse mapping is defined   by the \emph{body decoding judgement} $\decodeBody{\ebody}{b}$.
\[\begin{array}{ll}
\textbf{Judgement Form} & \textbf{Description}\\
\encodeBody{b}{e} & \text{$b$ has encoding $e$}\\
\decodeBody{e}{b} & \text{$e$ has decoding $b$}
\end{array}\]
Rather than defining $\tBody$ explicitly, and these judgements inductively against that definition (which would be tedious and uninteresting), it suffices to define the following condition, which establishes an isomorphism between literal bodies and values of type $\tBody$ mediated by the judgements above.
\begin{condition}[Body Isomorphism] All of the following must hold:
\begin{enumerate}
\item For every literal body $b$, we have that $\encodeBody{b}{\ebody}$ for some $\ebody$ such that $\hastypeUC{\ebody}{\tBody}$ and $\isvalU{\ebody}$.
\item If $\hastypeUC{\ebody}{\tBody}$ and $\isvalU{\ebody}$ then $\decodeBody{\ebody}{b}$ for some $b$.
\item If $\encodeBody{b}{\ebody}$ then $\decodeBody{\ebody}{b}$.
\item If $\hastypeUC{\ebody}{\tBody}$ and $\isvalU{\ebody}$ and $\decodeBody{\ebody}{b}$ then $\encodeBody{b}{\ebody}$. 
\item If $\encodeBody{b}{\ebody}$ and $\encodeBody{b}{\ebody'}$ then $\ebody = \ebody'$.
\item If $\hastypeUC{\ebody}{\tBody}$ and $\isvalU{\ebody}$ and $\decodeBody{\ebody}{b}$ and $\decodeBody{\ebody}{b'}$ then $b=b'$.
\end{enumerate}
\end{condition}

$\tParseResultExp$ abbreviates a labeled sum type that distinguishes successful parses from parse errors\footnote{In VerseML, the \li{ParseError} constructor of \li{ParseResult} required an error message and an error location, but we omit these in our formalization for simplicity}:
\[\tParseResultExp \triangleq [\mapitem{\lbltxt{Success}}{\tCEExp}, \mapitem{\lbltxt{ParseError}}{\prodt{}}]\] 

The type abbreviated $\tCEExp$ classifies encodings of \emph{candidate expansion expressions} (or \emph{ce-expressions}), $\ce$ (pronounced ``grave $e$''). The syntax of ce-expressions will be described in Sec. \ref{sec:ce-syntax-U}. The mapping from ce-expressions to values of type $\tCEExp$ is defined by the \emph{ce-expression encoding judgement}, $\encodeCondE{\ce}{e}$. An inverse mapping is defined by the \emph{ce-expression decoding judgement}, $\decodeCondE{e}{\ce}$.

\[\begin{array}{ll}
\textbf{Judgement Form} & \textbf{Description}\\
\encodeCondE{\ce}{e} & \text{$\ce$ has encoding $e$}\\
\decodeCondE{e}{\ce} & \text{$e$ has decoding $\ce$}
\end{array}\]

Again, rather than picking a particular definition of $\tCEExp$ and defining the judgements above inductively against it, we only state the following condition, which establishes an isomorphism between values of type $\tCEExp$ and ce-expressions.

\begin{condition}[Candidate Expansion Expression Isomorphism] All of the following must hold:
\begin{enumerate}
\item For every $\ce$, we have $\encodeCondE{\ce}{\ecand}$ for some $\ecand$ such that $\hastypeUC{\ecand}{\tCEExp}$ and $\isvalU{\ecand}$.
\item If $\hastypeUC{\ecand}{\tCEExp}$ and $\isvalU{\ecand}$ then $\decodeCondE{\ecand}{\ce}$ for some $\ce$.
\item If $\encodeCondE{\ce}{\ecand}$ then $\decodeCondE{\ecand}{\ce}$.
\item If $\hastypeUC{\ecand}{\tCEExp}$ and $\isvalU{\ecand}$ and $\decodeCondE{\ecand}{\ce}$ then $\encodeCondE{\ce}{\ecand}$.
\item If $\encodeCondE{\ce}{\ecand}$ and $\encodeCondE{\ce}{\ecand'}$ then $\ecand=\ecand'$.
\item If $\hastypeUC{\ecand}{\tCEExp}$ and $\isvalU{\ecand}$ and $\decodeCondE{\ecand}{\ce}$ and $\decodeCondE{\ecand}{\ce'}$ then $\ce=\ce'$.
\end{enumerate}
\end{condition}


\item The final premise of Rule (\ref{rule:expandsU-syntax}) extends the ueTSM context, $\uSigma$, with the newly determined {ueTSM definition}, and proceeds to assign a type, $\tau'$, and expansion, $e$, to $\ue$. The conclusion of Rule (\ref{rule:expandsU-syntax}) assigns this type and expansion to the ueTSM definition as a whole.% i.e. TSMs define behavior that is relevant during typed expansion, but not during evaluation. 



\emph{ueTSM contexts}, $\uSigma$, are of the form $\uAS{\uA}{\Sigma}$, where $\uA$ is a \emph{TSM naming context} and $\Sigma$ is a \emph{ueTSM definition context}. 

A \emph{TSM naming context}, $\uA$, is a finite function mapping each TSM name $\tsmv \in \domof{\uA}$ to the \emph{TSM name-symbol mapping}, $\vExpands{\tsmv}{a}$, for some \emph{symbol}, $a$. We write $\ctxUpdate{\uA}{\tsmv}{a}$ for the ueTSM naming context that maps $\tsmv$ to $\vExpands{\tsmv}{a}$, and defers to $\uA$ for all other TSM names (i.e. the previous mapping, if it exists, is updated).

A \emph{ueTSM definition context}, $\Sigma$, is a finite function mapping each symbol $a \in \domof{\Sigma}$ to an \emph{expanded ueTSM definition}, $\xuetsmbnd{a}{\tau}{\eparse}$, where $\tau$ is the ueTSM's type annotation, and $\eparse$ is its parse function. We write $\Sigma, \xuetsmbnd{a}{\tau}{\eparse}$ when $a \notin \domof{\Sigma}$ for the extension of $\Sigma$ that maps $a$ to $\xuetsmbnd{a}{\tau}{\eparse}$. We write $\uetsmenv{\Delta}{\Sigma}$  when all the type annotations in $\Sigma$ are well-formed assuming $\Delta$, and the parse functions in $\Sigma$ are closed and of type $\parr{\tBody}{\tParseResultExp}$.

\begin{definition}[ueTSM Definition Context Formation]\label{def:ueTSM-def-ctx-formation} $\uetsmenv{\Delta}{\Sigma}$ iff for each $\xuetsmbnd{\tsmv}{\tau}{\eparse} \in \Sigma$, we have $\istypeU{\Delta}{\tau}$ and $\hastypeU{\emptyset}{\emptyset}{\eparse}{\parr{\tBody}{\tParseResultExp}}$.\end{definition}

We define $\uSigma, \uShyp{\tsmv}{a}{\tau}{\eparse}$, when $\uSigma=\uAS{\uA}{\Sigma}$, as an abbreviation of \[\uAS{\ctxUpdate{\uA}{\tsmv}{a}}{\Sigma, \xuetsmbnd{a}{\tau}{\eparse}}\]

\end{enumerate}

% \[\begin{array}{ll}
% \textbf{Judgement Form} & \textbf{Description}\\
% \uetsmenv{\Delta}{\uSigma} & \text{$\uSigma$ is well-formed assuming $\Delta$}\end{array}\]
% This judgement is inductively defined by the following rules:
% \begin{subequations}[intermezzo]\label{rules:uetsmenv-U}
% \begin{equation}\label{rule:uetsmenv-empty}
% \inferrule{ }{\uetsmenv{\Delta}{\emptyset}}
% \end{equation}
% \begin{equation}\label{rule:uetsmenv-ext}
% \inferrule{
%   \uetsmenv{\Delta}{\uSigma}\\
%   \istypeU{\Delta}{\tau}\\
%   \hastypeU{\emptyset}{\emptyset}{\eparse}{\aparr{\tBody}{\tParseResultExp}}
% }{
%   \uetsmenv{\Delta}{\uSigma, \xuetsmbnd{\tsmv}{\tau}{\eparse}}
% }
% \end{equation}
% \end{subequations}

\subsection{ueTSM Application}\label{sec:U-uetsm-application}
The stylized unexpanded expression form for applying a ueTSM named $\tsmv$ to a literal form with literal body $b$ is:
\[
\utsmap{\tsmv}{b}
\] 
This stylized form uses forward slashes to delimit the literal body, but stylized variants of any of the literal forms specified in Figure \ref{fig:literal-forms} could also be added to Figure \ref{fig:U-unexpanded-terms}. % (we omit them for simplicity).
The corresponding operational form is $\autsmap{b}{\tsmv}$. %i.e. for each literal body $b$, the operator $\texttt{uapuetsm}[b]$ is indexed by the TSM name $\tsmv$ and takes no arguments. %\footnote{This is in following the conventions in \emph{PFPL} \cite{pfpl}, where operators parameters allow for the use of metatheoretic objects that are not syntax trees or binding trees, e.g. $\mathsf{str}[s]$ and $\mathsf{num}[n]$.} This operator is indexed by the TSM name $\tsmv$ and takes no arguments. 

The typed expansion rule governing ueTSM application is below:
\begin{subequations}[resume]
% \begin{equation}\label{rule:expandsU-tsmap}
% \inferrule{
%   \encodeBody{b}{\ebody}\\
%   \evalU{\ap{\eparse}{\ebody}}{\inj{\lbltxt{Success}}{\ecand}}\\
%   \decodeCondE{\ecand}{\ce}\\\\
%   \cvalidE{\emptyset}{\emptyset}{\esceneU{\Delta}{\Gamma}{\uSigma, \xuetsmbnd{\tsmv}{\tau}{\eparse}}{b}}{\ce}{e}{\tau}
% }{
%   \expandsU{\Delta}{\Gamma}{\uSigma, \xuetsmbnd{\tsmv}{\tau}{\eparse}}{\autsmap{b}{\tsmv}}{e}{\tau}
% }
% \end{equation}
\begin{equation}\label{rule:expandsU-tsmap}
\inferrule{
  \encodeBody{b}{\ebody}\\
  \evalU{\ap{\eparse}{\ebody}}{\inj{\lbltxt{Success}}{\ecand}}\\
  \decodeCondE{\ecand}{\ce}\\\\
  \cvalidE{\emptyset}{\emptyset}{\esceneU{\uDelta}{\uGamma}{\uSigma, \uShyp{\tsmv}{a}{\tau}{\eparse}}{b}}{\ce}{e}{\tau}
}{
  \expandsU{\uDelta}{\uGamma}{\uSigma, \uShyp{\tsmv}{a}{\tau}{\eparse}}{\utsmap{\tsmv}{b}}{e}{\tau}
}
\end{equation}
\end{subequations}
The premises of Rule (\ref{rule:expandsU-tsmap}) can be understood as follows, in order:
\begin{enumerate}
\item The first premise determines the encoding of the literal body, $\ebody$ (see above).
\item The second premise applies the parse function $\eparse$, which appears in the ueTSM context associated with $\tsmv$, to $\ebody$. If parsing succeeds, i.e. a value of the (stylized) form $\inj{\lbltxt{Success}}{\ecand}$ results from evaluation, then $\ecand$ will be a value of type $\tCEExp$ (assuming a well-formed ueTSM context, by application of Assumption \ref{condition:preservation-U}). We call $\ecand$ the \emph{encoding of the candidate expansion}.

If the parse function produces a value labeled $\lbltxt{ParseError}$, then typed expansion fails. No rule is necessary to handle this case. 

\item The third premise decodes the encoding of the candidate expansion to produce the \emph{candidate expansion}, $\ce$ (see above).




\item The final premise of Rule (\ref{rule:expandsU-tsmap}) \emph{validates} the candidate expansion and simultaneously generates the \emph{final expansion}, $e$. This is the topic of Sec. \ref{sec:ce-validation-U}.
\end{enumerate}
\subsection{Syntax of Candidate Expansions}\label{sec:ce-syntax-U}

\begin{figure}
\hspace{-5px}$\arraycolsep=3.5pt\begin{array}{lllllll}
\textbf{Sort} & & & \textbf{Operational Form} & \textbf{Stylized Form} & \textbf{Description}\\
\mathsf{CETyp} & \ctau & ::= & t & t & \text{variable}\\
&&& \aceparr{\ctau}{\ctau} & \parr{\ctau}{\ctau} & \text{partial function}\\
&&& \aceall{t}{\ctau} & \forallt{t}{\ctau} & \text{polymorphic}\\
&&& \acerec{t}{\ctau} & \rect{t}{\ctau} & \text{recursive}\\
&&& \aceprod{\labelset}{\mapschema{\ctau}{i}{\labelset}} & \prodt{\mapschema{\ctau}{i}{\labelset}} & \text{labeled product}\\
&&& \acesum{\labelset}{\mapschema{\ctau}{i}{\labelset}} & \sumt{\mapschema{\ctau}{i}{\labelset}} & \text{labeled sum}\\
\LCC &&& \lightgray & \lightgray & \lightgray\\
&&& \acesplicedt{m}{n} & \splicedt{m}{n} & \text{spliced}\\\ECC
\mathsf{CEExp} & \ce & ::= & x & x & \text{variable}\\
&&& \acelam{\ctau}{x}{\ce} & \lam{x}{\ctau}{\ce} & \text{abstraction}\\
&&& \aceap{\ce}{\ce} & \ap{\ce}{\ce} & \text{application}\\
&&& \acetlam{t}{\ce} & \Lam{t}{\ce} & \text{type abstraction}\\
&&& \acetap{\ce}{\ctau} & \App{\ce}{\ctau} & \text{type application}\\
&&& \acefold{t}{\ctau}{\ce} & \fold{\ce} & \text{fold}\\
&&& \aceunfold{\ce} & \unfold{\ce} & \text{unfold}\\
&&& \acetpl{\labelset}{\mapschema{\ce}{i}{\labelset}} & \tpl{\mapschema{\ce}{i}{\labelset}} & \text{labeled tuple}\\
&&& \acepr{\ell}{\ce} & \prj{\ce}{\ell} & \text{projection}\\
&&& \acein{\labelset}{\ell}{\mapschema{\ctau}{i}{\labelset}}{\ce} & \inj{\ell}{\ce} & \text{injection}\\
&&& \acecase{\labelset}{\tau}{\ce}{\mapschemab{x}{\ce}{i}{\labelset}} & \caseof{\ce}{\mapschemab{x}{\ce}{i}{\labelset}} & \text{case analysis}\\
\LCC &&& \lightgray & \lightgray & \lightgray\\
&&& \acesplicede{m}{n} & \splicede{m}{n} & \text{spliced}\ECC
\end{array}$
\caption[Syntax of candidate expansion types and expressions in $\miniVerseUE$]{Abstract syntax of candidate expansion types and expressions in $\miniVerseUE$. Metavariables $m$ and $n$ range over natural numbers. Candidate expansion types and expressions are identified up to $\alpha$-equivalence.}
\label{fig:U-candidate-terms}
\end{figure}

Figure \ref{fig:U-candidate-terms} defines the syntax of candidate expansion types (or \emph{ce-types}), $\ctau$, and candidate expansion expressions  (or \emph{ce-expressions}), $\ce$. Candidate expansion types and expressions are identified up to $\alpha$-equivalence in the usual manner.

Each inner core form maps onto a candidate expansion form. We refer to these as the \emph{shared forms}. In particular:

\begin{itemize}
  \item Each type form maps onto a ce-type form according to the metafunction $\Cof{\tau}$, defined as follows:
  \begin{align*}
  \Cof{t} & = t\\
  \Cof{\aparr{\tau_1}{\tau_2}} & = \aceparr{\Cof{\tau_1}}{\Cof{\tau_2}}\\
  \Cof{\aall{t}{\tau}} & = \aceall{t}{\Cof{\tau}}\\
  \Cof{\arec{t}{\tau}} & = \acerec{t}{\Cof{\tau}}\\
  \Cof{\aprod{\labelset}{\mapschema{\tau}{i}{\labelset}}} & = \aceprod{\labelset}{\mapschemax{\Cofv}{\ctau}{i}{\labelset}}\\
  \Cof{\asum{\labelset}{\mapschema{\tau}{i}{\labelset}}} & = \acesum{\labelset}{\mapschemax{\Cofv}{\ctau}{i}{\labelset}}
  \end{align*}
  \item Each expanded expression form maps onto a ce-expression form according to the metafunction $\Cof{e}$, defined as follows:
  \begin{align*}
  \Cof{x} & = x\\
  \Cof{\aelam{\tau}{x}{e}} & = \acelam{\Cof{\tau}}{x}{\Cof{e}}\\
  \Cof{\aeap{e_1}{e_2}} & = \aceap{\Cof{e_1}}{\Cof{e_2}}\\
  \Cof{\aetlam{t}{e}} & = \acetlam{t}{\Cof{e}}\\
  \Cof{\aetap{e}{\tau}} & = \acetap{\Cof{e}}{\Cof{\tau}}\\
  \Cof{\aefold{t}{\tau}{e}} & = \acefold{t}{\Cof\tau}{\Cof e}\\
  \Cof{\aeunfold{e}} & = \aceunfold{\Cof{e}}\\
  \Cof{\aetpl{\labelset}{\mapschema{e}{i}{\labelset}}} & = \acetpl{\labelset}{\mapschemax{\Cofv}{e}{i}{\labelset}}\\
  \Cof{\aein{\labelset}{\ell}{\mapschema{\tau}{i}{\labelset}}{e}} &= \acein{\labelset}{\ell}{\mapschemax{\Cofv}{\tau}{i}{\labelset}}{\Cof{e}}\\
  \Cof{\aecase{\labelset}{\tau}{e}{\mapschemab{x}{e}{i}{\labelset}}} & = \acecase{\labelset}{\Cof\tau}{\Cof{e}}{\mapschemacx{\Cofv}{x}{e}{i}{\labelset}}
\end{align*}

\end{itemize}

There are two other candidate expansion forms, highlighted in gray in Figure \ref{fig:U-candidate-terms}: a ce-type form for \emph{references to spliced unexpanded types}, $\acesplicedt{m}{n}$, and a ce-expression form for \emph{references to spliced unexpanded expressions}, $\acesplicede{m}{n}$. %TSM utilize these to splice types and unexpanded expressions out of literal bodies.

\subsection{Candidate Expansion Validation}\label{sec:ce-validation-U}



The \emph{candidate expansion validation judgements} validate ce-types and ce-expressions and simultaneously generate their final expansions.% are types and expanded expressions, respectively.
\[\begin{array}{ll}
\textbf{Judgement Form} & \textbf{Description}\\
\cvalidT{\Delta}{\tscenev}{\ctau}{\tau} & \text{Candidate expansion type $\ctau$ is well-formed and has expansion $\tau$}\\
& \text{assuming $\Delta$ and type splicing scene $\tscenev$.}\\
\cvalidE{\Delta}{\Gamma}{\escenev}{\ce}{e}{\tau} & \text{Candidate expansion expression $\ce$ has expansion $e$ and type $\tau$}\\
& \text{assuming $\Delta$ and $\Gamma$ and expression splicing scene $\escenev$.}
\end{array}\]
\emph{Expression splicing scenes}, $\escenev$, are of the form $\esceneU{\uDelta}{\uGamma}{\uSigma}{b}$, and \emph{type splicing scenes}, $\tscenev$, are of the form $\tsceneU{\Delta}{b}$. We write $\tsfrom{\escenev}$ for the type splicing scene constructed by dropping the unexpanded typing context and ueTSM context from $\escenev$:
\[\tsfrom{\esceneU{\uDelta}{\uGamma}{\uSigma}{b}} = \tsceneU{\uDelta}{b}\]

The purpose of splicing scenes is to ``remember'', during the candidate expansion validation process, the unexpanded type formation context, $\uDelta$, unexpanded typing context, $\uGamma$, ueTSM context, $\uSigma$, and the literal body, $b$, from the ueTSM application site (cf. Rule (\ref{rule:expandsU-tsmap})), because these are necessary to validate references to spliced unexpanded types and expressions that appear within a candidate expansion.

\subsubsection{Candidate Expansion Type Validation}
The \emph{candidate expansion type validation judgement}, $\cvalidT{\Delta}{\tscenev}{\ctau}{\tau}$, is inductively defined by Rules (\ref*{rules:cvalidT-U}) as follows.

\paragraph{Shared Forms} Rules (\ref*{rule:cvalidT-U-tvar}) through (\ref*{rule:cvalidT-U-sum}), which validate ce-types of shared form, are defined below.
%Each of these rules is defined based on the corresponding type formation rule, i.e. Rules (\ref{rule:istypeU-var}) through (\ref{rule:istypeU-sum}), respectively. For example, the following candidate expansion type validation rules are based on type formation rules (\ref{rule:istypeU-var}), (\ref{rule:istypeU-parr}) and (\ref{rule:istypeU-all}), respectively: 
\begin{subequations}\label{rules:cvalidT-U}
\begin{equation}\label{rule:cvalidT-U-tvar}
\inferrule{ }{
  \cvalidT{\Delta, \Dhyp{t}}{\tscenev}{t}{t}
}
\end{equation}
\begin{equation}\label{rule:cvalidT-U-parr}
  \inferrule{
    \cvalidT{\Delta}{\tscenev}{\ctau_1}{\tau_1}\\
    \cvalidT{\Delta}{\tscenev}{\ctau_2}{\tau_2}
  }{
    \cvalidT{\Delta}{\tscenev}{\aceparr{\ctau_1}{\ctau_2}}{\aparr{\tau_1}{\tau_2}}
  }
\end{equation}
\begin{equation}\label{rule:cvalidT-U-all}
  \inferrule {
    \cvalidT{\Delta, \Dhyp{t}}{\tscenev}{\ctau}{\tau}
  }{
    \cvalidT{\Delta}{\tscenev}{\aceall{t}{\ctau}}{\aall{t}{\tau}}
  }
\end{equation}
\begin{equation}\label{rule:cvalidT-U-rec}
  \inferrule{
    \cvalidT{\Delta, \Dhyp{t}}{\tscenev}{\ctau}{\tau}
  }{
    \cvalidT{\Delta}{\tscenev}{\acerec{t}{\ctau}}{\arec{t}{\tau}}
  }
\end{equation}
\begin{equation}\label{rule:cvalidT-U-prod}
  \inferrule{
    \{\cvalidT{\Delta}{\tscenev}{\ctau_i}{\tau_i}\}_{i \in \labelset}
  }{
    \cvalidT{\Delta}{\tscenev}{\aceprod{\labelset}{\mapschema{\ctau}{i}{\labelset}}}{\aprod{\labelset}{\mapschema{\tau}{i}{\labelset}}}
  }
\end{equation}
\begin{equation}\label{rule:cvalidT-U-sum}
  \inferrule{
    \{\cvalidT{\Delta}{\tscenev}{\ctau_i}{\tau_i}\}_{i \in \labelset}
  }{
    \cvalidT{\Delta}{\tscenev}{\acesum{\labelset}{\mapschema{\ctau}{i}{\labelset}}}{\asum{\labelset}{\mapschema{\tau}{i}{\labelset}}}
  }
\end{equation}

Observe that, in each of these rules, the ce-type form and the type form in the conclusion correspond, and the premises correspond to those of the corresponding type formation rule, i.e. Rules (\ref{rules:istypeU}). The type splicing scene, $\tscenev$, passes opaquely through these rules. 
The following lemma establishes that each type can be expressed as a well-formed ce-type, under the same type formation context and any type splicing scene.
\begin{lemma}[Candidate Expansion Type Expressibility]\label{lemma:ce-type-expressibility-U} If $\istypeU{\Delta}{\tau}$ then $\cvalidT{\Delta}{\tscenev}{\Cof{\tau}}{\tau}$. \end{lemma}
\begin{proof}
By rule induction over Rules (\ref{rules:istypeU}). In each case, we apply the IH on or over each premise, then apply the corresponding ce-type validation rule in Rules (\ref{rules:cvalidT-U}).
\end{proof}
% We can express this scheme more precisely with the following rule transformation. For each rule in Rules (\ref{rules:istypeU}), 
% \begin{mathpar}
% % \refstepcounter{equation}
% % \label{rule:cvalidT-U-rec}
% % \refstepcounter{equation}
% % \label{rule:cvalidT-U-prod}
% % \refstepcounter{equation}
% % \label{rule:cvalidT-U-sum}
% % \inferrule{J_1\\\cdots\\J_k}{J}
% \end{mathpar}
% the corresponding candidate expansion type validation rule is
% \begin{mathpar}
% \inferrule{
%   \VTypof{J_1}\\
%   \cdots\\
%   \VTypof{J_k}
% }{
%   \VTypof{J}
% }
% \end{mathpar}
% where 
% \[\begin{split}
% \VTypof{\istypeU{\Delta}{\tau}} & = \cvalidT{\Delta}{\tscenev}{\VTypof{\tau}}{\tau}\\
% \VTypof{\{J_i\}_{i \in \labelset}} & = \{\VTypof{J_i}\}_{i \in \labelset}
% \end{split}\]
% and where $\VTypof{\tau}$, when $\tau$ is a metapattern of sort $\mathsf{Typ}$, is a metapattern of sort $\mathsf{CETyp}$ defined as follows:
% \begin{itemize}
% \item When $\tau$ is of definite form, $\VTypof{\tau}$ is defined as follows:
% \begin{align*}
% \VTypof{t} & = t\\
% \VTypof{\aparr{\tau_1}{\tau_2}} & = \aceparr{\VTypof{\tau_1}}{\VTypof{\tau_2}}\\
% \VTypof{\aall{t}{\tau}} & = \aceall{t}{\VTypof{\tau}}\\
% \VTypof{\arec{t}{\tau}} & = \acerec{t}{\VTypof{\tau}}\\
% \VTypof{\aprod{\labelset}{\mapschema{\tau}{i}{\labelset}}} & = \aceprod{\labelset}{\mapschemax{\VTypofv}{\tau}{i}{\labelset}}\\
% \VTypof{\asum{\labelset}{\mapschema{\tau}{i}{\labelset}}} & = \acesum{\labelset}{\mapschemax{\VTypofv}{\tau}{i}{\labelset}}
% \end{align*}
% \item When $\tau$ is of indefinite form, $\VTypof{\tau}$ is a uniquely corresponding metapattern also of indefinite form. For example, $\VTypof{\tau_1}=\ctau_1$ and $\VTypof{\tau_2}=\ctau_2$.
% \end{itemize}

% It is instructive to use this rule transformation to generate Rules (\ref{rule:cvalidT-U-tvar}) through (\ref{rule:cvalidT-U-all}) above. We omit the remaining rules, i.e. Rules (\ref*{rule:cvalidT-U-rec}) through (\ref*{rule:cvalidT-U-sum}). 

Notice that in Rule (\ref{rule:cvalidT-U-tvar}), only type variables tracked by the candidate expansion type formation context, $\Delta$, are validated. Type variables in the application site unexpanded type formation context, which appears within the type splicing scene, $\tscenev$, are not validated. Indeed, $\tscenev$ is not inspected by any of the rules above. This achieves \emph{context-independent expansion} as described in Sec. \ref{sec:splicing-and-hygiene} for type variables -- ueTSMs cannot impose ``hidden constraints'' on the application site unexpanded type formation context, because the type variables bound at the application site are simply not directly available to ce-types.

\paragraph{References to Spliced Types} The only ce-type form that does not correspond to a type form is $\acesplicedt{m}{n}$, which is a \emph{reference to a spliced unexpanded type}, i.e. it indicates that an unexpanded type should be parsed out from the literal body, $b$, which appears in the type splicing scene, beginning at position $m$ and ending at position $n$. 

Rule (\ref*{rule:cvalidT-U-splicedt}) governs this form:
\begin{equation}\label{rule:cvalidT-U-splicedt}
  \inferrule{
    \parseUTyp{\bsubseq{b}{m}{n}}{\utau}\\
    \expandsTU{\uDD{\uD}{\Delta_\text{app}}}{\utau}{\tau}\\
    \Delta \cap \Delta_\text{app} = \emptyset
  }{
    \cvalidT{\Delta}{\tsceneU{\uDD{\uD}{\Delta_\text{app}}}{b}}{\acesplicedt{m}{n}}{\tau}
  }
\end{equation}
The first premise of this rule extracts the indicated subsequence of $b$ using the partial metafunction $\bsubseq{b}{m}{n}$ and parses it using the partial metafunction $\mathsf{parseUTyp}(b)$, described in Sec. \ref{sec:syntax-U}, to produce the spliced unexpanded type itself, $\utau$.

The second premise of Rule (\ref{rule:cvalidT-U-splicedt}) performs type expansion of $\utau$ under the application site unexpanded type formation context, $\uDD{\uD}{\Delta_\text{app}}$, which appears in the type splicing scene. The hypotheses in the candidate expansion type formation context, $\Delta$, are not made available to $\tau$. %This enforces \emph{expansion independent splicing} as described in Sec. \ref{sec:splicing-and-hygiene} for type variables that appear in candidate expansion types. 

The third premise of Rule (\ref{rule:cvalidT-U-splicedt}) imposes the constraint that the candidate expansion's type formation context, $\Delta$, be disjoint from the application site type formation context, $\Delta_\text{app}$. This premise can always be discharged by $\alpha$-varying the candidate expansion that the reference to the spliced type appears within. 

This achieves \emph{expansion-independent splicing} as described in Sec. \ref{sec:splicing-and-hygiene} for type variables -- the TSM provider can choose type variable names freely within a candidate expansion, because the language prevents them from shadowing type variables at the application site (by $\alpha$-varying the candidate expansion as needed).%Such a change in bound variable names is possible again because variables bound by the ueTSM provider in a candidate expansion cannot ``leak into'' spliced terms because the hypotheses in $\Delta$ are not made available to the spliced type, $\tau$. 

Rules (\ref{rules:cvalidT-U}) validate the following lemma, which establishes that the final expansion of a valid ce-type is a well-formed type under the combined type formation context.
\begin{lemma}[Candidate Expansion Type Validation]\label{lemma:candidate-expansion-type-validation}
If $\cvalidT{\Delta}{\tsceneU{\uDD{\uD}{\Delta_\text{app}}}{b}}{\ctau}{\tau}$ then $\istypeU{\Dcons{\Delta}{\Delta_\text{app}}}{\tau}$.
\end{lemma}
\begin{proof} By rule induction over Rules (\ref{rules:cvalidT-U}).
\begin{byCases}
\item[\text{(\ref{rule:cvalidT-U-tvar})}] We have 
\begin{pfsteps*}
   \item $\Delta=\Delta', \Dhyp{t}$ \BY{assumption}
   \item $\ctau=t$ \BY{assumption}
   \item $\tau=t$ \BY{assumption}
   \item $\istypeU{\Delta', \Dhyp{t}}{t}$ \BY{Rule (\ref{rule:istypeU-var})} \pflabel{istype}
   \item $\istypeU{\Dcons{\Delta', \Dhyp{t}}{\Delta_\text{app}}}{t}$ \BY{Lemma \ref{lemma:weakening-U} over $\Delta_\text{app}$ to \pfref{istype}}
 \end{pfsteps*} 
\resetpfcounter

\item[\text{(\ref{rule:cvalidT-U-parr})}] We have
\begin{pfsteps*}
  \item $\ctau=\aceparr{\ctau_1}{\ctau_2}$ \BY{assumption}
  \item $\tau=\aparr{\tau_1}{\tau_2}$ \BY{assumption}
  \item $\cvalidT{\Delta}{\tsceneU{\uDD{\uD}{\Delta_\text{app}}}{b}}{\ctau_1}{\tau_1}$ \BY{assumption} \pflabel{cvalid-ctau1}
  \item $\cvalidT{\Delta}{\tsceneU{\uDD{\uD}{\Delta_\text{app}}}{b}}{\ctau_2}{\tau_2}$ \BY{assumption} \pflabel{cvalid-ctau2}
  \item $\istypeU{\Dcons{\Delta}{\Delta_\text{app}}}{\tau_1}$ \BY{IH on \pfref{cvalid-ctau1}} \pflabel{istype1}
  \item $\istypeU{\Dcons{\Delta}{\Delta_\text{app}}}{\tau_2}$ \BY{IH on \pfref{cvalid-ctau2}} \pflabel{istype2}
  \item $\istypeU{\Dcons{\Delta}{\Delta_\text{app}}}{\aparr{\tau_1}{\tau_2}}$ \BY{Rule (\ref{rule:istypeU-parr}) on \pfref{istype1} and \pfref{istype2}}
\end{pfsteps*}
\resetpfcounter

\item[\text{(\ref{rule:cvalidT-U-all})}] We have
\begin{pfsteps*}
  \item $\ctau=\aceall{t}{\ctau'}$ \BY{assumption}
  \item $\tau=\aall{t}{\tau'}$ \BY{assumption}
  \item $\cvalidT{\Delta, \Dhyp{t}}{\tsceneU{\uDD{\uD}{\Delta_\text{app}}}{b}}{\ctau'}{\tau'}$ \BY{assumption} \label{cvalidT}
  \item $\istypeU{\Dcons{\Delta, \Dhyp{t}}{\Delta_\text{app}}}{\tau'}$ \BY{IH on \pfref{cvalidT}} \pflabel{istypeU1}
  \item $\istypeU{\Dcons{\Delta}{\Delta_\text{app}}, \Dhyp{t}}{\tau'}$ \BY{exchange over $\Delta_\text{app}$ on \pfref{istypeU1}} \pflabel{istypeU2}
  \item $\istypeU{\Dcons{\Delta}{\Delta_\text{app}}}{\aall{t}{\tau'}}$ \BY{Rule (\ref{rule:istypeU-all}) on \pfref{istypeU2}}
\end{pfsteps*}
\resetpfcounter

\item[{\text{(\ref{rule:cvalidT-U-rec})}}~\text{through}~{\text{(\ref{rule:cvalidT-U-sum})}}] These cases follow analagously, i.e. we apply the IH to or over all ce-type validation premises, apply exchange as needed, and then apply the corresponding type formation rule.
\\
% \item[\text{(\ref{rule:cvalidT-U-rec})}] We have
% \begin{pfsteps*}
%   \item $\ctau=\acerec{t}{\ctau'}$ \BY{assumption}
%   \item $\tau=\arec{t}{\tau'}$ \BY{assumption}
%   \item $\cvalidT{\Delta, \Dhyp{t}}{\tsceneU{\Delta_\text{app}}{b}}{\ctau'}{\tau'}$ \BY{assumption} \label{cvalidT}
%   \item $\istypeU{\Dcons{\Delta, \Dhyp{t}}{\Delta_\text{app}}}{\tau'}$ \BY{IH on \pfref{cvalidT}} \pflabel{istypeU1}
%   \item $\istypeU{\Dcons{\Delta}{\Delta_\text{app}}, \Dhyp{t}}{\tau'}$ \BY{exchange over $\Delta_\text{app}$ on \pfref{istypeU1}} \pflabel{istypeU2}
%   \item $\istypeU{\Dcons{\Delta}{\Delta_\text{app}}}{\arec{t}{\tau'}}$ \BY{Rule (\ref{rule:istypeU-rec}) on \pfref{istypeU2}}
% \end{pfsteps*}
% \resetpfcounter

% \item[\text{(\ref{rule:cvalidT-U-prod})}] We have
% \begin{pfsteps*}
% \item $\ctau=\aceprod{\labelset}{\mapschema{\ctau}{i}{\labelset}}$ \BY{assumption}  
% \item $\tau=\aprod{\labelset}{\mapschema{\tau}{i}{\labelset}}$ \BY{assumption}
% \item $\{\cvalidT{\Delta}{\tsceneU{\Delta_\text{app}}{b}}{\ctau_i}{\tau_i}\}_{i \in \labelset}$ \BY{assumption} \pflabel{cvalidT-ass}
% \item $\{\istypeU{\Dcons{\Delta}{\Delta_\text{app}}}{\tau_i}\}_{i \in \labelset}$ \BY{IH on \pfref{cvalidT-ass}$_i$ for each $i \in \labelset$} \pflabel{istype}
% \item $\istypeU{\Dcons{\Delta}{\Delta_\text{app}}}{\aprod{\labelset}{\mapschema{\tau}{i}{\labelset}}}$ \BY{Rule (\ref{rule:istypeU-prod}) on \pfref{istype}}
% \end{pfsteps*}
% \resetpfcounter 

% \item[\text{(\ref{rule:cvalidT-U-sum})}] We have
% \begin{pfsteps*}
% \item $\ctau=\acesum{\labelset}{\mapschema{\ctau}{i}{\labelset}}$ \BY{assumption}  
% \item $\tau=\asum{\labelset}{\mapschema{\tau}{i}{\labelset}}$ \BY{assumption}
% \item $\{\cvalidT{\Delta}{\tsceneU{\Delta_\text{app}}{b}}{\ctau_i}{\tau_i}\}_{i \in \labelset}$ \BY{assumption} \pflabel{cvalidT-ass}
% \item $\{\istypeU{\Dcons{\Delta}{\Delta_\text{app}}}{\tau_i}\}_{i \in \labelset}$ \BY{IH on \pfref{cvalidT-ass}$_i$ for each $i \in \labelset$} \pflabel{istype}
% \item $\istypeU{\Dcons{\Delta}{\Delta_\text{app}}}{\asum{\labelset}{\mapschema{\tau}{i}{\labelset}}}$ \BY{Rule (\ref{rule:istypeU-sum}) on \pfref{istype}}
% \end{pfsteps*}
% \resetpfcounter

\item[\text{(\ref{rule:cvalidT-U-splicedt})}] We have
\begin{pfsteps*}
\item $\ctau=\acesplicedt{m}{n}$ \BY{assumption}
\item $\parseUTyp{\bsubseq{b}{m}{n}}{\utau}$ \BY{assumption}
\item $\expandsTU{\uDD{\uD}{\Delta_\text{app}}}{\utau}{\tau}$ \BY{assumption} \label{expandsTU}
\item $\istypeU{\Delta_\text{app}}{\tau}$ \BY{Lemma \ref{lemma:type-expansion-U} on \pfref{expandsTU}}\pflabel{istype}
\item $\istypeU{\Dcons{\Delta}{\Delta_\text{app}}}{\tau}$ \BY{Lemma \ref{lemma:weakening-U} over $\Delta$ on \pfref{istype} and exchange over $\Delta$}
\end{pfsteps*}
\resetpfcounter
\end{byCases}
\end{proof}
\end{subequations}
% \begin{subequations}\label{rules:cvalidT-U}
% \begin{equation}\label{rule:cvalidT-U-tvar}
% \inferrule{ }{
%   \cvalidT{\Delta, \Dhyp{t}}{\tscenev}{t}{t}
% }
% \end{equation}
% \begin{equation}\label{rule:cvalidT-U-parr}
%   \inferrule{
%     \cvalidT{\Delta}{\tscenev}{\ctau_1}{\tau_1}\\
%     \cvalidT{\Delta}{\tscenev}{\ctau_2}{\tau_2}
%   }{
%     \cvalidT{\Delta}{\tscenev}{\aceparr{\ctau_1}{\ctau_2}}{\aparr{\tau_1}{\tau_2}}
%   }
% \end{equation}
% \begin{equation}\label{rule:cvalidT-U-all}
%   \inferrule {
%     \cvalidT{\Delta, \Dhyp{t}}{\tscenev}{\ctau}{\tau}
%   }{
%     \cvalidT{\Delta}{\tscenev}{\aceall{t}{\ctau}}{\aall{t}{\tau}}
%   }
% \end{equation}
% \begin{equation}\label{rule:cvalidT-U-rec}
%   \inferrule{
%     \cvalidT{\Delta, \Dhyp{t}}{\tscenev}{\ctau}{\tau}
%   }{
%     \cvalidT{\Delta}{\tscenev}{\acerec{t}{\ctau}}{\arec{t}{\tau}}
%   }
% \end{equation}
% \begin{equation}\label{rule:cvalidT-U-prod}
%   \inferrule{
%     \{\cvalidT{\Delta}{\tscenev}{\ctau_i}{\tau_i}\}_{i \in \labelset}
%   }{
%     \cvalidT{\Delta}{\tscenev}{\aceprod{\labelset}{\mapschema{\ctau}{i}{\labelset}}}{\aprod{\labelset}{\mapschema{\tau}{i}{\labelset}}}
%   }
% \end{equation}
% \begin{equation}\label{rule:cvalidT-U-sum}
%   \inferrule{
%     \{\cvalidT{\Delta}{\tscenev}{\ctau_i}{\tau_i}\}_{i \in \labelset}
%   }{
%     \cvalidT{\Delta}{\tscenev}{\acesum{\labelset}{\mapschema{\ctau}{i}{\labelset}}}{\asum{\labelset}{\mapschema{\tau}{i}{\labelset}}}
%   }
% \end{equation}
% \end{subequations}
%Each form of type, $\tau$, corresponds to a form of candidate expansion type, $\ctau$ (compare Figures \ref{fig:U-expanded-terms} and \ref{fig:U-candidate-terms}). For each type formation rule in Rules (\ref{rules:istypeU}), there is a corresponding candidate expansion type validation rule -- Rules (\ref{rule:cvalidT-U-tvar}) to (\ref{rule:cvalidT-U-sum}) -- where the candidate expansion type and the final expansion correspond. The premises also correspond. 



\subsubsection{Candidate Expansion Expression Validation}
The \emph{candidate expansion expression validation judgement}, $\cvalidE{\Delta}{\Gamma}{\escenev}{\ce}{e}{\tau}$, is defined mutually inductively with the typed expansion judgement by Rules (\ref*{rules:cvalidE-U}) as follows.% This is necessary because a typed expansion judgement appears as a premise in Rule (\ref{rule:cvalidE-U-splicede}) below, and a candidate expansion expression validation judgement appears as a premise in Rule (\ref{rule:expandsU-tsmap}) above.

\paragraph{Shared Forms} Rules (\ref*{rule:cvalidE-U-var}) through (\ref*{rule:cvalidE-U-case}) validate ce-expressions of shared form. The first three of these rules are defined below:
%For each expanded expression form defined in Figure \ref{fig:U-expanded-terms}, Figure \ref{fig:U-candidate-terms} defines a corresponding candidate expansion expression form. The validation rules for candidate expansion expressions of these forms are each based on the corresponding typing rule in Rules (\ref{rules:hastypeU}). For example, the validation rules for candidate expansion expressions of variable, function and function application form  are based on Rules (\ref{rule:hastypeU-var}) through (\ref{rule:hastypeU-ap}), respectively:
\begin{subequations}\label{rules:cvalidE-U}
\begin{equation}\label{rule:cvalidE-U-var}
\inferrule{ }{
  \cvalidE{\Delta}{\Gamma, \Ghyp{x}{\tau}}{\escenev}{x}{x}{\tau}
}
\end{equation}
\begin{equation}\label{rule:cvalidE-U-lam}
\inferrule{
  \cvalidT{\Delta}{\tsfrom{\escenev}}{\ctau}{\tau}\\
  \cvalidE{\Delta}{\Gamma, \Ghyp{x}{\tau}}{\escenev}{\ce}{e}{\tau'}
}{
  \cvalidE{\Delta}{\Gamma}{\escenev}{\acelam{\ctau}{x}{\ce}}{\aelam{\tau}{x}{e}}{\aparr{\tau}{\tau'}}
}
\end{equation}
\begin{equation}\label{rule:cvalidE-U-ap}
  \inferrule{
    \cvalidE{\Delta}{\Gamma}{\escenev}{\ce_1}{e_1}{\aparr{\tau}{\tau'}}\\
    \cvalidE{\Delta}{\Gamma}{\escenev}{\ce_2}{e_2}{\tau}
  }{
    \cvalidE{\Delta}{\Gamma}{\escenev}{\aceap{\ce_1}{\ce_2}}{\aeap{e_1}{e_2}}{\tau'}
  }
\end{equation}
Observe that, in each of these rules, the ce-expression form and the expanded expression form in the conclusion correspond, and the premises correspond to those of the corresponding typing rule, i.e. Rules (\ref{rule:hastypeU-var}) through (\ref{rule:hastypeU-ap}), respectively. The expression splicing scene, $\escenev$, passes opaquely through these rules.

We can express this scheme more precisely with the following rule transformation. For each rule in Rules (\ref{rules:hastypeU}),
\begin{mathpar}\refstepcounter{equation}
\label{rule:cvalidE-U-tlam}
\refstepcounter{equation}
\label{rule:cvalidE-U-tap}
\refstepcounter{equation}
\label{rule:cvalidE-U-fold}
\refstepcounter{equation}
\label{rule:cvalidE-U-unfold}
\refstepcounter{equation}
\label{rule:cvalidE-U-tpl}
\refstepcounter{equation}
\label{rule:cvalidE-U-pr}
\refstepcounter{equation}
\label{rule:cvalidE-U-in}
\refstepcounter{equation}
\label{rule:cvalidE-U-case}
  \inferrule{
    J_1\\
    \cdots\\
    J_k
  }{
    J
  }
\end{mathpar}
the corresponding candidate expansion expression validation rule is 
\begin{mathpar}
  \inferrule{
    \Cof{J_1}\\
    \cdots\\
    \Cof{J_k}
  }{
    \Cof{J}
  }
\end{mathpar}
where 
\[\begin{split}
  \Cof{\istypeU{\Delta}{\tau}} & = \cvalidT{\Delta}{\tsfrom{\escenev}}{\Cof{\tau}}{\tau}\\
  \Cof{\hastypeU{\Delta}{\Gamma}{e}{\tau}} & = \cvalidE{\Delta}{\Gamma}{\escenev}{\Cof{e}}{e}{\tau}\\
  \Cof{\{J_i\}_{i \in \labelset}} & = \{\Cof{J_i}\}_{i \in \labelset}
\end{split}\]
and where:
\begin{itemize}
\item $\Cof{\tau}$ is defined as follows:
  \begin{itemize}
  \item When $\tau$ is of definite form, $\Cof{\tau}$ is defined as in Sec. \ref{sec:ce-syntax-U}.
  \item When $\tau$ is of indefinite form, $\Cof{\tau}$ is a uniquely corresponding metavariable of sort $\mathsf{CETyp}$ also of indefinite form. For example, $\Cof{\tau_1}=\ctau_1$ and $\Cof{\tau_2}=\ctau_2$.
  \end{itemize}
\item $\Cof{e}$ is defined as follows
  \begin{itemize}
  \item When $e$ is of definite form, $\Cof{e}$ is defined as in Sec. \ref{sec:ce-syntax-U}. 
  \item When $e$ is of indefinite form, $\Cof{e}$ is a uniquely corresponding metavariable of sort $\mathsf{CEExp}$ also of indefinite form. For example, $\Cof{e_1}=\ce_1$ and $\Cof{e_2}=\ce_2$.
  \end{itemize}
\end{itemize}

It is instructive to use this rule transformation to generate Rules (\ref{rule:cvalidE-U-var}) through (\ref{rule:cvalidE-U-ap}) above. We omit the remaining rules for shared forms, i.e. Rules (\ref*{rule:cvalidE-U-tlam}) through (\ref*{rule:cvalidE-U-case}).

The following lemma establishes that each well-typed expanded expression, $e$, can be expressed as a valid ce-expression, $\Cof{e}$, that is assigned the same type under any expression splicing scene.
\begin{theorem}[Candidate Expansion Expression Expressibility]\label{theorem:ce-expressions-expressibility-U} If $\hastypeU{\Delta}{\Gamma}{e}{\tau}$ then $\cvalidE{\Delta}{\Gamma}{\escenev}{\Cof{e}}{e}{\tau}$.\end{theorem}
\begin{proof} By rule induction over Rules (\ref{rules:hastypeU}). The rule transformation above guarantees that this lemma holds by construction. In particular, in each case, we apply Lemma \ref{lemma:ce-type-expressibility-U} to or over each type formation premise, the IH to or over each typing premise, then apply the corresponding ce-expression validation rule in Rules (\ref{rule:cvalidE-U-var}) through (\ref{rule:cvalidE-U-case}).
\end{proof}

Notice that in Rule (\ref{rule:cvalidE-U-var}), only variables tracked by the candidate expansion typing context, $\Gamma$, are validated. Variables  in the application site unexpanded typing context, which appears within the expression splicing scene $\escenev$, are not validated. Indeed, $\escenev$ is not inspected by any of the rules above. This achieves \emph{context-independent expansion} as described in Sec. \ref{sec:splicing-and-hygiene} -- ueTSMs cannot impose ``hidden constraints'' on the application site unexpanded typing context, because the variable bindings at the application site are not directly available to candidate expansions.

\paragraph{References to Spliced Unexpanded Expressions} The only ce-expression form that does not correspond to an expanded expression form is $\acesplicede{m}{n}$, which is a \emph{reference to a spliced unexpanded expression}, i.e. it indicates that an unexpanded expression should be parsed out from the literal body beginning at position $m$ and ending at position $n$. Rule (\ref*{rule:cvalidE-U-splicede}) governs this form:
\begin{equation}\label{rule:cvalidE-U-splicede}
\inferrule{
  \parseUExp{\bsubseq{b}{m}{n}}{\ue}\\
  \expandsU{\uDD{\uD}{\Delta_\text{app}}}{\uGG{\uG}{\Gamma_\text{app}}}{\uSigma}{\ue}{e}{\tau}\\\\
  \Delta \cap \Delta_\text{app} = \emptyset\\
  \domof{\Gamma} \cap \domof{\Gamma_\text{app}} = \emptyset
}{
  \cvalidE{\Delta}{\Gamma}{\esceneU{\uDD{\uD}{\Delta_\text{app}}}{\uGG{\uG}{\Gamma_\text{app}}}{\uSigma}{b}}{\acesplicede{m}{n}}{e}{\tau}
}
\end{equation}
% \begin{equation}\label{rule:cvalidE-U-splicede}
% \inferrule{
%   \parseUExp{\bsubseq{b}{m}{n}}{\ue}\\\\
%   \expandsU{\Delta_\text{app}}{\Gamma_\text{app}}{\uSigma}{\ue}{e}{\tau}\\
%   \Delta \cap \Delta_\text{app} = \emptyset\\
%   \domof{\Gamma} \cap \domof{\Gamma_\text{app}} = \emptyset
% }{
%   \cvalidE{\Delta}{\Gamma}{\esceneU{\Delta_\text{app}}{\Gamma_\text{app}}{\uSigma}{b}}{\splicede{m}{n}}{e}{\tau}
% }
% \end{equation}
The first premise of this rule extracts the indicated subsequence of $b$ using the partial metafunction $\bsubseq{b}{m}{n}$ and parses it using the partial metafunction $\mathsf{parseUExp}(b)$, described in Sec. \ref{sec:syntax-U}, to produce the referenced spliced unexpanded expression, $\ue$.

The second premise of Rule (\ref{rule:cvalidE-U-splicede}) types and expands the spliced unexpanded expression $\ue$ assuming the application site contexts that appear in the expression splicing scene. The hypotheses in the candidate expansion type formation context, $\Delta$, and typing context, $\Gamma$, are not made available to $\ue$. 

The third premise of Rule (\ref{rule:cvalidE-U-splicede}) imposes the constraint that the candidate expansion's type formation context, $\Delta$, be disjoint from the application site type formation context, $\Delta_\text{app}$. Similarly, the fourth premise requires that the candidate expansion's typing context, $\Gamma$, be disjoint from the application site typing context, $\Gamma_\text{app}$. These two premises can always be discharged by $\alpha$-varying the ce-expression that the reference to the spliced unexpanded expression appears within. 

This achieves \emph{expansion-independent splicing} as described in Sec. \ref{sec:splicing-and-hygiene} -- the TSM provider can choose variable names freely within a candidate expansion, because the language prevents them from shadowing those at the application site (by $\alpha$-varying the candidate expansion as needed).
\end{subequations}
% \begin{subequations}\label{rules:cvalidE-U}
% \begin{equation}\label{rule:cvalidE-U-var}
% \inferrule{ }{
%   \cvalidE{\Delta}{\Gamma, \Ghyp{x}{\tau}}{\escenev}{x}{x}{\tau}
% }
% \end{equation}
% \begin{equation}\label{rule:cvalidE-U-lam}
% \inferrule{
%   \cvalidT{\Delta}{\tsfrom{\escenev}}{\ctau}{\tau}\\
%   \cvalidE{\Delta}{\Gamma, \Ghyp{x}{\tau}}{\escenev}{\ce}{e}{\tau'}
% }{
%   \cvalidE{\Delta}{\Gamma}{\escenev}{\acelam{\ctau}{x}{\ce}}{\aelam{\tau}{x}{e}}{\aparr{\tau}{\tau'}}
% }
% \end{equation}
% \begin{equation}\label{rule:cvalidE-U-ap}
%   \inferrule{
%     \cvalidE{\Delta}{\Gamma}{\escenev}{\ce_1}{e_1}{\aparr{\tau}{\tau'}}\\
%     \cvalidE{\Delta}{\Gamma}{\escenev}{\ce_2}{e_2}{\tau}
%   }{
%     \cvalidE{\Delta}{\Gamma}{\escenev}{\aceap{\ce_1}{\ce_2}}{\aeap{e_1}{e_2}}{\tau'}
%   }
% \end{equation}
% \begin{equation}\label{rule:cvalidE-U-tlam}
%   \inferrule{
%     \cvalidE{\Delta, \Dhyp{t}}{\Gamma}{\escenev}{\ce}{e}{\tau}
%   }{
%     \cvalidEX{\acetlam{t}{\ce}}{\aetlam{t}{e}}{\aall{t}{\tau}}
%   }
% \end{equation}
% \begin{equation}\label{rule:cvalidE-U-tap}
%   \inferrule{
%     \cvalidEX{\ce}{e}{\aall{t}{\tau}}\\
%     \cvalidT{\Delta}{\tsfrom{\escenev}}{\ctau'}{\tau'}
%   }{
%     \cvalidEX{\acetap{\ce}{\ctau'}}{\aetap{e}{\tau'}}{[\tau'/t]\tau}
%   }
% \end{equation}
% \begin{equation}\label{rule:cvalidE-U-fold}
%   \inferrule{
%     \cvalidT{\Delta, \Dhyp{t}}{\escenev}{\ctau}{\tau}\\
%     \cvalidEX{\ce}{e}{[\arec{t}{\tau}/t]\tau}
%   }{
%     \cvalidEX{\acefold{t}{\ctau}{\ce}}{\aefold{t}{\tau}{e}}{\arec{t}{\tau}}
%   }
% \end{equation}
% \begin{equation}\label{rule:cvalidE-U-unfold}
%   \inferrule{
%     \cvalidEX{\ce}{e}{\arec{t}{\tau}}
%   }{
%     \cvalidEX{\aceunfold{\ce}}{\aeunfold{e}}{[\arec{t}{\tau}/t]\tau}
%   }
% \end{equation}
% \begin{equation}\label{rule:cvalidE-U-tpl}
%   \inferrule{
%     \{\cvalidEX{\ce_i}{e_i}{\tau_i}\}_{i \in \labelset}
%   }{
%     \cvalidEX{\acetpl{\labelset}{\mapschema{\ce}{i}{\labelset}}}{\aetpl{\labelset}{\mapschema{e}{i}{\labelset}}}{\aprod{\labelset}{\mapschema{\tau}{i}{\labelset}}}
%   }
% \end{equation}
% \begin{equation}\label{rule:cvalidE-U-pr}
%   \inferrule{
%     \cvalidEX{\ce}{e}{\aprod{\labelset, \ell}{\mapschema{\tau}{i}{\labelset}; \mapitem{\ell}{\tau}}}
%   }{
%     \cvalidEX{\acepr{\ell}{\ce}}{\aepr{\ell}{e}}{\tau}
%   }
% \end{equation}
% \begin{equation}\label{rule:cvalidE-U-in}
%   \inferrule{
%     \{\cvalidT{\Delta}{\tsfrom{\escenev}}{\ctau_i}{\tau_i}\}_{i \in \labelset}\\
%     \cvalidT{\Delta}{\tsfrom{\escenev}}{\ctau}{\tau}\\
%     \cvalidEX{\ce}{e}{\tau}
%   }{
%     \left\{\shortstack{$\Delta~\Gamma \vdash_\uSigma \acein{\labelset, \ell}{\ell}{\mapschema{\ctau}{i}{\labelset}; \mapitem{\ell}{\ctau}}{\ce}$\\$\leadsto$\\$\aein{\labelset, \ell}{\ell}{\mapschema{\tau}{i}{\labelset}; \mapitem{\ell}{\tau}}{e} : \asum{\labelset, \ell}{\mapschema{\tau}{i}{\labelset}; \mapitem{\ell}{\tau}}$\vspace{-1.2em}}\right\}
%   }
% \end{equation}
% \begin{equation}\label{rule:cvalidE-U-case}
%   \inferrule{
%     \cvalidEX{\ce}{e}{\asum{\labelset}{\mapschema{\tau}{i}{\labelset}}}\\
%     \{\cvalidE{\Delta}{\Gamma, \Ghyp{x_i}{\tau_i}}{\escenev}{\ue_i}{e_i}{\tau}\}_{i \in \labelset}
%   }{
%     \cvalidEX{\acecase{\labelset}{\ce}{\mapschemab{x}{\ce}{i}{\labelset}}}{\aecase{\labelset}{e}{\mapschemab{x}{e}{i}{\labelset}}}{\tau}
%   }
% \end{equation}
% \begin{equation}\label{rule:cvalidE-U-splicede}
% \inferrule{
%   \parseUExp{\bsubseq{b}{m}{n}}{\ue}\\\\
%   \Delta \cap \Delta_\text{app} = \emptyset\\
%   \domof{\Gamma} \cap \domof{\Gamma_\text{app}} = \emptyset\\
%   \expandsU{\Delta_\text{app}}{\Gamma_\text{app}}{\uSigma}{\ue}{e}{\tau}
% }{
%   \cvalidE{\Delta}{\Gamma}{\esceneU{\Delta_\text{app}}{\Gamma_\text{app}}{\uSigma}{b}}{\acesplicede{m}{n}}{e}{\tau}
% }
% \end{equation}
% \end{subequations}

% Each form of expanded expression, $e$, corresponds to a form of candidate expansion expression, $\ce$ (compare Figure \ref{fig:U-expanded-terms} and Figure \ref{fig:U-candidate-terms}). For each typing rule in Rules \ref{rules:hastypeU}, there is a corresponding candidate expansion expression validation rule -- Rules (\ref{rule:cvalidE-U-var}) to (\ref{rule:cvalidE-U-case}) -- where the candidate expansion expression and expanded expression correspond. The premises also correspond.


%Candidate expansions cannot themselves define or apply TSMs. This simplifies our metatheory, though it can be inconvenient at times for TSM providers. We discuss adding the ability to use TSMs within candidate expansions in Sec. \ref{sec:tsms-in-expansions}.


\subsection{Metatheory}
For the judgements we have defined to form a sensible language, we must have that typed expansion and candidate expansion expression validation be consistent with typing. Formally, this can be expressed as follows.

\begin{theorem}[Typed Expansion]\label{thm:typed-expansion-U} Both of the following hold:
\begin{enumerate}
\item If $\expandsU{\uDD{\uD}{\Delta}}{\uGG{\uG}{\Gamma}}{\uAS{\uA}{\Sigma}}{\ue}{e}{\tau}$ then $\hastypeU{\Delta}{\Gamma}{e}{\tau}$.
\item If $\cvalidE{\Delta}{\Gamma}{\esceneU{\uDD{\uD}{\Delta_\text{app}}}{\uGG{\uG}{\Gamma_\text{app}}}{\uAS{\uA}{\Sigma}}{b}}{\ce}{e}{\tau}$ and $\Delta \cap \Delta_\text{app} = \emptyset$ and $\domof{\Gamma} \cap \domof{\Gamma_\text{app}} = \emptyset$ then $\hastypeU{\Dcons{\Delta}{\Delta_\text{app}}}{\Gcons{\Gamma}{\Gamma_\text{app}}}{e}{\tau}$.
\end{enumerate}
\end{theorem}
\begin{proof}
By mutual rule induction over Rules (\ref{rules:expandsU}) and Rules (\ref{rules:cvalidE-U}). 

The proof of part 1 proceeds by inducting over the typed expansion assumption. In the following cases, let $\uDelta=\uDD{\uD}{\Delta}$ and $\uGamma=\uGG{\uG}{\Gamma}$ and $\uSigma=\uAS{\uA}{\Sigma}$.
\begin{byCases}
\item[\text{(\ref{rule:expandsU-var})}] We have:
\begin{pfsteps}
  \item \ue=\ux \BY{assumption}
  \item e=x \BY{assumption}
  \item \Gamma=\Gamma', \Ghyp{x}{\tau} \BY{assumption}
  \item \hastypeU{\Delta}{\Gamma', \Ghyp{x}{\tau}}{x}{\tau} \BY{Rule (\ref{rule:hastypeU-var})}
\end{pfsteps}
\resetpfcounter

\item[\text{(\ref{rule:expandsU-lam})}] We have:
\begin{pfsteps}
  \item \ue=\aulam{\utau_1}{\ux}{\ue'} \BY{assumption}
  \item e=\aelam{\tau_1}{x}{e'} \BY{assumption}
  \item \tau=\aparr{\tau_1}{\tau_2} \BY{assumption}
  \item \expandsTU{\uDelta}{\utau_1}{\tau_1} \BY{assumption} \pflabel{istype}
  \item \expandsU{\uDelta}{\uGamma, \uGhyp{\ux}{x}{\tau_1}}{\uSigma}{\ue'}{e'}{\tau_2} \BY{assumption} \pflabel{expandsU}
%  \item \uetsmenv{\Delta}{\Sigma} \BY{assumption} \pflabel{uetsmenv}
  \item \istypeU{\Delta}{\tau_1} \BY{Lemma \ref{lemma:type-expansion-U} on \pfref{istype}} \pflabel{istype2}
  \item \hastypeU{\Delta}{\Gamma, \Ghyp{x}{\tau_1}}{e'}{\tau_2} \BY{IH, part 1 on \pfref{expandsU}} \pflabel{hastypeU}
  \item \hastypeU{\Delta}{\Gamma}{\aelam{\tau_1}{x}{e'}}{\aparr{\tau_1}{\tau_2}} \BY{Rule (\ref{rule:hastypeU-lam}) on \pfref{istype2} and \pfref{hastypeU}}
\end{pfsteps}
\resetpfcounter

\item[\text{(\ref{rule:expandsU-ap})}] We have:
\begin{pfsteps}
  \item \ue=\auap{\ue_1}{\ue_2} \BY{assumption}
  \item e=\aeap{e_1}{e_2} \BY{assumption}
  \item \expandsU{\uDelta}{\uGamma}{\uSigma}{\ue_1}{e_1}{\aparr{\tau_1}{\tau}} \BY{assumption}\pflabel{expandsU1}
  \item \expandsU{\uDelta}{\uGamma}{\uSigma}{\ue_2}{e_2}{\tau_1} \BY{assumption}\pflabel{expandsU2}
%  \item \uetsmenv{\Delta}{\Sigma} \BY{assumption} \pflabel{uetsmenv}
  \item \hastypeU{\Delta}{\Gamma}{e_1}{\aparr{\tau_1}{\tau}} \BY{IH, part 1 on \pfref{expandsU1}}\pflabel{hastypeU1}
  \item \hastypeU{\Delta}{\Gamma}{e_2}{\tau_1} \BY{IH on \pfref{expandsU2}}\pflabel{hastypeU2}
  \item \hastypeU{\Delta}{\Gamma}{\aeap{e_1}{e_2}}{\tau} \BY{Rule (\ref{rule:hastypeU-ap}) on \pfref{hastypeU1} and \pfref{hastypeU2}}
\end{pfsteps}
\resetpfcounter

\item[\text{(\ref{rule:expandsU-tlam})}~\text{through}~\text{(\ref{rule:expandsU-case})}] These cases follow analagously, i.e. we apply Lemma \ref{lemma:type-expansion-U} to or over the type expansion premises and the IH, part 1, to or over the typed expression expansion premises and then apply the corresponding typing rule in Rules (\ref{rule:hastypeU-tlam}) through (\ref{rule:hastypeU-case}).
\\

\item[\text{(\ref{rule:expandsU-syntax})}] We have 
\begin{pfsteps}
  \item \ue=\audefuetsm{\utau'}{\eparse}{\tsmv}{\ue'} \BY{assumption}
  \item \expandsTU{\uDelta}{\utau'}{\tau'} \BY{assumption} \pflabel{expandsTU}
%  \item \hastypeU{\emptyset}{\emptyset}{\eparse}{\aparr{\tBody}{\tParseResultExp}} \BY{assumption}\pflabel{eparse}
  \item \expandsU{\uDelta}{\uGamma}{\uSigma, \uShyp{\tsmv}{a}{\tau'}{\eparse}}{\ue'}{e}{\tau} \BY{assumption}\pflabel{expandsU}
%  \item \uetsmenv{\Delta}{\Sigma} \BY{assumption}\pflabel{uetsmenv1}
%  \item \istypeU{\Delta}{\tau'} \BY{Lemma \ref{lemma:type-expansion-U} to \pfref{expandsTU}} \pflabel{istype}
%  \item \uetsmenv{\Delta}{\Sigma, \xuetsmbnd{\tsmv}{\tau'}{\eparse}} \BY{Definition \ref{def:ueTSM-def-ctx-formation} on \pfref{uetsmenv1}, \pfref{istype} and \pfref{eparse}}\pflabel{uetsmenv3}
  \item \hastypeU{\Delta}{\Gamma}{e}{\tau} \BY{IH, part 1 on \pfref{expandsU}}
\end{pfsteps}
\resetpfcounter 

\item[\text{(\ref{rule:expandsU-tsmap})}] We have 
\begin{pfsteps}
  \item \ue=\autsmap{b}{\tsmv} \BY{assumption}
  \item \uA = \uA', \vExpands{\tsmv}{a} \BY{assumption}
  \item \Sigma=\Sigma', \xuetsmbnd{a}{\tau}{\eparse} \BY{assumption}
  \item \encodeBody{b}{\ebody} \BY{assumption}
  \item \evalU{\eparse(\ebody)}{\inj{\lbltxt{Success}}{\ecand}} \BY{assumption}
  \item \decodeCondE{\ecand}{\ce} \BY{assumption}
  \item \cvalidE{\emptyset}{\emptyset}{\esceneU{\uDelta}{\uGamma}{\uSigma}{b}}{\ce}{e}{\tau} \BY{assumption}\pflabel{cvalidE}
%  \item \uetsmenv{\Delta}{\Sigma} \BY{assumption} \pflabel{uetsmenv}
  \item \emptyset \cap \Delta = \emptyset \BY{finite set intersection identity} \pflabel{delta-cap}
  \item {\emptyset} \cap \domof{\Gamma} = \emptyset \BY{finite set intersection identity} \pflabel{gamma-cap}
  \item \hastypeU{\emptyset \cup \Delta}{\emptyset \cup \Gamma}{e}{\tau} \BY{IH, part 2 on \pfref{cvalidE}, \pfref{delta-cap}, and \pfref{gamma-cap}} \pflabel{penultimate}
  \item \hastypeU{\Delta}{\Gamma}{e}{\tau} \BY{definition of finite set union over \pfref{penultimate}}
\end{pfsteps}
\resetpfcounter
\end{byCases}

The second part of the theorem proceeds by induction over the candidate expansion expression validation assumption as follows. In the following cases, let $\uDelta_\text{app}=\uDD{\uD}{\Delta_\text{app}}$ and $\uGamma_\text{app}=\uGG{\uG}{\Gamma_\text{app}}$ and $\uSigma = \uAS{\uA}{\Sigma}$.
\begin{byCases}
\item[\text{(\ref{rule:cvalidE-U-var})}] We have
\begin{pfsteps*}
  \item $\ce=x$ \BY{assumption}
  \item $e=x$ \BY{assumption}
  \item $\Gamma=\Gamma', \Ghyp{x}{\tau}$ \BY{assumption}
  \item $\hastypeU{\Dcons{\Delta}{\Delta_\text{app}}}{\Gamma', \Ghyp{x}{\tau}}{x}{\tau}$ \BY{Rule (\ref{rule:hastypeU-var})} \pflabel{hastypeU}
  \item $\hastypeU{\Dcons{\Delta}{\Delta_\text{app}}}{\Gcons{\Gamma', \Ghyp{x}{\tau}}{\Gamma_\text{app}}}{x}{\tau}$ \BY{Lemma \ref{lemma:weakening-U} over $\Gamma_\text{app}$ to \pfref{hastypeU}}
\end{pfsteps*}
\resetpfcounter

\item[\text{(\ref{rule:cvalidE-U-lam})}] We have
\begin{pfsteps*}
  \item $\ce=\acelam{\ctau_1}{x}{\ce'}$ \BY{assumption}
  \item $e=\aelam{\tau_1}{x}{e'}$ \BY{assumption}
  \item $\tau=\aparr{\tau_1}{\tau_2}$ \BY{assumption}
  \item $\cvalidT{\Delta}{\tsceneU{\uDelta_\text{app}}{b}}{\ctau_1}{\tau_1}$ \BY{assumption} \pflabel{cvalidT}
  \item $\cvalidE{\Delta}{\Gamma, \Ghyp{x}{\tau_1}}{\esceneU{\uDelta_\text{app}}{\uGamma_\text{app}}{\uSigma}{b}}{\ce'}{e'}{\tau_2}$ \BY{assumption} \pflabel{cvalidE}
%  \item $\uetsmenv{\Delta_\text{app}}{\Sigma}$ \BY{assumption} \pflabel{uetsmenv}
  \item $\Delta \cap \Delta_\text{app}=\emptyset$ \BY{assumption} \pflabel{delta-disjoint}
  \item $\domof{\Gamma} \cap \domof{\Gamma_\text{app}}=\emptyset$ \BY{assumption} \pflabel{gamma-disjoint}
  \item $x \notin \domof{\Gamma_\text{app}}$ \BY{identification convention} \pflabel{x-fresh}
  \item $\domof{\Gamma, x : \tau_1} \cap \domof{\Gamma_\text{app}}=\emptyset$ \BY{\pfref{gamma-disjoint} and \pfref{x-fresh}} \pflabel{gamma-disjoint2}
  \item $\istypeU{\Dcons{\Delta}{\Delta_\text{app}}}{\tau_1}$ \BY{Lemma \ref{lemma:candidate-expansion-type-validation} on \pfref{cvalidT}} \pflabel{istype}
  \item $\hastypeU{\Dcons{\Delta}{\Delta_\text{app}}}{\Gcons{\Gamma, \Ghyp{x}{\tau_1}}{\Gamma_\text{app}}}{e'}{\tau_2}$ \BY{IH, part 2 on \pfref{cvalidE}, \pfref{delta-disjoint} and \pfref{gamma-disjoint2}} \pflabel{hastype1}
  \item $\hastypeU{\Dcons{\Delta}{\Delta_\text{app}}}{\Gcons{\Gamma}{\Gamma_\text{app}}, \Ghyp{x}{\tau_1}}{e'}{\tau_2}$ \BY{exchange over $\Gamma_\text{app}$ on \pfref{hastype1}} \pflabel{hastype2}
  \item $\hastypeU{\Dcons{\Delta}{\Delta_\text{app}}}{\Gcons{\Gamma}{\Gamma_\text{app}}}{\aelam{\tau_1}{x}{e'}}{\aparr{\tau_1}{\tau_2}}$ \BY{Rule (\ref{rule:hastypeU-lam}) on \pfref{istype} and \pfref{hastype2}}
\end{pfsteps*}
\resetpfcounter

\item[\text{(\ref{rule:cvalidE-U-ap})}] We have
\begin{pfsteps*}
  \item $\ce=\aceap{\ce_1}{\ce_2}$ \BY{assumption}
  \item $e=\aeap{e_1}{e_2}$ \BY{assumption}
  \item $\cvalidE{\Delta}{\Gamma}{\esceneU{\uDelta_\text{app}}{\uGamma_\text{app}}{\uSigma}{b}}{\ce_1}{e_1}{\aparr{\tau_2}{\tau}}$ \BY{assumption} \pflabel{cvalidE1}
  \item $\cvalidE{\Delta}{\Gamma}{\esceneU{\uDelta_\text{app}}{\uGamma_\text{app}}{\uSigma}{b}}{\ce_2}{e_2}{\tau_2}$ \BY{assumption} \pflabel{cvalidE2}
%  \item $\uetsmenv{\Delta_\text{app}}{\Sigma}$ \BY{assumption} \pflabel{uetsmenv}
  \item $\Delta \cap \Delta_\text{app}=\emptyset$ \BY{assumption} \pflabel{delta-disjoint}
  \item $\domof{\Gamma} \cap \domof{\Gamma_\text{app}}=\emptyset$ \BY{assumption} \pflabel{gamma-disjoint}
  \item $\hastypeU{\Dcons{\Delta}{\Delta_\text{app}}}{\Gcons{\Gamma}{\Gamma_\text{app}}}{e_1}{\aparr{\tau_2}{\tau}}$ \BY{IH, part 2 on \pfref{cvalidE1}, \pfref{delta-disjoint} and \pfref{gamma-disjoint}} \pflabel{hastypeU1}
  \item $\hastypeU{\Dcons{\Delta}{\Delta_\text{app}}}{\Gcons{\Gamma}{\Gamma_\text{app}}}{e_2}{\tau_2}$ \BY{IH, part 2 on \pfref{cvalidE2}, \pfref{delta-disjoint} and \pfref{gamma-disjoint}} \pflabel{hastypeU2}
  \item $\hastypeU{\Dcons{\Delta}{\Delta_\text{app}}}{\Gcons{\Gamma}{\Gamma_\text{app}}}{\aeap{e_1}{e_2}}{\tau}$ \BY{Rule (\ref{rule:hastypeU-ap}) on \pfref{hastypeU1} and \pfref{hastypeU2}}
\end{pfsteps*}
\resetpfcounter

\item[\text{(\ref{rule:cvalidE-U-tlam})}] We have
\begin{pfsteps}
  \item \ce=\acetlam{t}{\ce'} \BY{assumption}
  \item e = \aetlam{t}{e'} \BY{assumption}
  \item \tau = \aall{t}{\tau'}\BY{assumption}
  \item \cvalidE{\Delta, \Dhyp{t}}{\Gamma}{\esceneU{\uDelta_\text{app}}{\uGamma_\text{app}}{\uSigma}{b}}{\ce'}{e'}{\tau'} \BY{assumption} \pflabel{cvalidE}
%  \item \uetsmenv{\Delta_\text{app}}{\Sigma} \BY{assumption} \pflabel{uetsmenv}
  \item \Delta \cap \Delta_\text{app}=\emptyset \BY{assumption} \pflabel{delta-disjoint}
  \item \domof{\Gamma} \cap \domof{\Gamma_\text{app}}=\emptyset \BY{assumption} \pflabel{gamma-disjoint}
  \item \Dhyp{t} \notin \Delta_\text{app} \BY{identification convention}\pflabel{t-fresh}
  \item \Delta, \Dhyp{t} \cap \Delta_\text{app} = \emptyset \BY{\pfref{delta-disjoint} and \pfref{t-fresh}}\pflabel{delta-disjoint2}
  \item \hastypeU{\Dcons{\Delta, \Dhyp{t}}{\Delta_\text{app}}}{\Gcons{\Gamma}{\Gamma_\text{app}}}{e'}{\tau'} \BY{IH, part 2 on \pfref{cvalidE}, \pfref{delta-disjoint2} and \pfref{gamma-disjoint}}\pflabel{hastype1}
  \item \hastypeU{\Dcons{\Delta}{\Delta_\text{app}, \Dhyp{t}}}{\Gcons{\Gamma}{\Gamma_\text{app}}}{e'}{\tau'} \BY{exchange over $\Delta_\text{app}$ on \pfref{hastype1}}\pflabel{hastype2}
  \item \hastypeU{\Dcons{\Delta}{\Delta_\text{app}}}{\Gcons{\Gamma}{\Gamma_\text{app}}}{\aetlam{t}{e'}}{\aall{t}{\tau'}} \BY{Rule (\ref{rule:hastypeU-tlam}) on \pfref{hastype2}}
\end{pfsteps}
\resetpfcounter

\item[{\text{(\ref{rule:cvalidE-U-tap})}}~\text{through}~{\text{(\ref{rule:cvalidE-U-case})}}] These cases follow analagously, i.e. we apply the IH, part 2 to all ce-expression validation judgements, Lemma \ref{lemma:candidate-expansion-type-validation} to all ce-type validation judgements, the identification convention to ensure that extended contexts remain disjoint, weakening and exchange as needed, and the corresponding typing rule in Rules (\ref{rule:hastypeU-tap}) through (\ref{rule:hastypeU-case}).
\\

\item[\text{(\ref{rule:cvalidE-U-splicede})}] We have
\begin{pfsteps*}
  \item $\ce=\acesplicede{m}{n}$ \BY{assumption}
  \item $\parseUExp{\bsubseq{b}{m}{n}}{\ue}$ \BY{assumption}
  \item $\expandsU{\uDelta_\text{app}}{\uGamma_\text{app}}{\uSigma}{\ue}{e}{\tau}$ \BY{assumption} \pflabel{expands}
%  \item $\uetsmenv{\Delta_\text{app}}{\Sigma}$ \BY{assumption} \pflabel{uetsmenv}
  \item $\Delta \cap \Delta_\text{app}=\emptyset$ \BY{assumption} \pflabel{delta-disjoint}
  \item $\domof{\Gamma} \cap \domof{\Gamma_\text{app}}=\emptyset$ \BY{assumption} \pflabel{gamma-disjoint}
  \item $\hastypeU{\Delta_\text{app}}{\Gamma_\text{app}}{e}{\tau}$ \BY{IH, part 1 on \pfref{expands}} \pflabel{hastype}
  \item $\hastypeU{\Dcons{\Delta}{\Delta_\text{app}}}{\Gcons{\Gamma}{\Gamma_\text{app}}}{e}{\tau}$ \BY{Lemma \ref{lemma:weakening-U} over $\Delta$ and $\Gamma$ and exchange on \pfref{hastype}}
\end{pfsteps*}
\resetpfcounter
\end{byCases}

The mutual induction can be shown to be well-founded by showing that the following numeric metric on the judgements that we induct over is decreasing:
\begin{align*}
\sizeof{\expandsU{\uDelta}{\uGamma}{\uSigma}{\ue}{e}{\tau}} & = \sizeof{\ue}\\
\sizeof{\cvalidE{\Delta}{\Gamma}{\esceneU{\uDelta_\text{app}}{\uGamma_\text{app}}{\uSigma}{b}}{\ce}{e}{\tau}} & = \sizeof{b}
\end{align*}
where $\sizeof{b}$ is the length of $b$ and $\sizeof{\ue}$ is the sum of the lengths of the literal bodies in $\ue$,
\begin{align*}
\sizeof{\ux} & = 0\\
\sizeof{\aulam{\utau}{\ux}{\ue}} &= \sizeof{\ue}\\
\sizeof{\auap{\ue_1}{\ue_2}} & = \sizeof{\ue_1} + \sizeof{\ue_2}\\
\sizeof{\autlam{\ut}{\ue}} & = \sizeof{\ue}\\
\sizeof{\autap{\ue}{\utau}} & = \sizeof{\ue}\\
\sizeof{\aufold{\ut}{\utau}{\ue}} & = \sizeof{\ue}\\
\sizeof{\auunfold{\ue}} & = \sizeof{\ue}\\
%\end{align*}
%\begin{align*}
\sizeof{\autpl{\labelset}{\mapschema{\ue}{i}{\labelset}}} & = \sum_{i \in \labelset} \sizeof{\ue_i}\\
\sizeof{\aupr{\ell}{\ue}} & = \sizeof{\ue}\\
\sizeof{\auin{\labelset}{\ell}{\mapschema{\utau}{i}{\labelset}}{\ue}} & = \sizeof{\ue}\\
\sizeof{\aucase{\labelset}{\utau}{\ue}{\mapschemab{\ux}{\ue}{i}{\labelset}}} & = \sizeof{\ue} + \sum_{i \in \labelset} \sizeof{\ue_i}\\
\sizeof{\audefuetsm{\utau}{\eparse}{\tsmv}{\ue}} & = \sizeof{\ue}\\
\sizeof{\autsmap{b}{\tsmv}} & = \sizeof{b}
\end{align*}

The only case in the proof of part 1 that invokes part 2 is Case (\ref{rule:expandsU-tsmap}). There, we have that the metric remains stable: \begin{align*}
 & \sizeof{\expandsU{\uDelta}{\uGamma}{\uSigma, \uShyp{\tsmv}{a}{\tau}{\eparse}}{\autsmap{b}{\tsmv}}{e}{\tau}}\\
=& \sizeof{\cvalidE{\emptyset}{\emptyset}{\esceneU{\uDelta}{\uGamma}{\uSigma, \uShyp{\tsmv}{a}{\tau}{\eparse}}{b}}{\ce}{e}{\tau}}\\
=&\sizeof{b}\end{align*}

The only case in the proof of part 2 that invokes part 1 is Case (\ref{rule:cvalidE-U-splicede}). There, we have that $\parseUExp{\bsubseq{b}{m}{n}}{\ue}$ and the IH is applied to the judgement $\expandsU{\uDelta_\text{app}}{\uGamma_\text{app}}{\uSigma}{\ue}{e}{\tau}$ where $\uDelta_\text{app}=\uDD{\uD}{\Delta_\text{app}}$ and $\uGamma_\text{app}=\uGG{\uG}{\Gamma_\text{app}}$ and $\uSigma=\uAS{\uA}{\Sigma}$. Because the metric is stable when passing from part 1 to part 2, we must have that it is strictly decreasing in the other direction:
\[\sizeof{\expandsU{\uDelta_\text{app}}{\uGamma_\text{app}}{\uSigma}{\ue}{e}{\tau}} < \sizeof{\cvalidE{\Delta}{\Gamma}{\esceneU{\uDelta_\text{app}}{\uGamma_\text{app}}{\uSigma}{b}}{\acesplicede{m}{n}}{e}{\tau}}\]
i.e. by the definitions above, 
\[\sizeof{\ue} < \sizeof{b}\]

This is established by appeal to the following two conditions. The first condition simply states that subsequences of $b$ are no longer than $b$.
\begin{condition}[Body Subsequencing]\label{condition:body-subsequences} If $\bsubseq{b}{m}{n}=b'$ then $\sizeof{b'} \leq \sizeof{b}$. \end{condition}
The second condition states that an unexpanded expression constructed by parsing a textual sequence $b$ is strictly smaller, as measured by the metric defined above, than the length of $b$, because some characters must necessarily be used to invoke a TSM and delimit each literal body.
\begin{condition}[Expression Parsing Monotonicity]\label{condition:body-parsing} If $\parseUExp{b}{\ue}$ then $\sizeof{\ue} < \sizeof{b}$.\end{condition}

Combining Conditions \ref{condition:body-subsequences} and \ref{condition:body-parsing}, we have that $\sizeof{\ue} < \sizeof{b}$ as needed.
\end{proof}
% We need to define the following theorem about candidate expansion expression validation mutually with Theorem \ref{thm:typed-expansion-U}. 
% \begin{theorem}[Candidate Expansion Expression Validation]\label{thm:candidate-expansion-validation-U}
% If $\cvalidE{\Delta}{\Gamma}{\esceneU{\Delta_\text{app}}{\Gamma_\text{app}}{\uSigma}{b}}{\ce}{e}{\tau}$ and $\uetsmenv{\Delta_\text{app}}{\uSigma}$ then $\hastypeU{\Dcons{\Delta}{\Delta_\text{app}}}{\Gcons{\Gamma}{\Gamma_\text{app}}}{e}{\tau}$.
% \end{theorem}
% \begin{proof} By rule induction over Rules (\ref{rules:cvalidE-U}).
% \begin{byCases}
% \item[\text{(\ref{rule:cvalidE-U-var})}] We have
% \begin{pfsteps*}
%   \item $\ce=x$ \BY{assumption}
%   \item $e=x$ \BY{assumption}
%   \item $\Gamma=\Gamma', \Ghyp{x}{\tau}$ \BY{assumption}
%   \item $\hastypeU{\Dcons{\Delta}{\Delta_\text{app}}}{\Gamma', \Ghyp{x}{\tau}}{x}{\tau}$ \BY{Rule (\ref{rule:hastypeU-var})} \pflabel{hastypeU}
%   \item $\hastypeU{\Dcons{\Delta}{\Delta_\text{app}}}{\Gcons{\Gamma', \Ghyp{x}{\tau}}{\Gamma_\text{app}}}{x}{\tau}$ \BY{Lemma \ref{lemma:weakening-U} over $\Gamma_\text{app}$ to \pfref{hastypeU}}
% \end{pfsteps*}
% \resetpfcounter

% \item[\text{(\ref{rule:cvalidE-U-lam})}] We have
% \begin{pfsteps*}
%   \item $\ce=\acelam{\ctau_1}{x}{\ce'}$ \BY{assumption}
%   \item $e=\aelam{\tau_1}{x}{e'}$ \BY{assumption}
%   \item $\tau=\aparr{\tau_1}{\tau_2}$ \BY{assumption}
%   \item $\cvalidT{\Delta}{\esceneU{\Delta_\text{app}}{\Gamma_\text{app}}{\uSigma}{b}}{\ctau_1}{\tau_1}$ \BY{assumption} \pflabel{cvalidT}
%   \item $\cvalidE{\Delta}{\Gamma, \Ghyp{x}{\tau_1}}{\esceneU{\Delta_\text{app}}{\Gamma_\text{app}}{\uSigma}{b}}{\ce'}{e'}{\tau_2}$ \BY{assumption} \pflabel{cvalidE}
%   \item $\uetsmenv{\Delta_\text{app}}{\uSigma}$ \BY{assumption} \pflabel{uetsmenv}
%   \item $\istypeU{\Dcons{\Delta}{\Delta_\text{app}}}{\tau_1}$ \BY{Lemma \ref{lemma:candidate-expansion-type-validation} on \pfref{cvalidT}} \pflabel{istype}
%   \item $\hastypeU{\Dcons{\Delta}{\Delta_\text{app}}}{\Gcons{\Gamma, \Ghyp{x}{\tau_1}}{\Gamma_\text{app}}}{e'}{\tau_2}$ \BY{IH on \pfref{cvalidE} and \pfref{uetsmenv}} \pflabel{hastype1}
%   \item $\hastypeU{\Dcons{\Delta}{\Delta_\text{app}}}{\Gcons{\Gamma}{\Gamma_\text{app}}, \Ghyp{x}{\tau_1}}{e'}{\tau_2}$ \BY{exchange over $\Gamma_\text{app}$ on \pfref{hastype1}} \pflabel{hastype2}
%   \item $\hastypeU{\Dcons{\Delta}{\Delta_\text{app}}}{\Gcons{\Gamma}{\Gamma_\text{app}}}{\aelam{\tau_1}{x}{e'}}{\aparr{\tau_1}{\tau_2}}$ \BY{Rule (\ref{rule:hastypeU-lam}) on \pfref{istype} and \pfref{hastype2}}
% \end{pfsteps*}
% \resetpfcounter

% \item[\text{(\ref{rule:cvalidE-U-ap})}] We have
% \begin{pfsteps*}
%   \item $\ce=\aceap{\ce_1}{\ce_2}$ \BY{assumption}
%   \item $e=\aeap{e_1}{e_2}$ \BY{assumption}
%   \item $\cvalidE{\Delta}{\Gamma}{\esceneU{\Delta_\text{app}}{\Gamma_\text{app}}{\uSigma}{b}}{\ce_1}{e_1}{\aparr{\tau_1}{\tau}}$ \BY{assumption} \pflabel{cvalidE1}
%   \item $\cvalidE{\Delta}{\Gamma}{\esceneU{\Delta_\text{app}}{\Gamma_\text{app}}{\uSigma}{b}}{\ce_2}{e_2}{\tau_1}$ \BY{assumption} \pflabel{cvalidE2}
%   \item $\uetsmenv{\Delta_\text{app}}{\uSigma}$ \BY{assumption} \pflabel{uetsmenv}
%   \item $\hastypeU{\Dcons{\Delta}{\Delta_\text{app}}}{\Gcons{\Gamma}{\Gamma_\text{app}}}{e_1}{\aparr{\tau_1}{\tau}}$ \BY{IH on \pfref{cvalidE1} and \pfref{uetsmenv}} \pflabel{hastypeU1}
%   \item $\hastypeU{\Dcons{\Delta}{\Delta_\text{app}}}{\Gcons{\Gamma}{\Gamma_\text{app}}}{e_2}{\tau_1}$ \BY{IH on \pfref{cvalidE2} and \pfref{uetsmenv}} \pflabel{hastypeU2}
%   \item $\hastypeU{\Dcons{\Delta}{\Delta_\text{app}}}{\Gcons{\Gamma}{\Gamma_\text{app}}}{\aeap{e_1}{e_2}}{\tau}$ \BY{Rule (\ref{rule:hastypeU-ap}) on \pfref{hastypeU1} and \pfref{hastypeU2}}
% \end{pfsteps*}
% \resetpfcounter

% \item[\VExpof{\text{\ref{rule:hastypeU-tlam}}}~\text{through}~\VExpof{\text{\ref{rule:hastypeU-case}}}] These cases follow analagously, i.e. we apply the IH to all candidate expansion expression validation premises, Lemma \ref{lemma:candidate-expansion-type-validation} to all candidate expansion type validation premises, weakening and exchange as needed, and then apply the corresponding typing rule.
% \\

% \item[\text{(\ref{rule:cvalidE-U-splicede})}] We have
% \begin{pfsteps*}
%   \item $\ce=\acesplicede{m}{n}$ \BY{assumption}
%   \item $\parseUExp{\bsubseq{b}{m}{n}}{\ue}$ \BY{assumption}
%   \item $\expandsU{\Delta_\text{app}}{\Gamma_\text{app}}{\uSigma}{\ue}{e}{\tau}$ \BY{assumption} \pflabel{expands}
%   \item $\uetsmenv{\Delta_\text{app}}{\uSigma}$ \BY{assumption} \pflabel{uetsmenv}
%   \item $\hastypeU{\Delta_\text{app}}{\Gamma_\text{app}}{e}{\tau}$ \BY{Theorem \ref{thm:typed-expansion-U} on \pfref{expands} and \pfref{uetsmenv}} \pflabel{hastype}
%   \item $\hastypeU{\Dcons{\Delta}{\Delta_\text{app}}}{\Gcons{\Gamma}{\Gamma_\text{app}}}{e}{\tau}$ \BY{Lemma \ref{lemma:weakening-U} on \pfref{hastype}}
% \end{pfsteps*}
% \resetpfcounter
% \end{byCases}
% \end{proof}


%\qed


